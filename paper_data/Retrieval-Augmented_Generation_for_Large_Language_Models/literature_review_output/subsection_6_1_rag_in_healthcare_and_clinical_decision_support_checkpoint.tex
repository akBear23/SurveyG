\subsection{RAG in Healthcare and Clinical Decision Support}
The application of Large Language Models (LLMs) in the high-stakes medical domain presents both immense opportunities and significant challenges, primarily due to their propensity for generating "hallucinations" or factually incorrect information. Retrieval-Augmented Generation (RAG) has emerged as a critical technique to ground LLMs in authoritative clinical guidelines, electronic health records (EHRs), and biomedical knowledge graphs, thereby reducing hallucinations and substantially improving accuracy for tasks like medical question answering, guideline interpretation, and clinical trial screening.

A systematic review and meta-analysis by \cite{liu2025p6t} quantitatively demonstrates RAG's effectiveness, showing a 1.35 odds ratio increase in performance compared to baseline LLMs in biomedicine. To systematically understand RAG's capabilities in this critical field, \cite{xiong2024exb} introduced MIRAGE, the first comprehensive benchmark for medical RAG, alongside the MEDRAG toolkit. Their large-scale evaluation of 41 RAG configurations revealed that RAG can improve LLM accuracy by up to 18\% and elevate smaller models like GPT-3.5 to rival GPT-4's performance without RAG, while also identifying challenges such as the "lost-in-the-middle" phenomenon.

Numerous studies have since demonstrated RAG's practical utility across diverse clinical applications. For instance, \cite{kresevic2024uel} showcased RAG's potential for reliable clinical decision support by achieving near-perfect (99.0\%) accuracy in interpreting hepatological clinical guidelines. This was accomplished through meticulous data reformatting, converting complex tables and non-textual elements into LLM-friendly structured text, and advanced prompt engineering, proving these steps to be more impactful than few-shot learning alone. Similarly, \cite{ke20248bm} developed an optimized RAG pipeline for preoperative medicine, integrating 35 guidelines and achieving 91.4\% accuracy, non-inferior to human experts, while significantly reducing response time. Expanding on this, \cite{ke2025wm0} further evaluated RAG's generalizability across ten LLMs for medical fitness assessments, finding that RAG-augmented GPT-4 models consistently outperformed human evaluators in accuracy, consistency, and safety when grounded in local and international guidelines.

RAG has also been successfully applied to specialized medical tasks and data types. For clinical trial screening, \cite{unlu2024yc8} introduced RECTIFIER, a RAG-enabled GPT-4 system that efficiently extracts information from unstructured EHRs, outperforming human study staff in accuracy and significantly reducing screening time. For patient communication, \cite{ge20237yq} developed LiVersa, a liver disease-specific, PHI-compliant RAG chatbot, demonstrating a secure architecture for integrating authoritative guidelines. In multilingual contexts, \cite{zhou20249ba} created GastroBot, a Chinese gastrointestinal disease chatbot, which achieved high context recall and faithfulness by fine-tuning a domain-specific embedding model on Chinese guidelines and literature. \cite{lee20240to} further explored multilingual capabilities with a dual RAG system for diabetes guidelines, optimizing ensemble retrievers for both Korean and English texts. Other applications include lung cancer staging using RAG-LLM NotebookLM \cite{tozuka2024nau}, emergency patient triage with RAG-enhanced LLMs \cite{yazaki20245js}, and breast cancer nursing care, where RAG significantly improved response accuracy and overall satisfaction without compromising empathy \cite{xu2024w5j}. RAG also plays a crucial role in medical education, as demonstrated by \cite{ghadban2023j9e} with SMARThealth GPT, a RAG-based tool for frontline health worker capacity building in low- and middle-income countries, emphasizing traceability and scalability.

Beyond plain text, researchers are integrating RAG with structured knowledge. \cite{soman2023m86} developed KG-RAG, a token-optimized framework that leverages a biomedical knowledge graph (SPOKE) to ground LLMs, achieving over 50\% token reduction and enhanced robustness to prompt perturbations compared to traditional KG-RAG methods. Building on this, \cite{matsumoto2024b7a} introduced KRAGEN, a knowledge graph-enhanced RAG framework that uses advanced prompting techniques like Graph-of-Thoughts to dynamically break down and solve complex biomedical problems. Similarly, \cite{liu2025rz6} utilized a knowledge graph-based RAG to detect emergencies in patient portal messages, significantly improving accuracy, sensitivity, and specificity compared to LLMs without RAG.

Further advancements focus on enhancing LLM reasoning and self-correction within medical RAG systems. \cite{jeong2024cey} proposed Self-BioRAG, a framework incorporating domain-specific instruction sets, a specialized retriever, and a critic LLM for self-reflection, leading to improved medical reasoning and explanation generation. \cite{hammane2024hdb} also explored RAG with self-evaluation (SelfRewardRAG) to enhance medical reasoning by integrating real-time clinical records. Finally, hybrid approaches like those explored by \cite{bora20242mq} investigate combining RAG with fine-tuning for optimal performance in medical chatbot applications.

Despite these significant strides, challenges remain. Continuous updating of dynamic medical knowledge bases, ensuring data privacy (especially with sensitive patient data), and developing robust evaluation metrics that reliably assess factual correctness and clinical relevance (beyond lexical similarity) are ongoing areas of research. The integration of multimodal data (e.g., images, videos) into RAG for comprehensive clinical decision support also presents a promising future direction.