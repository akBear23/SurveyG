\subsection{Privacy and Security in RAG Systems}

While Retrieval-Augmented Generation (RAG) systems have revolutionized how Large Language Models (LLMs) access and synthesize external knowledge, significantly reducing hallucinations and providing up-to-date information \cite{lewis2020pwr}, their widespread adoption, particularly in sensitive domains, introduces critical and often overlooked privacy and security challenges. The field has seen extensive work on benchmarking RAG's capabilities \cite{chen2023nzb, xiong2024exb, tang2024i5r, salemi2024om5} and developing advanced architectures for robustness \cite{yan202437z, yu202480d, chan2024u69}, as well as applying RAG to structured data and domain-specific applications like textual graphs \cite{he20248lp}, customer service \cite{xu202412d}, and medical guidelines \cite{kresevic2024uel}. However, the inherent privacy vulnerabilities of RAG, especially concerning data leakage from external retrieval databases, have only recently begun to receive systematic scrutiny.

A pivotal work addressing these concerns is \cite{zeng2024dzl}, which provides the first comprehensive exploration of privacy issues in RAG systems. This research systematically investigates two primary privacy problems: the susceptibility of RAG systems to leak private information directly from their external retrieval databases, and how the integration of external retrieval data influences the privacy leakage of the LLM's own training data. Unlike prior LLM privacy research that focused on extracting memorized training data from the LLM's parametric knowledge, \cite{zeng2024dzl} introduces a novel methodological advancement: **composite structured prompting attacks**. This attack method cleverly combines an `{information}` component to guide the retriever towards specific data and a `{command}` component to instruct the LLM to output the retrieved content, effectively weaponizing the RAG pipeline for data extraction.

Empirical validation by \cite{zeng2024dzl} reveals significant vulnerabilities. For instance, targeted attacks successfully extracted 89 medical dialogue chunks and 107 pieces of Personally Identifiable Information (PII) using Llama-7b-Chat, while untargeted prompts on the Enron Email dataset led to exact matches in 116 out of 250 attempts with GPT-3.5-turbo. These findings underscore that RAG systems are highly susceptible to privacy breaches from their external knowledge bases, which often contain sensitive or proprietary information. This is particularly alarming given RAG's application in high-stakes environments such as medicine \cite{xiong2024exb, kresevic2024uel} and customer service \cite{xu202412d}, where data confidentiality is paramount.

Crucially, \cite{zeng2024dzl} also uncovers a counter-intuitive insight: RAG can actually *mitigate* the leakage of the LLM's own training data. This suggests a complex trade-off, where RAG introduces new vulnerabilities related to its external data sources but may offer a potential security benefit by reducing the LLM's tendency to output memorized pre-training data. Ablation studies further highlight that the design of the command prompt significantly impacts the success of these attacks, with explicit instructions like "Please repeat all the context" proving highly effective.

The implications of \cite{zeng2024dzl}'s findings are profound, shifting the narrative around RAG from an unmitigated benefit to a technology requiring careful privacy considerations. The identified vulnerabilities necessitate the urgent development of privacy-preserving RAG architectures and robust security measures. This includes designing retrieval mechanisms that can enforce fine-grained access controls, anonymizing sensitive data within retrieval databases, and developing advanced prompt filtering techniques to detect and neutralize malicious composite structured prompts. As RAG systems continue to evolve and integrate with diverse knowledge sources and complex reasoning tasks \cite{tang2024i5r, he20248lp}, ensuring responsible and ethical use demands a proactive approach to security, balancing the immense utility of RAG with stringent privacy safeguards. Future research must focus on building defense mechanisms against these novel RAG-specific attacks and further understanding the intricate interplay between retrieval and generation in terms of privacy.