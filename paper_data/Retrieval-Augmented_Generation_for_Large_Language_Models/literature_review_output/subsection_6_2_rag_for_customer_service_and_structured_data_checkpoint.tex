\subsection{RAG for Customer Service and Structured Data}
The application of Retrieval-Augmented Generation (RAG) in enterprise settings, particularly for customer service question answering and interaction with structured data, presents unique challenges and opportunities. While foundational RAG models \cite{lewis2020pwr} demonstrated the power of augmenting Large Language Models (LLMs) with external knowledge, their effectiveness diminishes when dealing with inherently structured and interconnected enterprise knowledge bases. Traditional RAG often treats documents as flat text, overlooking crucial intra-document structures and inter-document relationships, which can lead to compromised retrieval accuracy and suboptimal answer quality, as general RAG benchmarks have highlighted limitations in information integration and noise robustness \cite{chen2023nzb}.

To address these limitations, recent research emphasizes the integration of RAG with structured knowledge representations, such as Knowledge Graphs (KGs). A prime example in the customer service domain is the work by \cite{xu202412d}, which introduces a novel RAG framework leveraging KGs for customer service question answering. This approach constructs a dual-level KG that preserves both intra-issue structure (parsing individual tickets into trees of sections) and inter-issue relations (connecting tickets via explicit and implicit links). During question answering, an LLM-driven subgraph retrieval mechanism parses consumer queries for entities and intents, translating them into graph database queries (e.g., Cypher) to extract highly pertinent subgraphs. This sophisticated method yielded substantial empirical benefits, including a 77.6\% improvement in Mean Reciprocal Rank (MRR) and a 0.32 improvement in BLEU score over conventional RAG baselines, and significantly reduced median per-issue resolution time by 28.6\% in a real-world deployment.

Extending beyond customer service, the principle of leveraging structured knowledge for RAG applies to various domains with complex textual graphs. \cite{he20248lp} introduces G-Retriever, a pioneering RAG framework for general textual graph understanding and question answering. G-Retriever formulates subgraph retrieval as a Prize-Collecting Steiner Tree (PCST) optimization problem, enabling it to effectively leverage structural information to mitigate hallucination and improve scalability when "chatting with graphs." This highlights a broader trend towards graph-aware retrieval for complex, interconnected data. Similarly, in high-stakes domains like medicine, \cite{kresevic2024uel} demonstrates that meticulous data reformatting of clinical guidelines (e.g., converting tables and images into structured text-based lists) combined with advanced prompt engineering is paramount for achieving near-perfect accuracy (99.0\%) in interpreting hepatological guidelines, even more so than few-shot learning. This underscores that for domain-specific, structured information, the quality of data preparation and its structured representation for the LLM is a critical factor, a finding further supported by general medical RAG benchmarking efforts \cite{xiong2024exb}.

While integrating structured data significantly enhances RAG performance, it also introduces new considerations. General advancements in RAG, such as corrective retrieval strategies \cite{yan202437z} or unified context ranking and generation \cite{yu202480d}, can further bolster the robustness and efficiency of structured RAG systems. Query refinement techniques, like those proposed by \cite{chan2024u69} for rewriting, decomposing, and disambiguating queries, can be particularly effective when guided by the underlying structure of a knowledge graph. However, a critical concern for enterprise applications dealing with sensitive structured data is privacy. \cite{zeng2024dzl} systematically explores privacy issues in RAG, revealing significant vulnerabilities to data leakage from external retrieval databases through composite structured prompting attacks. This paper also presents a counter-intuitive finding: RAG can actually mitigate the leakage of the LLM's own training data, offering a nuanced perspective on RAG's privacy implications. Furthermore, accurately evaluating the utility of retrieved structured information to the LLM remains crucial, with methods like eRAG \cite{salemi2024om5} proposing to align retrieval evaluation directly with the LLM's downstream performance.

In conclusion, the literature clearly demonstrates that moving beyond plain-text retrieval to actively leverage structured knowledge representations, such as Knowledge Graphs, is essential for RAG systems operating in complex enterprise environments like customer service. This approach significantly improves retrieval accuracy, answer quality, and operational efficiency by preserving the inherent structure and relationships within domain-specific data. However, the development of these sophisticated systems necessitates careful consideration of data engineering, specialized retrieval algorithms, and critical privacy implications to ensure robust and trustworthy deployment. Future research must continue to explore automated KG construction, dynamic knowledge updates, and advanced privacy-preserving mechanisms for structured RAG.