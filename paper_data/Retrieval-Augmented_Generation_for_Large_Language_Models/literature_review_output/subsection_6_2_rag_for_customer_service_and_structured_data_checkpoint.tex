\subsection{RAG for Customer Service and Structured Data}
The application of Retrieval-Augmented Generation (RAG) in enterprise settings, particularly for customer service question answering and interaction with structured data, presents unique challenges and opportunities. While foundational RAG models \cite{lewis2020pwr} demonstrated the power of augmenting Large Language Models (LLMs) with external knowledge, their effectiveness diminishes when dealing with inherently structured and interconnected enterprise knowledge bases. Traditional RAG often treats documents as flat text, overlooking crucial intra-document structures and inter-document relationships, which can lead to compromised retrieval accuracy and suboptimal answer quality. General RAG benchmarks have highlighted limitations in information integration and noise robustness when faced with complex data \cite{chen2023nzb}. This subsection explores how RAG can effectively leverage structured knowledge representations, such as Knowledge Graphs (KGs) and tabular data, to enhance performance in domains where information has inherent structure and relationships.

To address these limitations, recent research emphasizes the integration of RAG with structured knowledge representations. A prime example in the customer service domain is the work by \cite{xu202412d}, which introduces a novel RAG framework leveraging KGs for customer service question answering. This approach constructs a dual-level KG that preserves both intra-issue structure (parsing individual tickets into trees of sections) and inter-issue relations (connecting tickets via explicit and implicit links). During question answering, an LLM-driven subgraph retrieval mechanism parses consumer queries for entities and intents, translating them into graph database queries (e.g., Cypher) to extract highly pertinent subgraphs. This sophisticated method yielded substantial empirical benefits, including a 77.6\% improvement in Mean Reciprocal Rank (MRR) and a 0.32 improvement in BLEU score over conventional RAG baselines, and significantly reduced median per-issue resolution time by 28.6\% in a real-world deployment. Similarly, \cite{debellis2024bv0} demonstrates the benefits of using ontologies and knowledge graphs to form domain-specific knowledge bases for RAG, enabling agile development and improved retrieval through reformulation browsing in support contexts. These approaches underscore the critical role of explicit structural information in mitigating hallucination and improving the precision of retrieved context, as further detailed by surveys on GraphRAG \cite{zhang2025gnc} which highlight its ability to support multi-step reasoning and capture complex relationships beyond flat text.

Extending beyond knowledge graphs, RAG for tabular data, such as querying relational databases via Text-to-SQL, represents another significant application in enterprise settings. Traditional LLMs struggle with the intricacies of SQL schema linking and complex query generation. \cite{thorpe2024l37} introduces Dubo-SQL, a method that combines diverse RAG with fine-tuning for Text-to-SQL tasks, achieving state-of-the-art execution accuracy (EX) on benchmarks like BIRD-SQL. This approach demonstrates how RAG can be tailored to generate precise, executable queries by retrieving relevant schema information and example queries, thereby transforming natural language questions into structured database operations. The challenge here lies not just in retrieving relevant text, but in translating intent into a formal, executable language that accurately reflects the underlying data structure, a distinct problem from graph traversal but equally critical for structured data interaction.

General advancements in RAG can be strategically adapted to further enhance structured RAG systems. For instance, the pre-retrieval phase, as categorized by \cite{huang2024a59}, is crucial for structured data. Query refinement techniques, such as those proposed by \cite{chan2024u69} for rewriting, decomposing, and disambiguating queries, can be specifically engineered to generate more effective graph traversal commands or SQL queries, guided by the underlying schema. This involves training LLMs to understand the structure of the knowledge base (e.g., entity types, relation properties, table schemas) and formulate queries that are syntactically correct and semantically aligned with the structured data. Furthermore, the post-retrieval and generation phases benefit from techniques like unified context ranking and answer generation \cite{yu202480d}. In structured RAG, this could involve ranking retrieved subgraphs or SQL query results based on their relevance to the LLM's generation task, ensuring the most pertinent structured information is prioritized. Corrective retrieval strategies, such as CRAG \cite{yan202437z}, can dynamically assess the quality of a generated SQL query or a retrieved subgraph, triggering refinement or re-querying if initial results are suboptimal or lead to errors, thereby enhancing robustness, especially for complex, multi-hop queries over structured data \cite{zhao2024931}.

While integrating structured data significantly enhances RAG performance, it also introduces new considerations, particularly regarding privacy and evaluation. Enterprise applications often deal with highly sensitive structured data, making privacy a paramount concern. \cite{zeng2024dzl} systematically explores privacy issues in RAG, revealing significant vulnerabilities to data leakage from external retrieval databases through composite structured prompting attacks. This risk is amplified when querying explicit knowledge graphs or relational databases, where the structure itself can inadvertently reveal sensitive relationships or infer private information. Further, \cite{li2024w6r} highlights membership inference attacks against RAG's external database, demonstrating that semantic similarity between generated content and a sample can reveal if the sample was part of the database, a critical vulnerability for proprietary structured datasets. Conversely, \cite{zeng2024dzl} also presents a nuanced finding that RAG can mitigate the leakage of the LLM's own training data, offering a complex perspective on RAG's privacy implications. Accurately evaluating the utility of retrieved structured information to the LLM remains crucial, with methods like eRAG \cite{salemi2024om5} proposing to align retrieval evaluation directly with the LLM's downstream performance, which is vital for assessing the true value of complex graph traversals or SQL query results.

In conclusion, the literature clearly demonstrates that moving beyond plain-text retrieval to actively leverage structured knowledge representations, such as Knowledge Graphs and tabular data, is essential for RAG systems operating in complex enterprise environments like customer service. This approach significantly improves retrieval accuracy, answer quality, and operational efficiency by preserving the inherent structure and relationships within domain-specific data. However, the development of these sophisticated systems necessitates careful consideration of data engineering, specialized retrieval algorithms, and critical privacy implications to ensure robust and trustworthy deployment. Future research must continue to explore hybrid retrieval mechanisms that can seamlessly query both graph-based knowledge, tabular data, and unstructured text within a single enterprise RAG system. Additionally, developing automated KG construction, dynamic schema inference for tabular data, advanced privacy-preserving graph traversal algorithms, and robust evaluation metrics for complex reasoning over structured data are crucial for the continued advancement of RAG in these critical domains.