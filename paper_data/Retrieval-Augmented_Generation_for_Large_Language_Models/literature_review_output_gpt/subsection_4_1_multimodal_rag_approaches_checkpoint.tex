\subsection{Multimodal RAG Approaches}

The integration of multimodal data in Retrieval-Augmented Generation (RAG) systems represents a significant advancement in the ability of models to process and generate information across diverse modalities, such as text and images. Traditional RAG models are primarily limited to textual knowledge, which constrains their effectiveness in scenarios requiring multimodal reasoning. Recent works, particularly the introduction of models like MuRAG, illustrate the potential of multimodal RAG approaches to enhance contextual understanding and broaden the applicability of RAG in real-world applications.

The foundational work by Chen et al. in \cite{chen2022j8c} introduces MuRAG, the first Multimodal Retrieval-Augmented Transformer, which effectively combines visual and textual information for open-domain question answering. MuRAG employs a unified encoder that integrates pre-trained T5 and ViT models, allowing it to process queries and memory candidates across both modalities. A key innovation in MuRAG is its joint pre-training objective, which utilizes a contrastive loss for retrieval and a generative loss for knowledge integration, alongside a two-stage fine-tuning pipeline to manage large external multimodal memories. This approach addresses the critical limitation of traditional RAG models by enabling the retrieval of visual information, thereby facilitating responses to visually-grounded queries.

Building on the foundation laid by MuRAG, the work presented in \cite{chen20245d2} introduces RagVL, which enhances multimodal RAG through knowledge-enhanced reranking and noise-injected training. RagVL addresses the static nature of Multimodal Large Language Models (MLLMs), which often rely on outdated training data, by employing a reranking mechanism to filter top-k retrieved images more accurately. Additionally, the introduction of visual noise during training enhances the robustness of the generator. This work highlights the ongoing challenge of effectively managing the multi-granularity noisy correspondence (MNC) problem, which can hinder accurate retrieval and generation in multimodal contexts.

Further advancing the discussion, the hybrid RAG framework proposed in \cite{omrani2024i22} synergizes various methodologies to optimize query response capabilities of Large Language Models (LLMs). By integrating external knowledge sources more effectively, this framework improves the accuracy and relevance of model-generated responses. The introduction of an innovative re-ranking mechanism within this hybrid approach showcases a significant step forward in enhancing the interaction between LLMs and external knowledge, thereby addressing limitations observed in previous models that struggled with contextually appropriate outputs.

Despite these advancements, challenges remain in the scalability and efficiency of multimodal RAG systems. While MuRAG demonstrates state-of-the-art performance on multimodal QA datasets, the computational costs associated with managing large external multimodal memories continue to pose significant hurdles. RagVL's approach to reranking and noise injection offers promising solutions, yet the inherent complexity of multimodal retrieval and integration suggests that further research is needed to optimize these processes for broader applications.

In conclusion, the evolution of multimodal RAG approaches, as exemplified by MuRAG, RagVL, and the hybrid frameworks, marks a critical shift towards integrating diverse knowledge modalities in language generation. These developments not only enhance the contextual understanding of models but also expand their applicability across various real-world scenarios. Future research should focus on addressing the unresolved issues of computational efficiency and the effective management of multimodal knowledge bases to fully realize the potential of multimodal RAG systems in dynamic environments.
```