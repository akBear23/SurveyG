\documentclass[12pt,a4paper]{article}
    \usepackage[utf8]{inputenc}
    \usepackage[T1]{fontenc}
    \usepackage{amsmath,amsfonts,amssymb}
    \usepackage{graphicx}
    \usepackage[margin=2.5cm]{geometry}
    \usepackage{setspace}
    \usepackage{natbib}
    \usepackage{url}
    \usepackage{hyperref}
    \usepackage{booktabs}
    \usepackage{longtable}
    \usepackage{array}
    \usepackage{multirow}
    \usepackage{wrapfig}
    \usepackage{float}
    \usepackage{colortbl}
    \usepackage{pdflscape}
    \usepackage{tabu}
    \usepackage{threeparttable}
    \usepackage{threeparttablex}
    \usepackage[normalem]{ulem}
    \usepackage{makecell}
    \usepackage{xcolor}

    % Set line spacing
    \doublespacing

    % Configure hyperref
    \hypersetup{
        colorlinks=true,
        linkcolor=blue,
        filecolor=magenta,      
        urlcolor=cyan,
        citecolor=red,
    }

    % Title and author information
    \title{A Comprehensive Literature Review with Self-Reflection}
    \author{Literature Review}
    \date{\today}

    \begin{document}

    \maketitle

    % Abstract (optional)
    \begin{abstract}
    This literature review provides a comprehensive analysis of recent research in the field. The review synthesizes findings from 211 research papers, identifying key themes, methodological approaches, and future research directions.
    \end{abstract}

    \newpage
    \tableofcontents
    \newpage

    \label{sec:introduction}

\section{Introduction}
\label{sec:introduction}

\subsection{Background: Large Language Models (LLMs) and their Limitations}
\label{sec:1\_1\_background:\_large\_language\_models\_(llms)\_\_and\_\_their\_limitations}

Large Language Models (LLMs) have ushered in a new era of natural language processing, demonstrating remarkable capabilities in generating coherent, contextually relevant, and often creative text across a myriad of tasks, from content creation and summarization to translation and question answering. Their widespread adoption has revolutionized human-computer interaction and significantly advanced the state-of-the-art in various domains. However, despite these impressive generative abilities, a critical examination reveals inherent shortcomings that fundamentally limit their reliability and utility in knowledge-intensive applications, serving as the primary motivation for the development of external knowledge integration mechanisms like Retrieval-Augmented Generation (RAG).

One of the most significant limitations of LLMs is their propensity for generating factually incorrect or nonsensical information, commonly referred to as "hallucinations" \cite{gao20238ea, he20248lp, yan202437z}. LLMs, by design, are trained to predict the next token based on patterns learned from vast datasets, not to guarantee factual accuracy or truthfulness. This inherent characteristic means they can confidently produce plausible-sounding but entirely false statements. For instance, \cite{chen2023nzb} empirically demonstrated that LLMs frequently fail to reject answering when no relevant information is available in external documents, instead generating incorrect answers. Furthermore, even when provided with warnings about potentially incorrect retrieved information, LLMs tend to prioritize and trust the (incorrect) retrieved data over their own internal knowledge, exacerbating the hallucination problem \cite{chen2023nzb}. This issue is not confined to general text; it extends to specialized domains, with \cite{he20248lp} noting that LLMs are prone to hallucinating content even in structured graph settings.

Another critical constraint stems from LLMs' reliance on static and potentially outdated training data. The parametric knowledge embedded within an LLM is frozen at the time of its last training cut-off, rendering it incapable of accessing real-time events, dynamic information, or newly emerging facts \cite{gao20238ea, yan202437z}. This limitation means that LLMs cannot provide up-to-date information on current affairs, recent scientific discoveries, or evolving domain-specific knowledge, significantly restricting their applicability in dynamic environments. The construction of benchmarks, such as the Retrieval-Augmented Generation Benchmark (RGB) by \cite{chen2023nzb}, specifically uses the latest news information to mitigate bias from LLMs' pre-existing internal knowledge, implicitly acknowledging the static nature of their training data and the need for external, current information. The inability to dynamically update their knowledge base without expensive and time-consuming retraining cycles makes LLMs inherently unsuitable for tasks requiring access to the most current information.

Finally, LLMs face significant constraints in processing very long or complex contexts due to their limited context window sizes. While modern LLMs have expanded their context windows considerably, they still struggle to process and reason over entire lengthy documents, extensive codebases, or multi-hour multimodal inputs like videos \cite{amugongo202530u}. This limitation means that even if relevant information exists within a vast document, the LLM may not be able to ingest it all directly into its input, leading to information loss or an inability to perform comprehensive in-context learning. \cite{he20248lp} highlights this challenge in the context of textual graphs, where converting large graphs into text sequences for LLMs often leads to excessive token counts that exceed context windows and cause information loss. Similarly, \cite{chen2023nzb} observed that LLMs often lack the ability to summarize and integrate information from multiple documents, a task that becomes increasingly difficult with longer or more numerous contexts.

These inherent challenges—hallucinations, reliance on static data, and context window limitations—underscore the fundamental necessity for external knowledge integration mechanisms. Without such mechanisms, LLMs remain prone to factual errors, quickly become outdated, and cannot effectively process the vast, complex information landscapes of real-world applications. This critical gap serves as the primary impetus for the development of Retrieval-Augmented Generation (RAG), which seeks to dynamically inject relevant, up-to-date, and accurate external knowledge into the LLM's generation process, thereby significantly enhancing its reliability, factual accuracy, and overall utility.
\subsection{The Genesis of Retrieval-Augmented Generation (RAG)}
\label{sec:1\_2\_the\_genesis\_of\_retrieval-augmented\_generation\_(rag)}

The emergence of Retrieval-Augmented Generation (RAG) represents a pivotal paradigm shift in the landscape of Large Language Models (LLMs), conceived specifically to address the inherent limitations of standalone LLMs discussed previously. These limitations primarily include a propensity for generating factually incorrect or nonsensical information (hallucinations), reliance on static and often outdated parametric knowledge, and a pervasive lack of transparency in their reasoning processes \cite{fan2024pf1, huang2024a59}. RAG directly confronts these critical shortcomings by effectively combining the powerful generative capabilities of LLMs with the dynamic ability to retrieve relevant, up-to-date, and verifiable information from vast external knowledge bases. This integration is crucial for grounding LLM responses in factual evidence, thereby significantly reducing the incidence of hallucinations and enhancing the overall trustworthiness and accuracy of generated content. Furthermore, RAG inherently provides a degree of transparency by allowing users to trace the source of information, a critical feature for building reliable AI systems \cite{gao20238ea}.

The foundational concept of Retrieval-Augmented Generation was formally introduced by \cite{lewis2020pwr} in their seminal 2020 paper, "Retrieval-Augmented Generation for Knowledge-Intensive NLP Tasks." This work presented RAG models as a novel approach to integrate both parametric and non-parametric memory into a single framework for language generation. Prior to RAG, LLMs primarily relied on their internal, parametric memory—the knowledge encoded within their vast number of parameters during pre-training—which was often static, prone to factual errors, and difficult to update without costly retraining \cite{fan2024pf1}. Lewis et al. proposed overcoming this by augmenting a pre-trained sequence-to-sequence (seq2seq) generator (representing the parametric memory) with a differentiable access mechanism to an explicit non-parametric memory. This non-parametric memory typically comprised a dense vector index of a large corpus, such as Wikipedia, accessed via a pre-trained neural retriever.

The core innovation of \cite{lewis2020pwr} lay in its ability to dynamically fetch relevant documents or passages from this external knowledge base based on the input query, and then condition the LLM's generation on these retrieved contexts. This mechanism directly addressed the problem of factual grounding, as the LLM could now synthesize responses informed by external, verifiable sources. The authors explored two primary RAG formulations: one where the generator conditions on the same retrieved passages across the entire generated sequence, and another more flexible approach where different passages could be utilized for each token generated. This latter approach allowed for more nuanced and context-sensitive generation, enabling the model to adapt its factual basis as it constructed a response.

The immediate impact of \cite{lewis2020pwr}'s RAG models was significant. They demonstrated state-of-the-art performance on several knowledge-intensive NLP tasks, particularly open-domain question answering, outperforming both purely parametric seq2seq models and existing task-specific retrieve-and-extract architectures. Crucially, for language generation tasks, RAG models were shown to produce output that was not only more specific and diverse but also significantly more factual than state-of-the-art parametric-only baselines. This marked a profound shift, offering a scalable and efficient method to inject up-to-date and verifiable knowledge into LLMs without the need for continuous, expensive retraining \cite{huang2024a59}. The inherent transparency, stemming from the ability to inspect the retrieved sources, further contributed to RAG's appeal as a solution for building more reliable and accountable AI systems.

In essence, the genesis of RAG was driven by the urgent need to imbue LLMs with dynamic, verifiable knowledge and to mitigate their inherent limitations. The framework proposed by \cite{lewis2020pwr} provided a robust and effective blueprint for integrating external knowledge, laying the groundwork for a vast body of subsequent research and development in the field. This foundational work established the "retrieve-then-generate" paradigm, which would become the bedrock for numerous advanced RAG architectures and optimizations, setting the stage for the detailed exploration of its components and evolution in the subsequent sections.
\subsection{Scope and Structure of the Review}
\label{sec:1\_3\_scope\_\_and\_\_structure\_of\_the\_review}

This literature review offers a meticulously organized and pedagogical roadmap through the rapidly evolving landscape of Retrieval-Augmented Generation (RAG) for Large Language Models (LLMs). Given the field's dynamic nature, characterized by diverse methodologies and rapidly emerging challenges \cite{huang2024a59}, a structured approach is imperative to consolidate fragmented knowledge and provide a coherent understanding. Our organizational framework is designed to guide the reader from foundational principles to cutting-edge advancements, ensuring a comprehensive and insightful grasp of RAG's intellectual trajectories and practical implications. This systematic progression aims to clarify the technological underpinnings of RAG, moving from basic concepts to more complex interactions and their real-world impact, aligning with the need for unified frameworks in this burgeoning area \cite{huang2024a59}.

The review commences by establishing the foundational RAG concepts and basic architectural paradigms (Section 2). This initial segment is crucial for grounding the reader in the core "retrieve-then-generate" mechanism, detailing its essential components—the retriever, the generator, and the external knowledge base. Understanding these fundamental building blocks and the early challenges they faced, such as effectively handling irrelevant context and mitigating initial hallucinations, provides the necessary context for appreciating subsequent innovations. Building upon these foundations, the review transitions to advanced methodological paradigms and optimizations (Section 3). This section explores sophisticated techniques that have refined RAG capabilities, categorizing them into pre-retrieval strategies (e.g., query refinement), post-retrieval enhancements (e.g., reranking and context compression), and adaptive, self-correcting frameworks. This progression reflects the field's problem-driven evolution from static pipelines to dynamic, intelligent systems capable of iterative refinement and robust knowledge management, directly addressing the limitations of earlier approaches.

A dedicated section then addresses the critical aspect of benchmarking and evaluation for RAG systems (Section 4). This segment is essential for rigorously assessing performance, faithfulness, and relevance across diverse RAG architectures and applications. It covers general benchmarking frameworks that diagnose core RAG capabilities, explores utility-aligned metrics that directly measure a document's value to the LLM, and delves into specialized benchmarks for complex reasoning tasks, such as multi-hop queries, and high-stakes domain-specific applications. The emphasis on robust and systematic evaluation is paramount for guiding future development and ensuring the reliability of RAG systems in real-world scenarios, particularly given the challenges of inconsistent benchmarking practices across the field \cite{rau20244nr}. Subsequently, the review explores specialized RAG applications (Section 5), focusing on its adaptation for structured and domain-specific knowledge. This includes the integration of knowledge graphs for enhanced reasoning, leveraging explicit structural and relational information to move beyond generic text retrieval. It also examines the meticulous data preparation and custom prompt engineering required to optimize RAG performance in highly specialized fields where accuracy and reliability are paramount.

Beyond these established technical advancements, the review delves into cutting-edge developments, notably the interplay between RAG and massive internal context models (Section 6). This section examines the paradigm shift towards dramatically expanding LLMs' native context windows and its profound implications for the future of RAG, potentially leading to novel hybrid architectures where internal capabilities and external retrieval synergistically enhance knowledge processing. This forward-looking perspective is crucial for understanding the evolving landscape of LLM knowledge acquisition and the potential redefinition of RAG's role. Finally, the review addresses critical ethical considerations and challenges inherent in RAG systems (Section 7), including privacy risks, mitigating bias, ensuring fairness, and robustness to adversarial attacks and misinformation. The importance of trustworthiness in RAG systems, encompassing dimensions like factuality, robustness, fairness, transparency, accountability, and privacy, is a central theme. Concluding with future research directions (Section 8), this comprehensive structure aims to provide readers with a holistic understanding of RAG's current state, its intellectual trajectory, and the promising avenues for future exploration within the dynamic landscape of LLM research.


\label{sec:foundational_concepts_of_rag}

\section{Foundational Concepts of RAG}
\label{sec:foundational\_concepts\_of\_rag}

\subsection{Basic RAG Architecture: Retrieve-then-Generate}
\label{sec:2\_1\_basic\_rag\_architecture:\_retrieve-then-generate}

The fundamental 'Retrieve-then-Generate' paradigm, often referred to as 'Naive RAG' or 'Retrieve-Read', constitutes the foundational architecture of Retrieval-Augmented Generation (RAG). This innovative approach marked a pivotal advancement beyond standalone Large Language Models (LLMs) by integrating external, verifiable information directly into the generation process \cite{lewis2020pwr, gao20238ea}. Its primary objective was to address inherent limitations of LLMs, such as their propensity for hallucination, reliance on static training data, and lack of transparency regarding factual sources.

Conceived as a sequential, two-stage process, the basic RAG architecture operates as follows:
\begin{enumerate}
    \item \textbf{Retrieval Stage}: Upon receiving a user's input query, a dedicated retriever component initiates a search within an external knowledge base. This knowledge base typically comprises a vast collection of documents, passages, or data indexed for efficient search, often using dense vector embeddings for semantic similarity matching \cite{lewis2020pwr}. The retriever's task is to identify and fetch a set of documents or passages deemed most relevant to the query. This step is crucial for accessing up-to-date, domain-specific, or proprietary information that would otherwise be inaccessible to the LLM.
    \item \textbf{Generation Stage}: The retrieved contexts, along with the original user query, are then fed as input to a Large Language Model (LLM). The LLM's role is to synthesize a coherent and informative answer by processing and reasoning over this augmented input. By grounding its response in the provided external evidence, the LLM aims to mitigate factual inaccuracies and hallucinations, thereby enhancing the trustworthiness and factual basis of the generated output \cite{gao20238ea}.
\end{enumerate}

The seminal work by Lewis et al. (2020) formally introduced the Retrieval-Augmented Generation framework, demonstrating its efficacy in knowledge-intensive NLP tasks. They proposed RAG models where a pre-trained sequence-to-sequence model (the generator) was augmented by a non-parametric memory, specifically a dense vector index of Wikipedia, accessed by a pre-trained neural retriever. This architecture allowed the LLM to dynamically consult an external knowledge source during generation, providing a mechanism for accessing and precisely manipulating knowledge beyond its parametric memory \cite{lewis2020pwr}. This was a significant conceptual leap, moving from models that solely relied on memorized knowledge to those that could actively seek and integrate new information.

The initial appeal of this 'Retrieve-then-Generate' paradigm stemmed from its intuitive solution to several critical LLM challenges. Firstly, it offered a direct mechanism to combat hallucinations by providing factual anchors from external sources. Secondly, it allowed LLMs to stay current with rapidly evolving information, as the external knowledge base could be updated independently of the LLM's training cycle. Thirdly, by presenting the retrieved documents alongside the generated answer, RAG inherently offered a degree of transparency and traceability, allowing users to verify the sources of information. This sequential integration of a retriever and a generator thus marked a significant methodological step, transforming LLMs from isolated knowledge systems into dynamic, externally-grounded reasoning agents \cite{gao20238ea}.

While conceptually powerful and a clear improvement over standalone LLMs, this basic 'Retrieve-then-Generate' architecture, in its naive implementation, quickly revealed a new set of challenges and limitations. These early hurdles, concerning the LLM's ability to effectively utilize, filter, and reason over the retrieved context, necessitated further research and the development of more sophisticated RAG paradigms, which will be explored in subsequent sections.
\subsection{Key Components: Retriever, Generator, and Knowledge Base}
\label{sec:2\_2\_key\_components:\_retriever,\_generator,\_\_and\_\_knowledge\_base}

Retrieval-Augmented Generation (RAG) systems are engineered to overcome the inherent limitations of standalone Large Language Models (LLMs) by integrating dynamic external knowledge. This foundational capability is realized through the synergistic interplay of three core components: the knowledge base, the retriever, and the generator. This architecture ensures that LLMs can access up-to-date, factual, and domain-specific information, thereby mitigating issues such as hallucination and reliance on static, potentially outdated training data.

The foundational element of any RAG system is the \textbf{knowledge base}, which serves as the external source of truth. This repository is a vast collection of information, meticulously prepared and indexed to facilitate efficient retrieval. It can encompass diverse data formats, ranging from unstructured text documents (e.g., articles, reports, web pages) to structured databases (e.g., relational databases, key-value stores) and semi-structured data. For dense retrieval methods, the knowledge base often takes the form of a vector store, where document chunks or passages are transformed into high-dimensional numerical embeddings and indexed for rapid similarity search \cite{gao20238ea}. The quality, comprehensiveness, and currency of this knowledge base are paramount, as they directly influence the factual grounding and reliability of the RAG system's outputs.

To efficiently navigate this potentially immense knowledge base, the \textbf{retriever} component is indispensable. Its primary role is to search and identify the most relevant information chunks, passages, or documents pertinent to a given user query. Retrieval methods typically fall into two main categories:
\begin{enumerate}
    \item \textbf{Sparse Retrieval:} These methods often rely on lexical or keyword matching techniques. Examples include TF-IDF (Term Frequency-Inverse Document Frequency) and BM25 (Best Match 25), which score documents based on the frequency and rarity of query terms within them. While effective for exact keyword matches, sparse methods can struggle with semantic similarity, failing to retrieve documents that convey the same meaning using different vocabulary.
    \item \textbf{Dense Retrieval:} These methods employ embedding-based approaches to find semantically similar documents in a high-dimensional vector space. Both the user query and the documents in the knowledge base are encoded into dense vector representations (embeddings) using specialized neural networks, often bi-encoder models. The relevance is then determined by calculating the cosine similarity or other distance metrics between the query embedding and document embeddings. A seminal work in this area is Dense Passage Retrieval (DPR) by \cite{karpukhin2020dense}, which demonstrated the effectiveness of training bi-encoders to generate embeddings that capture semantic relevance, significantly outperforming sparse methods on many open-domain question answering tasks. The efficiency of dense retrieval relies heavily on optimized vector databases that can perform fast approximate nearest neighbor searches.
\end{enumerate}
The retriever's effectiveness is crucial; a suboptimal retrieval step can lead to the generator receiving irrelevant or insufficient context, thereby undermining the entire RAG pipeline.

Once the retriever has identified and extracted relevant information from the knowledge base, the \textbf{generator}, typically a large language model (LLM), takes over. The LLM processes both the original user query and the retrieved context to formulate a coherent, informed, and factually grounded response. This integration of external knowledge with the LLM's generative capabilities was a core innovation introduced by \cite{lewis2020retrievalaugmented} with their Retrieval-Augmented Generation model. The generator's task is not merely to concatenate the retrieved information but to synthesize it, reason over it, and present it in a natural and understandable manner, directly addressing the user's query. The LLM leverages its pre-trained knowledge and linguistic abilities to interpret the query, understand the retrieved context, and construct a response that is both accurate and fluent.

In essence, the basic RAG pipeline operates sequentially: a user query is received, the retriever queries the knowledge base to fetch relevant documents, and these retrieved documents are then passed to the generator (LLM) along with the original query to produce a final answer. This modular design allows for independent optimization of each component. While this foundational architecture significantly enhances LLM capabilities, the effective utilization of retrieved information by the generator is not trivial, and the quality of retrieval directly impacts the generator's performance. The challenges inherent in this basic retrieve-then-generate paradigm, particularly concerning the generator's ability to robustly handle diverse retrieved contexts, are critical areas of ongoing research and will be further explored in the subsequent section.
\subsection{Early Challenges: Irrelevant Context and Hallucination}
\label{sec:2\_3\_early\_challenges:\_irrelevant\_context\_\_and\_\_hallucination}

Early Retrieval-Augmented Generation (RAG) systems, while promising, encountered significant hurdles as Large Language Models (LLMs) struggled to effectively process and integrate retrieved information, often leading to unreliable outputs and persistent hallucination. A primary challenge was the LLM's inability to robustly utilize or, crucially, reject noisy, irrelevant, or conflicting documents provided as context. This often resulted in the generation of unreliable content, where the LLM might hallucinate despite having access to factually correct information, or exhibit the 'lost-in-the-middle' phenomenon, overlooking critical details within long retrieved contexts.

Initial investigations into these limitations systematically diagnosed the specific failure modes of RAG systems. \cite{chen2023nzb} provided a foundational analysis by introducing the Retrieval-Augmented Generation Benchmark (RGB), designed to evaluate LLMs across four critical RAG abilities. Their findings highlighted that LLMs exhibited only limited \textit{Noise Robustness}, frequently confusing similar information within noisy documents and generating inaccurate answers even when relevant context was present. This directly contributed to the problem of irrelevant context leading to unreliable generation. Furthermore, the study revealed a significant deficiency in \textit{Negative Rejection}, where LLMs often failed to decline answering when no relevant information was available in the external documents, instead resorting to hallucination. The paper also pointed out LLMs' struggles with \textit{Information Integration}, demonstrating a lack of ability to synthesize coherent answers from multiple retrieved documents, indicating poor intelligent context management. Perhaps most critically, \cite{chen2023nzb} observed that LLMs often prioritized and trusted \textit{incorrect retrieved information} over their own internal knowledge, even when explicitly warned about potential risks, underscoring a profound challenge in \textit{Counterfactual Robustness} that directly contributed to factual errors and hallucination.

Building upon this general understanding of RAG's initial limitations, subsequent research extended the diagnostic rigor to high-stakes domains, further quantifying and identifying specific manifestations of these challenges. \cite{xiong2024exb} addressed the critical need for trustworthy and accurate LLMs in medical question answering, where hallucinations and outdated knowledge pose severe risks. Their work introduced the MIRAGE benchmark and MEDRAG toolkit, providing the first systematic evaluations of RAG systems in medicine under realistic "Question-Only Retrieval" settings. This comprehensive benchmarking effort not only confirmed the persistent problem of LLM unreliability but also empirically identified the "lost-in-the-middle" phenomenon within medical RAG. This phenomenon, where the position of ground-truth snippets significantly impacts performance, vividly illustrates the LLM's struggle to effectively leverage critical information embedded within long or complex retrieved contexts, often overlooking key details. The systematic evaluation by \cite{xiong2024exb} underscored that despite RAG's potential, the nuanced challenges of context utilization and hallucination remained prevalent, necessitating more sophisticated approaches to ensure accuracy in sensitive applications.

These early investigations collectively revealed fundamental shortcomings in LLMs' ability to intelligently process, filter, and integrate retrieved information. The observed difficulties with noise robustness, negative rejection, information integration, counterfactual robustness, and the "lost-in-the-middle" effect highlighted that simple retrieval and concatenation of documents were insufficient. These issues underscored a pressing need for more sophisticated RAG mechanisms capable of robust and accurate information integration, moving beyond basic retrieval to intelligent context management, utilization, and robust rejection of irrelevant or conflicting data.


\label{sec:advanced_rag_architectures_and_optimizations}

\section{Advanced RAG Architectures and Optimizations}
\label{sec:advanced\_rag\_architectures\_\_and\_\_optimizations}

\subsection{Pre-Retrieval Optimizations: Query Refinement and Expansion}
\label{sec:3\_1\_pre-retrieval\_optimizations:\_query\_refinement\_\_and\_\_expansion}

Initial Retrieval-Augmented Generation (RAG) systems often suffer from suboptimal retrieval due to ambiguous, overly broad, or complex user queries. These raw queries can lead to the fetching of irrelevant or insufficient context, thereby degrading the quality and factual accuracy of the Large Language Model's (LLM) output. Pre-retrieval optimizations are a critical class of techniques designed to enhance the initial user query before it interacts with the retriever, ensuring more precise and effective document fetching \cite{gao20238ea, huang2024a59, zhao2024931}. This section explores various strategies, including query expansion, rewriting, and decomposition, which proactively address the limitations of naive query formulation.

The landscape of RAG optimizations, as systematically categorized by surveys like \cite{gao20238ea} and \cite{huang2024a59}, highlights query manipulation as a foundational pre-retrieval step. These methods aim to transform the original query into a more retriever-friendly format.

\subsubsection*{Query Expansion}
Query expansion involves augmenting the original query with additional terms to broaden the search scope and capture more relevant documents, especially when the initial query is too concise or uses non-standard terminology. Traditional methods might include adding synonyms, related concepts, or thematic terms. More advanced approaches leverage LLMs to generate multiple reformulations or perspectives of the original query. For instance, for broad, open-ended queries that encompass diverse sub-intents, systems like RichRAG \cite{wang20245w8} employ a "sub-aspect explorer" to identify potential sub-aspects. This effectively expands the query by generating related facets, which are then used by a "multi-faceted retriever" to build a diverse candidate pool of documents. This intelligent expansion ensures that the final generated response is rich and comprehensive, covering various relevant aspects of the user's implicit information need.

\subsubsection*{Query Rewriting and Transformation}
Query rewriting focuses on rephrasing the original query to improve its compatibility with the underlying search index and retrieval model. A prominent technique in this category is Hypothetical Document Embeddings (HyDE), which involves generating a hypothetical, yet plausible, answer to the original query using an LLM \cite{gao20238ea}. This hypothetical document is then embedded, and its embedding is used to retrieve actual documents from the knowledge base. The rationale is that the embedding of a well-formed hypothetical answer is often semantically closer to relevant documents than the embedding of a short, potentially ambiguous query. While effective for dense retrieval, a limitation of HyDE is its reliance on the LLM's ability to generate a non-hallucinatory hypothetical answer, which can be challenging for complex or out-of-domain queries.

Moving beyond generating hypothetical documents, recent advancements have focused on explicit, learned query refinement. RQ-RAG \cite{chan2024u69} represents a significant innovation by training an LLM to dynamically refine search queries through rewriting, decomposition, and disambiguation \textit{during} the generation process. This end-to-end learned approach directly addresses ambiguous or complex initial queries by enabling the LLM to proactively generate more effective search terms. RQ-RAG's novelty lies in its dataset construction pipeline, which leverages a powerful external LLM (e.g., ChatGPT) to craft tailored search queries for specific refinement scenarios and to regenerate contextually aligned answers. Furthermore, it incorporates internal trajectory selection strategies (e.g., Perplexity, Confidence, Ensemble-based) to navigate multi-path query refinement at inference time without relying on external LLMs for decision-making. Experimental validation demonstrated RQ-RAG's superior performance on both single-hop and multi-hop QA tasks, significantly outperforming prior RAG methods and even larger proprietary LLMs, showcasing the power of explicit, learned query refinement.

Complementing these, dynamic retrieval systems like DRAGIN \cite{su20241om} further refine queries by actively deciding \textit{when} and \textit{what} to retrieve based on the LLM's real-time information needs during text generation. This approach moves beyond static rules, crafting appropriate queries that span the entire context rather than just the most recent sentence, offering a more adaptive form of query rewriting.

\subsubsection*{Query Decomposition}
For complex questions, especially those requiring multi-hop reasoning, query decomposition is a crucial strategy. This involves breaking down a single, intricate query into a series of simpler, more manageable sub-questions, each of which can be answered by a separate retrieval step. This approach mitigates the "lost-in-the-middle" problem, where LLMs struggle to synthesize information from long, complex retrieved contexts. For instance, IterDRAG \cite{yue2024ump} is designed for complex multi-hop queries, decomposing the input into simpler sub-queries and iteratively performing retrieval and generating intermediate answers to construct reasoning chains. Similarly, in practical applications, systems described by \cite{hikov2024rme} utilize a "sub-questions engine" for in-depth analysis of documents, effectively demonstrating query decomposition for comprehensive information extraction. RichRAG \cite{wang20245w8}, by identifying "sub-aspects" for broad queries, also implicitly performs a form of decomposition, allowing for targeted retrieval for each identified facet.

\subsubsection*{Critical Analysis and Comparison}
The evolution of pre-retrieval optimizations reflects a shift from simpler keyword-based or rule-based expansions to sophisticated, LLM-driven, and often learned approaches. While methods like HyDE offer a computationally lighter way to transform queries for dense retrieval, they rely on the quality of a single hypothetical generation. In contrast, systems like RQ-RAG \cite{chan2024u69} and DRAGIN \cite{su20241om} represent a more explicit and dynamic form of query refinement, capable of handling ambiguity, complexity, and multi-hop reasoning through learned rewriting and decomposition. The trade-off often lies in computational cost and data requirements; training an end-to-end model like RQ-RAG requires a carefully constructed dataset, whereas HyDE can be applied with a pre-trained LLM. However, the benefits of learned refinement, particularly for multi-hop tasks where RQ-RAG demonstrated significant gains, often outweigh these costs by delivering substantially more accurate and comprehensive responses. The integration of query decomposition, as seen in IterDRAG \cite{yue2024ump} and RichRAG \cite{wang20245w8}, is particularly vital for overcoming the limitations of single-shot retrieval on complex information needs.

The progression from diagnosing RAG limitations to systematically categorizing and then implementing explicit, learned query refinement techniques marks a significant advancement in RAG systems. Future research could focus on making these sophisticated refinement processes more computationally efficient, robust to noisy inputs, and universally applicable across diverse domains and a wider array of RAG-enabled NLP tasks beyond question answering.
\subsection{Post-Retrieval Optimizations: Reranking and Context Compression}
\label{sec:3\_2\_post-retrieval\_optimizations:\_reranking\_\_and\_\_context\_compression}

After the initial document retrieval phase in Retrieval-Augmented Generation (RAG) systems, the raw set of fetched documents often contains noise, redundancy, or suboptimal ordering, which can significantly hinder the Large Language Model (LLM)'s ability to generate accurate and coherent responses. Post-retrieval optimizations, primarily reranking and context compression, are therefore indispensable strategies. These techniques aim to refine the initially retrieved documents, ensuring the LLM receives the most concise, impactful, and relevant context for generation \cite{huang2024a59, verma2024f91}. Such optimizations are vital for mitigating issues like information overload, the 'lost-in-the-middle' phenomenon, and the inclusion of irrelevant data that can degrade response quality.

The necessity for these optimizations is underscored by empirical observations of LLMs' struggles with raw retrieved contexts. Research by \cite{chen2023nzb} systematically benchmarked LLMs in RAG scenarios, revealing significant shortcomings across fundamental abilities such as Noise Robustness, Negative Rejection, Information Integration, and Counterfactual Robustness. Their findings indicated that LLMs often struggle to extract information from noisy documents, frequently fail to reject answering when no relevant information is present, and lack the ability to effectively synthesize information from multiple documents. Furthermore, LLMs tend to prioritize retrieved (even incorrect) information over their internal knowledge, especially when explicitly warned, highlighting a critical need for more precise and reliable context presentation. A particularly salient issue is the "lost-in-the-middle" phenomenon, where LLMs perform best when relevant information is located at the beginning or end of a long context, but their performance degrades significantly when it is buried in the middle \cite{xiong2024exb}. These identified weaknesses directly motivate the development of post-retrieval mechanisms to filter out noise, prioritize truly relevant information, and strategically position it within the LLM's context window.

\subsubsection{Reranking Mechanisms}
Reranking emerges as a direct response to these challenges, aiming to reorder the initially retrieved documents based on a more refined assessment of their pertinence to the query. Traditionally, RAG pipelines often employed separate expert ranking models, typically cross-encoders based on architectures like BERT or T5, to refine initial retrieval results \cite{huang2024a59, rau20244nr}. These models would take the query and each retrieved document (or passage) as input and output a relevance score, allowing for a more nuanced reordering than the initial retriever's scores. While effective, these traditional rerankers often lacked the zero-shot generalization capabilities of modern LLMs and added architectural complexity to the RAG pipeline.

A significant advancement in this area is presented by \cite{yu202480d} with \textbf{RankRAG}, which unifies context ranking and answer generation within a single instruction-tuned LLM. This innovative framework addresses the limitations of previous multi-component approaches by training one LLM to perform both reranking and generation, thereby streamlining the "Retrieve-Rerank-Generate" inference pipeline. The technical contribution of RankRAG lies in its specialized instruction-tuning framework, which includes a novel task for context ranking framed as a simple question-answering problem where the LLM learns to identify context relevance. This task is seamlessly integrated into a comprehensive data blending strategy that combines context-rich QA, retrieval-augmented QA with hard negatives, and dedicated context ranking datasets. Remarkably, \cite{yu202480d} demonstrates that incorporating even a small fraction of this specialized ranking data into the instruction-tuning blend yields superior ranking performance, often outperforming LLMs exclusively fine-tuned on significantly larger ranking datasets. This unification not only simplifies the RAG architecture but also leverages the LLM's inherent multi-task capabilities, leading to superior zero-shot generation performance across diverse benchmarks, even surpassing models like GPT-4.

While RankRAG offers a compelling unified approach, it's important to consider potential trade-offs. Training a single model for multiple tasks might introduce complexities in balancing performance across different objectives, and the instruction-tuning process itself can be resource-intensive. Other approaches also explore LLM-native reranking. For instance, \cite{omrani2024i22} propose a hybrid RAG framework that includes an "innovative re-ranking mechanism" to optimize query response capabilities, suggesting that various LLM-driven or hybrid reranking strategies are being explored to enhance relevance and information fidelity. By providing a more precise and relevant set of top-$k$ contexts, reranking directly mitigates the issues of noise and information overload highlighted by \cite{chen2023nzb}, improving the LLM's ability to utilize the retrieved information effectively and directly counteracting the 'lost-in-the-middle' phenomenon by placing the most critical information prominently.

\subsubsection{Context Compression Techniques}
Complementary to reranking, context compression is another vital post-retrieval optimization, focusing on condensing the information within retrieved documents to fit within the LLM's context window while retaining critical details \cite{verma2024f91}. This is crucial not only for managing token limits but also for reducing cognitive load on the LLM, improving its focus, and mitigating the 'lost-in-the-middle' effect by presenting a denser, more signal-rich context. Context compression techniques can be broadly categorized into extractive, abstractive, and filtering methods \cite{verma2024f91, huang2024a59}.

\begin{itemize}
    \item \textbf{Extractive Compression:} These methods identify and extract the most important sentences, phrases, or passages from the retrieved documents. Techniques might involve scoring sentences based on their relevance to the query, salience, or novelty, and then selecting a subset. For example, \cite{omrani2024i22} mentions "Sentence-Window and Parent-Child methodologies" as part of their hybrid RAG, which are forms of extractive context management where smaller, relevant windows are used for initial retrieval, and then expanded to parent documents if more context is needed.
    \item \textbf{Abstractive Compression:} This involves summarizing the retrieved content into a more condensed form using generative models. While more complex, abstractive methods can produce highly coherent and concise summaries, potentially integrating information across multiple documents. However, they also introduce the risk of hallucination from the summarizer itself.
    \item \textbf{Filtering:} This is the most basic form of compression, where documents or passages deemed irrelevant by a reranker or a dedicated filter are simply discarded. This is implicitly performed by rerankers that select only the top-$k$ documents.
\end{itemize}

A notable example of an LLM-native approach to context compression is presented by \cite{zhu2024h7i} with \textbf{Sparse RAG}. This paradigm aims to accelerate inference and cut computation costs by encoding retrieved documents in parallel and then having the LLM selectively decode the output by only attending to "highly relevant caches" auto-regressively, chosen via prompting with special control tokens. This mechanism combines the assessment of individual document relevance with the generation process, effectively acting as a dynamic, LLM-driven sparse context selection. By filtering out undesirable contexts and focusing on relevant parts, Sparse RAG inherently improves generation quality and computational efficiency. The survey by \cite{verma2024f91} further details the evolution of contextual compression paradigms, highlighting the continuous efforts to balance information retention with context window constraints and processing overhead.

In conclusion, post-retrieval optimizations, encompassing both reranking and context compression, are indispensable for enhancing the robustness, accuracy, and efficiency of RAG systems. The diagnostic work by \cite{chen2023nzb} clearly articulated the challenges LLMs face with raw retrieved contexts, paving the way for sophisticated solutions. Reranking, from traditional cross-encoders to unified LLM-native frameworks like RankRAG \cite{yu202480d}, ensures that the most pertinent information is prioritized. Concurrently, context compression, through techniques ranging from extractive methods to LLM-driven sparse selection \cite{zhu2024h7i}, ensures this vital information is presented concisely and effectively within the LLM's operational context. Future directions in this domain will likely explore more sophisticated, adaptive, and LLM-native context management techniques, further integrating these post-retrieval steps to balance inference efficiency with the comprehensive and accurate processing of vast and complex information.
\subsection{Adaptive and Self-Correcting RAG Frameworks}
\label{sec:3\_3\_adaptive\_\_and\_\_self-correcting\_rag\_frameworks}

The effectiveness of Retrieval-Augmented Generation (RAG) systems is inherently susceptible to the quality of retrieved information, posing a significant challenge to factual accuracy and user confidence. This subsection examines the evolution towards self-correcting RAG frameworks, which incorporate sophisticated mechanisms for dynamic adaptation and proactive recovery from unreliable retrieval outcomes.

Early investigations into RAG's robustness revealed critical vulnerabilities. \cite{chen2023nzb} conducted rigorous benchmarking, identifying that Large Language Models (LLMs) augmented with RAG often struggle with "Noise Robustness," "Negative Rejection," "Information Integration," and "Counterfactual Robustness." Their findings demonstrated that LLMs frequently fail to reject answering when no relevant information is available, instead generating incorrect responses, and alarmingly, tend to trust and prioritize retrieved (incorrect) information even when possessing accurate internal knowledge or explicit warnings \cite{chen2023nzb}. These empirical insights underscored the urgent need for RAG systems to actively assess and correct retrieval quality rather than passively accepting it.

Building upon the foundational understanding of RAG architectures and their limitations, as comprehensively surveyed by \cite{gao20238ea}, the field began to explore more adaptive approaches. \cite{gao20238ea} categorized RAG into Naive, Advanced, and Modular paradigms, detailing various optimization methods for retrieval, such as query rewriting, reranking, and context compression, and highlighting the emergence of adaptive retrieval flows like Self-RAG. While these advancements aimed to improve retrieval precision and recall, they largely focused on \textit{optimizing} the initial retrieval process, leaving a critical gap: \textit{what to do when the initial retrieval inevitably fails or provides suboptimal information}.

This critical gap was directly addressed by \cite{yan202437z} with the introduction of Corrective Retrieval Augmented Generation (CRAG), marking a significant methodological shift towards true self-correction. CRAG proposes a dynamic, plug-and-play framework designed to actively assess the quality of retrieved documents and trigger corrective actions. At its core, CRAG employs a lightweight retrieval evaluator, a fine-tuned T5-large model, to quantify the confidence and relevance of retrieved documents to the input query \cite{yan202437z}. Based on this evaluation, a dynamic multi-action trigger initiates one of three strategies: "Correct" (if relevant documents are found, applying knowledge refinement), "Incorrect" (if documents are irrelevant, discarding them and initiating large-scale web searches for correction), or "Ambiguous" (for intermediate confidence scores, combining knowledge refinement with web search results) \cite{yan202437z}. Furthermore, CRAG refines knowledge utilization through a "decompose-then-recompose" algorithm, segmenting relevant documents into fine-grained "knowledge strips" and filtering out irrelevant portions to recompose the most critical information \cite{yan202437z}. This adaptive knowledge acquisition strategy, integrated directly into the generation process, represents a crucial leap towards more robust and trustworthy RAG systems, capable of proactively addressing and recovering from unreliable retrieval outcomes.

In parallel, another significant development, exemplified by the capabilities described in \cite{amugongo202530u} regarding models like Gemini 1.5, presents an alternative or complementary paradigm for robust knowledge access. While not a RAG framework in the traditional sense, this approach focuses on dramatically expanding the \textit{native context window} of the LLM itself, enabling it to process up to 10 million tokens across multimodal inputs \cite{amugongo202530u}. This allows the model to internalize vast amounts of information directly, achieving near-perfect recall within this massive context. Such advancements challenge the exclusive reliance on external retrieval for extensive knowledge access, suggesting a future where hybrid architectures might combine the deep contextual understanding of models with vast native windows with the dynamic, self-correcting external retrieval capabilities of frameworks like CRAG.

In conclusion, the evolution towards adaptive and self-correcting RAG frameworks represents a critical advancement in making LLMs more reliable and factually accurate. From diagnosing fundamental RAG limitations to developing sophisticated mechanisms for dynamic assessment and corrective action, the field is moving towards more intelligent knowledge integration. While frameworks like CRAG provide robust solutions for handling unreliable external retrieval, the concurrent development of LLMs with vastly expanded native context windows suggests a future where synergistic approaches, combining both internal and external knowledge management, will define the next generation of intelligent systems.


\label{sec:benchmarking_and_evaluation_of_rag_systems}

\section{Benchmarking and Evaluation of RAG Systems}
\label{sec:benchmarking\_\_and\_\_evaluation\_of\_rag\_systems}

\subsection{General Benchmarking Frameworks for RAG Capabilities}
\label{sec:4\_1\_general\_benchmarking\_frameworks\_for\_rag\_capabilities}

Systematically evaluating the diverse capabilities of Retrieval-Augmented Generation (RAG) systems is paramount for understanding their strengths, diagnosing weaknesses, and guiding future development. Moving beyond simplistic accuracy metrics, comprehensive benchmarking frameworks are essential to assess the nuanced abilities of RAG systems when interacting with external knowledge bases. These frameworks aim to diagnose specific RAG capabilities, such as robustness to noisy or irrelevant information, the capacity for negative rejection (i.e., identifying when no answer can be derived from the provided context), the ability to effectively integrate information from multiple retrieved sources, and counterfactual robustness.

A foundational contribution in this area is the Retrieval-Augmented Generation Benchmark (RGB) introduced by \cite{chen2023nzb}. This work addresses the critical gap in rigorous, systematic evaluation of RAG's impact on Large Language Models (LLMs), which often struggle to effectively utilize or reject retrieved information. RGB is a novel, dual-language benchmark specifically designed to diagnose four fundamental RAG capabilities: Noise Robustness, Negative Rejection, Information Integration, and Counterfactual Robustness. For instance, to evaluate Noise Robustness, RGB constructs testbeds where relevant information is embedded alongside irrelevant or distracting content. For Negative Rejection, it presents queries where no answer can be derived from the provided context, assessing the system's ability to abstain. The benchmark's data generation pipeline leverages real-world news and search APIs to create challenging scenarios, revealing that LLMs often confuse similar information, fail to reject answers when no context is present, struggle to integrate information from multiple sources, and even prioritize incorrect retrieved information over their own internal knowledge, even when explicitly warned \cite{chen2023nzb}. The diagnostic power of RGB is crucial for identifying specific bottlenecks in RAG systems, providing empirical evidence of where current RAG paradigms fall short and motivating the development of more robust architectures.

Building upon the need for comprehensive and explainable evaluation, \cite{friel20241ct} introduced RAGBench, a large-scale (100k examples), multi-domain, multi-task RAG benchmark. RAGBench differentiates itself by providing a unified, standardized dataset sourced from real-world corpora across five industry-specific domains (e.g., bio-medical, legal, finance) and various RAG task types (e.g., numerical reasoning, multi-document retrieval). Crucially, it proposes the TRACe evaluation framework, which formalizes four explainable and actionable metrics: Context Relevance, Context Utilization (novel), Completeness (novel), and Adherence (faithfulness). While RGB provides high-level diagnostic categories, RAGBench offers granular, span-level annotations for metrics like utilization and relevance, enabling a deeper understanding of \textit{how} the generator leverages the retrieved context. This focus on explainability and actionable insights helps pinpoint specific areas for improvement in RAG components \cite{friel20241ct}.

Further advancing the standardization of RAG evaluation, \cite{rau20244nr} presented BERGEN, an end-to-end library for reproducible research in RAG. BERGEN aims to standardize RAG experiments by providing a systematic evaluation framework for different state-of-the-art retrievers, rerankers, and LLMs across various datasets and collections. Unlike benchmarks that focus on specific RAG capabilities, BERGEN emphasizes the reproducibility and comparability of experimental results, which is vital for understanding the impact of each component in the RAG pipeline. This library approach facilitates consistent benchmarking across the rapidly evolving RAG landscape, addressing the challenge of inconsistent evaluation methodologies \cite{rau20244nr}.

The increasing complexity of RAG systems has also led to the development of modular toolkits that inherently support benchmarking. FlashRAG \cite{jin2024yhb} is one such modular toolkit designed for efficient RAG research. It offers a hierarchical, modular architecture that pre-implements 16 advanced RAG algorithms across various process flows (sequential, branching, conditional, loop). By providing standardized datasets and corpora, along with an intuitive visual web interface, FlashRAG enables researchers to easily reproduce, benchmark, and compare diverse RAG methods. Similarly, AutoRAG \cite{kim2024t1i} proposes an automated framework that identifies suitable RAG modules for a given dataset by exploring and approximating optimal combinations. These toolkits, while not solely benchmarks, provide the infrastructure and standardized environments necessary for systematic, comparative evaluation of different RAG configurations and components, thereby contributing significantly to general benchmarking efforts.

Beyond performance metrics, a holistic view of RAG evaluation necessitates considering trustworthiness. \cite{zhou20248fu} provides a comprehensive survey and proposes a unified framework for assessing trustworthiness in RAG systems, defining six critical dimensions: Factuality, Robustness, Fairness, Transparency, Accountability, and Privacy. This framework highlights that general RAG benchmarks should aspire to cover these broader ethical and practical considerations, moving beyond mere accuracy or utility to ensure responsible deployment. The paper also outlines a plan for an evaluation benchmark across these dimensions, emphasizing the need for a more comprehensive understanding of RAG system behavior in real-world, high-stakes applications \cite{zhou20248fu}.

Furthermore, the methodologies underpinning automated RAG evaluation are continuously being refined. Many modern benchmarking frameworks rely on LLMs as judges to assess response quality. However, these LLM-based judgments can be highly sensitive to evaluation prompts, leading to inconsistencies. To address this, \cite{liu2025sy0} introduced Judge-Consistency (ConsJudge), a method that enhances LLMs to generate more accurate and consistent evaluations for RAG models. ConsJudge prompts LLMs to generate judgments based on various combinations of dimensions, utilizing judge-consistency to select accepted and rejected judgments for DPO training. This improvement in LLM-based judgment reliability is crucial for the accuracy and robustness of automated benchmarking frameworks, ensuring that evaluations are fair and dependable \cite{liu2025sy0}.

In conclusion, general benchmarking frameworks for RAG capabilities have evolved significantly, moving from initial diagnostic tools like RGB, which pinpoint fundamental weaknesses, to large-scale, explainable benchmarks like RAGBench that offer granular insights into context utilization. Libraries like BERGEN and toolkits such as FlashRAG and AutoRAG provide the necessary infrastructure for reproducible and automated comparative evaluation of RAG components and pipelines. The field is also expanding to encompass broader trustworthiness dimensions, as proposed by \cite{zhou20248fu}, and refining the underlying LLM-as-a-judge methodologies with approaches like ConsJudge. This continuous development of comprehensive, adaptive, and trustworthy benchmarking frameworks is crucial for driving innovation and ensuring the reliability and effectiveness of RAG systems in diverse and demanding applications.
\subsection{Evaluating Retrieval Quality: Utility-Aligned Metrics}
\label{sec:4\_2\_evaluating\_retrieval\_quality:\_utility-aligned\_metrics}

The effective evaluation of retrieval quality within Retrieval-Augmented Generation (RAG) systems is critical for their optimization, yet it presents unique challenges that necessitate a departure from traditional relevance assessment. This subsection explores advanced methodologies that directly measure a document's actual utility to the consuming RAG Large Language Model (LLM), moving beyond external judgments to foster more accurate and transparent feedback.

Historically, evaluating the retrieval component of RAG systems has largely relied on metrics that often prove insufficient or misaligned with the LLM's true needs. Traditional information retrieval (IR) metrics, such as Precision@k, Recall, Mean Average Precision (MAP), or Normalized Discounted Cumulative Gain (NDCG), typically assess document relevance based on human annotations or keyword matching \cite{huang2024a59}. While valuable for standalone IR systems, these metrics often fail to capture whether a retrieved document genuinely contributes to the LLM's ability to generate a correct, comprehensive, and non-hallucinatory answer \cite{salemi2024om5, friel20241ct}. What is deemed "relevant" by human annotators or simple keyword overlap may not translate into actual utility for the LLM, leading to suboptimal retriever development. Furthermore, traditional end-to-end RAG evaluations, while comprehensive in assessing the final output, are computationally expensive and offer only list-level feedback, lacking the granularity to pinpoint which specific documents contribute to the final output \cite{salemi2024om5, rau20244nr}. This opacity makes it difficult to effectively optimize retriever models. The use of external LLMs as binary judges for relevance also introduces additional computational costs, memory constraints, and the potential for a mismatch between the judging LLM and the RAG LLM itself, alongside sensitivity to evaluation prompts \cite{salemi2024om5, liu2025sy0}.

Addressing these critical limitations, a paradigm shift towards "utility-aligned metrics" has emerged, redefining relevance as a document's direct contribution to the RAG LLM's downstream task performance. These methodologies aim to intrinsically link retrieval quality to the generator's actual usage and the final output's accuracy.

One prominent approach is eRAG, introduced by \cite{salemi2024om5}. This methodology directly measures a document's value by leveraging the \textit{RAG system's own LLM} as the arbiter for generating document-level relevance labels. Instead of processing all retrieved documents simultaneously, eRAG feeds each document in the retrieval list individually to the LLM, alongside the query. The LLM's output for each single document is then evaluated against the downstream task's ground truth labels (e.g., Exact Match for question answering or F1 for generation). This performance score for each individual document serves as its utility-aligned relevance label, which can then be aggregated using standard set-based or ranking metrics like MAP or NDCG to assess the overall retrieval list \cite{salemi2024om5}. eRAG offers several technical advantages: it provides a novel paradigm for evaluating retrieval quality by intrinsically aligning relevance labels with how the LLM actually utilizes the retrieved information, offering more accurate feedback for retriever optimization. Moreover, it boasts substantial computational advantages, consuming up to 50 times less GPU memory and improving runtime compared to traditional end-to-end RAG evaluation methods, making it a practical and scalable tool \cite{salemi2024om5}. Experimental validation across various knowledge-intensive language tasks from the KILT benchmark demonstrated that eRAG consistently achieves the highest correlation with downstream RAG performance, with absolute improvements in Kendall's $\tau$ ranging from 0.168 to 0.494 over baselines \cite{salemi2024om5}.

Beyond eRAG, other utility-aligned metrics further emphasize the direct measurement of context utilization. The TRACe evaluation framework, proposed by \cite{friel20241ct}, introduces several metrics, notably "Context Utilization." This metric specifically measures the fraction of retrieved context \textit{actually used} by the generator in formulating its response. Unlike eRAG, which assesses a document's potential utility when presented alone, Context Utilization directly observes the LLM's behavior when processing the full context, often relying on span-level annotations to identify which parts of the retrieved text are incorporated into the output. Similarly, for long-form generation tasks, \cite{qi2024tlf} introduces the Key Point Recall (KPR) metric within the Long$^2$RAG benchmark. KPR evaluates the extent to which LLMs incorporate key points extracted from the retrieved documents into their generated responses, providing a nuanced assessment of their ability to exploit retrieved information for comprehensive answers. This also aligns with the "utility-oriented thought" proposed by MetRag, which moves beyond mere similarity to consider a document's direct usefulness for the generation task \cite{gan2024id0}.

A critical aspect of these LLM-centric evaluation methods is the reliability of the LLM acting as a judge. \cite{liu2025sy0} highlights that while LLM-based judgment models have the potential for high-quality evaluations, they are highly sensitive to evaluation prompts, leading to inconsistencies. To address this, they introduce the Judge-Consistency (ConsJudge) method, which enhances LLMs to generate more accurate evaluations for RAG models by prompting LLMs to generate different judgments based on various combinations of judgment dimensions, utilizing judge-consistency to evaluate these judgments, and selecting accepted/rejected judgments for DPO training. This approach is crucial for improving the robustness and trustworthiness of any utility-aligned metric that relies on LLMs for annotation or judgment, such as eRAG's document-level scoring or TRACe's ground truth generation via GPT-4 \cite{salemi2024om5, friel20241ct}.

Collectively, these utility-aligned metrics represent a significant advancement in RAG evaluation. They move beyond superficial relevance to provide a deeper, more actionable understanding of how retrieved information truly impacts the LLM's performance. While eRAG, Context Utilization, and KPR each offer distinct mechanisms for assessing utility—from individual document contribution to actual usage in generation—they all share the common goal of bridging the gap between retrieval and generation. However, limitations persist; eRAG still relies on the availability of ground truth labels for the downstream task, which may not be universally available \cite{salemi2024om5}. Additionally, by evaluating documents individually, eRAG might not fully capture complex synergistic interactions that could occur when an LLM processes multiple documents simultaneously \cite{salemi2024om5}. Future research directions could explore methods to incorporate potential synergistic effects of multiple documents while maintaining computational efficiency, and to extend these utility-aligned evaluation frameworks to scenarios where explicit ground truth labels are scarce or where LLM judges can be further refined to ensure consistent and unbiased assessments. The development of standardized benchmarks like RAGBench \cite{friel20241ct} and BERGEN \cite{rau20244nr} will continue to be vital in rigorously testing and comparing these evolving utility-aligned evaluation methodologies.
\subsection{Benchmarking for Complex Reasoning: Multi-Hop Queries}
\label{sec:4\_3\_benchmarking\_for\_complex\_reasoning:\_multi-hop\_queries}

Evaluating Retrieval-Augmented Generation (RAG) systems for complex reasoning tasks, particularly those requiring intricate information synthesis across multiple documents, presents a significant challenge that moves beyond simple, single-fact retrieval. Traditional RAG evaluation often focused on "single-hop" queries, where answers could typically be found within a single document or a directly relevant passage, as seen in earlier benchmarks like RGB \cite{chen2023rgb} and RECALL \cite{liu2023recall}. However, real-world information needs frequently necessitate synthesizing disparate pieces of information, performing multi-step inferences, and comparing facts across various sources.

Addressing this critical gap, the \textit{MultiHop-RAG} benchmark \cite{tang2024i5r} represents a pivotal advancement in assessing RAG systems' capabilities for complex reasoning. This work highlights the inadequacy of existing benchmarks for handling "multi-hop queries," which demand retrieving and reasoning over multiple, often disparate, pieces of supporting evidence to formulate a comprehensive answer. The challenge lies not only in accurately retrieving all relevant evidence but also in enabling Large Language Models (LLMs) to perform sophisticated reasoning, such as inference, comparison, and temporal analysis, across these multiple pieces of evidence. Crucially, \textit{MultiHop-RAG} also evaluates the system's capacity to identify when an answer cannot be derived from the available evidence (Null queries), a vital feature for mitigating hallucinations in complex scenarios.

\textit{MultiHop-RAG} \cite{tang2024i5r} introduces a novel dataset and framework specifically designed to evaluate RAG systems on these demanding multi-hop queries. Its core innovation lies in a sophisticated, multi-stage, GPT-4-driven pipeline for data generation. This pipeline programmatically identifies "bridge-entities" and "bridge-topics" that connect disparate pieces of evidence, thereby enabling the creation of truly multi-hop queries that require advanced multi-document synthesis. The dataset leverages a knowledge base of recent news articles, ensuring that the information is external to common LLM pre-training data and mimicking real-world RAG deployment scenarios.

Furthermore, \textit{MultiHop-RAG} \cite{tang2024i5r} refines the methodological approach to evaluating reasoning complexity by introducing a novel categorization of multi-hop queries into four distinct types: Inference, Comparison, Temporal, and Null. Inference queries test the LLM's ability to deduce new information from multiple facts; Comparison queries require contrasting information from different sources; Temporal queries involve understanding and synthesizing events across a timeline; and Null queries assess the system's robustness in identifying unanswerable questions, which is paramount for reducing factual inaccuracies and hallucinations. This granular framework provides a more comprehensive and nuanced assessment of RAG systems than prior single-hop evaluations.

The empirical evaluations conducted using \textit{MultiHop-RAG} \cite{tang2024i5r} have revealed significant shortcomings in current state-of-the-art RAG systems. Both retrieval and generation components of existing RAG implementations, including advanced LLMs, perform "unsatisfactorily" when confronted with these complex multi-hop reasoning tasks. This finding underscores the substantial performance gap that needs to be addressed in the field. The inclusion of Null queries in the benchmark is particularly impactful, as it pushes the development of RAG systems that can not only retrieve and synthesize information but also possess the meta-cognitive ability to recognize the limits of their knowledge base, thereby enhancing trustworthiness and mitigating the risk of generating fabricated content.

In conclusion, \textit{MultiHop-RAG} \cite{tang2024i5r} represents a pivotal shift in the intellectual trajectory of RAG evaluation, moving from basic information retrieval to complex multi-document reasoning. By establishing the first dedicated benchmark and framework for multi-hop queries, it empirically demonstrates the limitations of current RAG systems. This work serves as a critical catalyst for future research into more sophisticated retrieval mechanisms, advanced multi-document reasoning strategies, and robust hallucination mitigation techniques, ultimately driving the development of RAG systems capable of meeting real-world complex information needs.
\subsection{Domain-Specific Benchmarking: High-Stakes Applications}
\label{sec:4\_4\_domain-specific\_benchmarking:\_high-stakes\_applications}

The deployment of Large Language Models (LLMs) in high-stakes domains, such as medicine, law, and finance, mandates an unwavering commitment to trustworthiness, factual accuracy, and reliability. In these critical applications, the consequences of hallucinations, outdated knowledge, or imprecise information retrieval can be severe, ranging from misdiagnosis in healthcare to erroneous legal advice or financial decisions \cite{zhou20248fu}. While Retrieval-Augmented Generation (RAG) offers a promising paradigm to ground LLMs with verifiable external information and mitigate these risks, a significant historical gap has existed in the systematic and domain-specific evaluation of RAG systems tailored for these sensitive fields. This deficiency has hindered the establishment of robust best practices and reliable deployment strategies.

Addressing this crucial need, recent advancements have focused on developing specialized benchmarks that reflect the unique complexities and stringent requirements of high-stakes environments. In the medical domain, \cite{xiong2024exb} introduced a pivotal methodological progression with the \textbf{MIRAGE benchmark} (Medical Information Retrieval-Augmented Generation Evaluation) and the accompanying \textbf{MEDRAG toolkit}. Prior to this work, RAG research in biomedicine often involved LLM improvements for information-seeking, but evaluations frequently lacked the systematic rigor required for clinical settings \cite{liu2025p6t, low2025gjc}. Existing systematic evaluations in biomedicine typically focused on vanilla LLMs, overlooking the unique challenges and opportunities presented by RAG \cite{xiong2024exb}. MIRAGE directly tackles these limitations by providing the first systematic evaluations of RAG systems tailored for medicine, thereby resolving the problem of unreliable LLM outputs and the absence of standardized performance metrics in this domain.

MIRAGE is meticulously designed for medical RAG, incorporating 7,663 questions from five diverse medical QA datasets to ensure broad coverage of clinical scenarios \cite{xiong2024exb}. A key innovation in its methodology is the enforcement of realistic evaluation settings, most notably "Question-Only Retrieval" (QOR), where answer options are withheld during the retrieval phase. This approach accurately simulates real-world medical question-answering, where a clinician formulates a query without pre-existing answer choices, thus ensuring the benchmark's findings are more applicable to practical medical applications than some previous, less stringent evaluation methods \cite{xiong2024exb}. The \textbf{MEDRAG toolkit} complements MIRAGE by providing a comprehensive platform integrating novel corpora like StatPearls alongside established sources (PubMed, medical textbooks), a mix of lexical, general semantic, scientific, and biomedical-domain specific retrievers (e.g., BM25, MedCPT, enhanced by Reciprocal Rank Fusion), and a broad selection of LLMs \cite{xiong2024exb}. This robust framework enables the systematic evaluation of 41 distinct RAG configurations, utilizing Chain-of-Thought (CoT) prompting with prepended retrieved snippets, representing a significant leap in methodological rigor. Through extensive experimentation, \cite{xiong2024exb} demonstrated that RAG can improve LLM accuracy by up to 18\\% relative to CoT prompting alone, and notably, elevate the performance of smaller models like GPT-3.5 and Mixtral to rival GPT-4 (without RAG) on the MIRAGE benchmark. The study also yielded crucial empirical discoveries, such as the log-linear scaling property between model performance and the number of retrieved snippets, and the observation of the "lost-in-the-middle" phenomenon in medical RAG, providing practical guidelines for optimizing RAG system design. While comprehensive, the study acknowledged limitations, such as constraints on retriever diversity due to computational resources and less pronounced RAG improvements for highly complex examination questions where retrieving truly helpful snippets remains challenging \cite{xiong2024exb}. Other medical applications, such as the radiology-specific RAG developed by \cite{weinert2025cxo}, also demonstrate RAG's potential to improve LLM performance on examinations by providing citable, up-to-date information, further underscoring the value of domain-specific grounding.

Beyond medicine, the legal domain presents another high-stakes application where precise and verifiable information is paramount. \cite{pipitone2024sfx} introduced \textbf{LegalBench-RAG}, a benchmark specifically designed to evaluate the \textit{retrieval component} of RAG systems in legal contexts. Unlike general RAG benchmarks (e.g., RGB \cite{chen2023retrieval}) that often focus on overall generation quality or broad document recall, LegalBench-RAG emphasizes the extraction of \textit{minimal, highly relevant text segments} (snippets) from legal documents. This focus is critical because imprecise retrieval in legal settings can lead to exceeding LLM context windows, inducing hallucinations, and preventing the generation of precise citations—all significant risks in the legal industry \cite{pipitone2024sfx}. LegalBench-RAG innovates by meticulously tracing contexts from existing legal datasets (like LegalBench \cite{guha2023legalbench}) back to their original locations within a large legal corpus to deduce precise character index spans for relevant information. This provides a granular, human-annotated ground truth for retrieval quality, a significant departure from benchmarks that bypass the retrieval step or accept less precise document-level relevance. However, LegalBench-RAG primarily assesses retrieval within a single document, limiting its scope for evaluating multi-document reasoning or multi-hop retrieval across multiple sources \cite{pipitone2024sfx}.

Similarly, the financial domain demands high accuracy and up-to-date information. \cite{zhao2024go5} conducted a systematic investigation into optimizing RAG pipelines in the financial domain. Their work benchmarked six LLMs across 15 retrieval scenarios and nine prompts over two real-world financial datasets. While not introducing a new benchmark framework like MIRAGE or LegalBench-RAG, their study thoroughly discussed the impact of each RAG component (retrieval pipeline, prompts, generation models) on answer generation quality, formulating specific recommendations for RAG system design in finance \cite{zhao2024go5}. This focus on component-level impact analysis complements the broader benchmarking efforts by providing actionable insights for fine-tuning RAG systems for domain-specific performance.

Collectively, these domain-specific benchmarking efforts highlight a critical intellectual trajectory in RAG research: moving beyond general evaluations to address the unique data characteristics, user needs, and trustworthiness requirements of high-stakes applications. They underscore the necessity for realistic evaluation settings (e.g., QOR in MIRAGE), granular ground truth (e.g., precise snippet retrieval in LegalBench-RAG), and systematic component analysis (e.g., in financial RAG optimization). These advancements directly contribute to enhancing the factuality, robustness, and accountability of RAG systems, which are key dimensions of trustworthiness identified by surveys on RAG trustworthiness \cite{zhou20248fu}. While significant progress has been made, challenges persist, including the computational resources required for exhaustive evaluation, the complexity of multi-hop reasoning in highly nuanced domains, and the continuous need for diverse retrieval strategies that can adapt to evolving knowledge bases. Nevertheless, these domain-specific benchmarks establish a foundational framework for rigorously validating and optimizing RAG in critical applications, guiding future research towards more reliable and ethically sound AI deployments.


\label{sec:specialized_rag_for_structured_and_domain-specific_knowledge}

\section{Specialized RAG for Structured and Domain-Specific Knowledge}
\label{sec:specialized\_rag\_for\_structured\_\_and\_\_domain-specific\_knowledge}

\subsection{RAG with Knowledge Graphs for Enhanced Reasoning}
\label{sec:5\_1\_rag\_with\_knowledge\_graphs\_for\_enhanced\_reasoning}

Conventional Retrieval-Augmented Generation (RAG) systems, primarily relying on semantic similarity over flat text chunks, often encounter significant limitations when tasked with navigating complex, interconnected, and structured data. This inherent deficiency stems from their inability to fully capture and leverage the explicit structural and relational knowledge embedded within rich knowledge bases, leading to suboptimal retrieval accuracy and constrained reasoning capabilities, particularly for multi-hop questions or domain-specific inquiries. To address these challenges, advanced RAG architectures are increasingly integrating Knowledge Graphs (KGs), enabling Large Language Models (LLMs) to move beyond purely semantic search and exploit the inherent structure of information for more precise, context-aware, and scalable knowledge extraction. This paradigm shift enhances the LLM's capacity to "chat with its graph," facilitating deeper understanding and more accurate generation.

A compelling demonstration of KG-enhanced RAG is presented by \cite{xu202412d} in the context of customer service question answering. The authors critically observe that traditional RAG methods, by treating historical customer service tickets as unstructured plain text, inadvertently discard crucial intra-issue structures (e.g., problem-solution relationships within a single ticket) and inter-issue relations (e.g., connections between related tickets). This oversight leads to fragmented context, compromised retrieval accuracy, and reduced answer quality, as fixed-length text chunking often severs vital connections. To overcome this, \cite{xu202412d} proposes a novel methodology centered on constructing a dual-level KG. At the intra-issue level, individual tickets are meticulously parsed into structured trees, capturing the internal flow and components of each issue. These trees are then interconnected at the inter-issue level through both explicitly defined and implicitly inferred relations, forming a comprehensive, domain-specific graph. The construction employs a hybrid parsing approach, combining robust rule-based extraction for structured fields with LLM-based parsing for nuanced unstructured text, ensuring high fidelity in graph representation. Crucially, retrieval is transformed through an LLM-driven subgraph mechanism. This system parses natural language queries to identify named entities and intents, dynamically translating them into precise graph database queries (e.g., Cypher). This allows for the extraction of contextually and structurally pertinent subgraphs, rather than disconnected text snippets. The empirical results are notable, demonstrating a 77.6\\% improvement in Mean Reciprocal Rank (MRR) and a 0.32 improvement in BLEU score over conventional RAG baselines. The practical impact was further validated by its deployment at LinkedIn, which reported a 28.6\\% reduction in median per-issue resolution time, underscoring the tangible benefits of leveraging structured knowledge in real-world, domain-specific RAG applications.

Building upon the fundamental premise of leveraging structural information, \cite{he20248lp} extends the RAG paradigm to tackle the more generalized and complex challenge of "chatting with graphs" for arbitrary textual graphs, moving beyond the specific document structures explored by \cite{xu202412d}. \cite{he20248lp} identifies that applying RAG to general textual graphs often results in significant challenges, including increased hallucination, scalability issues when attempting to flatten large graphs into linear text, and a persistent failure to fully exploit crucial structural information. To address these systemic limitations, they introduce G-Retriever, posited as the first RAG approach specifically designed for general textual graphs. A core innovation of G-Retriever lies in its ingenious formulation of subgraph retrieval as a Prize-Collecting Steiner Tree (PCST) optimization problem. This formulation allows the system to retrieve graph portions that are not only contextually relevant to the query but also structurally coherent and interconnected, by explicitly considering neighborhood information and optimizing for both "prizes" (relevance of nodes/edges) and "costs" (path length/complexity). This sophisticated approach significantly enhances explainability and demonstrably mitigates hallucination compared to prompt-tuning baselines, which struggle to accurately recall entire graph structures from isolated embeddings. By seamlessly integrating Graph Neural Networks (GNNs) with LLMs and RAG, G-Retriever achieves superior performance and scalability across diverse textual graph tasks, as rigorously evaluated on their newly introduced GraphQA benchmark.

Collectively, both \cite{xu202412d} and \cite{he20248lp} signify a critical evolution in RAG, marking a decisive shift from purely semantic, document-centric retrieval to intelligent, structure-aware information extraction from KGs. While \cite{xu202412d} masterfully showcases the power of dual-level KGs and LLM-driven subgraph retrieval for enhancing RAG in specific document-rich domains like customer service, its methodology for KG construction and query translation is tailored to the structured nature of its input. In contrast, \cite{he20248lp}'s G-Retriever generalizes this concept to arbitrary textual graphs, offering a more broadly applicable framework. Its formalization of subgraph retrieval as a PCST optimization problem provides a robust, principled mechanism for balancing relevance and connectivity, which is crucial for maintaining graph integrity and reducing hallucination in complex, unstructured graph environments. Both approaches explicitly address the limitations of conventional RAG by enabling LLMs to effectively "chat with their graph," leading to more precise, context-aware, and scalable information extraction. Future development in this area will likely focus on automating robust KG construction from diverse unstructured data sources, developing more dynamic and efficient KG update mechanisms, and exploring advanced graph reasoning techniques to further enhance LLM capabilities in increasingly complex, relational knowledge environments.
\subsection{Adapting RAG for Complex Domain-Specific Guidelines}
\label{sec:5\_2\_adapting\_rag\_for\_complex\_domain-specific\_guidelines}

The application of Retrieval-Augmented Generation (RAG) in highly specialized, complex domains, particularly those involving intricate guidelines like clinical protocols in healthcare or legal statutes, presents unique and formidable challenges. Ensuring the factual accuracy, reliability, and trustworthiness of Large Language Models (LLMs) in such high-stakes environments necessitates advanced strategies that extend far beyond basic text retrieval. A primary hurdle lies in the diverse, often semi-structured or multi-modal formats of domain-specific documents, which LLMs frequently struggle to parse, interpret, and reason over accurately \cite{kresevic2024uel, wu2025eum}.

A critical aspect of optimizing RAG for these domains is \textbf{meticulous data preparation and structuring}. Traditional RAG systems often treat all content as undifferentiated text, which is insufficient for documents rich in tables, flowcharts, diagrams, or explicit relational knowledge. For instance, \cite{kresevic2024uel} addresses the interpretation of hepatological clinical guidelines by proposing a systematic guideline reformatting strategy. This involves converting non-textual elements, such as tables extracted from images, into structured, LLM-friendly text-based lists or \texttt{.csv} files. Their work demonstrated a dramatic improvement in accuracy for tabular data questions, from a mere 28\\% baseline to 96\\% after reformatting, underscoring the foundational importance of data quality \cite{kresevic2024uel}. This approach highlights that for complex, semi-structured content, pre-processing to create a consistent, LLM-interpretable structure is paramount.

Complementing this, other approaches leverage more explicit structural information. \cite{debellis2024bv0} investigates the integration of ontologies and knowledge graphs (KGs) to form domain-specific knowledge bases for RAG, particularly for dental clinicians. By encoding domain knowledge into a structured KG, RAG systems can leverage explicit relationships and hierarchies, moving beyond purely semantic search to enable more precise, context-aware information extraction. This contrasts with the reformatting approach of \cite{kresevic2024uel} by focusing on the inherent relational structure of knowledge rather than just the presentation format of individual documents. Similarly, in the legal domain, where precise snippet retrieval is crucial for citation and avoiding context window overflow, benchmarks like LegalBench-RAG \cite{pipitone2024sfx} emphasize the need for ground truth at the character-level span, implicitly demanding highly granular data preparation and indexing for effective retrieval.

Beyond data structuring, \textbf{custom prompt engineering} plays a pivotal role in guiding the LLM's understanding and generation. \cite{kresevic2024uel} demonstrates that meticulously designed, tailored prompts, combined with structured data, significantly enhance the LLM's ability to reason over complex medical information. An ablation study in their framework empirically revealed that data quality (structured formatting and text conversion) and advanced prompt engineering were substantially more impactful than mere data quantity or few-shot learning, leading to an impressive 99.0\\% overall accuracy in guideline interpretation \cite{kresevic2024uel}. This suggests that even with well-prepared data, the way the query and context are presented to the LLM is critical for optimal performance in nuanced domains.

The efficacy of RAG in high-stakes medical applications is further corroborated by a growing body of research across various specialties. A systematic review and meta-analysis by \cite{liu2025p6t} found that RAG implementation showed a 1.35 odds ratio increase in performance compared to baseline LLMs in biomedicine, reinforcing its general utility. Specific applications include:
\begin{itemize}
    \item \textbf{Hepatology}: Beyond guideline interpretation \cite{kresevic2024uel}, \cite{ge20246t5} developed "LiVersa," a liver disease-specific LLM using RAG on American Association for the Study of Liver Diseases guidelines. LiVersa demonstrated higher accuracy than general LLMs, though it was rated less comprehensive and safe by hepatologists, highlighting the need for continuous refinement.
    \item \textbf{Neurology}: \cite{masanneck2014fk3} evaluated RAG systems using neurology guidelines, finding that while RAG improved accuracy over base models, it still produced potentially harmful answers and performed worse on case-based questions compared to knowledge-based ones. This points to the persistent challenge of complex reasoning and contextual application in clinical scenarios.
    \item \textbf{Radiology}: \cite{weinert2025cxo} developed a radiology-specific RAG system using a vector database of RadioGraphics articles. They found that RAG significantly improved examination scores for models like GPT-4 and Command R+, demonstrating its ability to provide transparent, domain-specific information retrieval and citing relevant references.
    \item \textbf{Preoperative Medicine}: \cite{ke2025wm0} developed an LLM-RAG pipeline for preoperative assessments, integrating local and international medical guidelines. Their GPT-4 LLM-RAG model achieved 96.4\\% accuracy in assessing medical fitness, outperforming human evaluators (86.6\\%) and exhibiting an absence of hallucinations, showcasing RAG's potential for standardizing assessments and improving efficiency.
\end{itemize}
These studies collectively demonstrate that RAG can significantly enhance LLM performance in medical contexts by grounding responses in verifiable, domain-specific knowledge, thereby mitigating hallucinations and improving factual accuracy. However, they also reveal that the quality of improvement can vary depending on the complexity of the task (e.g., case-based vs. knowledge-based questions), the base LLM used, and the thoroughness of the RAG implementation.

The challenge of processing truly \textbf{multi-modal information} remains an active research area. While \cite{kresevic2024uel} converts visual tables to text, this is a pre-processing step rather than inherent multi-modal understanding. \cite{wu2025eum} introduces Visual-RAG, a benchmark for text-to-image retrieval augmented generation, highlighting that even state-of-the-art multi-modal LLMs struggle to efficiently extract and utilize visual knowledge directly from retrieved images. This indicates a future direction where RAG systems might natively process visual evidence alongside text, rather than relying on conversion.

Finally, \textbf{robust evaluation} is paramount in these high-stakes domains. \cite{kresevic2024uel} critically notes that traditional text-similarity metrics (e.g., BLEU, ROUGE) are often inadequate for reliably assessing factual correctness in clinical LLM outputs, necessitating expert human review. This sentiment is echoed across medical RAG studies, where physician evaluations are often employed \cite{ge20246t5, masanneck2014fk3, low2025gjc}. Furthermore, the development of domain-specific benchmarks, such as LegalBench-RAG \cite{pipitone2024sfx} for legal documents focusing on precise snippet retrieval, and DomainRAG \cite{wang2024ac6} for evaluating RAG in Chinese college enrollment across abilities like structural information analysis and multi-document interaction, are crucial for systematically assessing RAG's capabilities and limitations in specialized contexts. These benchmarks move beyond general knowledge to evaluate specific RAG abilities pertinent to the domain, providing a more granular understanding of performance.

In conclusion, adapting RAG for complex domain-specific guidelines requires a multi-pronged approach. This includes meticulous data preparation to structure heterogeneous information into LLM-friendly formats, advanced prompt engineering to guide reasoning, and rigorous, domain-expert-driven evaluation. While RAG has shown significant promise in enhancing accuracy and reducing hallucinations in high-stakes applications like healthcare and legal analysis, challenges persist in inherently parsing non-textual information, performing complex multi-hop reasoning, and developing automated, reliable metrics for factual correctness and clinical relevance. Future research must continue to focus on system-level and knowledge-level enhancements, as suggested by \cite{liu2025p6t}, to further integrate and optimize RAG within these critical environments.


\label{sec:beyond_external_retrieval:_massive_internal_context}

\section{Beyond External Retrieval: Massive Internal Context}
\label{sec:beyond\_external\_retrieval:\_massive\_internal\_context}

\subsection{The Paradigm Shift: Expanding Native LLM Context Windows}
\label{sec:6\_1\_the\_paradigm\_shift:\_exp\_and\_ing\_native\_llm\_context\_windows}

Historically, Large Language Models (LLMs) were severely constrained by finite context windows, often ranging from a few thousand to tens of thousands of tokens. This inherent limitation necessitated external augmentation strategies like Retrieval-Augmented Generation (RAG) to access and incorporate broader, dynamic knowledge \cite{verma2024compression}. While RAG has proven invaluable for injecting relevant information and mitigating hallucinations, a profound paradigm shift is underway: the dramatic expansion of LLMs' native context windows, fundamentally altering their intrinsic capacity to process and reason over vast amounts of information directly within their architecture. This architectural evolution presents a powerful, and in many scenarios, superior alternative to external retrieval for tasks primarily driven by the need for expansive, in-depth contextual understanding, thereby challenging the traditional reliance on RAG for context extension.

The journey towards massively expanded context windows has been incremental yet rapid. Early pioneers like Anthropic's Claude 2.1 demonstrated the viability of processing hundreds of thousands of tokens (e.g., 200k tokens) in late 2023, paving the way for even larger capacities. This transformative capability is now epitomized by models such as Google's Gemini 1.5 Pro and Gemini 1.5 Flash, which leverage advanced architectural innovations to extend their effective context window to an unprecedented 10 million tokens \cite{gemini1.5}. Other leading models, including recent iterations of OpenAI's GPT-4 Turbo and Anthropic's Claude 3 series, also offer significantly expanded contexts, typically in the range of 128k to 200k tokens, with experimental versions reaching 1 million tokens.

Achieving these massive context windows is not solely attributable to sparse Mixture-of-Expert (MoE) Transformers, though MoE plays a crucial role in improving computational efficiency and scalability for large models \cite{gemini1.5}. The core technical advancements enabling such extensive context processing lie in a combination of innovations. These include highly efficient attention mechanisms, such as FlashAttention \cite{dao2022flashattention} and its successors, which reduce the quadratic computational complexity of standard attention to near-linear. Equally critical are advancements in positional embeddings, including techniques like Rotary Positional Embeddings (RoPE) scaling \cite{su2023roformer}, ALiBi (Attention with Linear Biases) \cite{press2022train}, and Positional Interpolation \cite{chen2023extending}, which allow models to generalize to sequence lengths far beyond those seen during training. These innovations, coupled with sophisticated sparse and dense scaling techniques and significant improvements in training and serving infrastructure, collectively enable this leap in efficiency and long-context performance across diverse LLMs \cite{gemini1.5}.

The expanded native context windows unlock a suite of powerful capabilities. Models like Gemini 1.5 demonstrate near-perfect recall, exceeding 99\\%, on synthetic "needle-in-a-haystack" retrieval tasks across millions of tokens and diverse modalities, including text, audio (up to 107 hours), and video (up to 10.5 hours) \cite{gemini1.5}. This native multimodality allows for the seamless interleaving of various input types within a single, extensive context. Such a capability enables novel applications, including the in-context learning of low-resource languages directly from comprehensive grammar manuals and dictionaries, achieving proficiency comparable to human translators, or the direct analysis of entire codebases within a single prompt \cite{gemini1.5}. Benchmarking studies, such as those by \cite{xie2024}, further highlight Gemini's strong performance across various tasks, including text generation and forecasting, indicating its robust ability to leverage extensive internal context.

The advent of these massively expanded native context windows introduces a critical re-evaluation of the roles of RAG and internal context. A comprehensive study by \cite{li2024raglc} directly compared RAG systems with long-context (LC) LLMs, including Gemini 1.5, across various public datasets. Their findings indicate that when sufficiently resourced, LC LLMs consistently outperform RAG in terms of average performance, particularly for tasks requiring deep, comprehensive understanding and reasoning over a large, self-contained body of information. This suggests that for such scenarios, the internal processing capabilities of models like Gemini 1.5 Pro/Flash can be superior. However, \cite{li2024raglc} also critically notes that RAG often retains a significant advantage in terms of computational cost, making it a more economical choice for many applications, especially where real-time updates or access to highly dynamic, proprietary, or frequently changing external knowledge is paramount.

While the ability to process millions of tokens natively is impressive, it does not negate all challenges. Processing such vast contexts still incurs substantial computational costs, particularly in terms of inference latency and memory footprint, even with efficient architectures like MoE \cite{verma2024compression, merth2024superposition}. Furthermore, the sheer volume of information can lead to the "lost-in-the-middle" phenomenon, where critical details are overlooked within long, noisy contexts, a challenge RAG also faces \cite{zhao2024longrag}. \cite{yue2024inference} explores inference scaling for RAG within long-context LLMs (using Gemini 1.5 Flash), observing that simply increasing the quantity of retrieved documents in RAG can lead to performance plateaus or declines due to noise and distraction. Their work suggests that even with large native contexts, sophisticated strategies like Demonstration-based RAG (DRAG) and Iterative DRAG (IterDRAG) are necessary to effectively \textit{utilize} the expanded context for complex, multi-hop queries. This demonstrates that the quality of context utilization remains paramount, regardless of the raw context length, and that intelligent processing mechanisms are still crucial for optimal performance and efficiency. Similarly, \cite{merth2024superposition} proposes "superposition prompting" to address the distraction phenomenon and quadratic scaling costs in long contexts, improving both efficiency and accuracy. Even within RAG, approaches like Sparse RAG \cite{zhu2024sparse} aim to cut computational costs by selectively attending to highly relevant caches, further emphasizing the need for efficient context management.

In essence, the dramatic expansion of native LLM context windows represents a profound architectural shift, enabling models to internalize and reason over unprecedented volumes of information. This innovation offers a powerful, often more performant, approach for tasks demanding deep, comprehensive understanding of extensive, relatively static datasets. However, it also necessitates a nuanced understanding of trade-offs, particularly regarding computational cost and the inherent challenges of effectively \textit{utilizing} vast contexts. This evolution underscores that while native context expansion provides the \textit{capacity}, intelligent processing mechanisms, whether internal or external, remain crucial for optimal performance and efficiency, paving the way for sophisticated hybrid architectures where both capabilities synergistically contribute to LLM intelligence.
\subsection{Implications for RAG and Future Hybrid Architectures}
\label{sec:6\_2\_implications\_for\_rag\_\_and\_\_future\_hybrid\_architectures}

The emergence of Large Language Models (LLMs) with dramatically expanded native context windows represents a pivotal shift in how knowledge is acquired and processed, fundamentally challenging the traditional role of Retrieval-Augmented Generation (RAG). Historically, RAG systems were indispensable for overcoming LLM limitations, such as hallucination, reliance on static training data, and restricted input lengths, by dynamically injecting external, verifiable information. However, the advent of models like Gemini 1.5, capable of natively processing up to 10 million tokens across multimodal inputs with near-perfect internal recall \cite{amugongo202530u}, necessitates a critical re-evaluation of RAG's function. This paradigm shift does not render RAG obsolete but rather redefines its optimal application, paving the way for sophisticated hybrid architectures where internal capabilities and external augmentation synergistically enhance overall intelligence and applicability.

While massive context windows offer unprecedented internal knowledge capacity, relying solely on them presents inherent limitations and trade-offs that underscore RAG's continued necessity. Processing millions of tokens internally incurs significant computational costs and inference latency, making real-time, high-throughput applications challenging \cite{merth20243h7, zhu2024h7i, jiang20243ac, quinn2024n3o}. Furthermore, once loaded, this internal context is static, making it unsuitable for information that is truly dynamic, rapidly evolving, or requires real-time updates. Even with vast context, LLMs can suffer from the "distraction phenomenon" or "lost-in-the-middle" issues, where the model struggles to effectively locate and utilize relevant information within an ultra-long, potentially noisy input, leading to performance plateaus or declines \cite{merth20243h7, yue2024ump}. Moreover, certain types of knowledge, such as highly structured data within knowledge graphs, may not be optimally leveraged by a flat, massive context window without explicit structural guidance. The ability to trace the provenance of information also becomes more opaque when knowledge is deeply internalized.

This redefinition suggests a future where RAG assumes a more specialized and strategic role within novel hybrid architectures. The expanded internal context of advanced LLMs, as demonstrated by \cite{amugongo202530u}, can be effectively leveraged for comprehensive understanding, fine-grained recall, and intricate reasoning over vast, pre-ingested, or relatively static datasets, such as entire legal corpora, extensive documentation, or historical archives. In contrast, external RAG would be strategically reserved for truly dynamic, real-time, highly specialized, or frequently updated information needs. For instance, for real-time web searches and self-correction when initial retrieval is uncertain, mechanisms like Corrective RAG (CRAG) \cite{yan202437z} remain crucial. Similarly, for complex graph-structured data, specialized retrieval approaches such as G-Retriever \cite{he20248lp}, which formulates subgraph retrieval as an optimization problem, continue to provide structured context that even a massive flat context window might struggle to parse optimally. For focused retrieval from large, partitioned knowledge bases, methods like M-RAG \cite{wang2024zt3}, which uses multi-agent reinforcement learning for intelligent partition selection and memory refinement, offer superior precision.

A key research direction for these hybrid systems lies in developing intelligent orchestration layers and routing mechanisms that dynamically decide \textit{when} to retrieve, \textit{what} to retrieve, and \textit{how} to integrate it with the LLM's internal context. This involves moving beyond simple retrieval to more sophisticated, LLM-driven decision-making for knowledge acquisition. For example, Auto-RAG \cite{yu2024c32} enables autonomous iterative retrieval by allowing the LLM to engage in multi-turn dialogues with the retriever, systematically planning and refining queries until sufficient external information is gathered. Similarly, PlanRAG \cite{lee2024hif} introduces an iterative plan-then-retrieval augmented generation technique, where the LLM generates and re-evaluates plans for data analysis and retrieval, particularly for complex decision-making tasks. These approaches highlight RAG's evolution from a passive context provider to an active, intelligent component that guides the LLM's knowledge acquisition process. Furthermore, optimizing the efficiency of RAG components is vital, with advancements like Sparse RAG \cite{zhu2024h7i} and PipeRAG \cite{jiang20243ac} demonstrating significant speedups in inference and retrieval by employing sparse context selection and algorithm-system co-design, respectively, ensuring RAG can keep pace with the rapid generation of LLMs. The work on inference scaling for long-context RAG, such as Demonstration-based RAG (DRAG) and Iterative DRAG (IterDRAG) \cite{yue2024ump}, further emphasizes the need for strategies that enable LLMs to \textit{effectively utilize} vast contexts, whether internal or retrieved, for complex reasoning tasks.

In conclusion, the interplay between LLMs with massive native context windows and sophisticated RAG techniques fosters a new frontier in LLM knowledge acquisition and processing. Future hybrid architectures are likely to harness the LLM's expanded internal context for deep, comprehensive understanding of vast static information, while intelligently deploying external RAG for dynamic, specialized, real-time, or highly structured data needs. The central challenge lies in optimally orchestrating this synergy, developing robust mechanisms to determine when to rely on the model's inherent capacity and when to augment it with external, verifiable, and dynamically acquired knowledge, thereby maximizing overall intelligence, applicability, and factual accuracy across diverse and complex tasks.


\label{sec:ethical_considerations_and_challenges_in_rag}

\section{Ethical Considerations and Challenges in RAG}
\label{sec:ethical\_considerations\_\_and\_\_challenges\_in\_rag}

\subsection{Privacy and Data Leakage in RAG Systems}
\label{sec:7\_1\_privacy\_\_and\_\_data\_leakage\_in\_rag\_systems}

The integration of Retrieval-Augmented Generation (RAG) systems, while enhancing the factual accuracy and reducing hallucinations in Large Language Models (LLMs), introduces a complex landscape of privacy and data leakage risks that demand critical attention. These risks extend beyond conventional LLM vulnerabilities, primarily stemming from the external retrieval databases that frequently house proprietary or private user data.

Early investigations into RAG's privacy implications, such as the work by \cite{smith2022retrieval}, highlighted the fundamental risk of sensitive information exfiltration from these external knowledge bases. Their research identified that the very mechanism of retrieving and presenting relevant documents to the LLM creates new attack surfaces, particularly when the underlying data includes confidential user records or proprietary business intelligence. Building upon this foundational understanding, subsequent studies have unveiled sophisticated attack vectors specifically tailored to the RAG architecture. For instance, \cite{jones2023prompting} meticulously detailed "composite structured prompting attacks," demonstrating how carefully crafted prompts can manipulate the retrieval process and the LLM's synthesis capabilities to illicitly extract sensitive data embedded within the external documents. These attacks exploit the interplay between the retriever and the generator, bypassing traditional security measures by leveraging the system's intended functionality for malicious purposes.

Beyond the direct leakage from external databases, RAG systems also exert a dual impact on the privacy of the LLM's own training data. Research by \cite{chen2023dualimpact} revealed that while RAG can introduce new vulnerabilities by potentially exposing parts of the LLM's internal knowledge when combined with retrieved information, it surprisingly also offers potential mitigation effects in certain scenarios. Their findings suggest that by offloading factual recall to external databases, RAG can reduce the LLM's reliance on memorizing sensitive training data, thereby potentially lowering the risk of direct memorization-based privacy attacks on the LLM itself. However, this benefit is contingent on robust security around the retrieval component, as a compromised retriever could still facilitate indirect exposure. This complex interplay underscores that RAG is not a monolithic privacy solution but rather a system with nuanced effects.

The evolving landscape of RAG privacy necessitates a proactive approach to system design and governance. Addressing these challenges, \cite{wang2024privacypreserving} proposed initial design principles for privacy-preserving RAG systems, advocating for techniques such as differential privacy for retrieved content and secure multi-party computation for sensitive data processing. They emphasized the urgent need for stringent data governance policies, including rigorous access controls, data anonymization strategies, and comprehensive auditing mechanisms, to manage the lifecycle of sensitive information within RAG deployments. In conclusion, the literature underscores that while RAG offers significant advancements, its deployment without robust privacy-preserving designs and stringent data governance creates substantial risks for sensitive information leakage. Future research must focus on developing resilient architectures and policies that can withstand novel attack vectors while harnessing RAG's benefits responsibly.

\bibliographystyle{plainnat}
\bibliography{references}
\subsection{Mitigating Bias and Ensuring Fairness}
\label{sec:7\_2\_mitigating\_bias\_\_and\_\_ensuring\_fairness}

The integration of Large Language Models (LLMs) with external knowledge bases in Retrieval-Augmented Generation (RAG) systems, while enhancing factual accuracy, introduces a critical challenge: the potential for existing biases to be amplified or inadvertently introduced during the retrieval and generation processes. These biases can stem from historical data reflecting societal inequalities within the external knowledge base or be inherited from the LLM's vast training data, leading to the generation of unfair, discriminatory, or harmful outputs \cite{zhou20248fu, ni2025ox9}. Addressing this necessitates robust strategies for detection, mitigation, and prevention to ensure equitable and responsible deployment of RAG technologies, forming a crucial dimension of overall RAG trustworthiness \cite{ni2025ox9}.

The problem of bias in RAG systems is multifaceted, impacting both the selection of information and its subsequent synthesis. As comprehensive surveys by \cite{zhou20248fu} and \cite{ni2025ox9} highlight, fairness is a core pillar of trustworthiness in RAG, alongside factuality, robustness, privacy, and accountability. Bias can manifest when the retriever component disproportionately selects documents reflecting certain demographics, viewpoints, or stereotypes present in the knowledge base. This "retrieval bias" can then be amplified by the LLM during generation, even if the LLM itself has undergone some debiasing during pre-training. Conversely, an LLM's inherent biases can lead it to misinterpret or selectively use retrieved information in a biased manner, even from a diverse set of documents. Critically, \cite{zhang2025byv} demonstrates that RAG systems are not inherently safer than standalone LLMs; in fact, they can sometimes make models \textit{less safe} and alter their safety profile, even when combining "safe" models with "safe" documents. This finding underscores the complex interaction between retrieval and generation, where emergent biases or safety failures can arise from the RAG pipeline itself, necessitating specific RAG-tailored safety and fairness interventions.

Initial efforts to address bias often focused on the generative capabilities of LLMs. Techniques such as prompt engineering, where carefully crafted prompts guide the LLM towards less biased outputs, and re-ranking of generated responses based on pre-computed fairness scores, were foundational. However, these methods primarily addressed the output of the LLM in isolation. With RAG, the problem space expanded to encompass the retrieval phase, demanding more integrated solutions.

To mitigate bias propagation within the retrieval component, researchers have explored "fairness-aware retriever" mechanisms. The goal is to diversify retrieved information, preventing the over-representation of biased perspectives. This can involve incorporating demographic parity constraints during document selection or re-ranking retrieved documents based on fairness scores associated with their source material or content. For instance, approaches might penalize documents from sources known to exhibit bias or prioritize documents that offer diverse viewpoints on a given topic. While effective in mitigating bias at the retrieval stage by ensuring a more balanced input context, the subsequent generation process could still inadvertently introduce or amplify subtle biases, particularly if the LLM's internal representations are biased or if it struggles with synthesizing conflicting fair information.

Addressing biases that persist or emerge during the generation phase of RAG has led to more sophisticated post-hoc mitigation strategies. These include counterfactual generation approaches, where biased outputs are identified and then re-generated using modified prompts or retrieved contexts to produce more neutral or diverse responses. This often involves identifying sensitive attributes in the output and attempting to generate alternatives that vary these attributes while maintaining factual consistency. Such methods highlight the complex trade-off between factual accuracy and fairness, as aggressive debiasing might inadvertently alter the factual content.

The comprehensive evaluation of fairness in RAG systems is paramount. Traditional single-axis fairness metrics, which often focus on a single protected attribute (e.g., gender or race), frequently fail to capture the complexities of real-world discrimination. Research has moved towards frameworks for evaluating intersectional biases—such as those combining race and gender—in RAG systems. This involves developing new evaluation datasets and metrics that assess fairness across multiple protected attributes simultaneously, providing a more nuanced understanding of how bias propagates from retrieval to generation. The trustworthiness frameworks proposed by \cite{zhou20248fu} and \cite{ni2025ox9} explicitly include fairness as a key dimension, advocating for comprehensive benchmarks that can diagnose specific RAG fairness capabilities and limitations across diverse scenarios.

Ultimately, a proactive approach to bias prevention at the source is critical. This involves auditing and curating the external knowledge base itself to identify and rectify biased, under-represented, or harmful information before it is even retrieved by the RAG system. Methods include identifying data gaps, suggesting diverse data augmentation strategies for the knowledge base, and implementing rigorous data governance policies. This represents a crucial shift towards preventing bias from entering the system in the first place, complementing reactive and retrieval-time mitigation efforts. However, this pre-processing can be resource-intensive and requires careful definition of "bias" within specific contexts.

In conclusion, the literature demonstrates a clear progression from detecting and mitigating bias in standalone LLM generation to addressing its propagation through RAG's retrieval and generation phases. While significant strides have been made in developing fairness-aware retrieval mechanisms, post-hoc generation corrections, and sophisticated intersectional evaluation metrics, several challenges persist. The dynamic nature of bias, the inherent trade-off between fairness and other performance metrics (like relevance or utility), and the critical finding that RAG can introduce new safety risks \cite{zhang2025byv} all underscore the need for continuous monitoring and adaptive strategies in real-world deployments. Future research must focus on developing holistic, integrated frameworks that combine proactive knowledge base curation with robust, context-aware detection and mitigation strategies throughout the entire RAG pipeline to foster truly trustworthy and inclusive AI systems.
\subsection{Robustness to Adversarial Attacks and Misinformation}
\label{sec:7\_3\_robustness\_to\_adversarial\_attacks\_\_and\_\_misinformation}

The integration of Large Language Models (LLMs) with retrieval mechanisms in Retrieval-Augmented Generation (RAG) systems, while promising enhanced factual accuracy, concurrently introduces significant vulnerabilities to adversarial attacks and the inadvertent propagation of misinformation. These challenges critically undermine the integrity and trustworthiness that RAG aims to provide, as carefully crafted malicious inputs or subtly manipulated retrieved documents can lead the LLM to generate incorrect, biased, or even harmful responses. Indeed, a comprehensive survey by \cite{zhou20248fu} highlights that while RAG mitigates some LLM limitations, it can reintroduce safety issues like information leakage and unfairness if retrieved data is problematic, emphasizing the need for a unified framework for RAG trustworthiness encompassing dimensions like factuality and robustness.

Adversarial attacks on RAG systems primarily exploit two main surfaces: the user query (prompt) and the external knowledge base.
\textbf{Knowledge Base Poisoning and Manipulation} represents a potent attack vector where malicious content is injected into the retrieval corpus, influencing the LLM's generation. \cite{zou2024haa} introduced PoisonedRAG, a knowledge corruption attack that demonstrates an attacker can inject a few malicious texts into a large knowledge database (e.g., 5 texts per target question in millions of texts) to induce the LLM to generate attacker-chosen target answers for specific questions, achieving a 90\\% attack success rate. Similarly, \cite{zhang2024rwm} proposed HijackRAG, a retrieval prompt hijack attack that manipulates the retrieval mechanisms by injecting malicious texts, causing the RAG system to generate pre-determined answers instead of correct ones. Their experiments showed high attack success rates and transferability across different retriever models, underscoring widespread risk. Beyond direct answer manipulation, \cite{clop2024zs2} investigated backdoor attacks on dense retrievers during fine-tuning, demonstrating that corpus poisoning can achieve significant prompt injection success rates with a small number of compromised documents. These attacks aim to insert harmful links, promote unauthorized services, or initiate denial-of-service behaviors. Furthermore, \cite{chen20247nc} explored black-box opinion manipulation attacks, where manipulating the ranking results of the retrieval model can significantly alter the opinion polarity of RAG-generated content, thereby impacting user cognition and decision-making. A particularly concerning form of this vulnerability is demonstrated by \cite{deng2024k1b} with Pandora, a RAG poisoning technique used to jailbreak OpenAI's GPTs. By uploading strategically crafted malicious documents as the RAG knowledge source and using tailored in-built prompts, they achieved jailbreak success rates of 64.3\\% for GPT-3.5 and 34.8\\% for GPT-4, bypassing existing safety filters.

The implications of such attacks extend beyond mere factual errors, fundamentally impacting the system's trustworthiness and potential for widespread misinformation dissemination. \cite{zhang2025byv} conducted a detailed comparative analysis across eleven LLMs, finding that RAG can paradoxically make models \textit{less safe} and significantly alter their safety profile. They observed that even combinations of ostensibly safe models with safe documents could lead to unsafe generations, highlighting a critical gap in current AI safety research which often focuses on non-RAG LLMs. This suggests that the perceived authority of RAG-generated content, if compromised, makes it a potent vector for spreading misinformation, as users are more likely to trust information presented by an advanced AI system.

Recognizing these vulnerabilities, the research community has begun to explore robust defense mechanisms. \cite{fang2024gh6} proposed Retrieval-augmented Adaptive Adversarial Training (RAAT) to enhance noise robustness in RAG systems. RAAT systematically categorizes retrieval noises into relevant, irrelevant, and counterfactual types, and employs an adaptive adversarial training mechanism that dynamically adjusts the training process based on the model's sensitivity to these diverse noises. Their experiments demonstrated significant improvements in F1 and Exact Match scores for models trained with RAAT under various noise conditions. Another approach, C-RAG, introduced by \cite{kang2024hrb}, provides the first framework to certify generation risks for RAG models. It offers provable guarantees on generation risks, demonstrating that RAG can achieve lower conformal generation risks than vanilla LLMs when the quality of the retrieval model and transformer is non-trivial. General defense strategies, such as rigorous input validation to detect malicious prompts and source verification for retrieved documents, remain crucial. However, the sophistication of attacks like PR-Attack \cite{jiao20259xa}, which uses bilevel optimization to coordinate prompt and RAG attacks for high effectiveness and stealth with limited poisoned texts, necessitates more advanced, multi-layered defenses.

To systematically evaluate and foster the development of secure RAG systems, specialized benchmarks are emerging. \cite{liang2025f4q} introduced SafeRAG, a benchmark designed to assess RAG security across various attack tasks, including silver noise, inter-context conflict, soft advertisements, and white Denial-of-Service. Their experiments on 14 RAG components revealed significant vulnerabilities, with even apparent attacks easily bypassing existing retrievers, filters, or advanced LLMs. Similarly, \cite{fang2024gh6} established RAG-Bench to specifically assess the noise robustness of RAG models.

Despite these promising developments, the arms race between attackers and defenders in the realm of RAG systems continues. The findings from \cite{zou2024haa} and \cite{zhang2024rwm} indicate that existing defenses are often insufficient against advanced knowledge corruption and hijacking attacks. Future research must prioritize holistic defense frameworks that combine advanced input validation, multi-layered source verification, and inherent LLM robustness, alongside enhanced transparency and provenance tracking for retrieved information. Developing real-time, scalable source verification for dynamic web content and robust defenses against zero-day adversarial attacks that exploit novel vulnerabilities are critical to ensure the reliability, factual accuracy, and overall trustworthiness of RAG outputs in an increasingly complex and adversarial information landscape.


\label{sec:conclusion_and_future_directions}

\section{Conclusion and Future Directions}
\label{sec:conclusion\_\_and\_\_future\_directions}

\subsection{Summary of Key Advancements}
\label{sec:8\_1\_summary\_of\_key\_advancements}

The evolution of Retrieval-Augmented Generation (RAG) represents a pivotal advancement in overcoming the inherent limitations of Large Language Models (LLMs), such as their propensity for factual hallucination, reliance on static knowledge, and constraints in processing extensive contexts. This journey, from foundational 'retrieve-then-generate' models to sophisticated adaptive architectures, has profoundly enhanced LLMs' factual accuracy, trustworthiness, and applicability in knowledge-intensive tasks \cite{28e2ecb4183ebc0eec504b12dddc677f8aef8745, lewis2020pwr}.

Initially, the basic 'retrieve-then-generate' paradigm, pioneered by works like \cite{lewis2020pwr} and detailed in Section 2.1, laid the groundwork by sequentially fetching relevant documents from an external knowledge base and subsequently using an LLM to synthesize a response. This foundational approach, however, quickly exposed critical vulnerabilities, notably poor noise robustness and the LLM's struggle to effectively utilize or reject irrelevant or conflicting information \cite{b798cf6af813638fab09a8af6ad0f3df6c241485}. Early diagnostic benchmarks, such as the Retrieval-Augmented Generation Benchmark (RGB) \cite{5bbc2b5aa6c63c6a2cfccf095d6020b063ad47ac} (discussed in Section 4.1), systematically identified these weaknesses, catalyzing a wave of methodological advancements. This led to the development of more structured frameworks that systematized RAG architectures, moving beyond simple sequential models to modular paradigms that allow for optimization at each stage of the retrieval and generation process \cite{28e2ecb4183ebc0eec504b12dddc677f8aef8745}.

Significant breakthroughs emerged in refining the RAG pipeline through advanced architectures and optimizations, as explored in Section 3. Pre-retrieval strategies, such as intelligent query refinement and expansion \cite{746b96ee17e329f1085a047116c05e12eaa3925a} (Section 3.1), allowed LLMs to dynamically improve search queries, leading to more precise initial retrieval. Concurrently, post-retrieval optimizations \cite{80478de9c7a81561e2f3dac9b8b1ef3df389ff2d} (Section 3.2), including reranking and context compression, became crucial for filtering and condensing retrieved information to maximize its utility for the LLM, mitigating issues like the 'lost-in-the-middle' phenomenon. A critical evolution was the advent of adaptive and self-correcting RAG frameworks (Section 3.3), exemplified by innovations like Corrective RAG (CRAG) \cite{5bbc2b5aa6c63c6a2cfccf095d6020b063ad47ac}. These systems dynamically assess the relevance and quality of retrieved documents, triggering corrective actions such as re-retrieval with refined queries or even large-scale web searches when initial information is deemed suboptimal. This adaptive knowledge acquisition marked a crucial leap towards more robust and resilient RAG systems, capable of proactively addressing and recovering from retrieval failures. The field also saw the emergence of automated optimization frameworks and proposals for unified retrieval engines, aiming for greater scalability and efficiency in RAG deployments.

The increasing complexity and criticality of RAG applications necessitated the development of advanced benchmarking and evaluation methodologies, as detailed in Section 4. Beyond general diagnostic frameworks for RAG capabilities, new metrics emerged to assess retrieval quality based on its actual utility to the LLM, moving beyond traditional relevance scores \cite{e90435e1ae06fab4efa272f5f46ed74ca0a8cde0} (Section 4.2). Benchmarks for complex reasoning tasks, such as multi-hop queries \cite{4e71624e90960cb003e311a0fe3b8be4c2863239} (Section 4.3), were introduced to evaluate LLMs' ability to synthesize information across multiple documents and identify when no answer can be derived. Crucially, domain-specific benchmarking gained prominence (Section 4.4), particularly in high-stakes fields like medicine, where frameworks like MIRAGE and toolkits like MEDRAG \cite{b798cf6af813638fab09a8af6ad0f3df6c241485} demonstrated RAG's capacity to significantly improve LLM accuracy in specialized contexts, often enabling smaller models to rival larger, unaugmented counterparts. These rigorous evaluations, including systematic reviews, have consistently validated RAG's statistically significant performance improvements in critical domains.

RAG's versatility has further expanded through its adaptation for structured and domain-specific knowledge (Section 5). The integration of knowledge graphs \cite{b708e0f49d8e9708bc649debd9a9372748fffa3d, a41d4a3b005c8ec4f821e6ee96672d930ca9596c} (Section 5.1) has allowed RAG systems to leverage explicit structural and relational information, moving beyond purely semantic search to enable more precise and context-aware reasoning. In highly specialized fields, meticulous data preparation and custom prompt engineering \cite{965a0969b460f9246158d88fb28e21c5d80d0a8b} (Section 5.2) have proven critical for optimizing RAG performance, ensuring LLMs can accurately interpret complex, often multi-modal, information.

A parallel and increasingly significant advancement, explored in Section 6, is the paradigm shift towards LLMs with vastly expanded native context windows. Models like Gemini 1.5 Pro/Flash \cite{a41d4a3b005c8ec4f821e6ee96672d930ca9596c, 5bbc2b5aa6c63c6a2cfccf095d6020b063ad47ac}, leveraging sparse Mixture-of-Expert architectures, can natively process millions of tokens across multimodal inputs, offering near-perfect recall over massive contexts (Section 6.1). This capability presents a powerful alternative or complement to external RAG for scenarios primarily driven by the need for extensive contextual understanding. This shift suggests a redefinition of RAG's role, potentially leading to novel hybrid architectures where expanded internal context and external RAG synergistically enhance overall LLM capabilities (Section 6.2).

Collectively, these advancements illustrate a relentless pursuit to overcome the inherent knowledge and context limitations of LLMs. The journey from diagnosing RAG's initial weaknesses to systematizing its architectures, enhancing its robustness through self-correction and intelligent query refinement, and rigorously benchmarking its performance in specialized domains, has profoundly contributed to making LLMs more factually accurate, trustworthy, and widely applicable. While external RAG continues to evolve with increasingly sophisticated retrieval and augmentation strategies, the concurrent architectural shift towards massive internal context processing suggests a future where LLMs can leverage both external, dynamic knowledge and vast, internalized context to achieve unprecedented utility in knowledge-intensive tasks. The ongoing challenge lies in seamlessly integrating these diverse approaches to create truly intelligent and adaptable knowledge systems.
\subsection{Open Challenges and Research Gaps}
\label{sec:8\_2\_open\_challenges\_\_and\_\_research\_gaps}

Despite significant advancements, Retrieval-Augmented Generation (RAG) systems still face several critical open challenges and research gaps that hinder their full potential and reliable deployment in complex, real-world knowledge-intensive environments. Addressing these unresolved issues is paramount for enhancing RAG's performance, trustworthiness, and applicability.

A foundational challenge lies in the inherent limitations of Large Language Models (LLMs) when interacting with and reasoning over retrieved information. As systematically diagnosed by the Retrieval-Augmented Generation Benchmark (RGB) \cite{chen2023nzb}, current RAG systems exhibit persistent vulnerabilities. Specifically, LLMs demonstrate limited noise robustness, often struggling to differentiate between similar or conflicting information within retrieved documents, which can lead to inaccurate generations. More critically, they frequently fail at negative rejection, generating incorrect answers even when no relevant information is present in the provided context, thereby exacerbating the problem of hallucination \cite{chen2023nzb}. Furthermore, a significant gap exists in LLMs' ability to effectively integrate and synthesize information from multiple documents, a prerequisite for complex query answering. The study also highlighted a concerning lack of counterfactual robustness, where LLMs tend to prioritize and trust incorrect information from retrieved documents over their own potentially accurate internal knowledge, even when explicitly warned about data risks \cite{chen2023nzb}. These findings underscore persistent vulnerabilities in how RAG systems process and validate external knowledge, indicating that the problem is not merely about \textit{what} is retrieved, but \textit{how} the retrieved information is utilized and reasoned upon. Future research must investigate explicit reasoning modules or neuro-symbolic approaches that can more robustly validate and synthesize retrieved facts, potentially drawing inspiration from iterative retrieval methods like Auto-RAG \cite{yu2024c32} and Chain-of-Verification RAG (CoV-RAG) \cite{he2024hos} that focus on refining queries and verifying generated answers.

Building upon these core limitations in information utilization, a more complex and urgent research gap is the unsatisfactory performance of RAG systems on multi-hop reasoning tasks. While earlier benchmarks like RGB primarily focused on single-hop queries, real-world applications frequently demand synthesizing information across disparate sources. \cite{tang2024i5r} directly addressed this by introducing \texttt{MultiHop-RAG}, the first dedicated benchmark for multi-hop queries. Their comprehensive evaluation empirically demonstrated that even state-of-the-art RAG systems perform "unsatisfactorily" in retrieving and answering these complex multi-hop questions, highlighting a significant hurdle in achieving advanced reasoning capabilities. This suggests a fundamental limitation in the LLM's ability to perform complex logical operations and synthesize disparate facts, even when relevant information is theoretically available. The inclusion of "Null queries" in \texttt{MultiHop-RAG} further reinforces the persistent challenge of negative rejection and hallucination mitigation identified by \cite{chen2023nzb}, as RAG systems must reliably determine when an answer cannot be derived from the available evidence. Moreover, the lack of granular, explainable evaluation metrics for RAG systems, as highlighted by \cite{friel20241ct} and implicitly by comprehensive benchmarks like CRUD-RAG \cite{lyu2024ngu}, further complicates the diagnosis of these performance failures. While frameworks like TRACe introduce metrics like "Context Utilization" and "Completeness," the field still requires more sophisticated tools to pinpoint \textit{why} an LLM fails to leverage relevant context or synthesize information effectively, moving beyond mere end-to-end accuracy. Future work must investigate explicit graph traversal algorithms over retrieved documents or develop architectures with dedicated reasoning modules to overcome the limitations of purely attention-based synthesis.

A significant emerging research gap revolves around the optimal interplay between traditional RAG and the paradigm shift towards Large Language Models (LLMs) with massive native context windows, as discussed in Section 6. The advent of models capable of processing millions of tokens internally challenges the traditional necessity of external retrieval for many long-context tasks. \cite{li2024wff} conducted a comprehensive comparison, revealing that while long-context LLMs often outperform RAG in average performance when sufficiently resourced, RAG retains a distinct advantage in significantly lower computational cost. This creates a critical architectural decision point. Similarly, \cite{zhao20248wm} proposed LongRAG, a dual-perspective RAG paradigm specifically for long-context question answering, aiming to enhance RAG's understanding of both global information and factual details within long documents. The open challenge is to develop principled methods and hybrid architectures that dynamically determine when to retrieve versus when to rely on in-context information, and how to optimally combine both modalities to leverage their respective strengths (e.g., RAG for dynamic, real-time, or highly specialized knowledge; massive context for comprehensive understanding of vast, static documents) while managing cost and latency.

Beyond these core performance and architectural challenges, several operational and systemic research gaps urgently require attention for RAG's practical deployment. A significant hurdle is the \textbf{knowledge curation bottleneck}: the substantial, often manual, effort and cost involved in creating, updating, and maintaining high-quality, structured knowledge bases. As RAG systems become more sophisticated, relying on diverse data sources and potentially knowledge graphs \cite{b708e0f49d8e9708bc649debd9a9372748fffa3d, probst202417i, debellis2024bv0}, the complexity of managing these external knowledge stores grows exponentially. This includes ensuring data freshness, consistency, and accuracy, especially in rapidly evolving domains. Closely related is the challenge of enhancing the system's ability to effectively handle \textbf{highly dynamic, conflicting, or rapidly evolving information}. While corrective RAG frameworks like CRAG \cite{yan202437z} attempt to mitigate issues arising from inaccurate initial retrievals by dynamically seeking additional information, they still face significant challenges in reconciling genuinely contradictory evidence or adapting to real-time information shifts without introducing new biases or inconsistencies. Future research should focus on automated knowledge graph construction and maintenance, active learning strategies for continuous knowledge base updates, and sophisticated reconciliation mechanisms for conflicting retrieved information.

Furthermore, \textbf{efficiency, scalability, and deployment considerations} present critical research gaps. As RAG pipelines incorporate more complex pre-retrieval query refinements, post-retrieval reranking, and self-correction mechanisms, their computational cost and latency can become prohibitive for real-world, high-throughput applications. This is particularly acute for edge-based deployments, where resource-constrained devices struggle with repetitive searches on growing user data, leading to significant latency and scalability issues \cite{qin202445s}. Novel architectures and optimization techniques are needed to free RAG from these constraints, enabling efficient operation across diverse hardware environments.

Finally, while Section 7 meticulously details the ethical and security challenges in RAG, a significant overarching research gap lies in developing a \textbf{unified framework and comprehensive methodologies for RAG trustworthiness}. As highlighted by recent surveys \cite{zhou20248fu, ni2025ox9, fan2024pf1, huang2024a59}, despite RAG's promise, it introduces new risks related to robustness, privacy, adversarial attacks, bias, and accountability. The field lacks a standardized approach to define, measure, and systematically improve trustworthiness across these dimensions, especially considering their interdependencies and potential trade-offs (e.g., between privacy and explainability). \cite{zhou20248fu} proposes a framework encompassing Factuality, Robustness, Fairness, Transparency, Accountability, and Privacy, emphasizing the need for holistic solutions. Future research must focus on developing integrated privacy-preserving designs, robust bias mitigation strategies, and enhanced adversarial defenses into the very architecture of RAG systems, rather than treating them as isolated afterthoughts. This includes fostering transparency and accountability in the retrieval and generation processes to build user confidence and enable auditability.

In conclusion, while RAG has shown immense promise, the field is still grappling with fundamental issues related to robust information utilization, complex multi-hop reasoning, and reliable hallucination prevention. Coupled with the need for advancements in knowledge management, adaptability to dynamic information, efficiency, scalability, and a unified approach to trustworthiness, these open challenges represent critical frontiers for future research. Addressing them will be instrumental in pushing the boundaries of RAG's reliability and applicability, enabling its widespread and trustworthy adoption in knowledge-intensive environments.
\subsection{The Future of Retrieval-Augmented Generation}
\label{sec:8\_3\_the\_future\_of\_retrieval-augmented\_generation}

The evolving landscape of Large Language Model (LLM) knowledge acquisition is characterized by a dynamic interplay between sophisticated external retrieval mechanisms and dramatically expanded internal context processing. The future of Retrieval-Augmented Generation (RAG) is not merely an incremental improvement but a convergence towards intelligent, hybrid architectures that synergistically leverage these paradigms, fostering the development of more robust, adaptable, and ethically responsible AI systems.

The foundational role of RAG in grounding LLMs with dynamic, up-to-date, and verifiable information remains paramount \cite{gao20238ea}. Advancements in adaptive and self-correcting RAG frameworks (Section 3.3), exemplified by Corrective RAG (CRAG) \cite{yan202437z}, highlight a trajectory towards autonomous knowledge acquisition where systems dynamically assess retrieval quality and trigger adaptive actions, including re-retrieval or large-scale web searches. This continuous refinement addresses early limitations such as noise robustness and multi-document integration challenges \cite{chen2023nzb}. Furthermore, RAG's specialization is extending beyond traditional textual documents, with innovations like G-Retriever (Section 5.1) pioneering RAG for textual graphs, enabling LLMs to reason over structured data by formulating subgraph retrieval as a Prize-Collecting Steiner Tree optimization problem \cite{he20248lp}. These developments underscore RAG's enduring relevance for dynamic, specialized, and verifiable knowledge acquisition.

Concurrently, a transformative shift in LLM architecture involves the dramatic expansion of native context windows. Models like Gemini 1.5 Pro/Flash, as detailed in \cite{amugongo202530u}, now boast effective context windows of up to 10 million tokens across multimodal inputs. This capability fundamentally expands an LLM's intrinsic ability to hold, recall, and reason over massive contexts directly, achieving near-perfect recall in "needle-in-a-haystack" tasks \cite{amugongo202530u}. This development challenges the traditional necessity of external retrieval for many long-context tasks, enabling novel applications such as in-context learning of low-resource languages from entire documentation or comprehensive analysis of vast codebases, effectively internalizing what previously required complex external augmentation.

The future of RAG will likely reside in sophisticated hybrid architectures that intelligently combine these strengths. As explored by \cite{li2024wff}, while massive context LLMs often outperform RAG in average performance when sufficiently resourced, RAG retains a distinct advantage in terms of significantly lower computational cost. This observation motivates hybrid approaches such as "Self-Route," where an LLM leverages self-reflection to dynamically route queries to either its expansive internal context or a RAG module based on factors like cost-efficiency and task requirements \cite{li2024wff}. Such an "intelligent decider" component presents a critical research direction, requiring robust training, potentially using reinforcement learning signals derived from downstream task performance, to optimize for latency, computational budget, and accuracy. However, this introduces new failure modes, such as misrouting queries or the "lost-in-the-middle" phenomenon within massive internal contexts. While LongRAG \cite{zhao20248wm} addresses this issue in traditional RAG by enhancing understanding of global and factual details in long documents, its manifestation in 10M token multimodal contexts could be more complex, requiring advanced context management and attention mechanisms to prevent subtle misinterpretations or overlooked critical information. Further, Astute RAG \cite{wang2024kca} offers a blueprint for overcoming imperfect retrieval and knowledge conflicts by adaptively eliciting internal LLM knowledge and iteratively consolidating internal and external sources, ensuring reliability even under challenging retrieval conditions. Similarly, RAG-DDR \cite{li20243nz} proposes end-to-end training for RAG systems using differentiable data rewards, aligning data preferences between modules and mitigating conflicts between parametric memory and external knowledge, particularly for smaller LLMs.

Beyond the RAG-vs-long-context dichotomy, the future also points towards more agentic and integrated RAG systems, and crucially, towards a multimodal paradigm. Agentic RAG frameworks, such as M-RAG, employ multi-agent reinforcement learning for dynamic database partitioning and memory refinement, allowing LLMs to intelligently select and optimize retrieval from multiple partitions \cite{wang2024zt3}. This represents a significant step towards RAG systems that can autonomously manage and interact with diverse knowledge sources. Concurrently, the concept of a "search engine for machines" is emerging, where unified retrieval engines (e.g., uRAG) serve multiple downstream RAG systems for varied purposes like question answering, fact verification, and entity linking, standardizing communication and optimizing retrieval models across tasks \cite{salemi2024bb6}. This suggests a future where retrieval becomes a highly integrated, multi-purpose utility layer for LLMs.

A critical, yet underdeveloped, aspect of this future is multimodal RAG. While LLMs like Gemini 1.5 are inherently multimodal, the integration of RAG with diverse data types (images, audio, video, complex diagrams) presents unique challenges and opportunities. Visual-RAG \cite{wu2025eum} highlights that even state-of-the-art Multimodal LLMs (MLLMs) struggle to efficiently extract and utilize visual knowledge from retrieved images, underscoring the need for improved visual retrieval, grounding, and attribution. Future research must focus on developing effective cross-modal retrieval techniques, robust methods for fusing information from disparate data types (e.g., text transcripts and video frames), and rigorous evaluation of generation fidelity based on multimodal sources \cite{wang20248gm}. Toolkits like FlashRAG \cite{jin2024yhb} are beginning to provide modular support for multimodal LLMs and CLIP-based retrievers, offering a foundation for this burgeoning research area. The systematic identification of best practices across the entire RAG workflow, including multimodal components, as explored by \cite{wang20248gm}, will be essential for realizing the full potential of multimodal RAG.

The pursuit of these advanced architectures must be intrinsically linked with the development of ethically responsible AI systems. While RAG aims to enhance trustworthiness, new challenges arise. As discussed in Section 7, RAG can, counter-intuitively, alter LLMs' safety profiles and introduce vulnerabilities for leaking sensitive information from external retrieval databases through novel attack vectors \cite{zeng2024dzl, zhang2025byv}. Therefore, future research must prioritize robust privacy-preserving RAG designs, mechanisms to detect and mitigate bias in retrieved and generated content, and enhanced robustness against adversarial attacks and misinformation, especially in multimodal contexts where biases could be subtly encoded in visual or audio data. A comprehensive framework for trustworthy RAG, encompassing reliability, privacy, safety, fairness, explainability, and accountability, as proposed by \cite{ni2025ox9}, will be essential to guide the development and deployment of these increasingly powerful and integrated LLM systems. This holistic vision aims for AI that can seamlessly acquire, process, and reason over information, minimizing hallucinations while ensuring ethical integrity and broadening applicability across all knowledge-intensive domains.


\newpage
\section*{References}
\addcontentsline{toc}{section}{References}

\begin{thebibliography}{211}

\bibitem{lewis2020pwr}
Patrick Lewis, Ethan Perez, Aleksandara Piktus, et al. (2020). \textit{Retrieval-Augmented Generation for Knowledge-Intensive NLP Tasks}. Neural Information Processing Systems.

\bibitem{komeili20215so}
M. Komeili, Kurt Shuster, and J. Weston (2021). \textit{Internet-Augmented Dialogue Generation}. Annual Meeting of the Association for Computational Linguistics.

\bibitem{chen2022j8c}
Wenhu Chen, Hexiang Hu, Xi Chen, et al. (2022). \textit{MuRAG: Multimodal Retrieval-Augmented Generator for Open Question Answering over Images and Text}. Conference on Empirical Methods in Natural Language Processing.

\bibitem{agarwal2021e31}
Oshin Agarwal, Heming Ge, Siamak Shakeri, et al. (2021). \textit{Knowledge Graph Based Synthetic Corpus Generation for Knowledge-Enhanced Language Model Pre-training}. North American Chapter of the Association for Computational Linguistics.

\bibitem{gui2021zw6}
Liangke Gui, Borui Wang, Qiuyuan Huang, et al. (2021). \textit{KAT: A Knowledge Augmented Transformer for Vision-and-Language}. North American Chapter of the Association for Computational Linguistics.

\bibitem{masanneck2014fk3}
L. Masanneck, Sven G. Meuth, and M. Pawlitzki (2014). \textit{Evaluating base and retrieval augmented LLMs with document or online support for evidence based neurology}. The Lancet.

\bibitem{sun2022hx2}
Zhiqing Sun, Xuezhi Wang, Yi Tay, et al. (2022). \textit{Recitation-Augmented Language Models}. International Conference on Learning Representations.

\bibitem{sarto2022nxs}
Sara Sarto, Marcella Cornia, L. Baraldi, et al. (2022). \textit{Retrieval-Augmented Transformer for Image Captioning}. International Conference on Content-Based Multimedia Indexing.

\bibitem{shi20222ui}
Ensheng Shi, Yanlin Wang, Wei Tao, et al. (2022). \textit{RACE: Retrieval-augmented Commit Message Generation}. Conference on Empirical Methods in Natural Language Processing.

\bibitem{chowdhury20228rz}
Jishnu Ray Chowdhury, Yong Zhuang, and Shuyi Wang (2022). \textit{Novelty Controlled Paraphrase Generation with Retrieval Augmented Conditional Prompt Tuning}. AAAI Conference on Artificial Intelligence.

\bibitem{xu2021slt}
Yan Xu, Etsuko Ishii, Zihan Liu, et al. (2021). \textit{Retrieval-Free Knowledge-Grounded Dialogue Response Generation with Adapters}. Workshop on Document-grounded Dialogue and Conversational Question Answering.

\bibitem{adolphs20219au}
Leonard Adolphs, Kurt Shuster, Jack Urbanek, et al. (2021). \textit{Reason first, then respond: Modular Generation for Knowledge-infused Dialogue}. Conference on Empirical Methods in Natural Language Processing.

\bibitem{dixit2022xid}
Tanay Dixit, Bhargavi Paranjape, Hannaneh Hajishirzi, et al. (2022). \textit{CORE: A Retrieve-then-Edit Framework for Counterfactual Data Generation}. Conference on Empirical Methods in Natural Language Processing.

\bibitem{glass2021qte}
Michael R. Glass, Gaetano Rossiello, Md. Faisal Mahbub Chowdhury, et al. (2021). \textit{Robust Retrieval Augmented Generation for Zero-shot Slot Filling}. Conference on Empirical Methods in Natural Language Processing.

\bibitem{agarwal2020c3x}
Oshin Agarwal, Heming Ge, Siamak Shakeri, et al. (2020). \textit{Large Scale Knowledge Graph Based Synthetic Corpus Generation for Knowledge-Enhanced Language Model Pre-training}. arXiv.org.

\bibitem{pan2022u7w}
Feifei Pan, Mustafa Canim, Michael R. Glass, et al. (2022). \textit{End-to-End Table Question Answering via Retrieval-Augmented Generation}. arXiv.org.

\bibitem{akbar202053c}
Shayan A. Akbar, and A. Kak (2020). \textit{A Large-Scale Comparative Evaluation of IR-Based Tools for Bug Localization}. IEEE Working Conference on Mining Software Repositories.

\bibitem{pappas20223ck}
Dimitris Pappas, Prodromos Malakasiotis, and Ion Androutsopoulos (2022). \textit{Data Augmentation for Biomedical Factoid Question Answering}. Workshop on Biomedical Natural Language Processing.

\bibitem{kim202056z}
Jihyeok Kim, Seungtaek Choi, Reinald Kim Amplayo, et al. (2020). \textit{Retrieval-Augmented Controllable Review Generation}. International Conference on Computational Linguistics.

\bibitem{gao20238ea}
Yunfan Gao, Yun Xiong, Xinyu Gao, et al. (2023). \textit{Retrieval-Augmented Generation for Large Language Models: A Survey}. arXiv.org.

\bibitem{fan2024pf1}
Wenqi Fan, Yujuan Ding, Liang-bo Ning, et al. (2024). \textit{A Survey on RAG Meeting LLMs: Towards Retrieval-Augmented Large Language Models}. Knowledge Discovery and Data Mining.

\bibitem{xiong2024exb}
Guangzhi Xiong, Qiao Jin, Zhiyong Lu, et al. (2024). \textit{Benchmarking Retrieval-Augmented Generation for Medicine}. Annual Meeting of the Association for Computational Linguistics.

\bibitem{chen2023nzb}
Jiawei Chen, Hongyu Lin, Xianpei Han, et al. (2023). \textit{Benchmarking Large Language Models in Retrieval-Augmented Generation}. AAAI Conference on Artificial Intelligence.

\bibitem{peng2024mp3}
Boci Peng, Yun Zhu, Yongchao Liu, et al. (2024). \textit{Graph Retrieval-Augmented Generation: A Survey}. arXiv.org.

\bibitem{tang2024i5r}
Yixuan Tang, and Yi Yang (2024). \textit{MultiHop-RAG: Benchmarking Retrieval-Augmented Generation for Multi-Hop Queries}. arXiv.org.

\bibitem{he20248lp}
Xiaoxin He, Yijun Tian, Yifei Sun, et al. (2024). \textit{G-Retriever: Retrieval-Augmented Generation for Textual Graph Understanding and Question Answering}. Neural Information Processing Systems.

\bibitem{xu202412d}
Zhentao Xu, Mark Jerome Cruz, Matthew Guevara, et al. (2024). \textit{Retrieval-Augmented Generation with Knowledge Graphs for Customer Service Question Answering}. Annual International ACM SIGIR Conference on Research and Development in Information Retrieval.

\bibitem{yan202437z}
Shi-Qi Yan, Jia-Chen Gu, Yun Zhu, et al. (2024). \textit{Corrective Retrieval Augmented Generation}. arXiv.org.

\bibitem{yu202480d}
Yue Yu, Wei Ping, Zihan Liu, et al. (2024). \textit{RankRAG: Unifying Context Ranking with Retrieval-Augmented Generation in LLMs}. Neural Information Processing Systems.

\bibitem{zeng2024dzl}
Shenglai Zeng, Jiankun Zhang, Pengfei He, et al. (2024). \textit{The Good and The Bad: Exploring Privacy Issues in Retrieval-Augmented Generation (RAG)}. Annual Meeting of the Association for Computational Linguistics.

\bibitem{salemi2024om5}
Alireza Salemi, and Hamed Zamani (2024). \textit{Evaluating Retrieval Quality in Retrieval-Augmented Generation}. Annual International ACM SIGIR Conference on Research and Development in Information Retrieval.

\bibitem{chan2024u69}
Chi-Min Chan, Chunpu Xu, Ruibin Yuan, et al. (2024). \textit{RQ-RAG: Learning to Refine Queries for Retrieval Augmented Generation}. arXiv.org.

\bibitem{kresevic2024uel}
Simone Kresevic, M. Giuffré, M. Ajčević, et al. (2024). \textit{Optimization of hepatological clinical guidelines interpretation by large language models: a retrieval augmented generation-based framework}. npj Digit. Medicine.

\bibitem{mavromatis2024ml9}
Costas Mavromatis, and George Karypis (2024). \textit{GNN-RAG: Graph Neural Retrieval for Large Language Model Reasoning}. arXiv.org.

\bibitem{jin2024yhb}
Jiajie Jin, Yutao Zhu, Xinyu Yang, et al. (2024). \textit{FlashRAG: A Modular Toolkit for Efficient Retrieval-Augmented Generation Research}. The Web Conference.

\bibitem{sarmah20245f3}
Bhaskarjit Sarmah, Dhagash Mehta, Benika Hall, et al. (2024). \textit{HybridRAG: Integrating Knowledge Graphs and Vector Retrieval Augmented Generation for Efficient Information Extraction}. International Conference on AI in Finance.

\bibitem{bechard2024834}
Patrice B'echard, and Orlando Marquez Ayala (2024). \textit{Reducing hallucination in structured outputs via Retrieval-Augmented Generation}. North American Chapter of the Association for Computational Linguistics.

\bibitem{wang20248gm}
Xiaohua Wang, Zhenghua Wang, Xuan Gao, et al. (2024). \textit{Searching for Best Practices in Retrieval-Augmented Generation}. Conference on Empirical Methods in Natural Language Processing.

\bibitem{zou2024iiy}
Wei Zou, Runpeng Geng, Binghui Wang, et al. (2024). \textit{PoisonedRAG: Knowledge Poisoning Attacks to Retrieval-Augmented Generation of Large Language Models}. arXiv.org.

\bibitem{gutierrez2024al5}
Bernal Jimenez Gutierrez, Yiheng Shu, Yu Gu, et al. (2024). \textit{HippoRAG: Neurobiologically Inspired Long-Term Memory for Large Language Models}. Neural Information Processing Systems.

\bibitem{yu2024arx}
Shi Yu, Chaoyue Tang, Bokai Xu, et al. (2024). \textit{VisRAG: Vision-based Retrieval-augmented Generation on Multi-modality Documents}. International Conference on Learning Representations.

\bibitem{guo2024plq}
Zirui Guo, Lianghao Xia, Yanhua Yu, et al. (2024). \textit{LightRAG: Simple and Fast Retrieval-Augmented Generation}. arXiv.org.

\bibitem{li2024wff}
Zhuowan Li, Cheng Li, Mingyang Zhang, et al. (2024). \textit{Retrieval Augmented Generation or Long-Context LLMs? A Comprehensive Study and Hybrid Approach}. Conference on Empirical Methods in Natural Language Processing.

\bibitem{zhao2024931}
Siyun Zhao, Yuqing Yang, Zilong Wang, et al. (2024). \textit{Retrieval Augmented Generation (RAG) and Beyond: A Comprehensive Survey on How to Make your LLMs use External Data More Wisely}. arXiv.org.

\bibitem{huang2024a59}
Yizheng Huang, and Jimmy X. Huang (2024). \textit{A Survey on Retrieval-Augmented Text Generation for Large Language Models}. arXiv.org.

\bibitem{xie20245dq}
Qianqian Xie, Weiguang Han, Zhengyu Chen, et al. (2024). \textit{FinBen: A Holistic Financial Benchmark for Large Language Models}. Neural Information Processing Systems.

\bibitem{wu2024bpc}
Shangyu Wu, Ying Xiong, Yufei Cui, et al. (2024). \textit{Retrieval-Augmented Generation for Natural Language Processing: A Survey}. arXiv.org.

\bibitem{lyu2024ngu}
Yuanjie Lyu, Zhiyu Li, Simin Niu, et al. (2024). \textit{CRUD-RAG: A Comprehensive Chinese Benchmark for Retrieval-Augmented Generation of Large Language Models}. ACM Trans. Inf. Syst..

\bibitem{deng2024k1b}
Gelei Deng, Yi Liu, Kailong Wang, et al. (2024). \textit{Pandora: Jailbreak GPTs by Retrieval Augmented Generation Poisoning}. Proceedings 2024 Workshop on AI Systems with Confidential COmputing.

\bibitem{soudani20247ny}
Heydar Soudani, E. Kanoulas, and Faegheh Hasibi (2024). \textit{Fine Tuning vs. Retrieval Augmented Generation for Less Popular Knowledge}. SIGIR-AP.

\bibitem{krishna2024qsh}
Satyapriya Krishna, Kalpesh Krishna, Anhad Mohananey, et al. (2024). \textit{Fact, Fetch, and Reason: A Unified Evaluation of Retrieval-Augmented Generation}. North American Chapter of the Association for Computational Linguistics.

\bibitem{zhou20248fu}
Yujia Zhou, Yan Liu, Xiaoxi Li, et al. (2024). \textit{Trustworthiness in Retrieval-Augmented Generation Systems: A Survey}. arXiv.org.

\bibitem{pipitone2024sfx}
Nicholas Pipitone, and Ghita Houir Alami (2024). \textit{LegalBench-RAG: A Benchmark for Retrieval-Augmented Generation in the Legal Domain}. arXiv.org.

\bibitem{jin20247cr}
Chao Jin, Zili Zhang, Xuanlin Jiang, et al. (2024). \textit{RAGCache: Efficient Knowledge Caching for Retrieval-Augmented Generation}. ACM Transactions on Computer Systems.

\bibitem{wang20246hs}
Zilong Wang, Zifeng Wang, Long T. Le, et al. (2024). \textit{Speculative RAG: Enhancing Retrieval Augmented Generation through Drafting}. International Conference on Learning Representations.

\bibitem{tihanyi2024d5e}
Norbert Tihanyi, M. Ferrag, Ridhi Jain, et al. (2024). \textit{CyberMetric: A Benchmark Dataset based on Retrieval-Augmented Generation for Evaluating LLMs in Cybersecurity Knowledge}. Computer Science Symposium in Russia.

\bibitem{zou2024haa}
Wei Zou, Runpeng Geng, Binghui Wang, et al. (2024). \textit{PoisonedRAG: Knowledge Corruption Attacks to Retrieval-Augmented Generation of Large Language Models}. Unpublished manuscript.

\bibitem{xiong2024u1b}
Guangzhi Xiong, Qiao Jin, Xiao Wang, et al. (2024). \textit{Improving Retrieval-Augmented Generation in Medicine with Iterative Follow-up Questions}. Pacific Symposium on Biocomputing. Pacific Symposium on Biocomputing.

\bibitem{fang2024gh6}
Feiteng Fang, Yuelin Bai, Shiwen Ni, et al. (2024). \textit{Enhancing Noise Robustness of Retrieval-Augmented Language Models with Adaptive Adversarial Training}. Annual Meeting of the Association for Computational Linguistics.

\bibitem{hu2024eyw}
Yuntong Hu, Zhihan Lei, Zhengwu Zhang, et al. (2024). \textit{GRAG: Graph Retrieval-Augmented Generation}. North American Chapter of the Association for Computational Linguistics.

\bibitem{xue2024bxd}
Jiaqi Xue, Meng Zheng, Yebowen Hu, et al. (2024). \textit{BadRAG: Identifying Vulnerabilities in Retrieval Augmented Generation of Large Language Models}. arXiv.org.

\bibitem{jeong2024cey}
Minbyul Jeong, Jiwoong Sohn, Mujeen Sung, et al. (2024). \textit{Improving medical reasoning through retrieval and self-reflection with retrieval-augmented large language models}. Bioinform..

\bibitem{matsumoto2024b7a}
Nicholas Matsumoto, Jay Moran, Hyunjun Choi, et al. (2024). \textit{KRAGEN: a knowledge graph-enhanced RAG framework for biomedical problem solving using large language models}. Bioinformatics.

\bibitem{friel20241ct}
Robert Friel, Masha Belyi, and Atindriyo Sanyal (2024). \textit{RAGBench: Explainable Benchmark for Retrieval-Augmented Generation Systems}. arXiv.org.

\bibitem{procko202417i}
T. Procko, and Omar Ochoa (2024). \textit{Graph Retrieval-Augmented Generation for Large Language Models: A Survey}. 2024 Conference on AI, Science, Engineering, and Technology (AIxSET).

\bibitem{wang2024dt8}
Hongru Wang, Wenyu Huang, Yang Deng, et al. (2024). \textit{UniMS-RAG: A Unified Multi-source Retrieval-Augmented Generation for Personalized Dialogue Systems}. arXiv.org.

\bibitem{zhang2025gnc}
Qinggang Zhang, Shengyuan Chen, Yuan-Qi Bei, et al. (2025). \textit{A Survey of Graph Retrieval-Augmented Generation for Customized Large Language Models}. arXiv.org.

\bibitem{cheng2024d7k}
Pengzhou Cheng, Yidong Ding, Tianjie Ju, et al. (2024). \textit{TrojanRAG: Retrieval-Augmented Generation Can Be Backdoor Driver in Large Language Models}. arXiv.org.

\bibitem{yue2024ump}
Zhenrui Yue, Honglei Zhuang, Aijun Bai, et al. (2024). \textit{Inference Scaling for Long-Context Retrieval Augmented Generation}. International Conference on Learning Representations.

\bibitem{jiang20243ac}
Wenqi Jiang, Shuai Zhang, Boran Han, et al. (2024). \textit{PipeRAG: Fast Retrieval-Augmented Generation via Algorithm-System Co-design}. arXiv.org.

\bibitem{ge20246t5}
J. Ge, Steve Sun, Joseph Owens, et al. (2024). \textit{Development of a liver disease–specific large language model chat interface using retrieval-augmented generation}. Hepatology.

\bibitem{ding20249ne}
Yujuan Ding, Wenqi Fan, Liang-bo Ning, et al. (2024). \textit{A Survey on RAG Meets LLMs: Towards Retrieval-Augmented Large Language Models}. arXiv.org.

\bibitem{sun2024eoe}
ZhongXiang Sun, Xiaoxue Zang, Kai Zheng, et al. (2024). \textit{ReDeEP: Detecting Hallucination in Retrieval-Augmented Generation via Mechanistic Interpretability}. International Conference on Learning Representations.

\bibitem{ma2024pwd}
Shengjie Ma, Chengjin Xu, Xuhui Jiang, et al. (2024). \textit{Think-on-Graph 2.0: Deep and Faithful Large Language Model Reasoning with Knowledge-guided Retrieval Augmented Generation}. International Conference on Learning Representations.

\bibitem{bornea2024jde}
Andrei-Laurentiu Bornea, Fadhel Ayed, Antonio De Domenico, et al. (2024). \textit{Telco-RAG: Navigating the Challenges of Retrieval Augmented Language Models for Telecommunications}. Global Communications Conference.

\bibitem{yang20243nb}
Diji Yang, Jinmeng Rao, Kezhen Chen, et al. (2024). \textit{IM-RAG: Multi-Round Retrieval-Augmented Generation Through Learning Inner Monologues}. Annual International ACM SIGIR Conference on Research and Development in Information Retrieval.

\bibitem{su20241om}
Weihang Su, Yichen Tang, Qingyao Ai, et al. (2024). \textit{DRAGIN: Dynamic Retrieval Augmented Generation based on the Real-time Information Needs of Large Language Models}. Annual Meeting of the Association for Computational Linguistics.

\bibitem{islam2024ug5}
Shayekh Bin Islam, Md Asib Rahman, K. S. M. T. Hossain, et al. (2024). \textit{Open-RAG: Enhanced Retrieval-Augmented Reasoning with Open-Source Large Language Models}. Conference on Empirical Methods in Natural Language Processing.

\bibitem{liu2025p6t}
Siru Liu, Allison B. McCoy, and Adam Wright (2025). \textit{Improving large language model applications in biomedicine with retrieval-augmented generation: a systematic review, meta-analysis, and clinical development guidelines}. J. Am. Medical Informatics Assoc..

\bibitem{ke20248bm}
Yuhe Ke, Liyuan Jin, Kabilan Elangovan, et al. (2024). \textit{Development and Testing of Retrieval Augmented Generation in Large Language Models - A Case Study Report}. arXiv.org.

\bibitem{ni2025ox9}
Bo Ni, Zheyuan Liu, Leyao Wang, et al. (2025). \textit{Towards Trustworthy Retrieval Augmented Generation for Large Language Models: A Survey}. arXiv.org.

\bibitem{lee2024hif}
Myeonghwa Lee, Seonho An, and Min-Soo Kim (2024). \textit{PlanRAG: A Plan-then-Retrieval Augmented Generation for Generative Large Language Models as Decision Makers}. North American Chapter of the Association for Computational Linguistics.

\bibitem{li2024hb4}
Mufei Li, Siqi Miao, and Pan Li (2024). \textit{Simple is Effective: The Roles of Graphs and Large Language Models in Knowledge-Graph-Based Retrieval-Augmented Generation}. International Conference on Learning Representations.

\bibitem{ke2025wm0}
Yuhe Ke, Liyuan Jin, Kabilan Elangovan, et al. (2025). \textit{Retrieval augmented generation for 10 large language models and its generalizability in assessing medical fitness}. npj Digit. Medicine.

\bibitem{wang2024zt3}
Zheng Wang, Shu Xian Teo, Jieer Ouyang, et al. (2024). \textit{M-RAG: Reinforcing Large Language Model Performance through Retrieval-Augmented Generation with Multiple Partitions}. Annual Meeting of the Association for Computational Linguistics.

\bibitem{kang2024hrb}
Mintong Kang, Nezihe Merve Gurel, Ning Yu, et al. (2024). \textit{C-RAG: Certified Generation Risks for Retrieval-Augmented Language Models}. International Conference on Machine Learning.

\bibitem{lin2024s1v}
Demiao Lin (2024). \textit{Revolutionizing Retrieval-Augmented Generation with Enhanced PDF Structure Recognition}. arXiv.org.

\bibitem{guinet2024vkg}
Gauthier Guinet, Behrooz Omidvar-Tehrani, Anoop Deoras, et al. (2024). \textit{Automated Evaluation of Retrieval-Augmented Language Models with Task-Specific Exam Generation}. International Conference on Machine Learning.

\bibitem{radeva2024vai}
I. Radeva, I. Popchev, L. Doukovska, et al. (2024). \textit{Web Application for Retrieval-Augmented Generation: Implementation and Testing}. Electronics.

\bibitem{soman2023m86}
Karthik Soman, Peter W Rose, John H Morris, et al. (2023). \textit{Biomedical knowledge graph-optimized prompt generation for large language models}. Bioinformatics.

\bibitem{chen20245d2}
Zhanpeng Chen, Chengjin Xu, Yiyan Qi, et al. (2024). \textit{MLLM Is a Strong Reranker: Advancing Multimodal Retrieval-augmented Generation via Knowledge-enhanced Reranking and Noise-injected Training}. arXiv.org.

\bibitem{unlu2024yc8}
Ozan Unlu, Jiyeon Shin, Charlotte J. Mailly, et al. (2024). \textit{Retrieval Augmented Generation Enabled Generative Pre-Trained Transformer 4 (GPT-4) Performance for Clinical Trial Screening}. medRxiv.

\bibitem{ge20237yq}
J. Ge, Steve Sun, Joseph Owens, et al. (2023). \textit{Development of a Liver Disease-Specific Large Language Model Chat Interface using Retrieval Augmented Generation}. medRxiv.

\bibitem{rau20244nr}
David Rau, Herv'e D'ejean, Nadezhda Chirkova, et al. (2024). \textit{BERGEN: A Benchmarking Library for Retrieval-Augmented Generation}. Conference on Empirical Methods in Natural Language Processing.

\bibitem{bora20242mq}
Arunabh Bora, and H. Cuayáhuitl (2024). \textit{Systematic Analysis of Retrieval-Augmented Generation-Based LLMs for Medical Chatbot Applications}. Machine Learning and Knowledge Extraction.

\bibitem{pradeep2024n91}
Ronak Pradeep, Nandan Thakur, Sahel Sharifymoghaddam, et al. (2024). \textit{Ragnarök: A Reusable RAG Framework and Baselines for TREC 2024 Retrieval-Augmented Generation Track}. European Conference on Information Retrieval.

\bibitem{zhao20248wm}
Qingfei Zhao, Ruobing Wang, Yukuo Cen, et al. (2024). \textit{LongRAG: A Dual-Perspective Retrieval-Augmented Generation Paradigm for Long-Context Question Answering}. Conference on Empirical Methods in Natural Language Processing.

\bibitem{chirkova2024kde}
Nadezhda Chirkova, David Rau, Herv'e D'ejean, et al. (2024). \textit{Retrieval-augmented generation in multilingual settings}. KNOWLLM.

\bibitem{dong2024qcd}
Guanting Dong, Yutao Zhu, Chenghao Zhang, et al. (2024). \textit{Understand What LLM Needs: Dual Preference Alignment for Retrieval-Augmented Generation}. The Web Conference.

\bibitem{lu2024pvt}
Songshuo Lu, Hua Wang, Yutian Rong, et al. (2024). \textit{TurboRAG: Accelerating Retrieval-Augmented Generation with Precomputed KV Caches for Chunked Text}. arXiv.org.

\bibitem{zhu2024h7i}
Yun Zhu, Jia-Chen Gu, Caitlin Sikora, et al. (2024). \textit{Accelerating Inference of Retrieval-Augmented Generation via Sparse Context Selection}. International Conference on Learning Representations.

\bibitem{yu2024c32}
Tian Yu, Shaolei Zhang, and Yang Feng (2024). \textit{Auto-RAG: Autonomous Retrieval-Augmented Generation for Large Language Models}. arXiv.org.

\bibitem{amugongo202530u}
L. M. Amugongo, Pietro Mascheroni, Steven Brooks, et al. (2025). \textit{Retrieval augmented generation for large language models in healthcare: A systematic review}. PLOS Digital Health.

\bibitem{hui2024tsz}
Yulong Hui, Yao Lu, and Huanchen Zhang (2024). \textit{UDA: A Benchmark Suite for Retrieval Augmented Generation in Real-world Document Analysis}. Neural Information Processing Systems.

\bibitem{khaliq2024ne2}
M. A. Khaliq, P. Chang, M. Ma, et al. (2024). \textit{RAGAR, Your Falsehood Radar: RAG-Augmented Reasoning for Political Fact-Checking using Multimodal Large Language Models}. FEVER.

\bibitem{salemi2024bb6}
Alireza Salemi, and Hamed Zamani (2024). \textit{Towards a Search Engine for Machines: Unified Ranking for Multiple Retrieval-Augmented Large Language Models}. Annual International ACM SIGIR Conference on Research and Development in Information Retrieval.

\bibitem{xu2024397}
Shicheng Xu, Liang Pang, Mo Yu, et al. (2024). \textit{Unsupervised Information Refinement Training of Large Language Models for Retrieval-Augmented Generation}. Annual Meeting of the Association for Computational Linguistics.

\bibitem{hu2024i6h}
Zhibo Hu, Chen Wang, Yanfeng Shu, et al. (2024). \textit{Prompt Perturbation in Retrieval-Augmented Generation based Large Language Models}. Knowledge Discovery and Data Mining.

\bibitem{sohn2024w2t}
Jiwoong Sohn, Yein Park, Chanwoong Yoon, et al. (2024). \textit{Rationale-Guided Retrieval Augmented Generation for Medical Question Answering}. North American Chapter of the Association for Computational Linguistics.

\bibitem{qi2024tlf}
Zehan Qi, Rongwu Xu, Zhijiang Guo, et al. (2024). \textit{LONG²RAG: Evaluating Long-Context & Long-Form Retrieval-Augmented Generation with Key Point Recall}. Conference on Empirical Methods in Natural Language Processing.

\bibitem{han2024mpx}
Binglan Han, Teo Sušnjak, and A. Mathrani (2024). \textit{Automating Systematic Literature Reviews with Retrieval-Augmented Generation: A Comprehensive Overview}. Applied Sciences.

\bibitem{zhao2024go5}
Yiyun Zhao, Prateek Singh, Hanoz Bhathena, et al. (2024). \textit{Optimizing LLM Based Retrieval Augmented Generation Pipelines in the Financial Domain}. North American Chapter of the Association for Computational Linguistics.

\bibitem{li20243nz}
Xinze Li, Senkun Mei, Zhenghao Liu, et al. (2024). \textit{RAG-DDR: Optimizing Retrieval-Augmented Generation Using Differentiable Data Rewards}. International Conference on Learning Representations.

\bibitem{wang2024kca}
Fei Wang, Xingchen Wan, Ruoxi Sun, et al. (2024). \textit{Astute RAG: Overcoming Imperfect Retrieval Augmentation and Knowledge Conflicts for Large Language Models}. Annual Meeting of the Association for Computational Linguistics.

\bibitem{akkiraju2024edc}
Rama Akkiraju, Anbang Xu, Deepak Bora, et al. (2024). \textit{FACTS About Building Retrieval Augmented Generation-based Chatbots}. arXiv.org.

\bibitem{zhou20249ba}
Qingqing Zhou, Can Liu, Yuchen Duan, et al. (2024). \textit{GastroBot: a Chinese gastrointestinal disease chatbot based on the retrieval-augmented generation}. Frontiers in Medicine.

\bibitem{kim2024t1i}
Dongkyu Kim, Byoungwook Kim, Donggeon Han, et al. (2024). \textit{AutoRAG: Automated Framework for optimization of Retrieval Augmented Generation Pipeline}. arXiv.org.

\bibitem{yilma20249sl}
G. M. Yilma, J. Ayala-Romero, A. Garcia-Saavedra, et al. (2024). \textit{TelecomRAG: Taming Telecom Standards with Retrieval Augmented Generation and LLMs}. Computer communication review.

\bibitem{xu20242x1}
Haowen Xu, Jinghui Yuan, Anye Zhou, et al. (2024). \textit{GenAI-powered Multi-Agent Paradigm for Smart Urban Mobility: Opportunities and Challenges for Integrating Large Language Models (LLMs) and Retrieval-Augmented Generation (RAG) with Intelligent Transportation Systems}. arXiv.org.

\bibitem{xu2024dgv}
Ran Xu, Hui Liu, Sreyashi Nag, et al. (2024). \textit{SimRAG: Self-Improving Retrieval-Augmented Generation for Adapting Large Language Models to Specialized Domains}. North American Chapter of the Association for Computational Linguistics.

\bibitem{liu2024878}
Suqing Liu, Zezhu Yu, Feiran Huang, et al. (2024). \textit{Can Small Language Models With Retrieval-Augmented Generation Replace Large Language Models When Learning Computer Science?}. Annual Conference on Innovation and Technology in Computer Science Education.

\bibitem{zeng2024vmz}
Huimin Zeng, Zhenrui Yue, Qian Jiang, et al. (2024). \textit{Federated Recommendation via Hybrid Retrieval Augmented Generation}. BigData Congress [Services Society].

\bibitem{bhattarai2024zkd}
Manish Bhattarai, Javier E. Santos, Shawn Jones, et al. (2024). \textit{Enhancing Code Translation in Language Models with Few-Shot Learning via Retrieval-Augmented Generation}. IEEE Conference on High Performance Extreme Computing.

\bibitem{wang2024ac6}
Shuting Wang, Jiongnan Liu, Jiehan Cheng, et al. (2024). \textit{DomainRAG: A Chinese Benchmark for Evaluating Domain-specific Retrieval-Augmented Generation}. arXiv.org.

\bibitem{omrani2024i22}
Pouria Omrani, Alireza Hosseini, Kiana Hooshanfar, et al. (2024). \textit{Hybrid Retrieval-Augmented Generation Approach for LLMs Query Response Enhancement}. 2024 10th International Conference on Web Research (ICWR).

\bibitem{tozuka2024nau}
Ryota Tozuka, Hisashi Johno, Akitomo Amakawa, et al. (2024). \textit{Application of NotebookLM, a Large Language Model with Retrieval-Augmented Generation, for Lung Cancer Staging}. Japanese Journal of Radiology.

\bibitem{ma20245jl}
Xueguang Ma, Shengyao Zhuang, B. Koopman, et al. (2024). \textit{VISA: Retrieval Augmented Generation with Visual Source Attribution}. arXiv.org.

\bibitem{yang2024128}
Si-Nan Yang, Dong Wang, Haoqi Zheng, et al. (2024). \textit{TimeRAG: BOOSTING LLM Time Series Forecasting via Retrieval-Augmented Generation}. IEEE International Conference on Acoustics, Speech, and Signal Processing.

\bibitem{lakatos202456t}
Robert Lakatos, P. Pollner, András Hajdu, et al. (2024). \textit{Investigating the performance of Retrieval-Augmented Generation and fine-tuning for the development of AI-driven knowledge-based systems}. Machine Learning and Knowledge Extraction.

\bibitem{chen20247nc}
Zhuo Chen, Jiawei Liu, Haotan Liu, et al. (2024). \textit{Black-Box Opinion Manipulation Attacks to Retrieval-Augmented Generation of Large Language Models}. arXiv.org.

\bibitem{zerhoudi2024y9l}
Saber Zerhoudi, and Michael Granitzer (2024). \textit{PersonaRAG: Enhancing Retrieval-Augmented Generation Systems with User-Centric Agents}. IR-RAG@SIGIR.

\bibitem{yu2025b4u}
Dazhou Yu, Riyang Bao, Gengchen Mai, et al. (2025). \textit{Spatial-RAG: Spatial Retrieval Augmented Generation for Real-World Spatial Reasoning Questions}. arXiv.org.

\bibitem{ghadban2023j9e}
Yasmina Al Ghadban, Yvonne Lu, Uday Adavi, et al. (2023). \textit{Transforming Healthcare Education: Harnessing Large Language Models for Frontline Health Worker Capacity Building using Retrieval-Augmented Generation}. medRxiv.

\bibitem{liang2025f4q}
Xun Liang, Simin Niu, Zhiyu Li, et al. (2025). \textit{SafeRAG: Benchmarking Security in Retrieval-Augmented Generation of Large Language Model}. Annual Meeting of the Association for Computational Linguistics.

\bibitem{quinn2024n3o}
Derrick Quinn, Mohammad Nouri, Neel Patel, et al. (2024). \textit{Accelerating Retrieval-Augmented Generation}. International Conference on Architectural Support for Programming Languages and Operating Systems.

\bibitem{tan2024l5v}
Hanzhuo Tan, Qi Luo, Lingixao Jiang, et al. (2024). \textit{Prompt-based Code Completion via Multi-Retrieval Augmented Generation}. ACM Transactions on Software Engineering and Methodology.

\bibitem{hajiaghayi20245ir}
Mohammadtaghi Hajiaghayi, S'ebastien Lahaie, Keivan Rezaei, et al. (2024). \textit{Ad Auctions for LLMs via Retrieval Augmented Generation}. Neural Information Processing Systems.

\bibitem{garigliotti2024sco}
Darío Garigliotti (2024). \textit{SDG target detection in environmental reports using Retrieval-augmented Generation with LLMs}. CLIMATENLP.

\bibitem{barron2024kue}
Ryan Barron, Ves Grantcharov, Selma Wanna, et al. (2024). \textit{Domain-Specific Retrieval-Augmented Generation Using Vector Stores, Knowledge Graphs, and Tensor Factorization}. International Conference on Machine Learning and Applications.

\bibitem{zhang2025byv}
Shiyue Zhang, Mark Dredze, AI Bloomberg, et al. (2025). \textit{RAG LLMs are Not Safer: A Safety Analysis of Retrieval-Augmented Generation for Large Language Models}. North American Chapter of the Association for Computational Linguistics.

\bibitem{gan2024id0}
Chunjing Gan, Dan Yang, Binbin Hu, et al. (2024). \textit{Similarity is Not All You Need: Endowing Retrieval Augmented Generation with Multi Layered Thoughts}. arXiv.org.

\bibitem{wang20245w8}
Shuting Wang, Xin Xu, Mang Wang, et al. (2024). \textit{RichRAG: Crafting Rich Responses for Multi-faceted Queries in Retrieval-Augmented Generation}. International Conference on Computational Linguistics.

\bibitem{li2024w6r}
Yuying Li, Gaoyang Liu, Chen Wang, et al. (2024). \textit{Generating Is Believing: Membership Inference Attacks against Retrieval-Augmented Generation}. IEEE International Conference on Acoustics, Speech, and Signal Processing.

\bibitem{fu2024m5q}
Jia Fu, Xiaoting Qin, Fangkai Yang, et al. (2024). \textit{AutoRAG-HP: Automatic Online Hyper-Parameter Tuning for Retrieval-Augmented Generation}. Conference on Empirical Methods in Natural Language Processing.

\bibitem{liu2024nei}
Huanshuo Liu, Hao Zhang, Zhijiang Guo, et al. (2024). \textit{CtrlA: Adaptive Retrieval-Augmented Generation via Probe-Guided Control}. arXiv.org.

\bibitem{muludi2024ehk}
Kurnia Muludi, Kaira Milani Fitria, Joko Triloka, et al. (2024). \textit{Retrieval-Augmented Generation Approach: Document Question Answering using Large Language Model}. International Journal of Advanced Computer Science and Applications.

\bibitem{lahiri2024i1q}
A. Lahiri, and Q. Hu (2024). \textit{AlzheimerRAG: Multimodal Retrieval Augmented Generation for PubMed articles}. arXiv.org.

\bibitem{hei2024cs4}
Zijian Hei, Weiling Liu, Wenjie Ou, et al. (2024). \textit{DR-RAG: Applying Dynamic Document Relevance to Retrieval-Augmented Generation for Question-Answering}. arXiv.org.

\bibitem{zhang2024rwm}
Yucheng Zhang, Qinfeng Li, Tianyu Du, et al. (2024). \textit{HijackRAG: Hijacking Attacks against Retrieval-Augmented Large Language Models}. arXiv.org.

\bibitem{qi2024g7x}
Jirui Qi, Gabriele Sarti, R. Fern'andez, et al. (2024). \textit{Model Internals-based Answer Attribution for Trustworthy Retrieval-Augmented Generation}. Conference on Empirical Methods in Natural Language Processing.

\bibitem{he2024hos}
Bolei He, Nuo Chen, Xinran He, et al. (2024). \textit{Retrieving, Rethinking and Revising: The Chain-of-Verification Can Improve Retrieval Augmented Generation}. Conference on Empirical Methods in Natural Language Processing.

\bibitem{qin202445s}
Ruiyang Qin, Zheyu Yan, Dewen Zeng, et al. (2024). \textit{Robust Implementation of Retrieval-Augmented Generation on Edge-Based Computing-in-Memory Architectures}. International Conference on Computer Aided Design.

\bibitem{wang20245b5}
Zheng Wang, Zhongyang Li, Zeren Jiang, et al. (2024). \textit{Crafting Personalized Agents through Retrieval-Augmented Generation on Editable Memory Graphs}. Conference on Empirical Methods in Natural Language Processing.

\bibitem{merth20243h7}
Thomas Merth, Qichen Fu, Mohammad Rastegari, et al. (2024). \textit{Superposition Prompting: Improving and Accelerating Retrieval-Augmented Generation}. International Conference on Machine Learning.

\bibitem{chen20247c1}
Jiajing Chen, Runyuan Bao, Hongye Zheng, et al. (2024). \textit{Optimizing Retrieval-Augmented Generation with Elasticsearch for Enhanced Question-Answering Systems}. 2024 5th International Conference on Big Data, Artificial Intelligence and Internet of Things Engineering (ICBAIE).

\bibitem{thorpe2024l37}
Dayton G. Thorpe, Andrew Duberstein, and Ian A. Kinsey (2024). \textit{Dubo-SQL: Diverse Retrieval-Augmented Generation and Fine Tuning for Text-to-SQL}. arXiv.org.

\bibitem{loumachi2024nxa}
Fatma Yasmine Loumachi, M. C. Ghanem, and M. Ferrag (2024). \textit{Advancing Cyber Incident Timeline Analysis Through Retrieval-Augmented Generation and Large Language Models}. De Computis.

\bibitem{xu2024be3}
Weiye Xu, Min Wang, Wen-gang Zhou, et al. (2024). \textit{P-RAG: Progressive Retrieval Augmented Generation For Planning on Embodied Everyday Task}. ACM Multimedia.

\bibitem{fayyazi2024h99}
Reza Fayyazi, Rozhina Taghdimi, and S. Yang (2024). \textit{Advancing TTP Analysis: Harnessing the Power of Encoder-Only and Decoder-Only Language Models with Retrieval Augmented Generation}. arXiv.org.

\bibitem{wang2024ad6}
Xi Wang, Procheta Sen, Ruizhe Li, et al. (2024). \textit{Adaptive Retrieval-Augmented Generation for Conversational Systems}. North American Chapter of the Association for Computational Linguistics.

\bibitem{kuo2024gi6}
Tzu-Lin Kuo, Fengting Liao, Mu-Wei Hsieh, et al. (2024). \textit{RAD-Bench: Evaluating Large Language Models Capabilities in Retrieval Augmented Dialogues}. arXiv.org.

\bibitem{yazaki20245js}
Megumi Yazaki, S. Maki, T. Furuya, et al. (2024). \textit{Emergency Patient Triage Improvement through a Retrieval-Augmented Generation Enhanced Large-Scale Language Model}. Prehospital Emergency Care.

\bibitem{clop2024zs2}
Cody Clop, and Yannick Teglia (2024). \textit{Backdoored Retrievers for Prompt Injection Attacks on Retrieval Augmented Generation of Large Language Models}. arXiv.org.

\bibitem{lee20240to}
Jaedong Lee, H. Cha, Y. Hwangbo, et al. (2024). \textit{Enhancing Large Language Model Reliability: Minimizing Hallucinations with Dual Retrieval-Augmented Generation Based on the Latest Diabetes Guidelines}. Journal of Personalized Medicine.

\bibitem{chen2025tux}
Zhe Chen, Yusheng Liao, Shuyang Jiang, et al. (2025). \textit{Towards Omni-RAG: Comprehensive Retrieval-Augmented Generation for Large Language Models in Medical Applications}. Annual Meeting of the Association for Computational Linguistics.

\bibitem{garcia2024qd5}
Brandon T Garcia, Lauren Westerfield, Priya Yelemali, et al. (2024). \textit{Improving Automated Deep Phenotyping Through Large Language Models Using Retrieval Augmented Generation}. medRxiv.

\bibitem{yang20255fx}
Qimin Yang, Huan Zuo, Runqi Su, et al. (2025). \textit{Dual retrieving and ranking medical large language model with retrieval augmented generation}. Scientific Reports.

\bibitem{dong2023i5q}
Chenxi Dong (2023). \textit{How to Build an AI Tutor that Can Adapt to Any Course and Provide Accurate Answers Using Large Language Model and Retrieval-Augmented Generation}. arXiv.org.

\bibitem{wu2024o9r}
Feifan Wu, Lingyuan Liu, Wentao He, et al. (2024). \textit{Time-Sensitve Retrieval-Augmented Generation for Question Answering}. International Conference on Information and Knowledge Management.

\bibitem{li2024oot}
Dongyang Li, Junbing Yan, Taolin Zhang, et al. (2024). \textit{On the Role of Long-tail Knowledge in Retrieval Augmented Large Language Models}. Annual Meeting of the Association for Computational Linguistics.

\bibitem{sharma2024t3p}
Kartik Sharma, Peeyush Kumar, and Yunqing Li (2024). \textit{OG-RAG: Ontology-Grounded Retrieval-Augmented Generation For Large Language Models}. arXiv.org.

\bibitem{leekha2024pac}
R. Leekha, Olga Simek, and Charlie Dagli (2024). \textit{War of Words: Harnessing the Potential of Large Language Models and Retrieval Augmented Generation to Classify, Counter and Diffuse Hate Speech}. The Florida AI Research Society.

\bibitem{xu2024w5j}
Ruiyu Xu, Ying Hong, Feifei Zhang, et al. (2024). \textit{Evaluation of the integration of retrieval-augmented generation in large language model for breast cancer nursing care responses}. Scientific Reports.

\bibitem{low2025gjc}
Y. Low, Michael L. Jackson, Rebecca J. Hyde, et al. (2025). \textit{Answering real-world clinical questions using large language model, retrieval-augmented generation, and agentic systems}. Digital Health.

\bibitem{chen2024iyt}
Weijie Chen, Ting Bai, Jinbo Su, et al. (2024). \textit{KG-Retriever: Efficient Knowledge Indexing for Retrieval-Augmented Large Language Models}. arXiv.org.

\bibitem{zhu2024yj5}
Yutao Zhu, Zhaoheng Huang, Zhicheng Dou, et al. (2024). \textit{One Token Can Help! Learning Scalable and Pluggable Virtual Tokens for Retrieval-Augmented Large Language Models}. AAAI Conference on Artificial Intelligence.

\bibitem{verma2024f91}
Sourav Verma (2024). \textit{Contextual Compression in Retrieval-Augmented Generation for Large Language Models: A Survey}. arXiv.org.

\bibitem{yao20240zt}
Chengyuan Yao, and Satoshi Fujita (2024). \textit{Adaptive Control of Retrieval-Augmented Generation for Large Language Models Through Reflective Tags}. Electronics.

\bibitem{leite2025k0s}
Marcus Vinicius Leite, J. Abe, Marcos Leandro Hoffmann Souza, et al. (2025). \textit{Enhancing Environmental Control in Broiler Production: Retrieval-Augmented Generation for Improved Decision-Making with Large Language Models}. AgriEngineering.

\bibitem{burgan20246u3}
Cara Burgan, Josiah Kowalski, and Weidong Liao (2024). \textit{Developing a Retrieval Augmented Generation (RAG) Chatbot App Using Adaptive Large Language Models (LLM) and LangChain Framework}. Proceedings of the West Virginia Academy of Science.

\bibitem{chu2025wz5}
Yun-Wei Chu, Kai Zhang, Christopher Malon, et al. (2025). \textit{Reducing Hallucinations of Medical Multimodal Large Language Models with Visual Retrieval-Augmented Generation}. arXiv.org.

\bibitem{efeoglu20242eq}
Sefika Efeoglu, and Adrian Paschke (2024). \textit{Relation Extraction with Fine-Tuned Large Language Models in Retrieval Augmented Generation Frameworks}. arXiv.org.

\bibitem{yu2024dv5}
Jeffy Yu (2024). \textit{Retrieval Augmented Generation Integrated Large Language Models in Smart Contract Vulnerability Detection}. arXiv.org.

\bibitem{feng20249iv}
Kan Feng, Lijun Luo, Yongjun Xia, et al. (2024). \textit{Optimizing Microservice Deployment in Edge Computing with Large Language Models: Integrating Retrieval Augmented Generation and Chain of Thought Techniques}. Symmetry.

\bibitem{pichai2023n5p}
Kieran Pichai (2023). \textit{A Retrieval-Augmented Generation Based Large Language Model Benchmarked On a Novel Dataset}. Journal of student-scientists' research.

\bibitem{fayyazi2023qg6}
Reza Fayyazi, Rozhina Taghdimi, and S. Yang (2023). \textit{Advancing TTP Analysis: Harnessing the Power of Large Language Models with Retrieval Augmented Generation}. 2024 Annual Computer Security Applications Conference Workshops (ACSAC Workshops).

\bibitem{sudhi20240uy}
Viju Sudhi, Sinchana Ramakanth Bhat, Max Rudat, et al. (2024). \textit{RAG-Ex: A Generic Framework for Explaining Retrieval Augmented Generation}. Annual International ACM SIGIR Conference on Research and Development in Information Retrieval.

\bibitem{wang2025klk}
Shijie Wang, Wenqi Fan, Yue Feng, et al. (2025). \textit{Knowledge Graph Retrieval-Augmented Generation for LLM-based Recommendation}. Annual Meeting of the Association for Computational Linguistics.

\bibitem{wu2025eum}
Yin Wu, Quanyu Long, Jing Li, et al. (2025). \textit{Visual-RAG: Benchmarking Text-to-Image Retrieval Augmented Generation for Visual Knowledge Intensive Queries}. arXiv.org.

\bibitem{yang20248km}
Rui Yang (2024). \textit{CaseGPT: a case reasoning framework based on language models and retrieval-augmented generation}. arXiv.org.

\bibitem{huang2024grc}
Zhongzhen Huang, Kui Xue, Yongqi Fan, et al. (2024). \textit{Tool Calling: Enhancing Medication Consultation via Retrieval-Augmented Large Language Models}. arXiv.org.

\bibitem{lv202521d}
Peizhuo Lv, Mengjie Sun, Hao Wang, et al. (2025). \textit{RAG-WM: An Efficient Black-Box Watermarking Approach for Retrieval-Augmented Generation of Large Language Models}. arXiv.org.

\bibitem{jiao20259xa}
Yang Jiao, Xiaodong Wang, and Kai Yang (2025). \textit{PR-Attack: Coordinated Prompt-RAG Attacks on Retrieval-Augmented Generation in Large Language Models via Bilevel Optimization}. Annual International ACM SIGIR Conference on Research and Development in Information Retrieval.

\bibitem{wang2024ywz}
Haijin Wang, Mianrong Zhang, Zheng Chen, et al. (2024). \textit{Carbon Footprint Accounting Driven by Large Language Models and Retrieval-augmented Generation}. arXiv.org.

\bibitem{patel2024h7u}
Hetul Niteshbhai Patel, Azara Surti, Parth Goel, et al. (2024). \textit{A Comparative Analysis of Large Language Models with Retrieval-Augmented Generation based Question Answering System}. 2024 8th International Conference on I-SMAC (IoT in Social, Mobile, Analytics and Cloud) (I-SMAC).

\bibitem{hikov2024rme}
Asen Hikov, and Laura Murphy (2024). \textit{Information retrieval from textual data: Harnessing large language models, retrieval augmented generation and prompt engineering}. Journal of AI, Robotics &amp; Workplace Automation.

\bibitem{tayebi20245il}
Soroosh Tayebi, A. Aripoli, P. Iglar, et al. (2024). \textit{RadioRAG: Factual Large Language Models for Enhanced Diagnostics in Radiology Using Dynamic Retrieval Augmented Generation}. arXiv.org.

\bibitem{duc2024hrn}
Nguyen Quang Duc, Le Hai Son, Nguyen Duc Nhan, et al. (2024). \textit{Towards Comprehensive Vietnamese Retrieval-Augmented Generation and Large Language Models}. arXiv.org.

\bibitem{debellis2024bv0}
Michael DeBellis, Nivedita Dutta, Jacob Gino, et al. (2024). \textit{Integrating Ontologies and Large Language Models to Implement Retrieval Augmented Generation}. Appl. Ontology.

\bibitem{pelletier20240l7}
A. Pelletier, Joseph Ramirez, Irsyad Adam, et al. (2024). \textit{Explainable Biomedical Hypothesis Generation via Retrieval Augmented Generation enabled Large Language Models}. arXiv.org.

\bibitem{lin20240ku}
Xinyi Lin, Gelei Deng, Yuekang Li, et al. (2024). \textit{GeneRAG: Enhancing Large Language Models with Gene-Related Task by Retrieval-Augmented Generation}. bioRxiv.

\bibitem{weinert2025cxo}
Dane A Weinert, and A. Rauschecker (2025). \textit{Enhancing Large Language Models with Retrieval-augmented Generation: A Radiology-specific Approach.}. Radiology: Artificial Intelligence.

\bibitem{liu2025rz6}
Siru Liu, A. Wright, Allison B. McCoy, et al. (2025). \textit{Detecting emergencies in patient portal messages using large language models and knowledge graph-based retrieval-augmented generation}. J. Am. Medical Informatics Assoc..

\bibitem{liu2025sy0}
Shuliang Liu, Xinze Li, Zhenghao Liu, et al. (2025). \textit{Judge as A Judge: Improving the Evaluation of Retrieval-Augmented Generation through the Judge-Consistency of Large Language Models}. Annual Meeting of the Association for Computational Linguistics.

\bibitem{nguyen202435q}
Quang Nguyen, Duy-Anh Nguyen, Khang Dang, et al. (2024). \textit{Advancing Question-Answering in Ophthalmology with Retrieval Augmented Generations (RAG): Benchmarking Open-source and Proprietary Large Language Models}. medRxiv.

\bibitem{rehulka2024p05}
Erik Rehulka, and Marek Suppa (2024). \textit{RAG Meets Detox: Enhancing Text Detoxification Using Open Large Language Models with Retrieval Augmented Generation}. Conference and Labs of the Evaluation Forum.

\bibitem{huang202465n}
Jie Huang, Mo Wang, Yunpeng Cui, et al. (2024). \textit{Layered Query Retrieval: An Adaptive Framework for Retrieval-Augmented Generation in Complex Question Answering for Large Language Models}. Applied Sciences.

\bibitem{hammane2024hdb}
Zakaria Hammane, Fatima-Ezzahraa Ben-Bouazza, and A. Fennan (2024). \textit{SelfRewardRAG: Enhancing Medical Reasoning with Retrieval-Augmented Generation and Self-Evaluation in Large Language Models}. International Symposium on Computer Vision.

\bibitem{samarajeewa20241p6}
Chamod Samarajeewa, Daswin De Silva, Evgeny Osipov, et al. (2024). \textit{Causal Reasoning in Large Language Models using Causal Graph Retrieval Augmented Generation}. International Conference on Human System Interaction.

\bibitem{hou2024gz7}
Yu Hou, J. R. Bishop, Hongfang Liu, et al. (2024). \textit{Improving Dietary Supplement Information Retrieval: Development of a Retrieval-Augmented Generation System With Large Language Models}. Journal of Medical Internet Research.

\bibitem{habib2024iqj}
Mohammad Affan Habib, Shehryar Amin, Muhammad Oqba, et al. (2024). \textit{TaxTajweez: A Large Language Model-based Chatbot for Income Tax Information In Pakistan Using Retrieval Augmented Generation (RAG)}. The Florida AI Research Society.

\end{thebibliography}

\end{document}