\subsection*{Mitigating Bias and Ensuring Fairness}

The integration of Large Language Models (LLMs) with external knowledge bases in Retrieval-Augmented Generation (RAG) systems, while enhancing factual accuracy, introduces a critical challenge: the potential for existing biases to be amplified or inadvertently introduced during the retrieval and generation processes. These biases can stem from historical data reflecting societal inequalities within the external knowledge base or be inherited from the LLM's vast training data, leading to the generation of unfair, discriminatory, or harmful outputs \cite{zhou20248fu, ni2025ox9}. Addressing this necessitates robust strategies for detection, mitigation, and prevention to ensure equitable and responsible deployment of RAG technologies, forming a crucial dimension of overall RAG trustworthiness \cite{ni2025ox9}.

The problem of bias in RAG systems is multifaceted, impacting both the selection of information and its subsequent synthesis. As comprehensive surveys by \cite{zhou20248fu} and \cite{ni2025ox9} highlight, fairness is a core pillar of trustworthiness in RAG, alongside factuality, robustness, privacy, and accountability. Bias can manifest when the retriever component disproportionately selects documents reflecting certain demographics, viewpoints, or stereotypes present in the knowledge base. This "retrieval bias" can then be amplified by the LLM during generation, even if the LLM itself has undergone some debiasing during pre-training. Conversely, an LLM's inherent biases can lead it to misinterpret or selectively use retrieved information in a biased manner, even from a diverse set of documents. Critically, \cite{zhang2025byv} demonstrates that RAG systems are not inherently safer than standalone LLMs; in fact, they can sometimes make models *less safe* and alter their safety profile, even when combining "safe" models with "safe" documents. This finding underscores the complex interaction between retrieval and generation, where emergent biases or safety failures can arise from the RAG pipeline itself, necessitating specific RAG-tailored safety and fairness interventions.

Initial efforts to address bias often focused on the generative capabilities of LLMs. Techniques such as prompt engineering, where carefully crafted prompts guide the LLM towards less biased outputs, and re-ranking of generated responses based on pre-computed fairness scores, were foundational. However, these methods primarily addressed the output of the LLM in isolation. With RAG, the problem space expanded to encompass the retrieval phase, demanding more integrated solutions.

To mitigate bias propagation within the retrieval component, researchers have explored "fairness-aware retriever" mechanisms. The goal is to diversify retrieved information, preventing the over-representation of biased perspectives. This can involve incorporating demographic parity constraints during document selection or re-ranking retrieved documents based on fairness scores associated with their source material or content. For instance, approaches might penalize documents from sources known to exhibit bias or prioritize documents that offer diverse viewpoints on a given topic. While effective in mitigating bias at the retrieval stage by ensuring a more balanced input context, the subsequent generation process could still inadvertently introduce or amplify subtle biases, particularly if the LLM's internal representations are biased or if it struggles with synthesizing conflicting fair information.

Addressing biases that persist or emerge during the generation phase of RAG has led to more sophisticated post-hoc mitigation strategies. These include counterfactual generation approaches, where biased outputs are identified and then re-generated using modified prompts or retrieved contexts to produce more neutral or diverse responses. This often involves identifying sensitive attributes in the output and attempting to generate alternatives that vary these attributes while maintaining factual consistency. Such methods highlight the complex trade-off between factual accuracy and fairness, as aggressive debiasing might inadvertently alter the factual content.

The comprehensive evaluation of fairness in RAG systems is paramount. Traditional single-axis fairness metrics, which often focus on a single protected attribute (e.g., gender or race), frequently fail to capture the complexities of real-world discrimination. Research has moved towards frameworks for evaluating intersectional biases—such as those combining race and gender—in RAG systems. This involves developing new evaluation datasets and metrics that assess fairness across multiple protected attributes simultaneously, providing a more nuanced understanding of how bias propagates from retrieval to generation. The trustworthiness frameworks proposed by \cite{zhou20248fu} and \cite{ni2025ox9} explicitly include fairness as a key dimension, advocating for comprehensive benchmarks that can diagnose specific RAG fairness capabilities and limitations across diverse scenarios.

Ultimately, a proactive approach to bias prevention at the source is critical. This involves auditing and curating the external knowledge base itself to identify and rectify biased, under-represented, or harmful information before it is even retrieved by the RAG system. Methods include identifying data gaps, suggesting diverse data augmentation strategies for the knowledge base, and implementing rigorous data governance policies. This represents a crucial shift towards preventing bias from entering the system in the first place, complementing reactive and retrieval-time mitigation efforts. However, this pre-processing can be resource-intensive and requires careful definition of "bias" within specific contexts.

In conclusion, the literature demonstrates a clear progression from detecting and mitigating bias in standalone LLM generation to addressing its propagation through RAG's retrieval and generation phases. While significant strides have been made in developing fairness-aware retrieval mechanisms, post-hoc generation corrections, and sophisticated intersectional evaluation metrics, several challenges persist. The dynamic nature of bias, the inherent trade-off between fairness and other performance metrics (like relevance or utility), and the critical finding that RAG can introduce new safety risks \cite{zhang2025byv} all underscore the need for continuous monitoring and adaptive strategies in real-world deployments. Future research must focus on developing holistic, integrated frameworks that combine proactive knowledge base curation with robust, context-aware detection and mitigation strategies throughout the entire RAG pipeline to foster truly trustworthy and inclusive AI systems.