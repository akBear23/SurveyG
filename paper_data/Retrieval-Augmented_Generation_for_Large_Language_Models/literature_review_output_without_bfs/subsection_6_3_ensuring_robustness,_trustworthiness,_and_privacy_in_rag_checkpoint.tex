\subsection{Ensuring Robustness, Trustworthiness, and Privacy in RAG}
While Retrieval-Augmented Generation (RAG) systems significantly enhance Large Language Models (LLMs) by grounding their responses in external knowledge, their real-world deployment necessitates rigorous attention to non-functional requirements such as robustness, trustworthiness, and privacy. These concerns are paramount, as RAG's reliance on external data introduces new vulnerabilities and amplifies existing challenges related to data quality, security, and ethical implications \cite{zhou20248fu}.

A primary concern for RAG systems is their resilience to irrelevant or noisy retrieved information. Early benchmarking efforts, such as RGB \cite{chen2023nzb}, systematically diagnosed fundamental weaknesses of RAG, revealing LLMs' struggles with noise robustness, negative rejection (failing to decline answers when no information is present), and information integration from multiple documents. To address these, corrective strategies have emerged, such as Corrective RAG (CRAG) \cite{yan202437z}, which dynamically assesses retrieval quality using a lightweight evaluator and triggers actions like knowledge refinement or web searches to mitigate the impact of poor initial retrieval. Complementing this, proactive approaches like RQ-RAG \cite{chan2024u69} train LLMs to refine queries through rewriting, decomposition, or disambiguation, thereby improving retrieval effectiveness from the outset. RankRAG \cite{yu202480d} further enhances robustness by unifying context ranking and answer generation within a single instruction-tuned LLM, simplifying the pipeline and improving performance.

Beyond architectural refinements, more sophisticated defense mechanisms against diverse noise types are crucial. Retrieval-augmented Adaptive Adversarial Training (RAAT) \cite{fang2024gh6} systematically categorizes retrieval noises (relevant, irrelevant, counterfactual) and employs an adaptive adversarial training mechanism to dynamically adjust the model's training, enhancing its internal capacity to discern and resist noisy contexts. Similarly, Astute RAG \cite{wang2024kca} tackles imperfect retrieval and knowledge conflicts by adaptively eliciting essential information from LLMs' internal knowledge and iteratively consolidating internal and external sources based on reliability. Efficiency and quality are also improved by Sparse RAG \cite{zhu2024h7i}, which selectively decodes by attending only to highly relevant caches, filtering undesirable contexts, and Speculative RAG \cite{wang20246hs}, which uses a smaller drafter LLM to generate multiple answer drafts in parallel, then efficiently verifies them with a larger LLM. In domain-specific applications, such as medicine, the MIRAGE benchmark \cite{xiong2024exb} highlights RAG's "lost-in-the-middle" phenomenon, while i-MedRAG \cite{xiong2024u1b} improves robustness for complex medical QA by enabling iterative follow-up questions and multi-step reasoning. The quality of the underlying data is also paramount; for instance, enhanced PDF structure recognition by ChatDOC \cite{lin2024s1v} significantly boosts RAG performance by ensuring accurate extraction from complex professional documents. Evaluating the utility of retrieved documents directly from the LLM's perspective, as proposed by eRAG \cite{salemi2024om5}, offers a more reliable metric for optimizing retrieval quality and, by extension, system robustness.

Privacy is another critical dimension of trustworthiness in RAG systems \cite{zhou20248fu}. A systematic exploration by \cite{zeng2024dzl} revealed significant vulnerabilities of RAG systems to data leakage from their external retrieval databases through composite structured prompting attacks. This work demonstrated that attackers can craft specific prompts to direct the retriever to fetch sensitive data and instruct the LLM to output it. Counter-intuitively, \cite{zeng2024dzl} also found that RAG can mitigate the leakage of the LLM's *own* training data, suggesting a complex interplay of privacy risks.

Beyond privacy, RAG systems are susceptible to a range of adversarial attacks and broader trustworthiness concerns. RAG poisoning, where malicious content is injected into the retrieval database, poses a severe threat. Pandora \cite{deng2024k1b} demonstrated how to jailbreak GPTs by poisoning RAG knowledge sources with policy-violating content and then triggering its retrieval through tailored prompts. Extending this, BadRAG \cite{xue2024bxd} introduced methods for crafting adversarial passages that are selectively retrieved by semantic triggers, enabling indirect generative attacks like sentiment steering or denial-of-service (DoS) by exploiting LLM alignment features. Opinion manipulation attacks \cite{chen20247nc} further show how RAG's ranking results can be manipulated to alter the opinion polarity of generated content, potentially misleading users. Even subtle prompt perturbations can significantly affect RAG outputs, leading to factually incorrect answers \cite{hu2024i6h}. To systematically evaluate these security vulnerabilities, SafeRAG \cite{liang2025f4q} provides a benchmark for assessing RAG security against various attack tasks, including silver noise, inter-context conflict, soft ads, and white DoS. Addressing bias in retrieved information is also crucial for trustworthiness; RAG$^2$ \cite{sohn2024w2t} mitigates retriever bias by retrieving snippets evenly from a comprehensive set of corpora. Furthermore, enhancing explainability, as seen in G-Retriever's \cite{he20248lp} subgraph retrieval for textual graphs, contributes to overall trustworthiness by providing transparent reasoning paths.

In conclusion, ensuring robustness, trustworthiness, and privacy in RAG systems is a multifaceted challenge. While significant progress has been made in developing advanced architectures, adaptive defense mechanisms against noise and adversarial attacks, and identifying critical privacy vulnerabilities, the dynamic nature of RAG and the continuous evolution of LLMs necessitate ongoing research. Future work must focus on developing more resilient RAG pipelines, robust content moderation for external knowledge bases, and comprehensive evaluation frameworks that encompass the full spectrum of trustworthiness dimensions for responsible RAG deployment in sensitive, real-world applications.