\subsection{The Promise and Limitations of Large Language Models}

Large Language Models (LLMs) have ushered in a transformative era in artificial intelligence, demonstrating state-of-the-art performance across a diverse spectrum of natural language processing tasks, from sophisticated content generation and summarization to complex reasoning and question answering \cite{brown2020language}. Their remarkable capabilities stem from training on colossal datasets, enabling them to capture intricate linguistic patterns and generate highly coherent and contextually relevant text. However, despite these advancements, LLMs inherently grapple with significant limitations that curtail their reliability and applicability, particularly in dynamic and knowledge-intensive domains. These core challenges include the 'knowledge cutoff' problem, the pervasive tendency for 'hallucination,' and a fundamental lack of inherent attributability or verifiability for their generated outputs. These shortcomings collectively underscore an urgent need for robust mechanisms to provide LLMs with up-to-date, accurate, and verifiable knowledge, thereby unlocking their full potential while mitigating substantial risks.

The 'knowledge cutoff' represents a fundamental temporal constraint inherent to the LLM pre-training paradigm. Models are trained on vast datasets collected up to a specific point in time, rendering them incapable of accessing or generating information about events, discoveries, or developments that occurred subsequently \cite{brown2020language, procko202417i}. This means that LLMs, despite their impressive linguistic fluency, can quickly become outdated in rapidly evolving fields such as technology, current events, or scientific research. For instance, an LLM trained in 2022 would be unable to provide accurate details about events from 2023 or 2024 without external intervention. Furthermore, their training data, while extensive, is necessarily finite and cannot encompass all domain-specific, proprietary, or niche knowledge critical for specialized applications, such as intricate medical guidelines \cite{chen2025tux} or specific telecommunication standards \cite{yilma20249sl}. This inability to access real-time or highly specific external information severely restricts their utility in dynamic and knowledge-intensive environments where factual accuracy and currency are paramount \cite{wang2024ac6}.

Complementing the knowledge cutoff, the phenomenon of 'hallucination' poses a more profound and insidious limitation. Hallucination refers to the generation of content that is factually incorrect, nonsensical, or inconsistent with the provided context or real-world knowledge, despite being presented with high confidence \cite{ji2023survey}. This tendency stems from the LLMs' probabilistic nature, where they excel at predicting the most plausible sequence of tokens based on learned statistical patterns, rather than possessing a true understanding of factual truth or logical consistency \cite{gudibande2023fantastically}. Mechanistic interpretability studies, such as those by \cite{sun2024eoe}, suggest that hallucinations can occur even in Retrieval-Augmented Generation (RAG) scenarios when the LLM's internal "Knowledge FFNs" overemphasize parametric knowledge in the residual stream, while "Copying Heads" fail to effectively integrate external knowledge from retrieved content. This indicates that the issue is not merely a lack of information but also a challenge in how LLMs process and prioritize information sources. Hallucinations can manifest in various forms, from fabricating non-existent facts, dates, or names to generating logically flawed arguments or citing non-existent sources \cite{gudibande2023fantastically}. Even when augmented with retrieval, LLMs can struggle with noise robustness, negative rejection (failing to decline answering when no information is present), information integration from multiple documents, and counterfactual robustness (prioritizing incorrect retrieved information over internal knowledge), leading to persistent factual errors \cite{chen2023nzb}. In high-stakes domains like healthcare, legal, or finance, the generation of such misleading information can have severe consequences, eroding user trust and posing significant risks to trustworthiness \cite{zhou20248fu, kang2024hrb}.

Moreover, a fundamental architectural challenge with purely parametric LLMs is the lack of inherent attributability, verifiability, or explainability for their generated content. When an LLM produces a response, it is often difficult, if not impossible, to trace the specific pieces of information or sources that contributed to its output \cite{sudhi20240uy}. This opaque nature, often referred to as the 'black box' problem in AI explainability (XAI), makes it challenging for users to ascertain the veracity of the generated text, especially when hallucinations occur. Without explicit provenance, the credibility of LLM-generated content is inherently compromised, hindering its adoption in applications requiring strict factual accuracy and accountability, such as verifying claims in legal documents or medical diagnoses \cite{yilma20249sl}. The internal "knowledge" of LLMs is implicitly encoded within billions of parameters, making it inaccessible for transparent validation and leading to a lack of transparency and accountability \cite{zhou20248fu}. This limitation is particularly pronounced when LLMs are tasked with complex reasoning or synthesizing information, as their internal "knowledge" is implicitly encoded within billions of parameters, making it inaccessible for transparent validation.

In summary, while LLMs offer unprecedented capabilities in language tasks, their reliance on static training data leads to inherent limitations in knowledge currency, factual accuracy, and verifiability. The 'knowledge cutoff' renders them perpetually behind the curve in dynamic domains, while 'hallucination' undermines their trustworthiness by generating confident but incorrect information, often due to complex internal processing issues. The absence of clear attributability further compounds these issues, making it difficult to trust and verify LLM outputs. These inherent architectural limitations underscore the urgent need for a paradigm that can dynamically ground LLMs in external, verifiable knowledge—a role that Retrieval-Augmented Generation aims to fill, as will be conceptualized in the following section.