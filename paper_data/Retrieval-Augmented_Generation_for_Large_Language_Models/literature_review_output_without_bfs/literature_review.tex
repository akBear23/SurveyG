\documentclass[12pt,a4paper]{article}
    \usepackage[utf8]{inputenc}
    \usepackage[T1]{fontenc}
    \usepackage{amsmath,amsfonts,amssymb}
    \usepackage{graphicx}
    \usepackage[margin=2.5cm]{geometry}
    \usepackage{setspace}
    \usepackage{natbib}
    \usepackage{url}
    \usepackage{hyperref}
    \usepackage{booktabs}
    \usepackage{longtable}
    \usepackage{array}
    \usepackage{multirow}
    \usepackage{wrapfig}
    \usepackage{float}
    \usepackage{colortbl}
    \usepackage{pdflscape}
    \usepackage{tabu}
    \usepackage{threeparttable}
    \usepackage{threeparttablex}
    \usepackage[normalem]{ulem}
    \usepackage{makecell}
    \usepackage{xcolor}

    % Set line spacing
    \doublespacing

    % Configure hyperref
    \hypersetup{
        colorlinks=true,
        linkcolor=blue,
        filecolor=magenta,      
        urlcolor=cyan,
        citecolor=red,
    }

    % Title and author information
    \title{A Comprehensive Literature Review with Self-Reflection}
    \author{Literature Review}
    \date{\today}

    \begin{document}

    \maketitle

    % Abstract (optional)
    \begin{abstract}
    This literature review provides a comprehensive analysis of recent research in the field. The review synthesizes findings from 211 research papers, identifying key themes, methodological approaches, and future research directions.
    \end{abstract}

    \newpage
    \tableofcontents
    \newpage

    \label{sec:introduction_to_retrieval-augmented_generation}

\section{Introduction to Retrieval-Augmented Generation}
\label{sec:introduction\_to\_retrieval-augmented\_generation}

\subsection{The Promise and Limitations of Large Language Models}
\label{sec:1\_1\_the\_promise\_\_and\_\_limitations\_of\_large\_language\_models}

Large Language Models (LLMs) have ushered in a transformative era in artificial intelligence, demonstrating state-of-the-art performance across a diverse spectrum of natural language processing tasks, from sophisticated content generation and summarization to complex reasoning and question answering \cite{brown2020language}. Their remarkable capabilities stem from training on colossal datasets, enabling them to capture intricate linguistic patterns and generate highly coherent and contextually relevant text. However, despite these advancements, LLMs inherently grapple with significant limitations that curtail their reliability and applicability, particularly in dynamic and knowledge-intensive domains. These core challenges include the 'knowledge cutoff' problem, the pervasive tendency for 'hallucination,' and a fundamental lack of inherent attributability or verifiability for their generated outputs. These shortcomings collectively underscore an urgent need for robust mechanisms to provide LLMs with up-to-date, accurate, and verifiable knowledge, thereby unlocking their full potential while mitigating substantial risks.

The 'knowledge cutoff' represents a fundamental temporal constraint inherent to the LLM pre-training paradigm. Models are trained on vast datasets collected up to a specific point in time, rendering them incapable of accessing or generating information about events, discoveries, or developments that occurred subsequently \cite{brown2020language, procko202417i}. This means that LLMs, despite their impressive linguistic fluency, can quickly become outdated in rapidly evolving fields such as technology, current events, or scientific research. For instance, an LLM trained in 2022 would be unable to provide accurate details about events from 2023 or 2024 without external intervention. Furthermore, their training data, while extensive, is necessarily finite and cannot encompass all domain-specific, proprietary, or niche knowledge critical for specialized applications, such as intricate medical guidelines \cite{chen2025tux} or specific telecommunication standards \cite{yilma20249sl}. This inability to access real-time or highly specific external information severely restricts their utility in dynamic and knowledge-intensive environments where factual accuracy and currency are paramount \cite{wang2024ac6}.

Complementing the knowledge cutoff, the phenomenon of 'hallucination' poses a more profound and insidious limitation. Hallucination refers to the generation of content that is factually incorrect, nonsensical, or inconsistent with the provided context or real-world knowledge, despite being presented with high confidence \cite{ji2023survey}. This tendency stems from the LLMs' probabilistic nature, where they excel at predicting the most plausible sequence of tokens based on learned statistical patterns, rather than possessing a true understanding of factual truth or logical consistency \cite{gudibande2023fantastically}. Mechanistic interpretability studies, such as those by \cite{sun2024eoe}, suggest that hallucinations can occur even in Retrieval-Augmented Generation (RAG) scenarios when the LLM's internal "Knowledge FFNs" overemphasize parametric knowledge in the residual stream, while "Copying Heads" fail to effectively integrate external knowledge from retrieved content. This indicates that the issue is not merely a lack of information but also a challenge in how LLMs process and prioritize information sources. Hallucinations can manifest in various forms, from fabricating non-existent facts, dates, or names to generating logically flawed arguments or citing non-existent sources \cite{gudibande2023fantastically}. Even when augmented with retrieval, LLMs can struggle with noise robustness, negative rejection (failing to decline answering when no information is present), information integration from multiple documents, and counterfactual robustness (prioritizing incorrect retrieved information over internal knowledge), leading to persistent factual errors \cite{chen2023nzb}. In high-stakes domains like healthcare, legal, or finance, the generation of such misleading information can have severe consequences, eroding user trust and posing significant risks to trustworthiness \cite{zhou20248fu, kang2024hrb}.

Moreover, a fundamental architectural challenge with purely parametric LLMs is the lack of inherent attributability, verifiability, or explainability for their generated content. When an LLM produces a response, it is often difficult, if not impossible, to trace the specific pieces of information or sources that contributed to its output \cite{sudhi20240uy}. This opaque nature, often referred to as the 'black box' problem in AI explainability (XAI), makes it challenging for users to ascertain the veracity of the generated text, especially when hallucinations occur. Without explicit provenance, the credibility of LLM-generated content is inherently compromised, hindering its adoption in applications requiring strict factual accuracy and accountability, such as verifying claims in legal documents or medical diagnoses \cite{yilma20249sl}. The internal "knowledge" of LLMs is implicitly encoded within billions of parameters, making it inaccessible for transparent validation and leading to a lack of transparency and accountability \cite{zhou20248fu}. This limitation is particularly pronounced when LLMs are tasked with complex reasoning or synthesizing information, as their internal "knowledge" is implicitly encoded within billions of parameters, making it inaccessible for transparent validation.

In summary, while LLMs offer unprecedented capabilities in language tasks, their reliance on static training data leads to inherent limitations in knowledge currency, factual accuracy, and verifiability. The 'knowledge cutoff' renders them perpetually behind the curve in dynamic domains, while 'hallucination' undermines their trustworthiness by generating confident but incorrect information, often due to complex internal processing issues. The absence of clear attributability further compounds these issues, making it difficult to trust and verify LLM outputs. These inherent architectural limitations underscore the urgent need for a paradigm that can dynamically ground LLMs in external, verifiable knowledge—a role that Retrieval-Augmented Generation aims to fill, as will be conceptualized in the following section.
\subsection{Conceptualizing Retrieval-Augmented Generation}
\label{sec:1\_2\_conceptualizing\_retrieval-augmented\_generation}

Large Language Models (LLMs), while demonstrating remarkable proficiency in natural language understanding and generation, are inherently constrained by several critical limitations, as discussed in Section 1.1. These include a susceptibility to factual inconsistencies, the tendency to "hallucinate" information not present in their training data, and a fundamental inability to provide explicit provenance for their outputs \cite{lewis2020pwr}. Such issues primarily arise from their reliance on parametric knowledge, which is static, bounded by a "knowledge cutoff" from their last training update, and lacks the dynamic capacity to access and integrate up-to-date external information. Retrieval-Augmented Generation (RAG) emerges as a transformative paradigm specifically designed to address these challenges by integrating external knowledge retrieval directly into the generation process, thereby augmenting the LLM's static parametric knowledge with a dynamic, non-parametric memory.

At its core, RAG introduces a fundamental shift from a purely "closed-book" generation model to an "open-book" approach. Conceptually, a RAG system comprises two primary interacting components: a \textit{retriever} and a \textit{generator}. The retriever's role is to efficiently query a vast, often dynamic, external knowledge base (e.g., a corpus of documents, a database, or the web) to identify and extract passages or documents most relevant to a given input query. This process leverages sophisticated indexing and semantic search techniques to ensure that the retrieved context is pertinent and informative. The generator, typically a pre-trained LLM, then takes both the original user query and the retrieved context as input. It synthesizes this combined information to formulate a coherent, contextually rich, and factually grounded response. This mechanism allows the LLM to dynamically access and synthesize information from sources far exceeding its internal parametric memory, providing a powerful means to overcome its inherent knowledge limitations.

The core value proposition of RAG lies in its ability to significantly enhance the reliability and trustworthiness of LLM outputs. By grounding responses in verifiable external sources, RAG intrinsically improves factual consistency and substantially reduces the incidence of hallucinations. When an LLM is provided with explicit, relevant information, its propensity to invent facts diminishes, as it can directly reference and reformulate the provided context. Furthermore, RAG inherently facilitates explicit provenance; the generated content can often be traced back to the specific retrieved documents, offering transparency and allowing users to verify the information's source. This capability is crucial for applications requiring high degrees of accuracy and accountability, such as in scientific, medical, or legal domains. The dynamic nature of the external knowledge base also means that RAG systems can be continuously updated with new information without requiring expensive and time-consuming retraining of the entire LLM, effectively addressing the knowledge cutoff problem.

The seminal work by \textcite{lewis2020pwr} formally introduced the RAG paradigm, demonstrating its efficacy by combining a pre-trained sequence-to-sequence generator with a dense vector index of Wikipedia. While the architectural specifics and training methodologies of such foundational models are detailed in subsequent sections (e.g., Section 2.1), the conceptual breakthrough was the realization that jointly leveraging a model's learned generative capabilities with a dynamic, searchable external memory could lead to superior performance on knowledge-intensive tasks. This established the essential groundwork for understanding RAG's fundamental mechanism of augmenting parametric knowledge with non-parametric memory, laying the foundation for a new era of more informed, factual, and attributable language generation. This conceptual framework underpins the diverse architectural advancements and strategic optimizations that have since characterized the rapid evolution of RAG systems, which will be explored in the subsequent sections of this review.


\label{sec:foundational_rag_architectures_and_end-to-end_integration}

\section{Foundational RAG Architectures and End-to-End Integration}
\label{sec:foundational\_rag\_architectures\_\_and\_\_end-to-end\_integration}

\subsection{Pioneering End-to-End RAG Models}
\label{sec:2\_1\_pioneering\_end-to-end\_rag\_models}

Prior to the advent of end-to-end Retrieval-Augmented Generation (RAG), systems addressing knowledge-intensive tasks typically relied on multi-stage "retrieve-then-read" pipelines, exemplified by models like DrQA \cite{chen2017reading}. In these architectures, a retriever first fetched documents, and a separate reader model then processed these retrieved documents to generate an answer. Critically, the retriever and reader were trained independently, often leading to a suboptimal coupling where the retriever was not explicitly optimized for the downstream generation task. This paradigm shifted significantly with the introduction of foundational RAG models, which pioneered the concept of jointly training a retriever and a generator in an end-to-end fashion. This marked a profound departure from purely parametric Large Language Models (LLMs) by integrating a non-parametric memory component, enabling models to learn to augment their generation capabilities with external, up-to-date knowledge directly.

A seminal contribution was the Retrieval-Augmented Generation (RAG) framework introduced by \cite{lewis2020pwr}. This work proposed combining a pre-trained sequence-to-sequence generator, such as BART, with a dense passage retriever (DPR) that queries a large non-parametric knowledge base like Wikipedia. The core innovation lay in its end-to-end fine-tuning approach, where the entire model was optimized to maximize the likelihood of the target sequence. This was achieved by marginalizing over the latent retrieved documents: for each target token, the model considers all possible retrieved passages, computes the probability of generating the token given each passage, and sums these probabilities, effectively allowing gradients to flow back to the retriever. The RAG framework further explored two distinct architectural variants: RAG-Sequence, which retrieves a single document for the entire generated sequence, and RAG-Token, which retrieves a new document at each token generation step. This nuanced design space highlighted the flexibility in integrating external knowledge, leading to more factual, specific, and diverse language generation.

Concurrently, the Retrieval-Augmented Language Model (REALM) proposed by \cite{guu2020realm} explored a similar concept, focusing on pre-training a masked language model to retrieve and incorporate knowledge. REALM integrated a differentiable retriever directly into the pre-training objective, enabling the model to learn to fetch relevant passages to complete masked tokens. The differentiability of the retriever was a key technical achievement, allowing gradients from the masked language model loss to propagate back to the retriever's parameters. This was typically achieved by treating document selection as a latent variable and employing an approximate gradient estimator (e.g., using a Gumbel-Softmax approximation or a straight-through estimator) to enable end-to-end optimization. Both RAG and REALM showcased substantial improvements on knowledge-intensive NLP tasks, effectively mitigating the hallucination problem inherent in purely parametric models by grounding responses in verifiable external knowledge.

While these initial models established the core RAG paradigm, they also highlighted certain challenges. The computational expense and complexity associated with end-to-end training, particularly for very large LLMs or massive knowledge bases, could be prohibitive. Furthermore, their performance was inherently tied to the quality of the retrieved documents, lacking explicit mechanisms to evaluate or filter out irrelevant or noisy passages during the generation process itself. Building upon these foundations, \cite{izacard2022atlas} extended the RAG framework to few-shot learning scenarios with Atlas. This model unified retrieval and generation within a single T5-based architecture, demonstrating how end-to-end trained RAG models could significantly boost performance in data-scarce settings by effectively leveraging external knowledge. Atlas showcased the versatility and immediate impact of the initial RAG framework, proving its applicability beyond standard knowledge-intensive tasks and into more challenging few-shot learning environments.

In conclusion, these pioneering works laid the groundwork for modern RAG systems by demonstrating the profound benefits of tightly integrating retrieval with generation. They established the core methodologies for learning to retrieve relevant passages and condition language generation on them, effectively overcoming the knowledge cutoff of parametric LLMs and providing a mechanism for factual grounding. However, the computational demands of end-to-end training and the implicit assumption of high-quality retrieval remained significant challenges. These limitations spurred subsequent research into more efficient training strategies, robust retrieval mechanisms, and adaptive RAG systems. The effectiveness of these integrated models was critically dependent on the two main architectural pillars: the retriever's ability to find relevant information and the generator's capacity to synthesize it. The next section will therefore deconstruct these systems to examine the core components in greater detail, laying the groundwork for understanding subsequent innovations in retrieval and generation strategies.
\subsection{Core Components: Retriever and Generator}
\label{sec:2\_2\_core\_components:\_retriever\_\_and\_\_generator}

The efficacy of Retrieval-Augmented Generation (RAG) systems hinges on the synergistic operation of two fundamental components: a retriever and a generator. These essential building blocks enable Large Language Models (LLMs) to transcend the limitations of their parametric knowledge by dynamically accessing and synthesizing information from vast, external knowledge bases. This architecture, exemplified by pioneering works, laid the crucial groundwork for subsequent innovations in the field by enhancing factual accuracy and mitigating hallucinations.

The first core component is the \textit{retriever}, whose primary function is to efficiently fetch relevant documents or passages from a large corpus in response to an input query. Prior to the advent of dense retrieval, traditional information retrieval systems predominantly relied on \textit{sparse retrieval} methods, such as BM25 or TF-IDF. These methods operate on lexical matching, identifying documents that share keywords with the query. While effective for exact-match queries, sparse retrievers often struggle with semantic understanding, failing to retrieve passages that are conceptually relevant but lexically dissimilar. To overcome this limitation, early RAG systems prominently adopted \textit{dense retrievers}, with the Dense Passage Retriever (DPR) \cite{lewis2020pwr} serving as a seminal example. DPR employs a dual-encoder (or bi-encoder) architecture, independently mapping input queries and passages into a shared, dense vector space using neural networks (e.g., BERT-based encoders).

The training of dense retrievers like DPR is typically achieved through contrastive learning. The model is optimized to embed semantically similar query-passage pairs (positive examples) close together in the vector space, while simultaneously pushing dissimilar pairs (negative examples) far apart. A common strategy involves using in-batch negatives, where other passages within the same training batch serve as negative examples for a given query. More advanced techniques, such as hard negative mining, further refine this process by selecting particularly challenging negative examples that are semantically close but factually irrelevant. For instance, \cite{glass2021qte} introduced Dense Negative Sampling (DNS), which mines hard negatives from the learned dense vector index itself, rather than relying solely on lexical methods like BM25. This iterative process of identifying and learning from hard negatives is crucial for creating a highly discriminative embedding space, significantly boosting retrieval performance by ensuring that only the most relevant passages are retrieved. During inference, the retriever encodes the input query into its vector representation, which is then used to perform an efficient maximum inner product search (MIPS) against a pre-indexed database of passage embeddings (often facilitated by libraries like FAISS), enabling rapid identification of the top-k relevant textual contexts.

Following the retrieval of relevant passages, the \textit{generator} component assumes the responsibility of synthesizing information from the retrieved context and the input query to produce a coherent, factual, and attributable response. Early RAG systems commonly leveraged pre-trained sequence-to-sequence (seq2seq) models, such as BART or T5, for this generation task \cite{lewis2020pwr}. The retrieved passages are typically concatenated with the original input query and fed into the generator. The generator then utilizes its sophisticated attention mechanisms to condition the output sequence on both the query and the provided context, allowing it to integrate the external knowledge effectively. The seminal work by \cite{lewis2020pwr} demonstrated how both the retriever and the generator could be fine-tuned end-to-end, enabling the generator to learn to effectively leverage the retrieved context, discerning between relevant and irrelevant information to formulate its response. This joint training fostered a tight coupling between the retrieval and generation stages, optimizing them for downstream task performance.

This tight integration of a neural retriever and a sequence-to-sequence generator represented a significant architectural innovation, enabling language models to overcome the inherent knowledge cutoff of their training data and substantially improve performance on knowledge-intensive NLP tasks, as demonstrated by models like RAG \cite{lewis2020pwr}, REALM \cite{REALM}, and Atlas \cite{Atlas}. However, this foundational paradigm also presented several challenges that motivated subsequent research. The end-to-end training process was computationally expensive and complex, requiring careful balancing of the optimization objectives for two distinct neural modules. Furthermore, the performance of the generator was highly dependent on the quality of the initial retrieval; poor retrieval could lead to irrelevant or even misleading contexts, resulting in hallucinated or inaccurate generations. The memory requirements for indexing vast corpora and for effective hard negative mining during contrastive learning (as noted by \cite{glass2021qte} regarding FAISS indexing) also posed practical scalability challenges. These limitations, particularly the implicit reliance on high-quality retrieval and the complexities of joint optimization, became focal points for subsequent advancements in retrieval strategies, multi-stage processing, and system-level optimizations.


\label{sec:enhancing_retrieval_strategies_and_adaptability}

\section{Enhancing Retrieval Strategies and Adaptability}
\label{sec:enhancing\_retrieval\_strategies\_\_and\_\_adaptability}

\subsection{Query Refinement and Multi-Perspective Retrieval}
\label{sec:3\_1\_query\_refinement\_\_and\_\_multi-perspective\_retrieval}

The initial effectiveness of Retrieval-Augmented Generation (RAG) systems \cite{lewis2020pwr} is frequently constrained by the quality of the user's input query. Ambiguous, underspecified, or inherently complex questions, such as those requiring multi-hop reasoning, often lead to the retrieval of irrelevant or insufficient context, thereby degrading the quality of the LLM's generated response \cite{tang2024i5r, huang2024a59}. This "query-context mismatch" is a critical bottleneck, as the retriever's performance is directly tied to how well the query expresses the underlying information need \cite{wu2024bpc}. Addressing this fundamental challenge, a significant body of research, categorized under "pre-retrieval" techniques by surveys like \cite{huang2024a59}, has emerged focusing on enhancing the query itself prior to retrieval, ensuring more precise and comprehensive information acquisition.

\#\#\# Multi-Perspective Query Generation

One prominent strategy to overcome underspecified queries is to generate multiple query perspectives. This approach aims to broaden the search space and increase retrieval recall by exploring diverse semantic angles of the user's intent. A widely popularized technique in this category is RAG-Fusion \cite{rag-fusion-2023}. RAG-Fusion leverages the generative capabilities of LLMs to create several rephrased or expanded versions of the original query. For instance, a query like "What is RAG?" might be expanded into "Retrieval-Augmented Generation definition," "How does RAG work?", and "Benefits of RAG." These multiple perspectives are then used to perform parallel retrievals from the knowledge base. The results from these individual retrievals are subsequently combined, often using rank fusion algorithms such as Reciprocal Rank Fusion (RRF), to produce a consolidated, broader, and potentially more relevant set of documents. While effective in boosting recall by mitigating the risk of a single, poorly formulated query, this method inherently incurs higher computational costs due to multiple parallel retrieval calls and the overhead of rank fusion. Furthermore, the quality of the generated perspectives heavily depends on the LLM's ability to accurately infer the user's multifaceted intent.

Building on the idea of capturing diverse facets, RichRAG \cite{wang20245w8} introduces a "sub-aspect explorer" specifically designed for broad, open-ended queries that require comprehensive, long-form answers. Instead of simply rephrasing, the sub-aspect explorer identifies potential sub-intents or key aspects within a complex query. This allows a "multi-faceted retriever" to build a diverse candidate pool of documents, ensuring comprehensive coverage for generating rich responses. This approach moves beyond simple rephrasing by performing a more structured decomposition of the query's underlying information needs, thereby enhancing both the diversity and relevance of the retrieved context.

\#\#\# LLM-Driven Query Rewriting and Decomposition

While generating multiple perspectives improves recall, an alternative line of research focuses on enhancing the precision of a single query through direct, model-driven rewriting or decomposition. This approach aims to produce a more focused and effective query, potentially reducing the need for extensive post-retrieval processing and lowering the inference cost compared to parallel multi-query retrieval. RQ-RAG \cite{chan2024u69} exemplifies this, introducing an end-to-end framework that trains a Llama2 model to dynamically rewrite, decompose complex questions into sub-questions, or disambiguate ambiguous queries. The innovation lies in its dataset construction pipeline, which uses a larger LLM (e.g., ChatGPT) to craft tailored search queries and contextually aligned answers for various refinement scenarios. By employing control tokens, the model is guided to perform specific actions (rewrite, decompose, disambiguate, or terminate search) at each step. This proactive, learned refinement significantly improves retrieval effectiveness, particularly for challenging multi-hop Question Answering (QA) tasks, achieving state-of-the-art results with remarkable data efficiency. The primary advantage of RQ-RAG is its ability to produce a more precise query, but it relies on the LLM's learned ability to correctly refine the query, and an incorrect refinement could lead to error propagation and poor retrieval.

Similarly, DPA-RAG \cite{dong2024qcd} introduces five novel query augmentation strategies aimed at aligning the retriever with the LLM's diverse knowledge preferences. This is a crucial insight: different LLMs might respond better to contexts retrieved by slightly different query formulations. By augmenting queries in ways that cater to the LLM's internal knowledge representation and reasoning patterns, DPA-RAG enhances the reliability and factual consistency of RAG systems. This highlights a shift from generic query optimization to LLM-specific query conditioning, recognizing the interplay between query formulation and the LLM's internal processing.

\#\#\# Adaptive Query Strategies and Domain-Specific Refinement

Recognizing that not all queries are equally complex or require the same refinement strategy, researchers have also developed adaptive frameworks. Layered Query Retrieval (LQR) \cite{huang202465n} proposes an adaptive RAG framework that focuses on query complexity classification. It employs a semantic rule-based approach to distinguish between different levels of multi-hop queries. Based on this classification, LQR dynamically selects appropriate retrieval strategies, beginning with keyword-based retrieval and then using a Natural Language Inference (NLI) model to assess document relevance. This adaptive selection of strategies, guided by query complexity, aims to balance retrieval efficiency with accuracy, particularly for complex multi-hop questions. The core idea is to avoid over-engineering simple queries while applying sophisticated refinement only when necessary.

Query enhancement also extends to specialized domain-specific contexts. Telco-RAG \cite{bornea2024jde}, designed for the highly technical telecommunications domain, incorporates lexicon-enhanced queries using a custom glossary. It further refines queries with LLM-generated candidate answers to clarify intent, leveraging domain-specific knowledge to disambiguate technical jargon and improve precision. This demonstrates how general query refinement principles can be tailored and augmented with domain-specific resources to achieve superior performance in specialized applications.

\#\#\# Critical Analysis and Remaining Challenges

The evolution of RAG systems clearly demonstrates a critical shift from passive context augmentation to active, intelligent pre-retrieval query processing. Techniques ranging from multi-perspective generation and LLM-driven rewriting to adaptive, corrective, and domain-specific query refinement significantly enhance the precision and recall of retrieved information. Multi-perspective approaches like RAG-Fusion prioritize recall and robustness against initial query ambiguity but come with increased computational costs due to multiple retrieval calls and the overhead of rank fusion. In contrast, LLM-driven rewriting (e.g., RQ-RAG) aims for higher precision and efficiency by generating a single, optimized query, but risks error propagation if the LLM's refinement is flawed or misinterprets the user's intent. Adaptive strategies, such as LQR, attempt to strike a balance by dynamically applying refinement based on query complexity, but require robust classification mechanisms that can accurately assess query difficulty without adding significant latency.

Despite these advancements, challenges remain. Optimizing the real-time performance of multi-stage refinement processes, especially for latency-sensitive applications, is crucial. Ensuring the generalizability of learned refinement strategies across diverse domains and query types without extensive re-training is another hurdle. Furthermore, developing robust mechanisms for dynamically selecting the most appropriate refinement approach for any given query without incurring excessive latency or computational overhead remains an open research question \cite{huang2024a59, zhao2024931}. The trade-off between the complexity of the refinement mechanism and its actual benefit in terms of downstream generation quality and efficiency requires continuous empirical investigation \cite{wu2024bpc}. These ongoing challenges highlight the need for further research into more efficient, robust, and context-aware query refinement techniques that can seamlessly integrate into real-world RAG deployments.
\subsection{Adaptive and Dynamic Retrieval Mechanisms}
\label{sec:3\_2\_adaptive\_\_and\_\_dynamic\_retrieval\_mechanisms}

Moving beyond the paradigm of passive augmentation, a significant frontier in Retrieval-Augmented Generation (RAG) research empowers the Large Language Model (LLM) to actively control and adapt its retrieval process. This subsection examines architectures where the LLM dynamically decides \textit{when} to retrieve, \textit{what} information to seek, or \textit{how} to refine its queries based on its ongoing generation and real-time information needs. This shift from static, pre-defined retrieval to intelligent, context-aware information acquisition marks a crucial step towards more autonomous and effective RAG systems.

Early RAG systems, such as the foundational work by \cite{lewis2020pwr}, established the principle of augmenting LLMs with external knowledge. However, these often relied on a single-shot retrieval at the beginning of the generation process, lacking the flexibility for dynamic adaptation or self-correction. The need for LLM agency in managing its knowledge acquisition became apparent as researchers sought to address limitations arising from irrelevant or insufficient initial retrievals.

A pivotal development in enabling LLM agency is \textit{Self-RAG} \cite{Self-RAG}. This framework represents a significant paradigm shift by integrating a self-reflection mechanism directly into the LLM's generation process. Self-RAG trains the LLM to critique its own generated segments, assessing their quality and factual consistency. Based on this internal evaluation, the LLM dynamically decides whether to trigger further retrieval, refine its query, or regenerate a response. This self-awareness allows the model to proactively seek out missing information or correct inaccuracies, making the retrieval process an integral, iterative part of generation rather than a pre-processing step. The core innovation lies in the LLM's ability to \textit{decide} its next action, including when to interact with the knowledge base, based on its internal state and perceived generation quality.

Building on the concept of dynamic decision-making during generation, subsequent works have explored various facets of LLM-driven retrieval control. \textit{DRAGIN} \cite{su20241om} (Dynamic Retrieval Augmented Generation based on the Real-time Information Needs of LLMs) directly addresses the challenge of identifying the optimal moment to activate retrieval and crafting appropriate queries \textit{during} the text generation process. Unlike static rules or limited context windows, DRAGIN enables the LLM to make these decisions based on its real-time information needs, which may span the entire generated context. This allows for more precise and timely retrieval, ensuring that the LLM fetches information exactly when it's most relevant to the evolving response.

Further extending the LLM's control over the retrieval workflow, \textit{Auto-RAG} \cite{yu2024c32} introduces an autonomous iterative retrieval model. Here, the LLM engages in multi-turn dialogues with the retriever, systematically planning retrievals and refining queries until sufficient external information is gathered. This framework leverages the LLM's powerful reasoning capabilities to make complex decisions about the \textit{depth} and \textit{breadth} of retrieval, dynamically adjusting the number of iterations based on query difficulty and the utility of retrieved knowledge. This iterative planning and refinement, driven by the LLM itself, moves beyond simple reactive retrieval to a more strategic, goal-oriented information-seeking process.

Beyond merely deciding \textit{when} to retrieve, LLMs can also be empowered to adaptively \textit{utilize} and \textit{consolidate} knowledge. \textit{Astute RAG} \cite{wang2024kca} focuses on making RAG resilient to imperfect retrieval and knowledge conflicts. It enables the LLM to adaptively elicit essential information from its internal knowledge and iteratively consolidate this with external sources, while being aware of the information's reliability. This represents a sophisticated form of LLM agency where the model not only retrieves but also critically evaluates and integrates diverse knowledge sources under its own guidance, resolving potential conflicts and improving trustworthiness. Similarly, \textit{M-RAG} \cite{wang2024zt3} introduces a multi-partition retrieval paradigm where multi-agent reinforcement learning (MARL) agents, operating under the LLM's broader objective, dynamically select the most suitable knowledge partition and refine retrieved memories. While the direct decision-making is delegated to RL agents, the overall system's goal is to optimize the LLM's performance through dynamic, fine-grained knowledge selection, highlighting an advanced form of adaptive knowledge acquisition.

These advancements collectively highlight a clear trajectory towards more intelligent, self-aware RAG agents that can strategically manage and critique their knowledge utilization. The core benefit of these dynamic mechanisms is the ability to acquire more precise, context-aware, and relevant information, thereby significantly reducing hallucinations and improving factual consistency. However, this increased sophistication introduces challenges such as heightened system complexity, potential for sub-optimal LLM decisions in complex scenarios, and the computational overhead associated with iterative retrieval and self-reflection. Future research must balance these advanced capabilities with the need for efficiency, generalizability, and robust self-evaluation mechanisms to ensure that dynamic RAG systems remain practical and scalable for real-world applications.
\subsection{Integrating RAG with Black-Box Language Models}
\label{sec:3\_3\_integrating\_rag\_with\_black-box\_language\_models}

The widespread adoption of Retrieval-Augmented Generation (RAG) has demonstrated its efficacy in grounding large language models (LLMs) with up-to-date, factual information, thereby mitigating hallucinations and overcoming knowledge cutoffs \cite{lewis2020pwr, fan2024pf1, huang2024a59}. However, a significant practical challenge arises when attempting to apply RAG to proprietary or black-box LLMs, where internal modifications, fine-tuning, or end-to-end training of the generator component are not feasible. Foundational RAG architectures, such as the original RAG model \cite{lewis2020pwr}, typically involve jointly training or fine-tuning both a retriever and a sequence-to-sequence generator. This tight integration, while powerful, becomes impractical or impossible when the LLM is only accessible via an API, presenting a critical barrier to deploying RAG in many commercial and closed-source environments \cite{zhao2024931}.

Addressing this limitation, recent research has focused on methodologies that enable RAG for black-box LLMs by strategically optimizing external components without altering the LLM itself. A pivotal contribution in this area is \textit{REPLUG: Retrieval-Augmented Black-Box Language Models} \cite{replug}. This approach ingeniously circumvents the need for LLM modification by training \textit{only the retriever component} to optimize the black-box LLM's output. The core innovation of REPLUG lies in its ability to adapt the retrieval mechanism to the specific behaviors and preferences of a fixed, pre-trained black-box LLM. By treating the LLM as an unmodifiable function, REPLUG frames retriever training as an optimization problem where the objective is to maximize the utility of the LLM's output given the retrieved context. This is often achieved through techniques like policy gradient methods or contrastive learning, where the retriever learns to prioritize documents that elicit better responses from the black-box LLM. Crucially, the reward signal for retriever training is derived from the black-box LLM's output, typically by evaluating the log-likelihood of target completions or using an external metric to assess the quality, factuality, or relevance of the LLM's generation given the retrieved context. This allows the retriever to effectively learn to retrieve passages that lead to higher-quality responses from the immutable LLM, significantly broadening RAG's applicability to a wider range of commercial and closed-source models.

Beyond REPLUG, other strategies have emerged to enhance retrieval for black-box LLMs. One common approach involves using the black-box LLM itself as a re-ranker or a component in a reward model. For instance, a preliminary retriever can fetch a larger set of documents, which are then re-ranked by the black-box LLM based on their relevance to the query and their potential to improve the LLM's answer. This external re-ranking step, often guided by prompt engineering, refines the context before final generation. Similarly, frameworks like uRAG \cite{salemi2024bb6} propose unified retrieval engines that can serve multiple downstream RAG systems, where the feedback from these systems (which might include black-box LLMs) can be used to optimize the shared retriever. This suggests a broader ecosystem where retriever optimization is informed by the performance of various LLM-based applications.

The challenge of defining a reliable reward signal for retriever training in a black-box setting is considerable. While log-likelihoods on reference answers provide a direct signal for generation quality, their availability can be limited. Alternative methods often rely on proxy reward signals, such as human annotations, automated evaluation metrics (e.g., ROUGE, BLEU, or RAG-specific metrics like those in \cite{rau20244nr, guinet2024vkg}), or even another, smaller LLM acting as a critic. The computational cost of repeatedly querying a black-box LLM during the retriever training loop can also be substantial, posing a practical limitation for large-scale deployments. Despite these challenges, empirical studies demonstrate the effectiveness of RAG with black-box models. For example, \cite{ke2025wm0} extensively evaluates RAG with various black-box LLMs (GPT-3.5, GPT-4, Gemini, Llama2, Llama3, Claude) in medical fitness assessment, showing that RAG significantly enhances accuracy and reduces hallucinations compared to standalone LLMs, though performance can vary across different base models. Similarly, \cite{fayyazi2023qg6} compares RAG with decoder-only LLMs (like GPT-3.5) against fine-tuned encoder-only models for TTP analysis in cybersecurity, finding RAG to be superior when relevant context is extracted.

Furthermore, many general RAG enhancements can be applied upstream or downstream of a black-box LLM without requiring internal access. Query refinement techniques, such as RAG-Fusion \cite{ragfusion}, which generate multiple query perspectives, can enhance the initial retrieval step for any retriever, including those optimized for black-box LLMs. Dynamic retrieval mechanisms, like DR-RAG \cite{hei2024cs4}, which employ a two-stage retrieval framework and a compact classifier to assess document relevance, can be integrated as external components to improve the quality of context provided to a black-box LLM. These methods focus on intelligent orchestration and information acquisition, rather than modifying the black-box LLM's internal mechanisms.

In conclusion, methodologies like REPLUG represent a crucial advancement in making RAG a more accessible and practical solution for real-world deployments involving proprietary LLMs. By decoupling the training of the retriever from the internal architecture of the LLM, these approaches overcome the significant hurdle posed by black-box models. However, this paradigm introduces its own set of challenges, including the computational expense of using the LLM in the training loop and the difficulty of defining robust and generalizable reward signals. Future directions in this domain will likely explore more sophisticated external reward modeling, adaptive multi-stage retrieval systems that intelligently orchestrate calls to black-box LLMs, and robust benchmarking frameworks tailored for black-box RAG systems \cite{guinet2024vkg}. This ongoing research promises to expand the utility of RAG to an even broader spectrum of applications and models, making RAG a more versatile and accessible solution for many real-world deployment scenarios where direct LLM access is limited.


\label{sec:advanced_rag_architectures_and_system_optimizations}

\section{Advanced RAG Architectures and System Optimizations}
\label{sec:advanced\_rag\_architectures\_\_and\_\_system\_optimizations}

\subsection{Multi-Stage and Corrective RAG Frameworks}
\label{sec:4\_1\_multi-stage\_\_and\_\_corrective\_rag\_frameworks}

The efficacy of Retrieval-Augmented Generation (RAG) is profoundly influenced by the quality of its initial retrieval, a critical vulnerability that can lead to continued factual inaccuracies or hallucinations despite the integration of external knowledge \cite{lewis2020pwr}. Benchmarking studies have consistently revealed that Large Language Models (LLMs) struggle with noise robustness, negative rejection, and the seamless integration of information from retrieved documents \cite{chen2023nzb}. Furthermore, the inherent limitations of single-shot retrieval in addressing complex multi-hop queries, which necessitate synthesizing information from disparate sources, underscore a significant gap in current RAG systems' reasoning capabilities \cite{tang2024i5r}. Traditional retrieval metrics often fail to correlate adequately with the actual utility of documents to the LLM, highlighting the need for more sophisticated, LLM-centric assessment mechanisms \cite{salemi2024om5}. These challenges necessitate the development of multi-stage and corrective RAG frameworks that dynamically evaluate retrieval quality and implement adaptive strategies to enhance factual consistency and overall reliability.

A pioneering framework in this domain is Corrective Retrieval Augmented Generation (CRAG) \cite{yan202437z}. CRAG introduces a plug-and-play architecture designed to actively mitigate the impact of suboptimal initial retrieval. It employs a lightweight retrieval evaluator (a fine-tuned T5-large model) to dynamically assess the relevance of retrieved documents to the input query. Based on this confidence score, CRAG triggers one of three corrective actions: knowledge refinement for highly relevant documents, large-scale web search for irrelevant documents, or a combination of both for ambiguous cases. This dynamic, multi-stage decision-making, coupled with a "decompose-then-recompose" algorithm for fine-grained knowledge extraction, significantly bolsters RAG's robustness against unreliable initial retrieval.

This approach stands in contrast to other adaptive RAG paradigms, such as Self-RAG \cite{Self-RAG}, which primarily relies on the LLM's internal self-reflection and critique to decide whether to retrieve, generate, or regenerate. While Self-RAG integrates self-correction directly into the LLM's generation process using special tokens, CRAG leverages an \textit{external, lightweight evaluator} to make dynamic decisions about knowledge acquisition and refinement. This distinction highlights a key trade-off: CRAG offers modularity and potentially greater efficiency due to its compact evaluator, alongside the ability to tap into dynamic web search for broader knowledge correction. Conversely, Self-RAG's internal critique might allow for a deeper, more context-aware assessment tied to the LLM's own reasoning, albeit at a higher computational cost by engaging the full LLM for meta-reasoning.

Building upon the concept of dynamic assessment and iterative refinement, other frameworks further enhance RAG's corrective capabilities. Astute RAG \cite{wang2024kca} specifically addresses the challenges of imperfect retrieval and knowledge conflicts between the LLM's internal knowledge and external sources. It operates in a post-retrieval stage, adaptively eliciting essential information from the LLM's internal knowledge and iteratively consolidating it with external knowledge, while being source-aware. This framework then finalizes the answer based on information reliability, demonstrating resilience to misleading or irrelevant retrieved content through an iterative, conflict-resolving mechanism.

Similarly, Auto-RAG \cite{yu2024c32} introduces an autonomous iterative retrieval model centered on the LLM's decision-making capabilities. Instead of a single retrieval pass, Auto-RAG engages in multi-turn dialogues with the retriever, systematically planning retrievals and refining queries until sufficient external information is gathered. This LLM-driven iterative process, which includes autonomously adjusting the number of iterations based on question difficulty and knowledge utility, represents a multi-stage, self-correcting approach to information acquisition, enhancing both relevance and completeness. DR-RAG \cite{hei2024cs4} also proposes a two-stage retrieval framework that applies dynamic document relevance to improve recall and answer accuracy. It utilizes a compact classifier in its second stage to determine the contribution of initially retrieved documents and selectively retrieve additional relevant information, optimizing efficiency by calling the LLM only once for generation.

Even frameworks like PlanRAG \cite{lee2024hif}, while primarily focused on enabling LLMs as decision-makers through strategic planning (as discussed in Section 4.2), incorporate a crucial corrective element. PlanRAG extends the ReAct framework with explicit 'Plan' and 'Re-plan' instructions, allowing the LLM to iteratively assess its current plan based on retrieval results. If the plan is deemed insufficient or incorrect, the LLM dynamically generates a new plan or corrects its analytical direction, embodying a multi-stage, self-correcting reasoning process for complex tasks.

These multi-stage and corrective frameworks collectively represent a significant evolution in RAG, transforming it from a passive augmentation system into an active, adaptive, and self-aware knowledge acquisition and generation pipeline. By integrating dynamic evaluation, iterative query refinement, and intelligent post-retrieval processing, they directly address the vulnerabilities of RAG to poor initial retrieval, leading to more reliable and factually consistent outputs. However, challenges persist, including optimizing the computational overhead associated with multiple processing stages, ensuring the accuracy and generalizability of lightweight evaluators across diverse domains, and developing more sophisticated mechanisms for orchestrating diverse knowledge sources (static corpora, dynamic web searches, internal LLM knowledge) and complex reasoning steps in real-time. Future research will likely focus on developing more efficient, robust, and generalizable adaptive RAG agents that can seamlessly integrate these corrective mechanisms, balancing performance with computational feasibility across various task complexities.
\subsection{Unified RAG Pipelines and Reasoning Capabilities}
\label{sec:4\_2\_unified\_rag\_pipelines\_\_and\_\_reasoning\_capabilities}

Traditional Retrieval-Augmented Generation (RAG) systems often consist of distinct, sequentially arranged components for retrieval, ranking, and answer generation. While modular, this multi-stage architecture can introduce complexity, propagate errors, and incur significant inference latency. Recent research addresses these challenges by pursuing two complementary thrusts: unifying traditionally separate RAG components into a single, instruction-tuned Large Language Model (LLM), and endowing LLMs with more sophisticated, human-like reasoning capabilities to tackle intricate, multi-step queries. These advancements aim to streamline the RAG process, enhance generalization, and push RAG towards more intelligent problem-solving.

A significant stride towards pipeline unification is exemplified by \textbf{RankRAG} \cite{yu202480d}. This framework trains a single LLM to perform both context ranking and answer generation through instruction fine-tuning. By collapsing these two critical stages into one model, RankRAG simplifies the RAG architecture, reducing the overhead associated with orchestrating multiple components and potentially lowering inference latency. The authors demonstrate that this unified approach achieves superior performance and generalization across 14 knowledge-intensive benchmarks, even outperforming leading open-source and proprietary models, including GPT-4. A key insight from RankRAG is the surprising effectiveness of integrating a small fraction of ranking-specific data into the instruction-tuning blend, which enables the LLM to learn robust ranking capabilities that transfer well across domains, even to biomedical tasks without domain-specific tuning. This suggests that instruction tuning can effectively imbue LLMs with multi-task competence, challenging the necessity of specialized, separate models for each RAG sub-task. Another approach to streamlining context utilization within a single LLM pass is \textbf{Demonstration-based RAG (DRAG)} \cite{yue2024ump}, which leverages in-context learning by integrating extensive retrieved documents directly within the LLM's input demonstrations. This allows long-context LLMs to learn how to extract relevant information and answer questions from a rich input context in a single inference step, thereby scaling effective context utilization beyond mere document quantity.

Beyond architectural unification, a parallel research direction focuses on empowering RAG systems with advanced reasoning capabilities to navigate complex, multi-step queries that demand more than a single retrieval-generation pass. This shift is motivated by the limitations of current RAG systems in handling intricate logical deductions or planning-intensive tasks. To address this, \textbf{PlanRAG} \cite{lee2024hif} introduces an iterative "plan-then-retrieval" augmented generation technique, positioning the LLM as a dynamic decision-maker. PlanRAG enables the LLM to generate an initial plan for data analysis, execute retrieval based on this plan, and critically, iteratively refine the plan (re-planning) if initial results are insufficient. This adaptive planning mechanism significantly outperforms existing iterative RAG methods on complex "Decision QA" tasks, demonstrating the power of explicit, iterative planning in guiding information acquisition and synthesis.

Complementing planning-based approaches, \textbf{IM-RAG} \cite{yang20243nb} advances multi-round RAG by leveraging learned inner monologues and multi-agent reinforcement learning. IM-RAG allows for flexible and interpretable multi-round retrieval and complex decision-making. The system dynamically switches between "Questioner" and "Answerer" roles, guided by mid-step rewards, enabling a more sophisticated, human-like reasoning process that can adapt to evolving information needs during a complex query. This mimics cognitive processes where internal deliberation and self-correction guide external information seeking. Similarly, \textbf{Iterative Demonstration-based RAG (IterDRAG)} \cite{yue2024ump} extends the concept of inference scaling to multi-hop reasoning. IterDRAG decomposes complex queries into simpler sub-queries, iteratively performing retrieval and generating intermediate answers to construct robust reasoning chains. This strategy, when combined with optimal computation allocation, demonstrates significant performance gains on multi-hop datasets, highlighting the efficacy of iterative, decompositional reasoning for complex knowledge-intensive tasks.

In conclusion, the evolution of RAG is characterized by a dual focus on architectural streamlining and enhanced reasoning. Approaches like RankRAG demonstrate the profound benefits of unifying RAG components into a single, instruction-tuned LLM, leading to simplified pipelines, superior performance, and improved generalization. Simultaneously, frameworks such as PlanRAG, IM-RAG, and IterDRAG are pushing the boundaries of RAG's reasoning capabilities, enabling LLMs to engage in iterative planning, inner monologues, and multi-round retrieval to address increasingly complex and multi-hop queries. While these advancements promise more sophisticated and human-like problem-solving, challenges persist in optimizing the computational overhead inherent in iterative processes and ensuring robust, generalizable reasoning across increasingly diverse and complex real-world knowledge bases. The effectiveness of instruction tuning in imbuing LLMs with these advanced capabilities, even with limited specialized data, underscores its critical role in shaping the future of intelligent RAG systems.
\subsection{Efficiency and System-Level Optimizations for RAG}
\label{sec:4\_3\_efficiency\_\_and\_\_system-level\_optimizations\_for\_rag}

The rapid evolution of Retrieval-Augmented Generation (RAG) systems, while significantly enhancing Large Language Model (LLM) capabilities by grounding responses in external knowledge, introduces substantial computational and architectural complexities \cite{gao20238ea, huang2024a59, wu2024bpc}. Deploying RAG in high-performance, real-time scenarios necessitates innovations that improve computational efficiency and scalability, moving beyond mere retrieval quality to address system-level bottlenecks. Initial efforts in RAG optimization have focused on identifying best practices for core components, such as chunking strategies, embedding models, and vector database selection, to balance performance and efficiency across the workflow \cite{wang20248gm}. However, the sheer volume of retrieved context often leads to prohibitively long input sequences for LLMs, escalating inference costs and latency, particularly due to the generation of Key-Value (KV) caches during the LLM's prefill phase.

A primary bottleneck in RAG systems stems from the computational overhead of processing these long augmented sequences. To mitigate this, novel caching mechanisms have emerged, directly targeting the expensive KV cache computation. \textcite{jin20247cr} introduced \texttt{RAGCache}, a multilevel dynamic caching system specifically designed for RAG. It caches the intermediate states (KV tensors) of retrieved documents in a prefix tree structure, allowing for sharing across multiple requests and adapting to GPU and host memory hierarchies. Coupled with a Prefix-aware Greedy-Dual-Size-Frequency (PGDSF) replacement policy and dynamic speculative pipelining, \texttt{RAGCache} demonstrated a reduction in Time to First Token (TTFT) by up to 4x and an improvement in throughput by up to 2.1x compared to state-of-the-art inference systems. While effective, the performance of such dynamic caching relies heavily on the similarity and recurrence of retrieval patterns across queries, and managing cache invalidation for dynamic knowledge bases remains a challenge. Building on this, \textcite{lu2024pvt} proposed \texttt{TurboRAG}, which takes an even more aggressive approach by pre-computing and storing the KV caches of document chunks offline. This eliminates the need for online KV cache computation during inference, leading to a remarkable reduction in TTFT by up to 9.4x while maintaining comparable accuracy to standard RAG systems. The trade-off here lies in the significant pre-computation cost and memory footprint required to store these pre-computed caches, which may not be feasible for extremely large or frequently updated knowledge bases. These caching strategies are crucial for accelerating the LLM inference stage by reusing expensive computations, but their practical applicability depends on the specific deployment context and data volatility.

Beyond caching, optimizing the interplay between the retrieval and generation components is vital for overall system throughput and reduced end-to-end latency. Traditional RAG systems often suffer from sequential dependencies, where retrieval must complete before generation can begin, or iterative retrieval introduces pauses. \textcite{jiang20243ac} addressed this with \texttt{PipeRAG}, an algorithm-system co-design approach for periodic RAG systems. \texttt{PipeRAG} introduces pipeline parallelism by using a "stale" query window to prefetch content, allowing concurrent execution of retrievals and inferences. It also employs flexible retrieval intervals and a performance-model-driven retrieval system that dynamically adjusts the Approximate Nearest Neighbor (ANN) search space to optimize latency. This co-design achieved up to a 2.6x speedup in end-to-end generation latency. However, the complexity of managing stale queries and the overhead of the performance model for dynamic ANN adjustments can introduce its own set of engineering challenges, requiring careful tuning to avoid performance degradation due to outdated context or inaccurate model predictions.

Furthermore, strategies for managing and retrieving from large knowledge bases are critical for scalability and precision, which indirectly contribute to efficiency by reducing the computational burden on the LLM. Traditional RAG often queries a monolithic database, leading to coarse-grained retrieval and the inclusion of irrelevant information that the LLM still has to process. \textcite{wang2024zt3} introduced \texttt{M-RAG}, a multiple partition paradigm that organizes external memories into distinct partitions. This framework uses a multi-agent reinforcement learning (MARL) approach, where Agent-S selects the most suitable partition for a query, and Agent-R refines retrieved memories within that partition. This fine-grained context selection, guided by RL agents, significantly improved performance across various language generation tasks, including an 11\\% gain for text summarization and 12\\% for dialogue generation, by reducing noise and focusing retrieval. By providing a more concise and relevant context, \texttt{M-RAG} effectively reduces the input length to the LLM, thereby lowering inference costs and latency. The main challenges for \texttt{M-RAG} include the complexity of training and maintaining the MARL agents, the overhead of dynamic partition selection, and the initial effort required to effectively partition the knowledge base.

In conclusion, the journey towards efficient and scalable RAG systems involves a multifaceted approach, encompassing sophisticated caching mechanisms to reduce LLM inference latency, algorithm-system co-design to overlap retrieval and generation, and advanced retrieval strategies that leverage knowledge partitioning for more precise context delivery. These innovations directly address the practical deployment challenges of RAG, such as high inference costs and latency, by optimizing the computational pipeline and reducing the amount of irrelevant information processed by the LLM. While these advancements significantly enhance RAG's practical deployability in high-performance, real-time environments, a persistent challenge remains in dynamically adapting these complex architectures to diverse data characteristics and evolving query patterns. This often requires careful tuning and a delicate trade-off between architectural complexity and raw computational efficiency. Future research will likely focus on more adaptive caching policies, autonomous optimization of retrieval intervals, and intelligent, data-driven partitioning strategies to further unlock RAG's full potential, alongside exploring hardware-aware RAG frameworks that can leverage specialized accelerators for even greater efficiency.


\label{sec:specialized_knowledge_integration_paradigms}

\section{Specialized Knowledge Integration Paradigms}
\label{sec:specialized\_knowledge\_integration\_paradigms}

\subsection{Multimodal Retrieval-Augmented Generation}
\label{sec:5\_1\_multimodal\_retrieval-augmented\_generation}

The inherent limitations of text-only Retrieval-Augmented Generation (RAG) systems, which struggle with queries requiring non-textual knowledge, have spurred a critical advancement towards integrating diverse modalities such as images, video, and audio into the knowledge retrieval and generation process. This expansion addresses the need for Large Language Models (LLMs) to access a broader spectrum of human knowledge, leading to more comprehensive and contextually rich responses, particularly for visually-grounded queries where visual evidence is paramount.

A pioneering effort in this domain is \textit{MuRAG: Multimodal Retrieval-Augmented Generator for Open Question Answering over Images and Text} by \cite{chen2022j8c}. MuRAG introduced a Multimodal Retrieval-Augmented Transformer, enabling LLMs to effectively leverage both visual and textual information. Its core methodology involves a unified multimodal encoder, combining pre-trained T5 and ViT models, to process queries and memory candidates across image and text modalities. A retriever stage, typically using Maximum Inner Product Search (MIPS), fetches Top-K multimodal items, which are then fed to a reader stage for text generation. A key innovation was its joint pre-training objective, combining a contrastive loss for effective retrieval and a generative loss for leveraging multimodal knowledge, alongside an efficient two-stage fine-tuning pipeline. While MuRAG significantly advanced the field by integrating diverse knowledge modalities and demonstrating state-of-the-art performance on multimodal QA datasets like WebQA, it also highlighted the inherent complexity and computational cost of managing massive external multimodal memories, posing scalability challenges for truly open-domain applications.

Building upon such foundational architectures, subsequent research has explored alternative strategies for multimodal document processing and retrieval. For instance, \cite{yu2024arx} introduced \textit{VisRAG}, a novel vision-language model (VLM)-based RAG pipeline that directly embeds entire multi-modality documents (including layout and images) as images, rather than first parsing them into text. This approach aims to maximize the retention and utilization of visual and layout information, eliminating the data loss often introduced during the parsing process in traditional text-based RAG. VisRAG's direct visual embedding offers a distinct architectural paradigm compared to MuRAG's separate text and image encoders, demonstrating superior performance by preserving the holistic context of multi-modality documents.

A significant challenge in multimodal RAG (MM-RAG) is the "multi-granularity noisy correspondence (MNC) problem," where retrieved multimodal evidence may contain irrelevant or conflicting information, hindering accurate generation. To address this, \cite{chen20245d2} proposed \textit{RagVL}, which leverages Multimodal Large Language Models (MLLMs) as powerful rerankers. RagVL instruction-tunes an MLLM to precisely filter the top-k retrieved images, thereby improving the relevance and quality of the multimodal context fed to the generator. Furthermore, to enhance the generator's robustness against noisy retrieval, RagVL injects visual noise during training at both data and token levels. This approach refines MuRAG's paradigm by introducing a more sophisticated mechanism for filtering and utilizing multimodal evidence, acknowledging that raw retrieval often benefits from an additional verification or ranking step. The importance of such robust retrieval and reranking strategies is underscored by broader RAG optimization efforts, as discussed by \cite{wang20248gm}, which systematically explores best practices across RAG components, including the integration of multimodal retrieval techniques for visual inputs and "retrieval as generation" for multimodal content.

Beyond improving retrieval quality, MM-RAG is crucial for addressing critical issues like hallucination and enhancing trustworthiness, particularly in high-stakes domains. MLLMs, despite their capabilities, are prone to generating factually incorrect information. \cite{chu2025wz5} demonstrated how Visual RAG (V-RAG) can significantly reduce hallucinations in medical MLLMs, specifically for tasks like chest X-ray report generation. Their work showed that V-RAG improves the accuracy of entity probing, ensuring that medical entities are properly grounded by retrieved images, thereby leading to more clinically accurate reports. This highlights MM-RAG's potential to provide verifiable grounding for sensitive applications. Complementing this, \cite{ma20245jl} introduced \textit{VISA: Retrieval-Augmented Generation with Visual Source Attribution}, which focuses on enhancing the verifiability of MM-RAG systems. VISA leverages large vision-language models to not only generate answers but also to identify and highlight the exact regions (with bounding boxes) in retrieved document screenshots that support the generated content. This visual attribution is critical for users to locate evidence, especially in complex, content-rich documents like Wikipedia pages or scientific papers, thereby significantly boosting the trustworthiness and explainability of MM-RAG outputs.

The capabilities of MM-RAG extend to complex reasoning and specialized applications. \cite{khaliq2024ne2} introduced \textit{RAGAR: Your Falsehood Radar}, applying multimodal RAG to the challenging task of political fact-checking for multimodal claims. RAGAR employs novel reasoning techniques like Chain of RAG (CoRAG) and Tree of RAG (ToRAG), which iteratively extract textual and image content, retrieve external information, and reason through subsequent questions based on prior evidence. This demonstrates how MM-RAG can move beyond simple question answering to support sophisticated, multi-step reasoning processes grounded in diverse evidence.

Despite these advancements, the field faces ongoing challenges. The scalability to even larger and more diverse multimodal knowledge bases, potentially including video and audio, remains a complex computational and architectural problem. Current research predominantly focuses on image-text pairs, with video and audio modalities often mentioned as future work rather than being deeply integrated. Benchmarking efforts, such as \textit{Visual-RAG} by \cite{wu2025eum}, specifically aim to isolate and measure the contribution of retrieved images in RAG for visually grounded queries. Their findings indicate that even state-of-the-art MLLMs struggle to efficiently extract and utilize visual knowledge, highlighting the need for improved visual retrieval, grounding, and attribution. Furthermore, handling complex inter-modal reasoning, ensuring dynamic knowledge updates, and developing robust evaluation metrics for truly open-domain multimodal understanding are critical areas for future research. The development of more advanced multimodal fusion techniques, adaptive retrieval strategies that can dynamically select modalities based on query context, and sophisticated mechanisms for handling ambiguity across modalities will be essential for realizing the full potential of multimodal RAG.
\subsection{Graph-Augmented RAG (GraphRAG)}
\label{sec:5\_2\_graph-augmented\_rag\_(graphrag)}

While Retrieval-Augmented Generation (RAG) significantly enhances Large Language Models (LLMs) by grounding responses in external knowledge, traditional RAG systems primarily operate on unstructured text, often struggling with complex relationships and explicit facts inherent in many real-world domains \cite{gao20238ea, huang2024a59}. This limitation can lead to persistent hallucinations and difficulties in handling intricate, multi-hop queries \cite{chen2023nzb, tang2024i5r}. Graph-Augmented RAG (GraphRAG) emerges as a powerful paradigm to address these challenges by integrating structured knowledge, particularly Knowledge Graphs (KGs) and textual graphs, into the retrieval process to provide LLMs with verifiable facts and relational context \cite{peng2024mp3, zhang2025gnc}.

Early explorations into GraphRAG focused on leveraging graph structures to improve retrieval and generation for specific tasks. \cite{he20248lp} introduced G-Retriever, a pioneering framework for textual graph understanding and question answering. It formulates subgraph retrieval as a Prize-Collecting Steiner Tree (PCST) problem, enabling the extraction of relevant subgraphs that capture neighborhood information, thereby mitigating hallucination and improving scalability for general textual graphs. Building on this, \cite{xu202412d} demonstrated the practical benefits of integrating RAG with KGs for customer service question answering. Their method constructs a dual-level KG that preserves both intra-issue structure and inter-issue relations, utilizing LLM-driven query parsing and Cypher queries for precise subgraph extraction, leading to substantial improvements in retrieval efficacy and answer quality.

Further advancements have focused on more sophisticated graph-aware retrieval algorithms and integration strategies. \cite{hu2024eyw} proposed GRAG for networked documents, employing a divide-and-conquer strategy for efficient textual subgraph retrieval and a graph soft pruning mechanism to filter irrelevant information. GRAG integrates graph context into LLMs through both hierarchical text descriptions (hard prompts) and GNN-based soft prompts, where message passing is guided by learned relevance factors. Addressing the challenge of multi-hop reasoning in KGQA, \cite{mavromatis2024ml9} introduced GNN-RAG, which repurposes Graph Neural Networks (GNNs) as "dense subgraph reasoners" to identify answer candidates and extract precise reasoning paths from KGs. This approach significantly outperforms LLM-based retrievers on complex multi-hop questions, achieving state-of-the-art results on benchmarks like WebQuestionsSP and Complex WebQuestions. In contrast, \cite{li2024hb4} argued for simplicity with SubgraphRAG, utilizing a lightweight Multilayer Perceptron (MLP) combined with Directional Distance Encoding (DDE) for efficient and flexible multi-hop subgraph retrieval. By framing subgraph retrieval as a triple factorization problem, SubgraphRAG allows for adjustable subgraph sizes and achieves competitive performance with smaller LLMs without fine-tuning, demonstrating that effective structural feature encoding can be highly efficient.

Recognizing the complementary strengths of vector-based and graph-based retrieval, hybrid GraphRAG approaches have emerged. \cite{sarmah20245f3} proposed HybridRAG, explicitly integrating VectorRAG and GraphRAG components. Their system employs a two-tiered LLM chain for robust KG construction from unstructured text, combining vector similarity search with graph traversal to overcome the limitations of each method individually, particularly in complex financial document analysis. Taking this integration a step further, \cite{ma2024pwd} introduced Think-on-Graph 2.0 (ToG-2), a hybrid RAG framework that performs iterative, "tight-coupling" knowledge exploration between unstructured documents and KGs. ToG-2 leverages LLMs for relation and entity pruning, guiding a continuous search for in-depth clues, which results in deeper and more faithful LLM reasoning across various knowledge-intensive tasks.

The field of GraphRAG is still evolving, with comprehensive surveys like \cite{peng2024mp3}, \cite{procko202417i}, and \cite{zhang2025gnc} formalizing its workflow and categorizing methodologies. \cite{zhang2025gnc} notably classifies GraphRAG into knowledge-based, index-based, and hybrid categories, highlighting the diverse ways graphs can be leveraged. Despite significant progress in mitigating hallucination and enhancing reasoning, challenges persist in the complexity of KG construction, ensuring the quality and completeness of graph knowledge, and developing highly efficient large-scale graph traversal and reasoning algorithms for real-time applications. Future research will likely focus on more automated and dynamic KG construction, adaptive graph-aware retrieval mechanisms, and seamless integration strategies to further ground LLMs in verifiable, structured knowledge.
\subsection{Internal Knowledge Recitation and Self-Awareness}
\label{sec:5\_3\_internal\_knowledge\_recitation\_\_and\_\_self-awareness}

A distinct paradigm in augmenting large language model (LLM) generation involves the model leveraging its own parametric memory, rather than solely relying on external corpora. This approach, often termed "internal knowledge recitation" or "parametric knowledge elicitation," allows LLMs to articulate knowledge already encoded within their parameters. It offers a compelling alternative or complementary method to traditional Retrieval-Augmented Generation (RAG) for certain types of information, particularly commonly known facts or knowledge acquired during extensive pre-training. This method enables LLMs to access and utilize their learned knowledge more effectively, potentially reducing inference latency and reliance on external infrastructure for information readily available within their weights.

A pioneering work in this domain is the RECITation-augmented gEneration (RECITE) framework proposed by \cite{sun2022hx2}. RECITE introduces a two-step process where, given an input, the LLM first "recites" one or several relevant passages sampled from its \textit{own internal memory}, and then proceeds to produce the final answer based on this self-generated information. This mechanism stands in stark contrast to traditional RAG systems that primarily fetch information from external, explicit knowledge bases. The core innovation lies in treating the LLM's parametric knowledge as a latent, retrievable corpus, accessible through carefully designed prompting. To facilitate this internal recitation, \cite{sun2022hx2} leverages prompt-based in-context learning, providing the LLM with exemplars of questions and corresponding recited evidences. For robustness, a self-consistency ensemble method generates multiple independent recitations, with a majority vote determining the final output. For multi-hop question answering, RECITE employs a multiple-recite-and-answer scheme, prompting sequential recitations where later recitations can build upon previous ones. While inference is closed-book, the model is fine-tuned on synthetically generated question-passage hint pairs to enhance its ability to recite relevant knowledge, demonstrating significant performance improvements on various closed-book QA tasks like Natural Questions and HotpotQA \cite{sun2022hx2}.

The paradigm of internal knowledge recitation presents a unique perspective compared to the extensive research focused on optimizing \textit{external} RAG pipelines. While many advanced RAG techniques concentrate on refining queries for external retrieval \cite{chan2024u69} or enhancing reasoning over retrieved external evidence \cite{islam2024ug5}, RECITE demonstrates that substantial knowledge can be extracted directly from the LLM's weights through improved prompting and architectural design. The choice between relying on internal parametric knowledge and external retrieval is often a trade-off, influenced by the nature and popularity of the required information. \cite{soudani20247ny} conducted a comprehensive empirical comparison, finding that while RAG generally outperforms fine-tuning (a method of injecting knowledge into parametric memory) for less popular or "long-tail" factual knowledge, the effectiveness of both approaches is highly dependent on factors like LM size, fine-tuning method, and data augmentation quality. This suggests that for widely known facts or knowledge frequently encountered during pre-training, internal elicitation might be efficient, whereas external RAG remains crucial for novel, dynamic, or highly specific information.

Furthermore, a critical challenge arises when an LLM's internal knowledge conflicts with external retrieved information, necessitating mechanisms for adaptive knowledge utilization. \cite{wang2024kca} addresses this by proposing a framework that enables LLMs to \textit{adaptively elicit} internal knowledge and resolve such conflicts. Their approach involves training a "knowledge conflict detector" to identify discrepancies between the LLM's internal beliefs and external evidence. The system then guides the LLM to either prioritize the external context or to generate a response based on its internal knowledge, potentially after further internal "reflection" or re-evaluation. This highlights a move towards hybrid systems that dynamically decide how to best leverage both internal and external knowledge sources, rather than relying on one exclusively. This adaptive strategy is particularly valuable for maintaining factual consistency and mitigating hallucinations that can arise from conflicting information.

Despite its promise, the internal knowledge recitation approach presents its own set of challenges and limitations. The reliance on "fuzzy memorization" means that the fidelity and specificity of internally recited knowledge might vary, potentially leading to "internal hallucinations" if the knowledge is not robustly encoded or accurately recalled \cite{sun2022hx2}. This inherent unreliability of parametric knowledge is a well-documented issue across LLMs, where models can confidently generate factually incorrect information even for topics they were extensively trained on \cite{yan202437z}. Unlike external RAG, where the provenance of information is explicit and verifiable, tracing the source or verifying the accuracy of internally recited facts is significantly more challenging. Moreover, updating or correcting outdated internal knowledge requires costly re-training or model editing techniques \cite{meng2022locating}, which are less agile than simply updating an external knowledge base in a RAG system. The process of fine-tuning for recitation, even with synthetic data, still requires some form of external knowledge to generate the training pairs, indicating that even "internal" knowledge is ultimately derived from external sources and subject to the biases and limitations of its training data.

The exploration of internal knowledge recitation is deeply intertwined with broader research into LLM knowledge representation, attribution, and model editing. Techniques for probing LLMs to understand what knowledge they possess and where it is stored (e.g., knowledge neurons \cite{dai2022knowledge}) provide foundational insights into how internal recitation might function. Similarly, advancements in model editing, such as ROME \cite{meng2022locating} and MEMIT \cite{meng2023mass}, demonstrate methods for directly modifying specific factual associations within a model's parameters, offering potential avenues for updating or refining internally accessible knowledge. Future research in this domain could focus on developing more sophisticated mechanisms for LLMs to \textit{quantify their confidence} in internal knowledge, enabling a more informed decision to either recite or seek external validation. This could involve uncertainty estimation techniques or explicit internal "verification" steps. Building on the work of \cite{wang2024kca}, hybrid models that dynamically decide between internal recitation and external retrieval based on the query's nature, the model's confidence, or the perceived popularity/recency of the required information could lead to more robust and efficient knowledge-augmented generation systems. For instance, a system might first attempt internal recitation for common queries, and only if confidence is low or the query is highly specific/recent, trigger an external RAG lookup. Further exploration into methods for improving the \textit{attributability} of internally recited facts, perhaps by linking them back to specific training examples or conceptual clusters within the model, would also be a significant step towards more trustworthy LLMs.


\label{sec:evaluation,_benchmarking,_and_robustness_of_rag_systems}

\section{Evaluation, Benchmarking, and Robustness of RAG Systems}
\label{sec:evaluation,\_benchmarking,\_\_and\_\_robustness\_of\_rag\_systems}

\subsection{Foundational Benchmarks and Retrieval Quality Assessment}
\label{sec:6\_1\_foundational\_benchmarks\_\_and\_\_retrieval\_quality\_assessment}

The robust evaluation of Retrieval-Augmented Generation (RAG) systems is paramount for understanding their capabilities, diagnosing limitations, and guiding iterative improvements. Initial research in this domain focused on establishing foundational benchmarks to systematically assess RAG's core functionalities and the Large Language Model's (LLM) ability to effectively integrate retrieved information. This foundational assessment provides a baseline for performance across critical aspects of retrieval and generation, moving beyond simplistic metrics to evaluate the true utility of the retrieved context.

A pioneering effort to systematically evaluate RAG's fundamental abilities was the Retrieval-Augmented Generation Benchmark (RGB) presented by \textcite{chen2023nzb}. RGB was designed to diagnose LLMs across four core RAG capabilities: noise robustness (handling irrelevant retrieved documents), negative rejection (refusing to answer when no relevant information is available), information integration (synthesizing information from multiple relevant documents), and counterfactual robustness (identifying and rejecting information that contradicts known facts). The findings from RGB were crucial, revealing that despite RAG's general improvements in accuracy, LLMs frequently struggled with these fundamental challenges. For instance, models often failed to reject answers when no relevant context was provided or inadequately synthesized information from disparate documents, highlighting specific bottlenecks in their ability to effectively leverage retrieved context. This diagnostic approach underscored the need for more sophisticated evaluation beyond simple accuracy scores, focusing on the LLM's reasoning and contextual understanding.

Building upon the need for deeper insights into retrieval utility, a crucial methodological shift emerged, moving beyond raw retrieval metrics (e.g., precision, recall) to directly evaluate the \textit{functional utility} of retrieved content to the LLM's downstream generation performance. \textcite{salemi2024om5} introduced eRAG, a novel methodology designed to assess retrieval quality by feeding each retrieved document individually to the LLM and evaluating its contribution to the final answer's correctness. This approach demonstrated a significantly higher correlation with downstream RAG performance compared to traditional metrics or even external LLM judges, offering a more accurate and efficient indicator of a RAG system's true efficacy. eRAG's strength lies in its ability to quantify the \textit{impact} of retrieval on generation, providing a more reliable signal for system optimization.

Further advancing the practical assessment of RAG systems, frameworks like RAGAS \cite{es2023ragas} have gained significant traction. RAGAS offers a set of metrics specifically tailored for RAG evaluation, focusing on aspects such as faithfulness (the degree to which the generated answer is grounded in the retrieved context), answer relevance (how relevant the answer is to the query), and context relevance (how relevant the retrieved context is to the query). These metrics often leverage LLMs as evaluators, providing a scalable and automated way to assess the quality of RAG outputs. RAGAS addresses the challenge of evaluating open-ended generation by breaking down the problem into measurable components that reflect the core objectives of RAG: accurate grounding and relevant response generation. Its widespread adoption highlights the community's shift towards more holistic and context-aware evaluation.

To provide even more granular and explainable feedback, \textcite{friel20241ct} developed RAGBench and the TRACe evaluation framework. TRACe introduced novel metrics like "Context Utilization" and "Completeness" alongside "Relevance" and "Adherence." These metrics go beyond simple correctness to provide actionable insights into \textit{how} the LLM leverages the retrieved context, such as whether it uses all relevant information or ignores parts of it. This level of detail is invaluable for diagnosing specific RAG system weaknesses and guiding targeted improvements. Pushing the boundaries of interpretability and component-level analysis, \textcite{guinet2024vkg} proposed an automated evaluation method using task-specific exam generation and Item Response Theory (IRT). This sophisticated framework decomposes a RAG system's overall ability into distinct contributions from its LLM, retrieval mechanism, and in-context learning components. By isolating these factors, IRT-based evaluation offers fine-grained, component-level insights into performance drivers, allowing developers to pinpoint whether issues stem from poor retrieval, inadequate LLM processing, or insufficient in-context learning.

In conclusion, the evolution of RAG evaluation has progressed from initial diagnostic benchmarks like RGB, which identified fundamental LLM limitations in handling retrieved information, to more sophisticated methodologies focused on the utility and impact of retrieval on the LLM's generation quality. Approaches like eRAG, RAGAS, RAGBench, and IRT-based evaluation represent a critical shift towards comprehensive, explainable, and component-level assessment. While these advancements offer more accurate and efficient indicators of RAG efficacy, the ongoing challenge remains in developing universally applicable, efficient, and deeply interpretable evaluation frameworks that can accurately reflect real-world RAG performance and guide continuous system optimization across diverse and dynamic applications, especially given the inherent complexities and potential biases of LLM-as-a-judge approaches.
\subsection{Benchmarking for Complex Queries and Domain-Specific Challenges}
\label{sec:6\_2\_benchmarking\_for\_complex\_queries\_\_and\_\_domain-specific\_challenges}

While foundational benchmarks have provided initial insights into Retrieval-Augmented Generation (RAG) systems, a deeper understanding of their performance under challenging conditions and in specialized application contexts necessitates the development of more targeted evaluation frameworks. General benchmarks often fall short in diagnosing nuanced failures related to complex reasoning, domain-specific factual accuracy, and the precise utilization of retrieved information, prompting a shift towards specialized testbeds.

Early efforts, such as the Retrieval-Augmented Generation Benchmark (RGB) by \cite{chen2023nzb}, laid crucial groundwork by systematically diagnosing RAG's fundamental abilities, including noise robustness, negative rejection, information integration, and counterfactual robustness. Despite revealing significant struggles for Large Language Models (LLMs) in these areas, particularly with noise and synthesizing information from multiple documents, RGB highlighted the need for benchmarks that could probe even more intricate reasoning patterns and domain-specific challenges.

Addressing the limitations of single-hop query evaluations, \cite{tang2024i5r} introduced \textit{MultiHop-RAG}, the first dedicated benchmark for complex multi-hop queries. This dataset, generated through a GPT-4-driven pipeline, categorizes queries into Inference, Comparison, Temporal, and Null types, demanding that RAG systems retrieve and synthesize information from multiple disparate sources. Initial evaluations on MultiHop-RAG revealed unsatisfactory performance from state-of-the-art RAG systems and LLMs, underscoring a significant gap in their multi-document reasoning capabilities.

In high-stakes application contexts, the need for specialized evaluation becomes even more critical. For the medical domain, \cite{xiong2024exb} developed \textit{MIRAGE} (Medical Information Retrieval-Augmented Generation Evaluation) and the accompanying MEDRAG toolkit. This benchmark features 7,663 questions across diverse medical datasets and enforces realistic evaluation settings like "Question-Only Retrieval." MIRAGE demonstrated that while RAG can significantly improve LLM accuracy in medicine, systems still struggle with complex examination questions and exhibit the "lost-in-the-middle" phenomenon, where relevant information is overlooked if not optimally positioned.

Similarly, the legal domain demands extreme precision and factual grounding. \cite{pipitone2024sfx} introduced \textit{LegalBench-RAG}, a benchmark specifically designed to evaluate the \textit{retrieval quality} in legal RAG systems, focusing on extracting minimal, highly relevant text snippets (character-level spans) from legal documents. Unlike general legal benchmarks that bypass the retrieval step, LegalBench-RAG's granular ground truth directly addresses critical concerns like context window limitations, hallucination risks, and the need for precise citations in legal applications.

Extending the scope to diverse tasks and languages, \cite{lyu2024ngu} presented \textit{CRUD-RAG}, a comprehensive Chinese benchmark for RAG systems. This novel framework categorizes RAG applications beyond traditional Question Answering (QA) into Create (e.g., text continuation), Read (QA), Update (e.g., hallucination modification), and Delete (e.g., multi-document summarization). By leveraging GPT-4 for high-quality data generation from recent news, CRUD-RAG provides a multifaceted testbed that reveals the impact of various RAG components across different task types, highlighting the need for tailored RAG optimizations.

Collectively, these specialized benchmarks demonstrate that current RAG systems exhibit significant shortcomings in complex reasoning, robustness to diverse information structures, and adaptability to domain-specific requirements. The consistent reliance on LLMs for sophisticated data generation and, in some cases, automated assessment methods, has been instrumental in creating these comprehensive and challenging testbeds. However, these evaluations consistently reveal that despite advancements, significant gaps remain in RAG systems' ability to reliably perform multi-hop reasoning, precisely utilize context in high-stakes domains, and adapt to a wide array of generative tasks, necessitating continued research into more intelligent and context-aware RAG architectures.
\subsection{Ensuring Robustness, Trustworthiness, and Privacy in RAG}
\label{sec:6\_3\_ensuring\_robustness,\_trustworthiness,\_\_and\_\_privacy\_in\_rag}

While Retrieval-Augmented Generation (RAG) systems significantly enhance Large Language Models (LLMs) by grounding their responses in external knowledge, their real-world deployment necessitates rigorous attention to non-functional requirements such as robustness, trustworthiness, and privacy. These concerns are paramount, as RAG's reliance on external data introduces new vulnerabilities and amplifies existing challenges related to data quality, security, and ethical implications \cite{zhou20248fu}.

A primary concern for RAG systems is their resilience to irrelevant or noisy retrieved information. Early benchmarking efforts, such as RGB \cite{chen2023nzb}, systematically diagnosed fundamental weaknesses of RAG, revealing LLMs' struggles with noise robustness, negative rejection (failing to decline answers when no information is present), and information integration from multiple documents. To address these, corrective strategies have emerged, such as Corrective RAG (CRAG) \cite{yan202437z}, which dynamically assesses retrieval quality using a lightweight evaluator and triggers actions like knowledge refinement or web searches to mitigate the impact of poor initial retrieval. Complementing this, proactive approaches like RQ-RAG \cite{chan2024u69} train LLMs to refine queries through rewriting, decomposition, or disambiguation, thereby improving retrieval effectiveness from the outset. RankRAG \cite{yu202480d} further enhances robustness by unifying context ranking and answer generation within a single instruction-tuned LLM, simplifying the pipeline and improving performance.

Beyond architectural refinements, more sophisticated defense mechanisms against diverse noise types are crucial. Retrieval-augmented Adaptive Adversarial Training (RAAT) \cite{fang2024gh6} systematically categorizes retrieval noises (relevant, irrelevant, counterfactual) and employs an adaptive adversarial training mechanism to dynamically adjust the model's training, enhancing its internal capacity to discern and resist noisy contexts. Similarly, Astute RAG \cite{wang2024kca} tackles imperfect retrieval and knowledge conflicts by adaptively eliciting essential information from LLMs' internal knowledge and iteratively consolidating internal and external sources based on reliability. Efficiency and quality are also improved by Sparse RAG \cite{zhu2024h7i}, which selectively decodes by attending only to highly relevant caches, filtering undesirable contexts, and Speculative RAG \cite{wang20246hs}, which uses a smaller drafter LLM to generate multiple answer drafts in parallel, then efficiently verifies them with a larger LLM. In domain-specific applications, such as medicine, the MIRAGE benchmark \cite{xiong2024exb} highlights RAG's "lost-in-the-middle" phenomenon, while i-MedRAG \cite{xiong2024u1b} improves robustness for complex medical QA by enabling iterative follow-up questions and multi-step reasoning. The quality of the underlying data is also paramount; for instance, enhanced PDF structure recognition by ChatDOC \cite{lin2024s1v} significantly boosts RAG performance by ensuring accurate extraction from complex professional documents. Evaluating the utility of retrieved documents directly from the LLM's perspective, as proposed by eRAG \cite{salemi2024om5}, offers a more reliable metric for optimizing retrieval quality and, by extension, system robustness.

Privacy is another critical dimension of trustworthiness in RAG systems \cite{zhou20248fu}. A systematic exploration by \cite{zeng2024dzl} revealed significant vulnerabilities of RAG systems to data leakage from their external retrieval databases through composite structured prompting attacks. This work demonstrated that attackers can craft specific prompts to direct the retriever to fetch sensitive data and instruct the LLM to output it. Counter-intuitively, \cite{zeng2024dzl} also found that RAG can mitigate the leakage of the LLM's \textit{own} training data, suggesting a complex interplay of privacy risks.

Beyond privacy, RAG systems are susceptible to a range of adversarial attacks and broader trustworthiness concerns. RAG poisoning, where malicious content is injected into the retrieval database, poses a severe threat. Pandora \cite{deng2024k1b} demonstrated how to jailbreak GPTs by poisoning RAG knowledge sources with policy-violating content and then triggering its retrieval through tailored prompts. Extending this, BadRAG \cite{xue2024bxd} introduced methods for crafting adversarial passages that are selectively retrieved by semantic triggers, enabling indirect generative attacks like sentiment steering or denial-of-service (DoS) by exploiting LLM alignment features. Opinion manipulation attacks \cite{chen20247nc} further show how RAG's ranking results can be manipulated to alter the opinion polarity of generated content, potentially misleading users. Even subtle prompt perturbations can significantly affect RAG outputs, leading to factually incorrect answers \cite{hu2024i6h}. To systematically evaluate these security vulnerabilities, SafeRAG \cite{liang2025f4q} provides a benchmark for assessing RAG security against various attack tasks, including silver noise, inter-context conflict, soft ads, and white DoS. Addressing bias in retrieved information is also crucial for trustworthiness; RAG$^2$ \cite{sohn2024w2t} mitigates retriever bias by retrieving snippets evenly from a comprehensive set of corpora. Furthermore, enhancing explainability, as seen in G-Retriever's \cite{he20248lp} subgraph retrieval for textual graphs, contributes to overall trustworthiness by providing transparent reasoning paths.

In conclusion, ensuring robustness, trustworthiness, and privacy in RAG systems is a multifaceted challenge. While significant progress has been made in developing advanced architectures, adaptive defense mechanisms against noise and adversarial attacks, and identifying critical privacy vulnerabilities, the dynamic nature of RAG and the continuous evolution of LLMs necessitate ongoing research. Future work must focus on developing more resilient RAG pipelines, robust content moderation for external knowledge bases, and comprehensive evaluation frameworks that encompass the full spectrum of trustworthiness dimensions for responsible RAG deployment in sensitive, real-world applications.


\label{sec:domain-specific_applications_and_practical_considerations}

\section{Domain-Specific Applications and Practical Considerations}
\label{sec:domain-specific\_applications\_\_and\_\_practical\_considerations}

\subsection{RAG in High-Stakes Domains: Healthcare and Legal}
\label{sec:7\_1\_rag\_in\_high-stakes\_domains:\_healthcare\_\_and\_\_legal}

In domains where factual accuracy, reliability, and precision are paramount, such as healthcare and legal, Retrieval-Augmented Generation (RAG) offers a critical solution to mitigate the inherent limitations of Large Language Models (LLMs), particularly their propensity for hallucination and reliance on outdated knowledge. The application of RAG in these high-stakes environments is driven by the need to optimize the interpretation of complex guidelines, enhance reasoning for critical decision-making, and improve sensitive customer or patient interactions, where errors can have severe consequences.

The foundational value of RAG in medicine is underscored by systematic evaluations. The \textit{MIRAGE} benchmark, introduced by \cite{xiong2024exb}, provides the first comprehensive assessment of RAG systems in medicine, demonstrating an accuracy improvement of up to 18\\% for LLMs and enabling smaller models like GPT-3.5 and Mixtral to achieve performance comparable to unaugmented GPT-4. This work also identified challenges such as the "lost-in-the-middle" phenomenon, where LLMs struggle to utilize relevant information if it is not optimally positioned within the context. Complementing this, a systematic review and meta-analysis by \cite{liu2025p6t} further validates RAG's overall positive impact in biomedicine, reporting a 1.35 odds ratio increase in performance compared to baseline LLMs. Similarly, \cite{bora20242mq} highlights RAG's crucial role in enhancing the performance of LLMs for medical chatbot applications, particularly in resource-constrained environments.

To achieve the high accuracy and reliability demanded by clinical settings, researchers have focused on meticulous data preparation, domain-specific fine-tuning, and advanced architectural designs. \cite{kresevic2024uel} demonstrated the critical importance of converting complex, non-textual clinical guideline elements (e.g., tables from images) into structured text and employing advanced prompt engineering, achieving a remarkable 99\\% accuracy in interpreting hepatological guidelines with GPT-4 Turbo. This highlights that data quality and presentation are often more impactful than few-shot learning. Building on this, \cite{ke20248bm} developed an optimized RAG pipeline for preoperative medicine, where GPT-4-RAG achieved 91.4\\% accuracy, non-inferior to human experts, while significantly reducing response time. \cite{ke2025wm0} extended this by evaluating RAG's generalizability across ten different LLMs for medical fitness assessments, showing that top-performing RAG-augmented models (e.g., GPT-4) achieved superior accuracy (96.4\\%) and consistency compared to human evaluators, with a near-absence of hallucinations. Further enhancing retrieval, \cite{lee20240to} introduced a dual RAG system for diabetes guidelines, optimizing both dense and sparse retrievers across Korean and English languages, demonstrating effective cross-regional capability. In radiology, \cite{weinert2025cxo} developed a radiology-specific RAG system using a vector database of 3,689 articles, significantly improving examination scores for GPT-4 (81.2\\% vs. 75.5\\%) and Command R+ (70.3\\% vs. 62.0\\%), showcasing the power of domain-specific corpus creation and retrieval for specialized medical knowledge tasks.

Advanced RAG architectures are also crucial for robust medical reasoning. \cite{jeong2024cey} proposed \textit{Self-BioRAG}, which enhances medical reasoning through domain-specific instruction sets, a specialized retriever, and a critic LM for self-reflection, showing significant improvements in multi-choice and long-form medical QA. This addresses the limitations of general Self-RAG in specialized contexts. Integrating structured knowledge, \cite{soman2023m86} developed a Knowledge Graph-based RAG (KG-RAG) framework for biomedicine, leveraging the SPOKE KG to generate token-optimized and robust responses, achieving over 50\\% reduction in token consumption and significantly improving accuracy and robustness to prompt perturbations compared to traditional Cypher-RAG. Similarly, \cite{liu2025rz6} utilized a knowledge graph-based RAG with global search to detect emergencies in patient portal messages, achieving an accuracy of 0.99, showcasing the power of structured knowledge for critical triage. \cite{hammane2024hdb} further explored self-evaluation in RAG for medical reasoning, leveraging real-time clinical records to generate precise and informed responses. While not exclusively medical, the \textit{G-Retriever} by \cite{he20248lp}, which formulates subgraph retrieval as a Prize-Collecting Steiner Tree problem for textual graphs, offers a promising approach for handling complex, interconnected medical records by mitigating hallucination and improving scalability. Similarly, \cite{bechard2024834}'s work on reducing hallucination in structured outputs (e.g., JSON workflows) via a fine-tuned retriever is highly relevant for generating structured reports or interpreting structured data from Electronic Health Records (EHRs).

RAG's practical value extends to various clinical and patient-facing applications. For \textbf{clinical decision support and administrative automation}, \cite{unlu2024yc8} introduced \textit{RECTIFIER}, a RAG-enabled GPT-4 system for clinical trial screening from unstructured EHR notes, which outperformed human study staff in accuracy and efficiency, streamlining a labor-intensive process. \cite{tozuka2024nau} demonstrated that a RAG-LLM (NotebookLM) successfully performed lung cancer staging with 86\\% diagnostic accuracy, outperforming GPT-4o and providing highly accurate reference locations. In \textbf{patient interaction and education}, \cite{ge20237yq} developed \textit{LiVersa}, a liver disease-specific, PHI-compliant RAG chatbot, further refined and evaluated by \cite{ge20246t5}, demonstrating a proof-of-concept for secure LLM deployment in healthcare. Addressing language-specific needs, \cite{zhou20249ba} created \textit{GastroBot}, a Chinese gastrointestinal disease chatbot that utilized a fine-tuned embedding model and a specialized knowledge base, achieving high context recall (95\\%) and faithfulness (93.73\\%). \cite{xu2024w5j} showed that a RAG-GPT system for breast cancer nursing care significantly improved response accuracy and overall patient satisfaction over direct GPT-4, without compromising empathy. In \textbf{public health and emergency settings}, \cite{ghadban2023j9e} applied RAG to build \textit{SMARThealth GPT} for frontline health worker education in low- and middle-income countries, emphasizing traceability, scalability, and adaptability to local guidelines. \cite{yazaki20245js} found that RAG-enhanced LLMs significantly improved emergency patient triage accuracy (70\\% correct rate) compared to human EMTs and physicians, while substantially reducing under-triage rates.

The principles of RAG's success in healthcare translate directly to the \textbf{legal domain}, where the need for factual accuracy, precise citation, and the interpretation of complex, often lengthy, documents is equally critical. Legal texts are characterized by their specialized jargon, intricate logical structures, and the severe consequences of misinterpretation or hallucination. \cite{yang20248km} introduced \textit{CaseGPT}, a framework leveraging LLMs and RAG for case-based reasoning in both healthcare and legal sectors. In the legal context, CaseGPT enables fuzzy searches based on imprecise descriptions, improving the searchability of legal precedents and generating insightful recommendations for case strategy formulation, demonstrating RAG's ability to enhance legal research and decision-making.

A significant challenge in legal RAG is the precise retrieval of relevant information, which is paramount for accurate citation and avoiding context window overflow. \cite{pipitone2024sfx} addressed this by introducing \textit{LegalBench-RAG}, the first dedicated benchmark for evaluating the \textit{retrieval component} of RAG systems in the legal domain. Unlike previous benchmarks like LegalBench, which focused on LLM generation given pre-selected context, LegalBench-RAG emphasizes extracting \textit{minimal, highly relevant text snippets} (character-level spans) from legal documents. This meticulous approach, derived by tracing back contexts to original sources, directly tackles issues of LLM hallucination due to irrelevant information and the inability to generate precise citations. The benchmark, comprising 6,858 human-annotated query-answer pairs over 79 million characters, provides a robust framework for developing RAG systems that can handle the granular precision required in legal practice.

Despite these advancements across both domains, critical considerations remain for RAG deployment in high-stakes environments. A significant concern is \textbf{privacy}, as highlighted by \cite{zeng2024dzl}, who revealed that RAG systems are vulnerable to leakage of sensitive data from their external retrieval databases through targeted prompting attacks. This necessitates robust privacy-preserving designs, especially when handling Protected Health Information (PHI) in medicine or confidential client information in law. Both domains also share challenges in constructing and maintaining high-quality, unbiased datasets for comprehensive benchmarking, and the need for continuous knowledge updates to keep pace with evolving guidelines, laws, and medical research.

In conclusion, RAG has demonstrated transformative potential in both healthcare and legal domains, moving beyond theoretical promise to deliver tangible improvements in accuracy, efficiency, and reliability across diverse applications. From optimizing clinical guideline interpretation and patient triage to enhancing legal research and case reasoning, RAG's ability to ground LLM responses in verifiable, up-to-date knowledge is invaluable. However, the journey towards widespread, safe, and ethical deployment requires continued research into robust evaluation metrics that capture domain-specific nuances, advanced privacy-preserving RAG architectures, and seamless integration with existing professional workflows, particularly for handling multimodal data and ensuring continuous knowledge updates while mitigating risks like the "lost-in-the-middle" phenomenon.
\subsection{RAG for Structured Output and Specialized Tasks}
\label{sec:7\_2\_rag\_for\_structured\_output\_\_and\_\_specialized\_tasks}

Large Language Models (LLMs) excel at generating free-form text, but their application in scenarios demanding precise, constrained, and structured outputs, such as JSON objects, or for highly specialized tasks like zero-shot slot filling, presents unique challenges. These applications necessitate high fidelity and strict adherence to predefined schemas, where hallucinations or deviations from format are unacceptable. Retrieval-Augmented Generation (RAG) offers a powerful paradigm to address these issues by grounding LLM generation in authoritative, structured external knowledge.

Early work demonstrated RAG's potential for precise information extraction in specialized contexts. For instance, \cite{glass2021qte} pioneered robust Retrieval Augmented Generation for zero-shot slot filling, a task requiring the extraction of specific values for predefined slots from documents. Their Knowledge Graph Induction (KGI) system, which combined a Dense Passage Retriever (DPR) and a RAG model, introduced \textbf{Dense Negative Sampling (DNS)} for DPR training. This innovation significantly boosted retrieval performance, leading to substantial improvements in slot filler quality and demonstrating RAG's capability for constrained generation by precisely identifying and extracting relevant information.

Building on the foundation of robust retrieval, subsequent research has extended RAG's utility to generating complex structured outputs. \cite{bechard2024834} directly tackled the challenge of reducing hallucination in structured JSON outputs, particularly for converting natural language requirements into executable workflows. They proposed a RAG-based system that fine-tuned a \textbf{domain-specific siamese transformer encoder} to retrieve relevant JSON workflow steps and database tables. This specialized retriever, trained with contrastive loss, proved crucial for aligning unstructured queries with structured JSON objects, outperforming larger, off-the-shelf general-purpose encoders. By explicitly prepending these retrieved JSON objects to the LLM prompt, their method drastically reduced hallucination, enabling smaller LLMs to achieve performance comparable to much larger models without RAG, while ensuring output adherence to predefined schemas.

Beyond structured JSON, RAG has proven versatile for other specialized tasks. \cite{fayyazi2023qg6} explored RAG's application in cybersecurity for analyzing Tactics, Techniques, and Procedures (TTPs). Their study compared Supervised Fine-Tuning (SFT) with RAG for comprehending and summarizing TTPs, finding that RAG with decoder-only LLMs performed better when directly relevant context was extracted. However, they noted that even with RAG, decoder-only models could still suffer from low precision (hallucinations) during the decoding phase, highlighting the persistent challenge of ensuring factual accuracy in specialized generation. This observation underscores the need for further enhancements in how LLMs utilize retrieved context.

Addressing this broader challenge of LLM utilization, \cite{xu2024397} introduced \textbf{InFO-RAG (Information Refinement Training)}, an unsupervised method designed to optimize LLMs for RAG by training them to act as "Information Refiners." This approach aims to teach LLMs to consistently integrate knowledge from retrieved texts (regardless of their initial quality) with their internal parameters to generate more concise, accurate, and complete outputs. While not exclusively focused on structured output, InFO-RAG demonstrated performance improvements across various tasks, including slot-filling. This meta-improvement is critical as it enhances the LLM's ability to leverage retrieved context effectively, thereby mitigating the hallucination issues observed in tasks like TTP analysis \cite{fayyazi2023qg6} and further solidifying the precision of structured outputs as seen in JSON generation \cite{bechard2024834} and slot filling \cite{glass2021qte}.

In conclusion, RAG has emerged as a vital tool for augmenting LLMs in generating structured outputs and performing specialized tasks. The progression from robust retrieval for basic slot filling to fine-tuned retrievers for complex JSON generation, and the ongoing efforts to improve LLM's intrinsic ability to refine retrieved information, demonstrate a clear trajectory towards more precise and constrained generation. Key challenges remain in ensuring perfect adherence to complex schemas under imperfect retrieval conditions and in further integrating structured reasoning capabilities directly into the generation process to eliminate residual hallucinations.
\subsection{The Interplay of RAG and Expanded LLM Context Windows}
\label{sec:7\_3\_the\_interplay\_of\_rag\_\_and\_\_exp\_and\_ed\_llm\_context\_windows}

The landscape of Large Language Models (LLMs) is undergoing a rapid transformation, characterized by a fascinating interplay between Retrieval-Augmented Generation (RAG) and the dramatic expansion of native LLM context windows. While RAG has established itself as a critical technique for grounding LLMs in factual, up-to-date, and domain-specific knowledge, the advent of models capable of processing millions of tokens natively necessitates a re-evaluation of their respective strengths and complementary roles.

Initially, RAG gained prominence by effectively addressing fundamental limitations of standalone LLMs, such as factual inaccuracies, knowledge cut-offs, and the inability to incorporate real-time or proprietary information \cite{liu2025p6t}. Its proven efficacy in providing explicit provenance and integrating easily updatable knowledge bases has made it invaluable for applications requiring verifiability and dynamic content across various domains. For instance, in radiology, RAG has been shown to significantly improve LLM performance on domain-specific examinations, providing transparent and citable information retrieval, especially for questions directly sourced from its knowledge base \cite{weinert2025cxo}.

However, recent breakthroughs in foundational LLM architectures, exemplified by models like Gemini 1.5, introduce an unprecedented capability: native context windows extending up to 10 million tokens \cite{amugongo202530u}. This allows LLMs to process entire books, extensive codebases, or hours of multimodal data (text, video, audio) within a single prompt, demonstrating impressive recall over these vast inputs \cite{amugongo202530u}. For certain long-document tasks where all relevant information can fit within this massive window, such as summarizing a single lengthy report or translating a language from an in-context grammar manual, the immediate need for external retrieval might appear diminished, as the model can directly "read" and reason over the full context provided.

Despite these impressive advancements, simply expanding the context window does not fully negate the need for RAG, and in fact, introduces new challenges. A critical limitation of large context windows is the "lost in the middle" problem, where LLMs struggle to recall information located in the central parts of a vast input context \cite{zhao20248wm, yue2024ump}. Furthermore, studies indicate that merely increasing the quantity of information within the context window can lead to performance plateaus or even declines due to increased noise and distraction, hindering the LLM's ability to locate and utilize relevant details \cite{yue2024ump}. The computational cost associated with processing multi-million token contexts for every query also remains a significant practical consideration \cite{li2024wff}.

In light of these nuances, RAG retains several enduring and distinct advantages that position it as a complementary, rather than replaceable, approach. Firstly, while 10 million tokens is substantial, it remains finite. Truly massive, enterprise-scale knowledge bases, encompassing petabytes of data across various formats (e.g., legal precedents, financial reports, AWS DevOps guides, scientific literature), can still vastly exceed even these expanded context windows \cite{guinet2024vkg}. For such scenarios, RAG's indexing and retrieval mechanisms remain essential for efficient and targeted information access. Secondly, RAG excels in handling dynamic knowledge. Updating a RAG knowledge base is typically a matter of re-indexing documents, a process far simpler, faster, and more cost-effective than repeatedly retraining or re-embedding a massive context window for every change in guidelines, new research, or evolving enterprise data \cite{ghadban2023j9e}. Thirdly, RAG inherently provides explicit provenance, linking generated answers directly to source documents. This verifiability is critical in high-stakes domains like medicine or finance, where trust, accountability, and auditability are paramount \cite{soman2023m86, weinert2025cxo}. Lastly, RAG can offer significant cost-efficiency. For many queries, retrieving and processing a few highly relevant chunks is substantially cheaper than feeding an entire multi-million token document to an LLM for every interaction \cite{li2024wff, bechard2024834}. Empirical comparisons have shown that while long-context LLMs can outperform RAG when sufficiently resourced, RAG maintains a distinct advantage in terms of significantly lower cost \cite{li2024wff}. Moreover, research suggests that "choosing the right retrieval algorithms often leads to bigger performance gains than simply using a larger language model" \cite{guinet2024vkg}, underscoring the continued importance of effective retrieval strategies.

The future likely lies in a synergistic relationship, where RAG and expanded context windows complement each other. Hybrid approaches are emerging that leverage the strengths of both. For instance, \texttt{Self-Route} dynamically routes queries to either RAG or a long-context LLM based on self-reflection, significantly reducing computation cost while maintaining comparable performance \cite{li2024wff}. \texttt{LongRAG} is a dual-perspective RAG paradigm specifically designed for Long-Context Question Answering (LCQA) that mitigates the "lost in the middle" issue, demonstrating significant performance gains over standalone long-context LLMs \cite{zhao20248wm}. Furthermore, strategies like Demonstration-based RAG (DRAG) and Iterative Demonstration-based RAG (IterDRAG) show how RAG can be scaled to effectively utilize long contexts, even with models like Gemini 1.5 Flash, by learning to extract relevant information and construct reasoning chains, thereby overcoming the limitations of simply increasing document quantity \cite{yue2024ump}. These advancements suggest that RAG can leverage larger context windows for more sophisticated multi-document synthesis or advanced retrieval strategies, while continuing to provide the scale, dynamism, verifiability, and cost-efficiency that even the most expansive native contexts cannot fully address.


\label{sec:conclusion_and_future_directions}

\section{Conclusion and Future Directions}
\label{sec:conclusion\_\_and\_\_future\_directions}

\subsection{Current Challenges and Limitations}
\label{sec:8\_1\_current\_challenges\_\_and\_\_limitations}

Despite the rapid advancements in Retrieval-Augmented Generation (RAG) systems, several significant challenges and persistent hurdles impede their widespread and reliable deployment. These encompass the inherent trade-off between increasing architectural complexity and computational overhead, the persistent difficulty in constructing and maintaining high-quality, unbiased datasets for comprehensive benchmarking, the critical tension in balancing utility with robust privacy and security safeguards, and the ongoing difficulties in ensuring explainability and mitigating bias in retrieved and generated content.

A primary challenge lies in the inherent trade-off between increasing architectural complexity and computational overhead. While foundational RAG models \cite{lewis2020pwr} established the paradigm of augmenting Large Language Models (LLMs) with external knowledge, subsequent research has introduced increasingly sophisticated, multi-stage, and specialized architectures to enhance robustness and precision. This pursuit of higher performance often comes at a direct cost of increased computational demands and latency. For instance, multi-stage frameworks like Corrective RAG (CRAG) \cite{yan202437z} dynamically assess retrieval quality and trigger corrective actions such as query refinement or web search. Similarly, RQ-RAG \cite{chan2024u69} iteratively refines queries through rewriting and decomposition. While these iterative and adaptive mechanisms significantly improve retrieval effectiveness and factual consistency, they inherently introduce additional sequential processing steps, leading to increased inference time and computational load, which can be prohibitive for real-time or high-throughput applications. Furthermore, the integration of structured knowledge, as seen in GraphRAG approaches like G-Retriever \cite{he20248lp} and GRAG \cite{hu2024eyw}, involves complex subgraph retrieval and Graph Neural Network (GNN) processing \cite{mavromatis2024ml9}. These methods, while offering richer context and enhanced reasoning capabilities, demand specialized infrastructure and higher computational budgets. Even efforts to streamline the RAG process, such as unifying ranking and generation into a single LLM as proposed by RankRAG \cite{yu202480d}, still involve a substantial reranking step that adds processing time, especially with large document sets. The emergence of dedicated system-level optimizations like PipeRAG \cite{jiang20243ac} for accelerating periodic retrievals and RAGCache \cite{jin20247cr} for optimizing memory and computation by caching Key-Value tensors, explicitly underscores that architectural complexity is a persistent practical bottleneck requiring continuous engineering efforts.

The persistent difficulty in constructing and maintaining high-quality, unbiased datasets for comprehensive benchmarking, coupled with inherent limitations in fully simulating real-world RAG complexities, remains a significant hurdle. Evaluation of RAG systems critically relies on benchmarks, yet creating datasets that accurately reflect real-world scenarios and avoid bias is profoundly challenging. Early benchmarks like RGB \cite{chen2023nzb} revealed fundamental weaknesses in LLMs' ability to handle noisy documents and integrate knowledge, but acknowledged the inherent difficulty of fully simulating real-world complexities. This limitation is further highlighted by specialized benchmarks such as MIRAGE \cite{xiong2024exb} for the medical domain and MultiHop-RAG \cite{tang2024i5r} for complex multi-hop queries, both of which expose significant gaps in current RAG systems' reasoning capabilities and the ongoing difficulty in creating truly comprehensive and unbiased evaluation datasets. The increasing reliance on LLM-generated data for scaling benchmarks, as seen in RAGBench \cite{friel20241ct} and FRAMES \cite{krishna2024qsh}, introduces the risk of perpetuating biases inherent in the generative models themselves, potentially leading to synthetic "ground truth" that may not fully capture human-level complexity or factual nuance. Moreover, real-world data often presents in highly unstructured formats, including raw text and tables embedded within HTML or PDF documents, which poses significant parsing and retrieval challenges, as highlighted by the UDA benchmark \cite{hui2024tsz}. The meticulous data reformatting required for applications like interpreting clinical guidelines \cite{kresevic2024uel} underscores that raw data quality, noise, and heterogeneity are not merely pre-processing steps but fundamental challenges that directly impact retrieval effectiveness and downstream generation performance. This gap between benchmark performance and real-world efficacy means that models optimized on synthetic or overly clean data may fail to generalize in deployment, leading to unreliable systems and a lack of trust.

Another crucial challenge is the ongoing tension in balancing utility with robust privacy and security safeguards in deployment. RAG systems derive their utility from dynamically accessing and integrating external, often sensitive, data. However, this very mechanism introduces unique attack vectors and significant privacy risks. \cite{zeng2024dzl} critically explores privacy issues in RAG, revealing significant vulnerabilities to data leakage from external retrieval databases through targeted prompting attacks. This finding is paramount for real-world applications, especially in sensitive domains like healthcare, where Protected Health Information (PHI) compliance is non-negotiable. The development of PHI-compliant RAG systems, as demonstrated by \cite{ge20237yq}, highlights the substantial engineering effort required to mitigate these risks. The core tension lies in maximizing RAG's utility by providing access to comprehensive knowledge while simultaneously minimizing the exposure of sensitive information. Overly restrictive privacy measures might limit the breadth of knowledge RAG can leverage, impacting its overall effectiveness, whereas insufficient safeguards pose severe regulatory, ethical, and reputational risks.

Finally, ensuring explainability and mitigating bias in retrieved and generated content remains a significant limitation. LLMs are inherently black boxes, and while RAG aims to improve transparency by providing explicit provenance (the retrieved documents), the LLM's internal processing of this context remains opaque. \cite{chen2023nzb} critically observed that LLMs often trust incorrect retrieved information even when explicitly warned, indicating a fundamental lack of robust internal fact-checking and critical reasoning over provided context. This challenges the notion that RAG inherently makes LLMs more "explainable." Bias can infiltrate RAG systems at multiple stages: from biased external knowledge bases and retrieval algorithms to the LLM's own pre-trained biases, which can be amplified or reinforced by retrieved problematic content. \cite{xu2024397} highlights a crucial "information refinement" gap, where LLMs struggle to effectively utilize retrieved information, sometimes ignoring relevant context or being misled by irrelevant passages. This suggests that simply providing context is insufficient; the LLM needs to be robustly capable of discerning, prioritizing, and integrating information without introducing or amplifying bias. Furthermore, the inadequacy of traditional quantitative metrics like BLEU and ROUGE for accurately reflecting factual correctness and bias in critical domains \cite{kresevic2024uel} complicates objective evaluation. This necessitates continuous human review and the integration of explicit safety features, moderation, and bias evaluation checks \cite{ghadban2023j9e} throughout the RAG pipeline, adding significant operational overhead and making comprehensive assessment difficult.

In conclusion, despite rapid advancements, RAG systems face multifaceted challenges spanning architectural efficiency, data quality and representativeness, privacy and security vulnerabilities, and the critical need for improved explainability and bias mitigation. Addressing these persistent hurdles requires continued innovation in designing more efficient and adaptive architectures, developing rigorous and unbiased evaluation methodologies that reflect real-world complexities, and integrating robust privacy-preserving mechanisms and ethical considerations into the core RAG pipeline.
\subsection{Emerging Trends and Open Research Questions}
\label{sec:8\_2\_emerging\_trends\_\_and\_\_open\_research\_questions}

The landscape of Retrieval-Augmented Generation (RAG) is undergoing a profound transformation, moving beyond foundational architectures towards more intelligent, adaptive, and integrated systems. This evolution is characterized by several converging trends that collectively point towards a future where RAG systems are not merely augmenters but sophisticated reasoning agents capable of interacting with diverse knowledge modalities. Simultaneously, this trajectory introduces critical open research questions that will shape the field's next generation. Comprehensive surveys by \cite{huang2024a59} and \cite{fan2024pf1} provide overarching frameworks for understanding these advancements and future directions.

One primary trajectory is the \textbf{holistic integration of diverse knowledge sources}, extending RAG far beyond its initial text-centric focus. As discussed in Section 5, while early RAG systems primarily leveraged unstructured text, the field is rapidly converging towards incorporating structured knowledge, multimodal data, and even the LLM's internal parametric memory. This trend is particularly evident in the burgeoning area of Graph-Augmented RAG (GraphRAG), which seeks to imbue LLMs with enhanced reasoning capabilities over interconnected data. Innovations range from pioneering approaches like G-Retriever \cite{he20248lp} for general textual graphs and dual-level KGs for customer service \cite{xu202412d}, to advanced techniques such as GNN-RAG \cite{mavromatis2024ml9} for dense subgraph reasoning and GRAG \cite{hu2024eyw} with its dual-view prompting. The drive for deeper, more faithful reasoning is further exemplified by Think-on-Graph 2.0 \cite{ma2024pwd}, which introduces a tight-coupling iterative exploration between knowledge graphs and unstructured text, mimicking human-like problem-solving by continuously digging into topics. This integration is crucial for domains requiring high factual consistency and complex relational understanding, as seen in biomedical problem-solving with KRAGEN \cite{matsumoto2024b7a} and the use of ontologies for domain-specific RAG \cite{debellis2024bv0}. Beyond structured text, the nascent integration of multimodal data (Section 5.1) is also gaining traction, with studies exploring multimodal retrieval for visual question answering \cite{wang20248gm}. Furthermore, the exploration of causal graphs in RAG \cite{samarajeewa20241p6} highlights a move towards augmenting LLMs with explicit causal relationships, enabling more robust causal reasoning. These efforts collectively aim to provide LLMs with a richer, more interconnected understanding of the world, moving beyond simple semantic similarity to encompass relational, visual, and even causal knowledge.

A second significant trend is the \textbf{development of more autonomous and self-aware RAG agents capable of complex reasoning}. This represents a shift from passive augmentation to active, adaptive, and intelligent information-seeking. As explored in Sections 3.2, 4.1, and 4.2, RAG systems are increasingly equipped with mechanisms that allow the LLM to dynamically control the retrieval process. This includes corrective strategies like CRAG \cite{yan202437z} that dynamically assess retrieval quality, proactive query refinement methods such as RQ-RAG \cite{chan2024u69}, and architectural simplifications like RankRAG \cite{yu202480d} that unify components. The frontier lies in truly agentic RAG systems, exemplified by IM-RAG \cite{yang20243nb} with its learned inner monologues and multi-agent reinforcement learning for flexible multi-round retrieval, and PlanRAG \cite{lee2024hif} which enables LLMs to generate and iteratively refine plans for complex decision-making. This agentic paradigm is not limited to general tasks; Self-BioRAG \cite{jeong2024cey} demonstrates its application in specialized domains like medical reasoning, using self-reflection to assess evidence. The vision for these systems is to act as "search engines for machines" \cite{salemi2024bb6}, capable of understanding complex information needs, strategizing retrieval, and synthesizing information in a human-like, iterative manner. This convergence of agentic capabilities with deep knowledge integration is a key future direction, particularly in high-stakes domains \cite{liu2025p6t}.

The continuous push for \textbf{greater efficiency and scalability in dynamic environments} remains a critical area. As RAG systems grow in complexity and are deployed in real-world, high-throughput applications, optimizing latency and resource consumption becomes paramount. Section 4.3 detailed innovations such as RAGCache \cite{jin20247cr} for dynamic caching, PipeRAG \cite{jiang20243ac} for algorithm-system co-design and pipeline parallelism, and TurboRAG \cite{lu2024pvt} for offline KV cache pre-computation. These system-level optimizations are crucial for ensuring RAG's practical viability, especially in scenarios where knowledge bases are constantly updated (e.g., medical guidelines \cite{ke20248bm, ke2025wm0, ghadban2023j9e}) and real-time performance is expected.

Despite these exciting trends, several \textbf{critical open research questions} demand attention. One major question is \textbf{how RAG's role will adapt and be redefined in an era of ever-larger LLM context windows}. With models like Gemini 1.5 Pro \cite{amugongo202530u} capable of processing millions of tokens across multimodal inputs, the traditional function of RAG in providing concise textual snippets (as discussed in Section 7.3) may shift. RAG's enduring value will likely pivot towards providing explicit provenance for generated content, facilitating real-time updates for truly massive and dynamic knowledge bases that still exceed even expanded context limits, and addressing the "lost in the middle" problem in long contexts \cite{zhao20248wm}. The interplay between RAG and large context windows is thus not one of replacement, but rather a complementary relationship where RAG provides a verifiable, dynamic, and potentially more cost-effective layer for specific knowledge needs.

Another pressing question concerns the \textbf{need for more sophisticated reasoning capabilities} within RAG systems. While GraphRAG and agentic approaches are making strides, achieving human-level complex reasoning, particularly across integrated multimodal and structured data, remains a significant challenge. Benchmarks (Section 6.2) consistently highlight limitations in multi-hop information integration, noise robustness, and counterfactual reasoning \cite{tang2024i5r, krishna2024qsh, chen2023nzb}. The future demands RAG systems that can not only retrieve relevant facts but also synthesize information from disparate sources, resolve conflicting evidence, and engage in deep inferential reasoning, including specialized forms like causal reasoning \cite{samarajeewa20241p6}. This requires substantial architectural and algorithmic advancements that can leverage the rich, interconnected knowledge structures now being integrated.

Finally, the \textbf{development of universally accepted, explainable, and robust evaluation metrics} that can capture the multifaceted performance of advanced RAG systems is paramount. As RAG architectures become more complex and autonomous, traditional metrics for retrieval relevance or generation accuracy (Section 6.1) become insufficient. There is a recognized need for more nuanced and explainable metrics that can diagnose \textit{why} a RAG system succeeds or fails. Approaches like eRAG \cite{salemi2024om5} evaluate retrieval utility to the LLM, while RAGBench \cite{friel20241ct} introduces explainable metrics like "Context Utilization" and "Completeness." The emergence of benchmarks like CRUD-RAG \cite{lyu2024ngu} for diverse Chinese tasks and methods for improving LLM-based judgments through judge-consistency \cite{liu2025sy0} underscore this need. Beyond accuracy, future evaluation must also encompass critical dimensions of trustworthiness, privacy (Section 6.3), and the ability to transparently assess the decision-making processes of agentic RAG systems, providing actionable insights for continuous improvement and responsible deployment.

In conclusion, the future of RAG research is characterized by an exciting convergence of holistic knowledge integration, autonomous agentic reasoning, and relentless pursuit of efficiency. These advancements are, however, inextricably linked to fundamental open questions regarding RAG's evolving role alongside expanding LLM context windows, the persistent quest for truly sophisticated reasoning, and the imperative for comprehensive, explainable, and robust evaluation methodologies to guide its development and ensure its responsible and effective application.


\newpage
\section*{References}
\addcontentsline{toc}{section}{References}

\begin{thebibliography}{211}

\bibitem{lewis2020pwr}
Patrick Lewis, Ethan Perez, Aleksandara Piktus, et al. (2020). \textit{Retrieval-Augmented Generation for Knowledge-Intensive NLP Tasks}. Neural Information Processing Systems.

\bibitem{komeili20215so}
M. Komeili, Kurt Shuster, and J. Weston (2021). \textit{Internet-Augmented Dialogue Generation}. Annual Meeting of the Association for Computational Linguistics.

\bibitem{chen2022j8c}
Wenhu Chen, Hexiang Hu, Xi Chen, et al. (2022). \textit{MuRAG: Multimodal Retrieval-Augmented Generator for Open Question Answering over Images and Text}. Conference on Empirical Methods in Natural Language Processing.

\bibitem{agarwal2021e31}
Oshin Agarwal, Heming Ge, Siamak Shakeri, et al. (2021). \textit{Knowledge Graph Based Synthetic Corpus Generation for Knowledge-Enhanced Language Model Pre-training}. North American Chapter of the Association for Computational Linguistics.

\bibitem{gui2021zw6}
Liangke Gui, Borui Wang, Qiuyuan Huang, et al. (2021). \textit{KAT: A Knowledge Augmented Transformer for Vision-and-Language}. North American Chapter of the Association for Computational Linguistics.

\bibitem{masanneck2014fk3}
L. Masanneck, Sven G. Meuth, and M. Pawlitzki (2014). \textit{Evaluating base and retrieval augmented LLMs with document or online support for evidence based neurology}. The Lancet.

\bibitem{sun2022hx2}
Zhiqing Sun, Xuezhi Wang, Yi Tay, et al. (2022). \textit{Recitation-Augmented Language Models}. International Conference on Learning Representations.

\bibitem{sarto2022nxs}
Sara Sarto, Marcella Cornia, L. Baraldi, et al. (2022). \textit{Retrieval-Augmented Transformer for Image Captioning}. International Conference on Content-Based Multimedia Indexing.

\bibitem{shi20222ui}
Ensheng Shi, Yanlin Wang, Wei Tao, et al. (2022). \textit{RACE: Retrieval-augmented Commit Message Generation}. Conference on Empirical Methods in Natural Language Processing.

\bibitem{chowdhury20228rz}
Jishnu Ray Chowdhury, Yong Zhuang, and Shuyi Wang (2022). \textit{Novelty Controlled Paraphrase Generation with Retrieval Augmented Conditional Prompt Tuning}. AAAI Conference on Artificial Intelligence.

\bibitem{xu2021slt}
Yan Xu, Etsuko Ishii, Zihan Liu, et al. (2021). \textit{Retrieval-Free Knowledge-Grounded Dialogue Response Generation with Adapters}. Workshop on Document-grounded Dialogue and Conversational Question Answering.

\bibitem{adolphs20219au}
Leonard Adolphs, Kurt Shuster, Jack Urbanek, et al. (2021). \textit{Reason first, then respond: Modular Generation for Knowledge-infused Dialogue}. Conference on Empirical Methods in Natural Language Processing.

\bibitem{dixit2022xid}
Tanay Dixit, Bhargavi Paranjape, Hannaneh Hajishirzi, et al. (2022). \textit{CORE: A Retrieve-then-Edit Framework for Counterfactual Data Generation}. Conference on Empirical Methods in Natural Language Processing.

\bibitem{glass2021qte}
Michael R. Glass, Gaetano Rossiello, Md. Faisal Mahbub Chowdhury, et al. (2021). \textit{Robust Retrieval Augmented Generation for Zero-shot Slot Filling}. Conference on Empirical Methods in Natural Language Processing.

\bibitem{agarwal2020c3x}
Oshin Agarwal, Heming Ge, Siamak Shakeri, et al. (2020). \textit{Large Scale Knowledge Graph Based Synthetic Corpus Generation for Knowledge-Enhanced Language Model Pre-training}. arXiv.org.

\bibitem{pan2022u7w}
Feifei Pan, Mustafa Canim, Michael R. Glass, et al. (2022). \textit{End-to-End Table Question Answering via Retrieval-Augmented Generation}. arXiv.org.

\bibitem{akbar202053c}
Shayan A. Akbar, and A. Kak (2020). \textit{A Large-Scale Comparative Evaluation of IR-Based Tools for Bug Localization}. IEEE Working Conference on Mining Software Repositories.

\bibitem{pappas20223ck}
Dimitris Pappas, Prodromos Malakasiotis, and Ion Androutsopoulos (2022). \textit{Data Augmentation for Biomedical Factoid Question Answering}. Workshop on Biomedical Natural Language Processing.

\bibitem{kim202056z}
Jihyeok Kim, Seungtaek Choi, Reinald Kim Amplayo, et al. (2020). \textit{Retrieval-Augmented Controllable Review Generation}. International Conference on Computational Linguistics.

\bibitem{gao20238ea}
Yunfan Gao, Yun Xiong, Xinyu Gao, et al. (2023). \textit{Retrieval-Augmented Generation for Large Language Models: A Survey}. arXiv.org.

\bibitem{fan2024pf1}
Wenqi Fan, Yujuan Ding, Liang-bo Ning, et al. (2024). \textit{A Survey on RAG Meeting LLMs: Towards Retrieval-Augmented Large Language Models}. Knowledge Discovery and Data Mining.

\bibitem{xiong2024exb}
Guangzhi Xiong, Qiao Jin, Zhiyong Lu, et al. (2024). \textit{Benchmarking Retrieval-Augmented Generation for Medicine}. Annual Meeting of the Association for Computational Linguistics.

\bibitem{chen2023nzb}
Jiawei Chen, Hongyu Lin, Xianpei Han, et al. (2023). \textit{Benchmarking Large Language Models in Retrieval-Augmented Generation}. AAAI Conference on Artificial Intelligence.

\bibitem{peng2024mp3}
Boci Peng, Yun Zhu, Yongchao Liu, et al. (2024). \textit{Graph Retrieval-Augmented Generation: A Survey}. arXiv.org.

\bibitem{tang2024i5r}
Yixuan Tang, and Yi Yang (2024). \textit{MultiHop-RAG: Benchmarking Retrieval-Augmented Generation for Multi-Hop Queries}. arXiv.org.

\bibitem{he20248lp}
Xiaoxin He, Yijun Tian, Yifei Sun, et al. (2024). \textit{G-Retriever: Retrieval-Augmented Generation for Textual Graph Understanding and Question Answering}. Neural Information Processing Systems.

\bibitem{xu202412d}
Zhentao Xu, Mark Jerome Cruz, Matthew Guevara, et al. (2024). \textit{Retrieval-Augmented Generation with Knowledge Graphs for Customer Service Question Answering}. Annual International ACM SIGIR Conference on Research and Development in Information Retrieval.

\bibitem{yan202437z}
Shi-Qi Yan, Jia-Chen Gu, Yun Zhu, et al. (2024). \textit{Corrective Retrieval Augmented Generation}. arXiv.org.

\bibitem{yu202480d}
Yue Yu, Wei Ping, Zihan Liu, et al. (2024). \textit{RankRAG: Unifying Context Ranking with Retrieval-Augmented Generation in LLMs}. Neural Information Processing Systems.

\bibitem{zeng2024dzl}
Shenglai Zeng, Jiankun Zhang, Pengfei He, et al. (2024). \textit{The Good and The Bad: Exploring Privacy Issues in Retrieval-Augmented Generation (RAG)}. Annual Meeting of the Association for Computational Linguistics.

\bibitem{salemi2024om5}
Alireza Salemi, and Hamed Zamani (2024). \textit{Evaluating Retrieval Quality in Retrieval-Augmented Generation}. Annual International ACM SIGIR Conference on Research and Development in Information Retrieval.

\bibitem{chan2024u69}
Chi-Min Chan, Chunpu Xu, Ruibin Yuan, et al. (2024). \textit{RQ-RAG: Learning to Refine Queries for Retrieval Augmented Generation}. arXiv.org.

\bibitem{kresevic2024uel}
Simone Kresevic, M. Giuffré, M. Ajčević, et al. (2024). \textit{Optimization of hepatological clinical guidelines interpretation by large language models: a retrieval augmented generation-based framework}. npj Digit. Medicine.

\bibitem{mavromatis2024ml9}
Costas Mavromatis, and George Karypis (2024). \textit{GNN-RAG: Graph Neural Retrieval for Large Language Model Reasoning}. arXiv.org.

\bibitem{jin2024yhb}
Jiajie Jin, Yutao Zhu, Xinyu Yang, et al. (2024). \textit{FlashRAG: A Modular Toolkit for Efficient Retrieval-Augmented Generation Research}. The Web Conference.

\bibitem{sarmah20245f3}
Bhaskarjit Sarmah, Dhagash Mehta, Benika Hall, et al. (2024). \textit{HybridRAG: Integrating Knowledge Graphs and Vector Retrieval Augmented Generation for Efficient Information Extraction}. International Conference on AI in Finance.

\bibitem{bechard2024834}
Patrice B'echard, and Orlando Marquez Ayala (2024). \textit{Reducing hallucination in structured outputs via Retrieval-Augmented Generation}. North American Chapter of the Association for Computational Linguistics.

\bibitem{wang20248gm}
Xiaohua Wang, Zhenghua Wang, Xuan Gao, et al. (2024). \textit{Searching for Best Practices in Retrieval-Augmented Generation}. Conference on Empirical Methods in Natural Language Processing.

\bibitem{zou2024iiy}
Wei Zou, Runpeng Geng, Binghui Wang, et al. (2024). \textit{PoisonedRAG: Knowledge Poisoning Attacks to Retrieval-Augmented Generation of Large Language Models}. arXiv.org.

\bibitem{gutierrez2024al5}
Bernal Jimenez Gutierrez, Yiheng Shu, Yu Gu, et al. (2024). \textit{HippoRAG: Neurobiologically Inspired Long-Term Memory for Large Language Models}. Neural Information Processing Systems.

\bibitem{yu2024arx}
Shi Yu, Chaoyue Tang, Bokai Xu, et al. (2024). \textit{VisRAG: Vision-based Retrieval-augmented Generation on Multi-modality Documents}. International Conference on Learning Representations.

\bibitem{guo2024plq}
Zirui Guo, Lianghao Xia, Yanhua Yu, et al. (2024). \textit{LightRAG: Simple and Fast Retrieval-Augmented Generation}. arXiv.org.

\bibitem{li2024wff}
Zhuowan Li, Cheng Li, Mingyang Zhang, et al. (2024). \textit{Retrieval Augmented Generation or Long-Context LLMs? A Comprehensive Study and Hybrid Approach}. Conference on Empirical Methods in Natural Language Processing.

\bibitem{zhao2024931}
Siyun Zhao, Yuqing Yang, Zilong Wang, et al. (2024). \textit{Retrieval Augmented Generation (RAG) and Beyond: A Comprehensive Survey on How to Make your LLMs use External Data More Wisely}. arXiv.org.

\bibitem{huang2024a59}
Yizheng Huang, and Jimmy X. Huang (2024). \textit{A Survey on Retrieval-Augmented Text Generation for Large Language Models}. arXiv.org.

\bibitem{xie20245dq}
Qianqian Xie, Weiguang Han, Zhengyu Chen, et al. (2024). \textit{FinBen: A Holistic Financial Benchmark for Large Language Models}. Neural Information Processing Systems.

\bibitem{wu2024bpc}
Shangyu Wu, Ying Xiong, Yufei Cui, et al. (2024). \textit{Retrieval-Augmented Generation for Natural Language Processing: A Survey}. arXiv.org.

\bibitem{lyu2024ngu}
Yuanjie Lyu, Zhiyu Li, Simin Niu, et al. (2024). \textit{CRUD-RAG: A Comprehensive Chinese Benchmark for Retrieval-Augmented Generation of Large Language Models}. ACM Trans. Inf. Syst..

\bibitem{deng2024k1b}
Gelei Deng, Yi Liu, Kailong Wang, et al. (2024). \textit{Pandora: Jailbreak GPTs by Retrieval Augmented Generation Poisoning}. Proceedings 2024 Workshop on AI Systems with Confidential COmputing.

\bibitem{soudani20247ny}
Heydar Soudani, E. Kanoulas, and Faegheh Hasibi (2024). \textit{Fine Tuning vs. Retrieval Augmented Generation for Less Popular Knowledge}. SIGIR-AP.

\bibitem{krishna2024qsh}
Satyapriya Krishna, Kalpesh Krishna, Anhad Mohananey, et al. (2024). \textit{Fact, Fetch, and Reason: A Unified Evaluation of Retrieval-Augmented Generation}. North American Chapter of the Association for Computational Linguistics.

\bibitem{zhou20248fu}
Yujia Zhou, Yan Liu, Xiaoxi Li, et al. (2024). \textit{Trustworthiness in Retrieval-Augmented Generation Systems: A Survey}. arXiv.org.

\bibitem{pipitone2024sfx}
Nicholas Pipitone, and Ghita Houir Alami (2024). \textit{LegalBench-RAG: A Benchmark for Retrieval-Augmented Generation in the Legal Domain}. arXiv.org.

\bibitem{jin20247cr}
Chao Jin, Zili Zhang, Xuanlin Jiang, et al. (2024). \textit{RAGCache: Efficient Knowledge Caching for Retrieval-Augmented Generation}. ACM Transactions on Computer Systems.

\bibitem{wang20246hs}
Zilong Wang, Zifeng Wang, Long T. Le, et al. (2024). \textit{Speculative RAG: Enhancing Retrieval Augmented Generation through Drafting}. International Conference on Learning Representations.

\bibitem{tihanyi2024d5e}
Norbert Tihanyi, M. Ferrag, Ridhi Jain, et al. (2024). \textit{CyberMetric: A Benchmark Dataset based on Retrieval-Augmented Generation for Evaluating LLMs in Cybersecurity Knowledge}. Computer Science Symposium in Russia.

\bibitem{zou2024haa}
Wei Zou, Runpeng Geng, Binghui Wang, et al. (2024). \textit{PoisonedRAG: Knowledge Corruption Attacks to Retrieval-Augmented Generation of Large Language Models}. Unpublished manuscript.

\bibitem{xiong2024u1b}
Guangzhi Xiong, Qiao Jin, Xiao Wang, et al. (2024). \textit{Improving Retrieval-Augmented Generation in Medicine with Iterative Follow-up Questions}. Pacific Symposium on Biocomputing. Pacific Symposium on Biocomputing.

\bibitem{fang2024gh6}
Feiteng Fang, Yuelin Bai, Shiwen Ni, et al. (2024). \textit{Enhancing Noise Robustness of Retrieval-Augmented Language Models with Adaptive Adversarial Training}. Annual Meeting of the Association for Computational Linguistics.

\bibitem{hu2024eyw}
Yuntong Hu, Zhihan Lei, Zhengwu Zhang, et al. (2024). \textit{GRAG: Graph Retrieval-Augmented Generation}. North American Chapter of the Association for Computational Linguistics.

\bibitem{xue2024bxd}
Jiaqi Xue, Meng Zheng, Yebowen Hu, et al. (2024). \textit{BadRAG: Identifying Vulnerabilities in Retrieval Augmented Generation of Large Language Models}. arXiv.org.

\bibitem{jeong2024cey}
Minbyul Jeong, Jiwoong Sohn, Mujeen Sung, et al. (2024). \textit{Improving medical reasoning through retrieval and self-reflection with retrieval-augmented large language models}. Bioinform..

\bibitem{matsumoto2024b7a}
Nicholas Matsumoto, Jay Moran, Hyunjun Choi, et al. (2024). \textit{KRAGEN: a knowledge graph-enhanced RAG framework for biomedical problem solving using large language models}. Bioinformatics.

\bibitem{friel20241ct}
Robert Friel, Masha Belyi, and Atindriyo Sanyal (2024). \textit{RAGBench: Explainable Benchmark for Retrieval-Augmented Generation Systems}. arXiv.org.

\bibitem{procko202417i}
T. Procko, and Omar Ochoa (2024). \textit{Graph Retrieval-Augmented Generation for Large Language Models: A Survey}. 2024 Conference on AI, Science, Engineering, and Technology (AIxSET).

\bibitem{wang2024dt8}
Hongru Wang, Wenyu Huang, Yang Deng, et al. (2024). \textit{UniMS-RAG: A Unified Multi-source Retrieval-Augmented Generation for Personalized Dialogue Systems}. arXiv.org.

\bibitem{zhang2025gnc}
Qinggang Zhang, Shengyuan Chen, Yuan-Qi Bei, et al. (2025). \textit{A Survey of Graph Retrieval-Augmented Generation for Customized Large Language Models}. arXiv.org.

\bibitem{cheng2024d7k}
Pengzhou Cheng, Yidong Ding, Tianjie Ju, et al. (2024). \textit{TrojanRAG: Retrieval-Augmented Generation Can Be Backdoor Driver in Large Language Models}. arXiv.org.

\bibitem{yue2024ump}
Zhenrui Yue, Honglei Zhuang, Aijun Bai, et al. (2024). \textit{Inference Scaling for Long-Context Retrieval Augmented Generation}. International Conference on Learning Representations.

\bibitem{jiang20243ac}
Wenqi Jiang, Shuai Zhang, Boran Han, et al. (2024). \textit{PipeRAG: Fast Retrieval-Augmented Generation via Algorithm-System Co-design}. arXiv.org.

\bibitem{ge20246t5}
J. Ge, Steve Sun, Joseph Owens, et al. (2024). \textit{Development of a liver disease–specific large language model chat interface using retrieval-augmented generation}. Hepatology.

\bibitem{ding20249ne}
Yujuan Ding, Wenqi Fan, Liang-bo Ning, et al. (2024). \textit{A Survey on RAG Meets LLMs: Towards Retrieval-Augmented Large Language Models}. arXiv.org.

\bibitem{sun2024eoe}
ZhongXiang Sun, Xiaoxue Zang, Kai Zheng, et al. (2024). \textit{ReDeEP: Detecting Hallucination in Retrieval-Augmented Generation via Mechanistic Interpretability}. International Conference on Learning Representations.

\bibitem{ma2024pwd}
Shengjie Ma, Chengjin Xu, Xuhui Jiang, et al. (2024). \textit{Think-on-Graph 2.0: Deep and Faithful Large Language Model Reasoning with Knowledge-guided Retrieval Augmented Generation}. International Conference on Learning Representations.

\bibitem{bornea2024jde}
Andrei-Laurentiu Bornea, Fadhel Ayed, Antonio De Domenico, et al. (2024). \textit{Telco-RAG: Navigating the Challenges of Retrieval Augmented Language Models for Telecommunications}. Global Communications Conference.

\bibitem{yang20243nb}
Diji Yang, Jinmeng Rao, Kezhen Chen, et al. (2024). \textit{IM-RAG: Multi-Round Retrieval-Augmented Generation Through Learning Inner Monologues}. Annual International ACM SIGIR Conference on Research and Development in Information Retrieval.

\bibitem{su20241om}
Weihang Su, Yichen Tang, Qingyao Ai, et al. (2024). \textit{DRAGIN: Dynamic Retrieval Augmented Generation based on the Real-time Information Needs of Large Language Models}. Annual Meeting of the Association for Computational Linguistics.

\bibitem{islam2024ug5}
Shayekh Bin Islam, Md Asib Rahman, K. S. M. T. Hossain, et al. (2024). \textit{Open-RAG: Enhanced Retrieval-Augmented Reasoning with Open-Source Large Language Models}. Conference on Empirical Methods in Natural Language Processing.

\bibitem{liu2025p6t}
Siru Liu, Allison B. McCoy, and Adam Wright (2025). \textit{Improving large language model applications in biomedicine with retrieval-augmented generation: a systematic review, meta-analysis, and clinical development guidelines}. J. Am. Medical Informatics Assoc..

\bibitem{ke20248bm}
Yuhe Ke, Liyuan Jin, Kabilan Elangovan, et al. (2024). \textit{Development and Testing of Retrieval Augmented Generation in Large Language Models - A Case Study Report}. arXiv.org.

\bibitem{ni2025ox9}
Bo Ni, Zheyuan Liu, Leyao Wang, et al. (2025). \textit{Towards Trustworthy Retrieval Augmented Generation for Large Language Models: A Survey}. arXiv.org.

\bibitem{lee2024hif}
Myeonghwa Lee, Seonho An, and Min-Soo Kim (2024). \textit{PlanRAG: A Plan-then-Retrieval Augmented Generation for Generative Large Language Models as Decision Makers}. North American Chapter of the Association for Computational Linguistics.

\bibitem{li2024hb4}
Mufei Li, Siqi Miao, and Pan Li (2024). \textit{Simple is Effective: The Roles of Graphs and Large Language Models in Knowledge-Graph-Based Retrieval-Augmented Generation}. International Conference on Learning Representations.

\bibitem{ke2025wm0}
Yuhe Ke, Liyuan Jin, Kabilan Elangovan, et al. (2025). \textit{Retrieval augmented generation for 10 large language models and its generalizability in assessing medical fitness}. npj Digit. Medicine.

\bibitem{wang2024zt3}
Zheng Wang, Shu Xian Teo, Jieer Ouyang, et al. (2024). \textit{M-RAG: Reinforcing Large Language Model Performance through Retrieval-Augmented Generation with Multiple Partitions}. Annual Meeting of the Association for Computational Linguistics.

\bibitem{kang2024hrb}
Mintong Kang, Nezihe Merve Gurel, Ning Yu, et al. (2024). \textit{C-RAG: Certified Generation Risks for Retrieval-Augmented Language Models}. International Conference on Machine Learning.

\bibitem{lin2024s1v}
Demiao Lin (2024). \textit{Revolutionizing Retrieval-Augmented Generation with Enhanced PDF Structure Recognition}. arXiv.org.

\bibitem{guinet2024vkg}
Gauthier Guinet, Behrooz Omidvar-Tehrani, Anoop Deoras, et al. (2024). \textit{Automated Evaluation of Retrieval-Augmented Language Models with Task-Specific Exam Generation}. International Conference on Machine Learning.

\bibitem{radeva2024vai}
I. Radeva, I. Popchev, L. Doukovska, et al. (2024). \textit{Web Application for Retrieval-Augmented Generation: Implementation and Testing}. Electronics.

\bibitem{soman2023m86}
Karthik Soman, Peter W Rose, John H Morris, et al. (2023). \textit{Biomedical knowledge graph-optimized prompt generation for large language models}. Bioinformatics.

\bibitem{chen20245d2}
Zhanpeng Chen, Chengjin Xu, Yiyan Qi, et al. (2024). \textit{MLLM Is a Strong Reranker: Advancing Multimodal Retrieval-augmented Generation via Knowledge-enhanced Reranking and Noise-injected Training}. arXiv.org.

\bibitem{unlu2024yc8}
Ozan Unlu, Jiyeon Shin, Charlotte J. Mailly, et al. (2024). \textit{Retrieval Augmented Generation Enabled Generative Pre-Trained Transformer 4 (GPT-4) Performance for Clinical Trial Screening}. medRxiv.

\bibitem{ge20237yq}
J. Ge, Steve Sun, Joseph Owens, et al. (2023). \textit{Development of a Liver Disease-Specific Large Language Model Chat Interface using Retrieval Augmented Generation}. medRxiv.

\bibitem{rau20244nr}
David Rau, Herv'e D'ejean, Nadezhda Chirkova, et al. (2024). \textit{BERGEN: A Benchmarking Library for Retrieval-Augmented Generation}. Conference on Empirical Methods in Natural Language Processing.

\bibitem{bora20242mq}
Arunabh Bora, and H. Cuayáhuitl (2024). \textit{Systematic Analysis of Retrieval-Augmented Generation-Based LLMs for Medical Chatbot Applications}. Machine Learning and Knowledge Extraction.

\bibitem{pradeep2024n91}
Ronak Pradeep, Nandan Thakur, Sahel Sharifymoghaddam, et al. (2024). \textit{Ragnarök: A Reusable RAG Framework and Baselines for TREC 2024 Retrieval-Augmented Generation Track}. European Conference on Information Retrieval.

\bibitem{zhao20248wm}
Qingfei Zhao, Ruobing Wang, Yukuo Cen, et al. (2024). \textit{LongRAG: A Dual-Perspective Retrieval-Augmented Generation Paradigm for Long-Context Question Answering}. Conference on Empirical Methods in Natural Language Processing.

\bibitem{chirkova2024kde}
Nadezhda Chirkova, David Rau, Herv'e D'ejean, et al. (2024). \textit{Retrieval-augmented generation in multilingual settings}. KNOWLLM.

\bibitem{dong2024qcd}
Guanting Dong, Yutao Zhu, Chenghao Zhang, et al. (2024). \textit{Understand What LLM Needs: Dual Preference Alignment for Retrieval-Augmented Generation}. The Web Conference.

\bibitem{lu2024pvt}
Songshuo Lu, Hua Wang, Yutian Rong, et al. (2024). \textit{TurboRAG: Accelerating Retrieval-Augmented Generation with Precomputed KV Caches for Chunked Text}. arXiv.org.

\bibitem{zhu2024h7i}
Yun Zhu, Jia-Chen Gu, Caitlin Sikora, et al. (2024). \textit{Accelerating Inference of Retrieval-Augmented Generation via Sparse Context Selection}. International Conference on Learning Representations.

\bibitem{yu2024c32}
Tian Yu, Shaolei Zhang, and Yang Feng (2024). \textit{Auto-RAG: Autonomous Retrieval-Augmented Generation for Large Language Models}. arXiv.org.

\bibitem{amugongo202530u}
L. M. Amugongo, Pietro Mascheroni, Steven Brooks, et al. (2025). \textit{Retrieval augmented generation for large language models in healthcare: A systematic review}. PLOS Digital Health.

\bibitem{hui2024tsz}
Yulong Hui, Yao Lu, and Huanchen Zhang (2024). \textit{UDA: A Benchmark Suite for Retrieval Augmented Generation in Real-world Document Analysis}. Neural Information Processing Systems.

\bibitem{khaliq2024ne2}
M. A. Khaliq, P. Chang, M. Ma, et al. (2024). \textit{RAGAR, Your Falsehood Radar: RAG-Augmented Reasoning for Political Fact-Checking using Multimodal Large Language Models}. FEVER.

\bibitem{salemi2024bb6}
Alireza Salemi, and Hamed Zamani (2024). \textit{Towards a Search Engine for Machines: Unified Ranking for Multiple Retrieval-Augmented Large Language Models}. Annual International ACM SIGIR Conference on Research and Development in Information Retrieval.

\bibitem{xu2024397}
Shicheng Xu, Liang Pang, Mo Yu, et al. (2024). \textit{Unsupervised Information Refinement Training of Large Language Models for Retrieval-Augmented Generation}. Annual Meeting of the Association for Computational Linguistics.

\bibitem{hu2024i6h}
Zhibo Hu, Chen Wang, Yanfeng Shu, et al. (2024). \textit{Prompt Perturbation in Retrieval-Augmented Generation based Large Language Models}. Knowledge Discovery and Data Mining.

\bibitem{sohn2024w2t}
Jiwoong Sohn, Yein Park, Chanwoong Yoon, et al. (2024). \textit{Rationale-Guided Retrieval Augmented Generation for Medical Question Answering}. North American Chapter of the Association for Computational Linguistics.

\bibitem{qi2024tlf}
Zehan Qi, Rongwu Xu, Zhijiang Guo, et al. (2024). \textit{LONG²RAG: Evaluating Long-Context & Long-Form Retrieval-Augmented Generation with Key Point Recall}. Conference on Empirical Methods in Natural Language Processing.

\bibitem{han2024mpx}
Binglan Han, Teo Sušnjak, and A. Mathrani (2024). \textit{Automating Systematic Literature Reviews with Retrieval-Augmented Generation: A Comprehensive Overview}. Applied Sciences.

\bibitem{zhao2024go5}
Yiyun Zhao, Prateek Singh, Hanoz Bhathena, et al. (2024). \textit{Optimizing LLM Based Retrieval Augmented Generation Pipelines in the Financial Domain}. North American Chapter of the Association for Computational Linguistics.

\bibitem{li20243nz}
Xinze Li, Senkun Mei, Zhenghao Liu, et al. (2024). \textit{RAG-DDR: Optimizing Retrieval-Augmented Generation Using Differentiable Data Rewards}. International Conference on Learning Representations.

\bibitem{wang2024kca}
Fei Wang, Xingchen Wan, Ruoxi Sun, et al. (2024). \textit{Astute RAG: Overcoming Imperfect Retrieval Augmentation and Knowledge Conflicts for Large Language Models}. Annual Meeting of the Association for Computational Linguistics.

\bibitem{akkiraju2024edc}
Rama Akkiraju, Anbang Xu, Deepak Bora, et al. (2024). \textit{FACTS About Building Retrieval Augmented Generation-based Chatbots}. arXiv.org.

\bibitem{zhou20249ba}
Qingqing Zhou, Can Liu, Yuchen Duan, et al. (2024). \textit{GastroBot: a Chinese gastrointestinal disease chatbot based on the retrieval-augmented generation}. Frontiers in Medicine.

\bibitem{kim2024t1i}
Dongkyu Kim, Byoungwook Kim, Donggeon Han, et al. (2024). \textit{AutoRAG: Automated Framework for optimization of Retrieval Augmented Generation Pipeline}. arXiv.org.

\bibitem{yilma20249sl}
G. M. Yilma, J. Ayala-Romero, A. Garcia-Saavedra, et al. (2024). \textit{TelecomRAG: Taming Telecom Standards with Retrieval Augmented Generation and LLMs}. Computer communication review.

\bibitem{xu20242x1}
Haowen Xu, Jinghui Yuan, Anye Zhou, et al. (2024). \textit{GenAI-powered Multi-Agent Paradigm for Smart Urban Mobility: Opportunities and Challenges for Integrating Large Language Models (LLMs) and Retrieval-Augmented Generation (RAG) with Intelligent Transportation Systems}. arXiv.org.

\bibitem{xu2024dgv}
Ran Xu, Hui Liu, Sreyashi Nag, et al. (2024). \textit{SimRAG: Self-Improving Retrieval-Augmented Generation for Adapting Large Language Models to Specialized Domains}. North American Chapter of the Association for Computational Linguistics.

\bibitem{liu2024878}
Suqing Liu, Zezhu Yu, Feiran Huang, et al. (2024). \textit{Can Small Language Models With Retrieval-Augmented Generation Replace Large Language Models When Learning Computer Science?}. Annual Conference on Innovation and Technology in Computer Science Education.

\bibitem{zeng2024vmz}
Huimin Zeng, Zhenrui Yue, Qian Jiang, et al. (2024). \textit{Federated Recommendation via Hybrid Retrieval Augmented Generation}. BigData Congress [Services Society].

\bibitem{bhattarai2024zkd}
Manish Bhattarai, Javier E. Santos, Shawn Jones, et al. (2024). \textit{Enhancing Code Translation in Language Models with Few-Shot Learning via Retrieval-Augmented Generation}. IEEE Conference on High Performance Extreme Computing.

\bibitem{wang2024ac6}
Shuting Wang, Jiongnan Liu, Jiehan Cheng, et al. (2024). \textit{DomainRAG: A Chinese Benchmark for Evaluating Domain-specific Retrieval-Augmented Generation}. arXiv.org.

\bibitem{omrani2024i22}
Pouria Omrani, Alireza Hosseini, Kiana Hooshanfar, et al. (2024). \textit{Hybrid Retrieval-Augmented Generation Approach for LLMs Query Response Enhancement}. 2024 10th International Conference on Web Research (ICWR).

\bibitem{tozuka2024nau}
Ryota Tozuka, Hisashi Johno, Akitomo Amakawa, et al. (2024). \textit{Application of NotebookLM, a Large Language Model with Retrieval-Augmented Generation, for Lung Cancer Staging}. Japanese Journal of Radiology.

\bibitem{ma20245jl}
Xueguang Ma, Shengyao Zhuang, B. Koopman, et al. (2024). \textit{VISA: Retrieval Augmented Generation with Visual Source Attribution}. arXiv.org.

\bibitem{yang2024128}
Si-Nan Yang, Dong Wang, Haoqi Zheng, et al. (2024). \textit{TimeRAG: BOOSTING LLM Time Series Forecasting via Retrieval-Augmented Generation}. IEEE International Conference on Acoustics, Speech, and Signal Processing.

\bibitem{lakatos202456t}
Robert Lakatos, P. Pollner, András Hajdu, et al. (2024). \textit{Investigating the performance of Retrieval-Augmented Generation and fine-tuning for the development of AI-driven knowledge-based systems}. Machine Learning and Knowledge Extraction.

\bibitem{chen20247nc}
Zhuo Chen, Jiawei Liu, Haotan Liu, et al. (2024). \textit{Black-Box Opinion Manipulation Attacks to Retrieval-Augmented Generation of Large Language Models}. arXiv.org.

\bibitem{zerhoudi2024y9l}
Saber Zerhoudi, and Michael Granitzer (2024). \textit{PersonaRAG: Enhancing Retrieval-Augmented Generation Systems with User-Centric Agents}. IR-RAG@SIGIR.

\bibitem{yu2025b4u}
Dazhou Yu, Riyang Bao, Gengchen Mai, et al. (2025). \textit{Spatial-RAG: Spatial Retrieval Augmented Generation for Real-World Spatial Reasoning Questions}. arXiv.org.

\bibitem{ghadban2023j9e}
Yasmina Al Ghadban, Yvonne Lu, Uday Adavi, et al. (2023). \textit{Transforming Healthcare Education: Harnessing Large Language Models for Frontline Health Worker Capacity Building using Retrieval-Augmented Generation}. medRxiv.

\bibitem{liang2025f4q}
Xun Liang, Simin Niu, Zhiyu Li, et al. (2025). \textit{SafeRAG: Benchmarking Security in Retrieval-Augmented Generation of Large Language Model}. Annual Meeting of the Association for Computational Linguistics.

\bibitem{quinn2024n3o}
Derrick Quinn, Mohammad Nouri, Neel Patel, et al. (2024). \textit{Accelerating Retrieval-Augmented Generation}. International Conference on Architectural Support for Programming Languages and Operating Systems.

\bibitem{tan2024l5v}
Hanzhuo Tan, Qi Luo, Lingixao Jiang, et al. (2024). \textit{Prompt-based Code Completion via Multi-Retrieval Augmented Generation}. ACM Transactions on Software Engineering and Methodology.

\bibitem{hajiaghayi20245ir}
Mohammadtaghi Hajiaghayi, S'ebastien Lahaie, Keivan Rezaei, et al. (2024). \textit{Ad Auctions for LLMs via Retrieval Augmented Generation}. Neural Information Processing Systems.

\bibitem{garigliotti2024sco}
Darío Garigliotti (2024). \textit{SDG target detection in environmental reports using Retrieval-augmented Generation with LLMs}. CLIMATENLP.

\bibitem{barron2024kue}
Ryan Barron, Ves Grantcharov, Selma Wanna, et al. (2024). \textit{Domain-Specific Retrieval-Augmented Generation Using Vector Stores, Knowledge Graphs, and Tensor Factorization}. International Conference on Machine Learning and Applications.

\bibitem{zhang2025byv}
Shiyue Zhang, Mark Dredze, AI Bloomberg, et al. (2025). \textit{RAG LLMs are Not Safer: A Safety Analysis of Retrieval-Augmented Generation for Large Language Models}. North American Chapter of the Association for Computational Linguistics.

\bibitem{gan2024id0}
Chunjing Gan, Dan Yang, Binbin Hu, et al. (2024). \textit{Similarity is Not All You Need: Endowing Retrieval Augmented Generation with Multi Layered Thoughts}. arXiv.org.

\bibitem{wang20245w8}
Shuting Wang, Xin Xu, Mang Wang, et al. (2024). \textit{RichRAG: Crafting Rich Responses for Multi-faceted Queries in Retrieval-Augmented Generation}. International Conference on Computational Linguistics.

\bibitem{li2024w6r}
Yuying Li, Gaoyang Liu, Chen Wang, et al. (2024). \textit{Generating Is Believing: Membership Inference Attacks against Retrieval-Augmented Generation}. IEEE International Conference on Acoustics, Speech, and Signal Processing.

\bibitem{fu2024m5q}
Jia Fu, Xiaoting Qin, Fangkai Yang, et al. (2024). \textit{AutoRAG-HP: Automatic Online Hyper-Parameter Tuning for Retrieval-Augmented Generation}. Conference on Empirical Methods in Natural Language Processing.

\bibitem{liu2024nei}
Huanshuo Liu, Hao Zhang, Zhijiang Guo, et al. (2024). \textit{CtrlA: Adaptive Retrieval-Augmented Generation via Probe-Guided Control}. arXiv.org.

\bibitem{muludi2024ehk}
Kurnia Muludi, Kaira Milani Fitria, Joko Triloka, et al. (2024). \textit{Retrieval-Augmented Generation Approach: Document Question Answering using Large Language Model}. International Journal of Advanced Computer Science and Applications.

\bibitem{lahiri2024i1q}
A. Lahiri, and Q. Hu (2024). \textit{AlzheimerRAG: Multimodal Retrieval Augmented Generation for PubMed articles}. arXiv.org.

\bibitem{hei2024cs4}
Zijian Hei, Weiling Liu, Wenjie Ou, et al. (2024). \textit{DR-RAG: Applying Dynamic Document Relevance to Retrieval-Augmented Generation for Question-Answering}. arXiv.org.

\bibitem{zhang2024rwm}
Yucheng Zhang, Qinfeng Li, Tianyu Du, et al. (2024). \textit{HijackRAG: Hijacking Attacks against Retrieval-Augmented Large Language Models}. arXiv.org.

\bibitem{qi2024g7x}
Jirui Qi, Gabriele Sarti, R. Fern'andez, et al. (2024). \textit{Model Internals-based Answer Attribution for Trustworthy Retrieval-Augmented Generation}. Conference on Empirical Methods in Natural Language Processing.

\bibitem{he2024hos}
Bolei He, Nuo Chen, Xinran He, et al. (2024). \textit{Retrieving, Rethinking and Revising: The Chain-of-Verification Can Improve Retrieval Augmented Generation}. Conference on Empirical Methods in Natural Language Processing.

\bibitem{qin202445s}
Ruiyang Qin, Zheyu Yan, Dewen Zeng, et al. (2024). \textit{Robust Implementation of Retrieval-Augmented Generation on Edge-Based Computing-in-Memory Architectures}. International Conference on Computer Aided Design.

\bibitem{wang20245b5}
Zheng Wang, Zhongyang Li, Zeren Jiang, et al. (2024). \textit{Crafting Personalized Agents through Retrieval-Augmented Generation on Editable Memory Graphs}. Conference on Empirical Methods in Natural Language Processing.

\bibitem{merth20243h7}
Thomas Merth, Qichen Fu, Mohammad Rastegari, et al. (2024). \textit{Superposition Prompting: Improving and Accelerating Retrieval-Augmented Generation}. International Conference on Machine Learning.

\bibitem{chen20247c1}
Jiajing Chen, Runyuan Bao, Hongye Zheng, et al. (2024). \textit{Optimizing Retrieval-Augmented Generation with Elasticsearch for Enhanced Question-Answering Systems}. 2024 5th International Conference on Big Data, Artificial Intelligence and Internet of Things Engineering (ICBAIE).

\bibitem{thorpe2024l37}
Dayton G. Thorpe, Andrew Duberstein, and Ian A. Kinsey (2024). \textit{Dubo-SQL: Diverse Retrieval-Augmented Generation and Fine Tuning for Text-to-SQL}. arXiv.org.

\bibitem{loumachi2024nxa}
Fatma Yasmine Loumachi, M. C. Ghanem, and M. Ferrag (2024). \textit{Advancing Cyber Incident Timeline Analysis Through Retrieval-Augmented Generation and Large Language Models}. De Computis.

\bibitem{xu2024be3}
Weiye Xu, Min Wang, Wen-gang Zhou, et al. (2024). \textit{P-RAG: Progressive Retrieval Augmented Generation For Planning on Embodied Everyday Task}. ACM Multimedia.

\bibitem{fayyazi2024h99}
Reza Fayyazi, Rozhina Taghdimi, and S. Yang (2024). \textit{Advancing TTP Analysis: Harnessing the Power of Encoder-Only and Decoder-Only Language Models with Retrieval Augmented Generation}. arXiv.org.

\bibitem{wang2024ad6}
Xi Wang, Procheta Sen, Ruizhe Li, et al. (2024). \textit{Adaptive Retrieval-Augmented Generation for Conversational Systems}. North American Chapter of the Association for Computational Linguistics.

\bibitem{kuo2024gi6}
Tzu-Lin Kuo, Fengting Liao, Mu-Wei Hsieh, et al. (2024). \textit{RAD-Bench: Evaluating Large Language Models Capabilities in Retrieval Augmented Dialogues}. arXiv.org.

\bibitem{yazaki20245js}
Megumi Yazaki, S. Maki, T. Furuya, et al. (2024). \textit{Emergency Patient Triage Improvement through a Retrieval-Augmented Generation Enhanced Large-Scale Language Model}. Prehospital Emergency Care.

\bibitem{clop2024zs2}
Cody Clop, and Yannick Teglia (2024). \textit{Backdoored Retrievers for Prompt Injection Attacks on Retrieval Augmented Generation of Large Language Models}. arXiv.org.

\bibitem{lee20240to}
Jaedong Lee, H. Cha, Y. Hwangbo, et al. (2024). \textit{Enhancing Large Language Model Reliability: Minimizing Hallucinations with Dual Retrieval-Augmented Generation Based on the Latest Diabetes Guidelines}. Journal of Personalized Medicine.

\bibitem{chen2025tux}
Zhe Chen, Yusheng Liao, Shuyang Jiang, et al. (2025). \textit{Towards Omni-RAG: Comprehensive Retrieval-Augmented Generation for Large Language Models in Medical Applications}. Annual Meeting of the Association for Computational Linguistics.

\bibitem{garcia2024qd5}
Brandon T Garcia, Lauren Westerfield, Priya Yelemali, et al. (2024). \textit{Improving Automated Deep Phenotyping Through Large Language Models Using Retrieval Augmented Generation}. medRxiv.

\bibitem{yang20255fx}
Qimin Yang, Huan Zuo, Runqi Su, et al. (2025). \textit{Dual retrieving and ranking medical large language model with retrieval augmented generation}. Scientific Reports.

\bibitem{dong2023i5q}
Chenxi Dong (2023). \textit{How to Build an AI Tutor that Can Adapt to Any Course and Provide Accurate Answers Using Large Language Model and Retrieval-Augmented Generation}. arXiv.org.

\bibitem{wu2024o9r}
Feifan Wu, Lingyuan Liu, Wentao He, et al. (2024). \textit{Time-Sensitve Retrieval-Augmented Generation for Question Answering}. International Conference on Information and Knowledge Management.

\bibitem{li2024oot}
Dongyang Li, Junbing Yan, Taolin Zhang, et al. (2024). \textit{On the Role of Long-tail Knowledge in Retrieval Augmented Large Language Models}. Annual Meeting of the Association for Computational Linguistics.

\bibitem{sharma2024t3p}
Kartik Sharma, Peeyush Kumar, and Yunqing Li (2024). \textit{OG-RAG: Ontology-Grounded Retrieval-Augmented Generation For Large Language Models}. arXiv.org.

\bibitem{leekha2024pac}
R. Leekha, Olga Simek, and Charlie Dagli (2024). \textit{War of Words: Harnessing the Potential of Large Language Models and Retrieval Augmented Generation to Classify, Counter and Diffuse Hate Speech}. The Florida AI Research Society.

\bibitem{xu2024w5j}
Ruiyu Xu, Ying Hong, Feifei Zhang, et al. (2024). \textit{Evaluation of the integration of retrieval-augmented generation in large language model for breast cancer nursing care responses}. Scientific Reports.

\bibitem{low2025gjc}
Y. Low, Michael L. Jackson, Rebecca J. Hyde, et al. (2025). \textit{Answering real-world clinical questions using large language model, retrieval-augmented generation, and agentic systems}. Digital Health.

\bibitem{chen2024iyt}
Weijie Chen, Ting Bai, Jinbo Su, et al. (2024). \textit{KG-Retriever: Efficient Knowledge Indexing for Retrieval-Augmented Large Language Models}. arXiv.org.

\bibitem{zhu2024yj5}
Yutao Zhu, Zhaoheng Huang, Zhicheng Dou, et al. (2024). \textit{One Token Can Help! Learning Scalable and Pluggable Virtual Tokens for Retrieval-Augmented Large Language Models}. AAAI Conference on Artificial Intelligence.

\bibitem{verma2024f91}
Sourav Verma (2024). \textit{Contextual Compression in Retrieval-Augmented Generation for Large Language Models: A Survey}. arXiv.org.

\bibitem{yao20240zt}
Chengyuan Yao, and Satoshi Fujita (2024). \textit{Adaptive Control of Retrieval-Augmented Generation for Large Language Models Through Reflective Tags}. Electronics.

\bibitem{leite2025k0s}
Marcus Vinicius Leite, J. Abe, Marcos Leandro Hoffmann Souza, et al. (2025). \textit{Enhancing Environmental Control in Broiler Production: Retrieval-Augmented Generation for Improved Decision-Making with Large Language Models}. AgriEngineering.

\bibitem{burgan20246u3}
Cara Burgan, Josiah Kowalski, and Weidong Liao (2024). \textit{Developing a Retrieval Augmented Generation (RAG) Chatbot App Using Adaptive Large Language Models (LLM) and LangChain Framework}. Proceedings of the West Virginia Academy of Science.

\bibitem{chu2025wz5}
Yun-Wei Chu, Kai Zhang, Christopher Malon, et al. (2025). \textit{Reducing Hallucinations of Medical Multimodal Large Language Models with Visual Retrieval-Augmented Generation}. arXiv.org.

\bibitem{efeoglu20242eq}
Sefika Efeoglu, and Adrian Paschke (2024). \textit{Relation Extraction with Fine-Tuned Large Language Models in Retrieval Augmented Generation Frameworks}. arXiv.org.

\bibitem{yu2024dv5}
Jeffy Yu (2024). \textit{Retrieval Augmented Generation Integrated Large Language Models in Smart Contract Vulnerability Detection}. arXiv.org.

\bibitem{feng20249iv}
Kan Feng, Lijun Luo, Yongjun Xia, et al. (2024). \textit{Optimizing Microservice Deployment in Edge Computing with Large Language Models: Integrating Retrieval Augmented Generation and Chain of Thought Techniques}. Symmetry.

\bibitem{pichai2023n5p}
Kieran Pichai (2023). \textit{A Retrieval-Augmented Generation Based Large Language Model Benchmarked On a Novel Dataset}. Journal of student-scientists' research.

\bibitem{fayyazi2023qg6}
Reza Fayyazi, Rozhina Taghdimi, and S. Yang (2023). \textit{Advancing TTP Analysis: Harnessing the Power of Large Language Models with Retrieval Augmented Generation}. 2024 Annual Computer Security Applications Conference Workshops (ACSAC Workshops).

\bibitem{sudhi20240uy}
Viju Sudhi, Sinchana Ramakanth Bhat, Max Rudat, et al. (2024). \textit{RAG-Ex: A Generic Framework for Explaining Retrieval Augmented Generation}. Annual International ACM SIGIR Conference on Research and Development in Information Retrieval.

\bibitem{wang2025klk}
Shijie Wang, Wenqi Fan, Yue Feng, et al. (2025). \textit{Knowledge Graph Retrieval-Augmented Generation for LLM-based Recommendation}. Annual Meeting of the Association for Computational Linguistics.

\bibitem{wu2025eum}
Yin Wu, Quanyu Long, Jing Li, et al. (2025). \textit{Visual-RAG: Benchmarking Text-to-Image Retrieval Augmented Generation for Visual Knowledge Intensive Queries}. arXiv.org.

\bibitem{yang20248km}
Rui Yang (2024). \textit{CaseGPT: a case reasoning framework based on language models and retrieval-augmented generation}. arXiv.org.

\bibitem{huang2024grc}
Zhongzhen Huang, Kui Xue, Yongqi Fan, et al. (2024). \textit{Tool Calling: Enhancing Medication Consultation via Retrieval-Augmented Large Language Models}. arXiv.org.

\bibitem{lv202521d}
Peizhuo Lv, Mengjie Sun, Hao Wang, et al. (2025). \textit{RAG-WM: An Efficient Black-Box Watermarking Approach for Retrieval-Augmented Generation of Large Language Models}. arXiv.org.

\bibitem{jiao20259xa}
Yang Jiao, Xiaodong Wang, and Kai Yang (2025). \textit{PR-Attack: Coordinated Prompt-RAG Attacks on Retrieval-Augmented Generation in Large Language Models via Bilevel Optimization}. Annual International ACM SIGIR Conference on Research and Development in Information Retrieval.

\bibitem{wang2024ywz}
Haijin Wang, Mianrong Zhang, Zheng Chen, et al. (2024). \textit{Carbon Footprint Accounting Driven by Large Language Models and Retrieval-augmented Generation}. arXiv.org.

\bibitem{patel2024h7u}
Hetul Niteshbhai Patel, Azara Surti, Parth Goel, et al. (2024). \textit{A Comparative Analysis of Large Language Models with Retrieval-Augmented Generation based Question Answering System}. 2024 8th International Conference on I-SMAC (IoT in Social, Mobile, Analytics and Cloud) (I-SMAC).

\bibitem{hikov2024rme}
Asen Hikov, and Laura Murphy (2024). \textit{Information retrieval from textual data: Harnessing large language models, retrieval augmented generation and prompt engineering}. Journal of AI, Robotics &amp; Workplace Automation.

\bibitem{tayebi20245il}
Soroosh Tayebi, A. Aripoli, P. Iglar, et al. (2024). \textit{RadioRAG: Factual Large Language Models for Enhanced Diagnostics in Radiology Using Dynamic Retrieval Augmented Generation}. arXiv.org.

\bibitem{duc2024hrn}
Nguyen Quang Duc, Le Hai Son, Nguyen Duc Nhan, et al. (2024). \textit{Towards Comprehensive Vietnamese Retrieval-Augmented Generation and Large Language Models}. arXiv.org.

\bibitem{debellis2024bv0}
Michael DeBellis, Nivedita Dutta, Jacob Gino, et al. (2024). \textit{Integrating Ontologies and Large Language Models to Implement Retrieval Augmented Generation}. Appl. Ontology.

\bibitem{pelletier20240l7}
A. Pelletier, Joseph Ramirez, Irsyad Adam, et al. (2024). \textit{Explainable Biomedical Hypothesis Generation via Retrieval Augmented Generation enabled Large Language Models}. arXiv.org.

\bibitem{lin20240ku}
Xinyi Lin, Gelei Deng, Yuekang Li, et al. (2024). \textit{GeneRAG: Enhancing Large Language Models with Gene-Related Task by Retrieval-Augmented Generation}. bioRxiv.

\bibitem{weinert2025cxo}
Dane A Weinert, and A. Rauschecker (2025). \textit{Enhancing Large Language Models with Retrieval-augmented Generation: A Radiology-specific Approach.}. Radiology: Artificial Intelligence.

\bibitem{liu2025rz6}
Siru Liu, A. Wright, Allison B. McCoy, et al. (2025). \textit{Detecting emergencies in patient portal messages using large language models and knowledge graph-based retrieval-augmented generation}. J. Am. Medical Informatics Assoc..

\bibitem{liu2025sy0}
Shuliang Liu, Xinze Li, Zhenghao Liu, et al. (2025). \textit{Judge as A Judge: Improving the Evaluation of Retrieval-Augmented Generation through the Judge-Consistency of Large Language Models}. Annual Meeting of the Association for Computational Linguistics.

\bibitem{nguyen202435q}
Quang Nguyen, Duy-Anh Nguyen, Khang Dang, et al. (2024). \textit{Advancing Question-Answering in Ophthalmology with Retrieval Augmented Generations (RAG): Benchmarking Open-source and Proprietary Large Language Models}. medRxiv.

\bibitem{rehulka2024p05}
Erik Rehulka, and Marek Suppa (2024). \textit{RAG Meets Detox: Enhancing Text Detoxification Using Open Large Language Models with Retrieval Augmented Generation}. Conference and Labs of the Evaluation Forum.

\bibitem{huang202465n}
Jie Huang, Mo Wang, Yunpeng Cui, et al. (2024). \textit{Layered Query Retrieval: An Adaptive Framework for Retrieval-Augmented Generation in Complex Question Answering for Large Language Models}. Applied Sciences.

\bibitem{hammane2024hdb}
Zakaria Hammane, Fatima-Ezzahraa Ben-Bouazza, and A. Fennan (2024). \textit{SelfRewardRAG: Enhancing Medical Reasoning with Retrieval-Augmented Generation and Self-Evaluation in Large Language Models}. International Symposium on Computer Vision.

\bibitem{samarajeewa20241p6}
Chamod Samarajeewa, Daswin De Silva, Evgeny Osipov, et al. (2024). \textit{Causal Reasoning in Large Language Models using Causal Graph Retrieval Augmented Generation}. International Conference on Human System Interaction.

\bibitem{hou2024gz7}
Yu Hou, J. R. Bishop, Hongfang Liu, et al. (2024). \textit{Improving Dietary Supplement Information Retrieval: Development of a Retrieval-Augmented Generation System With Large Language Models}. Journal of Medical Internet Research.

\bibitem{habib2024iqj}
Mohammad Affan Habib, Shehryar Amin, Muhammad Oqba, et al. (2024). \textit{TaxTajweez: A Large Language Model-based Chatbot for Income Tax Information In Pakistan Using Retrieval Augmented Generation (RAG)}. The Florida AI Research Society.

\end{thebibliography}

\end{document}