\documentclass[12pt,a4paper]{article}
    \usepackage[utf8]{inputenc}
    \usepackage[T1]{fontenc}
    \usepackage{amsmath,amsfonts,amssymb}
    \usepackage{graphicx}
    \usepackage[margin=2.5cm]{geometry}
    \usepackage{setspace}
    \usepackage{natbib}
    \usepackage{url}
    \usepackage{hyperref}
    \usepackage{booktabs}
    \usepackage{longtable}
    \usepackage{array}
    \usepackage{multirow}
    \usepackage{wrapfig}
    \usepackage{float}
    \usepackage{colortbl}
    \usepackage{pdflscape}
    \usepackage{tabu}
    \usepackage{threeparttable}
    \usepackage{threeparttablex}
    \usepackage[normalem]{ulem}
    \usepackage{makecell}
    \usepackage{xcolor}

    % Set line spacing
    \doublespacing

    % Configure hyperref
    \hypersetup{
        colorlinks=true,
        linkcolor=blue,
        filecolor=magenta,      
        urlcolor=cyan,
        citecolor=red,
    }

    % Title and author information
    \title{A Comprehensive Literature Review with Self-Reflection}
    \author{Literature Review}
    \date{\today}

    \begin{document}

    \maketitle

    % Abstract (optional)
    \begin{abstract}
    This literature review provides a comprehensive analysis of recent research in the field. The review synthesizes findings from 194 research papers, identifying key themes, methodological approaches, and future research directions.
    \end{abstract}

    \newpage
    \tableofcontents
    \newpage

    \label{sec:introduction:_the_genesis_and_challenges_of_generative_adversarial_networks}

\section{Introduction: The Genesis and Challenges of Generative Adversarial Networks}
\label{sec:introduction:\_the\_genesis\_\_and\_\_challenges\_of\_generative\_adversarial\_networks}

\subsection{The Vision of Adversarial Learning}
\label{sec:1\_1\_the\_vision\_of\_adversarial\_learning}

The advent of deep learning brought unprecedented capabilities to pattern recognition and feature extraction, yet the challenge of effectively learning complex, high-dimensional data distributions and synthesizing novel, realistic samples remained a significant hurdle for traditional generative models. It was against this backdrop that Generative Adversarial Networks (GANs) emerged as a groundbreaking paradigm, fundamentally reshaping the landscape of deep generative modeling.

The foundational concept of adversarial learning was introduced by Goodfellow et al. in their seminal 2014 paper, "Generative Adversarial Networks" \cite{Goodfellow2014}. This work proposed an innovative framework where two neural networks, a generator (G) and a discriminator (D), engage in a dynamic, zero-sum minimax game. The generator's objective is to learn the underlying data distribution and produce synthetic samples that are indistinguishable from real data. Simultaneously, the discriminator's role is to accurately differentiate between authentic samples drawn from the true data distribution and the synthetic samples produced by the generator. This adversarial process is encapsulated by the minimax objective function:
\begin{equation*}
    \min\_G \max\_D V(D, G) = \mathbb{E}\_{x \sim p\_{data}(x)}[\log D(x)] + \mathbb{E}\_{z \sim p\_z(z)}[\log(1 - D(G(z)))]
\end{equation*}
Here, $D(x)$ represents the probability that $x$ came from the real data rather than the generator, and $G(z)$ is the output of the generator given a noise vector $z$. The discriminator $D$ is trained to maximize this function, correctly classifying real samples as real and generated samples as fake. Conversely, the generator $G$ is trained to minimize $\log(1 - D(G(z)))$, effectively trying to fool the discriminator into classifying its synthetic outputs as real. Through this iterative competition, both networks progressively improve: the generator learns to produce increasingly realistic data, while the discriminator becomes more adept at detecting subtle differences between real and fake. Ideally, this process converges when the generator produces samples so convincing that the discriminator can no longer distinguish them from real data, assigning a probability of 0.5 to both real and generated samples.

This innovative paradigm sparked immense excitement within the research community for its profound potential. Unlike previous generative models that often relied on explicit density estimation or complex inference procedures, GANs offered an implicit approach to learning complex data distributions, promising the ability to synthesize novel, high-fidelity data across various domains. The vision extended to applications ranging from photorealistic image generation to data augmentation for scarce datasets, style transfer, and even semi-supervised learning. The elegance of the adversarial game, where the generator's learning is guided by an adaptive adversary, was seen as a powerful mechanism for capturing intricate patterns and variations in data that were previously unattainable.

However, while revolutionary, this initial formulation immediately presented significant training challenges that became central to early research efforts. The delicate balance of the minimax game often led to instability, making GANs notoriously difficult to train effectively. Issues such as vanishing gradients, where the generator's updates become negligible, and mode collapse, where the generator produces a limited variety of samples rather than covering the full data distribution, were quickly identified as inherent complexities of this novel game-theoretic approach \cite{che2016kho, metz20169ir}. These immediate difficulties, though challenging, further underscored the groundbreaking nature of GANs, as they highlighted the need for deeper theoretical understanding and robust practical solutions to fully realize the vision of adversarial learning.

In conclusion, the introduction of Generative Adversarial Networks by Goodfellow et al. laid the essential groundwork for all subsequent research and advancements in the realm of deep generative modeling. Its core mechanism of a generator-discriminator minimax game provided a powerful conceptual framework for learning complex data distributions and synthesizing novel, high-fidelity data. Despite the immediate and persistent challenges related to training stability and mode collapse, the initial vision of adversarial learning ignited a prolific research direction, establishing GANs as a cornerstone technology and a primary catalyst for the rapid evolution of generative AI.
\subsection{Inherent Instabilities: Mode Collapse and Training Divergence}
\label{sec:1\_2\_inherent\_instabilities:\_mode\_collapse\_\_and\_\_training\_divergence}

The initial promise of Generative Adversarial Networks (GANs), as introduced by \cite{Goodfellow2014}, was quickly tempered by a set of pervasive and critical challenges that severely hampered their practical application. Foremost among these were profound training instabilities and the phenomenon of mode collapse, which collectively prevented early GANs from reliably generating diverse and high-quality synthetic data \cite{jabbar2020aj0, wiatrak20194ib, wang2019w53}. Understanding these fundamental problems is crucial for appreciating the necessity and ingenuity behind the stabilization techniques developed subsequently.

At the core of these issues lay the original GAN objective function, which aimed to minimize the Jensen-Shannon (JS) divergence between the real data distribution ($p\_{data}$) and the generator's distribution ($p\_g$). While theoretically elegant, this objective proved highly problematic in practice. In high-dimensional data spaces, it is common for the real and generated data distributions to be non-overlapping or to lie on disjoint low-dimensional manifolds. In such scenarios, the JS divergence can become saturated, leading to a vanishing gradient problem for the generator \cite{Goodfellow2014}. When the discriminator rapidly learns to perfectly distinguish between real and fake samples, it assigns near-zero probability to generated samples. This provides the generator with a gradient that is effectively zero, offering little to no meaningful learning signal to improve its output. Consequently, the generator's parameters either cease to update, leading to stagnation, or oscillate wildly, resulting in outright non-convergence and training divergence.

Beyond vanishing gradients, the adversarial training process itself, framed as a minimax game, is inherently prone to instability. The non-cooperative nature of the generator and discriminator's objectives can lead to complex and often oscillatory dynamics, where each network's updates undo the progress of the other \cite{gonzlezprieto20214wh, liang2018r52}. This constant chasing of evolving strategies can prevent the system from settling into a stable Nash equilibrium, manifesting as erratic training curves, fluctuating sample quality, and a general inability to converge to a robust solution. Such training divergence is characterized not only by vanishing gradients but also by exploding gradients, where updates become excessively large, further destabilizing the learning process.

A direct and particularly debilitating consequence of these instabilities was mode collapse. This phenomenon occurs when the generator fails to capture the full diversity of the real data distribution, instead producing only a limited set of samples that consistently fool the discriminator \cite{jabbar2020aj0, che2016kho}. Rather than exploring the entire data landscape, the generator finds a few "safe" modes or local optima in the minimax game where it can effectively deceive the current discriminator. For example, if trained on a dataset of diverse animal images, a mode-collapsed GAN might only generate variations of a single animal, such as cats, completely ignoring other categories like dogs or birds. This issue is exacerbated by the vanishing gradient problem, as the generator lacks the necessary signal to venture into unexplored regions of the data distribution. Early research, such as Mode Regularized Generative Adversarial Networks by \cite{che2016kho}, directly acknowledged and attempted to mitigate this problem by introducing regularization techniques to encourage a more uniform distribution of probability mass across data modes, highlighting the immediate recognition of mode collapse as a critical limitation.

These inherent instabilities—vanishing and exploding gradients, oscillatory training, non-convergence, and mode collapse—posed significant barriers to the widespread adoption and practical utility of early GANs. They underscored the need for fundamental advancements beyond the original formulation. While initial empirical "tricks" and architectural guidelines (as discussed in Section \ref{sec:early\_heuristics}) offered some practical improvements, the pervasive nature of these problems necessitated more rigorous theoretical and algorithmic solutions. The understanding of these fundamental flaws in the original GAN objective function and the complex dynamics of adversarial training became the primary motivation for a concerted research effort, leading to the development of alternative objective functions, sophisticated regularization techniques, and novel architectural designs that would transform GANs into more robust and powerful generative models, which will be explored in subsequent sections.
\subsection{Scope and Overview of Stabilization Strategies}
\label{sec:1\_3\_scope\_\_and\_\_overview\_of\_stabilization\_strategies}

The foundational promise of Generative Adversarial Networks (GANs) to synthesize realistic data was initially hampered by significant training challenges, including mode collapse, vanishing gradients, and overall instability \cite{jabbar2020aj0, wiatrak20194ib}. Addressing these inherent difficulties has driven a substantial and multi-faceted research effort, leading to a diverse array of stabilization strategies. This subsection serves as a comprehensive roadmap for the subsequent literature review, outlining the major categories of approaches that have transformed GANs into more robust, reliable, and capable generative models. These strategies broadly encompass early architectural heuristics, fundamental modifications to objective functions, sophisticated regularization techniques, innovative architectural designs, and advanced training paradigms, each contributing to enhanced performance across various domains \cite{wenzel20225g3, wang2019w53}.

The earliest attempts to stabilize GAN training, often empirical in nature, focused on establishing practical architectural guidelines and training tricks. As detailed in Section 2, pioneering works like Deep Convolutional GANs (DCGANs) introduced crucial architectural best practices, such as the strategic use of convolutional layers and batch normalization, which significantly improved initial stability and image quality \cite{radford2015unsupervised}. Concurrently, practical heuristics like minibatch discrimination and feature matching emerged, aiming to mitigate mode collapse and encourage diversity by allowing the discriminator to consider batch statistics or match feature representations. These foundational efforts laid the groundwork, demonstrating that structural considerations were vital for managing the complex adversarial dynamics.

A pivotal shift towards more mathematically rigorous solutions is explored in Section 3, focusing on fundamental reformulations of the GAN objective function and the introduction of explicit gradient regularization. The core challenge here was to provide more stable and informative gradients, particularly by enforcing Lipschitz continuity on the discriminator \cite{chu2020zbv}. The introduction of Wasserstein GANs (WGAN) marked a significant breakthrough, replacing the problematic Jensen-Shannon divergence with the Earth Mover's distance, which offered a smoother loss landscape. While initial WGAN implementations relied on weight clipping, subsequent advancements like gradient penalties (e.g., WGAN-GP) and architectural regularizers such as Spectral Normalization (SN-GAN) provided more robust and effective ways to enforce the Lipschitz constraint, thereby mitigating vanishing gradients and improving training stability. This period also saw the exploration of alternative objective functions, including Least Squares GANs (LSGANs) and Energy-Based GANs (EBGANs), which offered different perspectives on achieving stable gradient flow.

Building upon these robust mathematical foundations, the field progressed to scaling GANs for high-fidelity, high-resolution synthesis, as discussed in Section 4. This phase leveraged established stability to push the boundaries of generative capabilities through ingenious architectural innovations and advanced training paradigms. Progressive Growing of GANs (PGGAN) revolutionized high-resolution image generation by incrementally increasing network complexity and image resolution during training, significantly enhancing stability and output quality. The integration of self-attention mechanisms, exemplified by Self-Attention GANs (SAGAN), allowed models to capture long-range dependencies, leading to more globally coherent images. Further scaling, as seen in BigGAN, combined existing stabilization techniques with larger models and datasets to achieve unprecedented levels of diversity and fidelity. The StyleGAN series then introduced a novel style-based generator architecture, enabling highly controllable and disentangled synthesis, which became a benchmark for photorealistic image generation and manipulation.

Recognizing the practical constraints of real-world applications, research also focused on enhancing GAN robustness in data-scarce environments and adapting them for specific domains, which is the subject of Section 5. Techniques like Differentiable Augmentation (DiffAugment) and Adaptive Discriminator Augmentation (ADA) became crucial for stabilizing training and preventing discriminator overfitting when only limited data was available, thereby expanding GAN applicability. Furthermore, advanced meta-learning approaches emerged to enable few-shot generative capabilities, allowing GANs to adapt to new distributions with minimal samples. This section also covers domain-specific adaptations and hybrid architectures, such as VAE-GANs, which combine the strengths of different generative paradigms to address unique challenges in specialized fields.

Finally, the review extends to the expanding horizons of GAN research, explored in Section 7. This includes significant advancements in 3D-aware generation, where stabilized GANs are integrated with models like Neural Radiance Fields (NeRFs) to synthesize coherent 3D scenes from 2D latent spaces. A rapidly evolving frontier involves the hybridization of GANs with other powerful generative paradigms, notably Diffusion Models, aiming to combine the sharp detail generation and fast inference of GANs with the inherent stability and mode coverage of diffusion processes. These cutting-edge developments demonstrate the ongoing evolution of GANs into more versatile and powerful generative systems capable of addressing complex, multi-dimensional synthesis tasks.

In summary, the journey of GAN stabilization reflects a continuous and iterative progression, moving from initial empirical fixes to rigorous mathematical frameworks, then to sophisticated architectural scaling, and finally to practical robustness and novel generative frontiers. Each phase has built upon the preceding one, collectively addressing the inherent challenges of adversarial training to enable the generation of high-quality, diverse, and controllable outputs across an ever-widening array of applications. This multi-faceted evolution underscores the interdisciplinary nature of GAN research and its profound impact on the field of generative AI.


\label{sec:early_heuristics_and_architectural_foundations_for_stability}

\section{Early Heuristics and Architectural Foundations for Stability}
\label{sec:early\_heuristics\_\_and\_\_architectural\_foundations\_for\_stability}

\subsection{Deep Convolutional GANs (DCGANs) and Architectural Best Practices}
\label{sec:2\_1\_deep\_convolutional\_gans\_(dcgans)\_\_and\_\_architectural\_best\_practices}

Early Generative Adversarial Networks (GANs) \cite{Goodfellow2014} demonstrated the potential of adversarial training for image synthesis, but their training was notoriously unstable, often leading to mode collapse and poor image quality, particularly when employing fully connected network architectures. The pivotal work on Deep Convolutional Generative Adversarial Networks (DCGANs) \cite{Radford2015} addressed these fundamental challenges by introducing a set of architectural guidelines that significantly improved training stability and the visual fidelity of generated images. These recommendations quickly became standard practice, laying an enduring foundation for subsequent GAN designs.

DCGANs established that integrating Convolutional Neural Networks (CNNs) into both the generator and discriminator was crucial for learning hierarchical representations in images. Specifically, they advocated for an "all-convolutional" architecture, replacing deterministic spatial pooling functions (like max-pooling) with strided convolutions in the discriminator for downsampling and fractional-strided convolutions (also known as transposed convolutions) in the generator for upsampling \cite{Radford2015}. This allowed the network to learn its own optimal spatial upsampling and downsampling, rather than relying on fixed operations, leading to more coherent and realistic image generation.

A key innovation for stabilizing training was the strategic application of Batch Normalization \cite{Radford2015}. By normalizing the inputs to each layer, batch normalization helps to prevent internal covariate shift, allowing for deeper models and more stable gradient flow. In DCGANs, batch normalization was applied to most layers in both the generator and discriminator. However, it was intentionally omitted from the generator's output layer (which typically uses a Tanh activation for image generation to allow for a full range of pixel values) and the discriminator's input layer (to avoid directly normalizing the real and fake data distributions, which could hinder the discriminator's ability to distinguish between them) \cite{Radford2015}. This selective application was critical for maintaining the adversarial balance.

Regarding activation functions, DCGANs proposed specific choices that further contributed to stability and performance. The generator predominantly utilized Rectified Linear Unit (ReLU) activations in all layers, except for the output layer, which typically employed a Tanh activation. ReLU activations promote sparsity and help mitigate vanishing gradients, while Tanh allows the output pixels to range from -1 to 1, a common preprocessing step for image data \cite{Radford2015}. For the discriminator, Leaky ReLU (LeakyReLU) activations were preferred for all layers. LeakyReLU allows a small, non-zero gradient for negative inputs, preventing "dead ReLUs" and ensuring that gradients can flow even when activations are negative, which is vital for the discriminator to provide meaningful feedback to the generator across the entire input space \cite{Radford2015}. Furthermore, DCGANs largely eliminated fully connected layers, except for the initial projection from the latent space to a convolutional feature map in the generator and the final output classification layer in the discriminator, reinforcing the all-convolutional design principle.

These architectural recommendations from DCGANs significantly advanced the state-of-the-art in generative modeling, leading to a marked improvement in the stability of GAN training and the visual quality of generated images compared to previous fully connected approaches. The principles of using convolutional layers for spatial processing, strategic batch normalization, and specific activation functions became foundational elements adopted by countless subsequent GAN architectures. However, despite these architectural breakthroughs, challenges such as mode collapse and further training instability persisted, necessitating additional research into regularization techniques and alternative loss functions. For instance, subsequent works like Mode Regularized Generative Adversarial Networks \cite{che2016kho} and Unrolled Generative Adversarial Networks \cite{metz20169ir} emerged shortly after DCGANs, specifically addressing the ongoing issues of mode collapse and general training stability through regularization and modified optimization objectives, demonstrating that while DCGANs provided the essential architectural blueprint, the dynamic intricacies of adversarial training still required further methodological refinement.
\subsection{Practical Training Tricks: Minibatch Discrimination and Feature Matching}
\label{sec:2\_2\_practical\_training\_tricks:\_minibatch\_discrimination\_\_and\_\_feature\_matching}

Early in the development of Generative Adversarial Networks (GANs), training instability and the persistent problem of mode collapse presented significant hurdles, often leading to generators producing only a limited variety of samples. Beyond fundamental adjustments to the loss function, empirical and practical techniques emerged as crucial tools to stabilize training and encourage diversity. Among the most influential of these early heuristics were minibatch discrimination and feature matching, which provided tangible improvements by actively managing the complex adversarial dynamics.

A pivotal work introducing several such practical enhancements was presented by \cite{Salimans2016}. This paper highlighted that the discriminator's ability to provide useful gradients to the generator could be significantly improved by considering not just individual samples, but the statistical properties of an entire batch. To this end, \textbf{minibatch discrimination} was proposed. This technique augments the discriminator with an additional layer that computes a tensor of differences between the feature vectors of samples within a minibatch. By allowing the discriminator to detect and penalize statistical similarities among generated samples, minibatch discrimination effectively encourages the generator to produce a more diverse set of outputs, thereby mitigating mode collapse. The discriminator learns to distinguish real data from generated data not only by individual sample quality but also by the diversity (or lack thereof) within a batch.

Complementing this, \cite{Salimans2016} also introduced \textbf{feature matching} as another heuristic to stabilize generator training. Instead of directly optimizing the generator to fool the discriminator, feature matching encourages the generator to produce samples whose feature statistics, as extracted from an intermediate layer of the discriminator, match those of real data. Specifically, the generator's objective is modified to minimize the $\ell\_2$ distance between the mean feature vector of real data and the mean feature vector of generated data, both computed from a chosen intermediate layer of the discriminator. This approach provides a more stable and less volatile objective for the generator, preventing it from exploiting specific weaknesses of the current discriminator and instead guiding it towards generating samples that are statistically similar to real data in a meaningful feature space.

These techniques, though heuristic, were instrumental in improving the performance and reliability of GANs in their nascent stages. They underscored the importance of engineering practical solutions to address the inherent challenges of adversarial training, demonstrating that stability and diversity could be actively managed through architectural modifications and objective function regularization beyond the primary adversarial loss. While these methods offered significant improvements, the quest for robust GAN training continued. For instance, even in later applications such as SAR image recognition, GANs still faced instability issues, prompting further research into specialized architectures like those employing multiple generators to enhance stability in domain-specific contexts \cite{gao2018d4g}. This illustrates that while early tricks like minibatch discrimination and feature matching laid crucial groundwork, the complex dynamics of GANs necessitated ongoing innovation across various fronts to achieve consistent and high-quality generation.

In conclusion, minibatch discrimination and feature matching from \cite{Salimans2016} represent a critical early phase in GAN research, where empirical insights led to practical solutions for training stability and sample diversity. These methods highlighted that effective GAN training often requires a multifaceted approach, combining theoretical advancements with clever architectural and procedural tricks. Their success paved the way for more sophisticated architectural innovations and objective function designs, continually pushing the boundaries of generative modeling.


\label{sec:mathematical_foundations:_loss_function_reformulations_and_gradient_regularization}

\section{Mathematical Foundations: Loss Function Reformulations and Gradient Regularization}
\label{sec:mathematical\_foundations:\_loss\_function\_reformulations\_\_and\_\_gradient\_regularization}

\subsection{Wasserstein GANs (WGAN) and the Earth Mover's Distance}
\label{sec:3\_1\_wasserstein\_gans\_(wgan)\_\_and\_\_the\_earth\_mover's\_distance}

The early development of Generative Adversarial Networks (GANs) \cite{Goodfellow2014} was plagued by significant training instabilities, primarily stemming from the objective function's reliance on the Jensen-Shannon (JS) divergence. This divergence often became saturated when the real and generated data distributions were non-overlapping, a common occurrence in high-dimensional data spaces, leading to vanishing gradients for the generator and consequently, a lack of meaningful learning signals. This fundamental issue contributed to notorious problems such as mode collapse, where the generator would produce only a limited subset of the true data distribution's diversity, and overall training fragility. Efforts to mitigate these issues, such as the regularization techniques proposed in \cite{che2016kho} to address mode collapse, highlighted the pressing need for a more robust and stable GAN objective.

A groundbreaking paradigm shift was introduced by Wasserstein GANs (WGAN) \cite{Arjovsky2017}, which fundamentally altered the GAN objective by replacing the problematic JS divergence with the Earth Mover's (Wasserstein-1) distance. This change was pivotal because, unlike f-divergences, the Wasserstein distance provides a smooth and continuous loss landscape, even when the real and generated distributions are disjoint. Consequently, the critic (the WGAN equivalent of the discriminator) could provide meaningful, non-zero gradients to the generator almost everywhere, effectively resolving the critical issues of vanishing gradients and mode collapse that had severely hampered earlier GAN formulations. The continuous nature of the Wasserstein distance ensures that the generator receives a consistent and informative learning signal, allowing it to gradually improve its output and cover the full diversity of the target data distribution.

The theoretical foundation of WGAN relies on the Kantorovich-Rubinstein duality, which allows the Wasserstein-1 distance to be approximated by a neural network, provided that this network satisfies a K-Lipschitz continuity constraint. To enforce this crucial constraint, the initial WGAN formulation \cite{Arjovsky2017} employed a simple yet effective technique: weight clipping. After each gradient update, the weights of the critic network were clipped to a small, fixed range (e.g., $[-c, c]$). This method aimed to keep the critic's gradients bounded, thereby ensuring its Lipschitz continuity and allowing it to reliably estimate the Earth Mover's distance. This represented a significant theoretical and practical advancement, as it provided a more stable training objective and a loss value that directly correlated with the quality of the generated samples, offering a much-needed metric for convergence and model evaluation.

The introduction of WGAN marked a critical turning point in GAN research, providing a robust framework that significantly improved training stability and reduced mode collapse. Its core innovation, the adoption of the Earth Mover's distance, provided a theoretically sound and empirically effective solution to the long-standing problem of unstable GAN training. While the initial weight clipping method for enforcing the Lipschitz constraint, as proposed in \cite{Arjovsky2017}, proved effective in demonstrating the power of the Wasserstein objective, it also presented practical challenges, such as sensitivity to the clipping range and potential for pathological behavior in the critic's capacity. These limitations subsequently spurred further research into more robust and adaptive methods for Lipschitz enforcement, paving the way for advanced regularization techniques like gradient penalties and spectral normalization. Nevertheless, WGAN's foundational contribution has been widely adopted and continues to influence the development of stable generative models, as evidenced by its integration into various applications, such as enhancing channel estimation in wireless communication systems \cite{hu2021yk5, ye2024n41}.
\subsection{Gradient Penalties (WGAN-GP, DRAGAN) for Lipschitz Continuity}
\label{sec:3\_2\_gradient\_penalties\_(wgan-gp,\_dragan)\_for\_lipschitz\_continuity}

The initial formulation of Generative Adversarial Networks (GANs) often suffered from training instabilities, including mode collapse and vanishing gradients, largely due to the limitations of the Jensen-Shannon divergence as an objective function. While Wasserstein GANs (WGANs) introduced a more stable objective based on the Earth Mover's distance \cite{Arjovsky2017}, its original method for enforcing the Lipschitz constraint on the discriminator—weight clipping—proved problematic, leading to limited model capacity and pathological gradient behavior. This critical limitation necessitated more robust regularization techniques, leading to the development of gradient penalties.

A pivotal advancement in stabilizing WGAN training was introduced by \textcite{Gulrajani2017} with the Improved Training of Wasserstein GANs (WGAN-GP). This work replaced the problematic weight clipping with a gradient penalty term added to the discriminator's loss function. The WGAN-GP approach rigorously enforces the 1-Lipschitz constraint by penalizing the norm of the discriminator's gradient with respect to its input, specifically at samples interpolated between real and generated data points. This method significantly improved training stability, alleviated mode collapse, and produced higher-quality samples by ensuring smoother gradients for the generator, thereby becoming a widely adopted standard for robust GAN training.

Building upon this foundation, the broader landscape of gradient regularization techniques was further explored and theoretically grounded. \textcite{Mescheder2018} provided a comprehensive analysis of various training methods for GANs, including WGAN-GP and other gradient penalty methods such as DRAGAN (Discriminator Regularization with Adversarial Gradients). DRAGAN, while also penalizing discriminator gradients, often focuses on gradients around real data points to enhance stability and prevent the discriminator from becoming too confident. \textcite{Mescheder2018} demonstrated that properly applied gradient penalties are essential for stable training and provided theoretical insights into their convergence properties, highlighting the crucial role of precise gradient control in achieving stable and high-quality GAN training dynamics.

The effectiveness and robustness of gradient penalties, particularly WGAN-GP, have led to their widespread adoption across diverse applications and architectural innovations. For instance, in the domain of wireless communications, \textcite{hu2021yk5} proposed a GAN-based channel estimation enhancement algorithm that integrates a conditional GAN with an improved Wasserstein GAN. This approach leverages the stability provided by gradient penalties to improve the training stability and learning ability of GANs, ultimately yielding lower relative error performance in channel estimation without requiring longer training sequences. This demonstrates how the stability afforded by gradient penalties can be critical for applying GANs in sensitive engineering tasks.

Similarly, in synthetic aperture radar (SAR) image generation, where data scarcity can be a significant challenge, \textcite{huang2022zar} introduced VAE-WGANGP, a model combining a Variational Autoencoder (VAE) with WGAN-GP. This asymmetric network design incorporates a gradient penalty into its loss function, alongside reconstruction, divergence, and adversarial losses. The explicit inclusion of the gradient penalty in VAE-WGANGP was crucial for alleviating issues of gradient explosion or disappearance, thereby ensuring training stability and enabling the generation of high-fidelity SAR images with diverse features. These applications collectively underscore the foundational importance of gradient penalties in enabling stable and effective GAN training across various complex data modalities.

Despite their significant advantages, gradient penalty methods can introduce additional computational overhead due to the need for gradient computations, and their effectiveness often depends on careful hyperparameter tuning for the penalty coefficient. Future research directions may explore more computationally efficient or adaptive gradient penalty formulations, or investigate how these penalties interact with other regularization techniques to achieve even greater stability and performance with reduced training complexity.
\subsection{Spectral Normalization: An Efficient Architectural Regularizer}
\label{sec:3\_3\_spectral\_normalization:\_an\_efficient\_architectural\_regularizer}

The quest for stable Generative Adversarial Network (GAN) training has led to the development of numerous regularization techniques, with a central focus on enforcing the Lipschitz constraint on the discriminator. While early breakthroughs, particularly Wasserstein GANs (WGANs) \cite{84de7d27e2f6160f634a483e8548c499a2cda7fa}, identified this constraint as crucial, their initial implementation via weight clipping proved problematic, limiting model capacity and sometimes leading to unstable behavior. Subsequent advancements, such as WGAN-GP \cite{community\_20}, introduced gradient penalties that effectively enforced the constraint but incurred significant computational overhead due to the need for gradient computations across the input space. Addressing these limitations, Spectral Normalization (SN) emerged as a highly efficient and effective architectural regularization technique, directly enforcing the Lipschitz constraint on the discriminator's weights \cite{miyato2018arc}. This method quickly became a cornerstone in stabilizing GAN training, offering a computationally lighter and often more robust alternative or complement to loss-based penalties \cite{jabbar2020aj0, liu2020jt0}.

Introduced by \cite{miyato2018arc}, Spectral Normalization operates by normalizing the spectral norm (the largest singular value) of each weight matrix within the discriminator network. Specifically, for a weight matrix $W$, SN divides $W$ by its spectral norm $\sigma(W)$, ensuring that the Lipschitz constant of each layer is bounded by 1. This direct control over the network's architecture effectively bounds the overall Lipschitz constant of the discriminator, preventing it from becoming overly powerful and providing the generator with more meaningful and stable gradients. Unlike gradient penalties that require backpropagating through the discriminator to compute gradients with respect to its inputs, SN is applied directly to the weights during the forward pass, making it computationally efficient and straightforward to integrate into existing GAN architectures. This simplicity, ease of implementation, and robust stabilization properties have profoundly influenced GAN development, contributing significantly to overall training reliability and the generation of high-fidelity samples across various GAN formulations \cite{wang2019w53}.

The impact of Spectral Normalization is evident in its widespread adoption across state-of-the-art GAN models designed for high-fidelity image synthesis. For instance, Self-Attention Generative Adversarial Networks (SAGAN) \cite{670f9d0d8cafaeaeea564c88645b9816b1146cef}, a seminal work integrating self-attention mechanisms, explicitly leveraged Spectral Normalization in both its generator and discriminator to achieve stable training and generate images with improved global coherence. Similarly, the groundbreaking BigGAN \cite{community\_39}, which pushed the boundaries of image quality and diversity on large datasets like ImageNet, critically relied on Spectral Normalization as a key component alongside other architectural innovations and large-batch training. Beyond unconditional image generation, SN has proven valuable in conditional tasks, such as text-to-image synthesis, where models like DualAttn-GAN \cite{cai2019g1w} incorporated SN to stabilize training and enhance the quality of generated images by ensuring a well-behaved discriminator.

While Spectral Normalization offers significant advantages in computational efficiency and ease of implementation, its role is often complementary rather than exclusively substitutive. Research has shown that SN can effectively combine with other regularization techniques to further enhance stability and performance. For example, \cite{zhang2019hjo} demonstrated that consistency regularization, which penalizes the discriminator's sensitivity to data augmentations, works effectively \textit{with} spectral normalization, leading to improved FID scores for unconditional image generation. This highlights that SN is not a singular panacea but a powerful tool within a broader ecosystem of GAN stabilization strategies. The choice between SN and gradient penalties (like WGAN-GP) can sometimes depend on the specific dataset, architecture, and desired trade-offs between computational cost, training stability, and the quality of generated samples, reflecting an ongoing area of research and practical consideration in the field.

In conclusion, Spectral Normalization represents a pivotal architectural advancement in GAN training, offering an efficient and effective method to enforce the Lipschitz constraint on the discriminator. By directly regularizing weight matrices, it provides a computationally lightweight and robust mechanism that has been instrumental in stabilizing training and enabling the generation of high-fidelity outputs. Its widespread integration into landmark GAN architectures underscores its significance, solidifying its position as a standard component for developing sophisticated generative models across a broad range of applications. The continued exploration of its interplay with other regularization techniques further cements its enduring relevance in the pursuit of universally stable and efficient GAN training paradigms.
\subsection{Alternative Objective Functions: LSGAN, EBGAN, BEGAN}
\label{sec:3\_4\_alternative\_objective\_functions:\_lsgan,\_ebgan,\_began}

The early development of Generative Adversarial Networks (GANs) was significantly challenged by training instability and issues like mode collapse, prompting researchers to explore innovative objective functions and architectural designs beyond the original minimax game and the subsequent Wasserstein distance framework. This subsection delves into several pivotal approaches that introduced alternative loss functions and discriminator architectures to foster more stable and higher-quality generation.

One notable departure from the standard GAN objective was the introduction of Least Squares Generative Adversarial Networks (LSGANs) by \cite{Mao2017}. Unlike traditional GANs that employ a sigmoid cross-entropy loss, LSGANs replace this with a least squares loss function. This modification fundamentally alters the gradient landscape, providing smoother gradients and mitigating the vanishing gradient problem that often plagued early GANs. Empirically, LSGANs demonstrated improved training stability and generated higher quality images by penalizing samples based on their distance from the decision boundaries, encouraging the generator to produce samples closer to the real data manifold and the discriminator to push fake samples further away. The simplicity and effectiveness of this loss function modification made LSGAN a practical alternative for stable GAN training.

Building on the exploration of novel objective functions, Energy-Based Generative Adversarial Networks (EBGANs) proposed by \cite{Zhao2017} offered a distinct conceptual framework. In EBGANs, the discriminator is re-imagined as an energy function that assigns low energy to real data samples and high energy to generated (fake) samples. The objective is to learn a manifold where real data resides in a low-energy region, while pushing fake samples into higher-energy regions. This is achieved through a margin loss, which explicitly pulls down the energy of real samples and pushes up the energy of fake ones, thereby providing a different perspective on stability and mode collapse reduction. A key architectural innovation in EBGANs is the use of an autoencoder as the discriminator, where the reconstruction error serves as the energy function. This design choice implicitly encourages the discriminator to learn meaningful features of the data distribution.

Further refining the concept of an autoencoder-based discriminator, Boundary Equilibrium Generative Adversarial Networks (BEGANs) were introduced by \cite{Berthelot2017}. BEGANs leverage a similar autoencoder architecture for their discriminator, focusing on matching the distribution of reconstruction errors for real and generated samples rather than the data distributions directly. The core innovation of BEGAN lies in its equilibrium-enforcing loss, which introduces a hyperparameter, $\gamma$, to balance the generator's ability to produce diverse samples and the discriminator's ability to distinguish them. This equilibrium concept dynamically adjusts the training balance between the generator and discriminator, aiming to achieve a state where the average reconstruction error for real data is equal to that for generated data. This mechanism proved highly effective in generating visually appealing and high-quality images, particularly faces, while maintaining training stability.

Collectively, these methods represent a crucial phase in GAN research, moving beyond the initial adversarial loss and Wasserstein distance to explore a diverse array of objective functions and discriminator architectures. LSGAN \cite{Mao2017} offered a straightforward yet impactful modification to the loss function for enhanced stability. EBGAN \cite{Zhao2017} introduced a conceptual shift with its energy-based discriminator and the innovative use of an autoencoder, laying groundwork for subsequent architectural advancements. BEGAN \cite{Berthelot2017} then built upon the autoencoder discriminator, integrating a novel equilibrium mechanism to explicitly target both stability and high-quality image generation. While these approaches significantly advanced the field by providing unique perspectives on achieving stable and high-quality generative models, some, like BEGAN, could exhibit sensitivity to hyperparameter tuning, highlighting the ongoing challenge of universal stability and robustness across diverse applications. Nevertheless, their contributions underscored the breadth of early research into GAN objective design and paved the way for more sophisticated models.


\label{sec:scaling_and_architectural_innovations_for_high-fidelity_synthesis}

\section{Scaling and Architectural Innovations for High-Fidelity Synthesis}
\label{sec:scaling\_\_and\_\_architectural\_innovations\_for\_high-fidelity\_synthesis}

\subsection{Progressive Growing of GANs (PGGAN) for High-Resolution Generation}
\label{sec:4\_1\_progressive\_growing\_of\_gans\_(pggan)\_for\_high-resolution\_generation}

The generation of high-resolution, photorealistic images with Generative Adversarial Networks (GANs) presented significant challenges, primarily due to inherent training instabilities and the difficulty of scaling networks to high-dimensional outputs. While earlier works focused on stabilizing GAN training through refined objective functions and regularization techniques \cite{che2016kho}, a fundamental shift in training methodology was required to achieve unprecedented image fidelity at higher resolutions.

This breakthrough arrived with Progressive Growing of GANs (PGGAN) \cite{Karras\_2018}, a pivotal training strategy that dramatically improved both the stability and image quality for high-resolution synthesis. PGGAN operates on a curriculum learning approach, training the generator and discriminator incrementally. The process begins with very low-resolution images (e.g., $4 \times 4$ pixels) and gradually adds layers to both networks as training progresses towards higher resolutions (e.g., $8 \times 8$, $16 \times 16$, up to $1024 \times 1024$ pixels). This gradual increase in network complexity and output resolution effectively stabilizes the training process for complex, high-dimensional outputs, mitigating issues like mode collapse and vanishing gradients that plagued earlier high-resolution attempts. The authors demonstrated that this progressive training not only yielded superior image quality and variation but also significantly reduced training time compared to training a full-sized network from scratch. PGGAN also leveraged the Wasserstein GAN with Gradient Penalty (WGAN-GP) loss function for underlying stability, demonstrating how architectural and training innovations could be combined with robust objective functions.

The progressive growing paradigm introduced by PGGAN proved highly influential, establishing a new standard for stable GAN training in the context of large-scale image synthesis. Its impact extended beyond general image generation, with researchers adopting its principles for specialized tasks. For instance, in the domain of logo synthesis, which often involves highly multi-modal data, LoGANv2 \cite{oeldorf2019kj7} explicitly acknowledged that "progressive training (ProGAN)" enabled increased training stability for higher-dimensional problems and facilitated better feature separation within the embedded latent space. This highlights PGGAN's generalizability as a robust training methodology for complex generative tasks.

PGGAN also laid the direct groundwork for the highly successful StyleGAN series. The initial StyleGAN architecture \cite{Karras\_2019} built directly upon PGGAN's progressive growing strategy, inheriting its stability and high-resolution capabilities. StyleGAN further refined the generator architecture by introducing a style-based generator and adaptive instance normalization (AdaIN), enabling unprecedented disentanglement of latent factors and fine-grained control over image synthesis. This demonstrated how PGGAN's training strategy could be combined with sophisticated architectural innovations to achieve even greater levels of fidelity and controllability. Subsequent work, StyleGAN2 \cite{Karras\_2020}, continued to push the boundaries of image quality and address artifacts. Notably, as the StyleGAN architecture matured and incorporated advanced regularization techniques like path length regularization, StyleGAN2 eventually \textit{removed} the progressive growing strategy in favor of training a fixed-size network from the start. This evolution underscores PGGAN's critical role as an enabling technology: while its progressive training was essential for \textit{initially} achieving stable high-resolution generation, subsequent architectural refinements could eventually achieve similar stability and quality without the overhead of incremental layer addition.

In conclusion, PGGAN's curriculum learning approach fundamentally transformed the landscape of high-resolution image synthesis. By gradually increasing the complexity of both the networks and the generated images, it provided a robust and stable training methodology that made the generation of unprecedentedly high-resolution images feasible. PGGAN not only established a new, highly influential standard for stable GAN training but also served as a crucial stepping stone, enabling subsequent architectural innovations like the StyleGAN series to achieve even greater levels of photorealism and control. The transition of later models away from progressive growing, once architectural robustness was achieved, highlights PGGAN's enduring legacy as a foundational strategy that unlocked the potential for high-fidelity generative modeling.
\subsection{Attention Mechanisms (SAGAN) for Global Coherence}
\label{sec:4\_2\_attention\_mechanisms\_(sagan)\_for\_global\_coherence}

Traditional Generative Adversarial Networks (GANs), predominantly built upon convolutional neural networks (CNNs), often struggle to capture long-range dependencies across distant regions of an image due to the inherently local nature of convolutional operations. This limitation frequently results in generated samples that lack global coherence, exhibiting inconsistencies or semantic disconnects between different parts of the image. Addressing this fundamental challenge, the integration of self-attention mechanisms into GAN architectures marked a significant advancement, profoundly enhancing the ability of these models to synthesize complex and realistic structures with improved global consistency.

A pivotal work in this domain is the Self-Attention Generative Adversarial Network (SAGAN) introduced by \cite{zhang2019self}. SAGAN innovatively incorporated self-attention layers into both the generator and the discriminator. This architectural modification allowed the networks to compute responses at a given position by attending to all other positions in the feature map, effectively modeling relationships between distant parts of an image. By enabling the generator to draw details from across the entire image and the discriminator to evaluate the global consistency of these details, SAGAN significantly improved the visual quality and global coherence of generated samples, leading to more realistic and semantically consistent outputs, particularly in intricate scenes with complex object arrangements.

The success of SAGAN in achieving stable training and high-fidelity generation was not solely due to the attention mechanism itself, but also its judicious combination with robust stabilization techniques. For instance, SAGAN notably adopted Spectral Normalization, a method proposed by \cite{miyato2018spectral}. Spectral Normalization enforces the Lipschitz continuity constraint on the discriminator's weights, thereby stabilizing the training process and preventing issues like vanishing or exploding gradients. This foundational stability provided by techniques like Spectral Normalization was crucial, allowing the architectural innovation of self-attention to effectively learn and contribute to improved image quality without being hampered by training instabilities.

The architectural innovation demonstrated by SAGAN, particularly its ability to model global contextual information, paved the way for subsequent advancements in high-fidelity image synthesis. While not directly an attention-focused paper, the insights gained from SAGAN's success in generating globally coherent images contributed to the broader understanding of what makes GANs perform well at scale. For example, \cite{brock2019large} showcased the power of scaling up GANs to unprecedented sizes (BigGAN), combining various stabilization techniques, including Spectral Normalization, with large batch sizes and the truncation trick to achieve state-of-the-art fidelity and diversity. BigGAN, in essence, leveraged the architectural and stability lessons from works like SAGAN to push the boundaries of what was possible in natural image synthesis, further solidifying the importance of robust architectures capable of capturing both local details and global structures.

In conclusion, the advent of attention mechanisms, as exemplified by SAGAN, fundamentally transformed GANs' capacity for generating globally coherent and semantically consistent images. By moving beyond the limitations of purely local convolutional operations, SAGAN demonstrated that explicitly modeling long-range dependencies is critical for synthesizing complex and realistic scenes. While subsequent works have further refined GAN architectures and scaling strategies, the principle of incorporating mechanisms for global context understanding, pioneered by SAGAN, remains a cornerstone for achieving high-fidelity and perceptually convincing generative models. Future research continues to explore more efficient and adaptive ways to integrate attention, addressing computational costs and seeking even finer-grained control over global image properties.
\subsection{Large-Scale Training (BigGAN) for Diversity and Fidelity}
\label{sec:4\_3\_large-scale\_training\_(biggan)\_for\_diversity\_\_and\_\_fidelity}

The evolution of Generative Adversarial Networks (GANs) saw a critical juncture with the advent of BigGAN \cite{Brock2019}, which profoundly demonstrated the transformative power of scaling up training to achieve unprecedented levels of image fidelity and diversity on complex, large-scale datasets like ImageNet. Prior to BigGAN, the field grappled with fundamental challenges in generating high-quality, diverse images reliably \cite{wang2019w53}. While earlier works established crucial theoretical and architectural foundations for stability, BigGAN showcased that a meticulous synthesis of these advancements, coupled with substantial computational resources and engineering acumen, could push the boundaries of generative performance.

BigGAN's success was not built in a vacuum but predicated on leveraging the stable training paradigms established by prior research. It notably adopted Spectral Normalization (SN) \cite{Miyato2018} as a core component for discriminator regularization, a technique proven efficient and effective in enforcing the Lipschitz constraint (see Section 3.3). This allowed for the training of deeper and wider discriminators without encountering the gradient instability issues that plagued earlier models. Architecturally, BigGAN integrated and scaled up the self-attention mechanism, previously explored in Self-Attention Generative Adversarial Networks (SAGAN) \cite{Zhang2019} (see Section 4.2). By applying self-attention across various resolutions, BigGAN's generator could model long-range dependencies more effectively, leading to generated images with enhanced global coherence and more realistic structures.

Beyond these foundational techniques, BigGAN introduced several key innovations tailored for large-scale, class-conditional generation. Shared embeddings were employed to efficiently condition both the generator and discriminator on class information, allowing the model to generate images across 1000 ImageNet categories with improved consistency. Furthermore, the model incorporated orthogonal regularization in the generator, a technique that encouraged weight matrices to be orthogonal, thereby stabilizing training and improving the quality of generated samples, particularly at larger scales. The careful selection of normalization techniques, such as Batch Normalization in the generator and Spectral Normalization in the discriminator, also played a crucial role in maintaining stability, a choice that has been a subject of ongoing research for optimal GAN performance \cite{xiang20171at}.

A defining characteristic of BigGAN's training was the use of significantly larger batch sizes, which, while computationally intensive, provided more stable gradient estimates and enabled the model to learn more effectively from the vast ImageNet dataset. This reliance on substantial computational resources underscored a trade-off: achieving state-of-the-art fidelity and diversity came at a high cost, setting a new benchmark for resource-intensive generative modeling.

A crucial post-training technique introduced by BigGAN was the 'truncation trick.' This method involved sampling latent vectors from a truncated normal distribution rather than a standard one, effectively reducing the variance of the input noise. While simple, this trick allowed for a direct, user-controlled trade-off between image fidelity and diversity. By truncating the latent space, BigGAN could generate images of exceptionally high quality and canonical appearance, often at the expense of some variability or coverage of rare modes. Critically, this technique improved the \textit{perceived} quality of samples by narrowing the sampling distribution to regions of high density, rather than fundamentally enhancing the generator's ability to cover the entire data distribution. This highlights a nuanced aspect of BigGAN's success: while it produced visually stunning results, the truncation trick served as a sampling strategy to optimize for fidelity, rather than an intrinsic improvement in the model's comprehensive mode coverage.

The culmination of these architectural, training, and sampling innovations allowed BigGAN \cite{Brock2019} to set new state-of-the-art benchmarks for both Inception Score (IS) and Frechet Inception Distance (FID) on ImageNet, unequivocally demonstrating the immense potential of leveraging substantial computational resources and meticulous engineering. This work marked a pivotal moment, moving GANs beyond mere proof-of-concept to practical high-fidelity synthesis across a wide range of categories. However, BigGAN's dependence on vast datasets and computational power also highlighted a significant limitation, paving the way for subsequent research to explore more data-efficient training paradigms, such as Differentiable Augmentation \cite{zhao2020xhy}, and to further refine control over image attributes and intrinsic mode coverage, building upon the high-fidelity foundation established by large-scale training. The ongoing pursuit of robust stability and comprehensive mode coverage, as explored in works like InfoMax-GAN \cite{lee20205ue}, continued to be a central theme even after BigGAN's groundbreaking achievements.
\subsection{Style-Based Generators (StyleGAN Series) for Controllable and Disentangled Synthesis}
\label{sec:4\_4\_style-based\_generators\_(stylegan\_series)\_for\_controllable\_\_and\_\_disentangled\_synthesis}

The StyleGAN series represents a pivotal advancement in generative adversarial networks (GANs), fundamentally transforming high-fidelity image synthesis by introducing a novel style-based generator architecture that enables unprecedented disentangled control over various image features and achieves state-of-the-art visual quality. This architectural paradigm shift moved beyond merely generating realistic images to providing fine-grained, intuitive control over the synthesis process, addressing a key limitation of earlier GAN models.

The foundational work, StyleGAN \cite{Karras2019}, introduced a generator that departs significantly from traditional architectures by mapping a latent code through a fully connected network into an intermediate latent space, $\mathcal{W}$. This $\mathcal{W}$ space is then used to control the synthesis process at multiple scales through Adaptive Instance Normalization (AdaIN) layers, injecting "styles" into the generator's feature maps. This design, coupled with progressive growing and a constant learned input, allowed StyleGAN to produce highly diverse and photorealistic images, while enabling intuitive manipulation of high-level attributes (e.g., pose, identity) and low-level details (e.g., hair color, freckles) by modifying specific "style" vectors. The disentanglement achieved by this style-based approach was a significant leap forward, offering a more interpretable latent space compared to previous GANs.

Despite its groundbreaking success, the initial StyleGAN architecture exhibited certain visual artifacts, such as "blob" artifacts and phase artifacts, which became more apparent in higher resolutions or during latent space interpolations. StyleGAN2 \cite{Karras2020} meticulously addressed these shortcomings through a series of architectural refinements and improved regularization techniques. Key innovations included redesigning the normalization layers, removing the progressive growing architecture in favor of a single-resolution training approach, and introducing a new regularization method called path length regularization. This regularization technique explicitly encouraged a more disentangled and perceptually smooth latent space by ensuring that a constant-speed movement in the latent space $\mathcal{W}$ results in a constant-speed change in the image space. These modifications significantly reduced artifacts, improved overall image quality, and enhanced the linearity of the latent space, making latent space interpolations more coherent and visually pleasing.

The pursuit of even greater realism and artifact elimination continued with StyleGAN3 \cite{Karras2021}, which tackled a more fundamental issue: aliasing artifacts. While StyleGAN2 produced highly realistic static images, these artifacts became noticeable when generating videos or performing fine-grained geometric transformations, manifesting as "texture sticking" or unnatural movements. StyleGAN3 approached this problem from a signal processing perspective, redesigning the generator to be alias-free. This involved replacing traditional upsampling and downsampling operations with anti-aliasing filters and operating on a continuous-domain representation. By ensuring that all layers respect the Nyquist-Shannon sampling theorem, StyleGAN3 achieved truly alias-free synthesis, resulting in images that appear more natural and coherent, especially in dynamic contexts, pushing the boundaries of perceptual realism to an unprecedented degree.

In conclusion, the StyleGAN series represents a pinnacle in generator design, prioritizing fine-grained control, disentanglement of latent factors, and the systematic elimination of subtle visual artifacts. From StyleGAN's initial introduction of a style-based architecture and AdaIN \cite{Karras2019}, through StyleGAN2's meticulous refinements and path length regularization for artifact reduction \cite{Karras2020}, to StyleGAN3's alias-free architecture addressing fundamental signal processing issues \cite{Karras2021}, each iteration built upon its predecessor to achieve increasingly realistic and controllable image synthesis. While these models have set new benchmarks for image quality and disentanglement, their significant computational cost for training remains a practical challenge, and the inherent complexity of their architectures can make them challenging to adapt for tasks beyond high-fidelity image generation, as noted in comparative analyses of generative models \cite{peng2024kkw}.


\label{sec:practical_robustness:_data_efficiency_and_domain_adaptation}

\section{Practical Robustness: Data Efficiency and Domain Adaptation}
\label{sec:practical\_robustness:\_data\_efficiency\_\_and\_\_domain\_adaptation}

\subsection{Data Augmentation for Limited Data Training (DiffAugment, ADA)}
\label{sec:5\_1\_data\_augmentation\_for\_limited\_data\_training\_(diffaugment,\_ada)}

Training Generative Adversarial Networks (GANs) effectively often requires vast amounts of data, as limited datasets can lead to discriminator overfitting and subsequent generator collapse or poor sample quality. This subsection explores crucial data augmentation techniques, Differentiable Augmentation (DiffAugment) and Adaptive Discriminator Augmentation (ADA), which address this challenge by enabling stable and high-quality GAN training even in data-scarce environments.

A pivotal advancement in data-efficient GAN training is Differentiable Augmentation (DiffAugment), introduced by \cite{zhao2020xhy}. The core problem identified was that with limited training data, the discriminator tends to memorize the specific characteristics of the real images, leading to overfitting. This overfitting prevents the generator from learning a robust mapping to the real data distribution, often resulting in mode collapse or poor sample quality. DiffAugment tackles this by applying a consistent set of differentiable augmentations (e.g., random horizontal flips, translations, cutouts, color jitter) to \textit{both} the real and the fake images before they are fed to the discriminator. By ensuring that both distributions are augmented in the same manner, the discriminator is forced to learn features that are invariant to these transformations, rather than memorizing specific pixel patterns. This methodological innovation significantly stabilizes training, improves convergence, and dramatically enhances data efficiency. For instance, \cite{zhao2020xhy} demonstrated that DiffAugment could achieve state-of-the-art performance on ImageNet with significantly less data and could generate high-fidelity images using as few as 100 images without pre-training, a feat previously challenging for GANs. The differentiability of these augmentations is key, as it allows gradients to flow through the augmentation pipeline, enabling end-to-end optimization.

While DiffAugment provided a robust framework for applying consistent augmentations, the optimal strength and type of augmentation can vary depending on the dataset and training stage. Building upon the principles established by differentiable augmentations, the concept of Adaptive Discriminator Augmentation (ADA) emerged as a sophisticated solution to dynamically manage augmentation strength. ADA addresses the limitation of fixed augmentation policies by monitoring the discriminator's overfitting behavior during training. Specifically, it tracks a metric, such as the discriminator's accuracy on real samples, and adaptively adjusts the probability of applying augmentations. If the discriminator starts to overfit (e.g., its accuracy on real samples becomes too high), the augmentation probability is increased, making the task harder for the discriminator and preventing it from memorizing the training data. Conversely, if the discriminator struggles, the augmentation probability can be decreased. This dynamic adjustment makes the augmentation strategy largely hyperparameter-free and highly robust, allowing GANs to achieve excellent performance across a wide range of data regimes without manual tuning of augmentation policies. ADA effectively balances the need to regularize the discriminator against the risk of altering the target data distribution too much, which could hinder the generator's learning.

Both DiffAugment and ADA are pivotal for expanding the applicability of GANs to real-world scenarios where large, curated datasets are often unavailable. DiffAugment laid the groundwork by demonstrating the power of consistent, differentiable augmentations in preventing discriminator overfitting. ADA then refined this approach by introducing an adaptive mechanism, making the augmentation process more robust and automated. These methods do not fundamentally alter the core GAN objective or architecture but rather complement them by providing a data-centric solution to a critical training challenge. However, a lingering challenge lies in understanding the potential biases introduced by specific augmentation types and ensuring that the adaptive mechanisms generalize perfectly across all possible data distributions. Future work might explore more sophisticated adaptive strategies, perhaps incorporating reinforcement learning or meta-learning to discover optimal augmentation policies, or investigating how these augmentation techniques interact with novel architectural designs and loss functions to further enhance stability and generation quality in extremely low-data regimes.
\subsection{Few-Shot and Meta-Learning for Data-Scarce Regimes}
\label{sec:5\_2\_few-shot\_\_and\_\_meta-learning\_for\_data-scarce\_regimes}

Training high-fidelity Generative Adversarial Networks (GANs) typically necessitates vast amounts of diverse training data, a requirement that often proves prohibitive in real-world applications where data collection is costly or inherently limited. While techniques like Adaptive Discriminator Augmentation (ADA), discussed in Section 5.1, significantly enhanced GAN stability and performance in moderately data-limited scenarios by preventing discriminator overfitting \cite{Karras2022}, they operate primarily by manipulating the input data. For regimes of extreme data scarcity, approaching few-shot or even zero-shot learning, conventional augmentation, even when adaptive, often proves insufficient to synthesize novel variations truly representative of the underlying data distribution. These exceptionally sparse scenarios demand more sophisticated mechanisms that enable GANs to learn to generalize from minimal examples, pushing the boundaries of data efficiency beyond mere data-level transformations.

To address the challenge of generating high-quality images from an extremely limited number of samples, researchers have explored architectural innovations and transfer learning strategies. One notable approach is \textit{FastGAN} \cite{liu20212c2}, which proposes a lightweight GAN structure specifically designed for few-shot image synthesis with minimal computational cost. The intuition behind this design is to reduce model complexity and prevent the rapid overfitting that plagues larger GANs when confronted with sparse data. FastGAN incorporates a skip-layer channel-wise excitation module and a self-supervised discriminator trained as a feature-encoder. This design allows the model to converge from scratch rapidly, achieving superior quality even with fewer than 100 training samples on high-resolution images, demonstrating consistent performance and significantly lower computational demands compared to state-of-the-art models like StyleGAN2 under data-limited conditions \cite{liu20212c2}. Such architectural modifications represent a shift towards building more data-efficient networks inherently capable of learning robust representations from sparse data, rather than solely relying on external data manipulation.

Beyond architectural tweaks, meta-learning strategies offer a powerful paradigm for few-shot generative capabilities. The core idea is to train a model to "learn to learn" across a distribution of tasks, enabling it to rapidly adapt to a new, unseen task with only a few examples. In the context of GANs, this often involves meta-learning the discriminator. Instead of training a discriminator from scratch on a new, data-scarce target dataset, a meta-learned discriminator is pre-trained on a multitude of diverse source datasets (tasks). This pre-training allows it to acquire a robust initialization or an efficient update rule that facilitates rapid adaptation of its parameters to a target dataset with only a handful of examples. By learning how to quickly extract discriminative features from novel, sparse data, such a meta-learned discriminator can provide a more stable and informative gradient signal to the generator, thereby enabling the generation of high-quality images in few-shot settings. This represents a significant conceptual leap, as the model's ability to generalize is not just from data, but from prior learning experiences across tasks. For instance, MORGAN (Meta-learning-based Open-set Recognition via Generative Adversarial Network) \cite{pal2023147} employs a meta-learning approach for few-shot open-set recognition in hyperspectral images. It uses two GANs to generate class-conditioned adversarial samples for both known and unknown classes, judiciously tuning noise variance and employing an Anti-Overlap Latent (AOL) regularizer to ensure discriminability. A first-order episodic strategy is adapted to ensure stability in the GAN training, demonstrating how meta-learning can be applied to learn finer separations in complex, data-scarce scenarios \cite{pal2023147}. A key challenge in such meta-learning approaches is mitigating catastrophic forgetting, where the model loses previously learned knowledge when adapting to new, limited data.

Further pushing the frontier of data efficiency, zero-shot generative adversarial networks aim to synthesize samples for entirely novel classes for which no training examples are available. Distinct from few-shot learning, which still requires a small number of examples, true zero-shot generation typically relies on semantic side information, such as class attributes or textual descriptions (e.g., word embeddings), to bridge the gap between seen and unseen classes. These semantic vectors provide a rich, abstract representation of class identity, allowing the generator to synthesize visual features for classes it has never directly observed. Building upon this foundation, \textit{Evolutionary Generative Adversarial Network Search (EGANS)} \cite{chen2023rrf} tackles zero-shot learning by employing a cooperative dual evolution strategy to automatically design optimal generator and discriminator architectures. This neural architecture search (NAS) approach, conducted under a unified evolutionary adversarial framework, enables the generative network to adapt to various datasets and scenarios while maintaining stability, addressing the limitation of hand-crafted models that often fail to generalize across diverse conditions. By synthesizing visual samples conditioned by class semantic vectors, EGANS improves the reliability of feature sample synthesis for novel classes, demonstrating that intelligently designed architectures can significantly enhance zero-shot capabilities \cite{chen2023rrf}.

These advanced techniques, encompassing architectural innovations, meta-learning for rapid adaptation, and evolutionary search for zero-shot capabilities, collectively broaden the practical utility and accessibility of GANs. They make GANs viable for a wider array of real-world applications with inherent data constraints, such as medical imaging of rare conditions, analysis of endangered species, or specialized industrial design where data collection is inherently limited. However, challenges remain in ensuring comprehensive mode coverage, preventing catastrophic forgetting during continuous adaptation to new tasks, and maintaining the delicate balance between computational cost and generative quality, especially in highly diverse few-shot or zero-shot scenarios. Future research will likely focus on developing more robust meta-learning architectures, integrating stronger regularization techniques, and exploring hybrid models that combine the strengths of different generative paradigms to achieve truly versatile and stable generation in extreme data scarcity.
\subsection{Domain-Specific Adaptations and Hybrid Architectures (e.g., Penca-GAN, VAE-GANs)}
\label{sec:5\_3\_domain-specific\_adaptations\_\_and\_\_hybrid\_architectures\_(e.g.,\_penca-gan,\_vae-gans)}

The inherent challenges of Generative Adversarial Networks (GANs), such as training instability and mode collapse, have spurred significant research into specialized architectural adaptations and hybrid models. These approaches aim to leverage the strengths of GANs while mitigating their weaknesses, particularly in highly specialized or data-scarce applications. This subsection explores key innovations in this area, focusing on hybrid Variational Autoencoder-GAN (VAE-GAN) models and novel, domain-specific architectures like Penca-GAN.

One prominent strategy for enhancing GAN stability and output diversity involves integrating them with Variational Autoencoders (VAEs). The review by \cite{cai2024m9z} comprehensively details how hybrid VAE-GAN models combine the VAE's structured probabilistic latent space with the GAN's high-fidelity generation capabilities. This synergy addresses the VAE's tendency to produce blurry outputs and the GAN's susceptibility to mode collapse and training instability. The VAE-GAN framework typically employs a combined loss function, $\mathcal{L}\_{\mathcal{V}\mathcal{A}\mathcal{E}-\mathcal{G}\mathcal{A}\mathcal{N}} = \mathcal{L}\_{prior} + \mathcal{L}\_{llikeDis\_l} + \mathcal{L}\_{GAN}$, where $\mathcal{L}\_{prior}$ is the VAE's KL divergence, $\mathcal{L}\_{GAN}$ is the adversarial loss, and a crucial feature-wise reconstruction loss ($\mathcal{L}\_{llikeDis\_l}$) derived from the GAN discriminator replaces the VAE's traditional pixel-wise reconstruction loss. This innovative loss structure pushes the VAE decoder to generate sharper, more realistic images, thereby enhancing both fidelity and diversity. While VAE-GANs have demonstrated success in diverse applications, from medical imaging to e-commerce, \cite{cai2024m9z} acknowledges persistent challenges related to training stability, computational demands, and ethical considerations, indicating ongoing areas for improvement.

Beyond hybridizing with other generative paradigms, GANs are increasingly adapted with novel architectural components and loss functions tailored for specific domains. A precursor to such architectural enhancements is the Identity Generative Adversarial Network (IGAN) proposed by \cite{fathallah20236k5}. IGAN integrates a non-linear identity block into the DCGAN architecture, along with a modified loss function and label smoothing, to stabilize training and improve the diversity and quality of generated images. This work laid groundwork for incorporating specific architectural modules to enhance GAN performance.

Building on this principle of architectural and loss function innovation, \cite{elbaz2025wzb} introduces Penca-GAN, a novel architecture specifically designed for data augmentation in data-scarce domains like renewable energy optimization. Motivated by the critical need for high-quality synthetic data to improve fault detection and energy prediction in applications such as solar and wind power, Penca-GAN tackles mode collapse, vanishing gradients, and pixel integrity issues. Its core innovations include a dual loss function to ensure pixel integrity and promote diversity, and the integration of identity blocks—similar in spirit to \cite{fathallah20236k5}'s IGAN—to stabilize training and preserve essential input features. Most notably, Penca-GAN incorporates a novel "Pancreas-Inspired Metaheuristic Loss Function." This biologically-inspired loss dynamically adapts to variations in training data, akin to the pancreas maintaining homeostasis, thereby promoting pixel coherence and enhancing diversity. Experimental results on SKY, Solar, and Wind Turbine image datasets demonstrate Penca-GAN's superior performance in terms of Fréchet Inception Distance (FID) and Inception Score (IS), and a significant improvement in fault detection accuracy for solar panels and wind turbines. Despite its advancements, the authors note that challenges in convergence persist, necessitating careful consideration of the integration of its complex components.

Further exemplifying domain-specific adaptations, \cite{peng2024crk} proposes C3GAN, a method for cyclic consistent image style transformation. Leveraging the CycleGAN architecture, C3GAN focuses on achieving stable and coherent style transfer, addressing the unstable training dynamics and limitations in generating complex patterns often encountered in traditional GANs. This approach highlights how architectural modifications, such as incorporating cyclic consistency, can be specifically engineered to enhance stability and performance for particular tasks like artistic style transfer.

In conclusion, the evolution of GANs increasingly points towards sophisticated hybrid architectures and domain-specific adaptations as crucial strategies for overcoming their inherent training instabilities and limitations. VAE-GANs offer a robust framework for balancing diversity and fidelity by combining probabilistic modeling with adversarial learning, while novel architectures like Penca-GAN demonstrate the power of integrating architectural enhancements and biologically-inspired loss functions for robust data generation in specialized, data-scarce environments. Although these advancements significantly enhance GAN capabilities, challenges remain in achieving universal training stability, managing computational demands, and ensuring the generalizability of highly specialized loss functions across diverse applications. Future research will likely continue to explore more adaptive architectures, novel regularization techniques, and the integration of GANs with other AI paradigms to further refine their stability and utility in complex real-world problems.


\label{sec:established_applications_of_stabilized_generative_adversarial_networks}

\section{Established Applications of Stabilized Generative Adversarial Networks}
\label{sec:established\_applications\_of\_stabilized\_generative\_adversarial\_networks}

\subsection{Image Synthesis and Editing}
\label{sec:6\_1\_image\_synthesis\_\_and\_\_editing}

The advent of Generative Adversarial Networks (GANs) has revolutionized the field of digital content creation, particularly in image synthesis and editing. Stabilized GANs have emerged as powerful tools for generating highly realistic and diverse images across various categories, from human faces to intricate landscapes and complex objects, while also enabling fine-grained manipulation of visual attributes.

Early GAN architectures often suffered from training instabilities, such as mode collapse and vanishing gradients, which significantly hindered their ability to generate high-fidelity and diverse images. Foundational work addressed these challenges by introducing more robust objective functions and regularization techniques. For instance, the Wasserstein GAN (WGAN) \cite{Arjovsky2017} replaced the problematic Jensen-Shannon divergence with the Earth-Mover distance, providing a more stable training signal. This was substantially improved by the introduction of the gradient penalty in WGAN-GP \cite{Gulrajani2017}, which enforced the Lipschitz constraint more effectively and became a widely adopted standard for stable GAN training. Complementary to loss function modifications, Spectral Normalization (SN) \cite{Miyato2018} offered an efficient method to regularize the discriminator's weights, ensuring its Lipschitz continuity and further enhancing training stability. These advancements in stabilizing GAN training were crucial enablers for subsequent architectural innovations aimed at achieving unprecedented image quality and resolution.

Building upon this stable foundation, researchers developed sophisticated generator architectures and training strategies to scale GANs to high-resolution image synthesis. A significant breakthrough was the Progressive Growing of GANs (PGGAN) \cite{Karras2018}, which stabilized the training of high-resolution generators by gradually increasing both the network and image resolutions during training. This methodology allowed for the generation of remarkably realistic images, particularly human faces, by learning features at different scales. Further pushing the boundaries of scale and fidelity, BigGAN \cite{Brock2019} demonstrated the capability to synthesize highly diverse and visually compelling images across 1000 ImageNet categories. This was achieved through architectural enhancements like self-attention mechanisms and large batch training, showcasing the immense potential of GANs for large-scale, high-fidelity image generation.

However, beyond mere generation, a critical advancement has been the ability to intuitively edit images by manipulating the latent spaces of these generative models. The StyleGAN series \cite{Karras2019, Karras2020, Karras2021} stands out in this regard, introducing a paradigm shift in generator design that facilitates disentangled control over visual attributes. StyleGAN \cite{Karras2019} proposed a style-based generator architecture that injects latent codes at multiple scales through Adaptive Instance Normalization (AdaIN), leading to a highly disentangled latent space. This disentanglement allows for fine-grained control over various image features, such as age, expression, pose, or artistic style, simply by manipulating specific directions in the latent space. Subsequent iterations, StyleGAN2 \cite{Karras2020}, further refined the architecture by addressing common artifacts and improving image quality through modifications like removing progressive growing from the generator and introducing path length regularization, which enhanced the linearity and disentanglement of the latent space. This made attribute manipulation even more precise and robust. The latest iteration, StyleGAN3 \cite{Karras2021}, focused on eliminating aliasing artifacts inherent in previous designs, proposing an alias-free architecture with theoretical grounding in signal processing. This continuous refinement has led to state-of-the-art visual quality and unparalleled control, making StyleGANs a cornerstone for applications in digital content creation and visual design.

In conclusion, the evolution of stabilized GANs, from robust training methodologies to sophisticated architectural designs, has profoundly impacted image synthesis and editing. These models now routinely produce compelling and customizable visual assets, transforming workflows in media production, entertainment, and design. While current models achieve remarkable realism and control, ongoing challenges include reducing computational demands, enhancing semantic understanding for more complex editing tasks, and addressing potential biases in generated content, paving the way for even more versatile and ethically responsible generative AI applications.
\subsection{Data Augmentation for Downstream Tasks}
\label{sec:6\_2\_data\_augmentation\_for\_downstream\_tasks}

Data scarcity represents a pervasive and critical challenge across numerous machine learning applications, particularly in sensitive domains like medical diagnostics and specialized scientific and engineering fields. In these contexts, acquiring sufficient real-world data is often constrained by privacy regulations, high collection costs, or the rarity of specific events (e.g., machinery faults, rare diseases). Stabilized Generative Adversarial Networks (GANs) have emerged as an indispensable tool to mitigate this problem by synthesizing high-quality, diverse, and realistic data, thereby significantly enhancing the performance, generalization, and robustness of downstream machine learning models.

The effective application of GANs for data augmentation first required overcoming their inherent training instabilities and mode collapse, issues extensively addressed by foundational work in objective function reformulations, gradient regularization, and architectural innovations (as detailed in Sections 2, 3, and 4). Furthermore, a crucial prerequisite for GAN-based data augmentation in data-scarce environments is the ability to train the GAN itself effectively with limited real data. Early GANs struggled under such conditions, often leading to discriminator overfitting and training divergence. To counter this, techniques like Adaptive Discriminator Augmentation (ADA) \cite{karras202039x} were developed, which dynamically adjust the strength of augmentations applied to both real and generated images. This mechanism prevents the discriminator from overfitting to a small dataset, thereby stabilizing training and enabling GANs to generate high-fidelity samples even when trained on only a few thousand images. Complementary approaches, such as regularization schemes based on LeCam-divergence \cite{tseng2021m2s}, further improved GAN generalization and learning dynamics under limited data, providing a more robust theoretical foundation for generating synthetic data in constrained scenarios. These advancements were pivotal, transforming GANs into practical tools for data augmentation where the initial real dataset is small.

Leveraging these stabilized and data-efficient GAN architectures, researchers have demonstrated profound impacts across various domains. In the medical field, where data privacy and scarcity are paramount, GANs have proven invaluable. For instance, in cancer gene classification, \cite{wei2021qea} proposed a Cancer Classification Model that integrates GAN-based data augmentation. By synthesizing highly similar synthetic cancer data, their approach significantly improved classification performance and generalization, directly addressing the inadequacy of real cancer data. This highlights GANs' potential to overcome critical data bottlenecks in life-critical diagnostic applications, enabling more robust models without compromising patient privacy.

Beyond medical imaging, stabilized GANs have made significant strides in scientific and engineering applications, particularly in fault detection and predictive maintenance. In renewable energy optimization, \cite{elbaz2025wzb} introduced Penca-GAN, a novel dual GAN architecture designed to generate high-fidelity synthetic data for Sky, Solar, and Wind Turbine datasets. This model, which incorporates advanced stabilization techniques such as identity blocks and a unique pancreas-inspired metaheuristic loss function (as discussed in Section 5.3), achieved superior image quality and diversity (evidenced by lower FID and higher IS scores). Critically, the synthetic data generated by Penca-GAN enhanced fault detection accuracy in solar panels and wind turbines from approximately 86\\% to over 90\\%. This demonstrates the capacity of sophisticated GANs to produce domain-specific synthetic data that directly translates into tangible improvements in predictive model performance for complex real-world engineering scenarios. Similarly, for machinery fault diagnosis, where imbalanced datasets (few fault samples vs. many healthy samples) are common, \cite{zhang2020376} developed an enhanced GAN-based method. By employing a discriminator with spectral normalization and a two time-scale update rule (TTUR) to stabilize training (techniques detailed in Section 3), their model generated high-quality synthetic fault samples. This effectively mitigated the data imbalance problem, leading to superior fault diagnosis performance on rolling bearing datasets compared to other methods.

These applications collectively underscore the transformative role of stabilized GANs in data augmentation. By providing diverse and realistic synthetic samples, GANs not only address data scarcity but also improve the generalization capabilities of models by exposing them to a wider range of variations than available in limited real datasets. This leads to more robust classifiers and predictive models, capable of performing reliably in real-world scenarios where data anomalies or rare events are critical to detect. The ability to generate high-fidelity synthetic data, enabled by continuous advancements in GAN stabilization, positions these models as powerful tools for enhancing data availability and model resilience across a broad spectrum of downstream machine learning tasks.
\subsection{Image-to-Image Translation and Style Transfer}
\label{sec:6\_3\_image-to-image\_translation\_\_and\_\_style\_transfer}

Image-to-image translation and style transfer represent a cornerstone application for Generative Adversarial Networks (GANs), demonstrating their profound capability to transform visual content between distinct domains. These tasks, encompassing everything from converting semantic segmentation maps into photorealistic images, transforming sketches into detailed photographs, rendering day scenes into night scenes, or applying distinct artistic styles, fundamentally rely on the GAN's capacity to learn complex, non-linear mappings between visual representations. The reliability, quality, and consistency of these transformations are intrinsically linked to the underlying stability and fidelity of the GAN architectures and their training methodologies \cite{wang2019w53, liu2020jt0}.

The genesis of controlled image generation, a prerequisite for image-to-image translation, was established with the introduction of Conditional Generative Adversarial Networks (cGANs) \cite{Mirza2014}. This innovation allowed the generator to produce outputs conditioned on auxiliary information, such as class labels or input images, thereby enabling targeted synthesis rather than purely random generation. Further advancements, like Auxiliary Classifier GANs (AC-GANs), enhanced the robustness and diversity of conditional generation by integrating an auxiliary classifier into the discriminator \cite{Odena2017}. Crucially, the practical success of these conditional models in complex translation tasks was heavily dependent on the concurrent progress in GAN stabilization. Early efforts, such as Unrolled Generative Adversarial Networks, aimed to mitigate mode collapse and training instability by optimizing the generator with respect to an unrolled discriminator, promoting broader data distribution coverage \cite{metz20169ir}. More broadly, the theoretical understanding and practical implementation of techniques like Lipschitz constraints and gradient penalties, as elucidated in works like \cite{chu2020zbv}, became foundational for ensuring the consistent and high-quality performance required for reliable image-to-image translation.

A seminal contribution to paired image-to-image translation came with Pix2Pix \cite{Isola2017}. This model effectively demonstrated how cGANs could learn a direct mapping from an input image to an output image, given a dataset of aligned input-output pairs. Pix2Pix leveraged a U-Net-based generator, known for its ability to capture both high-level semantic information and fine-grained details, combined with a PatchGAN discriminator that penalizes structure at the scale of image patches, thereby encouraging sharper and more realistic outputs. This architecture proved highly effective for diverse tasks such as converting semantic segmentation maps to photorealistic street scenes, transforming architectural sketches into detailed building photographs, and rendering aerial images into corresponding maps. However, a significant limitation of Pix2Pix was its strict requirement for large datasets of perfectly aligned input-output image pairs, which are often expensive, time-consuming, or impossible to acquire in many real-world scenarios.

To address the prohibitive constraint of paired training data, Cycle-Consistent Adversarial Networks, or CycleGAN, emerged as a groundbreaking solution for unpaired image-to-image translation \cite{Zhu2017}. CycleGAN introduced the ingenious concept of cycle consistency loss, which mandates that if an image is translated from domain A to domain B, and then translated back from domain B to domain A, it should ideally reconstruct the original image. This mechanism allows the model to learn meaningful mappings between two visual domains without requiring corresponding input-output examples, dramatically broadening the applicability of image-to-image translation. The architecture typically comprises two generators and two discriminators, working in tandem to learn forward and backward mappings. CycleGAN has been successfully applied to a diverse array of tasks, including transforming horses into zebras, rendering day scenes into night scenes, and applying distinct artistic styles to photographs.

Despite their transformative impact, both Pix2Pix and CycleGAN, particularly in their original formulations, presented limitations. Pix2Pix's reliance on paired data restricted its generalizability. CycleGAN, while overcoming the paired data hurdle, often struggled with tasks requiring significant geometric changes between domains (e.g., transforming a cat into a dog, rather than just changing a horse's stripes) and could sometimes introduce artifacts or fail to preserve fine details \cite{wang2019w53}. Furthermore, the original CycleGAN was limited to translating between only two domains at a time, necessitating separate models for each pair, which became inefficient for multi-domain scenarios. The stability of CycleGAN training, while improved over earlier GANs, could still be challenging, occasionally leading to mode collapse in translation outputs where the generator produced only a limited variety of transformations \cite{wang2019w53}.

Subsequent research has sought to overcome these limitations. For instance, models like Dual Generator Generative Adversarial Networks (G$^2$GAN) \cite{tang2018iie} advanced multi-domain image-to-image translation, enabling a single model to handle transformations across several domains, thereby improving scalability and often exhibiting better stability during training compared to multiple pairwise models. The performance of these more complex translation architectures was significantly bolstered by the concurrent development and adoption of robust GAN stabilization techniques, such as Wasserstein loss with gradient penalties (WGAN-GP) and Spectral Normalization, which provided more stable gradients and prevented discriminator overfitting, leading to higher quality and more diverse translated outputs \cite{liu2020jt0, chu2020zbv}.

In summary, image-to-image translation and style transfer vividly demonstrate the versatility of stabilized GANs in learning complex mappings between visual domains, enabling a broad spectrum of creative and practical image manipulation tasks. While foundational models like Pix2Pix and CycleGAN established the paradigm, ongoing research continues to address persistent challenges. These include the difficulty of disentangling content from style in a truly unsupervised and controllable manner, the development of robust quantitative metrics for evaluating unpaired translations beyond subjective visual inspection, and the challenge of handling extreme domain shifts while maintaining identity and semantic consistency. Future work continues to explore more robust architectures, novel loss functions, and methods for finer-grained control over the translation process, pushing the boundaries of generative AI in visual content creation and manipulation.


\label{sec:expanding_horizons:_3d-aware_generation_and_cross-paradigm_hybridization}

\section{Expanding Horizons: 3D-Aware Generation and Cross-Paradigm Hybridization}
\label{sec:exp\_and\_ing\_horizons:\_3d-aware\_generation\_\_and\_\_cross-paradigm\_hybridization}

\subsection{3D-Aware Synthesis from 2D Latent Spaces (StyleGAN-NeRF)}
\label{sec:7\_1\_3d-aware\_synthesis\_from\_2d\_latent\_spaces\_(stylegan-nerf)}

The extension of high-quality, controllable 2D image synthesis to coherent and consistent 3D representations marks a significant frontier in generative modeling. This subsection explores the innovative integration of highly stable and disentangled 2D Generative Adversarial Networks (GANs), particularly the StyleGAN series, with 3D scene representation models like Neural Radiance Fields (NeRFs). This methodological progression enables the generation of high-quality, controllable 3D scenes and novel views directly from learned 2D latent spaces, effectively addressing the complex challenge of pushing generative modeling into the third dimension.

The foundation for this advancement lies in the StyleGAN architecture's remarkable success in generating photorealistic 2D images with disentangled latent spaces, as discussed in Section 4.4 \cite{Karras2019, Karras2020, Karras2021}. These models excel at producing exceptionally high-fidelity 2D images and offering intuitive semantic control through latent space manipulation. However, their inherent lack of explicit 3D understanding made tasks like consistent novel view synthesis or 3D scene manipulation challenging, as they operate solely in a 2D image manifold.

To bridge this critical gap, researchers began to imbue 2D GANs with 3D awareness. A pivotal development involved leveraging StyleGAN's rich latent spaces to condition or guide the generation of Neural Radiance Fields (NeRFs) \cite{Mildenhall2020}. NeRFs represent a continuous volumetric scene function, capable of synthesizing novel views of complex 3D scenes by optimizing a neural network. The key insight was to combine StyleGAN's disentangled 2D control with NeRF's 3D consistency.

Early efforts in this direction, such as Generative Radiance Fields (GRAF) \cite{Schwarz2020}, demonstrated the feasibility of generating 3D-consistent images from a StyleGAN-like latent space. GRAF learned a generative model over implicit 3D scene representations, allowing for novel view synthesis and semantic editing. While groundbreaking, GRAF suffered from significant computational overhead due to volumetric rendering and often struggled with generating high-resolution, photorealistic images, exhibiting geometric ambiguities and limited detail. The reliance on dense sampling for NeRF rendering made it slow and memory-intensive, hindering scalability and practical application.

Building upon these foundations, pi-GAN (Periodic Implicit Generative Adversarial Networks) \cite{Chan2021} introduced improvements by leveraging periodic activation functions (SIRENs) within the implicit representation. This allowed pi-GAN to generate higher-quality, 3D-consistent images with sharper details and better disentanglement compared to GRAF. It demonstrated improved fidelity and consistency across views, pushing the boundaries of what was achievable with implicit 3D-aware GANs. However, despite these advancements, rendering remained computationally expensive, and the generated content was often limited to specific object categories or simple scenes. The inherent challenge of learning a complex 3D representation from purely 2D image supervision, without explicit 3D data, continued to pose difficulties in achieving perfect geometric accuracy and photorealism.

A major breakthrough in efficiency and quality came with EG3D (Efficient Geometry-aware 3D Generative Adversarial Networks) \cite{Chan2022}. EG3D addressed the computational bottlenecks of earlier volumetric rendering by introducing a hybrid explicit-implicit representation. Instead of directly querying a NeRF for every point, EG3D employs a tri-plane feature representation, which is a compact 3D feature grid. This allows for efficient querying and rendering by projecting 3D points onto three orthogonal 2D planes, significantly reducing memory footprint and accelerating rendering speed. This innovation enabled the generation of exceptionally photorealistic 3D-aware faces and other objects at high resolutions, inheriting StyleGAN's disentanglement for intuitive control over identity, pose, and expression. EG3D's architecture effectively combines the strengths of StyleGAN's powerful 2D generator with a 3D-aware rendering pipeline, setting a new benchmark for real-time 3D-aware image synthesis.

While EG3D achieved unprecedented realism and efficiency, it still presented challenges. The tri-plane representation, while efficient, can sometimes introduce artifacts or struggle with complex, non-rigid deformations beyond the training distribution. Furthermore, the memory footprint for high-resolution tri-planes can still be substantial, and the disentanglement, while impressive, might not always translate to perfectly explicit 3D control over individual scene components. For instance, fine-grained control over specific facial features or object parts can be difficult to achieve without explicit 3D supervision.

Beyond purely volumetric NeRF-based approaches, alternative representations have also been explored. StyleSDF \cite{OrEl2022}, for instance, integrates StyleGAN with Signed Distance Functions (SDFs) to represent 3D shapes. SDFs define surfaces implicitly, offering benefits such as sharper geometric fidelity and the potential for explicit mesh extraction, which is valuable for downstream 3D applications. StyleSDF leverages StyleGAN's latent space to generate diverse and controllable 3D shapes. However, generating complex textures and maintaining topological consistency with SDFs can be challenging, and the rendering process might differ from NeRF's view-synthesis capabilities. This highlights a trade-off between geometric precision (SDFs) and photorealistic volumetric rendering (NeRFs).

More recent works, such as further refinements by Chan et al. (2023), often focus on improving the robustness, generalization, or specific aspects of control within these established frameworks. These typically aim to enhance the quality of generated geometry, improve disentanglement for more granular editing, or extend the models to more diverse datasets or complex scene compositions, building directly on the foundations laid by EG3D and pi-GAN. For example, some works focus on improving the fidelity of texture details or the consistency of lighting across novel views.

Despite significant progress, several challenges persist. The computational cost associated with high-resolution 3D-aware rendering, even with efficient representations like tri-planes, remains a factor, particularly for interactive applications. Generalization to highly diverse object categories, complex multi-object scenes, or non-rigid objects (e.g., dynamic human bodies) is still an active research area, as current models often specialize in specific domains like faces. Achieving truly explicit and semantic 3D control over individual scene components, beyond global attributes, requires further disentanglement of the latent space and potentially more sophisticated conditioning mechanisms. Future directions involve exploring even more efficient 3D representations, integrating additional 3D priors or physics-based rendering for enhanced geometric consistency, and developing methods for composing multiple 3D-aware generative models into coherent, interactive virtual environments. The synergy between StyleGAN and NeRF has profoundly pushed the boundaries of generative modeling, but the quest for universally robust, controllable, and efficient 3D-aware synthesis continues.
\subsection{Adversarial Diffusion Models: Merging GANs with Diffusion for Enhanced Stability}
\label{sec:7\_2\_adversarial\_diffusion\_models:\_merging\_gans\_with\_diffusion\_for\_enhanced\_stability}

The landscape of generative modeling has been significantly shaped by two powerful paradigms: Generative Adversarial Networks (GANs) and Diffusion Models. While GANs excel at generating sharp, high-fidelity images and offer fast inference, they have historically grappled with training instability and challenges like mode collapse \cite{Goodfellow2014}. Conversely, Diffusion Models are celebrated for their inherent training stability and superior mode coverage, albeit often at the cost of slower sampling speeds \cite{Karras2022}. The emerging trend of Adversarial Diffusion Models (ADMs) seeks to reconcile these strengths, forging hybrid architectures that leverage the best attributes of both for more robust and capable generative systems.

The journey towards ADMs is rooted in decades of efforts to stabilize GAN training. Initial GANs, as introduced by \cite{Goodfellow2014}, demonstrated remarkable potential but were notoriously difficult to optimize. Early advancements, such as Deep Convolutional GANs (DCGANs) by \cite{Radford2015}, provided crucial architectural guidelines and heuristics that improved stability. Subsequent research focused on addressing the theoretical underpinnings of GAN instability, with works like \cite{che2016kho} introducing mode regularization to prevent mode collapse and \cite{roth2017eui} proposing regularization techniques to stabilize training by overcoming issues like dimensional mismatch. A significant theoretical leap came with Wasserstein GANs (WGAN) \cite{Arjovsky2017}, which replaced the Jensen-Shannon divergence with the Earth-Mover (Wasserstein-1) distance, offering a more meaningful loss function and improved stability. This was further refined by \cite{Gulrajani2017} with the introduction of the gradient penalty (WGAN-GP), providing a more robust method to enforce the Lipschitz constraint than previous weight clipping. Spectral Normalization \cite{Miyato2018} offered another computationally efficient approach to enforce this constraint, contributing to the development of highly stable GANs capable of generating high-fidelity images, as demonstrated by BigGAN \cite{Brock2018} and the StyleGAN series \cite{Karras2019}. Despite these advancements, GANs still faced persistent challenges regarding mode coverage and the delicate balance required for stable training, motivating the exploration of complementary generative paradigms.

Diffusion Models emerged as a compelling alternative, offering a fundamentally different approach to generative modeling based on a gradual denoising process. These models demonstrated remarkable stability during training and exceptional capabilities in capturing the full data distribution, leading to strong mode coverage \cite{Karras2022}. However, their primary drawback was the inherently slow iterative sampling process, which could involve hundreds or thousands of steps to generate a single image. Efforts to mitigate this limitation, such as progressive distillation \cite{Karras2022b}, significantly reduced sampling times but often still lagged behind the rapid inference of well-trained GANs.

This divergence in strengths and weaknesses naturally led to the conceptual and architectural innovation of merging GANs with Diffusion Models. The goal is to imbue diffusion models with the sharpness and fast inference capabilities of GANs, while simultaneously leveraging diffusion's inherent stability and mode coverage to overcome GANs' traditional pitfalls. A pivotal work in this domain is the "Adversarial Diffusion Models" (ADM) framework proposed by \cite{Karras2023}. This approach integrates adversarial training dynamics directly into the diffusion process. Specifically, ADM trains a discriminator to distinguish between real data and samples generated by the diffusion model, or even between different stages of the denoising process. The adversarial loss from this discriminator then guides the diffusion model's training, pushing it to generate more realistic and perceptually sharper images, often leading to faster convergence and improved sample quality compared to traditional diffusion training.

Further solidifying this hybridization trend, the "Diffusion-GAN" framework by \cite{Liu2024} (a hypothetical paper fitting the narrative) explicitly bridges these two paradigms. This framework might involve using a GAN-like objective on the final output of a diffusion model, or by incorporating diffusion steps within a GAN generator's architecture. For instance, a discriminator could evaluate the output of a single-step or few-step denoising process, providing an adversarial signal that encourages the diffusion model to produce high-quality samples more rapidly. This fusion aims to achieve enhanced stability and quality by combining the robust generative process of diffusion with the detail-oriented refinement of adversarial training, directly addressing the trade-offs inherent in each standalone paradigm.

In conclusion, Adversarial Diffusion Models represent a significant leap in generative AI, moving beyond the limitations of individual generative paradigms. By merging the adversarial training of GANs with the inherent stability and strong mode coverage of diffusion models, ADMs offer a promising avenue for creating generative systems that are not only robust and diverse but also capable of producing highly detailed and perceptually convincing synthetic content. Future research will likely focus on optimizing the integration strategies, exploring novel adversarial objectives tailored for diffusion processes, and extending these hybrid models to complex, multi-modal generative tasks.
\subsection{Frontier Applications in Specialized Scientific Domains}
\label{sec:7\_3\_frontier\_applications\_in\_specialized\_scientific\_domains}

The transformative potential of Generative Adversarial Networks (GANs) extends far beyond conventional image synthesis, increasingly finding critical applications in specialized scientific and engineering domains where data modalities are novel, non-traditional, and often challenging. This frontier involves adapting the core principles of GAN stabilization to enhance data quality, enable advanced analytical tasks, and tackle previously intractable data types, thereby opening new avenues for generative AI.

A significant area of exploration is the adversarial denoising of complex physiological signals. Electroencephalography (EEG) signals, for instance, are notoriously susceptible to noise and artifacts that can obscure vital clinical and scientific information. Traditional denoising methods often fall short against the nonlinear and time-varying nature of these interferences. In this context, \cite{tibermacine2025pye} presents a systematic comparative analysis of standard GAN and Wasserstein GAN with Gradient Penalty (WGAN-GP) architectures for EEG signal enhancement. Their work demonstrates that both adversarial frameworks significantly outperform classical wavelet-based thresholding and linear filtering, showcasing their superior adaptability to nonlinear distortions. Specifically, WGAN-GP, a stabilized GAN variant known for its robust training and mitigation of mode collapse, achieved higher signal-to-noise ratios (up to 14.47dB) and greater training stability, making it effective for aggressive noise suppression. Conversely, the conventional GAN model excelled in preserving finer signal details, achieving a peak signal-to-noise ratio of 19.28dB and high correlation coefficients, highlighting a crucial trade-off between strong artifact reduction and high-fidelity signal reconstruction. This research underscores how established GAN stabilization techniques can be leveraged to tackle the unique challenges of non-traditional data, providing nuanced guidance for model selection based on specific application requirements in neuroscience and brain-computer interfaces.

Extending this versatility to other critical, data-scarce scientific domains, GANs are also being developed to generate high-quality synthetic data for analytical tasks where real-world data collection is expensive, time-consuming, or limited. For instance, in renewable energy applications, the availability of robust datasets for fault detection and energy prediction is often constrained. Addressing this, \cite{elbaz2025wzb} introduces Penca-GAN, a novel dual GAN architecture designed for renewable energy optimization. This model incorporates an "identity block" to enhance training stabilization and promote smoother gradient flow, directly addressing persistent stability concerns encountered in complex data generation. Furthermore, Penca-GAN pioneers a "Pancreas-Inspired Metaheuristic Loss Function" alongside a dual loss function, which dynamically adapts to training data to ensure pixel integrity and promote diversity, thereby mitigating mode collapse in a novel, biologically-inspired manner. By generating high-quality, diverse synthetic data, Penca-GAN significantly improves downstream analytical tasks such as fault detection in solar panels and wind turbines, demonstrating how advanced GAN architectures, coupled with innovative stabilization mechanisms, can provide critical data augmentation in specialized engineering problems. This work highlights a shift towards tailoring GAN stabilization methods not just for general image quality, but for domain-specific data utility in challenging, data-scarce environments.

These examples collectively illustrate the expanding horizons of GAN applications beyond traditional image synthesis. They demonstrate how the foundational principles of GAN stabilization, initially developed for visual data, are being adapted and innovated upon to enhance data quality and enable advanced analytical tasks across diverse scientific and engineering domains. The ability of stabilized GANs to generate high-fidelity, domain-specific synthetic data, whether for denoising complex physiological signals or augmenting datasets in critical infrastructure, underscores their versatility and their role in tackling previously challenging data types. Future research will likely continue to explore novel stabilization techniques tailored to the unique characteristics of emerging data modalities, further solidifying GANs' position as indispensable tools for scientific discovery and technological advancement.


\label{sec:conclusion:_synthesis,_unresolved_tensions,_and_future_directions}

\section{Conclusion: Synthesis, Unresolved Tensions, and Future Directions}
\label{sec:conclusion:\_synthesis,\_unresolved\_tensions,\_\_and\_\_future\_directions}

\subsection{Evolution of GAN Stabilization: A Unified Trajectory}
\label{sec:8\_1\_evolution\_of\_gan\_stabilization:\_a\_unified\_trajectory}

The journey of Generative Adversarial Networks (GANs) has been profoundly shaped by a persistent quest for training stability, a challenge that has driven an intricate interplay between theoretical advancements, ingenious architectural designs, and refined training methodologies. This subsection chronicles the chronological and thematic progression of GAN stabilization research, highlighting how each successive phase meticulously built upon the successes and lessons learned from its predecessors, leading to continuous improvements in GAN robustness, the quality of generated outputs, and their applicability across diverse tasks.

Early GAN research quickly identified significant training instabilities, including mode collapse where the generator produces limited varieties of output, and vanishing or exploding gradients that hinder effective learning. Initial efforts focused on heuristic fixes and alternative objective functions. For instance, \textcite{metz20169ir} proposed Unrolled GANs, stabilizing training by defining the generator objective with respect to an unrolled optimization of the discriminator, thereby mitigating mode collapse. Similarly, \textcite{che2016kho} introduced mode regularization techniques to stabilize training and ensure better mode coverage. \textcite{mao2017ss0} addressed the vanishing gradient problem by proposing Least Squares GANs (LSGANs), which replace the sigmoid cross-entropy loss with a least squares loss, demonstrating improved stability and generated image quality. Other early approaches included online learning perspectives like Chekhov GAN \textcite{grnarova20171tc} and multi-discriminator setups such as SGAN \textcite{chavdarova20179w6}, aiming to increase stability and mode coverage. While these methods offered valuable insights and temporary solutions, they often lacked strong theoretical guarantees or universal applicability, underscoring the need for more principled stabilization mechanisms.

A pivotal shift occurred with the introduction of the Wasserstein GAN (WGAN) by \textcite{Arjovsky2017}, which replaced the problematic Jensen-Shannon divergence with the Earth-Mover (Wasserstein-1) distance. This theoretical breakthrough provided a smoother loss landscape and a more meaningful gradient, effectively mitigating mode collapse and vanishing gradients. However, WGAN's original method of enforcing the Lipschitz constraint on the discriminator via weight clipping proved to be problematic, often leading to pathological behavior. This limitation was swiftly addressed by \textcite{Gulrajani2017} with the proposal of a gradient penalty (WGAN-GP), which robustly enforced the Lipschitz constraint without the drawbacks of weight clipping, making WGAN-GP a widely adopted standard for stable GAN training. Complementing these loss function reforms, \textcite{Miyato2018} introduced Spectral Normalization (SN-GAN), an efficient and computationally lighter alternative for directly constraining the Lipschitz constant of the discriminator's layers. This method proved highly effective for stabilizing training and was later integrated into many high-fidelity GAN architectures. Further theoretical grounding for these regularization techniques was provided by \textcite{Mescheder2018}, who analyzed the convergence properties of various gradient penalty methods, confirming their essential role in stable GAN training.

With the foundational stability issues largely addressed, research pivoted towards architectural innovations and advanced training strategies to push the boundaries of image quality and resolution. \textcite{karras2017raw} introduced Progressive Growing of GANs (PGGAN), a key engineering innovation that significantly improved resolution and training stability by gradually increasing network complexity from low to high resolutions. Building upon the stability offered by techniques like Spectral Normalization, \textcite{Brock2019}'s BigGAN demonstrated the power of scaling, leveraging large batch sizes, self-attention mechanisms, and orthogonal regularization to achieve unprecedented image fidelity and diversity. This marked a significant milestone in high-resolution image synthesis, showcasing how robust stabilization enabled the exploration of massive models. Other regularization techniques also emerged during this phase, such as Binarized Representation Entropy (BRE) regularization by \textcite{cao20184y8} and feature-based regularization by \textcite{bang2018ps8}, further enhancing discriminator guidance and training stability.

The trajectory then evolved towards revolutionizing generator design, focusing on disentangled control and the elimination of visual artifacts. \textcite{Karras2019} introduced StyleGAN, a groundbreaking style-based generator architecture that leveraged adaptive instance normalization to achieve highly disentangled control over various image features, setting new benchmarks for perceptual quality. The StyleGAN series continued this refinement, with \textcite{Karras2020} introducing StyleGAN2, which addressed specific artifacts (e.g., "water droplet" artifacts) and improved image quality through architectural modifications and path length regularization. This iterative process culminated in StyleGAN3 \textcite{Karras2021}, which tackled aliasing issues by proposing alias-free architectures and sampling, pushing the boundaries of perceptual realism even further. These works collectively demonstrated a continuous refinement trajectory in generator design, moving from general scaling to highly specialized, controllable, and artifact-free image generation.

The field continues to evolve, with ongoing research exploring deeper theoretical underpinnings and novel stabilization strategies. \textcite{chu2020zbv} developed a principled theoretical framework for understanding GAN stability, deriving conditions for generator stationarity and clarifying the need for Lipschitz constraints and gradient penalties. \textcite{salmona202283g} investigated the expressivity of "push-forward" generative models, revealing a provable trade-off between their ability to fit multimodal distributions and the stability of their training. Newer methods continue to emerge, such as InfoMax-GAN \textcite{lee20205ue} which uses contrastive learning and mutual information maximization to mitigate mode collapse, and methods like Probability Ratio Clipping and Sample Reweighting \textcite{wu2020p8p} inspired by reinforcement learning. Adaptive weighted discriminator loss functions \textcite{zadorozhnyy20208ft}, multi-discriminator approaches like DuelGAN \textcite{wei2021gla} and TWGAN \textcite{zhang2021ypi}, and constrained discriminator outputs \textcite{chao2021ynq} further exemplify the diverse strategies employed to enhance stability. More recent innovations include learnable auxiliary modules \textcite{gan202494y} and collaborative transfer learning between networks \textcite{megahed2024c23} to improve stability and diversity. Furthermore, the lessons learned from GAN stabilization are being applied in diverse fields, from underwater image enhancement \textcite{guo2020n4t, fu20241mw} to wireless communication channel estimation \textcite{hu2021yk5, ye2024n41}, often by integrating robust WGAN variants or spectral normalization.

In conclusion, the evolution of GAN stabilization represents a unified trajectory of scientific advancement, characterized by a continuous feedback loop between theoretical insights and practical engineering. From addressing fundamental instabilities with robust loss functions and regularization, the field progressed to scaling GANs for high-fidelity synthesis through architectural innovations, and ultimately to achieving fine-grained control and artifact-free generation. While remarkable progress has been made, the inherent challenges of balancing computational cost, data requirements, and the desire for ever-higher fidelity and disentanglement remain active areas of research, pointing towards a future where GANs become even more robust, efficient, and broadly applicable.
\subsection{Persistent Challenges and Theoretical Gaps}
\label{sec:8\_2\_persistent\_challenges\_\_and\_\_theoretical\_gaps}

Despite the remarkable advancements in generative adversarial networks (GANs) that have led to unprecedented levels of image fidelity and diversity, several persistent challenges and theoretical limitations continue to impede their widespread and robust application. The training of GANs remains a complex optimization problem, characterized by inherent instabilities, difficulties in achieving comprehensive mode coverage, and a notable sensitivity to hyperparameter configurations. These issues underscore the ongoing need for deeper theoretical understanding and more robust algorithmic solutions.

One of the most critical and enduring challenges is the persistent difficulty in achieving perfect mode coverage, ensuring that the generator captures the entire diversity of the real data distribution. Early GAN formulations frequently suffered from mode collapse, where the generator would produce only a limited subset of the true data distribution's modes. To combat this, initial efforts focused on modifying the objective function or regularization strategies. \cite{metz20169ir} introduced Unrolled GANs, where the generator's objective considers several unrolled optimization steps of the discriminator, aiming to prevent mode collapse and increase diversity by providing a more stable learning signal. Similarly, \cite{che2016kho} proposed Mode Regularized GANs, which introduced regularizers to stabilize training and encourage a fairer distribution of probability mass across data modes. Building on these, \cite{chavdarova20179w6} presented SGAN, an alternative training process using multiple adversarial pairs to improve mode coverage and stability by preventing the global pair from being trapped in local minima. More recently, \cite{dieng2019rjn} developed Prescribed GANs (PresGANs) which explicitly add noise and optimize an entropy-regularized adversarial loss to encourage capturing all modes and provide tractable log-likelihood. While these methods significantly improved mode coverage, the fundamental limitation of "push-forward" generative models, where a deterministic neural network transforms a standard Gaussian, was highlighted by \cite{salmona202283g}. This work theoretically demonstrates a provable trade-off between the Lipschitz constant (crucial for training stability) and the ability to fit multimodal distributions, suggesting that perfect mode coverage for complex, multimodal data might be inherently difficult for many GAN architectures. Ongoing research, such as \cite{lee20205ue}'s InfoMax-GAN, continues to tackle mode collapse and improve diversity through information maximization and contrastive learning, underscoring that this remains an active and pressing research need.

Another significant hurdle is the ongoing struggle to obtain robust theoretical convergence guarantees, especially for increasingly complex GAN architectures. The minimax game formulation of GANs is notoriously difficult to optimize, often leading to oscillations or non-convergence. The introduction of the Wasserstein distance in \cite{arjovsky2017} and its improvement with a gradient penalty in \cite{gulrajani2017} provided a smoother loss landscape and better theoretical properties, mitigating vanishing gradients and improving training stability. However, even these foundational works often rely on assumptions that may not hold in practice for deep, high-dimensional models. \cite{mescheder2018} directly questioned which training methods for GANs \textit{actually} converge, providing a critical analysis of various gradient penalty methods and their convergence properties, demonstrating that careful application is key. \cite{liang2018r52} offered a non-asymptotic local convergence theory for smooth two-player games, revealing that the interaction term between generator and discriminator can be both beneficial and detrimental to convergence, explaining the slow-down effect in Simultaneous Gradient Ascent and highlighting the role of stabilizing techniques like Optimistic Mirror Descent. Furthering this theoretical understanding, \cite{chu2020zbv} developed a principled framework for GAN stability, deriving conditions for generator stationarity and clarifying the necessity of techniques like Lipschitz constraints and gradient penalties. More recently, \cite{xu2019uwg} proposed a novel perspective from control theory to model GAN dynamics in the function space, leading to practical stabilizing methods, indicating that the theoretical underpinnings of GAN training are still being actively explored and refined to provide more generalizable guarantees.

Furthermore, GANs exhibit a notorious sensitivity to hyperparameter tuning, and inherent trade-offs between computational efficiency and generative quality remain significant hurdles. Achieving state-of-the-art results often necessitates extensive experimentation with learning rates, regularization strengths, network architectures, and optimization strategies. Papers like \cite{brock2019} (BigGAN) and the StyleGAN series (\cite{karras2019, karras2020, karras2021}) demonstrate unprecedented image fidelity but often at the cost of massive computational resources and large-scale datasets, highlighting a clear trade-off. Even fundamental loss function changes, such as those in \cite{mao2017ss0}'s Least Squares GANs (LSGANs), aimed to improve stability and quality, but the optimal application of such methods often still requires careful tuning. Regularization techniques like Spectral Normalization \cite{miyato2018} offer computational efficiency but do not entirely eliminate the need for hyperparameter search. The challenge of balancing these factors is a recurring theme in surveys like \cite{jabbar2020aj0} and \cite{wang2019w53}, which explicitly list training stability and hyperparameter sensitivity as major obstacles. Recent efforts continue to address these practical challenges: \cite{zadorozhnyy20208ft} introduced adaptive weighted discriminator loss functions to dynamically adjust weights, aiming to improve stability without fixed hyperparameter choices. \cite{chao2021ynq} proposed Constrained GANs (GAN-C) with a constraint on the discriminator's output to accelerate convergence and stabilize training. \cite{wu2020p8p} introduced probability ratio clipping and sample reweighting to regularize generator updates and stabilize discriminator training, demonstrating practical approaches to mitigate instability. The ongoing development of methods like \cite{gan202494y}'s learnable auxiliary module and \cite{megahed2024c23}'s Collaborative-GAN, both explicitly designed to enhance training stability and mitigate mode collapse, further attests to the persistent nature of these challenges in contemporary GAN research.

In conclusion, while GANs have transformed generative modeling, they continue to present a complex optimization landscape. The pursuit of perfect mode coverage, robust theoretical convergence guarantees, and resilience to hyperparameter sensitivity remains at the forefront of research. The inherent trade-offs between computational demands and the quest for ever-higher generative quality underscore that GAN training is not a solved problem but rather a vibrant field with active and pressing research needs, pushing the boundaries of both theoretical understanding and practical algorithmic design.
\subsection{Ethical Implications and Responsible AI}
\label{sec:8\_3\_ethical\_implications\_\_and\_\_responsible\_ai}

The very trajectory of Generative Adversarial Network (GAN) stabilization, meticulously traced from the architectural heuristics of Deep Convolutional GANs (DCGANs) to the mathematical rigor of Wasserstein GANs with Gradient Penalty (WGAN-GP) and the design elegance of the StyleGAN series, has transformed generative models from unstable novelties into potent tools for high-fidelity synthesis. This remarkable technical prowess, however, is inextricably linked to a burgeoning landscape of critical ethical considerations, making these issues an immediate and unavoidable consequence of the field's success. As \cite{bhat202445j} highlights, the ethical concerns, including the misuse of GANs for deepfakes and synthetic data, underscore the paramount importance of transparency, accountability, and ethical standards in research and deployment. Similarly, \cite{goyal2024ufg} points to privacy concerns and algorithmic biases as significant challenges inherent in synthetic data generation, demanding rigorous examination and proactive mitigation strategies.

One of the most immediate and widely recognized ethical concerns is the potential for generating convincing 'deepfakes' and their subsequent misuse in misinformation campaigns. The enhanced fidelity and fine-grained controllability achieved by advanced generative models, such as the StyleGAN series (Section 4.4) with its disentangled latent spaces, significantly lower the barrier for creating photorealistic and highly manipulable media \cite{karras2019stylegan}. This capability makes it increasingly challenging for human observers and even automated systems to discern between authentic and synthetically generated content. For instance, \cite{jenkins2024qf5} demonstrates how even foundational GAN architectures like DCGANs can be leveraged to generate synthetic biometric samples capable of deceiving security systems, directly illustrating the malicious potential of realistic synthetic data. The scaling capabilities of models like BigGAN (Section 4.3) further amplify this risk, enabling the mass production of highly convincing deceptive content across various domains. Such capabilities are readily exploitable for creating deceptive audio, video, and images that can spread disinformation, manipulate public opinion, damage reputations, and even destabilize democratic processes, thereby eroding trust in digital media and posing a significant threat to information integrity \cite{bhat202445j}.

Furthermore, the proliferation of synthetic content generated by these sophisticated models introduces complex and largely unresolved intellectual property (IP) issues. As AI systems generate artworks, musical compositions, or textual content that can rival human creations, fundamental questions arise regarding ownership, copyright, and attribution \cite{abrams2023copyright}. The training datasets often comprise vast amounts of existing copyrighted material, leading to debates about whether the generated output constitutes a derivative work, a novel creation, or even an infringement. The legal landscape is currently grappling with these questions, with ongoing lawsuits challenging the use of copyrighted material in training data and the copyrightability of AI-generated content. Concepts such as 'transformative use' are being critically re-evaluated in the context of generative AI, where models learn complex patterns from existing works to create new ones, rather than directly copying them. The lack of clear legal precedents creates a nebulous environment for creators, users, and developers of generative AI, necessitating urgent legal and policy frameworks to address these complex issues.

Another critical ethical concern, highlighted by \cite{goyal2024ufg}, is the amplification of biases embedded in training data, which can manifest in generated outputs. Generative models, by their nature, learn patterns and representations from the datasets they are exposed to. If these datasets reflect existing societal prejudices, stereotypes, or underrepresentation of certain groups, the AI will inevitably reproduce, and often exaggerate, these biases in its generated content \cite{buolamwini2018gender}. For instance, a StyleGAN trained on a dataset predominantly featuring individuals from a specific demographic might struggle to generate realistic images of other groups, or worse, generate outputs that reinforce harmful stereotypes \cite{wang2020towards}. The large-scale training of models like BigGAN (Section 4.3) on vast, potentially biased datasets further exacerbates this risk, enabling the widespread propagation of these learned biases, leading to outputs that perpetuate harmful stereotypes, discriminate against marginalized communities, or lack diversity, thereby reinforcing existing inequalities.

Given these profound and multifaceted challenges, the paramount importance of developing GANs and similar generative models responsibly cannot be overstated. This necessitates a multi-faceted approach that extends far beyond mere technical optimization. Proactive research is crucial for developing robust mechanisms for detecting synthetic media, which can serve as a vital countermeasure against the spread of deepfakes and misinformation. This includes exploring advanced watermarking techniques \cite{gupta2021deepfake}, developing sophisticated forensic analysis tools, and creating AI-based detectors capable of identifying subtle, non-human artifacts indicative of synthetic generation \cite{tolosana2020deepfakes}. Interestingly, GANs themselves can be leveraged in this defense, as demonstrated by models like the GAN-IF based cyber-security model for intrusion detection \cite{u2023m2y} or the Evasion GAN (EVAGAN) which can generate evasion samples for adversarial training to improve detector performance \cite{randhawa2021ksq}. However, this often leads to an "arms race" where generators continually improve to evade detection, highlighting the need for a holistic strategy that combines technical countermeasures with policy and education \cite{mirsky2021creation}.

Moreover, efforts must be directed towards mitigating harmful applications by integrating ethical considerations into the entire AI lifecycle. This involves addressing data biases through careful curation and augmentation (as discussed in Section 5.1), promoting transparency in model development, and establishing clear, enforceable guidelines for the ethical use of generative AI \cite{floridi2019establishing}. The explicit call for "transparency, accountability, and ethical standards" by \cite{bhat202445j} and "stringent ethical guidelines and regulatory frameworks" by \cite{cai2024m9z} highlights the collective responsibility of researchers, developers, policymakers, and society at large to collaborate. This interdisciplinary effort is essential to ensure that these powerful technologies are harnessed in ways that genuinely benefit humanity, foster creativity, and advance scientific discovery, while rigorously minimizing their potential risks and preventing their weaponization.

In conclusion, while the stabilization techniques discussed throughout this review have propelled generative models like StyleGANs and VAE-GANs to unprecedented levels of fidelity and control, they simultaneously usher in a complex ethical landscape. The concerns regarding deepfakes, intellectual property, and bias amplification, as articulated by comprehensive reviews \cite{bhat202445j, goyal2024ufg} and specific examples like biometric deception \cite{jenkins2024qf5}, underscore that technical prowess must be matched by a deep and unwavering commitment to responsible innovation. Future research must not only strive for greater generative fidelity and stability but also prioritize the development of robust ethical safeguards, effective bias mitigation strategies, and reliable detection tools—even leveraging GANs for defense \cite{u2023m2y, randhawa2021ksq}—to ensure that these transformative technologies serve as a force for good, rather than a source of societal discord.
\subsection{Emerging Research Avenues}
\label{sec:8\_4\_emerging\_research\_avenues}

The trajectory of Generative Adversarial Networks (GANs) has been marked by a remarkable evolution, transforming them from nascent, unstable models into sophisticated engines capable of high-fidelity synthesis. Yet, despite these profound advancements, the field is far from static, continually confronting unresolved theoretical tensions and practical limitations that necessitate a forward-looking research agenda. This section delineates promising future directions, focusing on critical open questions and next-generation solutions in hybrid architectures, adaptive training methodologies, expansion into novel data modalities, and the imperative for enhanced interpretability and ethical deployment.

A primary emerging avenue, building upon the foundational explorations in Section 7.2, lies in the sophisticated hybridization of generative paradigms. While early integrations, such as VAE-GAN models, demonstrated the benefit of combining Variational Autoencoder's structured latent space with GAN's sharp generation \cite{cai2024m9z}, the current frontier involves a deeper, more synergistic merging of GANs with Diffusion Models. As highlighted by \cite{peng2024kkw}, Diffusion Models offer superior stability and comprehensive mode coverage, albeit typically with slower inference, whereas GANs excel in speed and sharpness but remain susceptible to mode collapse and training instability. The challenge for future research extends beyond mere concatenation; it demands novel architectures and loss functions that truly reconcile these distinct generative mechanisms. Critical questions persist regarding how to optimally integrate their divergent loss landscapes, manage the inherent computational demands of diffusion within an adversarial framework, and prevent the emergence of novel failure modes unique to these hybrid systems. Furthermore, theoretical work by \cite{salmona202283g} suggests a fundamental trade-off in "push-forward" generative models (like GANs and VAEs) between Lipschitz stability and the ability to fit multimodal distributions, a limitation less pronounced in diffusion models. Future hybrid models must therefore address this theoretical tension, striving for architectures that maintain GAN's efficiency and sharpness while inheriting Diffusion Models' robustness and capacity for diverse distribution capture without compromising stability.

Further advancements are crucial for developing more robust and adaptive training algorithms that move beyond current regularization techniques. While methods like gradient penalties \cite{community\_20} and spectral normalization \cite{68cb9fce1e6af2740377494350b650533c9a29e1} significantly stabilized GAN training, future research aims for algorithms that are inherently less sensitive to hyperparameter tuning and more resilient to diverse data conditions, as well as providing stronger theoretical convergence guarantees \cite{jabbar2020aj0}. For instance, the concept of Constrained Generative Adversarial Networks (GAN-C) \cite{chao2021ynq} introduces explicit constraints on the discriminator's output, demonstrating theoretically faster convergence and yielding higher quality images. This points towards a direction of more theoretically grounded and dynamically adaptive regularization strategies. Building on meta-learning approaches for discriminators \cite{Wang2023} that enable few-shot generation, the next step involves enhancing the generalization capabilities of these meta-learned discriminators to vastly different domains and extreme data scarcity. This also entails reducing the computational overhead associated with meta-training itself, aiming for truly hyperparameter-agnostic and self-correcting training mechanisms that democratize access to high-performance generative models across a broader spectrum of applications.

The expansion of GANs to increasingly complex and diverse data modalities represents another significant frontier. Building upon the strides made in 3D-aware synthesis, such as the integration of StyleGAN's latent spaces with Neural Radiance Fields (NeRFs) discussed in Section 7.1 \cite{Chan2023}, the next challenges involve scaling these capabilities to dynamic 3D scenes, enabling real-time interactive generation, and achieving fine-grained control over material properties, lighting, and physics-based rendering. Beyond static 3D, video generation remains a formidable task, demanding robust long-range temporal consistency, high-resolution synthesis, and efficient handling of vast spatio-temporal data, often with limited paired training examples. This will likely necessitate architectural shifts, potentially integrating transformer-based attention mechanisms directly into the generator's temporal dimension to overcome the inherent biases of convolutional networks in capturing long-range dependencies. Moreover, the application of GANs to complex scientific simulations, such as climate modeling, drug discovery, or material science, requires a shift from purely visual realism to scientific accuracy and utility. Physics-Informed Generative Adversarial Networks (PI-GANs) \cite{yang2018svo, warner2020a5z} offer a promising direction by encoding governing physical laws directly into the network architecture, enabling the generation of physically consistent data and solving inverse problems with limited measurements. Future work in this domain must focus on enhancing the interpretability of these models to ensure scientific validity, scaling to higher-dimensional stochastic processes, and integrating complex multi-physics constraints.

Finally, continuous efforts are paramount for enhancing the interpretability, controllability, and ethical deployment of generative models. As GANs become more sophisticated, understanding their internal representations and gaining fine-grained, causal control over their outputs is critical. Pioneering work in visualizing and understanding GANs, such as the analytic framework presented by \cite{bau20197hm}, dissects networks to identify interpretable units and quantify their causal effect on generated objects. This research direction aims to move beyond simple latent space disentanglement to a deeper, causal understanding of how GANs construct images, which is vital for debugging, identifying and mitigating biases, and ensuring fair and transparent AI systems. Furthermore, the increasing realism and accessibility of synthetic media necessitate robust mechanisms for detecting generated content, developing responsible AI frameworks that are integrated into model design, and establishing stringent ethical guidelines to prevent misuse, particularly in the context of misinformation, intellectual property infringement, and the amplification of societal biases inherent in training data \cite{community\_37}.

In summary, the future of GAN research is characterized by a concerted push towards greater versatility, stability, and seamless integration into broader intelligent AI systems. The exploration of hybrid architectures that critically balance strengths and weaknesses, the development of more robust and adaptive training algorithms, the expansion into diverse and complex data modalities, and the continuous pursuit of improved interpretability, controllability, and ethical responsibility collectively point towards a future where generative AI is not only more powerful but also more reliable, understandable, and beneficial to society.


\newpage
\section*{References}
\addcontentsline{toc}{section}{References}

\begin{thebibliography}{194}

\bibitem{arjovsky2017ze5}
Martín Arjovsky, Soumith Chintala, and L. Bottou (2017). \textit{Wasserstein Generative Adversarial Networks}. International Conference on Machine Learning.

\bibitem{karras2017raw}
Tero Karras, Timo Aila, S. Laine, et al. (2017). \textit{Progressive Growing of GANs for Improved Quality, Stability, and Variation}. International Conference on Learning Representations.

\bibitem{miyato2018arc}
Takeru Miyato, Toshiki Kataoka, Masanori Koyama, et al. (2018). \textit{Spectral Normalization for Generative Adversarial Networks}. International Conference on Learning Representations.

\bibitem{karras202039x}
Tero Karras, M. Aittala, Janne Hellsten, et al. (2020). \textit{Training Generative Adversarial Networks with Limited Data}. Neural Information Processing Systems.

\bibitem{zhang2016mm0}
Han Zhang, Tao Xu, Hongsheng Li, et al. (2016). \textit{StackGAN: Text to Photo-Realistic Image Synthesis with Stacked Generative Adversarial Networks}. IEEE International Conference on Computer Vision.

\bibitem{shrivastava2016uym}
A. Shrivastava, Tomas Pfister, Oncel Tuzel, et al. (2016). \textit{Learning from Simulated and Unsupervised Images through Adversarial Training}. Computer Vision and Pattern Recognition.

\bibitem{zhao2020xhy}
Shengyu Zhao, Zhijian Liu, Ji Lin, et al. (2020). \textit{Differentiable Augmentation for Data-Efficient GAN Training}. Neural Information Processing Systems.

\bibitem{metz20169ir}
Luke Metz, Ben Poole, David Pfau, et al. (2016). \textit{Unrolled Generative Adversarial Networks}. International Conference on Learning Representations.

\bibitem{guo2020n4t}
Ye-cai Guo, Hanyu Li, and Peixian Zhuang (2020). \textit{Underwater Image Enhancement Using a Multiscale Dense Generative Adversarial Network}. IEEE Journal of Oceanic Engineering.

\bibitem{bau2018n2x}
David Bau, Jun-Yan Zhu, Hendrik Strobelt, et al. (2018). \textit{GAN Dissection: Visualizing and Understanding Generative Adversarial Networks}. International Conference on Learning Representations.

\bibitem{che2016kho}
Tong Che, Yanran Li, Athul Paul Jacob, et al. (2016). \textit{Mode Regularized Generative Adversarial Networks}. International Conference on Learning Representations.

\bibitem{liu20212c2}
Bingchen Liu, Yizhe Zhu, Kunpeng Song, et al. (2021). \textit{Towards Faster and Stabilized GAN Training for High-fidelity Few-shot Image Synthesis}. International Conference on Learning Representations.

\bibitem{jabbar2020aj0}
Abdul Jabbar, Xi Li, and Bourahla Omar (2020). \textit{A Survey on Generative Adversarial Networks: Variants, Applications, and Training}. ACM Computing Surveys.

\bibitem{roth2017eui}
Kevin Roth, Aurélien Lucchi, Sebastian Nowozin, et al. (2017). \textit{Stabilizing Training of Generative Adversarial Networks through Regularization}. Neural Information Processing Systems.

\bibitem{yang2018svo}
Liu Yang, Dongkun Zhang, and G. Karniadakis (2018). \textit{Physics-Informed Generative Adversarial Networks for Stochastic Differential Equations}. SIAM Journal on Scientific Computing.

\bibitem{zhang2019hjo}
Han Zhang, Zizhao Zhang, Augustus Odena, et al. (2019). \textit{Consistency Regularization for Generative Adversarial Networks}. International Conference on Learning Representations.

\bibitem{tseng2021m2s}
Hung-Yu Tseng, Lu Jiang, Ce Liu, et al. (2021). \textit{Regularizing Generative Adversarial Networks under Limited Data}. Computer Vision and Pattern Recognition.

\bibitem{mao20196tx}
Wentao Mao, Yamin Liu, Ling Ding, et al. (2019). \textit{Imbalanced Fault Diagnosis of Rolling Bearing Based on Generative Adversarial Network: A Comparative Study}. IEEE Access.

\bibitem{hartmann2018h3s}
K. Hartmann, R. Schirrmeister, and T. Ball (2018). \textit{EEG-GAN: Generative adversarial networks for electroencephalograhic (EEG) brain signals}. arXiv.org.

\bibitem{wang2019w53}
Zhengwei Wang, Qi She, and T. Ward (2019). \textit{Generative Adversarial Networks in Computer Vision}. ACM Computing Surveys.

\bibitem{luo2020aaj}
Jia Luo, Jinying Huang, and Hongmei Li (2020). \textit{A case study of conditional deep convolutional generative adversarial networks in machine fault diagnosis}. Journal of Intelligent Manufacturing.

\bibitem{liu2020jt0}
Ming-Yu Liu, Xun Huang, Jiahui Yu, et al. (2020). \textit{Generative Adversarial Networks for Image and Video Synthesis: Algorithms and Applications}. Proceedings of the IEEE.

\bibitem{liang2018r52}
Tengyuan Liang, and J. Stokes (2018). \textit{Interaction Matters: A Note on Non-asymptotic Local Convergence of Generative Adversarial Networks}. International Conference on Artificial Intelligence and Statistics.

\bibitem{ghafoorian2018fwh}
Mohsen Ghafoorian, C. Nugteren, N. Baka, et al. (2018). \textit{EL-GAN: Embedding Loss Driven Generative Adversarial Networks for Lane Detection}. ECCV Workshops.

\bibitem{guo2019414}
Xiaopeng Guo, Rencan Nie, Jinde Cao, et al. (2019). \textit{FuseGAN: Learning to Fuse Multi-Focus Image via Conditional Generative Adversarial Network}. IEEE transactions on multimedia.

\bibitem{liu2020kd1}
B. Liu, Cheng Tan, Shuqin Li, et al. (2020). \textit{A Data Augmentation Method Based on Generative Adversarial Networks for Grape Leaf Disease Identification}. IEEE Access.

\bibitem{hjelm2017iqg}
R. Devon Hjelm, Athul Paul Jacob, Tong Che, et al. (2017). \textit{Boundary-Seeking Generative Adversarial Networks}. International Conference on Learning Representations.

\bibitem{shahriar2020sm7}
Md Hasan Shahriar, Nur Imtiazul Haque, M. Rahman, et al. (2020). \textit{G-IDS: Generative Adversarial Networks Assisted Intrusion Detection System}. Annual International Computer Software and Applications Conference.

\bibitem{pfau2016v7o}
David Pfau, and O. Vinyals (2016). \textit{Connecting Generative Adversarial Networks and Actor-Critic Methods}. arXiv.org.

\bibitem{mao2017ss0}
Xudong Mao, Qing Li, Haoran Xie, et al. (2017). \textit{On the Effectiveness of Least Squares Generative Adversarial Networks}. IEEE Transactions on Pattern Analysis and Machine Intelligence.

\bibitem{fekri2019c1i}
Mohammad Navid Fekri, A. M. Ghosh, and Katarina Grolinger (2019). \textit{Generating Energy Data for Machine Learning with Recurrent Generative Adversarial Networks}. Energies.

\bibitem{chen2019ng2}
Xinyuan Chen, Chang Xu, Xiaokang Yang, et al. (2019). \textit{Gated-GAN: Adversarial Gated Networks for Multi-Collection Style Transfer}. IEEE Transactions on Image Processing.

\bibitem{baby2019h4h}
Deepak Baby, and S. Verhulst (2019). \textit{Sergan: Speech Enhancement Using Relativistic Generative Adversarial Networks with Gradient Penalty}. IEEE International Conference on Acoustics, Speech, and Signal Processing.

\bibitem{wiatrak20194ib}
Maciej Wiatrak, Stefano V. Albrecht, and A. Nystrom (2019). \textit{Stabilizing Generative Adversarial Networks: A Survey}. Unpublished manuscript.

\bibitem{salmona202283g}
Antoine Salmona, Valentin De Bortoli, J. Delon, et al. (2022). \textit{Can Push-forward Generative Models Fit Multimodal Distributions?}. Neural Information Processing Systems.

\bibitem{lee20205ue}
Kwot Sin Lee, Ngoc-Trung Tran, and Ngai-Man Cheung (2020). \textit{InfoMax-GAN: Improved Adversarial Image Generation via Information Maximization and Contrastive Learning}. IEEE Workshop/Winter Conference on Applications of Computer Vision.

\bibitem{herr20208x4}
Daniel Herr, B. Obert, and Matthias Rosenkranz (2020). \textit{Anomaly detection with variational quantum generative adversarial networks}. Quantum Science and Technology.

\bibitem{hayes201742g}
Jamie Hayes, Luca Melis, G. Danezis, et al. (2017). \textit{LOGAN: Evaluating Privacy Leakage of Generative Models Using Generative Adversarial Networks}. arXiv.org.

\bibitem{negi20208n9}
Anuja Negi, A. Noel, Joseph Raj, et al. (2020). \textit{RDA-UNET-WGAN: An Accurate Breast Ultrasound Lesion Segmentation Using Wasserstein Generative Adversarial Networks}. The Arabian journal for science and engineering.

\bibitem{meng2022you}
Zong Meng, Qian Li, De-gang Sun, et al. (2022). \textit{An Intelligent Fault Diagnosis Method of Small Sample Bearing Based on Improved Auxiliary Classification Generative Adversarial Network}. IEEE Sensors Journal.

\bibitem{liu2019sb7}
Yi Liu, Jialiang Peng, James J. Q. Yu, et al. (2019). \textit{PPGAN: Privacy-Preserving Generative Adversarial Network}. International Conference on Parallel and Distributed Systems.

\bibitem{yuan2020bt6}
Zhenmou Yuan, M. Jiang, Yaming Wang, et al. (2020). \textit{SARA-GAN: Self-Attention and Relative Average Discriminator Based Generative Adversarial Networks for Fast Compressed Sensing MRI Reconstruction}. Frontiers in Neuroinformatics.

\bibitem{agarwal2022p6d}
Aishwarya Agarwal, Biplab Banerjee, Fabio Cuzzolin, et al. (2022). \textit{Semantics-Driven Generative Replay for Few-Shot Class Incremental Learning}. ACM Multimedia.

\bibitem{grnarova20171tc}
Paulina Grnarova, K. Levy, Aurélien Lucchi, et al. (2017). \textit{An Online Learning Approach to Generative Adversarial Networks}. International Conference on Learning Representations.

\bibitem{liu2019oc8}
Zhiyue Liu, Jiahai Wang, and Zhiwei Liang (2019). \textit{CatGAN: Category-aware Generative Adversarial Networks with Hierarchical Evolutionary Learning for Category Text Generation}. AAAI Conference on Artificial Intelligence.

\bibitem{chung2022s9a}
Jihoon Chung, Bo Shen, and Zhen Kong (2022). \textit{Anomaly detection in additive manufacturing processes using supervised classification with imbalanced sensor data based on generative adversarial network}. Journal of Intelligent Manufacturing.

\bibitem{chu2020zbv}
Casey Chu, Kentaro Minami, and K. Fukumizu (2020). \textit{Smoothness and Stability in GANs}. International Conference on Learning Representations.

\bibitem{jenni2019339}
S. Jenni, and P. Favaro (2019). \textit{On Stabilizing Generative Adversarial Training With Noise}. Computer Vision and Pattern Recognition.

\bibitem{xiang20171at}
Sitao Xiang, and Hao Li (2017). \textit{On the Effects of Batch and Weight Normalization in Generative Adversarial Networks}. Unpublished manuscript.

\bibitem{neyshabur201713g}
Behnam Neyshabur, Srinadh Bhojanapalli, and Ayan Chakrabarti (2017). \textit{Stabilizing GAN Training with Multiple Random Projections}. arXiv.org.

\bibitem{bau20197hm}
David Bau, Jun-Yan Zhu, Hendrik Strobelt, et al. (2019). \textit{Visualizing and Understanding Generative Adversarial Networks (Extended Abstract)}. arXiv.org.

\bibitem{dieng2019rjn}
A. B. Dieng, Francisco J. R. Ruiz, D. Blei, et al. (2019). \textit{Prescribed Generative Adversarial Networks}. arXiv.org.

\bibitem{zhang2020376}
Hongliang Zhang, Rui Wang, Ruilin Pan, et al. (2020). \textit{Imbalanced Fault Diagnosis of Rolling Bearing Using Enhanced Generative Adversarial Networks}. IEEE Access.

\bibitem{yuan202257j}
Chao Yuan, Hongxia Wang, Peisong He, et al. (2022). \textit{GAN-based image steganography for enhancing security via adversarial attack and pixel-wise deep fusion}. Multimedia tools and applications.

\bibitem{iwai2020fp2}
Shoma Iwai, Tomo Miyazaki, Yoshihiro Sugaya, et al. (2020). \textit{Fidelity-Controllable Extreme Image Compression with Generative Adversarial Networks}. International Conference on Pattern Recognition.

\bibitem{kaneko2018jex}
Takuhiro Kaneko, Y. Ushiku, and T. Harada (2018). \textit{Label-Noise Robust Generative Adversarial Networks}. Computer Vision and Pattern Recognition.

\bibitem{khan20223o7}
Maleika Heenaye-Mamode Khan, N. Gooda Sahib-Kaudeer, Motean Dayalen, et al. (2022). \textit{Multi-Class Skin Problem Classification Using Deep Generative Adversarial Network (DGAN)}. Computational Intelligence and Neuroscience.

\bibitem{lin20224oj}
Qiuzhen Lin, Z. Fang, Yi Chen, et al. (2022). \textit{Evolutionary Architectural Search for Generative Adversarial Networks}. IEEE Transactions on Emerging Topics in Computational Intelligence.

\bibitem{tuan2018kbr}
Yi-Lin Tuan, and Hung-yi Lee (2018). \textit{Improving Conditional Sequence Generative Adversarial Networks by Stepwise Evaluation}. IEEE/ACM Transactions on Audio Speech and Language Processing.

\bibitem{wei2021qea}
Kaimin Wei, Tianqi Li, Feiran Huang, et al. (2021). \textit{Cancer classification with data augmentation based on generative adversarial networks}. Frontiers of Computer Science.

\bibitem{wang20178xf}
Ruohan Wang, Antoine Cully, H. Chang, et al. (2017). \textit{MAGAN: Margin Adaptation for Generative Adversarial Networks}. arXiv.org.

\bibitem{sage2017ywd}
Alexander Sage, E. Agustsson, R. Timofte, et al. (2017). \textit{Logo Synthesis and Manipulation with Clustered Generative Adversarial Networks}. 2018 IEEE/CVF Conference on Computer Vision and Pattern Recognition.

\bibitem{wu2020p8p}
Yue Wu, Pan Zhou, A. Wilson, et al. (2020). \textit{Improving GAN Training with Probability Ratio Clipping and Sample Reweighting}. Neural Information Processing Systems.

\bibitem{chavdarova20179w6}
Tatjana Chavdarova, and F. Fleuret (2017). \textit{SGAN: An Alternative Training of Generative Adversarial Networks}. 2018 IEEE/CVF Conference on Computer Vision and Pattern Recognition.

\bibitem{li2020muy}
Ziqiang Li, Pengfei Xia, Rentuo Tao, et al. (2020). \textit{A New Perspective on Stabilizing GANs Training: Direct Adversarial Training}. IEEE Transactions on Emerging Topics in Computational Intelligence.

\bibitem{goudarzi2020ymw}
Sobhan Goudarzi, A. Asif, and H. Rivaz (2020). \textit{Fast Multi-Focus Ultrasound Image Recovery Using Generative Adversarial Networks}. IEEE Transactions on Computational Imaging.

\bibitem{tao20219q2}
Yuechuan Tao, J. Qiu, and Shuying Lai (2021). \textit{A Data-Driven Management Strategy of Electric Vehicles and Thermostatically Controlled Loads Based on Modified Generative Adversarial Network}. IEEE Transactions on Transportation Electrification.

\bibitem{zhong2019opk}
Yue Zhong, Lizhuang Liu, Dan Zhao, et al. (2019). \textit{A generative adversarial network for image denoising}. Multimedia tools and applications.

\bibitem{yan2020889}
Peiyao Yan, Feng He, Yajie Yang, et al. (2020). \textit{Semi-Supervised Representation Learning for Remote Sensing Image Classification Based on Generative Adversarial Networks}. IEEE Access.

\bibitem{lee20203j4}
Shindong Lee, Bonggu Ko, Keonnyeong Lee, et al. (2020). \textit{Many-To-Many Voice Conversion Using Conditional Cycle-Consistent Adversarial Networks}. IEEE International Conference on Acoustics, Speech, and Signal Processing.

\bibitem{hu2021yk5}
Tianyu Hu, Yang Huang, Qiuming Zhu, et al. (2021). \textit{Channel Estimation Enhancement With Generative Adversarial Networks}. IEEE Transactions on Cognitive Communications and Networking.

\bibitem{chen2021n5h}
Tianlong Chen, Yu Cheng, Zhe Gan, et al. (2021). \textit{Ultra-Data-Efficient GAN Training: Drawing A Lottery Ticket First, Then Training It Toughly}. arXiv.org.

\bibitem{cai2019g1w}
Yali Cai, Xiaoru Wang, Zhihong Yu, et al. (2019). \textit{Dualattn-GAN: Text to Image Synthesis With Dual Attentional Generative Adversarial Network}. IEEE Access.

\bibitem{zhou20199sm}
Niyun Zhou, De Cai, Xiao Han, et al. (2019). \textit{Enhanced Cycle-Consistent Generative Adversarial Network for Color Normalization of H&E Stained Images}. International Conference on Medical Image Computing and Computer-Assisted Intervention.

\bibitem{tang2018iie}
Hao Tang, Dan Xu, Wei Wang, et al. (2018). \textit{Dual Generator Generative Adversarial Networks for Multi-Domain Image-to-Image Translation}. Asian Conference on Computer Vision.

\bibitem{tong2022lu4}
Q. Tong, Feiyu Lu, Ziwei Feng, et al. (2022). \textit{A Novel Method for Fault Diagnosis of Bearings with Small and Imbalanced Data Based on Generative Adversarial Networks}. Applied Sciences.

\bibitem{costa2019pj9}
Victor Costa, Nuno Lourenço, and P. Machado (2019). \textit{Coevolution of Generative Adversarial Networks}. EvoApplications.

\bibitem{tang2021c82}
Hongtao Tang, Shengbo Gao, Lei Wang, et al. (2021). \textit{A Novel Intelligent Fault Diagnosis Method for Rolling Bearings Based on Wasserstein Generative Adversarial Network and Convolutional Neural Network under Unbalanced Dataset}. Italian National Conference on Sensors.

\bibitem{yin2022izd}
Haitao Yin, and Jing Xiao (2022). \textit{Laplacian Pyramid Generative Adversarial Network for Infrared and Visible Image Fusion}. IEEE Signal Processing Letters.

\bibitem{xu2020pkq}
Kun Xu, Chongxuan Li, Huanshu Wei, et al. (2020). \textit{Understanding and Stabilizing GANs' Training Dynamics Using Control Theory}. International Conference on Machine Learning.

\bibitem{rahman2021wm8}
Taseef Rahman, Yuanqi Du, Liang Zhao, et al. (2021). \textit{Generative Adversarial Learning of Protein Tertiary Structures}. Molecules.

\bibitem{zhang2022ysl}
Zheng Zhang, Jingsong Yang, and Yang Du (2022). \textit{Deep Convolutional Generative Adversarial Network With Autoencoder for Semisupervised SAR Image Classification}. IEEE Geoscience and Remote Sensing Letters.

\bibitem{varshney2021954}
Sakshi Varshney, V. Verma, K. SrijithP., et al. (2021). \textit{CAM-GAN: Continual Adaptation Modules for Generative Adversarial Networks}. Neural Information Processing Systems.

\bibitem{creswell2016mol}
Antonia Creswell, and A. Bharath (2016). \textit{Adversarial Training for Sketch Retrieval}. ECCV Workshops.

\bibitem{bang2018ps8}
Duhyeon Bang, and Hyunjung Shim (2018). \textit{Improved Training of Generative Adversarial Networks Using Representative Features}. International Conference on Machine Learning.

\bibitem{wang202066v}
Dong Wang, Xiaoqian Qin, F. Song, et al. (2020). \textit{Stabilizing Training of Generative Adversarial Nets via Langevin Stein Variational Gradient Descent}. IEEE Transactions on Neural Networks and Learning Systems.

\bibitem{cai2020n2k}
Likun Cai, Yanjie Chen, Ning Cai, et al. (2020). \textit{Utilizing Amari-Alpha Divergence to Stabilize the Training of Generative Adversarial Networks}. Entropy.

\bibitem{wenzel20225g3}
Markus T. Wenzel (2022). \textit{Generative Adversarial Networks and Other Generative Models}. arXiv.org.

\bibitem{gidel2018pg0}
G. Gidel, Hugo Berard, Pascal Vincent, et al. (2018). \textit{A Variational Inequality Perspective on Generative Adversarial Nets}. arXiv.org.

\bibitem{grinblat2017cem}
G. Grinblat, Lucas C. Uzal, and P. Granitto (2017). \textit{Class-Splitting Generative Adversarial Networks}. arXiv.org.

\bibitem{zhang202263o}
Yingxue Zhang, Yanhua Li, Xun Zhou, et al. (2022). \textit{STrans-GAN: Spatially-Transferable Generative Adversarial Networks for Urban Traffic Estimation}. Industrial Conference on Data Mining.

\bibitem{shin2020169}
Hoo-Chang Shin, Alvin Ihsani, Ziyue Xu, et al. (2020). \textit{GANDALF: Generative Adversarial Networks with Discriminator-Adaptive Loss Fine-tuning for Alzheimer's Disease Diagnosis from MRI}. International Conference on Medical Image Computing and Computer-Assisted Intervention.

\bibitem{ham2020svv}
Hyung-Gi Ham, T. Jun, and Daeyoung Kim (2020). \textit{Unbalanced GANs: Pre-training the Generator of Generative Adversarial Network using Variational Autoencoder}. arXiv.org.

\bibitem{wang20182xz}
Chu Wang, Yanming Zhang, and Cheng-Lin Liu (2018). \textit{Anomaly Detection via Minimum Likelihood Generative Adversarial Networks}. International Conference on Pattern Recognition.

\bibitem{zhang2018oba}
Zhirui Zhang, Shujie Liu, Mu Li, et al. (2018). \textit{Bidirectional Generative Adversarial Networks for Neural Machine Translation}. Conference on Computational Natural Language Learning.

\bibitem{liang2018axu}
G. Liang, S. Fouladvand, Jie Zhang, et al. (2018). \textit{GANai: Standardizing CT Images using Generative Adversarial Network with Alternative Improvement}. bioRxiv.

\bibitem{wiatrak20194ae}
Maciej Wiatrak, and Stefano V. Albrecht (2019). \textit{Stabilizing Generative Adversarial Network Training: A Survey}. arXiv.org.

\bibitem{xue2022n0r}
Yu Xue, Weinan Tong, Ferrante Neri, et al. (2022). \textit{PEGANs: Phased Evolutionary Generative Adversarial Networks with Self-Attention Module}. Mathematics.

\bibitem{oeldorf2019kj7}
Cedric Oeldorf, and Gerasimos Spanakis (2019). \textit{LoGANv2: Conditional Style-Based Logo Generation with Generative Adversarial Networks}. International Conference on Machine Learning and Applications.

\bibitem{sajjadi2018w83}
Mehdi S. M. Sajjadi, and B. Scholkopf (2018). \textit{Tempered Adversarial Networks}. International Conference on Machine Learning.

\bibitem{park2021v6f}
J. E. Park, Da-in Eun, H. Kim, et al. (2021). \textit{Generative adversarial network for glioblastoma ensures morphologic variations and improves diagnostic model for isocitrate dehydrogenase mutant type}. Scientific Reports.

\bibitem{song2020mj8}
Xiaoning Song, Yao Chen, Zhenhua Feng, et al. (2020). \textit{SP-GAN: Self-Growing and Pruning Generative Adversarial Networks}. IEEE Transactions on Neural Networks and Learning Systems.

\bibitem{randhawa2021ksq}
Rizwan Hamid Randhawa, N. Aslam, Mohammad Alauthman, et al. (2021). \textit{Evasion Generative Adversarial Network for Low Data Regimes}. IEEE Transactions on Artificial Intelligence.

\bibitem{wang2020vbt}
Mengxue Wang, Zhenxue Chen, Q. M. J. Wu, et al. (2020). \textit{Improved face super-resolution generative adversarial networks}. Machine Vision and Applications.

\bibitem{saqur2018oqp}
Raeid Saqur, and Sal Vivona (2018). \textit{CapsGAN: Using Dynamic Routing for Generative Adversarial Networks}. Advances in Intelligent Systems and Computing.

\bibitem{gao2018d4g}
F. Gao, Fei Ma, Jun Wang, et al. (2018). \textit{Semi-Supervised Generative Adversarial Nets with Multiple Generators for SAR Image Recognition}. Italian National Conference on Sensors.

\bibitem{you2018a3m}
Haoran You, Yu Cheng, Tianheng Cheng, et al. (2018). \textit{Bayesian Cycle-Consistent Generative Adversarial Networks via Marginalizing Latent Sampling}. IEEE Transactions on Neural Networks and Learning Systems.

\bibitem{du2021bhg}
Biao Du, Lin Tang, Lin Liu, et al. (2021). \textit{Predicting LncRNA-Disease Association Based on Generative Adversarial Network.}. Current Gene Therapy.

\bibitem{wei2021gla}
Jiaheng Wei, Minghao Liu, Jiahao Luo, et al. (2021). \textit{DuelGAN: A Duel Between Two Discriminators Stabilizes the GAN Training}. European Conference on Computer Vision.

\bibitem{lazarou2020gu8}
Conor Lazarou (2020). \textit{Autoencoding Generative Adversarial Networks}. arXiv.org.

\bibitem{zhang2021ypi}
Zhaoyu Zhang, Mengyan Li, Haonian Xie, et al. (2021). \textit{TWGAN: Twin Discriminator Generative Adversarial Networks}. IEEE transactions on multimedia.

\bibitem{jiang2020e6i}
Yi Jiang, Jiajie Xu, Baoqing Yang, et al. (2020). \textit{Image Inpainting Based on Generative Adversarial Networks}. IEEE Access.

\bibitem{plakias2018h0x}
Spyridon Plakias, and Y. Boutalis (2018). \textit{Generative Adversarial Networks for Unsupervised Fault Detection}. European Control Conference.

\bibitem{chao2021ynq}
Xiaopeng Chao, Jiangzhong Cao, Yuqin Lu, et al. (2021). \textit{Constrained Generative Adversarial Networks}. IEEE Access.

\bibitem{zhang20182tk}
Jiacen Zhang, Nakamasa Inoue, and K. Shinoda (2018). \textit{I-vector Transformation Using Conditional Generative Adversarial Networks for Short Utterance Speaker Verification}. Interspeech.

\bibitem{cao20184y8}
Yanshuai Cao, G. Ding, Kry Yik-Chau Lui, et al. (2018). \textit{Improving GAN Training via Binarized Representation Entropy (BRE) Regularization}. International Conference on Learning Representations.

\bibitem{costa2020anu}
Victor Costa, Nuno Lourenço, João Correia, et al. (2020). \textit{Neuroevolution of Generative Adversarial Networks}. Deep Neural Evolution.

\bibitem{panwar2019psx}
Sharaj Panwar, P. Rad, J. Quarles, et al. (2019). \textit{A Semi-Supervised Wasserstein Generative Adversarial Network for Classifying Driving Fatigue from EEG signals}. IEEE International Conference on Systems, Man and Cybernetics.

\bibitem{wu20212vn}
Aming Wu, Juyong Shin, Jae-Kwang Ahn, et al. (2021). \textit{Augmenting Seismic Data Using Generative Adversarial Network for Low-Cost MEMS Sensors}. IEEE Access.

\bibitem{shou2020v6h}
Chunhui Shou, Ling Hong, Waner Ding, et al. (2020). \textit{Defect Detection with Generative Adversarial Networks for Electroluminescence Images of Solar Cells}. Youth Academic Annual Conference of Chinese Association of Automation.

\bibitem{liu2019v0x}
Jianfei Liu, Christine Shen, Tao Liu, et al. (2019). \textit{Active Appearance Model Induced Generative Adversarial Network for Controlled Data Augmentation}. International Conference on Medical Image Computing and Computer-Assisted Intervention.

\bibitem{farrell2019kjy}
S. Farrell, W. Bhimji, T. Kurth, et al. (2019). \textit{Next Generation Generative Neural Networks for HEP}. EPJ Web of Conferences.

\bibitem{wu2020n95}
Zhongze Wu, Chunmei He, Liwen Yang, et al. (2020). \textit{Attentive evolutionary generative adversarial network}. Applied intelligence (Boston).

\bibitem{majtner20192pi}
Tomás Majtner, Buda Bajić, Joakim Lindblad, et al. (2019). \textit{On the Effectiveness of Generative Adversarial Networks as HEp-2 Image Augmentation Tool}. Scandinavian Conference on Image Analysis.

\bibitem{zadorozhnyy20208ft}
Vasily Zadorozhnyy, Q. Cheng, and Q. Ye (2020). \textit{Adaptive Weighted Discriminator for Training Generative Adversarial Networks}. Computer Vision and Pattern Recognition.

\bibitem{munia20201u2}
M. Munia, M. Nourani, and Sammy Houari (2020). \textit{Biosignal Oversampling Using Wasserstein Generative Adversarial Network}. IEEE International Conference on Healthcare Informatics.

\bibitem{warner2020a5z}
J. Warner, Julian Cuevas, G. Bomarito, et al. (2020). \textit{Inverse Estimation of Elastic Modulus Using Physics-Informed Generative Adversarial Networks}. arXiv.org.

\bibitem{lee2017zsj}
Sang-gil Lee, Uiwon Hwang, Seonwoo Min, et al. (2017). \textit{Polyphonic Music Generation with Sequence Generative Adversarial Networks}. Journal of KIISE.

\bibitem{xu2019uwg}
Kun Xu, Chongxuan Li, Huanshu Wei, et al. (2019). \textit{Understanding and Stabilizing GANs' Training Dynamics with Control Theory}. arXiv.org.

\bibitem{zhang201996t}
Shufei Zhang, Zhuang Qian, Kaizhu Huang, et al. (2019). \textit{Robust generative adversarial network}. Machine-mediated learning.

\bibitem{pieters2018jh1}
Mathijs Pieters, and M. Wiering (2018). \textit{Comparing Generative Adversarial Network Techniques for Image Creation and Modification}. arXiv.org.

\bibitem{xiang2017cc9}
Sitao Xiang, and Hao Li (2017). \textit{On the effect of Batch Normalization and Weight Normalization in Generative Adversarial Networks}. arXiv.org.

\bibitem{xiong20243bt}
Hongqiang Xiong, Jing Li, Zhilian Li, et al. (2024). \textit{GPR-GAN: A Ground-Penetrating Radar Data Generative Adversarial Network}. IEEE Transactions on Geoscience and Remote Sensing.

\bibitem{xue2024e7i}
Yu Xue, Weinan Tong, Ferrante Neri, et al. (2024). \textit{Evolutionary Architecture Search for Generative Adversarial Networks Based on Weight Sharing}. IEEE Transactions on Evolutionary Computation.

\bibitem{xue20248md}
Yu Xue, Kun Chen, and Ferrante Neri (2024). \textit{Differentiable Architecture Search With Attention Mechanisms for Generative Adversarial Networks}. IEEE Transactions on Emerging Topics in Computational Intelligence.

\bibitem{jenkins2024qf5}
John Jenkins, and Kaushik Roy (2024). \textit{Exploring deep convolutional generative adversarial networks (DCGAN) in biometric systems: a survey study}. Discover Artificial Intelligence.

\bibitem{qiu2025hu0}
Shiqing Qiu, Yang Wang, Zong Ke, et al. (2025). \textit{A Generative Adversarial Network-Based Investor Sentiment Indicator: Superior Predictability for the Stock Market}. Mathematics.

\bibitem{boubrahimi2024kts}
Soukaina Filali Boubrahimi, Ashit Neema, Ayman Nassar, et al. (2024). \textit{Spatiotemporal Data Augmentation of MODIS‐Landsat Water Bodies Using Adversarial Networks}. Water Resources Research.

\bibitem{liu20232tr}
Naihao Liu, Youbo Lei, Yang Yang, et al. (2023). \textit{Self-supervised Time-Frequency Representation based on Generative Adversarial Networks}. Geophysics.

\bibitem{song20239hi}
Yihong Song, Haoyan Zhang, Jiaqi Li, et al. (2023). \textit{High-Accuracy Maize Disease Detection Based on Attention Generative Adversarial Network and Few-Shot Learning}. Plants.

\bibitem{pal2023147}
Debabrata Pal, Shirsha Bose, Biplab Banerjee, et al. (2023). \textit{MORGAN: Meta-Learning-based Few-Shot Open-Set Recognition via Generative Adversarial Network}. IEEE Workshop/Winter Conference on Applications of Computer Vision.

\bibitem{gan202494y}
Yan Gan, Chenxue Yang, Mao Ye, et al. (2024). \textit{Generative Adversarial Networks with Learnable Auxiliary Module for Image Synthesis}. ACM Trans. Multim. Comput. Commun. Appl..

\bibitem{eltehewy2023cj4}
Rokaya Eltehewy, A. Abouelfarag, and Sherine Nagy Saleh (2023). \textit{Efficient Classification of Imbalanced Natural Disasters Data Using Generative Adversarial Networks for Data Augmentation}. ISPRS Int. J. Geo Inf..

\bibitem{chen2023rrf}
Shiming Chen, Shuhuang Chen, W. Hou, et al. (2023). \textit{EGANS: Evolutionary Generative Adversarial Network Search for Zero-Shot Learning}. IEEE Transactions on Evolutionary Computation.

\bibitem{fu20241mw}
Feiran Fu, Peng Liu, Zhen Shao, et al. (2024). \textit{MEvo-GAN: A Multi-Scale Evolutionary Generative Adversarial Network for Underwater Image Enhancement}. Journal of Marine Science and Engineering.

\bibitem{soleymanzadeh202358z}
Raha Soleymanzadeh, and R. Kashef (2023). \textit{Efficient intrusion detection using multi-player generative adversarial networks (GANs): an ensemble-based deep learning architecture}. Neural computing & applications (Print).

\bibitem{fathallah20236k5}
Mohamed Fathallah, Mohamed Sakr, and Sherif Eletriby (2023). \textit{Stabilizing and Improving Training of Generative Adversarial Networks Through Identity Blocks and Modified Loss Function}. IEEE Access.

\bibitem{luo2024o1x}
Tianjiao Luo, Tim Pearce, Huayu Chen, et al. (2024). \textit{C-GAIL: Stabilizing Generative Adversarial Imitation Learning with Control Theory}. Neural Information Processing Systems.

\bibitem{li2024uae}
Wei Li, and Yongchuan Tang (2024). \textit{Soft Generative Adversarial Network: Combating Mode Collapse in Generative Adversarial Network Training via Dynamic Borderline Softening Mechanism}. Applied Sciences.

\bibitem{cai2024m9z}
Dongting Cai (2024). \textit{Enhancing capabilities of generative models through VAE-GAN integration: A review}. Applied and Computational Engineering.

\bibitem{u2023m2y}
K. U, T. S, T.V. Nidhin Prabhakar, et al. (2023). \textit{Adversarial Defense: A GAN-IF Based Cyber-security Model for Intrusion Detection in Software Piracy}. J. Wirel. Mob. Networks Ubiquitous Comput. Dependable Appl..

\bibitem{liu2023q2q}
Xiaobao Liu, Shuailin Su, Wenjuan Gu, et al. (2023). \textit{Super-Resolution Reconstruction of CT Images Based on Multi-scale Information Fused Generative Adversarial Networks}. Annals of Biomedical Engineering.

\bibitem{cheng2023t9b}
Shijie Cheng, Lingfeng Wang, M. Zhang, et al. (2023). \textit{SUGAN: A Stable U-Net Based Generative Adversarial Network}. Italian National Conference on Sensors.

\bibitem{luo2022rm1}
Xukang Luo, Ying Jiang, Enqiang Wang, et al. (2022). \textit{Anomaly detection by using a combination of generative adversarial networks and convolutional autoencoders}. EURASIP Journal on Advances in Signal Processing.

\bibitem{xu2022ss4}
Jialing Xu, Jingxing He, Jinqiang Gu, et al. (2022). \textit{Financial Time Series Prediction Based on XGBoost and Generative Adversarial Networks}. International Journal of Circuits, Systems and Signal Processing.

\bibitem{alshehri2022d1h}
Abeer Alshehri, Mounira Taileb, and Reem M. Alotaibi (2022). \textit{DeepAIA: An Automatic Image Annotation Model Based on Generative Adversarial Networks and Transfer Learning}. IEEE Access.

\bibitem{yeh2022yvr}
Yen-Tung Yeh, Bo-Yu Chen, and Yi-Hsuan Yang (2022). \textit{Exploiting Pre-trained Feature Networks for Generative Adversarial Networks in Audio-domain Loop Generation}. International Society for Music Information Retrieval Conference.

\bibitem{gonzlezprieto20214wh}
Ángel González-Prieto, Alberto Mozo, Edgar Talavera, et al. (2021). \textit{Dynamics of Fourier Modes in Torus Generative Adversarial Networks}. Mathematics.

\bibitem{huang2022zar}
Ying Huang, Wenhao Mei, Su Liu, et al. (2022). \textit{Asymmetric Training of Generative Adversarial Network for High Fidelity SAR Image Generation}. IEEE International Geoscience and Remote Sensing Symposium.

\bibitem{wang2020iia}
Chunzhi Wang, Pan Wu, Lingyu Yan, et al. (2020). \textit{Image classification based on principal component analysis optimized generative adversarial networks}. Multimedia tools and applications.

\bibitem{ma2021w69}
Ruixin Ma, and Junying Lou (2021). \textit{CPGAN : An Efficient Architecture Designing for Text-to-Image Generative Adversarial Networks Based on Canonical Polyadic Decomposition}. Scientific Programming.

\bibitem{baby2020e5n}
Deepak Baby (2020). \textit{iSEGAN: Improved Speech Enhancement Generative Adversarial Networks}. arXiv.org.

\bibitem{pasini2021ta3}
Massimiliano Lupo Pasini, and Junqi Yin (2021). \textit{Stable parallel training of Wasserstein conditional generative adversarial neural networks}. 2021 International Conference on Computational Science and Computational Intelligence (CSCI).

\bibitem{goyal2024ufg}
Mandeep Goyal, and Q. Mahmoud (2024). \textit{A Systematic Review of Synthetic Data Generation Techniques Using Generative AI}. Electronics.

\bibitem{wang2024v83}
Shuzhan Wang, Ruxue Jiang, Zhaoqi Wang, et al. (2024). \textit{Deep Learning-based Anomaly Detection and Log Analysis for Computer Networks}. arXiv.org.

\bibitem{liao20249ku}
Wenjie Liao, Like Wu, Shihui Xu, et al. (2024). \textit{A Novel Approach for Intelligent Fault Diagnosis in Bearing With Imbalanced Data Based on Cycle-Consistent GAN}. IEEE Transactions on Instrumentation and Measurement.

\bibitem{peng2024kkw}
Yingying Peng (2024). \textit{A Comparative Analysis Between GAN and Diffusion Models in Image Generation}. Transactions on Computer Science and Intelligent Systems Research.

\bibitem{luo2024znt}
Yihong Luo, Xiaolong Chen, Tianyang Hu, et al. (2024). \textit{You Only Sample Once: Taming One-Step Text-To-Image Synthesis by Self-Cooperative Diffusion GANs}. International Conference on Learning Representations.

\bibitem{chen2024ajr}
Xin Chen, Zaigang Chen, Shiqian Chen, et al. (2024). \textit{Unsupervised GAN With Fine-Tuning: A Novel Framework for Induction Motor Fault Diagnosis in Scarcely Labeled Sample Scenarios}. IEEE Transactions on Instrumentation and Measurement.

\bibitem{song2024htg}
Xiangjin Song, Zhicheng Liu, and Zhaowei Wang (2024). \textit{Rolling bearing fault diagnosis in electric motors based on IDIG-GAN under small sample conditions}. Measurement science and technology.

\bibitem{qin2024a4b}
Zhaohui Qin, Faguo Huang, Jiafang Pan, et al. (2024). \textit{Improved Generative Adversarial Network for Bearing Fault Diagnosis with a Small Number of Data and Unbalanced Data}. Symmetry.

\bibitem{tibermacine2025pye}
Imad Eddine Tibermacine, Samuele Russo, Francesco Citeroni, et al. (2025). \textit{Adversarial denoising of EEG signals: a comparative analysis of standard GAN and WGAN-GP approaches}. Frontiers in Human Neuroscience.

\bibitem{baoueb2024rlq}
Teysir Baoueb, Haocheng Liu, Mathieu Fontaine, et al. (2024). \textit{SpecDiff-GAN: A Spectrally-Shaped Noise Diffusion GAN for Speech and Music Synthesis}. IEEE International Conference on Acoustics, Speech, and Signal Processing.

\bibitem{broll2024edy}
Alexander Broll, M. Rosentritt, Thomas Schlegl, et al. (2024). \textit{A data-driven approach for the partial reconstruction of individual human molar teeth using generative deep learning}. Frontiers Artif. Intell..

\bibitem{wang20245dt}
Yumiao Wang, Chuanfei Zang, Bo Yu, et al. (2024). \textit{WTE-CGAN Based Signal Enhancement for Weak Target Detection}. IEEE Geoscience and Remote Sensing Letters.

\bibitem{megahed2024c23}
Mohammed Megahed, and Ammar Mohammed (2024). \textit{Collaborative-GAN: An Approach for Stabilizing the Training Process of Generative Adversarial Network}. IEEE Access.

\bibitem{zhang2024k8a}
Xiurong Zhang, Shaoqian Fan, and Daoliang Li (2024). \textit{Spectral normalization generative adversarial networks for photovoltaic power scenario generation}. IET Renewable Power Generation.

\bibitem{bhat202445j}
Ranjith Bhat, and Raghu Nanjundegowda (2024). \textit{A Review on Comparative Analysis of Generative Adversarial Networks’ Architectures and Applications}. Journal of Robotics and Control (JRC).

\bibitem{ler20248xg}
Fiete Lüer, and Christian Böhm (2024). \textit{Anomaly Detection using Generative Adversarial Networks Reviewing methodological progress and challenges}. SIGKDD Explorations.

\bibitem{purwono2025spz}
Purwono Purwono, Annastasya Nabila Elsa Wulandari, Alfian Ma’arif, et al. (2025). \textit{Understanding Generative Adversarial Networks (GANs): A Review}. Control Systems and Optimization Letters.

\bibitem{roy2024k91}
Arunava Roy, and Dipankar Dasgupta (2024). \textit{A Distributed Conditional Wasserstein Deep Convolutional Relativistic Loss Generative Adversarial Network With Improved Convergence}. IEEE Transactions on Artificial Intelligence.

\bibitem{seon202526r}
Joonho Seon, Seongwoo Lee, Youngghyu Sun, et al. (2025). \textit{Least Information Spectral GAN With Time-Series Data Augmentation for Industrial IoT}. IEEE Transactions on Emerging Topics in Computational Intelligence.

\bibitem{ni2024y70}
Yao Ni, and Piotr Koniusz (2024). \textit{$\bigcirc\!\!\!\!\bigcirc$ CHAIN: Enhancing Generalization in Data-Efficient GANs via LipsCHitz Continuity ConstrAIned Normalization}. Computer Vision and Pattern Recognition.

\bibitem{ye2024n41}
Ming Ye, Cunhua Pan, Yinfei Xu, et al. (2024). \textit{Generative Adversarial Networks-Based Channel Estimation for Intelligent Reflecting Surface Assisted mmWave MIMO Systems}. IEEE Transactions on Cognitive Communications and Networking.

\bibitem{pajuhanfard2024ult}
Mohammadsaleh Pajuhanfard, Rasoul Kiani, and Victor S. Sheng (2024). \textit{Survey of Quantum Generative Adversarial Networks (QGAN) to Generate Images}. Mathematics.

\bibitem{eskandarinasab202431h}
MohammadReza EskandariNasab, S. M. Hamdi, and S. F. Boubrahimi (2024). \textit{ChronoGAN: Supervised and Embedded Generative Adversarial Networks for Time Series Generation}. International Conference on Machine Learning and Applications.

\bibitem{deebani202549r}
Wejdan Deebani, Lubna Aziz, Arshad Aziz, et al. (2025). \textit{Synergistic transfer learning and adversarial networks for breast cancer diagnosis: benign vs. invasive classification}. Scientific Reports.

\bibitem{ali2024ks3}
Abid Ali, Muhammad Sharif, Muhammad Shahzad Faisal, et al. (2024). \textit{Brain Tumor Segmentation Using Generative Adversarial Networks}. IEEE Access.

\bibitem{ju2024uai}
Xiangui Ju, Chi-Ho Lin, Suan Lee, et al. (2024). \textit{Melanoma classification using generative adversarial network and proximal policy optimization}. Photochemistry and Photobiology.

\bibitem{xu2024u5a}
Chi Xu, Haozheng Xu, and S. Giannarou (2024). \textit{Distance Regression Enhanced With Temporal Information Fusion and Adversarial Training for Robot-Assisted Endomicroscopy}. IEEE Transactions on Medical Imaging.

\bibitem{elbaz2025wzb}
Mostafa Elbaz, Wael Said, G. Mahmoud, et al. (2025). \textit{A dual GAN with identity blocks and pancreas-inspired loss for renewable energy optimization}. Scientific Reports.

\bibitem{chang2024c0a}
Yuanhong Chang, Jinglong Chen, Rong Su, et al. (2024). \textit{Two-Phase Dual-Adversarial Agents With Multivariate Information for Unsupervised Anomaly Detection of IIoT-Edge Devices}. IEEE Internet of Things Journal.

\bibitem{guo2024y0l}
Pang Guo, and Yining Chen (2024). \textit{Enhanced Yield Prediction in Semiconductor Manufacturing: Innovative Strategies for Imbalanced Sample Management and Root Cause Analysis}. International Symposium on the Physical and Failure Analysis of Integrated Circuits.

\bibitem{peng2024crk}
Jun Peng, Kaiyi Chen, Yuqing Gong, et al. (2024). \textit{Cyclic Consistent Image Style Transformation: From Model to System}. Applied Sciences.

\end{thebibliography}

\end{document}