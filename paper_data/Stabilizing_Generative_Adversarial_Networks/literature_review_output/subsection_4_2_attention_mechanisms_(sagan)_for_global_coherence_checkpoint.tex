\subsection*{Attention Mechanisms (SAGAN) for Global Coherence}

Traditional Generative Adversarial Networks (GANs), predominantly built upon convolutional neural networks (CNNs), often struggle to capture long-range dependencies across distant regions of an image due to the inherently local nature of convolutional operations. This limitation frequently results in generated samples that lack global coherence, exhibiting inconsistencies or semantic disconnects between different parts of the image. Addressing this fundamental challenge, the integration of self-attention mechanisms into GAN architectures marked a significant advancement, profoundly enhancing the ability of these models to synthesize complex and realistic structures with improved global consistency.

A pivotal work in this domain is the Self-Attention Generative Adversarial Network (SAGAN) introduced by \cite{zhang2019self}. SAGAN innovatively incorporated self-attention layers into both the generator and the discriminator. This architectural modification allowed the networks to compute responses at a given position by attending to all other positions in the feature map, effectively modeling relationships between distant parts of an image. By enabling the generator to draw details from across the entire image and the discriminator to evaluate the global consistency of these details, SAGAN significantly improved the visual quality and global coherence of generated samples, leading to more realistic and semantically consistent outputs, particularly in intricate scenes with complex object arrangements.

The success of SAGAN in achieving stable training and high-fidelity generation was not solely due to the attention mechanism itself, but also its judicious combination with robust stabilization techniques. For instance, SAGAN notably adopted Spectral Normalization, a method proposed by \cite{miyato2018spectral}. Spectral Normalization enforces the Lipschitz continuity constraint on the discriminator's weights, thereby stabilizing the training process and preventing issues like vanishing or exploding gradients. This foundational stability provided by techniques like Spectral Normalization was crucial, allowing the architectural innovation of self-attention to effectively learn and contribute to improved image quality without being hampered by training instabilities.

The architectural innovation demonstrated by SAGAN, particularly its ability to model global contextual information, paved the way for subsequent advancements in high-fidelity image synthesis. While not directly an attention-focused paper, the insights gained from SAGAN's success in generating globally coherent images contributed to the broader understanding of what makes GANs perform well at scale. For example, \cite{brock2019large} showcased the power of scaling up GANs to unprecedented sizes (BigGAN), combining various stabilization techniques, including Spectral Normalization, with large batch sizes and the truncation trick to achieve state-of-the-art fidelity and diversity. BigGAN, in essence, leveraged the architectural and stability lessons from works like SAGAN to push the boundaries of what was possible in natural image synthesis, further solidifying the importance of robust architectures capable of capturing both local details and global structures.

In conclusion, the advent of attention mechanisms, as exemplified by SAGAN, fundamentally transformed GANs' capacity for generating globally coherent and semantically consistent images. By moving beyond the limitations of purely local convolutional operations, SAGAN demonstrated that explicitly modeling long-range dependencies is critical for synthesizing complex and realistic scenes. While subsequent works have further refined GAN architectures and scaling strategies, the principle of incorporating mechanisms for global context understanding, pioneered by SAGAN, remains a cornerstone for achieving high-fidelity and perceptually convincing generative models. Future research continues to explore more efficient and adaptive ways to integrate attention, addressing computational costs and seeking even finer-grained control over global image properties.