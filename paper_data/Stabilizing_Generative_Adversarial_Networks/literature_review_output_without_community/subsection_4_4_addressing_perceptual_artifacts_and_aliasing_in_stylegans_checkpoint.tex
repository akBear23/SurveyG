\subsection*{Addressing Perceptual Artifacts and Aliasing in StyleGANs}

While early Generative Adversarial Networks (GANs) focused on achieving stable training and basic image synthesis, the StyleGAN series marked a significant shift towards unprecedented levels of photorealism and controllable generation. However, even the groundbreaking StyleGAN architecture \cite{Karras2019} exhibited certain perceptual artifacts and fundamental signal processing issues that limited its realism and consistency, particularly when generating high-resolution content or animating latent space interpolations. Subsequent refinements in the series, namely StyleGAN2 \cite{Karras2020} and StyleGAN3 \cite{Karras2021}, meticulously addressed these limitations, pushing the boundaries of synthetic image quality and robustness.

The initial StyleGAN model \cite{Karras2019} revolutionized image synthesis by introducing a style-based generator, a mapping network, and Adaptive Instance Normalization (AdaIN) layers. This architecture enabled highly disentangled control over various visual attributes, allowing for intuitive manipulation of generated images. Despite its success in generating visually compelling and diverse outputs, StyleGAN suffered from noticeable "droplet" artifacts and a lack of perfect disentanglement, where changes in one latent dimension could inadvertently affect unrelated visual features.

To overcome these shortcomings, StyleGAN2 \cite{Karras2020} undertook a comprehensive analysis of the generator architecture and training process, identifying several sources of these perceptual artifacts. The authors found that the progressive growing scheme, commonly used in earlier high-resolution GANs, and certain aspects of instance normalization contributed to these issues. StyleGAN2 introduced several key innovations, most notably **path length regularization**, which aimed to encourage a more "well-behaved" latent space. By regularizing the mapping from latent codes to features, path length regularization ensured that a fixed-size step in the latent space always resulted in a fixed-magnitude change in the image space, thereby improving disentanglement and significantly reducing the visually distracting "droplet" artifacts. Furthermore, StyleGAN2 redesigned the generator by removing the progressive growing and replacing instance normalization with a more robust weight demodulation technique, leading to enhanced image quality and stability.

Despite the significant improvements in StyleGAN2, a more fundamental limitation remained: aliasing. This problem, inherent in discrete signal processing, manifested as static high-frequency details that did not move correctly when images were translated or rotated, leading to a "texture sticking" effect and hindering smooth animation. StyleGAN3 \cite{Karras2021} meticulously diagnosed this issue, recognizing that previous GANs, including StyleGAN2, implicitly suffered from aliasing due to their reliance on discrete pixel grids and operations that were not perfectly translation-equivariant. To address this, StyleGAN3 introduced a radical redesign of the generator to be truly **alias-free**. This was achieved by incorporating **anti-aliasing filters** at every layer of the generator, ensuring that high-frequency information was handled correctly and consistently across different resolutions and transformations. By making the generator fully translation-equivariant, StyleGAN3 enabled generated content to appear consistent and realistic even under continuous transformations, drastically improving the quality of latent space interpolations and animation capabilities.

The advancements from StyleGAN2's path length regularization to StyleGAN3's alias-free architecture represent a continuous effort to refine the underlying signal processing principles of generative models. These innovations not only enhanced the realism and perceptual quality of generated images but also made GANs more robust and versatile for applications requiring high consistency, such as video synthesis or interactive content creation. While StyleGAN3 achieved unprecedented levels of control and fidelity, the computational cost associated with its alias-free design and the general challenge of scaling GANs to extremely diverse, large-scale datasets remain areas for ongoing research and optimization.