\documentclass[12pt,a4paper]{article}
    \usepackage[utf8]{inputenc}
    \usepackage[T1]{fontenc}
    \usepackage{amsmath,amsfonts,amssymb}
    \usepackage{graphicx}
    \usepackage[margin=2.5cm]{geometry}
    \usepackage{setspace}
    \usepackage{natbib}
    \usepackage{url}
    \usepackage{hyperref}
    \usepackage{booktabs}
    \usepackage{longtable}
    \usepackage{array}
    \usepackage{multirow}
    \usepackage{wrapfig}
    \usepackage{float}
    \usepackage{colortbl}
    \usepackage{pdflscape}
    \usepackage{tabu}
    \usepackage{threeparttable}
    \usepackage{threeparttablex}
    \usepackage[normalem]{ulem}
    \usepackage{makecell}
    \usepackage{xcolor}

    % Set line spacing
    \doublespacing

    % Configure hyperref
    \hypersetup{
        colorlinks=true,
        linkcolor=blue,
        filecolor=magenta,      
        urlcolor=cyan,
        citecolor=red,
    }

    % Title and author information
    \title{A Comprehensive Literature Review with Self-Reflection}
    \author{Literature Review}
    \date{\today}

    \begin{document}

    \maketitle

    % Abstract (optional)
    \begin{abstract}
    This literature review provides a comprehensive analysis of recent research in the field. The review synthesizes findings from 194 research papers, identifying key themes, methodological approaches, and future research directions.
    \end{abstract}

    \newpage
    \tableofcontents
    \newpage

    \label{sec:introduction}

\section{Introduction}
\label{sec:introduction}

\subsection{The Promise of Generative Adversarial Networks}
\label{sec:1\_1\_the\_promise\_of\_generative\_adversarial\_networks}

The introduction of Generative Adversarial Networks (GANs) by Goodfellow et al. in 2014 \cite{goodfellow2014generative} heralded a transformative era in artificial intelligence, particularly within generative modeling. This novel framework immediately captivated the research community by proposing a unique adversarial training paradigm that promised unprecedented capabilities in data synthesis and unsupervised learning \cite{jabbar2020aj0, bhat202445j}. At its core, a GAN comprises two competing neural networks: a generator (G) and a discriminator (D). The generator's objective is to learn the underlying distribution of real data and produce synthetic samples that are indistinguishable from authentic ones. Concurrently, the discriminator's role is to become highly proficient at differentiating between real data samples and those fabricated by the generator. This dynamic, zero-sum game, where both networks iteratively refine their strategies, enables the generator to progressively synthesize novel, high-quality content without explicit programming of features or rules \cite{goodfellow2014generative}.

This implicit learning of complex data distributions represented a significant departure from prior generative models, fundamentally reshaping the landscape of generative AI. Before GANs, models like Variational Autoencoders (VAEs) and autoregressive models were prominent. While VAEs offered a probabilistic framework for latent space representation, they often struggled to produce perceptually sharp and realistic samples, frequently yielding blurry outputs due to their reliance on reconstruction loss and explicit density modeling \cite{goyal2024ufg, salmona202283g}. Autoregressive models, on the other hand, could generate high-quality samples but suffered from slow, sequential generation processes. GANs, by contrast, leveraged their adversarial objective to directly push for high fidelity and realism, aiming for generated samples that could fool a sophisticated discriminator. This ability to generate sharp, coherent, and seemingly authentic data was a key driver of the initial excitement, showcasing a profound capability for creative and analytical tasks that was previously unattainable \cite{wang2019w53}.

The broad potential for generating novel, high-quality content across diverse domains was immediately apparent and widely discussed. A primary aspiration was photorealistic image synthesis, where GANs could create entirely new faces, landscapes, or objects that were virtually indistinguishable from real photographs \cite{pieters2018jh1, wang2019w53}. This capability opened doors for applications in digital art, entertainment, and virtual reality, offering tools for artists and designers to generate complex visual content with unprecedented ease. Beyond creative endeavors, GANs demonstrated significant promise in analytical applications, such as data augmentation for scientific research \cite{wang2019w53}. In fields like medical imaging or specialized industrial applications, where acquiring large, diverse, and annotated datasets is often costly, time-consuming, or ethically challenging, GANs offered a powerful solution to generate synthetic data. This synthetic data could then be used to expand training sets, improve the robustness of downstream machine learning models, and accelerate discoveries in areas like disease diagnosis or material science \cite{goyal2024ufg}. Furthermore, the framework showed potential for tasks like image-to-image translation, facial attribute manipulation, and style transfer, highlighting its versatility in learning complex mappings between visual domains \cite{wang2019w53}.

In conclusion, the initial conception of Generative Adversarial Networks presented a revolutionary approach to unsupervised learning, characterized by its unique adversarial training paradigm. The excitement stemmed from their unprecedented ability to implicitly learn and reproduce the complexities of real-world data, promising a future where AI could be a true partner in creativity, scientific discovery, and diverse analytical tasks. However, this profound promise was immediately tempered by the inherent difficulties of optimizing the adversarial minimax game. The delicate balance required for stable training, coupled with the non-convex nature of the objective function, led to early recognition of significant challenges such as vanishing gradients, mode collapse, and general training instability \cite{goodfellow2014generative, jabbar2020aj0, bhat202445j, chu2020zbv, salmona202283g}. These fundamental problems, present from GANs' inception, would soon become the central focus of extensive research, driving the field towards developing robust stabilization techniques.
\subsection{The Central Challenge: Training Instability}
\label{sec:1\_2\_the\_central\_challenge:\_training\_instability}

Despite their revolutionary potential in generative modeling, Generative Adversarial Networks (GANs) are fundamentally characterized by profound training instabilities. This inherent difficulty, widely acknowledged across the literature \cite{jabbar2020aj0, wiatrak20194ib, chu2020zbv}, manifests primarily as unreliable convergence, oscillating performance, and specific failure modes. These issues stem directly from the delicate, non-cooperative nature of the adversarial min-max game, where a generator ($G$) and a discriminator ($D$) are simultaneously optimized. This adversarial dynamic makes GANs exceptionally sensitive to a multitude of factors, including hyperparameter choices, network architectures, and initialization strategies, frequently leading to suboptimal, often uninterpretable, generated outputs \cite{wang2019w53}. This central challenge has not only defined much of the research trajectory in the field but also underscores the critical need for robust stabilization techniques, highlighting the core problem that this review addresses.

The core of GAN instability lies in the complex dynamics of their two-player game. Unlike traditional optimization problems that aim to minimize a single loss function towards a stable minimum, GANs involve a continuous competition to find a Nash equilibrium. This minimax objective often lacks a unique, stable equilibrium, or if one exists, it is notoriously difficult to reach through standard gradient-based optimization methods \cite{liang2018r52, grnarova20171tc}. The non-convex nature of the GAN objective, combined with the continuous interplay where one network's improvement alters the other's optimal strategy, often leads to complex dynamics such as limit cycles or rotational behavior in the parameter space rather than stable convergence \cite{gonzlezprieto20214wh, chu2020zbv}. This results in training curves that frequently oscillate wildly, with generated sample quality fluctuating significantly throughout the training process. Such erratic behavior makes GANs notoriously sensitive to hyperparameter choices, such as the relative learning rates for the generator and discriminator, batch sizes, and optimizer configurations \cite{xiang20171at}. Even slight deviations from optimal settings can lead to divergence, poor quality samples, or a complete failure to train, making the process of finding a stable configuration a significant practical hurdle.

This instability manifests in well-documented failure modes that severely limit GANs' utility. Most notably, these include vanishing gradients, where the generator ceases to receive meaningful learning signals, and mode collapse, where the generator produces only a limited variety of samples, failing to capture the full diversity of the real data distribution \cite{goodfellow2014generative, salimans2016improved, arjovsky2017wasserstein}. These persistent issues, which plagued early architectures and continue to challenge complex models, represent critical symptoms of the underlying optimization difficulties and will be examined in detail in Subsection 2.3 after the foundational GAN concepts are established.

Beyond these fundamental algorithmic challenges, the practical process of debugging and evaluating unstable GANs is notoriously complex and time-consuming. Unlike supervised learning where validation loss or accuracy directly correlates with model performance, GAN loss values often do not reliably indicate the perceptual quality of generated outputs \cite{wenzel20225g3, jabbar2020aj0}. A decreasing generator loss might not signify better samples, and an increasing discriminator loss could be a sign of either effective training (discriminator being fooled) or a failing generator. This disconnect forces researchers and practitioners to rely heavily on subjective visual inspection of generated samples, a process that is both labor-intensive and prone to misinterpretation. Without a clear, quantitative signal for convergence or quality, determining when to stop training, comparing different models, or diagnosing the root cause of poor performance becomes a significant practical challenge, further emphasizing the urgency of effective stabilization methods \cite{karras2017raw}.

In summary, the training instability of GANs, encompassing unreliable convergence, extreme sensitivity to configuration, the difficulty of debugging due to uninformative metrics, and specific failure modes like vanishing gradients and mode collapse, is not merely a practical inconvenience but a fundamental theoretical and algorithmic challenge. These issues directly impede the ability of GANs to reliably learn complex, high-dimensional data distributions, generate diverse and high-quality samples, and achieve stable training. The persistent nature of these problems has been the primary impetus for extensive research into more robust theoretical frameworks, novel loss functions, and advanced regularization techniques, driving the evolution of the field towards more stable and effective generative models.


\label{sec:foundational_concepts_and_early_challenges_of_gans}

\section{Foundational Concepts and Early Challenges of GANs}
\label{sec:foundational\_concepts\_\_and\_\_early\_challenges\_of\_gans}

\subsection{The Original GAN Framework: Adversarial Minimax Game}
\label{sec:2\_1\_the\_original\_gan\_framework:\_adversarial\_minimax\_game}

The seminal work by \cite{goodfellow2014generative} introduced Generative Adversarial Networks (GANs), a groundbreaking framework that recast generative modeling as a dynamic, zero-sum game between two competing neural networks. This innovative paradigm immediately captured significant attention for its potential to synthesize highly realistic data, particularly images, by learning complex data distributions implicitly \cite{jabbar2020aj0, bhat202445j}. The foundational understanding of this original framework is crucial for appreciating the subsequent extensive research aimed at stabilizing its delicate adversarial balance and overcoming its inherent optimization challenges.

At its core, the original GAN architecture comprises two distinct neural networks: a generator ($G$) and a discriminator ($D$). The generator's primary function is to learn a mapping from a simple prior noise distribution $p\_z(z)$ (typically a uniform or Gaussian distribution) to the intricate real data distribution $p\_{data}(x)$. Through this mapping, the generator produces synthetic samples, denoted as $G(z)$, that aim to be indistinguishable from authentic data. Conversely, the discriminator acts as a binary classifier, tasked with differentiating between real samples drawn directly from $p\_{data}(x)$ and fake samples generated by $G$. This adversarial interplay is conceptualized as a minimax game, where the discriminator strives to maximize its classification accuracy, while the generator simultaneously endeavors to minimize the discriminator's ability to discern between real and fake, thereby producing increasingly convincing synthetic data.

The objective function for this adversarial game is mathematically defined as:
$$ \min\_G \max\_D V(D, G) = \mathbb{E}\_{x \sim p\_{data}(x)}[\log D(x)] + \mathbb{E}\_{z \sim p\_z(z)}[\log(1 - D(G(z)))] $$
During the training process, the discriminator $D$ is optimized to maximize $V(D,G)$. This entails learning to assign high probabilities to real data samples ($D(x) \approx 1$) and low probabilities to generated samples ($D(G(z)) \approx 0$). Concurrently, the generator $G$ is updated to minimize $V(D,G)$, which is equivalent to maximizing $\mathbb{E}\_{z \sim p\_z(z)}[\log D(G(z))]$. This objective compels $G$ to produce samples $G(z)$ for which $D(G(z))$ approaches $1$, effectively making its outputs appear real to the discriminator. Theoretically, this adversarial game converges to a unique Nash equilibrium. At this equilibrium, the generator perfectly replicates the real data distribution ($p\_g = p\_{data}$), and the discriminator outputs $0.5$ for all inputs, signifying its inability to distinguish between real and fake samples.

A critical theoretical implication of this original formulation is that, at the global optimum, the objective function corresponds to minimizing the Jensen-Shannon Divergence (JSD) between the real data distribution ($p\_{data}$) and the generated data distribution ($p\_g$). While JSD is a symmetric and bounded measure of similarity between probability distributions, its properties proved to be a significant source of practical instability in GAN training \cite{jabbar2020aj0}. Specifically, when the supports of $p\_{data}$ and $p\_g$ are disjoint or have negligible overlap—a common occurrence, especially in high-dimensional data spaces like images—the JSD becomes a constant value. In such scenarios, the gradients of the discriminator with respect to the generator's parameters can vanish, providing little to no meaningful learning signal to the generator. This vanishing gradient problem makes it exceedingly difficult for the generator to learn effectively, particularly during the early stages of training when $p\_g$ is far from $p\_{data}$.

Beyond vanishing gradients, the delicate balance inherent in the adversarial minimax game often led to training instability and convergence issues \cite{bhat202445j}. The competitive nature meant that if one network became too powerful too quickly, the training process could derail. For instance, an overly strong discriminator could consistently output $0$ or $1$ for generated samples, leading to saturated gradients for the generator. Conversely, a weak discriminator might provide an insufficient learning signal, allowing the generator to produce poor-quality samples without significant penalty. This imbalance could manifest as oscillations in performance or, more critically, as mode collapse, where the generator produces only a limited variety of samples, failing to capture the full diversity of the real data distribution \cite{jabbar2020aj0}. The theoretical analysis of GAN dynamics, such as that by \cite{gonzlezprieto20214wh}, further elucidates why the training process is inherently unstable; they show that convergent orbits in GANs are often small perturbations of periodic orbits, implying that Nash equilibria can act as spiral attractors, which theoretically justifies the observed slow and unstable training.

In summary, the original GAN framework by Goodfellow et al. was a revolutionary contribution, offering a powerful new paradigm for generative modeling. However, its reliance on the Jensen-Shannon Divergence as the underlying objective function, coupled with the inherent competitive dynamics of the minimax game, immediately exposed fundamental challenges. These included the pervasive problem of vanishing gradients when data distributions had non-overlapping supports, leading to training instability and the notorious issue of mode collapse. These initial theoretical and practical difficulties underscored the need for significant advancements in objective functions, architectural designs, and training methodologies, setting the stage for the extensive research that followed to stabilize and enhance generative adversarial models.
\subsection{Early Architectural Guidelines: Deep Convolutional GANs (DCGANs)}
\label{sec:2\_2\_early\_architectural\_guidelines:\_deep\_convolutional\_gans\_(dcgans)}

The initial formulation of Generative Adversarial Networks (GANs) \cite{Goodfellow2014} presented a powerful theoretical framework for generative modeling, but their practical implementation was plagued by significant training instability and difficulties in convergence, often producing incoherent or limited-diversity outputs. A crucial methodological progression towards making GANs a more implementable framework was the introduction of Deep Convolutional Generative Adversarial Networks (DCGANs) by \cite{Radford2015}.

DCGANs marked the first significant step towards practical GANs by effectively integrating Convolutional Neural Networks (CNNs) into both the generator and discriminator architectures. This integration leveraged the hierarchical feature learning capabilities of CNNs, enabling the generation of more coherent and visually plausible images compared to earlier fully-connected architectures. Beyond simply using CNNs, \cite{Radford2015} introduced a set of architectural heuristics that provided initial stability to GAN training and enabled the generation of more coherent images, marking a crucial methodological progression from the abstract GAN concept to a more implementable framework.

Several key architectural guidelines were established. To address training instability and facilitate deeper networks, batch normalization layers were introduced in both the generator and discriminator. Batch normalization helps stabilize learning by normalizing the input to each layer, preventing internal covariate shift and allowing for higher learning rates, which was vital for the deeper convolutional structures. Specific activation functions were also prescribed: ReLU (Rectified Linear Unit) was predominantly used in the generator for all layers except the output, which typically used Tanh to produce pixel values in a normalized range. For the discriminator, LeakyReLU was employed, providing a non-zero gradient for negative inputs and helping to prevent 'dying ReLU' problems, thereby contributing to better gradient flow and more stable adversarial training.

A pivotal architectural choice was the avoidance of pooling layers in favor of strided convolutions. In the generator, fractional-strided convolutions (often referred to as transposed convolutions) were used for spatial upsampling, allowing the network to learn its own upsampling strategy rather than relying on fixed interpolation. Conversely, the discriminator utilized strided convolutions for spatial downsampling. This approach allowed the network to learn more effective spatial transformations, preserving more information and often leading to better image quality than traditional pooling operations.

These architectural heuristics provided initial stability to GAN training, moving the field from an abstract concept to a more robust and implementable framework. The structured use of CNNs and the proposed architectural choices enabled DCGANs to generate images with significantly improved visual quality and coherence, demonstrating the potential of GANs for unsupervised representation learning. For instance, \cite{Radford2015} showed that the learned features in the discriminator could be effectively used for classification tasks, and that latent space arithmetic could produce meaningful semantic manipulations in generated images, such as interpolating between gender or expressions.

Despite these advancements, DCGANs still faced limitations. While stability was improved, training remained sensitive to hyperparameter choices and could still suffer from issues like mode collapse, where the generator produces a limited variety of samples. The resolution of generated images was also relatively modest compared to later advancements. Thus, while DCGANs established fundamental architectural principles for deep generative models and showcased the immense potential of GANs, their inherent challenges in achieving consistent stability and scaling to higher resolutions laid the groundwork for subsequent research into more robust training methodologies and advanced architectures.
\subsection{Persistent Problems: Vanishing Gradients and Mode Collapse}
\label{sec:2\_3\_persistent\_problems:\_vanishing\_gradients\_\_and\_\_mode\_collapse}

Despite the initial promise of Generative Adversarial Networks (GANs) \cite{goodfellow2014generative}, early architectures were plagued by significant training instabilities, primarily manifesting as vanishing gradients and mode collapse. These issues severely hampered the models' ability to learn diverse and high-fidelity data distributions, highlighting fundamental limitations in the original adversarial training framework.

The problem of vanishing gradients arises when the discriminator becomes overly effective at distinguishing between real and fake samples. In such scenarios, particularly when the real and generated data distributions have non-overlapping supports, the Jensen-Shannon divergence (JSD) used in the original GAN objective saturates. This saturation means the discriminator's loss becomes constant and near zero, providing negligible gradients to the generator. Consequently, the generator receives no meaningful learning signal, effectively halting its progress and preventing it from improving its sample quality. This fundamental limitation was acknowledged as a critical barrier to stable GAN training \cite{roth2017eui}.

Concurrently, mode collapse emerged as another pervasive issue. Instead of capturing the full diversity of the real data distribution, the generator would often converge to producing only a limited variety of samples, frequently focusing on a few distinct "modes" that were particularly effective at fooling the discriminator. This behavior results in a generator that fails to represent the true complexity and richness of the target data, leading to repetitive and uninteresting outputs. For instance, if trained on a dataset of diverse animal images, a generator suffering from mode collapse might only produce images of cats, ignoring dogs, birds, and other animals present in the training data.

Early attempts to address these instabilities often involved regularization techniques. \cite{che2016kho} proposed Mode Regularized Generative Adversarial Networks (MRGANs) to tackle mode collapse and instability. They argued that the "bad behaviors" of GANs stem from the discriminator's functional shape in high-dimensional spaces, which can lead to training stagnation or misdirection of probability mass. Their approach introduced several regularizers to the objective function, aiming to stabilize training and promote a fairer distribution of probability mass across data modes, thereby mitigating the missing modes problem. Similarly, \cite{roth2017eui} introduced a regularization approach specifically to stabilize GAN training, directly addressing the fragility caused by dimensional mismatch or non-overlapping support between the model and data distributions. They noted that such non-overlapping supports cause the density ratio and associated f-divergence to be undefined, a direct precursor to vanishing gradients. Their low-computational-cost regularizer aimed to overcome this fundamental limitation, making GAN models more reliable.

A more fundamental theoretical solution to the vanishing gradient problem was introduced by \cite{arjovsky2017ze5} with the Wasserstein Generative Adversarial Network (WGAN). This work fundamentally altered the loss function by replacing the Jensen-Shannon divergence with the Earth-Mover (Wasserstein-1) distance. The key insight was that the Wasserstein distance provides a continuous and differentiable metric even when the distributions are disjoint, ensuring that the critic (discriminator) can always provide a meaningful gradient to the generator. This property directly addressed the vanishing gradient problem, as the generator would consistently receive a learning signal regardless of how well the critic performed. Furthermore, by providing a smoother loss landscape, the Wasserstein distance inherently contributed to alleviating mode collapse by encouraging the generator to explore a broader range of the data distribution.

These persistent problems of vanishing gradients and mode collapse underscored that mere architectural tweaks were insufficient to stabilize GAN training. Instead, they highlighted the critical need for more fundamental theoretical and algorithmic solutions that could provide robust learning signals and encourage comprehensive mode coverage. This realization propelled the next wave of research, moving beyond empirical fixes to explore deeper mathematical and algorithmic foundations for GAN stabilization.


\label{sec:core_stability_mechanisms:_divergence,_distance,_and_advanced_regularization}

\section{Core Stability Mechanisms: Divergence, Distance, and Advanced Regularization}
\label{sec:core\_stability\_mechanisms:\_divergence,\_distance,\_\_and\_\_advanced\_regularization}

\subsection{Shifting from Divergence to Distance: Wasserstein GANs}
\label{sec:3\_1\_shifting\_from\_divergence\_to\_distance:\_wasserstein\_gans}

Early Generative Adversarial Networks (GANs) frequently suffered from training instability, particularly vanishing gradients and mode collapse, largely attributable to the choice of divergence metric used to measure the distance between the generator's distribution and the true data distribution. This fundamental challenge was profoundly addressed by the introduction of Wasserstein GANs (WGANs), which marked a pivotal theoretical and practical advancement in the field.

The groundbreaking work by \cite{arjovsky2017ze5} introduced Wasserstein GANs, fundamentally altering the GAN loss function by replacing the problematic Jensen-Shannon (JS) divergence with the Earth Mover's (or Wasserstein-1) distance. The JS divergence, while theoretically sound for overlapping distributions, proved highly unsuitable for the typical scenario in GAN training where the generated and real data distributions often lie on low-dimensional manifolds and are non-overlapping. In such cases, the JS divergence becomes a constant, providing zero gradients almost everywhere, which severely hinders the discriminator's ability to provide meaningful feedback to the generator and leads to the notorious vanishing gradient problem.

The shift to the Earth Mover's distance offered a robust solution to this dilemma. Unlike f-divergences (like JS divergence), the Wasserstein distance provides a smoother and non-zero gradient everywhere, even when the two distributions are non-overlapping. This crucial property ensures that the critic (discriminator in WGANs) can always provide a useful gradient signal to the generator, regardless of how far apart the generated and real data distributions are. This directly mitigates the vanishing gradient problem, allowing for more stable and continuous learning throughout the training process. Furthermore, the Wasserstein distance offers a more meaningful loss metric that empirically correlates with the perceived quality of the generated samples, providing a reliable indicator of training progress that was often absent in traditional GANs.

A cornerstone of WGAN's theoretical stability is the requirement for its critic network to be a K-Lipschitz function. This constraint is derived from the Kantorovich-Rubinstein duality, which states that the Earth Mover's distance can be computed by finding the maximum value of a K-Lipschitz function. Enforcing this Lipschitz constraint on the critic is essential for ensuring that the critic's output is a valid approximation of the Wasserstein distance. Initially, \cite{arjovsky2017ze5} proposed weight clipping as a simple method to enforce this constraint, albeit with some practical limitations such as potentially reducing model capacity or requiring careful hyperparameter tuning. Despite these initial practical challenges, the theoretical foundation laid by WGANs, particularly the rigorous application of the Kantorovich-Rubinstein duality and the K-Lipschitz critic, represented a profound theoretical advancement. It moved GAN research from heuristic-driven stabilization to a more principled, mathematically grounded approach, paving the way for subsequent improvements in GAN training stability and performance.

In conclusion, the introduction of Wasserstein GANs by \cite{arjovsky2017ze5} marked a paradigm shift in generative modeling. By replacing the problematic Jensen-Shannon divergence with the Earth Mover's distance and introducing the K-Lipschitz critic requirement, WGANs provided a stable, theoretically sound framework that effectively addressed vanishing gradients and offered a more interpretable loss. This fundamental change not only stabilized GAN training but also opened new avenues for research into more robust and high-fidelity generative models, establishing a new benchmark for theoretical rigor in the field.
\subsection{Gradient Penalties for Robust Lipschitz Enforcement}
\label{sec:3\_2\_gradient\_penalties\_for\_robust\_lipschitz\_enforcement}

While the weight clipping strategy proposed in the original Wasserstein Generative Adversarial Network (WGAN) \cite{Arjovsky2017} was theoretically motivated to enforce the $K$-Lipschitz constraint on the critic (discriminator), it introduced significant practical limitations that often hindered model performance and stability. The fixed clipping range, a sensitive hyperparameter, could drastically limit the critic's capacity if too small, leading to underfitting and an inability to learn complex functions. Conversely, a large clipping range might not effectively enforce the Lipschitz constraint, resulting in unstable training dynamics akin to those WGAN aimed to mitigate. Furthermore, weight clipping could cause gradients to concentrate at the boundaries of the clipping range, leading to vanishing or exploding gradients in specific layers and further destabilizing the training process \cite{jabbar2020aj0, purwono2025spz}. This crude enforcement mechanism often restricted the critic's ability to learn a smooth function landscape, which is crucial for providing consistent and meaningful gradients to the generator.

Recognizing these limitations, \cite{Gulrajani2017} introduced a more robust and effective method for enforcing the Lipschitz constraint: the gradient penalty (WGAN-GP). Instead of directly manipulating the critic's weights, WGAN-GP added a regularization term to the critic's loss function that penalized the norm of its gradient with respect to its input. Specifically, this penalty term encourages the gradient norm to be close to one for samples interpolated linearly between real and generated data points. This approach ensures that the critic's gradients are smooth and well-behaved across the entire input space, without restricting the model's capacity or introducing the boundary effects observed with weight clipping \cite{Gulrajani2017}. The theoretical justification for this approach lies in the Kantorovich-Rubinstein duality, which requires the critic to be 1-Lipschitz (or $K$-Lipschitz, with $K=1$ being a common choice for simplicity and stability) for the Wasserstein distance to be accurately estimated. By penalizing deviations from a gradient norm of one, WGAN-GP directly addresses this requirement in a differentiable manner.

The introduction of the gradient penalty in WGAN-GP provided several critical advantages. Firstly, it allowed the critic to learn a much smoother function, which in turn provided more stable and informative gradients to the generator, significantly enhancing overall training stability. This explicit enforcement of smoothness and Lipschitz continuity is vital for GANs, as highlighted by \cite{chu2020zbv}, who demonstrate how such conditions contribute to the eventual stationarity of the generator during training. Secondly, by not directly constraining the weights, WGAN-GP allowed the critic to maintain its full representational capacity, enabling it to learn more complex decision boundaries and better distinguish between real and fake samples. This methodological improvement not only prevented capacity limitations but also further reduced the incidence of mode collapse, as the generator received consistent feedback across the data manifold.

However, this robustness came with a computational overhead. Calculating the gradient norm with respect to the input data requires computing second-order derivatives (or at least first-order derivatives of the critic's output with respect to its input, which are then used in the penalty term), which can be computationally expensive, especially for high-dimensional inputs and large batch sizes. This computational burden, while manageable, motivated the search for alternative and more efficient methods of Lipschitz enforcement. WGAN-GP represented a significant advancement in function space regularization, explicitly enforcing smoothness. This contrasts with implicit regularization methods that might arise from architectural choices or other normalization techniques. For instance, recent work such as CHAIN (Lipschitz Continuity Constrained Normalization) \cite{ni2024y70} has explored integrating Lipschitz constraints directly within normalization layers, offering another avenue for ensuring discriminator stability, particularly in data-efficient GANs.

In conclusion, the transition from WGAN's crude weight clipping to WGAN-GP's gradient penalty marked a crucial evolutionary step in stabilizing GAN training. By providing a robust and capacity-preserving method for Lipschitz enforcement, WGAN-GP laid the groundwork for the development of more advanced and stable GAN architectures, proving indispensable for generating high-quality and diverse samples \cite{purwono2025spz}. The theoretical soundness and practical efficacy of WGAN-GP quickly made it a standard practice in subsequent GAN architectures. Despite its widespread adoption and efficacy, the computational demands of WGAN-GP, particularly the need for gradient computation on interpolated samples, motivated the search for more computationally efficient and universally applicable methods for Lipschitz enforcement. This led to innovations like Spectral Normalization \cite{miyato2018arc}, which offered an alternative approach to constraining the discriminator's Lipschitz constant without explicit gradient penalties.
\subsection{Spectral Normalization and Dynamic Learning Rates}
\label{sec:3\_3\_spectral\_normalization\_\_and\_\_dynamic\_learning\_rates}

The pursuit of stable and efficient Generative Adversarial Network (GAN) training has been a central challenge since their inception. While early advancements like Wasserstein GANs with Gradient Penalties (WGAN-GP) significantly improved stability by enforcing the Lipschitz constraint on the discriminator, they often introduced computational overhead due to the need for gradient computations on interpolated samples \cite{gulrajani2017improved}. This computational burden and the sensitivity to the interpolation strategy motivated the search for more direct and efficient methods for Lipschitz enforcement, as well as optimized training dynamics to better manage the adversarial game \cite{jabbar2020aj0}. The theoretical underpinnings of GAN stability often point to the importance of discriminator smoothness and bounded Lipschitz constants to ensure meaningful gradients and prevent mode collapse \cite{chu2020zbv}.

A pivotal innovation addressing these challenges was \textbf{Spectral Normalization (SN)}, introduced by \cite{miyato2018arc}. SN offers an elegant and computationally efficient mechanism to enforce the 1-Lipschitz constraint on the discriminator, a critical requirement for stable training, particularly in Wasserstein-based GANs. Unlike gradient penalties, which regularize the discriminator's output gradients, SN directly normalizes the spectral norm of the weight matrices in each layer of the discriminator. The spectral norm of a matrix represents its largest singular value, and by normalizing it to 1, SN ensures that the Lipschitz constant of each individual layer, and consequently the entire discriminator network, is bounded. This direct approach makes SN computationally lighter than gradient penalties, as it avoids the need for explicit gradient computations on interpolated samples. Furthermore, SN is straightforward to implement and can be seamlessly integrated into various GAN architectures without extensive hyperparameter tuning or specific architectural modifications, making it a highly generalizable stabilization technique \cite{miyato2018arc}. By preventing the discriminator from becoming overly confident or powerful too rapidly, SN fosters smoother loss landscapes, provides more consistent and informative gradient signals to the generator, and significantly mitigates issues such as vanishing gradients and mode collapse, ultimately leading to the generation of higher-quality and more diverse samples. This method represents a refinement in the broader category of weight normalization techniques, which includes earlier approaches like Weight Normalization (WN) \cite{xiang20171at} that aimed to improve training stability by reparameterizing weights. SN, however, specifically targets the Lipschitz constant, providing a more theoretically grounded and effective solution for GANs.

Complementing Spectral Normalization, the paper by \cite{miyato2018arc} also effectively employed the \textbf{Two-Time-Scale Update Rule (TTUR)}, a technique originally proposed by \cite{heusel2017gans} to further optimize GAN training dynamics. TTUR is predicated on the understanding that the generator and discriminator, with their distinct objectives and learning challenges, often benefit from different learning rates. Instead of applying a single learning rate to both networks, TTUR allows for separate learning rates, typically setting the discriminator's learning rate to be higher than the generator's. This dynamic learning rate management is crucial for maintaining a healthy adversarial balance throughout the training process. In the context of two-player games like GANs, the interaction between the players' updates can lead to complex dynamics, where the choice of learning rate significantly impacts convergence and stability \cite{liang2018r52}. If the discriminator learns too slowly, it may fail to provide a sufficiently strong or accurate signal for the generator to improve. Conversely, if the discriminator learns too quickly and becomes overly powerful, the generator's gradients can vanish, leading to training stagnation or mode collapse. By allowing distinct update frequencies or magnitudes, TTUR prevents one network from dominating the other, facilitating a more balanced and effective adversarial game. This strategy enhances training stability, improves convergence properties, and contributes to better sample quality and diversity across a wide range of datasets.

The combined application of Spectral Normalization and the Two-Time-Scale Update Rule by \cite{miyato2018arc} marked a significant advancement in GAN research, moving towards more principled and efficient stabilization strategies. SN provides a robust, computationally light, and generalizable method for enforcing Lipschitz continuity, while TTUR optimizes the delicate adversarial balance through dynamic learning rate management. These innovations have become foundational, with SN, in particular, being widely adopted in subsequent state-of-the-art GAN architectures. The principles of Lipschitz-constrained normalization continue to be explored, with recent works like CHAIN (LipsCHitz Continuity ConstrAIned Normalization) \cite{ni2024y70} further refining normalization techniques for data-efficient GANs by focusing on gradient reduction and adaptive feature interpolation. Similarly, other regularization methods, such as Consistency Regularization \cite{zhang2019hjo} and constrained discriminator outputs \cite{chao2021ynq}, have shown to work effectively with SN, highlighting its compatibility and foundational role. Despite these advancements, the challenge of achieving perfect mode coverage and absolute training stability in increasingly complex generative models remains an active area of research, underscoring the ongoing relevance of adaptive and efficient regularization techniques exemplified by SN and TTUR.
\subsection{Alternative Loss Functions for Enhanced Stability}
\label{sec:3\_4\_alternative\_loss\_functions\_for\_enhanced\_stability}

The persistent challenge of training instability in Generative Adversarial Networks (GANs), characterized by issues such as vanishing gradients, mode collapse, and oscillating performance, has driven extensive research into modifying the core objective functions. Beyond the f-divergences initially explored, a significant line of inquiry has focused on designing alternative loss functions that provide smoother, non-saturating gradients and more robust convergence properties, thereby enhancing the overall training dynamics and generative quality. This quest highlights a continuous search for diverse mathematical perspectives to achieve stable and effective generative modeling.

A foundational contribution in this area is the introduction of Least Squares Generative Adversarial Networks (LSGANs) \cite{Mao2017}. LSGANs address the limitations of the traditional sigmoid cross-entropy loss, which can suffer from vanishing gradients when the discriminator becomes overly confident and saturates. By replacing this with a least squares loss function, LSGANs ensure that both the generator and discriminator receive meaningful, non-saturating gradients throughout training. This modification encourages the generator to produce samples closer to the decision boundary, leading to improved training stability and the generation of higher quality images compared to original GANs. Theoretically, the least squares objective implicitly minimizes the Pearson $\chi^2$ divergence, offering a distinct and often more stable mathematical perspective than the Jensen-Shannon divergence minimized by early GANs \cite{Mao2017}.

Following the principles of non-saturating objectives, the adversarial hinge loss emerged as another cornerstone for stable GAN training, particularly in high-fidelity models. The hinge loss provides a clear margin for classification, penalizing the discriminator only when its output for real samples falls below a certain positive margin, or when its output for fake samples rises above a negative margin. This margin-based formulation ensures that the discriminator does not become overly confident too early, preventing gradient saturation and providing a consistent learning signal to the generator. For the generator, the hinge loss encourages it to push fake samples beyond the discriminator's negative margin. This approach has been widely adopted in state-of-the-art architectures, including Self-Attention GANs (SAGAN) and BigGAN, due to its effectiveness in promoting stable training and high-quality image synthesis. \textcite{wang20178xf} further explored adaptive hinge loss functions, demonstrating how dynamically adjusting the margin based on the expected energy of the target distribution can lead to improved stability and performance, with theoretical proofs of convergence under certain assumptions.

These alternative loss functions represent a critical shift in GAN optimization strategies. Unlike Wasserstein GANs with Gradient Penalty (WGAN-GP) \cite{Gulrajani2017}, which primarily enforce a Lipschitz constraint on the discriminator to ensure meaningful gradients across the input space, LSGANs and hinge loss directly reshape the objective landscape itself. They achieve stability by preventing the discriminator's loss from saturating, thus providing a more consistent and robust gradient flow to the generator. This distinction is crucial: while WGAN-GP focuses on the \textit{smoothness} of the discriminator function, LSGANs and hinge loss focus on the \textit{shape} of the loss function to avoid regions of zero gradient. The theoretical framework proposed by \textcite{chu2020zbv} further elucidates the importance of smoothness and specific divergence properties for guaranteeing eventual stationarity of the generator, highlighting why non-saturating and margin-based losses contribute to stability.

Beyond these widely adopted approaches, researchers have continued to explore novel modifications to loss functions. \textcite{zadorozhnyy20208ft} introduced adaptive weighted discriminator loss functions, or "aw-loss functions." This method addresses the challenge that an equally weighted sum of real and fake losses can sometimes benefit one part of the training while harming the other, leading to instability and mode collapse. By adaptively weighting the real and fake components of the discriminator's loss based on their gradients, aw-loss functions guide the discriminator's training in a direction that explicitly benefits overall GAN stability. This dynamic balancing act has shown significant improvements in Inception Scores (IS) and Fréchet Inception Distance (FID) metrics on various datasets, demonstrating the value of fine-grained control over loss components.

Another innovative approach is presented by Constrained Generative Adversarial Networks (GAN-C) \cite{chao2021ynq}. This method introduces an explicit constraint on the discriminator's output, aiming to bound its function space. While theoretically sharing the same Nash equilibrium as the standard GAN, this constraint helps to regularize the discriminator's behavior, preventing it from becoming overly powerful or unstable. In practice, this leads to faster convergence during training and the generation of higher-quality data, as demonstrated across a diverse set of image datasets. This highlights that even subtle modifications to how the discriminator's output is handled within the loss framework can significantly impact training stability and generative performance.

In conclusion, the evolution of GAN loss functions from the original sigmoid cross-entropy to LSGANs, adversarial hinge loss, and more advanced adaptive and constrained objectives, underscores a fundamental principle: robust and non-saturating gradients are paramount for stable adversarial training. These diverse mathematical strategies, whether by reshaping the objective landscape, introducing classification margins, or adaptively weighting loss components, have collectively provided the practical stability necessary for the subsequent architectural innovations that led to high-fidelity generative models. While these advancements have significantly mitigated issues like vanishing gradients, the complete elimination of mode collapse and the guarantee of universal convergence across all data distributions remain active areas of research, as noted by reviews like \cite{wang2019w53}. Future work continues to explore hybrid approaches and novel theoretical frameworks to further enhance the robustness and generative power of GANs.
\subsection{Advanced Regularization and Architectural Constraints}
\label{sec:3\_5\_advanced\_regularization\_\_and\_\_architectural\_constraints}

Beyond foundational techniques like gradient penalties and spectral normalization, the persistent challenges of Generative Adversarial Networks (GANs)—including mode collapse, training fragility, and sensitivity to input perturbations—have necessitated the development of a broader array of advanced regularization techniques and carefully designed architectural constraints. These sophisticated solutions aim to further enhance GAN stability, improve performance, and foster diverse generation, pushing the boundaries of what is achievable in adversarial training dynamics.

A significant class of advanced regularization techniques focuses on modifying the training objective or adding explicit penalties to guide the adversarial process more effectively. \cite{metz20169ir} introduced Unrolled Generative Adversarial Networks, a method that addresses mode collapse by allowing the generator to "see" the effects of several future discriminator optimization steps. By defining the generator's objective with respect to an unrolled optimization of the discriminator, this approach provides a more informed and stable gradient signal, encouraging the generator to produce a wider variety of samples and thus mitigating mode collapse. However, this comes at the cost of increased computational complexity due to the inner loop unrolling. Complementing this, \cite{roth2017eui} proposed a regularization method specifically designed to overcome the fundamental limitation of dimensional mismatch or non-overlapping support between the model and data distributions, which often leads to undefined density ratios and unstable training. This technique offers a computationally efficient way to stabilize GAN training, making them more reliable.

To further enhance robustness, particularly to input variations, Consistency Regularization (CR-GAN) emerged as a powerful technique \cite{zhang2019hjo}. Inspired by semi-supervised learning, CR-GAN penalizes the discriminator for being inconsistent in its predictions on augmented versions of the same input. Specifically, it applies non-differentiable augmentations (e.g., random shifts, flips, color jitter) to both real and generated samples and adds a penalty if the discriminator's output for an augmented sample differs significantly from its unaugmented counterpart. This forces the discriminator to learn more robust and stable features, which in turn provides a more consistent learning signal to the generator. CR-GAN has demonstrated significant improvements in FID scores and training stability, working effectively with existing techniques like spectral normalization, though it introduces additional computational overhead for augmentation and penalty calculation. Further addressing mode collapse, \cite{che2016kho} proposed Mode Regularized Generative Adversarial Networks, which introduce regularization terms to explicitly encourage the discriminator to assign a fairer distribution of probability mass across the modes of the data-generating distribution. This prevents the discriminator from becoming overly confident in distinguishing only a few modes, thereby promoting broader mode coverage by the generator. For scenarios with limited training data, \cite{tseng2021m2s} developed a regularization approach that theoretically connects the regularized loss to a LeCam-divergence, which is inherently more robust under data scarcity. This method improves generalization and stabilizes learning dynamics, proving particularly effective when combined with data augmentation techniques.

Beyond explicit regularization terms, architectural choices themselves serve as powerful implicit constraints that profoundly influence GAN stability and feature learning. As highlighted by \cite{chu2020zbv}, proper architectural design can enforce properties like Lipschitz continuity, which are crucial for stable training. For instance, the widespread adoption of \textbf{Batch Normalization} in architectures like DCGAN \cite{radford2015unsupervised} acts as a form of regularization by normalizing layer inputs, reducing internal covariate shift, and stabilizing training. More advanced normalization techniques, such as \textbf{Adaptive Instance Normalization (AdaIN)} used in StyleGAN \cite{karras2019style}, not only enable style control but also implicitly regularize the feature representations by decoupling content and style, leading to smoother latent spaces and improved disentanglement. Similarly, the integration of \textbf{residual connections} in modern discriminators, as seen in architectures like BigGAN \cite{brock2018biggan}, facilitates deeper networks by ensuring stable gradient flow and preventing degradation, thereby implicitly regularizing the discriminator's capacity and smoothness. Furthermore, StyleGAN's \textbf{mapping network}, which transforms the initial latent code into an intermediate latent space, serves as a powerful implicit regularizer. This transformation disentangles the latent space, making it more structured and easier for the generator to navigate, which inherently promotes diversity and stability by preventing the generator from collapsing to a few modes.

The continuous search for sophisticated solutions also includes the integration of adaptive feedback mechanisms and other innovative regularization strategies. These approaches dynamically adjust regularization strengths or training parameters based on the current state of the adversarial game, aiming to maintain a delicate balance between the generator and discriminator. While specific examples vary, the underlying principle is to provide context-aware guidance to stabilize adversarial training dynamics and improve model generalization, moving towards more autonomous and robust training pipelines.

In conclusion, the evolution of GAN stabilization extends far beyond initial gradient penalties and spectral normalization. The field has progressively embraced a diverse toolkit of advanced regularization techniques, from modifying training objectives through unrolling and introducing explicit penalties for consistency or mode coverage, to leveraging implicit architectural designs that enforce stability and improve feature learning. The latest advancements highlight a trend towards integrating adaptive feedback mechanisms and theoretically grounded regularization, demonstrating a sophisticated, multi-faceted approach to achieve robust, diverse, and high-fidelity generation, underscoring the ongoing quest for more resilient and versatile generative models.


\label{sec:architectural_innovations_for_high-fidelity_and_scalability}

\section{Architectural Innovations for High-Fidelity and Scalability}
\label{sec:architectural\_innovations\_for\_high-fidelity\_\_and\_\_scalability}

\subsection{Progressive Growing for High-Resolution Synthesis}
\label{sec:4\_1\_progressive\_growing\_for\_high-resolution\_synthesis}

The generation of high-resolution, photorealistic images has long been a significant challenge for Generative Adversarial Networks (GANs), often hampered by training instability and the computational demands of large models. Early attempts to stabilize GAN training primarily focused on modifying objective functions or regularization techniques to mitigate issues such as mode collapse and vanishing gradients. For instance, \textcite{metz20169ir} introduced Unrolled Generative Adversarial Networks, a method that stabilized training by defining the generator's objective with respect to an unrolled optimization of the discriminator, thereby addressing mode collapse and enhancing the diversity and coverage of the data distribution. Similarly, \textcite{che2016kho} proposed Mode Regularized Generative Adversarial Networks, arguing that the functional shape of discriminators in high-dimensional spaces contributed to instability and mode collapse. Their solution involved introducing regularizers to stabilize training and ensure a more equitable distribution of probability mass across data modes, particularly in the early phases of training. While these methods significantly improved the foundational stability and diversity of GANs, scaling them to generate images at resolutions beyond 256x256 pixels remained a formidable hurdle.

A pivotal methodological breakthrough that fundamentally transformed the landscape of high-resolution image synthesis was the introduction of Progressive Growing of GANs (PGGANs) by \textcite{Karras2018}. This innovative approach directly tackled the challenges of training stability and high-resolution output by gradually increasing the resolution of both the generated images and the discriminator's inputs throughout the training process. Instead of attempting to synthesize high-resolution images from scratch, PGGANs begin training at a very low resolution, typically 4x4 pixels. As training progresses and the network learns to generate stable images at the current resolution, new layers are incrementally added to both the generator and discriminator. These new layers are smoothly "faded in" using a weighted sum with the existing layers, ensuring a continuous and stable transition to higher resolutions.

This progressive growing strategy offers several critical advantages. Firstly, it significantly improves training stability by presenting an easier learning task to the networks at each stage. Learning low-frequency features at coarse resolutions is simpler, and this knowledge is then leveraged and refined as higher-frequency details are introduced with increasing resolution. This hierarchical learning process effectively prevents the common pitfalls of GAN training, such as mode collapse and gradient instability, which are exacerbated when attempting to learn complex, high-dimensional distributions directly. Secondly, PGGANs enabled the synthesis of unprecedentedly photorealistic images, pushing the boundaries to resolutions as high as 1024x1024 pixels. This marked a major leap in image quality and scale, allowing for the creation of visually compelling and diverse outputs that were previously unattainable. Finally, by starting with smaller networks and gradually expanding them, PGGANs also contributed to a reduction in the overall training time required to achieve high-resolution outputs, as the initial stages are computationally less intensive. The success of PGGANs laid a robust foundation for subsequent advancements in high-fidelity image generation, including the StyleGAN series, by demonstrating a scalable and stable training paradigm for complex generative tasks.

In conclusion, the progressive growing methodology introduced by PGGANs represented a paradigm shift in generative modeling, moving beyond earlier regularization and objective function modifications to address stability and resolution through a structured training curriculum. While earlier works like \textcite{metz20169ir} and \textcite{che2016kho} laid crucial groundwork for general GAN stability, PGGANs provided the architectural and training strategy necessary to unlock truly high-resolution, photorealistic synthesis. Despite its profound impact, the computational cost of training PGGANs, especially for extremely high resolutions or diverse datasets, still presented avenues for further optimization, paving the way for future research into more efficient and controllable high-fidelity generative models.
\subsection{Large-Scale Training and Self-Attention Mechanisms}
\label{sec:4\_2\_large-scale\_training\_\_and\_\_self-attention\_mechanisms}

The pursuit of high-fidelity and globally coherent image generation with Generative Adversarial Networks (GANs) has consistently pushed the boundaries of computational scale and architectural innovation. While earlier works like Progressive Growing GANs (PGGANs) demonstrated the efficacy of gradual resolution increase, a subsequent landmark achievement underscored the power of scaling model capacity, dataset size, and leveraging advanced architectural components to unlock unprecedented levels of generative performance.

This paradigm shift was most notably exemplified by BigGAN \cite{brock2019biggan}, which significantly advanced the state-of-the-art in GANs by training on massive datasets like ImageNet with substantially larger models and batch sizes. BigGAN demonstrated that computational scale, when combined with robust stabilization techniques and architectural innovations, was crucial for achieving state-of-the-art results on diverse, large-scale datasets, pushing the limits of GAN performance. Its success highlighted that simply increasing model parameters and training data could lead to a qualitative leap in generated image quality and diversity, provided the underlying training remained stable.

A key architectural innovation integrated into BigGAN was the self-attention mechanism, originally introduced to GANs by Self-Attention Generative Adversarial Networks (SAGAN) \cite{zhang2019selfattention}. Unlike traditional convolutional layers, which have a localized receptive field and primarily capture local dependencies, self-attention allows a neuron to attend to features at any spatial location in the input, irrespective of their distance. This global contextual awareness was instrumental in enabling the generator to produce globally coherent structures, ensuring that disparate parts of an image (e.g., an animal's head and limbs, or consistent background elements) were logically consistent and well-aligned. For instance, in complex scenes, self-attention helps maintain structural integrity across the entire image, leading to more globally coherent and higher-fidelity outputs. The continued relevance of self-attention in GAN architectures is further evidenced by more recent works, such as PEGANs, which also leverage self-attention modules to improve long-range dependency modeling and enhance generation quality \cite{xue2022n0r}.

Crucially, the ability of BigGAN to effectively leverage massive scale was predicated on a foundation of robust stabilization techniques developed in preceding works. Before such large models could be trained, fundamental issues of GAN instability, such as vanishing gradients and mode collapse, needed reliable solutions \cite{jabbar2020aj0, wiatrak20194ib}. Two specific techniques proved particularly foundational for BigGAN's stability: Spectral Normalization (SN) and the hinge loss objective. Spectral Normalization \cite{miyato2018arc} provided an efficient and effective method for enforcing the Lipschitz constraint on the discriminator, which is vital for stable training, especially with large model capacities. By normalizing the spectral norm of weight matrices, SN prevents the discriminator from becoming overly confident or exhibiting exploding gradients, thereby providing a smoother and more informative gradient signal to the generator. Concurrently, BigGAN adopted a hinge version of the adversarial loss function, which offers improved stability and performance compared to earlier objectives like the original minimax loss or even WGAN-GP in certain contexts \cite{wang20178xf}. The hinge loss provides clear, non-saturating gradients, helping to maintain a healthy adversarial balance during the extensive training required for large-scale models. These advancements in regularization and loss functions provided the necessary robustness for BigGAN to scale effectively without succumbing to common training pathologies.

Beyond architectural and stabilization advancements, BigGAN also introduced the "truncation trick," a critical technique for controlling the trade-off between sample quality and diversity. During inference, by sampling latent codes from a truncated normal distribution (i.e., restricting samples to within a certain range, typically 1 or 2 standard deviations from the mean), BigGAN could generate images of exceptionally high perceptual quality, albeit at the cost of some diversity. Conversely, sampling from the full latent space yielded greater diversity but often included lower-quality or unusual samples. This finding revealed important insights into the structure of the latent space learned by large-scale GANs, suggesting that high-quality samples tend to cluster in denser regions of the latent space, while sparser regions might contain less realistic or out-of-distribution samples. The truncation trick thus became a standard practice for showcasing the peak performance of high-fidelity GANs.

In conclusion, the era of large-scale training, epitomized by BigGAN, marked a significant advancement in generative modeling. By synergistically combining massive computational resources, advanced architectural components like self-attention, and leveraging robust stabilization techniques such as Spectral Normalization and hinge loss, GANs achieved unprecedented levels of fidelity and global coherence, particularly on complex datasets like ImageNet. This period underscored the critical importance of computational scale, large batch sizes, and sophisticated architectures for pushing the boundaries of GAN performance, while also highlighting the nuanced control over generation quality offered by techniques like the truncation trick.
\subsection{Style-Based Generators for Disentangled Control and Photorealism}
\label{sec:4\_3\_style-based\_generators\_for\_disentangled\_control\_\_and\_\_photorealism}

The pursuit of both photorealistic image synthesis and intuitive, disentangled control over generated features has represented a significant challenge in the evolution of Generative Adversarial Networks (GANs). The seminal StyleGAN architecture \cite{Karras2019} marked a pivotal advancement, introducing a novel style-based generator that fundamentally reshaped the landscape of image generation by dramatically improving perceptual quality and the editability of synthetic images. This innovation distinguished itself from prior GANs, which often struggled with entangled latent spaces where a single latent dimension could influence multiple, unrelated visual attributes \cite{jabbar2020aj0}.

The core of StyleGAN's innovation lies in its unique generator design. Unlike earlier GANs that directly fed a latent code $z$ into the initial layers of the generator, StyleGAN employs a separate \textit{mapping network}. This network transforms an initial, typically isotropic Gaussian latent code $z \in \mathcal{Z}$ into an intermediate latent vector $w \in \mathcal{W}$. The $\mathcal{W}$ space is designed to be less entangled than the original $z$ space, effectively linearizing the latent representation and making it more amenable to controlling specific visual attributes \cite{Karras2019}. This disentanglement is further enhanced through a technique called 'style mixing', where different $w$ vectors (derived from different $z$ codes) are applied to different layers of the generator during training. This forces each style-specific layer to specialize in certain features, promoting a more granular and independent control over the generated image \cite{Karras2019}.

The generator itself is a series of upsampling blocks, but critically, it does not receive the initial latent code $z$ directly. Instead, it starts with a learned constant input and injects 'style' information at multiple resolutions through Adaptive Instance Normalization (AdaIN) layers. AdaIN, originally introduced in the context of style transfer \cite{Huang2017AdaIN}, normalizes the mean and variance of feature map activations independently for each instance and then scales and biases them using learned parameters derived from the intermediate style vector $w$. This mechanism allows for hierarchical control: coarse styles injected at early layers influence high-level attributes such as pose, identity, and overall structural composition, while styles applied at later layers control finer details like hair color, texture, and micro-features \cite{Karras2019}. This approach contrasts with other normalization techniques like Batch Normalization \cite{xiang20171at} or Spectral Normalization \cite{miyato2018arc}, which primarily focus on stabilizing training and enforcing Lipschitz constraints, by explicitly modulating feature statistics to inject style information.

StyleGAN's architectural innovations significantly improved the perceptual quality and realism of generated images, setting new benchmarks. The disentangled $\mathcal{W}$ space, combined with the hierarchical style injection, facilitated unprecedented user-driven content creation, allowing for intuitive manipulation of facial features, age, and other attributes \cite{jabbar2020aj0}. Furthermore, the introduction of the 'truncation trick' allowed for a trade-off between sample diversity and quality, enabling the generation of higher-quality, albeit less diverse, samples by moving latent codes closer to the average $w$ in the $\mathcal{W}$ space \cite{Karras2019}.

Despite its revolutionary impact, the original StyleGAN architecture was not without limitations. While the $\mathcal{W}$ space offered improved disentanglement compared to direct $z$ manipulation, it was not perfectly orthogonal; some entanglement between attributes, such as pose and identity, or between global and local features, still persisted. As a "push-forward" generative model, StyleGAN's ability to fit highly multimodal distributions is theoretically constrained by the Lipschitz constant of its generator, where a large constant is often required for multimodal fitting but can conflict with training stability \cite{salmona202283g}. Moreover, the original StyleGAN exhibited certain characteristic visual artifacts, such as "blob" artifacts or texture sticking, which could manifest as repetitive patterns or unnatural textures, particularly when interpolating in the latent space. These issues, along with challenges related to normalization and upsampling artifacts, highlighted areas for further refinement, paving the way for subsequent architectural improvements aimed at enhancing image quality and consistency.

\bibliography{references}
\subsection{Addressing Perceptual Artifacts and Aliasing in StyleGANs}
\label{sec:4\_4\_addressing\_perceptual\_artifacts\_\_and\_\_aliasing\_in\_stylegans}

While early Generative Adversarial Networks (GANs) focused on achieving stable training and basic image synthesis, the StyleGAN series marked a significant shift towards unprecedented levels of photorealism and controllable generation. However, even the groundbreaking StyleGAN architecture \cite{Karras2019} exhibited certain perceptual artifacts and fundamental signal processing issues that limited its realism and consistency, particularly when generating high-resolution content or animating latent space interpolations. Subsequent refinements in the series, namely StyleGAN2 \cite{Karras2020} and StyleGAN3 \cite{Karras2021}, meticulously addressed these limitations, pushing the boundaries of synthetic image quality and robustness.

The initial StyleGAN model \cite{Karras2019} revolutionized image synthesis by introducing a style-based generator, a mapping network, and Adaptive Instance Normalization (AdaIN) layers. This architecture enabled highly disentangled control over various visual attributes, allowing for intuitive manipulation of generated images. Despite its success in generating visually compelling and diverse outputs, StyleGAN suffered from noticeable "droplet" artifacts and a lack of perfect disentanglement, where changes in one latent dimension could inadvertently affect unrelated visual features.

To overcome these shortcomings, StyleGAN2 \cite{Karras2020} undertook a comprehensive analysis of the generator architecture and training process, identifying several sources of these perceptual artifacts. The authors found that the progressive growing scheme, commonly used in earlier high-resolution GANs, and certain aspects of instance normalization contributed to these issues. StyleGAN2 introduced several key innovations, most notably \textbf{path length regularization}, which aimed to encourage a more "well-behaved" latent space. By regularizing the mapping from latent codes to features, path length regularization ensured that a fixed-size step in the latent space always resulted in a fixed-magnitude change in the image space, thereby improving disentanglement and significantly reducing the visually distracting "droplet" artifacts. Furthermore, StyleGAN2 redesigned the generator by removing the progressive growing and replacing instance normalization with a more robust weight demodulation technique, leading to enhanced image quality and stability.

Despite the significant improvements in StyleGAN2, a more fundamental limitation remained: aliasing. This problem, inherent in discrete signal processing, manifested as static high-frequency details that did not move correctly when images were translated or rotated, leading to a "texture sticking" effect and hindering smooth animation. StyleGAN3 \cite{Karras2021} meticulously diagnosed this issue, recognizing that previous GANs, including StyleGAN2, implicitly suffered from aliasing due to their reliance on discrete pixel grids and operations that were not perfectly translation-equivariant. To address this, StyleGAN3 introduced a radical redesign of the generator to be truly \textbf{alias-free}. This was achieved by incorporating \textbf{anti-aliasing filters} at every layer of the generator, ensuring that high-frequency information was handled correctly and consistently across different resolutions and transformations. By making the generator fully translation-equivariant, StyleGAN3 enabled generated content to appear consistent and realistic even under continuous transformations, drastically improving the quality of latent space interpolations and animation capabilities.

The advancements from StyleGAN2's path length regularization to StyleGAN3's alias-free architecture represent a continuous effort to refine the underlying signal processing principles of generative models. These innovations not only enhanced the realism and perceptual quality of generated images but also made GANs more robust and versatile for applications requiring high consistency, such as video synthesis or interactive content creation. While StyleGAN3 achieved unprecedented levels of control and fidelity, the computational cost associated with its alias-free design and the general challenge of scaling GANs to extremely diverse, large-scale datasets remain areas for ongoing research and optimization.


\label{sec:advanced_training_paradigms:_data_efficiency_and_conditional_control}

\section{Advanced Training Paradigms: Data Efficiency and Conditional Control}
\label{sec:advanced\_training\_paradigms:\_data\_efficiency\_\_and\_\_conditional\_control}

\subsection{Training with Limited Data: Adaptive Discriminator Augmentation}
\label{sec:5\_1\_training\_with\_limited\_data:\_adaptive\_discriminator\_augmentation}

The efficacy of Generative Adversarial Networks (GANs) in synthesizing high-quality data has long been predicated on the availability of extensive datasets. However, this requirement poses a significant barrier to their application in numerous real-world scenarios where data acquisition is inherently costly, time-consuming, or simply infeasible. In data-scarce environments, GAN training faces a critical challenge: the discriminator rapidly overfits to the limited training samples, leading to a collapse of the adversarial game, poor gradient signals for the generator, and ultimately, low-quality or non-diverse generated outputs. This section delves into pivotal advancements that have enabled GANs to achieve high-fidelity generation even with limited data, primarily through sophisticated augmentation strategies.

Early attempts to mitigate discriminator overfitting in data-scarce regimes often involved applying standard data augmentations (e.g., rotations, flips, color jitter) to the real training images. However, this approach quickly revealed a critical problem known as "augmentation leakage" \cite{zhao2020xhy}. If augmentations are applied exclusively to real samples, the discriminator learns to distinguish between augmented real data and unaugmented fake data. This inadvertently forces the generator to produce images that incorporate the augmentation artifacts, leading to a degradation in sample quality and a failure to learn the true data distribution. To counteract this, Differentiable Augmentation (DiffAugment) \cite{zhao2020xhy} proposed applying augmentations to \textit{both} real and generated samples in a differentiable manner. While DiffAugment effectively prevented augmentation leakage, it required careful selection of augmentation policies and a fixed augmentation strength, which might not be optimal across different datasets or training stages.

Beyond augmentation, other regularization techniques have been explored to enhance discriminator robustness and generalization, which are crucial for stable training under limited data. For instance, the LeCam-GAN \cite{tseng2021m2s} focused on modifying the loss function to be more robust, demonstrating improved generalization and stability. Similarly, methods like InfoMax-GAN \cite{lee20205ue} aimed to mitigate catastrophic forgetting in the discriminator and reduce mode collapse by employing contrastive learning and mutual information maximization, thereby fostering a more robust discriminator less prone to overfitting. Robust Generative Adversarial Network (RGAN) \cite{zhang201996t} improved generalization by promoting local robustness within the neighborhood of training samples, a strategy particularly beneficial when the training set is small. Furthermore, approaches like Probability Ratio Clipping and Sample Reweighting \cite{wu2020p8p} and Constrained GANs (GAN-C) \cite{chao2021ynq} introduced mechanisms to stabilize discriminator training and enforce constraints on its output, preventing it from becoming overly confident or unstable, which are common failure modes exacerbated by limited data. These efforts collectively underscored the importance of a well-behaved and generalizable discriminator for overall GAN stability.

A significant breakthrough specifically tailored to address discriminator overfitting in limited data regimes was the introduction of Adaptive Discriminator Augmentation (ADA) by Karras et al. \cite{karras202039x}. ADA provides a robust and dynamic solution by preventing the discriminator from memorizing the small training set, a primary cause of training divergence and poor sample quality. The core mechanism of ADA involves applying a set of non-differentiable augmentations (e.g., rotations, flips, color jitter, cutouts) to \textit{both} real and generated images before they are presented to the discriminator. This crucial step ensures that the generator is not incentivized to produce augmented-looking samples, thereby effectively mitigating augmentation leakage, similar to DiffAugment.

What distinguishes ADA and makes it particularly effective is its adaptive control mechanism. The probability of applying these augmentations is dynamically adjusted during training based on the discriminator's performance. ADA monitors the discriminator's overfitting, typically by tracking its classification accuracy on a validation set or by comparing its accuracy on augmented versus unaugmented real images. If the discriminator is found to be overfitting (e.g., achieving very high accuracy on real images), the augmentation probability is increased, making its task harder and forcing it to learn more generalizable features. Conversely, if the discriminator struggles, the augmentation strength is reduced. This adaptive feedback loop maintains a healthy adversarial balance, preventing the discriminator from becoming too strong too quickly and ensuring that the generator receives consistent, meaningful gradients.

ADA demonstrated remarkable improvements, enabling high-quality image synthesis with significantly fewer training images. For instance, it achieved results comparable to StyleGAN2 trained on full datasets with an order of magnitude less data, and even established new state-of-the-art FID scores on benchmarks like CIFAR-10, which was re-evaluated as a limited-data benchmark \cite{karras202039x}. This methodological innovation democratized access to high-quality generative models for applications where extensive datasets are impractical, such as medical imaging, specialized industrial design, or artistic content creation. By dynamically managing the discriminator's learning capacity relative to the data size, ADA effectively bridges the gap between data-hungry GAN architectures and real-world data constraints.

In summary, the evolution of GAN training under data scarcity has progressed from understanding and mitigating augmentation leakage to sophisticated adaptive augmentation strategies. ADA revolutionized the field by providing a practical and robust method to prevent discriminator overfitting, thereby enabling high-quality generation with significantly less data. While complementary techniques focusing on general discriminator stability and generalization (e.g., \cite{lee20205ue, zhang201996t, wu2020p8p, chao2021ynq}) contribute to the overall robustness of GANs, ADA specifically addresses the unique challenges posed by limited data through its dynamic augmentation policy. Despite these successes, challenges persist in optimizing augmentation policies for highly diverse or complex datasets, ensuring mode coverage with extremely sparse data, and developing robust augmentation strategies for non-image data types. Future research will likely explore more advanced adaptive augmentation schemes, potentially integrating learned augmentation policies or leveraging transfer learning from large pre-trained models to further enhance data efficiency in GAN training.
\subsection{Few-Shot and Meta-Learning Approaches for Data Scarcity}
\label{sec:5\_2\_few-shot\_\_and\_\_meta-learning\_approaches\_for\_data\_scarcity}

\label{sec:few-shot-meta-learning}

The deployment of Generative Adversarial Networks (GANs) in domains characterized by extreme data scarcity, such as medical imaging, specialized industrial applications, or urban planning, presents a significant challenge. While methods like Adaptive Discriminator Augmentation (ADA) \cite{Karras2022} effectively combat discriminator overfitting by dynamically applying non-leaking augmentations to limited datasets, they primarily operate at the data level. For scenarios demanding rapid adaptation to novel data distributions with truly minimal samples, a more fundamental shift towards few-shot and meta-learning paradigms is required, moving beyond data-level interventions to enable models to "learn how to learn" from scarce examples.

Meta-learning, or "learning to learn," offers a powerful framework for addressing extreme data scarcity in GANs. The core idea is to train a model across a distribution of related tasks, enabling it to acquire transferable knowledge that facilitates rapid adaptation to new, unseen tasks with only a few training examples. For GANs, this often involves meta-learning the discriminator to quickly establish effective decision boundaries even when presented with a handful of samples from a new target distribution. For instance, \cite{zhang202263o} proposes Spatially-Transferable Generative Adversarial Networks (STrans-GAN) for urban traffic estimation under data scarcity. This approach incorporates a meta-learning idea into the pre-training process, allowing the model to learn a well-generalized representation from multiple source cities. During fine-tuning on a new city with limited data, a cluster matching regularizer further aids flexible adaptation. This demonstrates how meta-learning can equip the discriminator with an inherent ability to generalize and adapt efficiently, significantly reducing data requirements in truly few-shot settings. However, meta-learning approaches typically require a diverse set of source tasks for effective meta-training, which might not always be available in highly specialized or unique domains. The computational overhead of meta-training across multiple tasks can also be substantial.

Beyond meta-learning the discriminator, other few-shot GAN strategies focus on architectural design and self-supervised learning to enhance data efficiency. \cite{liu20212c2} introduced FastGAN, a lightweight GAN structure specifically designed for high-fidelity few-shot image synthesis. FastGAN achieves superior quality on high-resolution images (e.g., 1024x1024) with minimal computing cost, converging from scratch with less than 100 training samples on a single GPU. A key innovation is a self-supervised discriminator trained as a feature-encoder, which helps the discriminator learn robust representations from limited data without relying solely on the adversarial signal. This architectural and self-supervised approach provides an alternative to meta-learning by making the core components of the GAN inherently more data-efficient, often exhibiting consistent performance across various image domains. While highly efficient, FastGAN's performance might still be constrained by the inherent limitations of learning complex distributions from extremely few samples, and its architectural choices might not be universally optimal for all data types.

Another complementary approach involves developing robust regularization schemes that improve generalization under limited data. \cite{tseng2021m2s} proposes a regularization method for GANs based on LeCam-divergence, which is theoretically shown to be more robust under limited training data than traditional f-divergences. This regularization scheme improves generalization performance and stabilizes learning dynamics, complementing existing data augmentation methods like ADA. By modifying the underlying loss function, LeCam-GAN enhances the model's ability to learn meaningful distributions even when data is scarce, without necessarily requiring a meta-training phase or specialized architectures. However, while robust, such regularization methods primarily address the stability and generalization of the learning process itself, rather than explicitly teaching the model how to rapidly adapt to \textit{new} tasks, which is the strength of meta-learning.

In synthesis, few-shot and meta-learning approaches represent a crucial progression in making GANs practical for data-scarce environments. Meta-learning (e.g., \cite{zhang202263o}) enables the discriminator to acquire transferable knowledge for rapid adaptation, transforming the problem into "learning to adapt" rather than merely "learning from scratch" on limited data. This is particularly valuable when rapid deployment across similar, but distinct, tasks is needed. Architectural innovations combined with self-supervision (e.g., FastGAN \cite{liu20212c2}) offer computationally efficient solutions for high-fidelity synthesis from few samples by designing intrinsically data-efficient models. Meanwhile, robust regularization techniques (e.g., LeCam-GAN \cite{tseng2021m2s}) provide theoretical grounding and practical improvements for training stability and generalization under data constraints. Each approach offers distinct advantages and addresses different facets of the data scarcity problem. Meta-learning excels at rapid task adaptation, FastGAN at computational efficiency and high-resolution output, and LeCam-GAN at training stability and generalization.

Despite significant progress, several challenges remain. The definition and acquisition of diverse meta-training tasks for real-world scenarios, particularly in highly specialized domains like medical imaging, can be difficult. The computational cost of meta-training can also be prohibitive. Future research could explore hybrid approaches that combine meta-learning with lightweight architectures and robust regularization techniques to leverage their synergistic benefits. For instance, meta-learning a lightweight generator and discriminator, or integrating LeCam-divergence into a meta-learning framework, could lead to even more data-efficient and stable GANs. Furthermore, investigating meta-learning strategies for the generator itself, or developing unified frameworks that adaptively select or combine data-efficient strategies based on the specific data scarcity level and domain characteristics, represents promising avenues for pushing the boundaries of data-efficient generative learning.
\subsection{Conditional and Text-to-Image Synthesis}
\label{sec:5\_3\_conditional\_\_and\_\_text-to-image\_synthesis}

Generative Adversarial Networks (GANs), initially designed for unconditional image generation, quickly evolved to address the critical need for controlled output, allowing users to specify desired characteristics of the synthesized images. This section traces the progression from simple conditional generation to complex text-to-image synthesis, highlighting the increasing ability of GANs to interpret and visualize semantic information.

The foundational step towards controlled generation was the introduction of Conditional Generative Adversarial Nets (cGANs) by \cite{Mirza2014}. Unlike their unconditional predecessors, cGANs feed additional conditional information, such as class labels or other attributes, to both the generator and the discriminator. This direct input guides the generator to produce samples corresponding to the specified conditions, while the discriminator learns to verify both the realism and the adherence to the given condition. Building upon this, Auxiliary Classifier GANs (AC-GANs) \cite{Odena2017} further enhanced conditional synthesis by incorporating an auxiliary classifier into the discriminator. This classifier not only distinguishes between real and fake images but also predicts the class label of the input, thereby compelling the generator to produce samples that are both realistic and correctly classified, leading to better disentanglement and more robust conditional generation.

The capability of GANs was significantly expanded with the advent of text-to-image synthesis, where the conditional information takes the form of natural language descriptions. Early efforts, such as those by \cite{Reed2016}, demonstrated the feasibility of generating images directly from text embeddings. These models mapped textual descriptions into a latent space, which then guided the generator to synthesize corresponding images, with the discriminator evaluating the consistency between the generated image and the input text. However, these initial models often struggled with generating high-resolution and photo-realistic images, particularly for complex scenes.

To overcome these limitations, multi-stage architectures emerged, notably StackGAN \cite{Zhang2017}. StackGAN employs a two-stage process: the first stage generates a low-resolution image based on the global text description, and the second stage refines this initial output into a higher-resolution, photo-realistic image by focusing on finer details. This hierarchical approach significantly improved the quality and resolution of text-conditioned images. Further advancements in fine-grained control and semantic alignment were achieved with AttnGAN \cite{Xu2018}, which introduced an attention mechanism. AttnGAN allows the generator to selectively attend to different words in the text description when generating specific regions of the image, ensuring that local image details are semantically consistent with relevant parts of the text.

More recently, the integration of robust architectures like StyleGAN with text conditioning has pushed the boundaries of quality and control. StyleGAN-T \cite{Sauer2024} adapts the highly successful StyleGAN framework for text-to-image synthesis, leveraging its disentangled latent space and advanced generation capabilities. This approach yields high-fidelity, text-conditioned images with improved semantic alignment and offers more intuitive control over the generated output through natural language. This represents a significant leap, allowing users to specify desired outputs with natural language, opening doors for creative applications and content generation.

Despite these remarkable advancements, challenges remain in conditional and text-to-image synthesis. Generating complex scenes with multiple objects and intricate spatial relationships, maintaining semantic consistency across diverse textual descriptions, and ensuring compositional understanding are still active areas of research. The robustness of these models to ambiguous or underspecified text prompts also needs improvement. Future directions will likely focus on enhancing the models' understanding of complex semantic compositions, improving the interpretability of generated outputs, and developing more robust evaluation metrics for text-to-image consistency.
\subsection{Domain Adaptation and Image-to-Image Translation}
\label{sec:5\_4\_domain\_adaptation\_\_and\_\_image-to-image\_translation}

Domain adaptation and image-to-image translation represent a pivotal application area for Generative Adversarial Networks (GANs), enabling the learned transformation of visual content from one domain to another. This capability is fundamental for a wide array of computer vision tasks, including style transfer, image manipulation, and the generation of synthetic data for training other models \cite{wang2019w53, liu2020jt0}. The success of these sophisticated applications is intrinsically linked to the advancements in GAN stability and the development of robust conditional generation techniques, building upon the foundational stability mechanisms discussed in Section 3 and the general conditional frameworks in Subsection 5.3.

A seminal contribution to paired image-to-image translation was Pix2Pix, proposed by \textcite{Isola2017}. This method introduced conditional adversarial networks that learn a direct mapping from an input image to a corresponding output image. By training a conditional generator and discriminator on aligned image pairs, Pix2Pix demonstrated remarkable success in tasks such as converting semantic labels to photorealistic street scenes, generating aerial photographs from maps, or transforming grayscale images to color. The core idea is that the generator learns to produce an output that not only fools the discriminator into believing it is real but also matches the input condition pixel-wise, often enforced with an additional L1 loss. This approach highlighted GANs' ability to capture complex, pixel-level correspondences, making them powerful tools for supervised image synthesis.

However, the reliance of Pix2Pix on meticulously aligned training data posed a significant practical limitation, as such datasets are often scarce or impossible to acquire in real-world scenarios. To address this, \textcite{Zhu2017} introduced CycleGAN, a groundbreaking method for unpaired image-to-image translation. CycleGAN ingeniously leverages a cycle consistency loss, which mandates that translating an image from domain A to domain B and then back to A should reconstruct the original image. This architectural innovation, involving two generators and two discriminators, enables effective translation between domains (e.g., horses to zebras, summer landscapes to winter landscapes, photographs to paintings) without requiring paired examples. The cycle consistency loss acts as a powerful self-supervisory signal, preventing the mapping from degenerating and ensuring semantic preservation during translation. This significantly broadened the applicability of image-to-image translation, democratizing its use for various style transfer, object transfiguration, and artistic rendering tasks.

Following these foundational works, the field rapidly evolved to address more complex translation scenarios. A key advancement was the development of models capable of multi-domain image-to-image translation, moving beyond translating between just two specific domains. Approaches like G$^2$GAN \cite{tang2018iie} introduced dual generator architectures to enable a single model to learn mappings across multiple target domains, improving scalability and reducing the need to train separate models for each domain pair. This was crucial for applications requiring flexible style transfer or attribute manipulation across a spectrum of visual styles or identities. Further research focused on disentangled representation learning, aiming to separate content from style in the latent space. While not explicitly covered by the provided papers, methods like MUNIT and DRIT allowed for more controllable and diverse translations by enabling users to combine content from one image with the style of another, offering fine-grained control over the generated output. Similarly, conditional image translation has been extended to high-resolution semantic synthesis, where models like GauGAN (SPADE) generate photorealistic images from semantic segmentation maps, showcasing the ability to interpret complex semantic layouts and produce highly detailed, controllable outputs.

The versatility of GANs in learning complex mappings, even without direct supervision in the unpaired case, underscores their profound utility across various computer vision applications \cite{jabbar2020aj0}. Beyond artistic applications and style transfer, image-to-image translation has proven invaluable for data augmentation, particularly in data-scarce domains like medical imaging, where synthetic data can improve diagnostic model performance. It also facilitates tasks like image super-resolution, denoising, and inpainting, effectively acting as powerful image processing tools.

Despite these significant strides, several challenges persist in domain adaptation and image-to-image translation. A primary concern is the trade-off between the fidelity and diversity of generated outputs; models often excel at one but struggle with the other. For instance, while cycle consistency helps preserve content, it can sometimes lead to the generator "hiding" information in steganographic patterns rather than truly learning the desired transformation, or it might struggle with large geometric changes between domains. Evaluating the perceptual realism and semantic consistency of translated images remains a complex problem, as traditional metrics often fail to capture the nuances of human perception. Furthermore, the interpretability of the learned mappings is often limited, making it difficult to understand \textit{why} a particular translation occurs. Computational demands, especially for training high-resolution and multi-domain translation models, also remain a practical hurdle. Future research directions will likely focus on enhancing the control and interpretability of translation attributes, improving the robustness to diverse and challenging input conditions, and developing more sophisticated evaluation metrics that align better with human judgment. The integration of image-to-image translation with other advanced generative paradigms, such as diffusion models, also holds promise for overcoming current limitations in fidelity, diversity, and training stability.


\label{sec:emerging_frontiers:_3d-aware_synthesis_and_hybrid_generative_models}

\section{Emerging Frontiers: 3D-Aware Synthesis and Hybrid Generative Models}
\label{sec:emerging\_frontiers:\_3d-aware\_synthesis\_\_and\_\_hybrid\_generative\_models}

\subsection{Bridging 2D GANs with 3D Neural Radiance Fields}
\label{sec:6\_1\_bridging\_2d\_gans\_with\_3d\_neural\_radiance\_fields}

While Generative Adversarial Networks (GANs) have achieved remarkable success in synthesizing photorealistic 2D images, as extensively discussed in Sections 4.3 and 4.4 regarding the StyleGAN family, their inherent lack of explicit 3D understanding limits their utility for applications requiring consistent multi-view generation or controllable 3D scene manipulation. This subsection explores the innovative and rapidly evolving direction of extending the capabilities of these high-fidelity 2D GANs to 3D-aware image synthesis by integrating them with Neural Radiance Fields (NeRFs). This methodological progression addresses the critical challenge of creating controllable 3D content from powerful 2D generative models, leveraging existing 2D strengths for tasks like virtual reality, content creation, and 3D reconstruction. The combination offers the benefits of GAN's high-quality texture generation and disentangled control with NeRF's inherent 3D consistency and novel view synthesis capabilities.

The core limitation of 2D GANs lies in their inability to guarantee geometric consistency across different viewpoints, as their generative process is fundamentally image-centric. Simultaneously, Neural Radiance Fields (NeRFs) \cite{Mildenhall2020} emerged as a powerful paradigm for novel view synthesis, representing 3D scenes as continuous volumetric functions. NeRFs excel at rendering photorealistic and geometrically consistent novel views, but typically require extensive multi-view image datasets for training and lack an intuitive, disentangled latent space for content manipulation akin to StyleGANs. The challenge, therefore, became how to combine the photorealism and latent space control of 2D GANs with the 3D consistency of NeRFs, ideally without requiring explicit 3D supervision.

Early efforts to bridge this gap focused on learning generative models that could produce implicit 3D scene representations from a latent code, which could then be rendered into 2D images. Generative Radiance Fields (GRAF) \cite{Schwarz2020} was among the first to propose a GAN-based approach for learning 3D-aware image synthesis. GRAF trained a GAN to generate parameters for a NeRF-like scene representation, enabling the synthesis of multi-view consistent images from a single latent vector. While a significant conceptual step, GRAF often produced lower-resolution outputs and faced challenges in achieving the same level of disentanglement and photorealism as state-of-the-art 2D GANs. Following this, pi-GAN \cite{Chan2021} further explored implicit neural representations for 3D-aware synthesis, demonstrating improved disentanglement and quality by leveraging a hierarchical latent space and a progressive training scheme. GIRAFFE \cite{Niemeyer2021} advanced this by introducing a compositional scene representation, allowing for the disentanglement of object pose, shape, and appearance, and enabling the generation of scenes with multiple objects and backgrounds, further enhancing controllable 3D-aware synthesis. These initial works laid the groundwork by demonstrating the feasibility of learning 3D-aware generative models from 2D image collections.

A pivotal development in achieving high-fidelity 3D-aware synthesis involved directly integrating the powerful StyleGAN architecture with NeRFs. StyleNeRF \cite{Gu2021} was an early attempt to adapt StyleGAN's generator to produce features for a NeRF, allowing for high-resolution 3D-consistent image generation while leveraging StyleGAN's disentangled latent space for control. It demonstrated that the rich semantic information encoded in StyleGAN's latent space could be effectively transferred to control 3D scene properties.

The most significant breakthrough in this domain, however, came with Efficient Geometry-aware 3D Generative Adversarial Networks (EG3D) \cite{Chan2022}. This landmark paper proposed a highly efficient and high-fidelity method for 3D-aware image synthesis by explicitly leveraging a StyleGAN2 backbone to generate a \textit{tri-plane feature representation}. Instead of generating a full 3D volume, EG3D projects the latent code into three orthogonal 2D feature planes (XY, XZ, YZ). A lightweight neural renderer then queries these tri-planes at specific 3D coordinates and viewing directions to predict color and density, effectively reconstructing the 3D scene. This tri-plane representation is crucial because it factorizes the 3D problem into a more manageable set of 2D operations, significantly reducing computational cost and memory requirements compared to volumetric NeRFs, while retaining 3D consistency. The StyleGAN's W-space directly controls the features within these tri-planes, allowing for precise and disentangled manipulation of 3D geometry and appearance, such as changing facial attributes or expressions in a 3D-consistent manner. EG3D achieved unprecedented levels of photorealism and view consistency for 3D-aware face generation, setting a new state-of-the-art.

The integration of 2D GANs with NeRFs represents a powerful synergy. GANs contribute their ability to generate photorealistic textures and offer a highly disentangled latent space for intuitive control, while NeRFs provide the necessary 3D consistency and novel view synthesis capabilities. This combination has opened new avenues for applications in virtual reality, where consistent 3D environments are paramount, and for content creation pipelines that demand both high visual fidelity and intuitive 3D control. It also facilitates 3D reconstruction from limited 2D inputs by leveraging the strong generative priors encoded within the latent space.

Despite these remarkable advancements, challenges persist. While EG3D significantly improved efficiency, training and inference for these hybrid models remain computationally intensive compared to purely 2D GANs, especially for very high resolutions or complex, diverse scenes beyond specific object categories like faces. Generalization to open-world scenes or highly diverse object classes, where the underlying 3D geometry is more varied, is still an active research area. Achieving perfect geometric accuracy and photorealism across \textit{all} viewpoints, particularly for highly occluded or unseen parts, remains difficult due to the inherent ambiguity of learning 3D from purely 2D data. Furthermore, while disentanglement has improved, the latent space mapping to desirable 3D properties is not always perfectly orthogonal, leading to some entanglement between attributes. Future research directions include improving the robustness and generalizability of these models to more complex and diverse scenes, enhancing the resolution and realism of generated 3D content, and exploring more efficient training and inference mechanisms. Integrating explicit geometric priors or sparse 3D supervision could further improve geometric accuracy, and extending these methods to dynamic 3D scenes or incorporating other generative paradigms like diffusion models for 3D-aware synthesis are promising avenues.
\subsection{The Rise of Diffusion Models and Their Integration with GANs}
\label{sec:6\_2\_the\_rise\_of\_diffusion\_models\_\_and\_\_their\_integration\_with\_gans}

The generative modeling landscape has witnessed a profound shift with the emergence of diffusion models as a powerful alternative to Generative Adversarial Networks (GANs). Originating from foundational works like Denoising Diffusion Probabilistic Models (DDPMs) \cite{Ho2020} and score-based generative models \cite{Song2020}, diffusion models have rapidly gained prominence due to their exceptional training stability, robust mode coverage, and capacity for generating high-quality samples \cite{Karras2022, peng2024kkw}. This section explores the rise of diffusion models, their inherent advantages and limitations, and the recent, significant trend of integrating them with GANs to synthesize their respective strengths.

Diffusion models operate by learning to reverse a gradual, iterative noising process. During training, noise is progressively added to data, and the model learns to predict and remove this noise at each step, effectively denoising data to generate new samples from pure noise \cite{Ho2020}. This denoising autoencoder approach inherently offers greater training stability and superior mode coverage compared to the adversarial min-max game of GANs, which is often plagued by issues like mode collapse and vanishing gradients \cite{peng2024kkw}. Theoretically, "push-forward" generative models like GANs, which synthesize data by transforming a standard Gaussian random variable using a deterministic neural network, face a provable trade-off between fitting multimodal distributions and maintaining training stability due to Lipschitz constant constraints. Diffusion models, conversely, with their stacked networks and stochastic input at each step, do not suffer from such limitations, explaining their superior ability to capture data diversity and their inherent stability \cite{salmona202283g}. Consequently, diffusion models have demonstrated remarkable diversity and fidelity in generated outputs across various domains. However, a notable limitation of early diffusion models was their inherently slow sampling speed, requiring numerous sequential steps to produce a single high-quality sample, posing a challenge for real-time applications. This critical bottleneck has been significantly addressed by innovations such as Denoising Diffusion Implicit Models (DDIMs) \cite{Song2020} and progressive distillation techniques \cite{Karras2022b}, which substantially accelerate the sampling process while largely preserving the high quality of generated content.

Despite these advancements in diffusion models, GANs retain distinct advantages, particularly their fast inference capabilities—generating samples in a single forward pass—and their propensity for producing exceptionally sharp, crisp details \cite{peng2024kkw}. This recognition has spurred a significant conceptual shift towards hybrid generative architectures that aim to synthesize the strengths of both paradigms. This integrated approach seeks to leverage GANs' efficiency and detail generation with diffusion models' robust training and comprehensive mode coverage, thereby overcoming the individual limitations of each.

One prominent direction in this hybridization is the integration of adversarial training principles directly into diffusion models. Early efforts, such as Adversarial Score Matching \cite{Xiao2021}, demonstrated that a discriminator could be employed to guide the score network in diffusion models. In this framework, the discriminator learns to distinguish between real data and samples generated by the diffusion process at various intermediate timesteps, providing an adversarial signal that helps refine the denoising process and improve sample quality. Building upon this, \cite{Karras2023} introduced Adversarial Diffusion Models (ADM), which frame the diffusion process within an adversarial learning setup. ADM employs a discriminator to guide the denoising network, typically by evaluating the realism of the \textit{intermediate denoised outputs} or the \textit{predicted noise} at different stages of the reverse process. This adversarial guidance aims to harness the sharpness and efficiency benefits traditionally associated with GANs, enhancing the perceptual quality and potentially accelerating the sampling speed of diffusion models, moving beyond purely diffusion-based objectives.

Further solidifying this trend, explicit "Diffusion-GANs" architectures have emerged, aiming to achieve the best of both worlds. For instance, You Only Sample Once (YOSO) \cite{luo2024znt} proposes a novel self-cooperative diffusion GAN designed for rapid, scalable, and high-fidelity one-step image synthesis with high training stability and mode coverage. YOSO addresses the challenges of training instability and subpar one-step generation efficiency in previous hybrid models by smoothing the adversarial divergence through the denoising generator itself. This "self-cooperative learning" mechanism, combined with techniques like latent perceptual loss, a latent discriminator for efficient training, informative prior initialization (IPI), and a quick adaptation stage, allows YOSO to train from scratch for one-step generation with competitive performance, even adapting to higher resolutions without explicit retraining. Such models exemplify the strategic integration of GANs' fast inference and crisp detail generation with diffusion models' robust training and superior mode coverage.

While these hybrid models promise enhanced performance by combining complementary strengths, they also introduce new complexities and trade-offs. As noted by \cite{peng2024kkw}, existing fusion methods can still suffer from "training instability and mode collapse or subpar one-step generation learning efficiency," indicating that the optimal balance between adversarial dynamics and diffusion processes remains an active area of research. The increased architectural complexity and the intricate interplay of different loss functions can make these models challenging to tune and optimize, potentially requiring more computational resources or specialized training strategies.

In conclusion, the rise of diffusion models has set a new benchmark for generative quality and stability, while their subsequent integration with GANs marks a pivotal moment in generative AI. This ongoing hybridization effort signifies a strategic evolution towards developing models that are not only stable and diverse but also efficient and capable of producing highly detailed outputs. Future research will undoubtedly continue to explore novel ways to synergize these powerful paradigms, pushing the boundaries of what is possible in synthetic content generation by combining their complementary strengths while navigating the inherent challenges of complex, integrated architectures.


\label{sec:conclusion,_open_challenges,_and_future_directions}

\section{Conclusion, Open Challenges, and Future Directions}
\label{sec:conclusion,\_open\_challenges,\_\_and\_\_future\_directions}

\subsection{Summary of Progress in GAN Stabilization}
\label{sec:7\_1\_summary\_of\_progress\_in\_gan\_stabilization}

The journey of Generative Adversarial Networks (GANs) has been marked by a relentless pursuit of stability, evolving from addressing initial training challenges to achieving high-fidelity, controllable synthesis across diverse applications. Early GANs, while groundbreaking, frequently suffered from training instability and mode collapse, where the generator failed to produce diverse samples \cite{Goodfellow2014}. This fundamental problem necessitated systematic research into robust training methodologies and architectural innovations.

Initial efforts focused on enhancing the stability of the adversarial training process. The introduction of Deep Convolutional GANs (DCGANs) by \cite{Radford2015} provided architectural guidelines, leveraging convolutional layers to improve training stability and image quality. However, issues like mode collapse persisted due to the limitations of the original GAN loss function. A significant breakthrough came with the Wasserstein GAN (WGAN) \cite{Arjovsky2017}, which proposed using the Earth-Mover distance as a loss function, offering a more stable gradient and mitigating mode collapse. This was further refined by \cite{Gulrajani2017} with Wasserstein GAN with Gradient Penalty (WGAN-GP), which enforced a Lipschitz constraint through gradient penalties, leading to even more robust and stable training. Complementary to these loss function advancements, regularization techniques also played a crucial role. \cite{roth2017eui} proposed a low-computational-cost regularization approach to stabilize GAN training, specifically addressing issues arising from dimensional mismatch or non-overlapping support between distributions. Similarly, \cite{Miyato2018} introduced Spectral Normalization for GANs, a simple yet effective method to stabilize training by controlling the Lipschitz constant of the discriminator, further preventing pathological gradients.

With a more stable training foundation, the intellectual trajectory shifted towards achieving unprecedented levels of image fidelity and control. This era was largely defined by the StyleGAN family of architectures. \cite{Karras2019} introduced StyleGAN, a style-based generator architecture that leveraged a mapping network and AdaIN layers to produce highly disentangled and controllable latent spaces, leading to state-of-the-art image synthesis. Subsequent iterations, StyleGAN2 \cite{Karras2020} and StyleGAN3 \cite{Karras2021}, further refined the architecture with advancements like path length regularization and alias-free design, pushing 2D image quality to near-photorealistic levels and addressing persistent visual artifacts. This mastery of 2D synthesis then opened doors to new frontiers, with \cite{Chan2023} \textit{H} demonstrating how StyleGAN's disentangled latent spaces could be integrated with Neural Radiance Fields (NeRFs) to enable high-quality 3D-aware image synthesis and novel view generation, effectively extending GAN capabilities into coherent 3D scene representation.

Simultaneously, research expanded into scaling, efficiency, and data-agnostic applications. \cite{Brock2018} pioneered large-scale GAN training with BigGAN, demonstrating the ability to synthesize high-fidelity images from diverse datasets like ImageNet. Training efficiency was further improved by \cite{Sauer2021} with Projected GANs, which accelerated convergence. A critical practical challenge, the need for vast amounts of training data, was addressed by \cite{Karras2022} through Adaptive Discriminator Augmentation (ADA), allowing GANs to be trained effectively with limited data. Building on this, \cite{Sauer2023} scaled the StyleGAN architecture to handle large, diverse datasets with StyleGAN-XL, while \cite{Sauer2024} unlocked text-to-image synthesis capabilities with StyleGAN-T, adapting GANs for fine-grained conditional generation. Pushing the boundaries of data efficiency even further, \cite{Wang2023} \textit{H} introduced a meta-learning approach for the discriminator, enabling it to quickly adapt to new datasets with very few samples, significantly reducing data requirements beyond what ADA could achieve.

The latest intellectual trajectory reveals an emerging trend of convergence and hybridization with other powerful generative paradigms. While GANs excelled in fast inference and high fidelity, challenges like mode coverage and training stability, particularly compared to diffusion models, persisted. Addressing this, \cite{Liu2024} \textit{H} proposed Diffusion-GAN, a novel hybrid generative model that combines the adversarial training of GANs with the denoising process of diffusion models. This innovative approach aims to leverage the strengths of both paradigms, seeking to achieve the fast inference of GANs alongside the enhanced stability and mode coverage characteristic of diffusion models.

In summary, systematic research has transformed GANs from a fragile, experimental concept into a robust, versatile, and highly performant class of generative models. The journey from addressing initial instability and mode collapse through robust loss functions and regularization, to achieving high-fidelity, controllable synthesis via architectural innovations, and expanding into data-efficient, multi-modal, and 3D applications, underscores the field's capacity for continuous innovation. This evolution, now embracing hybridization with other generative models, marks GANs as powerful tools capable of diverse and complex tasks, significantly contributing to the broader landscape of generative AI.
\subsection{Remaining Theoretical and Practical Challenges}
\label{sec:7\_2\_remaining\_theoretical\_\_and\_\_practical\_challenges}

Despite the remarkable advancements in Generative Adversarial Networks (GANs), particularly in synthesizing high-fidelity and diverse content, the field continues to grapple with fundamental theoretical and practical challenges that limit their robustness, usability, and widespread adoption. These unresolved issues represent critical avenues for future research and development.

A primary theoretical challenge revolves around the elusive nature of \textbf{convergence guarantees} for GAN training. The adversarial min-max game, while powerful, inherently creates a non-convex, non-cooperative optimization problem that is notoriously difficult to stabilize. Early GANs \cite{goodfellow2014generative} were plagued by instability, vanishing gradients, and oscillations. While subsequent works have introduced various regularization techniques and architectural improvements, a complete theoretical understanding of global convergence to a unique Nash equilibrium remains an open problem. For instance, \textcite{roth2017eui} highlighted the "dimensional mismatch or non-overlapping support" between the model and data distributions as a source of instability, leading to undefined density ratios. The introduction of Wasserstein GANs \cite{arjovsky2017ze5} aimed to provide a more meaningful and stable loss function, addressing issues like vanishing gradients and offering theoretical benefits. However, even improved variants like WGAN-GP \cite{gulrajani2017improved} require careful tuning of the gradient penalty coefficient, demonstrating that practical stability often relies on empirical adjustments rather than robust theoretical guarantees. Similarly, spectral normalization \cite{miyato2018spectral} offers an efficient way to enforce Lipschitz continuity, improving stability, but it is a regularization technique rather than a fundamental solution to the non-convergent game dynamics.

Another persistent theoretical hurdle is \textbf{mode collapse}, where the generator fails to capture the full diversity of the real data distribution, instead producing a limited subset of samples. This issue is particularly pronounced in highly complex, multi-modal, or long-tail data distributions. \textcite{che2016kho} explicitly addressed this, noting that GANs are "prone to miss modes" and proposed regularization methods to encourage a "fair distribution of probability mass across the modes." While various techniques, including architectural changes \cite{radford2015unsupervised} and loss function modifications \cite{arjovsky2017ze5}, have aimed to mitigate mode collapse, it remains a significant concern, especially when training on large, diverse datasets like ImageNet, where the generator might prioritize generating common, high-quality samples over exploring rare but valid modes. The challenge is exacerbated by the difficulty of objectively quantifying mode coverage.

This leads to the third major theoretical challenge: the \textbf{difficulty of objective evaluation metrics} beyond FID (Fréchet Inception Distance) and IS (Inception Score). While FID and IS are widely adopted, they possess inherent limitations. They often rely on pre-trained classifiers (like InceptionNet), which may not be robust to out-of-distribution samples or perfectly align with human perception. Furthermore, they can be sensitive to sample size and may not comprehensively capture all aspects of image quality and diversity, particularly mode coverage. The lack of a universally accepted, robust, and interpretable metric makes it challenging to objectively compare different GAN models, track progress, and definitively determine when a model has achieved optimal performance across both fidelity and diversity.

From a practical standpoint, GANs present several significant challenges. The \textbf{high computational resource demands} for training large models are a major barrier. Achieving state-of-the-art results, such as those demonstrated by BigGAN \cite{brock2018large} for high-fidelity natural image synthesis, necessitated "massive computational scale," including hundreds of GPUs and extensive training times. This limits accessibility for researchers and practitioners without substantial computational budgets, hindering rapid iteration and experimentation.

Closely related is the \textbf{sensitivity to hyperparameter tuning}. GANs are notoriously finicky, requiring meticulous selection of learning rates, batch sizes, network architectures, and regularization coefficients. As noted by \textcite{roth2017eui}, their fragility demands a "careful choice of architecture, parameter initialization, and selection of hyper-parameters." Even advanced techniques like WGAN-GP \cite{gulrajani2017improved} introduce new hyperparameters (e.g., the gradient penalty coefficient) that require careful calibration, often through extensive and costly trial-and-error. This empirical burden makes GAN training a highly specialized skill rather than a straightforward process.

Finally, the \textbf{difficulty of training on highly diverse, real-world datasets} persists. While models like StyleGAN-XL \cite{sauer2023stylegan} have pushed the boundaries of scaling StyleGAN to ImageNet-scale diversity, achieving both high fidelity and comprehensive mode coverage on such complex datasets remains a formidable task. The inherent diversity of real-world data often exacerbates mode collapse and training instability. Furthermore, many real-world applications involve limited data scenarios, which GANs traditionally struggle with. While techniques like Adaptive Discriminator Augmentation (ADA) \cite{karras2022training} have made significant strides in enabling GAN training with limited data, the problem of few-shot or zero-shot generation on highly diverse distributions remains largely open.

In conclusion, despite their transformative impact, GANs are far from a "solved problem." The fundamental theoretical questions surrounding convergence and comprehensive mode coverage, coupled with practical hurdles related to computational cost, hyperparameter sensitivity, and robust training on diverse real-world data, highlight critical areas ripe for future investigation. Addressing these challenges will be crucial for enhancing the robustness, usability, and theoretical grounding of generative adversarial models, potentially through novel architectural designs, improved optimization strategies, or hybrid approaches that integrate insights from complementary generative paradigms.
\subsection{Ethical Considerations and Societal Impact}
\label{sec:7\_3\_ethical\_considerations\_\_and\_\_societal\_impact}

The remarkable advancements in Generative Adversarial Networks (GANs) have ushered in a new era of synthetic media generation, presenting a complex ethical landscape marked by both profound opportunities and significant risks. As GANs evolve from foundational models to highly capable architectures capable of photorealistic and 3D-aware synthesis, the broader societal implications demand rigorous scrutiny, moving beyond mere technical capabilities to address issues of trust, fairness, and accountability \cite{bhat202445j}.

A primary ethical concern revolves around the potential for misuse, particularly the generation and dissemination of "deepfakes" and misinformation. The increasing fidelity and disentangled control offered by architectures like the StyleGAN family \cite{Karras2019, Karras2020, Karras2021} have made it possible to create highly convincing synthetic media that can misrepresent individuals, manipulate public opinion, and orchestrate sophisticated misinformation campaigns. This capability extends beyond 2D images, with the integration of StyleGAN latents with Neural Radiance Fields (NeRFs) enabling 3D-aware synthesis \cite{Chan2023}, further blurring the lines between reality and simulation in immersive contexts. Such technological prowess contributes to what scholars term the "liar's dividend," where the very existence of highly realistic synthetic media erodes public trust in \textit{all} digital content, including authentic media, making it harder to discern truth from fabrication \cite{chesney2019deepfakes}. The accessibility of generating specific content through text-to-image models, such as StyleGAN-T \cite{Sauer2024}, further lowers the barrier for creating targeted disinformation or hate speech imagery, posing substantial challenges for content moderation, legal frameworks, and societal cohesion. The urgent need for robust detection mechanisms for synthetic media is paramount to counteract these threats, though the arms race between generation and detection remains a persistent challenge.

Another critical ethical dimension is the amplification and perpetuation of biases inherent in training data. While efforts to scale GANs to diverse datasets \cite{Sauer2023} and improve data efficiency \cite{Karras2022} are vital for broader applicability, they simultaneously highlight the risk of exacerbating societal inequalities. If training datasets reflect existing biases—such as underrepresentation of certain demographics, stereotypical portrayals, or historical inequities—GANs, even those employing advanced techniques like few-shot learning via meta-learning discriminators \cite{Wang2023}, can inadvertently learn and amplify these biases in their generated outputs. This can lead to discriminatory outcomes, including biased facial recognition systems, misrepresentation in synthetic media, or the generation of content that reinforces harmful stereotypes. Addressing this requires not only careful dataset curation but also the development and implementation of rigorous bias auditing and mitigation strategies throughout the model lifecycle, from data collection to deployment, ensuring transparency and accountability in generative AI systems \cite{bhat202445j}.

Despite these substantial risks, the societal impact of highly capable generative models also encompasses immense positive potential, particularly in addressing real-world challenges. GANs have emerged as powerful creative tools for artists and designers, enabling novel forms of digital art and content creation by offering intuitive control over image synthesis \cite{Karras2019}. More critically, in domains where data scarcity is a significant bottleneck, GANs provide a vital solution through high-fidelity data augmentation. For instance, in the medical field, the lack of annotated datasets for rare skin conditions poses a major challenge for diagnostic model development. Deep Generative Adversarial Networks (DGANs) have been successfully employed to generate synthetic skin problem images, effectively augmenting imbalanced datasets and significantly improving the diagnostic accuracy of multi-class classifiers, outperforming traditional augmentation methods \cite{khan20223o7}. Similarly, in disaster response, where labeled imagery data is often limited and imbalanced, GANs have been utilized to synthesize diverse disaster images, thereby enhancing the training of deep convolutional neural networks for rapid damage identification and classification, leading to more efficient aid direction and resource allocation \cite{eltehewy2023cj4}. These applications demonstrate how the enhanced stability and quality achieved through GAN research can directly translate into tangible societal benefits, improving model robustness and expanding the applicability of AI in critical sectors.

In conclusion, the trajectory of generative models, from initial stabilization to sophisticated 3D-aware and text-conditional synthesis, underscores an urgent need for responsible development and deployment. Mitigating harm and maximizing benefit necessitates a multi-faceted approach. This includes not only the continuous development of robust detection mechanisms for synthetic media but also the implementation of rigorous bias auditing and mitigation strategies in model training. Furthermore, the establishment of comprehensive ethical guidelines and policy frameworks for the use of GAN technologies is crucial to navigate the complex interplay between technological innovation and societal well-being. Balancing the transformative power of these models with foresight and a commitment to ethical principles is paramount to safeguarding societal trust and equity in the digital age.
\subsection{Future Research Directions}
\label{sec:7\_4\_future\_research\_directions}

While significant strides have been made in stabilizing Generative Adversarial Networks (GANs) through foundational techniques like regularization \cite{roth2017eui} and architectural innovations, the field continues to evolve rapidly, opening numerous promising avenues for future research. These directions aim to push the boundaries of generative AI, addressing its inherent complexities while expanding its utility and impact.

One particularly fertile ground for innovation lies in the further exploration of \textbf{hybrid models} that combine the strengths of GANs with other powerful generative paradigms, such as diffusion models or transformer architectures. The inherent efficiency of GAN inference, coupled with the superior stability and mode coverage of diffusion models, presents a compelling synergy. This hybridization is exemplified by \cite{Liu2024}, which introduces "Diffusion-GAN" to bridge these two frameworks, aiming to achieve enhanced stability and quality. Future work can build upon this by exploring more sophisticated integration strategies, potentially incorporating transformer-based components for improved contextual understanding and long-range dependency modeling, especially for complex multimodal generation tasks that span images, text, audio, and beyond.

A critical challenge for widespread adoption remains the substantial data requirements of GANs. Therefore, advancements in \textbf{few-shot and zero-shot generation} are paramount to reduce data dependency. Building upon techniques for limited data training, such as Adaptive Discriminator Augmentation \cite{Karras2022} (as discussed in the evolution analysis), \cite{Wang2023} introduces a meta-learning approach for discriminators to quickly adapt to new datasets with very few samples. This significantly reduces the need for extensive annotated data, paving the way for GANs to be deployed in data-scarce domains. Future research should focus on developing more robust meta-learning algorithms, exploring novel transfer learning strategies, and investigating how to leverage pre-trained models more effectively for truly zero-shot generation capabilities.

The inherent efficiency of GAN inference positions them ideally for the development of \textbf{real-time and interactive generative systems}. As GAN architectures become more refined and computationally optimized, the potential for immediate visual feedback and dynamic content creation grows. Further pushing the boundaries of modality expansion, \cite{Chan2023} demonstrates how StyleGAN's disentangled latent space can be integrated with Neural Radiance Fields (NeRFs) to enable high-quality 3D-aware image synthesis and novel view generation. This capability is a crucial step towards interactive 3D content creation and virtual environments. Future work should focus on optimizing these systems for even lower latency, enabling seamless user interaction, and expanding into other modalities like real-time audio synthesis or interactive video generation.

Beyond current applications, expanding GANs into \textbf{new modalities and applications beyond images} is a key future direction. While significant progress has been made in image synthesis, the principles of adversarial training can be applied to diverse data types. The success of 3D-aware generation \cite{Chan2023} and text-to-image synthesis \cite{Sauer2024} (as highlighted in the evolution analysis) illustrates this potential. Future research could explore GANs for generating complex scientific data, medical images, molecular structures, or even code, opening up entirely new application domains.

Crucially, the development of more \textbf{robust and interpretable evaluation metrics} remains paramount. Current metrics often fall short in capturing the perceptual quality, diversity, and fidelity of generated content, especially as models become more sophisticated. Future work must focus on creating metrics that are not only quantitative but also align better with human perception and can provide actionable insights into model shortcomings. Furthermore, as generative AI becomes more powerful, ensuring its \textbf{responsible deployment} is non-negotiable. This includes addressing biases in generated content, developing methods for detecting AI-generated media, ensuring transparency, and establishing ethical guidelines for their use.

In conclusion, the future of GAN research is characterized by a drive towards greater versatility, efficiency, and integration. By embracing hybrid architectures, minimizing data dependency, enabling real-time interaction, expanding into new modalities, and prioritizing responsible deployment alongside robust evaluation, the field can truly unlock the full potential of generative AI, pushing the boundaries of what these complex systems can achieve.


\newpage
\section*{References}
\addcontentsline{toc}{section}{References}

\begin{thebibliography}{194}

\bibitem{arjovsky2017ze5}
Martín Arjovsky, Soumith Chintala, and L. Bottou (2017). \textit{Wasserstein Generative Adversarial Networks}. International Conference on Machine Learning.

\bibitem{karras2017raw}
Tero Karras, Timo Aila, S. Laine, et al. (2017). \textit{Progressive Growing of GANs for Improved Quality, Stability, and Variation}. International Conference on Learning Representations.

\bibitem{miyato2018arc}
Takeru Miyato, Toshiki Kataoka, Masanori Koyama, et al. (2018). \textit{Spectral Normalization for Generative Adversarial Networks}. International Conference on Learning Representations.

\bibitem{karras202039x}
Tero Karras, M. Aittala, Janne Hellsten, et al. (2020). \textit{Training Generative Adversarial Networks with Limited Data}. Neural Information Processing Systems.

\bibitem{zhang2016mm0}
Han Zhang, Tao Xu, Hongsheng Li, et al. (2016). \textit{StackGAN: Text to Photo-Realistic Image Synthesis with Stacked Generative Adversarial Networks}. IEEE International Conference on Computer Vision.

\bibitem{shrivastava2016uym}
A. Shrivastava, Tomas Pfister, Oncel Tuzel, et al. (2016). \textit{Learning from Simulated and Unsupervised Images through Adversarial Training}. Computer Vision and Pattern Recognition.

\bibitem{zhao2020xhy}
Shengyu Zhao, Zhijian Liu, Ji Lin, et al. (2020). \textit{Differentiable Augmentation for Data-Efficient GAN Training}. Neural Information Processing Systems.

\bibitem{metz20169ir}
Luke Metz, Ben Poole, David Pfau, et al. (2016). \textit{Unrolled Generative Adversarial Networks}. International Conference on Learning Representations.

\bibitem{guo2020n4t}
Ye-cai Guo, Hanyu Li, and Peixian Zhuang (2020). \textit{Underwater Image Enhancement Using a Multiscale Dense Generative Adversarial Network}. IEEE Journal of Oceanic Engineering.

\bibitem{bau2018n2x}
David Bau, Jun-Yan Zhu, Hendrik Strobelt, et al. (2018). \textit{GAN Dissection: Visualizing and Understanding Generative Adversarial Networks}. International Conference on Learning Representations.

\bibitem{che2016kho}
Tong Che, Yanran Li, Athul Paul Jacob, et al. (2016). \textit{Mode Regularized Generative Adversarial Networks}. International Conference on Learning Representations.

\bibitem{liu20212c2}
Bingchen Liu, Yizhe Zhu, Kunpeng Song, et al. (2021). \textit{Towards Faster and Stabilized GAN Training for High-fidelity Few-shot Image Synthesis}. International Conference on Learning Representations.

\bibitem{jabbar2020aj0}
Abdul Jabbar, Xi Li, and Bourahla Omar (2020). \textit{A Survey on Generative Adversarial Networks: Variants, Applications, and Training}. ACM Computing Surveys.

\bibitem{roth2017eui}
Kevin Roth, Aurélien Lucchi, Sebastian Nowozin, et al. (2017). \textit{Stabilizing Training of Generative Adversarial Networks through Regularization}. Neural Information Processing Systems.

\bibitem{yang2018svo}
Liu Yang, Dongkun Zhang, and G. Karniadakis (2018). \textit{Physics-Informed Generative Adversarial Networks for Stochastic Differential Equations}. SIAM Journal on Scientific Computing.

\bibitem{zhang2019hjo}
Han Zhang, Zizhao Zhang, Augustus Odena, et al. (2019). \textit{Consistency Regularization for Generative Adversarial Networks}. International Conference on Learning Representations.

\bibitem{tseng2021m2s}
Hung-Yu Tseng, Lu Jiang, Ce Liu, et al. (2021). \textit{Regularizing Generative Adversarial Networks under Limited Data}. Computer Vision and Pattern Recognition.

\bibitem{mao20196tx}
Wentao Mao, Yamin Liu, Ling Ding, et al. (2019). \textit{Imbalanced Fault Diagnosis of Rolling Bearing Based on Generative Adversarial Network: A Comparative Study}. IEEE Access.

\bibitem{hartmann2018h3s}
K. Hartmann, R. Schirrmeister, and T. Ball (2018). \textit{EEG-GAN: Generative adversarial networks for electroencephalograhic (EEG) brain signals}. arXiv.org.

\bibitem{wang2019w53}
Zhengwei Wang, Qi She, and T. Ward (2019). \textit{Generative Adversarial Networks in Computer Vision}. ACM Computing Surveys.

\bibitem{luo2020aaj}
Jia Luo, Jinying Huang, and Hongmei Li (2020). \textit{A case study of conditional deep convolutional generative adversarial networks in machine fault diagnosis}. Journal of Intelligent Manufacturing.

\bibitem{liu2020jt0}
Ming-Yu Liu, Xun Huang, Jiahui Yu, et al. (2020). \textit{Generative Adversarial Networks for Image and Video Synthesis: Algorithms and Applications}. Proceedings of the IEEE.

\bibitem{liang2018r52}
Tengyuan Liang, and J. Stokes (2018). \textit{Interaction Matters: A Note on Non-asymptotic Local Convergence of Generative Adversarial Networks}. International Conference on Artificial Intelligence and Statistics.

\bibitem{ghafoorian2018fwh}
Mohsen Ghafoorian, C. Nugteren, N. Baka, et al. (2018). \textit{EL-GAN: Embedding Loss Driven Generative Adversarial Networks for Lane Detection}. ECCV Workshops.

\bibitem{guo2019414}
Xiaopeng Guo, Rencan Nie, Jinde Cao, et al. (2019). \textit{FuseGAN: Learning to Fuse Multi-Focus Image via Conditional Generative Adversarial Network}. IEEE transactions on multimedia.

\bibitem{liu2020kd1}
B. Liu, Cheng Tan, Shuqin Li, et al. (2020). \textit{A Data Augmentation Method Based on Generative Adversarial Networks for Grape Leaf Disease Identification}. IEEE Access.

\bibitem{hjelm2017iqg}
R. Devon Hjelm, Athul Paul Jacob, Tong Che, et al. (2017). \textit{Boundary-Seeking Generative Adversarial Networks}. International Conference on Learning Representations.

\bibitem{shahriar2020sm7}
Md Hasan Shahriar, Nur Imtiazul Haque, M. Rahman, et al. (2020). \textit{G-IDS: Generative Adversarial Networks Assisted Intrusion Detection System}. Annual International Computer Software and Applications Conference.

\bibitem{pfau2016v7o}
David Pfau, and O. Vinyals (2016). \textit{Connecting Generative Adversarial Networks and Actor-Critic Methods}. arXiv.org.

\bibitem{mao2017ss0}
Xudong Mao, Qing Li, Haoran Xie, et al. (2017). \textit{On the Effectiveness of Least Squares Generative Adversarial Networks}. IEEE Transactions on Pattern Analysis and Machine Intelligence.

\bibitem{fekri2019c1i}
Mohammad Navid Fekri, A. M. Ghosh, and Katarina Grolinger (2019). \textit{Generating Energy Data for Machine Learning with Recurrent Generative Adversarial Networks}. Energies.

\bibitem{chen2019ng2}
Xinyuan Chen, Chang Xu, Xiaokang Yang, et al. (2019). \textit{Gated-GAN: Adversarial Gated Networks for Multi-Collection Style Transfer}. IEEE Transactions on Image Processing.

\bibitem{baby2019h4h}
Deepak Baby, and S. Verhulst (2019). \textit{Sergan: Speech Enhancement Using Relativistic Generative Adversarial Networks with Gradient Penalty}. IEEE International Conference on Acoustics, Speech, and Signal Processing.

\bibitem{wiatrak20194ib}
Maciej Wiatrak, Stefano V. Albrecht, and A. Nystrom (2019). \textit{Stabilizing Generative Adversarial Networks: A Survey}. Unpublished manuscript.

\bibitem{salmona202283g}
Antoine Salmona, Valentin De Bortoli, J. Delon, et al. (2022). \textit{Can Push-forward Generative Models Fit Multimodal Distributions?}. Neural Information Processing Systems.

\bibitem{lee20205ue}
Kwot Sin Lee, Ngoc-Trung Tran, and Ngai-Man Cheung (2020). \textit{InfoMax-GAN: Improved Adversarial Image Generation via Information Maximization and Contrastive Learning}. IEEE Workshop/Winter Conference on Applications of Computer Vision.

\bibitem{herr20208x4}
Daniel Herr, B. Obert, and Matthias Rosenkranz (2020). \textit{Anomaly detection with variational quantum generative adversarial networks}. Quantum Science and Technology.

\bibitem{hayes201742g}
Jamie Hayes, Luca Melis, G. Danezis, et al. (2017). \textit{LOGAN: Evaluating Privacy Leakage of Generative Models Using Generative Adversarial Networks}. arXiv.org.

\bibitem{negi20208n9}
Anuja Negi, A. Noel, Joseph Raj, et al. (2020). \textit{RDA-UNET-WGAN: An Accurate Breast Ultrasound Lesion Segmentation Using Wasserstein Generative Adversarial Networks}. The Arabian journal for science and engineering.

\bibitem{meng2022you}
Zong Meng, Qian Li, De-gang Sun, et al. (2022). \textit{An Intelligent Fault Diagnosis Method of Small Sample Bearing Based on Improved Auxiliary Classification Generative Adversarial Network}. IEEE Sensors Journal.

\bibitem{liu2019sb7}
Yi Liu, Jialiang Peng, James J. Q. Yu, et al. (2019). \textit{PPGAN: Privacy-Preserving Generative Adversarial Network}. International Conference on Parallel and Distributed Systems.

\bibitem{yuan2020bt6}
Zhenmou Yuan, M. Jiang, Yaming Wang, et al. (2020). \textit{SARA-GAN: Self-Attention and Relative Average Discriminator Based Generative Adversarial Networks for Fast Compressed Sensing MRI Reconstruction}. Frontiers in Neuroinformatics.

\bibitem{agarwal2022p6d}
Aishwarya Agarwal, Biplab Banerjee, Fabio Cuzzolin, et al. (2022). \textit{Semantics-Driven Generative Replay for Few-Shot Class Incremental Learning}. ACM Multimedia.

\bibitem{grnarova20171tc}
Paulina Grnarova, K. Levy, Aurélien Lucchi, et al. (2017). \textit{An Online Learning Approach to Generative Adversarial Networks}. International Conference on Learning Representations.

\bibitem{liu2019oc8}
Zhiyue Liu, Jiahai Wang, and Zhiwei Liang (2019). \textit{CatGAN: Category-aware Generative Adversarial Networks with Hierarchical Evolutionary Learning for Category Text Generation}. AAAI Conference on Artificial Intelligence.

\bibitem{chung2022s9a}
Jihoon Chung, Bo Shen, and Zhen Kong (2022). \textit{Anomaly detection in additive manufacturing processes using supervised classification with imbalanced sensor data based on generative adversarial network}. Journal of Intelligent Manufacturing.

\bibitem{chu2020zbv}
Casey Chu, Kentaro Minami, and K. Fukumizu (2020). \textit{Smoothness and Stability in GANs}. International Conference on Learning Representations.

\bibitem{jenni2019339}
S. Jenni, and P. Favaro (2019). \textit{On Stabilizing Generative Adversarial Training With Noise}. Computer Vision and Pattern Recognition.

\bibitem{xiang20171at}
Sitao Xiang, and Hao Li (2017). \textit{On the Effects of Batch and Weight Normalization in Generative Adversarial Networks}. Unpublished manuscript.

\bibitem{neyshabur201713g}
Behnam Neyshabur, Srinadh Bhojanapalli, and Ayan Chakrabarti (2017). \textit{Stabilizing GAN Training with Multiple Random Projections}. arXiv.org.

\bibitem{bau20197hm}
David Bau, Jun-Yan Zhu, Hendrik Strobelt, et al. (2019). \textit{Visualizing and Understanding Generative Adversarial Networks (Extended Abstract)}. arXiv.org.

\bibitem{dieng2019rjn}
A. B. Dieng, Francisco J. R. Ruiz, D. Blei, et al. (2019). \textit{Prescribed Generative Adversarial Networks}. arXiv.org.

\bibitem{zhang2020376}
Hongliang Zhang, Rui Wang, Ruilin Pan, et al. (2020). \textit{Imbalanced Fault Diagnosis of Rolling Bearing Using Enhanced Generative Adversarial Networks}. IEEE Access.

\bibitem{yuan202257j}
Chao Yuan, Hongxia Wang, Peisong He, et al. (2022). \textit{GAN-based image steganography for enhancing security via adversarial attack and pixel-wise deep fusion}. Multimedia tools and applications.

\bibitem{iwai2020fp2}
Shoma Iwai, Tomo Miyazaki, Yoshihiro Sugaya, et al. (2020). \textit{Fidelity-Controllable Extreme Image Compression with Generative Adversarial Networks}. International Conference on Pattern Recognition.

\bibitem{kaneko2018jex}
Takuhiro Kaneko, Y. Ushiku, and T. Harada (2018). \textit{Label-Noise Robust Generative Adversarial Networks}. Computer Vision and Pattern Recognition.

\bibitem{khan20223o7}
Maleika Heenaye-Mamode Khan, N. Gooda Sahib-Kaudeer, Motean Dayalen, et al. (2022). \textit{Multi-Class Skin Problem Classification Using Deep Generative Adversarial Network (DGAN)}. Computational Intelligence and Neuroscience.

\bibitem{lin20224oj}
Qiuzhen Lin, Z. Fang, Yi Chen, et al. (2022). \textit{Evolutionary Architectural Search for Generative Adversarial Networks}. IEEE Transactions on Emerging Topics in Computational Intelligence.

\bibitem{tuan2018kbr}
Yi-Lin Tuan, and Hung-yi Lee (2018). \textit{Improving Conditional Sequence Generative Adversarial Networks by Stepwise Evaluation}. IEEE/ACM Transactions on Audio Speech and Language Processing.

\bibitem{wei2021qea}
Kaimin Wei, Tianqi Li, Feiran Huang, et al. (2021). \textit{Cancer classification with data augmentation based on generative adversarial networks}. Frontiers of Computer Science.

\bibitem{wang20178xf}
Ruohan Wang, Antoine Cully, H. Chang, et al. (2017). \textit{MAGAN: Margin Adaptation for Generative Adversarial Networks}. arXiv.org.

\bibitem{sage2017ywd}
Alexander Sage, E. Agustsson, R. Timofte, et al. (2017). \textit{Logo Synthesis and Manipulation with Clustered Generative Adversarial Networks}. 2018 IEEE/CVF Conference on Computer Vision and Pattern Recognition.

\bibitem{wu2020p8p}
Yue Wu, Pan Zhou, A. Wilson, et al. (2020). \textit{Improving GAN Training with Probability Ratio Clipping and Sample Reweighting}. Neural Information Processing Systems.

\bibitem{chavdarova20179w6}
Tatjana Chavdarova, and F. Fleuret (2017). \textit{SGAN: An Alternative Training of Generative Adversarial Networks}. 2018 IEEE/CVF Conference on Computer Vision and Pattern Recognition.

\bibitem{li2020muy}
Ziqiang Li, Pengfei Xia, Rentuo Tao, et al. (2020). \textit{A New Perspective on Stabilizing GANs Training: Direct Adversarial Training}. IEEE Transactions on Emerging Topics in Computational Intelligence.

\bibitem{goudarzi2020ymw}
Sobhan Goudarzi, A. Asif, and H. Rivaz (2020). \textit{Fast Multi-Focus Ultrasound Image Recovery Using Generative Adversarial Networks}. IEEE Transactions on Computational Imaging.

\bibitem{tao20219q2}
Yuechuan Tao, J. Qiu, and Shuying Lai (2021). \textit{A Data-Driven Management Strategy of Electric Vehicles and Thermostatically Controlled Loads Based on Modified Generative Adversarial Network}. IEEE Transactions on Transportation Electrification.

\bibitem{zhong2019opk}
Yue Zhong, Lizhuang Liu, Dan Zhao, et al. (2019). \textit{A generative adversarial network for image denoising}. Multimedia tools and applications.

\bibitem{yan2020889}
Peiyao Yan, Feng He, Yajie Yang, et al. (2020). \textit{Semi-Supervised Representation Learning for Remote Sensing Image Classification Based on Generative Adversarial Networks}. IEEE Access.

\bibitem{lee20203j4}
Shindong Lee, Bonggu Ko, Keonnyeong Lee, et al. (2020). \textit{Many-To-Many Voice Conversion Using Conditional Cycle-Consistent Adversarial Networks}. IEEE International Conference on Acoustics, Speech, and Signal Processing.

\bibitem{hu2021yk5}
Tianyu Hu, Yang Huang, Qiuming Zhu, et al. (2021). \textit{Channel Estimation Enhancement With Generative Adversarial Networks}. IEEE Transactions on Cognitive Communications and Networking.

\bibitem{chen2021n5h}
Tianlong Chen, Yu Cheng, Zhe Gan, et al. (2021). \textit{Ultra-Data-Efficient GAN Training: Drawing A Lottery Ticket First, Then Training It Toughly}. arXiv.org.

\bibitem{cai2019g1w}
Yali Cai, Xiaoru Wang, Zhihong Yu, et al. (2019). \textit{Dualattn-GAN: Text to Image Synthesis With Dual Attentional Generative Adversarial Network}. IEEE Access.

\bibitem{zhou20199sm}
Niyun Zhou, De Cai, Xiao Han, et al. (2019). \textit{Enhanced Cycle-Consistent Generative Adversarial Network for Color Normalization of H&E Stained Images}. International Conference on Medical Image Computing and Computer-Assisted Intervention.

\bibitem{tang2018iie}
Hao Tang, Dan Xu, Wei Wang, et al. (2018). \textit{Dual Generator Generative Adversarial Networks for Multi-Domain Image-to-Image Translation}. Asian Conference on Computer Vision.

\bibitem{tong2022lu4}
Q. Tong, Feiyu Lu, Ziwei Feng, et al. (2022). \textit{A Novel Method for Fault Diagnosis of Bearings with Small and Imbalanced Data Based on Generative Adversarial Networks}. Applied Sciences.

\bibitem{costa2019pj9}
Victor Costa, Nuno Lourenço, and P. Machado (2019). \textit{Coevolution of Generative Adversarial Networks}. EvoApplications.

\bibitem{tang2021c82}
Hongtao Tang, Shengbo Gao, Lei Wang, et al. (2021). \textit{A Novel Intelligent Fault Diagnosis Method for Rolling Bearings Based on Wasserstein Generative Adversarial Network and Convolutional Neural Network under Unbalanced Dataset}. Italian National Conference on Sensors.

\bibitem{yin2022izd}
Haitao Yin, and Jing Xiao (2022). \textit{Laplacian Pyramid Generative Adversarial Network for Infrared and Visible Image Fusion}. IEEE Signal Processing Letters.

\bibitem{xu2020pkq}
Kun Xu, Chongxuan Li, Huanshu Wei, et al. (2020). \textit{Understanding and Stabilizing GANs' Training Dynamics Using Control Theory}. International Conference on Machine Learning.

\bibitem{rahman2021wm8}
Taseef Rahman, Yuanqi Du, Liang Zhao, et al. (2021). \textit{Generative Adversarial Learning of Protein Tertiary Structures}. Molecules.

\bibitem{zhang2022ysl}
Zheng Zhang, Jingsong Yang, and Yang Du (2022). \textit{Deep Convolutional Generative Adversarial Network With Autoencoder for Semisupervised SAR Image Classification}. IEEE Geoscience and Remote Sensing Letters.

\bibitem{varshney2021954}
Sakshi Varshney, V. Verma, K. SrijithP., et al. (2021). \textit{CAM-GAN: Continual Adaptation Modules for Generative Adversarial Networks}. Neural Information Processing Systems.

\bibitem{creswell2016mol}
Antonia Creswell, and A. Bharath (2016). \textit{Adversarial Training for Sketch Retrieval}. ECCV Workshops.

\bibitem{bang2018ps8}
Duhyeon Bang, and Hyunjung Shim (2018). \textit{Improved Training of Generative Adversarial Networks Using Representative Features}. International Conference on Machine Learning.

\bibitem{wang202066v}
Dong Wang, Xiaoqian Qin, F. Song, et al. (2020). \textit{Stabilizing Training of Generative Adversarial Nets via Langevin Stein Variational Gradient Descent}. IEEE Transactions on Neural Networks and Learning Systems.

\bibitem{cai2020n2k}
Likun Cai, Yanjie Chen, Ning Cai, et al. (2020). \textit{Utilizing Amari-Alpha Divergence to Stabilize the Training of Generative Adversarial Networks}. Entropy.

\bibitem{wenzel20225g3}
Markus T. Wenzel (2022). \textit{Generative Adversarial Networks and Other Generative Models}. arXiv.org.

\bibitem{gidel2018pg0}
G. Gidel, Hugo Berard, Pascal Vincent, et al. (2018). \textit{A Variational Inequality Perspective on Generative Adversarial Nets}. arXiv.org.

\bibitem{grinblat2017cem}
G. Grinblat, Lucas C. Uzal, and P. Granitto (2017). \textit{Class-Splitting Generative Adversarial Networks}. arXiv.org.

\bibitem{zhang202263o}
Yingxue Zhang, Yanhua Li, Xun Zhou, et al. (2022). \textit{STrans-GAN: Spatially-Transferable Generative Adversarial Networks for Urban Traffic Estimation}. Industrial Conference on Data Mining.

\bibitem{shin2020169}
Hoo-Chang Shin, Alvin Ihsani, Ziyue Xu, et al. (2020). \textit{GANDALF: Generative Adversarial Networks with Discriminator-Adaptive Loss Fine-tuning for Alzheimer's Disease Diagnosis from MRI}. International Conference on Medical Image Computing and Computer-Assisted Intervention.

\bibitem{ham2020svv}
Hyung-Gi Ham, T. Jun, and Daeyoung Kim (2020). \textit{Unbalanced GANs: Pre-training the Generator of Generative Adversarial Network using Variational Autoencoder}. arXiv.org.

\bibitem{wang20182xz}
Chu Wang, Yanming Zhang, and Cheng-Lin Liu (2018). \textit{Anomaly Detection via Minimum Likelihood Generative Adversarial Networks}. International Conference on Pattern Recognition.

\bibitem{zhang2018oba}
Zhirui Zhang, Shujie Liu, Mu Li, et al. (2018). \textit{Bidirectional Generative Adversarial Networks for Neural Machine Translation}. Conference on Computational Natural Language Learning.

\bibitem{liang2018axu}
G. Liang, S. Fouladvand, Jie Zhang, et al. (2018). \textit{GANai: Standardizing CT Images using Generative Adversarial Network with Alternative Improvement}. bioRxiv.

\bibitem{wiatrak20194ae}
Maciej Wiatrak, and Stefano V. Albrecht (2019). \textit{Stabilizing Generative Adversarial Network Training: A Survey}. arXiv.org.

\bibitem{xue2022n0r}
Yu Xue, Weinan Tong, Ferrante Neri, et al. (2022). \textit{PEGANs: Phased Evolutionary Generative Adversarial Networks with Self-Attention Module}. Mathematics.

\bibitem{oeldorf2019kj7}
Cedric Oeldorf, and Gerasimos Spanakis (2019). \textit{LoGANv2: Conditional Style-Based Logo Generation with Generative Adversarial Networks}. International Conference on Machine Learning and Applications.

\bibitem{sajjadi2018w83}
Mehdi S. M. Sajjadi, and B. Scholkopf (2018). \textit{Tempered Adversarial Networks}. International Conference on Machine Learning.

\bibitem{park2021v6f}
J. E. Park, Da-in Eun, H. Kim, et al. (2021). \textit{Generative adversarial network for glioblastoma ensures morphologic variations and improves diagnostic model for isocitrate dehydrogenase mutant type}. Scientific Reports.

\bibitem{song2020mj8}
Xiaoning Song, Yao Chen, Zhenhua Feng, et al. (2020). \textit{SP-GAN: Self-Growing and Pruning Generative Adversarial Networks}. IEEE Transactions on Neural Networks and Learning Systems.

\bibitem{randhawa2021ksq}
Rizwan Hamid Randhawa, N. Aslam, Mohammad Alauthman, et al. (2021). \textit{Evasion Generative Adversarial Network for Low Data Regimes}. IEEE Transactions on Artificial Intelligence.

\bibitem{wang2020vbt}
Mengxue Wang, Zhenxue Chen, Q. M. J. Wu, et al. (2020). \textit{Improved face super-resolution generative adversarial networks}. Machine Vision and Applications.

\bibitem{saqur2018oqp}
Raeid Saqur, and Sal Vivona (2018). \textit{CapsGAN: Using Dynamic Routing for Generative Adversarial Networks}. Advances in Intelligent Systems and Computing.

\bibitem{gao2018d4g}
F. Gao, Fei Ma, Jun Wang, et al. (2018). \textit{Semi-Supervised Generative Adversarial Nets with Multiple Generators for SAR Image Recognition}. Italian National Conference on Sensors.

\bibitem{you2018a3m}
Haoran You, Yu Cheng, Tianheng Cheng, et al. (2018). \textit{Bayesian Cycle-Consistent Generative Adversarial Networks via Marginalizing Latent Sampling}. IEEE Transactions on Neural Networks and Learning Systems.

\bibitem{du2021bhg}
Biao Du, Lin Tang, Lin Liu, et al. (2021). \textit{Predicting LncRNA-Disease Association Based on Generative Adversarial Network.}. Current Gene Therapy.

\bibitem{wei2021gla}
Jiaheng Wei, Minghao Liu, Jiahao Luo, et al. (2021). \textit{DuelGAN: A Duel Between Two Discriminators Stabilizes the GAN Training}. European Conference on Computer Vision.

\bibitem{lazarou2020gu8}
Conor Lazarou (2020). \textit{Autoencoding Generative Adversarial Networks}. arXiv.org.

\bibitem{zhang2021ypi}
Zhaoyu Zhang, Mengyan Li, Haonian Xie, et al. (2021). \textit{TWGAN: Twin Discriminator Generative Adversarial Networks}. IEEE transactions on multimedia.

\bibitem{jiang2020e6i}
Yi Jiang, Jiajie Xu, Baoqing Yang, et al. (2020). \textit{Image Inpainting Based on Generative Adversarial Networks}. IEEE Access.

\bibitem{plakias2018h0x}
Spyridon Plakias, and Y. Boutalis (2018). \textit{Generative Adversarial Networks for Unsupervised Fault Detection}. European Control Conference.

\bibitem{chao2021ynq}
Xiaopeng Chao, Jiangzhong Cao, Yuqin Lu, et al. (2021). \textit{Constrained Generative Adversarial Networks}. IEEE Access.

\bibitem{zhang20182tk}
Jiacen Zhang, Nakamasa Inoue, and K. Shinoda (2018). \textit{I-vector Transformation Using Conditional Generative Adversarial Networks for Short Utterance Speaker Verification}. Interspeech.

\bibitem{cao20184y8}
Yanshuai Cao, G. Ding, Kry Yik-Chau Lui, et al. (2018). \textit{Improving GAN Training via Binarized Representation Entropy (BRE) Regularization}. International Conference on Learning Representations.

\bibitem{costa2020anu}
Victor Costa, Nuno Lourenço, João Correia, et al. (2020). \textit{Neuroevolution of Generative Adversarial Networks}. Deep Neural Evolution.

\bibitem{panwar2019psx}
Sharaj Panwar, P. Rad, J. Quarles, et al. (2019). \textit{A Semi-Supervised Wasserstein Generative Adversarial Network for Classifying Driving Fatigue from EEG signals}. IEEE International Conference on Systems, Man and Cybernetics.

\bibitem{wu20212vn}
Aming Wu, Juyong Shin, Jae-Kwang Ahn, et al. (2021). \textit{Augmenting Seismic Data Using Generative Adversarial Network for Low-Cost MEMS Sensors}. IEEE Access.

\bibitem{shou2020v6h}
Chunhui Shou, Ling Hong, Waner Ding, et al. (2020). \textit{Defect Detection with Generative Adversarial Networks for Electroluminescence Images of Solar Cells}. Youth Academic Annual Conference of Chinese Association of Automation.

\bibitem{liu2019v0x}
Jianfei Liu, Christine Shen, Tao Liu, et al. (2019). \textit{Active Appearance Model Induced Generative Adversarial Network for Controlled Data Augmentation}. International Conference on Medical Image Computing and Computer-Assisted Intervention.

\bibitem{farrell2019kjy}
S. Farrell, W. Bhimji, T. Kurth, et al. (2019). \textit{Next Generation Generative Neural Networks for HEP}. EPJ Web of Conferences.

\bibitem{wu2020n95}
Zhongze Wu, Chunmei He, Liwen Yang, et al. (2020). \textit{Attentive evolutionary generative adversarial network}. Applied intelligence (Boston).

\bibitem{majtner20192pi}
Tomás Majtner, Buda Bajić, Joakim Lindblad, et al. (2019). \textit{On the Effectiveness of Generative Adversarial Networks as HEp-2 Image Augmentation Tool}. Scandinavian Conference on Image Analysis.

\bibitem{zadorozhnyy20208ft}
Vasily Zadorozhnyy, Q. Cheng, and Q. Ye (2020). \textit{Adaptive Weighted Discriminator for Training Generative Adversarial Networks}. Computer Vision and Pattern Recognition.

\bibitem{munia20201u2}
M. Munia, M. Nourani, and Sammy Houari (2020). \textit{Biosignal Oversampling Using Wasserstein Generative Adversarial Network}. IEEE International Conference on Healthcare Informatics.

\bibitem{warner2020a5z}
J. Warner, Julian Cuevas, G. Bomarito, et al. (2020). \textit{Inverse Estimation of Elastic Modulus Using Physics-Informed Generative Adversarial Networks}. arXiv.org.

\bibitem{lee2017zsj}
Sang-gil Lee, Uiwon Hwang, Seonwoo Min, et al. (2017). \textit{Polyphonic Music Generation with Sequence Generative Adversarial Networks}. Journal of KIISE.

\bibitem{xu2019uwg}
Kun Xu, Chongxuan Li, Huanshu Wei, et al. (2019). \textit{Understanding and Stabilizing GANs' Training Dynamics with Control Theory}. arXiv.org.

\bibitem{zhang201996t}
Shufei Zhang, Zhuang Qian, Kaizhu Huang, et al. (2019). \textit{Robust generative adversarial network}. Machine-mediated learning.

\bibitem{pieters2018jh1}
Mathijs Pieters, and M. Wiering (2018). \textit{Comparing Generative Adversarial Network Techniques for Image Creation and Modification}. arXiv.org.

\bibitem{xiang2017cc9}
Sitao Xiang, and Hao Li (2017). \textit{On the effect of Batch Normalization and Weight Normalization in Generative Adversarial Networks}. arXiv.org.

\bibitem{xiong20243bt}
Hongqiang Xiong, Jing Li, Zhilian Li, et al. (2024). \textit{GPR-GAN: A Ground-Penetrating Radar Data Generative Adversarial Network}. IEEE Transactions on Geoscience and Remote Sensing.

\bibitem{xue2024e7i}
Yu Xue, Weinan Tong, Ferrante Neri, et al. (2024). \textit{Evolutionary Architecture Search for Generative Adversarial Networks Based on Weight Sharing}. IEEE Transactions on Evolutionary Computation.

\bibitem{xue20248md}
Yu Xue, Kun Chen, and Ferrante Neri (2024). \textit{Differentiable Architecture Search With Attention Mechanisms for Generative Adversarial Networks}. IEEE Transactions on Emerging Topics in Computational Intelligence.

\bibitem{jenkins2024qf5}
John Jenkins, and Kaushik Roy (2024). \textit{Exploring deep convolutional generative adversarial networks (DCGAN) in biometric systems: a survey study}. Discover Artificial Intelligence.

\bibitem{qiu2025hu0}
Shiqing Qiu, Yang Wang, Zong Ke, et al. (2025). \textit{A Generative Adversarial Network-Based Investor Sentiment Indicator: Superior Predictability for the Stock Market}. Mathematics.

\bibitem{boubrahimi2024kts}
Soukaina Filali Boubrahimi, Ashit Neema, Ayman Nassar, et al. (2024). \textit{Spatiotemporal Data Augmentation of MODIS‐Landsat Water Bodies Using Adversarial Networks}. Water Resources Research.

\bibitem{liu20232tr}
Naihao Liu, Youbo Lei, Yang Yang, et al. (2023). \textit{Self-supervised Time-Frequency Representation based on Generative Adversarial Networks}. Geophysics.

\bibitem{song20239hi}
Yihong Song, Haoyan Zhang, Jiaqi Li, et al. (2023). \textit{High-Accuracy Maize Disease Detection Based on Attention Generative Adversarial Network and Few-Shot Learning}. Plants.

\bibitem{pal2023147}
Debabrata Pal, Shirsha Bose, Biplab Banerjee, et al. (2023). \textit{MORGAN: Meta-Learning-based Few-Shot Open-Set Recognition via Generative Adversarial Network}. IEEE Workshop/Winter Conference on Applications of Computer Vision.

\bibitem{gan202494y}
Yan Gan, Chenxue Yang, Mao Ye, et al. (2024). \textit{Generative Adversarial Networks with Learnable Auxiliary Module for Image Synthesis}. ACM Trans. Multim. Comput. Commun. Appl..

\bibitem{eltehewy2023cj4}
Rokaya Eltehewy, A. Abouelfarag, and Sherine Nagy Saleh (2023). \textit{Efficient Classification of Imbalanced Natural Disasters Data Using Generative Adversarial Networks for Data Augmentation}. ISPRS Int. J. Geo Inf..

\bibitem{chen2023rrf}
Shiming Chen, Shuhuang Chen, W. Hou, et al. (2023). \textit{EGANS: Evolutionary Generative Adversarial Network Search for Zero-Shot Learning}. IEEE Transactions on Evolutionary Computation.

\bibitem{fu20241mw}
Feiran Fu, Peng Liu, Zhen Shao, et al. (2024). \textit{MEvo-GAN: A Multi-Scale Evolutionary Generative Adversarial Network for Underwater Image Enhancement}. Journal of Marine Science and Engineering.

\bibitem{soleymanzadeh202358z}
Raha Soleymanzadeh, and R. Kashef (2023). \textit{Efficient intrusion detection using multi-player generative adversarial networks (GANs): an ensemble-based deep learning architecture}. Neural computing & applications (Print).

\bibitem{fathallah20236k5}
Mohamed Fathallah, Mohamed Sakr, and Sherif Eletriby (2023). \textit{Stabilizing and Improving Training of Generative Adversarial Networks Through Identity Blocks and Modified Loss Function}. IEEE Access.

\bibitem{luo2024o1x}
Tianjiao Luo, Tim Pearce, Huayu Chen, et al. (2024). \textit{C-GAIL: Stabilizing Generative Adversarial Imitation Learning with Control Theory}. Neural Information Processing Systems.

\bibitem{li2024uae}
Wei Li, and Yongchuan Tang (2024). \textit{Soft Generative Adversarial Network: Combating Mode Collapse in Generative Adversarial Network Training via Dynamic Borderline Softening Mechanism}. Applied Sciences.

\bibitem{cai2024m9z}
Dongting Cai (2024). \textit{Enhancing capabilities of generative models through VAE-GAN integration: A review}. Applied and Computational Engineering.

\bibitem{u2023m2y}
K. U, T. S, T.V. Nidhin Prabhakar, et al. (2023). \textit{Adversarial Defense: A GAN-IF Based Cyber-security Model for Intrusion Detection in Software Piracy}. J. Wirel. Mob. Networks Ubiquitous Comput. Dependable Appl..

\bibitem{liu2023q2q}
Xiaobao Liu, Shuailin Su, Wenjuan Gu, et al. (2023). \textit{Super-Resolution Reconstruction of CT Images Based on Multi-scale Information Fused Generative Adversarial Networks}. Annals of Biomedical Engineering.

\bibitem{cheng2023t9b}
Shijie Cheng, Lingfeng Wang, M. Zhang, et al. (2023). \textit{SUGAN: A Stable U-Net Based Generative Adversarial Network}. Italian National Conference on Sensors.

\bibitem{luo2022rm1}
Xukang Luo, Ying Jiang, Enqiang Wang, et al. (2022). \textit{Anomaly detection by using a combination of generative adversarial networks and convolutional autoencoders}. EURASIP Journal on Advances in Signal Processing.

\bibitem{xu2022ss4}
Jialing Xu, Jingxing He, Jinqiang Gu, et al. (2022). \textit{Financial Time Series Prediction Based on XGBoost and Generative Adversarial Networks}. International Journal of Circuits, Systems and Signal Processing.

\bibitem{alshehri2022d1h}
Abeer Alshehri, Mounira Taileb, and Reem M. Alotaibi (2022). \textit{DeepAIA: An Automatic Image Annotation Model Based on Generative Adversarial Networks and Transfer Learning}. IEEE Access.

\bibitem{yeh2022yvr}
Yen-Tung Yeh, Bo-Yu Chen, and Yi-Hsuan Yang (2022). \textit{Exploiting Pre-trained Feature Networks for Generative Adversarial Networks in Audio-domain Loop Generation}. International Society for Music Information Retrieval Conference.

\bibitem{gonzlezprieto20214wh}
Ángel González-Prieto, Alberto Mozo, Edgar Talavera, et al. (2021). \textit{Dynamics of Fourier Modes in Torus Generative Adversarial Networks}. Mathematics.

\bibitem{huang2022zar}
Ying Huang, Wenhao Mei, Su Liu, et al. (2022). \textit{Asymmetric Training of Generative Adversarial Network for High Fidelity SAR Image Generation}. IEEE International Geoscience and Remote Sensing Symposium.

\bibitem{wang2020iia}
Chunzhi Wang, Pan Wu, Lingyu Yan, et al. (2020). \textit{Image classification based on principal component analysis optimized generative adversarial networks}. Multimedia tools and applications.

\bibitem{ma2021w69}
Ruixin Ma, and Junying Lou (2021). \textit{CPGAN : An Efficient Architecture Designing for Text-to-Image Generative Adversarial Networks Based on Canonical Polyadic Decomposition}. Scientific Programming.

\bibitem{baby2020e5n}
Deepak Baby (2020). \textit{iSEGAN: Improved Speech Enhancement Generative Adversarial Networks}. arXiv.org.

\bibitem{pasini2021ta3}
Massimiliano Lupo Pasini, and Junqi Yin (2021). \textit{Stable parallel training of Wasserstein conditional generative adversarial neural networks}. 2021 International Conference on Computational Science and Computational Intelligence (CSCI).

\bibitem{goyal2024ufg}
Mandeep Goyal, and Q. Mahmoud (2024). \textit{A Systematic Review of Synthetic Data Generation Techniques Using Generative AI}. Electronics.

\bibitem{wang2024v83}
Shuzhan Wang, Ruxue Jiang, Zhaoqi Wang, et al. (2024). \textit{Deep Learning-based Anomaly Detection and Log Analysis for Computer Networks}. arXiv.org.

\bibitem{liao20249ku}
Wenjie Liao, Like Wu, Shihui Xu, et al. (2024). \textit{A Novel Approach for Intelligent Fault Diagnosis in Bearing With Imbalanced Data Based on Cycle-Consistent GAN}. IEEE Transactions on Instrumentation and Measurement.

\bibitem{peng2024kkw}
Yingying Peng (2024). \textit{A Comparative Analysis Between GAN and Diffusion Models in Image Generation}. Transactions on Computer Science and Intelligent Systems Research.

\bibitem{luo2024znt}
Yihong Luo, Xiaolong Chen, Tianyang Hu, et al. (2024). \textit{You Only Sample Once: Taming One-Step Text-To-Image Synthesis by Self-Cooperative Diffusion GANs}. International Conference on Learning Representations.

\bibitem{chen2024ajr}
Xin Chen, Zaigang Chen, Shiqian Chen, et al. (2024). \textit{Unsupervised GAN With Fine-Tuning: A Novel Framework for Induction Motor Fault Diagnosis in Scarcely Labeled Sample Scenarios}. IEEE Transactions on Instrumentation and Measurement.

\bibitem{song2024htg}
Xiangjin Song, Zhicheng Liu, and Zhaowei Wang (2024). \textit{Rolling bearing fault diagnosis in electric motors based on IDIG-GAN under small sample conditions}. Measurement science and technology.

\bibitem{qin2024a4b}
Zhaohui Qin, Faguo Huang, Jiafang Pan, et al. (2024). \textit{Improved Generative Adversarial Network for Bearing Fault Diagnosis with a Small Number of Data and Unbalanced Data}. Symmetry.

\bibitem{tibermacine2025pye}
Imad Eddine Tibermacine, Samuele Russo, Francesco Citeroni, et al. (2025). \textit{Adversarial denoising of EEG signals: a comparative analysis of standard GAN and WGAN-GP approaches}. Frontiers in Human Neuroscience.

\bibitem{baoueb2024rlq}
Teysir Baoueb, Haocheng Liu, Mathieu Fontaine, et al. (2024). \textit{SpecDiff-GAN: A Spectrally-Shaped Noise Diffusion GAN for Speech and Music Synthesis}. IEEE International Conference on Acoustics, Speech, and Signal Processing.

\bibitem{broll2024edy}
Alexander Broll, M. Rosentritt, Thomas Schlegl, et al. (2024). \textit{A data-driven approach for the partial reconstruction of individual human molar teeth using generative deep learning}. Frontiers Artif. Intell..

\bibitem{wang20245dt}
Yumiao Wang, Chuanfei Zang, Bo Yu, et al. (2024). \textit{WTE-CGAN Based Signal Enhancement for Weak Target Detection}. IEEE Geoscience and Remote Sensing Letters.

\bibitem{megahed2024c23}
Mohammed Megahed, and Ammar Mohammed (2024). \textit{Collaborative-GAN: An Approach for Stabilizing the Training Process of Generative Adversarial Network}. IEEE Access.

\bibitem{zhang2024k8a}
Xiurong Zhang, Shaoqian Fan, and Daoliang Li (2024). \textit{Spectral normalization generative adversarial networks for photovoltaic power scenario generation}. IET Renewable Power Generation.

\bibitem{bhat202445j}
Ranjith Bhat, and Raghu Nanjundegowda (2024). \textit{A Review on Comparative Analysis of Generative Adversarial Networks’ Architectures and Applications}. Journal of Robotics and Control (JRC).

\bibitem{ler20248xg}
Fiete Lüer, and Christian Böhm (2024). \textit{Anomaly Detection using Generative Adversarial Networks Reviewing methodological progress and challenges}. SIGKDD Explorations.

\bibitem{purwono2025spz}
Purwono Purwono, Annastasya Nabila Elsa Wulandari, Alfian Ma’arif, et al. (2025). \textit{Understanding Generative Adversarial Networks (GANs): A Review}. Control Systems and Optimization Letters.

\bibitem{roy2024k91}
Arunava Roy, and Dipankar Dasgupta (2024). \textit{A Distributed Conditional Wasserstein Deep Convolutional Relativistic Loss Generative Adversarial Network With Improved Convergence}. IEEE Transactions on Artificial Intelligence.

\bibitem{seon202526r}
Joonho Seon, Seongwoo Lee, Youngghyu Sun, et al. (2025). \textit{Least Information Spectral GAN With Time-Series Data Augmentation for Industrial IoT}. IEEE Transactions on Emerging Topics in Computational Intelligence.

\bibitem{ni2024y70}
Yao Ni, and Piotr Koniusz (2024). \textit{$\bigcirc\!\!\!\!\bigcirc$ CHAIN: Enhancing Generalization in Data-Efficient GANs via LipsCHitz Continuity ConstrAIned Normalization}. Computer Vision and Pattern Recognition.

\bibitem{ye2024n41}
Ming Ye, Cunhua Pan, Yinfei Xu, et al. (2024). \textit{Generative Adversarial Networks-Based Channel Estimation for Intelligent Reflecting Surface Assisted mmWave MIMO Systems}. IEEE Transactions on Cognitive Communications and Networking.

\bibitem{pajuhanfard2024ult}
Mohammadsaleh Pajuhanfard, Rasoul Kiani, and Victor S. Sheng (2024). \textit{Survey of Quantum Generative Adversarial Networks (QGAN) to Generate Images}. Mathematics.

\bibitem{eskandarinasab202431h}
MohammadReza EskandariNasab, S. M. Hamdi, and S. F. Boubrahimi (2024). \textit{ChronoGAN: Supervised and Embedded Generative Adversarial Networks for Time Series Generation}. International Conference on Machine Learning and Applications.

\bibitem{deebani202549r}
Wejdan Deebani, Lubna Aziz, Arshad Aziz, et al. (2025). \textit{Synergistic transfer learning and adversarial networks for breast cancer diagnosis: benign vs. invasive classification}. Scientific Reports.

\bibitem{ali2024ks3}
Abid Ali, Muhammad Sharif, Muhammad Shahzad Faisal, et al. (2024). \textit{Brain Tumor Segmentation Using Generative Adversarial Networks}. IEEE Access.

\bibitem{ju2024uai}
Xiangui Ju, Chi-Ho Lin, Suan Lee, et al. (2024). \textit{Melanoma classification using generative adversarial network and proximal policy optimization}. Photochemistry and Photobiology.

\bibitem{xu2024u5a}
Chi Xu, Haozheng Xu, and S. Giannarou (2024). \textit{Distance Regression Enhanced With Temporal Information Fusion and Adversarial Training for Robot-Assisted Endomicroscopy}. IEEE Transactions on Medical Imaging.

\bibitem{elbaz2025wzb}
Mostafa Elbaz, Wael Said, G. Mahmoud, et al. (2025). \textit{A dual GAN with identity blocks and pancreas-inspired loss for renewable energy optimization}. Scientific Reports.

\bibitem{chang2024c0a}
Yuanhong Chang, Jinglong Chen, Rong Su, et al. (2024). \textit{Two-Phase Dual-Adversarial Agents With Multivariate Information for Unsupervised Anomaly Detection of IIoT-Edge Devices}. IEEE Internet of Things Journal.

\bibitem{guo2024y0l}
Pang Guo, and Yining Chen (2024). \textit{Enhanced Yield Prediction in Semiconductor Manufacturing: Innovative Strategies for Imbalanced Sample Management and Root Cause Analysis}. International Symposium on the Physical and Failure Analysis of Integrated Circuits.

\bibitem{peng2024crk}
Jun Peng, Kaiyi Chen, Yuqing Gong, et al. (2024). \textit{Cyclic Consistent Image Style Transformation: From Model to System}. Applied Sciences.

\end{thebibliography}

\end{document}