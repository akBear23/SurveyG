\subsection*{Conditional and Text-to-Image Synthesis}

Generative Adversarial Networks (GANs), initially designed for unconditional image generation, quickly evolved to address the critical need for controlled output, allowing users to specify desired characteristics of the synthesized images. This section traces the progression from simple conditional generation to complex text-to-image synthesis, highlighting the increasing ability of GANs to interpret and visualize semantic information.

The foundational step towards controlled generation was the introduction of Conditional Generative Adversarial Nets (cGANs) by \cite{Mirza2014}. Unlike their unconditional predecessors, cGANs feed additional conditional information, such as class labels or other attributes, to both the generator and the discriminator. This direct input guides the generator to produce samples corresponding to the specified conditions, while the discriminator learns to verify both the realism and the adherence to the given condition. Building upon this, Auxiliary Classifier GANs (AC-GANs) \cite{Odena2017} further enhanced conditional synthesis by incorporating an auxiliary classifier into the discriminator. This classifier not only distinguishes between real and fake images but also predicts the class label of the input, thereby compelling the generator to produce samples that are both realistic and correctly classified, leading to better disentanglement and more robust conditional generation.

The capability of GANs was significantly expanded with the advent of text-to-image synthesis, where the conditional information takes the form of natural language descriptions. Early efforts, such as those by \cite{Reed2016}, demonstrated the feasibility of generating images directly from text embeddings. These models mapped textual descriptions into a latent space, which then guided the generator to synthesize corresponding images, with the discriminator evaluating the consistency between the generated image and the input text. However, these initial models often struggled with generating high-resolution and photo-realistic images, particularly for complex scenes.

To overcome these limitations, multi-stage architectures emerged, notably StackGAN \cite{Zhang2017}. StackGAN employs a two-stage process: the first stage generates a low-resolution image based on the global text description, and the second stage refines this initial output into a higher-resolution, photo-realistic image by focusing on finer details. This hierarchical approach significantly improved the quality and resolution of text-conditioned images. Further advancements in fine-grained control and semantic alignment were achieved with AttnGAN \cite{Xu2018}, which introduced an attention mechanism. AttnGAN allows the generator to selectively attend to different words in the text description when generating specific regions of the image, ensuring that local image details are semantically consistent with relevant parts of the text.

More recently, the integration of robust architectures like StyleGAN with text conditioning has pushed the boundaries of quality and control. StyleGAN-T \cite{Sauer2024} adapts the highly successful StyleGAN framework for text-to-image synthesis, leveraging its disentangled latent space and advanced generation capabilities. This approach yields high-fidelity, text-conditioned images with improved semantic alignment and offers more intuitive control over the generated output through natural language. This represents a significant leap, allowing users to specify desired outputs with natural language, opening doors for creative applications and content generation.

Despite these remarkable advancements, challenges remain in conditional and text-to-image synthesis. Generating complex scenes with multiple objects and intricate spatial relationships, maintaining semantic consistency across diverse textual descriptions, and ensuring compositional understanding are still active areas of research. The robustness of these models to ambiguous or underspecified text prompts also needs improvement. Future directions will likely focus on enhancing the models' understanding of complex semantic compositions, improving the interpretability of generated outputs, and developing more robust evaluation metrics for text-to-image consistency.