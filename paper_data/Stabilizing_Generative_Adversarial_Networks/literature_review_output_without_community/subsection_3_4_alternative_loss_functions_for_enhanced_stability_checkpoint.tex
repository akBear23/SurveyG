\subsection{Alternative Loss Functions for Enhanced Stability}

The persistent challenge of training instability in Generative Adversarial Networks (GANs), characterized by issues such as vanishing gradients, mode collapse, and oscillating performance, has driven extensive research into modifying the core objective functions. Beyond the f-divergences initially explored, a significant line of inquiry has focused on designing alternative loss functions that provide smoother, non-saturating gradients and more robust convergence properties, thereby enhancing the overall training dynamics and generative quality. This quest highlights a continuous search for diverse mathematical perspectives to achieve stable and effective generative modeling.

A foundational contribution in this area is the introduction of Least Squares Generative Adversarial Networks (LSGANs) \cite{Mao2017}. LSGANs address the limitations of the traditional sigmoid cross-entropy loss, which can suffer from vanishing gradients when the discriminator becomes overly confident and saturates. By replacing this with a least squares loss function, LSGANs ensure that both the generator and discriminator receive meaningful, non-saturating gradients throughout training. This modification encourages the generator to produce samples closer to the decision boundary, leading to improved training stability and the generation of higher quality images compared to original GANs. Theoretically, the least squares objective implicitly minimizes the Pearson $\chi^2$ divergence, offering a distinct and often more stable mathematical perspective than the Jensen-Shannon divergence minimized by early GANs \cite{Mao2017}.

Following the principles of non-saturating objectives, the adversarial hinge loss emerged as another cornerstone for stable GAN training, particularly in high-fidelity models. The hinge loss provides a clear margin for classification, penalizing the discriminator only when its output for real samples falls below a certain positive margin, or when its output for fake samples rises above a negative margin. This margin-based formulation ensures that the discriminator does not become overly confident too early, preventing gradient saturation and providing a consistent learning signal to the generator. For the generator, the hinge loss encourages it to push fake samples beyond the discriminator's negative margin. This approach has been widely adopted in state-of-the-art architectures, including Self-Attention GANs (SAGAN) and BigGAN, due to its effectiveness in promoting stable training and high-quality image synthesis. \textcite{wang20178xf} further explored adaptive hinge loss functions, demonstrating how dynamically adjusting the margin based on the expected energy of the target distribution can lead to improved stability and performance, with theoretical proofs of convergence under certain assumptions.

These alternative loss functions represent a critical shift in GAN optimization strategies. Unlike Wasserstein GANs with Gradient Penalty (WGAN-GP) \cite{Gulrajani2017}, which primarily enforce a Lipschitz constraint on the discriminator to ensure meaningful gradients across the input space, LSGANs and hinge loss directly reshape the objective landscape itself. They achieve stability by preventing the discriminator's loss from saturating, thus providing a more consistent and robust gradient flow to the generator. This distinction is crucial: while WGAN-GP focuses on the *smoothness* of the discriminator function, LSGANs and hinge loss focus on the *shape* of the loss function to avoid regions of zero gradient. The theoretical framework proposed by \textcite{chu2020zbv} further elucidates the importance of smoothness and specific divergence properties for guaranteeing eventual stationarity of the generator, highlighting why non-saturating and margin-based losses contribute to stability.

Beyond these widely adopted approaches, researchers have continued to explore novel modifications to loss functions. \textcite{zadorozhnyy20208ft} introduced adaptive weighted discriminator loss functions, or "aw-loss functions." This method addresses the challenge that an equally weighted sum of real and fake losses can sometimes benefit one part of the training while harming the other, leading to instability and mode collapse. By adaptively weighting the real and fake components of the discriminator's loss based on their gradients, aw-loss functions guide the discriminator's training in a direction that explicitly benefits overall GAN stability. This dynamic balancing act has shown significant improvements in Inception Scores (IS) and Fréchet Inception Distance (FID) metrics on various datasets, demonstrating the value of fine-grained control over loss components.

Another innovative approach is presented by Constrained Generative Adversarial Networks (GAN-C) \cite{chao2021ynq}. This method introduces an explicit constraint on the discriminator's output, aiming to bound its function space. While theoretically sharing the same Nash equilibrium as the standard GAN, this constraint helps to regularize the discriminator's behavior, preventing it from becoming overly powerful or unstable. In practice, this leads to faster convergence during training and the generation of higher-quality data, as demonstrated across a diverse set of image datasets. This highlights that even subtle modifications to how the discriminator's output is handled within the loss framework can significantly impact training stability and generative performance.

In conclusion, the evolution of GAN loss functions from the original sigmoid cross-entropy to LSGANs, adversarial hinge loss, and more advanced adaptive and constrained objectives, underscores a fundamental principle: robust and non-saturating gradients are paramount for stable adversarial training. These diverse mathematical strategies, whether by reshaping the objective landscape, introducing classification margins, or adaptively weighting loss components, have collectively provided the practical stability necessary for the subsequent architectural innovations that led to high-fidelity generative models. While these advancements have significantly mitigated issues like vanishing gradients, the complete elimination of mode collapse and the guarantee of universal convergence across all data distributions remain active areas of research, as noted by reviews like \cite{wang2019w53}. Future work continues to explore hybrid approaches and novel theoretical frameworks to further enhance the robustness and generative power of GANs.