\subsection{Early Architectural Guidelines: Deep Convolutional GANs (DCGANs)}

The initial formulation of Generative Adversarial Networks (GANs) \cite{Goodfellow2014} presented a powerful theoretical framework for generative modeling, but their practical implementation was plagued by significant training instability and difficulties in convergence, often producing incoherent or limited-diversity outputs. A crucial methodological progression towards making GANs a more implementable framework was the introduction of Deep Convolutional Generative Adversarial Networks (DCGANs) by \cite{Radford2015}.

DCGANs marked the first significant step towards practical GANs by effectively integrating Convolutional Neural Networks (CNNs) into both the generator and discriminator architectures. This integration leveraged the hierarchical feature learning capabilities of CNNs, enabling the generation of more coherent and visually plausible images compared to earlier fully-connected architectures. Beyond simply using CNNs, \cite{Radford2015} introduced a set of architectural heuristics that provided initial stability to GAN training and enabled the generation of more coherent images, marking a crucial methodological progression from the abstract GAN concept to a more implementable framework.

Several key architectural guidelines were established. To address training instability and facilitate deeper networks, batch normalization layers were introduced in both the generator and discriminator. Batch normalization helps stabilize learning by normalizing the input to each layer, preventing internal covariate shift and allowing for higher learning rates, which was vital for the deeper convolutional structures. Specific activation functions were also prescribed: ReLU (Rectified Linear Unit) was predominantly used in the generator for all layers except the output, which typically used Tanh to produce pixel values in a normalized range. For the discriminator, LeakyReLU was employed, providing a non-zero gradient for negative inputs and helping to prevent 'dying ReLU' problems, thereby contributing to better gradient flow and more stable adversarial training.

A pivotal architectural choice was the avoidance of pooling layers in favor of strided convolutions. In the generator, fractional-strided convolutions (often referred to as transposed convolutions) were used for spatial upsampling, allowing the network to learn its own upsampling strategy rather than relying on fixed interpolation. Conversely, the discriminator utilized strided convolutions for spatial downsampling. This approach allowed the network to learn more effective spatial transformations, preserving more information and often leading to better image quality than traditional pooling operations.

These architectural heuristics provided initial stability to GAN training, moving the field from an abstract concept to a more robust and implementable framework. The structured use of CNNs and the proposed architectural choices enabled DCGANs to generate images with significantly improved visual quality and coherence, demonstrating the potential of GANs for unsupervised representation learning. For instance, \cite{Radford2015} showed that the learned features in the discriminator could be effectively used for classification tasks, and that latent space arithmetic could produce meaningful semantic manipulations in generated images, such as interpolating between gender or expressions.

Despite these advancements, DCGANs still faced limitations. While stability was improved, training remained sensitive to hyperparameter choices and could still suffer from issues like mode collapse, where the generator produces a limited variety of samples. The resolution of generated images was also relatively modest compared to later advancements. Thus, while DCGANs established fundamental architectural principles for deep generative models and showcased the immense potential of GANs, their inherent challenges in achieving consistent stability and scaling to higher resolutions laid the groundwork for subsequent research into more robust training methodologies and advanced architectures.