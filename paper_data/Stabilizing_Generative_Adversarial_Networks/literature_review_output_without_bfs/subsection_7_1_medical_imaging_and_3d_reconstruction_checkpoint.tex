\subsection{Medical Imaging and 3D Reconstruction}

The sensitive domains of medical imaging and 3D reconstruction present unique challenges for generative models, primarily due to data scarcity, privacy concerns, and the critical need for high-fidelity, clinically relevant synthetic data. Stabilized Generative Adversarial Networks (GANs) have emerged as transformative tools, addressing these limitations by enabling robust data augmentation and sophisticated anatomical reconstruction, thereby enhancing the performance and applicability of downstream analytical models. The inherent stability mechanisms, such as those discussed in Sections 3, 4, and 5, are paramount for generating diverse and realistic medical data, preventing artifacts or mode collapse that could compromise diagnostic accuracy.

A primary application of stabilized GANs in medical imaging is data augmentation, particularly vital in areas like cancer gene classification and medical image analysis where real-world data acquisition is often constrained. For instance, \cite{wei2021qea} highlights the problem of inadequate cancer gene expression data, which typically leads to poor generalization in classification models. While the specific GAN architecture is not detailed, the success of their proposed Cancer Classification Model, which leverages GANs to synthesize artificial cancer data "highly similar to the real one" and demonstrably improve classification results, implicitly relies on the GAN's ability to maintain stability and fidelity during generation. Without stable training, the generated data would lack the diversity or realism required to effectively augment limited datasets.

Building on this, more advanced and explicitly stabilized GAN architectures have shown significant promise. \cite{deebani202549r} utilized a conditional Wasserstein GAN (cWGAN) for data augmentation in breast cancer diagnosis. By employing the Wasserstein distance and gradient penalty (as discussed in Section 3.2), cWGAN ensures a smoother loss landscape and more stable training, crucial for generating high-quality synthetic breast cancer images. This approach, combined with multi-scale transfer learning, achieved an impressive 99.2\% accuracy for binary classification on the BreakHis dataset, significantly outperforming existing methods. The conditional aspect of cWGAN further enhances its utility by allowing targeted generation of specific cancer subtypes, a critical feature for augmenting minority classes in imbalanced medical datasets. Similarly, \cite{park2021v6f} demonstrated that GAN-based synthetic MRI images for glioblastoma ensured sufficient morphological variations, improving the diagnostic accuracy for isocitrate dehydrogenase (IDH)-mutant types from 84.1\% to 90.9\% in an independent validation set. The ability to generate morphologically variable yet realistic samples is a direct consequence of stable GAN training, preventing mode collapse where the generator might only produce a limited set of variations.

Beyond direct image synthesis, stabilized GANs are also employed for medical image standardization. \cite{liang2018axu} introduced GANai for standardizing CT images across non-standard imaging protocols. Their model incorporates an "alternative improvement training strategy" with phase-specific loss functions and training data, which significantly enhances the efficiency and stability of GAN training. This approach mitigates the "lack-of-detail problem" in CT image synthesis, ensuring that generated images maintain crucial radiomic features for consistent analysis. The stability provided by GANai's unique training strategy is essential for generating standardized images that are both realistic and diagnostically reliable. Furthermore, the application extends to biosignals, where \cite{munia20201u2} leveraged a Wasserstein GAN (WGAN) for oversampling electrocardiogram (ECG) signals. The inherent stability of WGAN (Section 3.1), with its loss function correlating with generated image quality, enabled the generation of synthetic ECG data that improved minority-class classification accuracy for imbalanced biomedical datasets, validated using the Fréchet Inception Distance (FID) score.

In the realm of complex 3D reconstruction, the enhanced stability and fidelity of modern GAN architectures are particularly impactful. Acquiring comprehensive 3D anatomical data is often challenging, invasive, or expensive. Advanced GANs, such as StyleGAN-2 (detailed in Section 5.2), have been successfully applied to these tasks. For instance, \cite{broll2024edy} utilized StyleGAN-2 for the partial reconstruction of individual human molar teeth. Their innovative approach involved training StyleGAN-2 on 3D mesh files of full dental crown restorations, learning the intricate morphology of teeth. A crucial aspect was a PCA-optimized 2D projection method to convert 3D data into 2D images for StyleGAN-2 input, followed by a downstream optimization process for reconstruction. The architectural refinements of StyleGAN-2, including its style-based generator and path length regularization (Section 5.2), were critical for generating highly realistic and diverse occlusal surfaces, achieving Root Mean Square Error (RMSE) values between 0.02mm and 0.18mm. Notably, dentists preferred the GAN-based restorations for 3 out of 4 inlay geometries compared to conventional CAD procedures, underscoring the clinical relevance of StyleGAN-2's high-fidelity output.

Another significant contribution to 3D medical generation comes from \cite{saqur2018oqp}, who proposed CapsGAN for generating 3D images with high geometric transformations. Their work explicitly tackles GAN training stability by experimenting with Wasserstein distance (including gradient clipping and penalty, as discussed in Sections 3.1 and 3.2) and Spectral Normalization (Section 4.1). These foundational stability mechanisms are crucial for CapsGAN's ability to generate robust 3D images, paving the way for applications in 3D medical imaging and potentially video generation where geometric consistency is paramount.

These diverse applications collectively underscore the pivotal role of stabilized GANs in advancing medical imaging and 3D reconstruction. The ability to generate realistic, diverse, and anatomically plausible synthetic data—whether for augmenting cancer gene datasets, standardizing CT images, or reconstructing complex dental structures—directly addresses the pervasive issue of data limitation in medical research and clinical practice. The explicit integration of stability mechanisms, from WGAN's gradient penalties to StyleGAN's architectural refinements, is not merely an improvement but a prerequisite for generating data that meets the stringent requirements of clinical validity and diagnostic utility.

Despite these significant strides, challenges remain in ensuring the absolute clinical validity, interpretability, and generalizability of GAN-generated medical data. Future research must focus on developing more robust quantitative validation metrics tailored for synthetic medical data, exploring explainable AI techniques for GANs to provide transparency in their generative process, and further enhancing the controllability and specificity of generative models to meet the stringent requirements of diverse medical applications. Addressing these challenges will be crucial for paving the way for their broader and more confident adoption in clinical settings, ultimately improving patient care and accelerating medical research.