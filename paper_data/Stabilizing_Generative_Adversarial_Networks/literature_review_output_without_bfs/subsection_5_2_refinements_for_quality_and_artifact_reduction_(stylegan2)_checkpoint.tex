\subsection{Refinements for Quality and Artifact Reduction (StyleGAN2)}

The pursuit of generating highly realistic and controllable images with Generative Adversarial Networks (GANs) has been significantly advanced by the StyleGAN architecture. While the original StyleGAN \cite{Karras2019} revolutionized image synthesis by introducing a style-based generator that enabled intuitive control over visual features and disentangled latent representations, it also exhibited several persistent visual artifacts, such as 'water droplet' textures and an inconsistent mapping from latent codes to images. Addressing these limitations became crucial for further progress, leading to the development of StyleGAN2 \cite{Karras2020}.

The foundational StyleGAN model \cite{Karras2019} built upon the progressive growing methodology introduced in PGGAN \cite{karras2017raw}, which gradually increases network complexity and resolution during training to stabilize the process and achieve high-resolution outputs. StyleGAN enhanced this by incorporating a mapping network to transform latent codes into an intermediate style representation, which then modulated the feature maps of the generator via Adaptive Instance Normalization (AdaIN) layers. This design successfully disentangled different levels of visual features, allowing for fine-grained control. However, the reliance on progressive growing and the specific implementation of AdaIN introduced undesirable side effects. For instance, the progressive growing approach, while stabilizing training, could lead to resolution-specific artifacts, and the AdaIN layers were found to inadvertently contribute to signal-to-noise ratio issues and the characteristic 'water droplet' artifacts, where generated textures appeared to stick to specific coordinates.

StyleGAN2 \cite{Karras2020} meticulously identified and addressed these shortcomings through a series of architectural and regularization refinements, significantly boosting overall image quality and stability. One of the most prominent changes was the **removal of progressive growing**. Instead of gradually adding layers, StyleGAN2 trains a fixed-size generator from the outset, which helps to eliminate resolution-specific artifacts and simplifies the training pipeline. To compensate for the stability benefits lost by removing progressive growing, StyleGAN2 redesigned the normalization layers. The original AdaIN was replaced with a **demodulation mechanism** applied directly to the convolutional weights. This modification ensures that the magnitude of feature maps remains consistent across different styles, effectively preventing the 'water droplet' artifacts and improving the signal-to-noise ratio by avoiding the introduction of unwanted high-frequency noise.

Beyond architectural changes, StyleGAN2 introduced a novel **path length regularization** technique. This regularization encourages a more consistent and smooth mapping from the latent space to the image space. Specifically, it penalizes the deviation of the generator's output from the ideal behavior where a fixed-size step in latent space corresponds to a fixed-size change in the generated image. By ensuring that the magnitude of the Jacobian of the generator mapping is consistent, path length regularization improves latent space disentanglement and prevents abrupt changes in generated images for small latent code perturbations. This leads to a more predictable and interpretable latent space, which is critical for applications requiring fine-grained control over image attributes.

The cumulative effect of these refinements in StyleGAN2 \cite{Karras2020} was a substantial improvement in image fidelity, a marked reduction in visual artifacts, and enhanced training stability. These advancements further solidified the StyleGAN architecture as a benchmark for state-of-the-art image synthesis, demonstrating that careful analysis of architectural components and the introduction of targeted regularization can overcome persistent challenges in GAN training. Subsequent work, such as StyleGAN3 \cite{Karras2021}, continued this trajectory of refinement by addressing aliasing issues in the generator, highlighting the continuous effort to achieve truly alias-free and perfectly realistic image generation. Despite these significant strides, the computational cost and substantial data requirements for training such high-fidelity models remain a common limitation, posing ongoing challenges for broader accessibility and application.