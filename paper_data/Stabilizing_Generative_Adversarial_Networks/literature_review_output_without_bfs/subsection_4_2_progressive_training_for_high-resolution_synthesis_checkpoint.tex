\subsection*{Progressive Training for High-Resolution Synthesis}

The pursuit of generating high-fidelity, photo-realistic images with Generative Adversarial Networks (GANs) has long been hampered by significant training instabilities and the inherent difficulty of synthesizing complex details at high resolutions. Early GAN models frequently suffered from issues such as mode collapse, where the generator produces a limited variety of outputs, and vanishing or exploding gradients, which impede stable learning \cite{wang2019w53, wiatrak20194ib, jabbar2020aj0}. While various regularization techniques and alternative loss functions were proposed to mitigate these problems, such as unrolled GANs \cite{metz20169ir}, mode regularization \cite{che2016kho}, and Least Squares GANs \cite{mao2017ss0}, achieving stable training at resolutions beyond 256x256 pixels remained a formidable challenge.

Prior to the advent of progressive training, some approaches attempted to tackle higher resolutions by decomposing the generation process into multiple stages. For instance, StackGAN \cite{zhang2016mm0} employed a stacked architecture where a Stage-I GAN generated low-resolution images based on text descriptions, and a subsequent Stage-II GAN refined these into higher-resolution, photo-realistic outputs. This method, while innovative, relied on a cascaded design of separate networks for different resolutions, which could be complex to coordinate and optimize.

A groundbreaking paradigm shift arrived with the introduction of Progressive Growing of GANs (PGGAN) by Karras et al. \cite{karras2017raw}. This method fundamentally re-imagined the training process by adopting a curriculum learning approach: instead of training a full-resolution GAN from scratch, PGGAN begins by generating very small, low-resolution images (e.g., 4x4 pixels). As training progresses, new layers are incrementally added to both the generator and discriminator networks, gradually increasing the output resolution and the complexity of learned features. For example, after training at 4x4, new layers are smoothly faded in to enable 8x8 generation, and this process continues up to resolutions like 1024x1024 pixels for datasets such as CelebA-HQ \cite{karras2017raw}.

This progressive growth strategy offered several critical advantages. By starting with simpler tasks, PGGAN significantly improved training stability, as the networks first learned broad structures before focusing on fine details. This gradual introduction of complexity also drastically reduced the problem of mode collapse, allowing the generator to capture a wider diversity of the data distribution. Furthermore, the method accelerated training convergence, as early stages could be learned quickly, providing a robust foundation for subsequent, higher-resolution layers. The ability to produce unprecedentedly high-quality and diverse images at resolutions previously unattainable, such as 1024x1024, underscored PGGAN's pivotal contribution to the field \cite{karras2017raw}. The authors also incorporated techniques to increase variation in generated images and detailed implementation specifics to prevent unhealthy competition between the generator and discriminator, further enhancing stability.

The success of PGGAN demonstrated that the training process itself, when structured intelligently, could be a powerful tool for stabilization and performance enhancement, rather than solely relying on architectural modifications or loss function changes. While PGGAN introduced the progressive growing strategy, subsequent advancements in GANs often integrated this approach with other stabilization techniques. For instance, Spectral Normalization (SN) \cite{miyato2018arc}, which constrains the Lipschitz constant of discriminator layers, became a widely adopted method for improving training stability and was frequently combined with progressive growing in later high-fidelity GAN architectures, as seen in works like BigGAN and StyleGAN. This synergistic combination of progressive training with robust regularization further pushed the boundaries of image synthesis quality and stability \cite{qin2024a4b}.

In conclusion, Progressive Growing of GANs marked a significant turning point in generative modeling, moving beyond mere architectural scaling to introduce a dynamic training methodology. By progressively increasing resolution and complexity, PGGAN effectively addressed long-standing issues of training instability and mode collapse, paving the way for the generation of high-resolution, diverse, and visually compelling images. However, despite its success, the computational cost associated with training such large models, especially at very high resolutions, remains a challenge, and the continuous pursuit of even greater fidelity and disentanglement continues to drive research in this area.