\subsection*{Signal Processing and Denoising}

The accurate analysis of sensitive biomedical signals, such as Electroencephalography (EEG), is frequently hampered by pervasive noise and artifacts that obscure critical information. These interferences, often nonlinear and dynamic, pose significant challenges for traditional denoising methods like linear filtering or wavelet-based thresholding, which often struggle to adapt to the complex, non-stationary nature of biological signals without distorting essential features \cite{tibermacine2025pye}. Consequently, there is a pressing need for more robust and adaptive approaches capable of distinguishing between noise and subtle physiological activity.

In this context, Generative Adversarial Networks (GANs), particularly their stabilized variants, have emerged as powerful tools for signal processing, demonstrating a unique capability to learn complex data distributions and generate realistic synthetic signals. This generative capacity can be leveraged for tasks such as data augmentation, super-sampling, and crucially, denoising and artifact removal \cite{hartmann2018h3s}. The inherent adversarial training mechanism allows GANs to learn a mapping from noisy to clean signal distributions, offering a distinct advantage over methods that rely on explicit noise models.

A significant area of research has focused on the utility of stabilized GANs in signal denoising, where the core challenge lies in balancing aggressive noise removal with the high-fidelity reconstruction of subtle signal features. For instance, \cite{tibermacine2025pye} conducted a comprehensive comparative analysis of a conventional GAN model and a Wasserstein GAN with Gradient Penalty (WGAN-GP) for the adversarial denoising of EEG signals. Their work was motivated by the limitations of existing methods against nonlinear EEG artifacts and the scarcity of direct, systematic comparisons between different stable GAN architectures under identical experimental conditions. The "conventional GAN" in this context typically refers to an architecture employing the original minimax objective function, often with convolutional layers (similar to DCGANs), which, as discussed in Section 2.1, is prone to training instabilities but can sometimes excel in specific aspects of generation if carefully tuned \cite{wang2019w53}. WGAN-GP, on the other hand, leverages the Wasserstein distance and gradient penalties to ensure a smoother loss landscape and more stable training, as detailed in Section 3.2.

The study by \cite{tibermacine2025pye} meticulously designed an adversarial pipeline, evaluating both GAN variants on two distinct EEG datasets (healthy and unhealthy recordings). They employed a comprehensive suite of quantitative metrics, including signal-to-noise ratio (SNR), peak signal-to-noise ratio (PSNR), correlation coefficient, mutual information, dynamic time warping (DTW) distance, and relative root mean squared error (RRMSE). The findings demonstrated that both adversarial frameworks significantly outperformed classical wavelet-based thresholding and linear filtering methods, showcasing their superior adaptability to nonlinear distortions prevalent in EEG data.

Critically, the comparison revealed a practical trade-off between the two GAN architectures. The WGAN-GP model exhibited greater training stability and achieved higher SNR values (up to 14.47dB compared to 12.37dB for the conventional GAN), alongside consistently lower RRMSE values. This indicates WGAN-GP's proficiency in more aggressive and robust noise suppression, making it suitable for scenarios where strong artifact reduction is paramount. Conversely, the conventional GAN model proved more adept at preserving finer signal details, reflected in a higher PSNR of 19.28dB and a correlation coefficient exceeding 0.90 in several recordings. This suggests that while potentially less stable in training, a conventional GAN, when successfully trained, might offer superior fidelity for subtle signal features, which is crucial in clinical settings where minute neural activity is diagnostically significant.

Beyond pure denoising, the principles of stabilized GANs have also been applied to related challenges in EEG signal processing. For instance, \cite{panwar2019psx} utilized a semi-supervised WGAN-GP (sWGAN-GP) for classifying driving fatigue from EEG signals. While their primary goal was classification, the sWGAN-GP framework was instrumental in augmenting limited labeled data with generated EEG samples, effectively addressing frequency artifacts and training instability. This highlights how the stability and generative power of WGAN-GP extend beyond direct denoising to improve the robustness of downstream analytical tasks by providing cleaner or augmented data. Similarly, \cite{hartmann2018h3s} explored EEG-GANs for generating naturalistic EEG signals, noting the importance of WGAN improvements for training stabilization and the potential for "restoration of corrupted data segments," directly linking generation capabilities to denoising applications.

In summary, the application of stabilized GANs to signal processing, particularly for denoising EEG, underscores their transformative potential. The comparative analysis by \cite{tibermacine2025pye} provides valuable insights into the performance trade-offs inherent in different stable GAN variants, demonstrating that "stabilization" is not merely about achieving convergence, but also about understanding the nuanced impact on output quality and fidelity. While WGAN-GP offers superior training stability and aggressive noise reduction, conventional GANs might be preferable for preserving subtle signal details. Future research should build upon such comparative analyses, exploring how novel stabilization methods (e.g., spectral normalization, progressive growing, or self-attention mechanisms adapted for time series, as discussed in Section 4) impact these application-specific trade-offs. Further investigation into hybrid models, real-time denoising capabilities, and generalization across diverse physiological signal types will be crucial for advancing the field and enhancing data quality for subsequent analysis in various real-world scenarios.