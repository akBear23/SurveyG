\documentclass[12pt,a4paper]{article}
    \usepackage[utf8]{inputenc}
    \usepackage[T1]{fontenc}
    \usepackage{amsmath,amsfonts,amssymb}
    \usepackage{graphicx}
    \usepackage[margin=2.5cm]{geometry}
    \usepackage{setspace}
    \usepackage{natbib}
    \usepackage{url}
    \usepackage{hyperref}
    \usepackage{booktabs}
    \usepackage{longtable}
    \usepackage{array}
    \usepackage{multirow}
    \usepackage{wrapfig}
    \usepackage{float}
    \usepackage{colortbl}
    \usepackage{pdflscape}
    \usepackage{tabu}
    \usepackage{threeparttable}
    \usepackage{threeparttablex}
    \usepackage[normalem]{ulem}
    \usepackage{makecell}
    \usepackage{xcolor}

    % Set line spacing
    \doublespacing

    % Configure hyperref
    \hypersetup{
        colorlinks=true,
        linkcolor=blue,
        filecolor=magenta,      
        urlcolor=cyan,
        citecolor=red,
    }

    % Title and author information
    \title{A Comprehensive Literature Review with Self-Reflection}
    \author{Literature Review}
    \date{\today}

    \begin{document}

    \maketitle

    % Abstract (optional)
    \begin{abstract}
    This literature review provides a comprehensive analysis of recent research in the field. The review synthesizes findings from 194 research papers, identifying key themes, methodological approaches, and future research directions.
    \end{abstract}

    \newpage
    \tableofcontents
    \newpage

    \label{sec:introduction}

\section{Introduction}
\label{sec:introduction}

\subsection{The Promise and Challenges of Generative Adversarial Networks}
\label{sec:1\_1\_the\_promise\_\_and\_\_challenges\_of\_generative\_adversarial\_networks}

Generative Adversarial Networks (GANs), introduced by Goodfellow et al. \cite{goodfellow2014generative}, represent a groundbreaking class of generative models that have revolutionized synthetic data generation. At their core, GANs operate on a minimax game between two neural networks: a generator ($G$) that learns to produce data mimicking a target distribution, and a discriminator ($D$) that learns to distinguish between real data and data generated by $G$. This adversarial process has demonstrated an immense promise, particularly in computer vision, where GANs have proven capable of synthesizing highly realistic images across various domains, from generating human faces \cite{karras2017raw} to artistic styles \cite{wang2019w53} and even scientific data \cite{herr20208x4}. Their potential extends to diverse applications, including data augmentation, image-to-image translation, text-to-image synthesis \cite{zhang2016mm0}, and learning from simulated environments \cite{shrivastava2016uym}.

Despite this revolutionary potential, the practical application of GANs has been consistently plagued by significant and persistent challenges since their inception. The adversarial training paradigm, while powerful, inherently leads to a complex, non-convex minimax optimization problem that is notoriously difficult to solve \cite{pfau2016v7o, liang2018r52}. A primary issue is training instability, which manifests as oscillations, non-convergence, or even divergence of the generator and discriminator \cite{wiatrak20194ib, jabbar2020aj0}. This instability is often attributed to the vanishing or exploding gradients that can occur when the discriminator becomes too strong or too weak, hindering the generator's ability to learn effectively \cite{mao2017ss0, arjovsky2017ze5}. The original GAN formulation, relying on Jensen-Shannon divergence, was particularly susceptible to these gradient problems, especially when there was little overlap between the real and generated data distributions \cite{roth2017eui}.

Another critical challenge is "mode collapse," where the generator produces a limited variety of outputs, failing to capture the full diversity and richness of the real data distribution \cite{che2016kho, metz20169ir}. Instead of exploring the entire data manifold, the generator converges to a few modes that consistently fool the discriminator, leading to a lack of diversity in the generated samples. This issue fundamentally undermines the utility of a generative model, as it fails to represent the true complexity of the underlying data. Theoretical analyses have further highlighted a provable trade-off between a generative model's ability to fit multimodal distributions and the stability of its training, often linked to the Lipschitz constant of the generator \cite{salmona202283g}.

Beyond these core training difficulties, other practical hurdles include the sensitivity to hyperparameter choices, which often necessitate extensive tuning and domain expertise \cite{miyato2018arc, chu2020zbv}. The computational cost and data requirements for training high-fidelity GANs are substantial, often demanding large-scale GPU clusters and vast datasets \cite{liu20212c2}. Furthermore, training with limited data can lead to discriminator overfitting and training divergence \cite{karras202039x}. The objective evaluation of GAN performance is also a challenge, as metrics like Inception Score (IS) and Fréchet Inception Distance (FID), while widely used, do not perfectly correlate with human perception or fully capture both quality and diversity \cite{wang2019w53, jabbar2020aj0}.

In summary, while Generative Adversarial Networks hold immense promise for synthesizing highly realistic and diverse data across numerous applications, their inherent practical difficulties—stemming from training instability, mode collapse, convergence issues, and demanding resource requirements—have presented a significant barrier to their widespread and reliable adoption. This fundamental tension between their powerful capabilities and their persistent practical hurdles has driven extensive research into developing more robust and stable training methodologies.
\subsection{Scope and Organization of the Review}
\label{sec:1\_2\_scope\_\_and\_\_organization\_of\_the\_review}

This literature review provides a comprehensive and pedagogically structured exploration of Generative Adversarial Networks (GANs), specifically focusing on the evolution of techniques designed to enhance their training stability and generative capabilities. Recognizing the inherent challenges that have plagued GANs since their inception, such as mode collapse, vanishing gradients, and general training instability \cite{jabbar2020aj0, bhat202445j}, this review systematically traces the intellectual journey of researchers in overcoming these hurdles. The organizational framework is meticulously designed to guide the reader through a logical progression, from foundational concepts and early stabilization attempts to advanced architectural innovations, practical enhancements, and real-world applications, ultimately providing a clear roadmap of the interconnected research landscape in GAN stabilization. This structure aims to not only describe the advancements but also to critically evaluate the progression and interdependencies of various solutions, highlighting why certain approaches emerged as responses to prior limitations.

The review commences in \textbf{Section 2, "Fundamentals of GANs and Initial Instabilities,"} by establishing the core principles of the original GAN framework and detailing the fundamental challenges that necessitated extensive research into stabilization techniques. This section lays the groundwork by explaining the adversarial interplay between the generator and discriminator, and elaborating on issues like mode collapse, vanishing or exploding gradients, and the difficulty in achieving stable convergence, which are critical for appreciating subsequent innovations \cite{bhat202445j}. It further introduces the essential quantitative metrics used to assess GAN stability and quality, providing a framework for evaluating the effectiveness of various stabilization techniques discussed later.

Building upon this understanding of initial problems, \textbf{Section 3, "Foundational Stability through Objective Function and Gradient Regularization,"} delves into the initial and fundamental breakthroughs in stabilizing GAN training. This section explores how critical reformulations of the objective function, such as the introduction of the Wasserstein distance, provided a smoother and more meaningful loss landscape, significantly mitigating early instabilities. It further examines the development of gradient penalties and other regularization techniques designed to enforce crucial mathematical properties, like Lipschitz continuity, on the discriminator, thereby transforming GANs from notoriously unstable models into more reliably trainable systems.

The review then transitions to \textbf{Section 4, "Architectural Innovations for Robustness and Scalability,"} which explores significant advancements in network design and training strategies that further enhanced GAN robustness and enabled high-resolution synthesis. This section details how methods like spectral normalization directly regularized network components and how multi-stage training paradigms, such as progressive growing, dramatically stabilized training and enabled the generation of high-resolution images. It also covers the integration of self-attention mechanisms and the scaling of GANs to large, complex datasets, demonstrating how ingenious architectural engineering pushed GANs beyond basic stability to achieve higher quality, more diverse, and larger-scale outputs.

Subsequently, \textbf{Section 5, "Advancing High-Fidelity and Controllable Generation,"} highlights a significant paradigm shift towards achieving unprecedented levels of image fidelity, resolution, and fine-grained control over generated content. This section primarily examines the StyleGAN series, which revolutionized generator architectures and training methodologies. These innovations enabled the disentanglement of latent space for intuitive manipulation of visual features at different scales and systematically addressed common artifacts. This progression showcases how, once foundational stability was largely achieved, the field pivoted towards engineering sophisticated models for pushing the boundaries of generative quality and semantic control, setting new benchmarks for synthetic image realism.

\textbf{Section 6, "Practical Considerations and Alternative Generative Paradigms,"} broadens the scope to address crucial practical limitations of GANs and explores alternative or hybrid generative modeling approaches that enhance stability and utility. It examines data-centric strategies, such as adaptive data augmentation, which are vital for achieving stable and high-quality GAN training in data-scarce environments. Furthermore, it delves into hybrid architectures, particularly VAE-GAN integrations, which combine the strengths of different generative models to overcome individual weaknesses, leading to more robust, diverse, and high-fidelity generation. This section underscores how the field moves beyond purely adversarial training to address real-world constraints and explore synergistic model designs.

The tangible impact of these advancements is showcased in \textbf{Section 7, "Applications of Stabilized Generative Adversarial Networks."} This section demonstrates the practical utility of robust GANs across various real-world domains, including medical imaging, 3D reconstruction, and signal processing. By highlighting how improved stability, quality, and control have enabled GANs to address critical challenges in diverse fields, this section bridges the gap between methodological innovations and their societal and scientific contributions, illustrating their growing role as indispensable tools in modern data science and engineering.

Finally, \textbf{Section 8, "Conclusion, Challenges, and Future Directions,"} synthesizes the key advancements in GAN stabilization, summarizing the intellectual trajectory from foundational theoretical fixes to sophisticated architectural and practical innovations. It critically examines unresolved tensions and persistent challenges that continue to face the field, including the trade-off between computational cost and performance, the elusive goal of universal stability, and the need for more robust theoretical guarantees for complex models \cite{goyal2024ufg}. The section concludes by discussing emerging trends, potential future research directions, and the crucial ethical considerations surrounding the development and deployment of increasingly powerful generative AI, emphasizing the ongoing quest for truly stable, controllable, and responsible generative models.

This structured approach ensures a clear understanding of the intellectual journey and the interconnectedness of various research contributions in the field of GAN stabilization, providing readers with a comprehensive and insightful overview of this dynamic research area.


\label{sec:fundamentals_of_gans_and_initial_instabilities}

\section{Fundamentals of GANs and Initial Instabilities}
\label{sec:fundamentals\_of\_gans\_\_and\_\_initial\_instabilities}

\subsection{The Original GAN Framework}
\label{sec:2\_1\_the\_original\_gan\_framework}

Generative modeling, the task of learning to produce new samples from the same distribution as a given training dataset, has long been a fundamental challenge in machine learning. Before the advent of Generative Adversarial Networks (GANs), approaches often relied on explicit density estimation or computationally intensive methods. The seminal work by Goodfellow et al. \cite{goodfellow2014generative} introduced a radically different paradigm, framing generative modeling as an adversarial game between two neural networks, which quickly became a cornerstone for unsupervised learning.

The core of the original GAN framework consists of two competing neural networks: a generator (G) and a discriminator (D). The generator, G, is a differentiable function that learns to map samples from a simple prior distribution, typically a uniform or Gaussian latent vector $z$, to the data space, producing synthetic data samples $G(z)$. Its objective is to generate samples that are indistinguishable from real data. Conversely, the discriminator, D, is a binary classifier that takes a data sample as input and outputs a single scalar representing the probability that the sample came from the real training data rather than from the generator. D's goal is to accurately distinguish between real data samples and the fake samples produced by G.

The training process of GANs is conceptualized as a minimax game, where G and D are simultaneously optimized in an adversarial fashion. The discriminator D is trained to maximize the probability of correctly classifying real data samples as real and generated samples as fake. In parallel, the generator G is trained to minimize D's ability to distinguish between real and fake samples, effectively trying to "fool" the discriminator into classifying its generated outputs as real. This dynamic creates a continuous feedback loop: as G improves its ability to generate realistic samples, D must become more sophisticated to detect fakes, and this improved D, in turn, provides a stronger learning signal for G to further refine its generation capabilities.

The original objective function, or value function $V(D, G)$, for this minimax game is defined as:
$$ \min\_G \max\_D V(D, G) = \mathbb{E}\_{x \sim p\_{data}(x)}[\log D(x)] + \mathbb{E}\_{z \sim p\_z(z)}[\log(1 - D(G(z)))] $$
Here, $p\_{data}(x)$ represents the true data distribution, and $p\_z(z)$ is the prior distribution over the input noise variables for the generator. The discriminator D aims to maximize this value function, while the generator G aims to minimize it. Goodfellow et al. \cite{goodfellow2014generative} theoretically demonstrated that at the optimal point of this adversarial game, the generator's distribution $p\_g$ converges to the true data distribution $p\_{data}$. At this equilibrium, the optimal discriminator $D^*(x)$ would output $0.5$ for all inputs, indicating that it can no longer differentiate between real and generated samples. The theoretical underpinnings further reveal that this objective function is equivalent to minimizing the Jensen-Shannon divergence between the data distribution and the generator's distribution, up to a constant term.

The initial promise of the original GAN framework was immense. It offered a novel and powerful approach to unsupervised learning, capable of implicitly learning complex, high-dimensional data distributions without requiring explicit density estimation. This opened up new possibilities for generating novel, high-fidelity samples, which was a significant advancement for tasks like image synthesis. However, despite its theoretical elegance and groundbreaking potential, the original GAN framework proved notoriously difficult to train in practice. Issues such as mode collapse, where the generator produces a limited variety of samples, and vanishing gradients, where the discriminator becomes too powerful too quickly and provides no useful learning signal to the generator, were prevalent. These inherent instabilities and training challenges made the original GAN a critical baseline, immediately highlighting the need for subsequent research into stabilization techniques, alternative objective functions, and architectural improvements that would define the next generation of generative models.

\bibliographystyle{plain}
\bibliography{references}
\subsection{Inherent Challenges: Mode Collapse and Training Instability}
\label{sec:2\_2\_inherent\_challenges:\_mode\_collapse\_\_and\_\_training\_instability}

Early Generative Adversarial Networks (GANs), despite their theoretical promise, were severely hampered by fundamental training difficulties that manifested as pervasive instability and a critical failure known as mode collapse \cite{jabbar2020aj0, wiatrak20194ib, chu2020zbv}. These issues stemmed directly from the adversarial training framework, where the generator (G) and discriminator (D) engage in a continuous minimax game, often leading to unstable dynamics, non-convergence, or oscillations during training rather than a stable Nash equilibrium \cite{grnarova20171tc}. Understanding these inherent challenges is crucial for appreciating the subsequent research trajectory aimed at GAN stabilization.

One of the most debilitating problems was \textit{mode collapse}, a phenomenon where the generator fails to capture the full diversity of the real data distribution \cite{jabbar2020aj0}. Instead of producing a wide variety of realistic outputs, the generator converges to a limited subset of the data's "modes," repeatedly generating similar, often high-quality, samples that represent only a fraction of the true data variability. This severely limited the practical utility of GANs for tasks requiring diverse sample generation, such as image synthesis or data augmentation. Mode collapse often arises when the generator finds a specific set of outputs that consistently fool the discriminator, and then exploits this weakness, ceasing to explore other parts of the data distribution. Conversely, if the discriminator becomes too strong too quickly, it might reject all generated samples, providing no useful gradient signal for the generator to learn from, thus causing the generator to collapse to a single, easily produced output. Early research quickly identified mode collapse as a central impediment, prompting immediate efforts to understand and mitigate this issue by encouraging the generator to explore the full data space \cite{metz20169ir, che2016kho}. The theoretical understanding of this challenge was further deepened by works highlighting a provable trade-off between the Lipschitz constant of generative networks and their ability to fit multimodal distributions, directly linking stability constraints to the challenge of capturing diversity \cite{salmona202283g}.

Beyond mode collapse, GANs also suffered from profound \textit{training instability}, primarily characterized by vanishing or exploding gradients, which made it exceedingly difficult for the networks to learn effectively or converge \cite{jabbar2020aj0, wiatrak20194ib}. The original GAN formulation, based on minimizing the Jensen-Shannon (JS) divergence between the real and generated data distributions, proved problematic. When the generator's distribution was far from the real data distribution—a common scenario early in training, especially in high-dimensional spaces—the JS divergence would often saturate. This saturation meant that the discriminator could easily distinguish real from fake samples, becoming "perfect" or near-perfect. In such cases, the gradients backpropagated to the generator would become vanishingly small, providing little to no learning signal and effectively halting the generator's progress \cite{arjovsky2017ze5}. Conversely, in some scenarios, gradients could explode, leading to erratic updates and divergence.

The adversarial nature of the training further exacerbated these instabilities. The continuous competition between the generator and discriminator often prevented the system from settling into a stable equilibrium. Instead, the training process would frequently oscillate, with one network overpowering the other, leading to a cycle of collapse and recovery, or simply non-convergence \cite{chu2020zbv}. This dynamic was also influenced by the unbounded function space of the discriminator, which, without proper constraints, could become overly confident and provide unreliable gradients \cite{chao2021ynq}. The inherent difficulty in finding a stable Nash equilibrium in this non-convex, high-dimensional game meant that training was highly sensitive to hyperparameter choices and initialization, making GANs notoriously fragile and challenging to deploy consistently \cite{bang2018ps8}.

These fundamental issues—mode collapse, vanishing/exploding gradients, and general training instability—resulted in poor quality generated samples, unreliable convergence, and a significant barrier to the widespread adoption of GANs. The recognition of these critical flaws underscored the urgent need for robust stabilization techniques, which became the primary focus of subsequent research. This foundational understanding of GAN failures directly motivated the development of alternative objective functions, novel regularization strategies, and architectural innovations, which are explored in the following sections, to transform GANs into reliably trainable and effective generative models.
\subsection{Quantitative Metrics for Evaluating GAN Stability and Quality}
\label{sec:2\_3\_quantitative\_metrics\_for\_evaluating\_gan\_stability\_\_and\_\_quality}

The inherent challenges of training Generative Adversarial Networks (GANs), including mode collapse, vanishing gradients, and training instability, necessitate robust quantitative metrics for objective evaluation of their stability, quality, and diversity \cite{jabbar2020aj0, wang2019w53}. These metrics provide a crucial framework for comparing different GAN architectures and assessing the effectiveness of various stabilization techniques.

Early efforts to quantify GAN performance moved beyond subjective visual inspection to more objective measures. The \textbf{Inception Score (IS)} emerged as one of the first widely adopted metrics \cite{salimans2016improved}. It assesses both the quality and diversity of generated images by leveraging a pre-trained Inception-v3 network. High IS values indicate that generated images are both semantically clear (low entropy of predicted class labels) and diverse (high entropy of marginal class probabilities). For instance, \cite{karras2017raw} utilized the Inception Score to demonstrate "unprecedented quality" and "increased variation" in their progressively grown GANs. While groundbreaking, IS has limitations, including its reliance on a classifier pre-trained on ImageNet, which may not generalize well to other datasets, and its sensitivity to image resolution.

To address some of the shortcomings of IS, the \textbf{Fréchet Inception Distance (FID)} was introduced as a more robust and widely accepted metric \cite{heusel2017gans}. FID measures the distance between the feature distributions of real and generated images, extracted from an intermediate layer of a pre-trained Inception-v3 network. By modeling these feature distributions as multivariate Gaussians, FID computes the Fréchet distance between them, providing a single score that correlates well with human perception of image quality and diversity. A lower FID score indicates better quality and diversity. FID has become the de facto standard for evaluating high-fidelity GANs, with architectural innovations like BigGAN \cite{brock2019} and the StyleGAN series \cite{karras2019, karras2020, karras2021} consistently reporting FID scores to demonstrate their state-of-the-art performance. Many stabilization techniques, such as those proposed by \cite{zadorozhnyy20208ft} with adaptive weighted discriminators and \cite{zhang2021ypi} with twin discriminators, explicitly aim to improve IS and FID, showcasing these metrics as key indicators of success.

Beyond overall quality and diversity, metrics for \textbf{mode coverage} are essential to diagnose and mitigate mode collapse, a common GAN failure where the generator produces only a limited subset of the true data distribution. While FID implicitly captures aspects of mode coverage, more explicit measures like precision and recall (though not explicitly detailed in the provided papers, they are widely used in the field) quantify how well the generator captures the full data distribution. The development of GANs like PresGAN \cite{dieng2019rjn} directly addresses mode collapse through an entropy regularizer, with evaluation focusing on mitigating this issue alongside perceptual quality. The ability of a GAN to fit multimodal distributions is also theoretically linked to its Lipschitz constant, highlighting a trade-off between stability and mode coverage \cite{salmona202283g}.

Finally, \textbf{convergence diagnostics} and measures of training stability are critical for assessing the effectiveness of stabilization techniques. While IS and FID evaluate the final output, monitoring training dynamics provides insights into the learning process itself. Techniques like Wasserstein GAN (WGAN) \cite{arjovsky2017ze5} and Least Squares GAN (LSGAN) \cite{mao2017ss0} were developed to provide more stable gradients and smoother loss landscapes, which are observed through less volatile training curves. Improved WGANs with gradient penalty (WGAN-GP) \cite{gulrajani2017improved} further enhanced stability, leading to better final image quality as measured by IS/FID. The work by \cite{xu2019uwg} models GAN training dynamics using control theory to directly improve stability, while \cite{gan202494y} and \cite{megahed2024c23} propose methods with learnable auxiliary modules and collaborative training, respectively, explicitly targeting training stability and evaluating their success through improved FID scores and reduced loss oscillations. The stability of the training process, often measured by the consistency of loss functions and the absence of exploding/vanishing gradients, is a direct diagnostic for the effectiveness of regularization techniques like spectral normalization \cite{miyato2018spectral} or various gradient penalties.

In conclusion, quantitative metrics such as Inception Score and Fréchet Inception Distance, alongside diagnostics for mode coverage and training stability, are indispensable for the rigorous evaluation and advancement of GAN research. They provide an objective basis for comparing the efficacy of different architectures and stabilization methods, guiding the field towards generators capable of producing increasingly high-quality, diverse, and stable outputs. However, ongoing challenges include the reliance on proxy metrics derived from pre-trained classifiers and the need for metrics that better align with nuanced aspects of human perception and specific application requirements.


\label{sec:foundational_stability_through_objective_function_and_gradient_regularization}

\section{Foundational Stability through Objective Function and Gradient Regularization}
\label{sec:foundational\_stability\_through\_objective\_function\_\_and\_\_gradient\_regularization}

\subsection{Wasserstein GANs: A New Distance Metric}
\label{sec:3\_1\_wasserstein\_gans:\_a\_new\_distance\_metric}

The initial formulation of Generative Adversarial Networks (GANs) by \textcite{goodfellow2014generative} encountered significant training challenges, primarily stemming from the properties of the Jensen-Shannon (JS) divergence employed in its objective function. When the real and generated data distributions had disjoint or non-overlapping supports—a common occurrence in high-dimensional data spaces, especially during early training stages—the JS divergence became constant. This led to vanishing gradients for the generator, effectively halting its learning process and preventing it from producing diverse samples, a phenomenon known as mode collapse \cite{arjovsky2017ze5, chu2020zbv}. Such instability underscored the urgent need for a more robust and continuous metric to measure the distance between probability distributions.

A pivotal innovation addressing these fundamental limitations was the introduction of Wasserstein GANs (WGANs) by \textcite{arjovsky2017ze5}. This seminal work proposed replacing the JS divergence with the Earth-Mover (EM) distance, also known as the Wasserstein-1 distance, as the metric for quantifying the discrepancy between the real and generated data distributions. The EM distance offers a crucial advantage: it provides a continuous and differentiable (almost everywhere) measure of distance between probability distributions, even when their supports are disjoint \cite{arjovsky2017ze5}. This continuity is paramount because it ensures that the loss function provides meaningful, non-zero gradients to the generator even when its output distribution is far from the real data, thereby directly mitigating the vanishing gradient problem that plagued original GANs. Consequently, WGANs offered a smoother loss landscape, allowing for more stable training and providing more informative gradients that guide the generator towards the true data distribution more effectively, thus contributing to alleviating mode collapse by encouraging broader exploration of the data manifold.

To theoretically guarantee that the discriminator—re-termed "critic" in WGANs—approximates the EM distance, its function must satisfy the 1-Lipschitz continuity condition \cite{arjovsky2017ze5}. This condition implies that the magnitude of the critic's gradients must be bounded by one, ensuring that the critic's output does not change too rapidly with respect to its input. The initial WGAN model enforced this Lipschitz constraint through a simple yet theoretically significant method: weight clipping \cite{arjovsky2017ze5}. This involved clamping the weights of the critic network to a small fixed range (e.g., $[-c, c]$) after each gradient update. From a theoretical standpoint, weight clipping attempts to bound the function space of the discriminator, which has been shown to be crucial for stabilizing training and achieving convergence \cite{chao2021ynq}. By directly limiting the magnitude of the weights, it aims to indirectly control the Lipschitz constant of the critic.

However, despite its theoretical appeal, the practical implementation of weight clipping introduced its own set of significant challenges and limitations. Firstly, the choice of the clipping range $c$ was highly sensitive and often required extensive hyperparameter tuning; an improperly chosen $c$ could severely impact performance. If $c$ was too small, it could drastically restrict the critic's capacity, forcing it to learn an overly simplistic function and preventing it from adequately distinguishing between real and fake samples. This under-capacity could lead to poor gradient signals for the generator. Conversely, if $c$ was too large, the Lipschitz constraint might not be effectively enforced, leading back to training instability \cite{arjovsky2017ze5}. Secondly, weight clipping often resulted in pathological gradient behavior, where gradients tended to concentrate at the boundaries of the clipping range. This phenomenon meant that the critic's weights would frequently be pushed to their maximum or minimum allowed values, leading to a "binary" or "sparse" critic that struggled to provide rich, informative gradients across the entire input space \cite{saqur2018oqp}. Such behavior could hinder the generator's ability to learn complex data distributions effectively, as the critic's feedback became less nuanced. The need for a more robust and principled method to enforce the Lipschitz constraint, without compromising the critic's expressive power or inducing pathological gradient dynamics, became evident.

In conclusion, the advent of Wasserstein GANs marked a profound theoretical shift in the understanding and training of generative adversarial networks. By leveraging the Earth-Mover distance, \textcite{arjovsky2017ze5} provided a theoretically sound framework for stable learning, offering a continuous loss function that yielded informative gradients even in challenging scenarios of disjoint distribution supports. The introduction of the 1-Lipschitz constraint and the initial attempt to enforce it via weight clipping were crucial steps towards stabilizing GAN training. However, the practical deficiencies of weight clipping, including its capacity limitations and pathological gradient behavior, highlighted the urgent need for a more principled and effective method to enforce the Lipschitz constraint without crippling the critic's expressive power. This challenge set the stage for subsequent advancements aimed at refining the stability mechanisms of Wasserstein GANs.
\subsection{Gradient Penalties for Lipschitz Continuity}
\label{sec:3\_2\_gradient\_penalties\_for\_lipschitz\_continuity}

The pursuit of stable and high-quality generative adversarial networks (GANs) has fundamentally relied on ensuring a well-behaved discriminator, primarily by enforcing its Lipschitz continuity. This mathematical constraint is crucial for the theoretical guarantees and practical efficacy of GANs, particularly those leveraging the Wasserstein distance. The seminal work on Wasserstein GANs (WGAN) \cite{arjovsky2017ze5} marked a significant paradigm shift by introducing the Earth Mover's (or Wasserstein) distance as a more stable and meaningful loss function compared to the Jensen-Shannon divergence used in original GANs. The theoretical framework of WGAN mandates that the discriminator (or critic) must be 1-Lipschitz, implying that its gradients must have a magnitude of at most one everywhere.

To enforce this 1-Lipschitz constraint, the original WGAN \cite{arjovsky2017ze5} employed a straightforward but ultimately problematic technique: weight clipping. This involved clamping the discriminator's weights within a small, fixed range (e.g., $[-c, c]$) after each update. As critically analyzed by \cite{gulrajani2017}, weight clipping suffered from several severe drawbacks. If the clipping range was too small, it drastically restricted the discriminator's capacity, leading to underfitting and hindering its ability to learn complex decision boundaries. Conversely, a large clipping range could allow gradients to explode, causing training instability and pathological behavior, often resulting in vanishing or exploding gradients \cite{jabbar2020aj0}. Furthermore, the optimal clipping value $c$ was a highly sensitive hyperparameter, difficult to tune and often dataset- and architecture-dependent.

A pivotal advancement that directly addressed these limitations was the introduction of gradient penalties, most notably in Improved Training of Wasserstein GANs (WGAN-GP) \cite{gulrajani2017}. WGAN-GP provided a robust and effective method for enforcing the Lipschitz constraint by directly penalizing the magnitude of the discriminator's gradients with respect to its input. Instead of clipping weights, WGAN-GP adds a penalty term to the discriminator's loss function that encourages the gradient norm to be close to one for samples interpolated between real and generated data points. This interpolation strategy ensures that the discriminator's gradients remain well-behaved across the entire input space, without sacrificing model capacity or introducing the pathological behaviors associated with weight clipping.

The innovation of gradient penalties in WGAN-GP profoundly improved training stability, sample quality, and convergence properties, establishing it as a widely adopted standard and a cornerstone for subsequent high-performance GAN architectures \cite{wang2019w53}. By maintaining a smoother loss landscape and providing more informative gradients to the generator, WGAN-GP enabled the stable training of deeper and more complex GAN models. Theoretical analyses, such as those by \cite{mescheder2018} and \cite{chu2020zbv}, further reinforced the efficacy of gradient penalties, demonstrating their role in ensuring convergence and stability in GAN training dynamics. These works highlight that a well-defined Lipschitz constant is crucial for the generator to effectively learn the data distribution, even when dealing with multimodal distributions \cite{salmona202283g}. The enhanced stability of WGAN-GP has also been demonstrated in diverse applications, such as in physics-informed GANs for solving stochastic differential equations, where it proved superior to vanilla GANs \cite{yang2018svo}.

Despite its significant advantages, WGAN-GP is not without its own considerations. The computation of gradient penalties introduces an additional computational overhead per training step, as it requires calculating second-order derivatives or approximations thereof. Moreover, WGAN-GP enforces the Lipschitz constraint by penalizing the gradient norm along straight lines between real and generated samples. While effective, this approach can be considered a stronger-than-necessary condition, as the true 1-Lipschitz constraint applies everywhere in the input space, not just on these interpolated paths. This might lead to over-regularization in certain regions of the input space. Recognizing these nuances, subsequent research continued to explore refinements and alternative methods for enforcing Lipschitz continuity or related discriminator constraints. For instance, \cite{chao2021ynq} proposed Constrained Generative Adversarial Networks (GAN-C) which introduce a direct constraint on the discriminator's output to stabilize training, sharing the goal of bounding the discriminator's function space. More recently, \cite{ni2024y70} introduced CHAIN (lipsCHitz continuity constrAIned Normalization), a novel normalization technique that integrates a Lipschitz continuity constraint into the scaling step, addressing gradient explosion and enhancing generalization in data-efficient GANs.

In conclusion, the transition from weight clipping to gradient penalties marked a critical maturation point for GAN training, fundamentally resolving the severe limitations of earlier regularization techniques. Gradient penalties, exemplified by WGAN-GP, provided a theoretically sound and practically effective mechanism for enforcing the Lipschitz constraint on the discriminator. This innovation not only laid the foundational stability necessary for the development of sophisticated and high-fidelity GAN architectures but also inspired a continuous line of research into more refined Lipschitz regularization techniques. This foundational concept of gradient regularization was later refined into techniques like the R1 and R2 penalties, which apply a zero-centered gradient penalty only to real data, proving crucial for the stability of subsequent state-of-the-art models like StyleGAN2 \cite{karras2020}. Future research continues to explore methods that balance strict theoretical guarantees with practical considerations like computational efficiency and the ability to model highly complex, multimodal data distributions.
\subsection{Alternative Loss Functions and Equilibrium Concepts}
\label{sec:3\_3\_alternative\_loss\_functions\_\_and\_\_equilibrium\_concepts}

The persistent instability of Generative Adversarial Networks (GANs), often manifested as mode collapse, vanishing gradients, and training oscillations, spurred significant research into objective functions and training dynamics beyond the original minimax game and the subsequent Wasserstein framework. These alternative approaches primarily aim to provide smoother loss landscapes, enforce specific architectural properties, or introduce explicit mechanisms to balance the adversarial training process, thereby improving stability and sample quality.

One of the earliest and most straightforward modifications to the GAN objective was introduced by Least Squares Generative Adversarial Networks (LSGANs) \cite{Mao2017}. LSGANs replace the traditional sigmoid cross-entropy loss with a least squares loss for both the discriminator and the generator. This change yields two primary benefits: first, it provides smoother gradients, particularly for samples far from the decision boundary, mitigating the vanishing gradient problem that plagues sigmoid-based losses when the discriminator becomes too confident. Second, \cite{Mao2017} theoretically demonstrated that minimizing the LSGAN objective corresponds to minimizing the Pearson $\chi^2$ divergence, a measure of statistical distance that offers more stable gradient signals compared to the Jensen-Shannon divergence implicitly minimized by original GANs. This direct mathematical connection to a well-behaved divergence contributed significantly to LSGANs' improved training stability and ability to generate higher quality images.

Moving beyond direct loss function replacements, architectural designs for the discriminator coupled with novel training objectives emerged as a powerful avenue. Energy-based Generative Adversarial Networks (EBGANs) \cite{Zhao2016} conceptualize the discriminator as an energy function. In this framework, the discriminator, often implemented as an autoencoder, learns to assign low energy to real data samples and high energy to generated (fake) data. The generator is then trained to produce samples that minimize this energy, effectively guiding it towards the real data distribution. This approach stabilizes training by focusing on the discriminator's ability to reconstruct real data accurately (low reconstruction error, thus low energy) while failing to reconstruct fake data (high reconstruction error, thus high energy). This mechanism helps mitigate mode collapse by encouraging the generator to produce diverse samples that the autoencoder discriminator cannot easily reconstruct as "real."

Building upon the autoencoder discriminator concept, Boundary Equilibrium Generative Adversarial Networks (BEGAN) \cite{Berthelot2017} further refined this idea by focusing on matching the autoencoder loss distributions for real and generated images. Crucially, BEGAN introduced an explicit \textit{equilibrium concept} through a diversity control parameter ($\gamma$). This parameter dynamically balances the generator and discriminator's training progress by adjusting the weight of the reconstruction loss for generated samples. By controlling this balance, BEGAN allows for a principled trade-off between image diversity and quality. This dynamic adjustment mechanism, based on the ratio of the discriminator's autoencoder loss for real versus fake samples, significantly improved visual quality and training stability for its time, demonstrating the power of an explicit equilibrium strategy to prevent either network from overpowering the other.

The broader theoretical landscape of GAN objective functions can be understood through the lens of \textit{f-divergences}. Many GAN variants implicitly or explicitly minimize different f-divergences, which are a class of statistical distances that generalize measures like Kullback-Leibler (KL) divergence, Jensen-Shannon (JS) divergence, and Pearson $\chi^2$ divergence. For instance, as noted, LSGANs correspond to Pearson $\chi^2$ divergence \cite{Mao2017}. Further exploring this framework, Alpha-GAN \cite{Cai2020AlphaGAN} adopts the alpha divergence as its minimization objective. Alpha divergence offers a flexible generalization of several common divergences, allowing for a tunable parameter ($\alpha$) to balance training stability and generated image quality. Similarly, LeCam-GAN \cite{Tseng2021LeCamGAN} proposes a regularization approach that theoretically connects the regularized loss to the LeCam-divergence, which is found to be more robust under limited training data. This highlights how selecting appropriate f-divergences can fundamentally alter the GAN's learning dynamics and stability properties.

Another prominent alternative loss function that gained widespread adoption is the Hinge Loss. Initially explored in variants like Margin Adaptation for Generative Adversarial Networks (MAGAN) \cite{Wang2017MAGAN}, Hinge Loss provides a clear margin for the discriminator's classification, penalizing it only when its prediction for real samples falls below a positive margin or its prediction for fake samples rises above a negative margin. This loss function is particularly effective because it provides strong, non-saturating gradients when the discriminator is performing poorly but yields zero gradients once the discriminator achieves a sufficient margin. This piecewise linear nature helps stabilize training by preventing the discriminator from becoming overly confident and thus providing useful gradients to the generator for a longer period, making it a robust choice for many high-performance GAN architectures.

Further innovations in objective functions and training dynamics include Relativistic GANs, where the discriminator estimates the probability that \textit{real data is more realistic than fake data}, or vice-versa, rather than just classifying samples as real or fake independently. This relativistic approach, exemplified by methods like DRL-GAN \cite{Roy2024DRLGAN}, can provide more informative gradient signals to the generator, contributing to improved stability and sample quality by directly comparing real and fake distributions. Other approaches focus on explicit regularization of the objective to prevent specific failure modes. For instance, Mode Regularized Generative Adversarial Networks \cite{Che2016ModeRegGAN} introduce regularizers to the objective function to encourage the fair distribution of probability mass across data modes, directly addressing mode collapse. Similarly, Robust Generative Adversarial Networks (RGAN) \cite{Zhang2019RobustGAN} propose a worst-case adversarial training setting, where both generator and discriminator compete within a small Wasserstein ball, aiming to improve generalization and stability by making the networks robust to perturbations. The theoretical underpinnings of such stability improvements, often linking to properties like Lipschitz continuity and the choice of divergence, have been further explored by works like \cite{Chu2020Smoothness}, which provides a principled framework for understanding GAN stability.

In summary, the quest for stable GAN training has led to a rich diversity of alternative loss functions and equilibrium concepts. From the simpler, yet effective, least squares loss \cite{Mao2017} and the energy-based autoencoder discriminator of EBGAN \cite{Zhao2016}, to the explicit equilibrium parameter of BEGAN \cite{Berthelot2017}, these methods offered distinct strategies. The f-divergence framework provides a unifying theoretical lens, explaining how different choices (e.g., Pearson $\chi^2$, Alpha-divergence, LeCam-divergence) impact training dynamics. The widespread adoption of Hinge Loss further underscores the value of non-saturating, margin-based objectives. While these approaches have significantly improved GAN stability and sample quality, they often introduce new hyperparameters (e.g., $\gamma$ in BEGAN, $\alpha$ in Alpha-GAN) that require careful tuning, and their theoretical guarantees can be complex. Nevertheless, these foundational innovations laid crucial groundwork, demonstrating that stability could be achieved not only through gradient regularization (as in WGAN-GP) but also through fundamentally rethinking the adversarial objective and the balance between the generator and discriminator. This paved the way for more sophisticated architectural and training strategy innovations.


\label{sec:architectural_innovations_for_robustness_and_scalability}

\section{Architectural Innovations for Robustness and Scalability}
\label{sec:architectural\_innovations\_for\_robustness\_\_and\_\_scalability}

\subsection{Spectral Normalization for Discriminator Stability}
\label{sec:4\_1\_spectral\_normalization\_for\_discriminator\_stability}

The quest for stable and high-performance Generative Adversarial Networks (GANs) has driven significant innovation in regularization techniques. While earlier methods, such as the gradient penalties employed in Wasserstein GAN with Gradient Penalty (WGAN-GP) \cite{gulrajani2017}, effectively enforced the Lipschitz constraint on the discriminator, they often incurred substantial computational overhead due to the necessity of second-order gradient computations. This computational burden motivated the search for more efficient and direct approaches to stabilize discriminator training.

A pivotal advancement in this regard was the introduction of Spectral Normalization (SN) by Miyato et al. \cite{miyato2018arc}. Unlike gradient-based methods, SN directly regularizes the weights of each convolutional or fully-connected layer within the discriminator network. The core mechanism involves normalizing the spectral norm of each weight matrix to be less than or equal to one. The spectral norm of a matrix $W$ is its largest singular value, $\sigma(W)$. By dividing $W$ by its spectral norm, i.e., $\hat{W} = W / \sigma(W)$, SN ensures that the Lipschitz constant of each layer is at most one. This property, when applied across all layers, effectively enforces a 1-Lipschitz constraint on the entire discriminator function, a crucial condition for stable GAN training and meaningful gradient signals, as theoretically elucidated by works like \cite{mescheder2018} and \cite{chu2020zbv}. Practically, the spectral norm is efficiently approximated using a few steps of the power iteration method during each training iteration, making SN computationally light and easy to integrate into existing GAN architectures without requiring explicit gradient computations for regularization.

The computational efficiency and robust stabilizing effect of Spectral Normalization quickly led to its widespread adoption across various high-performance GAN architectures. Its ability to control the discriminator's Lipschitz constant without complex gradient calculations made it a versatile tool for improving both training stability and the quality of generated samples. A prime example of SN's impact is its integral role in BigGAN \cite{brock2019}. BigGAN leveraged SN as a crucial component in its discriminator design, alongside other innovations like self-attention mechanisms and shared embeddings, to achieve unprecedented levels of fidelity and diversity in large-scale natural image synthesis on datasets like ImageNet. This demonstrated that SN could effectively scale to complex, high-dimensional data, becoming a standard component for ensuring discriminator stability in state-of-the-art generative models.

While SN provided a foundational method for discriminator stability, its effectiveness is often enhanced when combined with other regularization strategies. For instance, Zhang et al. \cite{zhang2019hjo} demonstrated that Consistency Regularization, which penalizes the discriminator's sensitivity to data augmentations, works effectively \textit{with} spectral normalization. This highlights SN's role not just as a standalone solution, but as a compatible and often essential building block within more sophisticated regularization schemes designed to further improve GAN performance and robustness. The ongoing relevance of Lipschitz constraints in balancing stability with generative capacity, including discussions on potential trade-offs, continues to be a topic of theoretical and practical importance \cite{salmona202283g}.

In summary, Spectral Normalization represents a significant paradigm shift in GAN regularization, offering a computationally efficient and direct method to enforce the Lipschitz constraint on the discriminator. By normalizing the spectral norm of weight matrices, SN effectively stabilizes training, mitigates issues like vanishing gradients and mode collapse, and contributes substantially to the generation of high-quality samples. Its integration into leading architectures like BigGAN underscores its foundational importance, establishing SN as a simple, yet powerful, tool that continues to underpin robust and high-fidelity generative adversarial models.
\subsection{Progressive Training for High-Resolution Synthesis}
\label{sec:4\_2\_progressive\_training\_for\_high-resolution\_synthesis}

The pursuit of generating high-fidelity, photo-realistic images with Generative Adversarial Networks (GANs) has long been hampered by significant training instabilities and the inherent difficulty of synthesizing complex details at high resolutions. Early GAN models frequently suffered from issues such as mode collapse, where the generator produces a limited variety of outputs, and vanishing or exploding gradients, which impede stable learning \cite{wang2019w53, wiatrak20194ib, jabbar2020aj0}. While various regularization techniques and alternative loss functions were proposed to mitigate these problems, such as unrolled GANs \cite{metz20169ir}, mode regularization \cite{che2016kho}, and Least Squares GANs \cite{mao2017ss0}, achieving stable training at resolutions beyond 256x256 pixels remained a formidable challenge.

Prior to the advent of progressive training, some approaches attempted to tackle higher resolutions by decomposing the generation process into multiple stages. For instance, StackGAN \cite{zhang2016mm0} employed a stacked architecture where a Stage-I GAN generated low-resolution images based on text descriptions, and a subsequent Stage-II GAN refined these into higher-resolution, photo-realistic outputs. This method, while innovative, relied on a cascaded design of separate networks for different resolutions, which could be complex to coordinate and optimize.

A groundbreaking paradigm shift arrived with the introduction of Progressive Growing of GANs (PGGAN) by Karras et al. \cite{karras2017raw}. This method fundamentally re-imagined the training process by adopting a curriculum learning approach: instead of training a full-resolution GAN from scratch, PGGAN begins by generating very small, low-resolution images (e.g., 4x4 pixels). As training progresses, new layers are incrementally added to both the generator and discriminator networks, gradually increasing the output resolution and the complexity of learned features. For example, after training at 4x4, new layers are smoothly faded in to enable 8x8 generation, and this process continues up to resolutions like 1024x1024 pixels for datasets such as CelebA-HQ \cite{karras2017raw}.

This progressive growth strategy offered several critical advantages. By starting with simpler tasks, PGGAN significantly improved training stability, as the networks first learned broad structures before focusing on fine details. This gradual introduction of complexity also drastically reduced the problem of mode collapse, allowing the generator to capture a wider diversity of the data distribution. Furthermore, the method accelerated training convergence, as early stages could be learned quickly, providing a robust foundation for subsequent, higher-resolution layers. The ability to produce unprecedentedly high-quality and diverse images at resolutions previously unattainable, such as 1024x1024, underscored PGGAN's pivotal contribution to the field \cite{karras2017raw}. The authors also incorporated techniques to increase variation in generated images and detailed implementation specifics to prevent unhealthy competition between the generator and discriminator, further enhancing stability.

The success of PGGAN demonstrated that the training process itself, when structured intelligently, could be a powerful tool for stabilization and performance enhancement, rather than solely relying on architectural modifications or loss function changes. While PGGAN introduced the progressive growing strategy, subsequent advancements in GANs often integrated this approach with other stabilization techniques. For instance, Spectral Normalization (SN) \cite{miyato2018arc}, which constrains the Lipschitz constant of discriminator layers, became a widely adopted method for improving training stability and was frequently combined with progressive growing in later high-fidelity GAN architectures, as seen in works like BigGAN and StyleGAN. This synergistic combination of progressive training with robust regularization further pushed the boundaries of image synthesis quality and stability \cite{qin2024a4b}.

In conclusion, Progressive Growing of GANs marked a significant turning point in generative modeling, moving beyond mere architectural scaling to introduce a dynamic training methodology. By progressively increasing resolution and complexity, PGGAN effectively addressed long-standing issues of training instability and mode collapse, paving the way for the generation of high-resolution, diverse, and visually compelling images. However, despite its success, the computational cost associated with training such large models, especially at very high resolutions, remains a challenge, and the continuous pursuit of even greater fidelity and disentanglement continues to drive research in this area.
\subsection{Self-Attention Mechanisms for Global Coherence}
\label{sec:4\_3\_self-attention\_mechanisms\_for\_global\_coherence}

Traditional Generative Adversarial Networks (GANs) primarily relied on convolutional layers, which are inherently effective at capturing local spatial correlations. However, a significant limitation of these architectures was their struggle to model long-range dependencies across distant regions of an image. This inductive bias often led to generated images that exhibited strong local realism but lacked global structural coherence, particularly evident in complex scenes with intricate object relationships or expansive backgrounds. Addressing this fundamental challenge, the integration of self-attention mechanisms marked a pivotal architectural innovation, enabling GANs to effectively capture relationships between features at arbitrary distances. This mechanism, originally popularized in Transformer architectures for natural language processing, provided a flexible way to model dependencies without the fixed receptive field constraints of convolutional kernels.

The seminal work by \cite{Zhang2019} introduced Self-Attention Generative Adversarial Networks (SAGANs), fundamentally transforming how GANs model global dependencies. SAGANs incorporated self-attention modules into both the generator and discriminator networks. In the generator, this mechanism allowed each generated feature location to attend to all other locations across the feature map, thereby synthesizing details that are consistent with distant parts of the image. For instance, when generating a complex scene like a landscape, self-attention could ensure that a distant mountain range is appropriately scaled and textured relative to a foreground river, maintaining global visual consistency. Concurrently, the discriminator leveraged self-attention to evaluate the global consistency of the generated images, assessing whether disparate elements of a scene logically cohere and adhere to the overall structure of real images. This dual application of self-attention significantly enhanced the networks' representational power, leading to a marked improvement in the quality and structural integrity of generated images, particularly for intricate and diverse visual content. However, the introduction of self-attention also brought increased computational and memory requirements, especially for high-resolution images, a trade-off noted by \cite{Zhang2019} who also explored the optimal placement of attention layers, finding them most effective in deeper layers of the network.

Building upon the foundation laid by SAGAN, self-attention mechanisms quickly became a vital component in subsequent high-fidelity generative models, demonstrating their versatility beyond purely image generation. For instance, \cite{cai2019g1w} proposed DualAttn-GAN for text-to-image synthesis, which employed dual attention modules to enhance both local details and global structures by attending to relevant words and different visual regions. This highlighted how attention could be used to bridge modalities and ensure semantic consistency across complex generative tasks. Similarly, \cite{xue2022n0r} integrated self-attention into Phased Evolutionary Generative Adversarial Networks (PEGANs) to improve training stability and enhance the modeling of long-range dependencies, directly addressing the limitations of convolutional operations and common GAN instabilities like mode collapse. These works underscore that self-attention was not merely a theoretical improvement but a practical necessity for achieving state-of-the-art results in various generative tasks by allowing models to obtain long-range dependency modeling abilities.

The effectiveness of self-attention in fostering global coherence was further underscored by its integration into large-scale GAN architectures. For example, \cite{Brock2019} incorporated self-attention, alongside other advancements like spectral normalization and hinge loss, into their Large Scale GAN Training for High Fidelity Natural Image Synthesis (BigGAN). While a detailed discussion of BigGAN is reserved for Section 4.4, its remarkable success in generating highly realistic and diverse images across a wide range of classes was partly attributable to the self-attention modules, which allowed the model to maintain global consistency even at very high resolutions and across complex compositions. This demonstrated how self-attention, originally proposed to address a specific architectural limitation, became an indispensable tool for pushing the boundaries of generative model performance.

In conclusion, the introduction of self-attention mechanisms, particularly through SAGANs, represented a pivotal advancement in GAN research by directly tackling the challenge of modeling global coherence. By enabling both generators to synthesize and discriminators to evaluate long-range dependencies, self-attention significantly elevated the quality and structural integrity of generated images, moving beyond the limitations of purely local convolutional processing. While self-attention has proven highly effective, its quadratic complexity with respect to feature map size remains a significant computational and memory bottleneck, especially in very high-resolution applications. This has prompted ongoing research into more efficient self-attention variants, such as linear-time approximations, sparse attention patterns, or differentiable architecture search methods like those proposed by \cite{xue20248md} (DAMGAN), which aim to optimize attention placement and limit computational cost during the search for stable GAN architectures. These efforts seek to maintain global modeling capabilities at higher resolutions while optimizing computational resources, thereby continuing to push the boundaries of realistic and efficient image generation.
\subsection{Scaling Up: Large-Scale GAN Training}
\label{sec:4\_4\_scaling\_up:\_large-scale\_gan\_training}

The ambition to generate highly realistic and diverse images across complex, large-scale datasets, such as ImageNet, represented a formidable challenge for Generative Adversarial Networks (GANs). Achieving this required not only overcoming inherent training instabilities but also developing architectural paradigms capable of managing immense model complexity and computational demands \cite{wang2019w53, purwono2025spz}. The foundational breakthroughs in GAN stability, including the more robust objective functions provided by Wasserstein GAN with Gradient Penalty (WGAN-GP) \cite{Gulrajani2017} and the efficient Lipschitz constraint enforcement of Spectral Normalization (SN) \cite{Miyato2018} (as discussed in Sections 3.2 and 4.1, respectively), laid critical groundwork. Furthermore, the Progressive Growing of GANs (PGGAN) \cite{Karras2018} (Section 4.2) demonstrated that a curriculum learning approach could enable stable synthesis of high-resolution images, setting the stage for models to explore the potential of sheer scale. These advancements collectively provided the necessary stability and architectural insights that allowed researchers to push the boundaries of GAN performance on demanding datasets.

A pivotal moment in this scaling trajectory was the introduction of BigGAN \cite{Brock2019}, which dramatically elevated the state-of-the-art in image fidelity and diversity on the challenging ImageNet dataset. BigGAN demonstrated that with substantial computational resources, large batch sizes, and meticulous engineering, GANs could generate remarkably realistic and varied images across a vast array of categories. This success was not merely a matter of increasing model size; it was a carefully orchestrated combination of architectural innovations and advanced regularization techniques designed to maintain stability and performance at an unprecedented scale \cite{liu2020jt0}.

Architecturally, BigGAN integrated the self-attention mechanism, a concept previously shown to be effective for modeling long-range dependencies in images by Self-Attention Generative Adversarial Networks (SAGAN) \cite{Zhang2019} (Section 4.3). By incorporating self-attention into both the generator and discriminator, BigGAN enabled the networks to capture global coherence and intricate relationships between distant parts of an image, which is crucial for synthesizing complex scenes with consistent structures. This adaptation of self-attention to a larger scale, alongside the use of shared embeddings for class conditioning, allowed BigGAN to leverage information across different classes more effectively, significantly enhancing sample diversity and the model's ability to generate distinct categories of images. The effectiveness of self-attention in improving image generation quality and stability has been further corroborated in subsequent works, such as SARA-GAN, which also utilized self-attention to build long-range dependencies \cite{yuan2020bt6}.

Beyond architectural enhancements, BigGAN introduced several key regularization techniques specifically tailored to maintain stability and enhance performance when operating at such a massive scale. One critical innovation was orthogonal regularization applied to the generator's weight matrices. This technique was introduced to combat the specific training collapse phenomena observed when scaling GANs to very large models and batch sizes, preventing gradient pathologies and encouraging a more stable learning process \cite{Brock2019}. This aligns with broader research efforts in stabilizing GANs through regularization, as explored by works like \cite{che2016kho} and \cite{roth2017eui}. Another significant contribution was the "truncation trick," which provided a practical mechanism to trade off sample fidelity for diversity. By adjusting the sampling range of the latent space, the truncation trick allowed for the generation of exceptionally high-quality, albeit sometimes less diverse, images, offering a valuable control knob for practitioners. Crucially, BigGAN built upon established stability principles, notably incorporating Spectral Normalization \cite{Miyato2018} within its discriminator. This ensured stable training even with the significantly increased model capacity and large batch sizes, demonstrating a synergistic approach where prior stability mechanisms were essential enablers for scaling.

The synergistic integration of these advancements—large batch sizes, self-attention, shared class embeddings, orthogonal regularization, and the truncation trick—allowed BigGAN to achieve unprecedented Inception Scores and Fréchet Inception Distances on ImageNet. This landmark work unequivocally demonstrated that scaling up GANs, when coupled with careful engineering and a deep understanding of training dynamics, was a viable and powerful path to state-of-the-art image synthesis \cite{wiatrak20194ib, jabbar2020aj0}. However, this monumental achievement came with a substantial computational cost, demanding extensive GPU resources and large batch sizes for training. This trade-off between performance and computational expense highlighted a significant challenge for future research: making such high-fidelity generative models more computationally efficient and accessible to a wider range of researchers and applications.


\label{sec:advancing_high-fidelity_and_controllable_generation}

\section{Advancing High-Fidelity and Controllable Generation}
\label{sec:advancing\_high-fidelity\_\_and\_\_controllable\_generation}

\subsection{The Style-Based Generator Architecture (StyleGAN)}
\label{sec:5\_1\_the\_style-based\_generator\_architecture\_(stylegan)}

The pursuit of high-fidelity image synthesis with intuitive control over visual features has been a central challenge in generative modeling. While early Generative Adversarial Networks (GANs) struggled with training stability and generating diverse, high-resolution images, the StyleGAN architecture emerged as a revolutionary advancement, significantly pushing the boundaries of what was achievable \cite{karras2019}. This breakthrough built upon foundational work in stable GAN training and architectural innovations, particularly the concept of progressive growing.

A crucial precursor to StyleGAN was the Progressive Growing of GANs (PGGAN) \cite{karras2017raw}. PGGAN introduced a novel training methodology where both the generator and discriminator progressively grow in resolution, starting from low-resolution images and gradually adding layers to model increasingly fine details. This approach not only stabilized training significantly but also enabled the generation of unprecedentedly high-resolution and high-quality images, such as $1024^2$ pixel celebrity faces. However, even with PGGAN, the latent space often remained entangled, making it difficult to control specific visual attributes without affecting others.

The original StyleGAN \cite{karras2019} addressed this limitation by fundamentally rethinking the generator's architecture, introducing a "style-based" design that decoupled the latent space from image features. Instead of feeding a latent code directly to the generator's first layer, StyleGAN's generator begins with a learned constant input. The core innovation lies in its use of a \textbf{mapping network}, a small multi-layer perceptron, which transforms an initial, unstructured latent code $z$ into an intermediate latent space of 'style' vectors $w$. This intermediate $w$ space is designed to be less entangled than $z$, making it easier to linearly interpolate and manipulate. These 'style' vectors $w$ are then injected into the generator at different scales (resolutions) through \textbf{Adaptive Instance Normalization (AdaIN)} layers. AdaIN normalizes the feature maps for each channel and then scales and biases them using learned parameters derived from the style vector $w$. This mechanism allows the style vector to control the statistical properties (mean and variance) of the feature maps, thereby dictating the visual style at that particular resolution.

This unique style-based design enables intuitive, scale-specific control over visual features. Coarse 'styles' injected at lower resolutions influence fundamental structural elements like a subject's pose, identity, or overall facial shape. Conversely, 'styles' injected at higher resolutions control finer details such as hair color, skin texture, or lighting effects. By disentangling these aspects, StyleGAN allowed for highly realistic and editable image generation, where users could manipulate specific attributes without unintended side effects. The architecture's ability to decouple latent space from image features resulted in a more linear and disentangled representation, a significant step towards controllable synthesis.

Subsequent iterations further refined this architecture. StyleGAN2 \cite{karras2020} aimed to analyze and improve the image quality by addressing several artifacts present in the original StyleGAN, such as the "blob" artifacts often seen in generated images. Key improvements included removing the progressive growing's fading mechanism, redesigning the AdaIN operation with weight demodulation, and introducing \textbf{path length regularization}. This regularization technique encouraged a more consistent mapping from the latent space to the image space, leading to even better disentanglement and significantly higher perceptual quality.

The latest iteration, StyleGAN3 \cite{karras2021}, represented a significant theoretical advancement by tackling the issue of aliasing. Previous StyleGAN models, despite their high quality, suffered from subtle aliasing artifacts that became apparent during transformations like rotation or translation. StyleGAN3 re-architected the generator to be \textbf{alias-free}, incorporating anti-aliasing filters at every upsampling step. This design choice ensured that the generated images were truly continuous and equivariant to transformations, leading to unprecedented consistency and realism, particularly in video synthesis or when applying geometric transformations to static images.

In conclusion, the StyleGAN series represents a monumental leap in generative adversarial networks. By introducing a style-based generator, a mapping network, and Adaptive Instance Normalization, StyleGAN \cite{karras2019} revolutionized the control and disentanglement of latent spaces, enabling intuitive manipulation of visual features across different scales. The subsequent refinements in StyleGAN2 \cite{karras2020} and StyleGAN3 \cite{karras2021} progressively enhanced image quality, mitigated artifacts, and addressed fundamental signal processing issues, solidifying StyleGAN's position as a cornerstone in high-fidelity and controllable image synthesis. While these models demand significant computational resources and large datasets, their architectural innovations have profoundly influenced the field, setting new benchmarks for realism and editability in generated imagery.
\subsection{Refinements for Quality and Artifact Reduction (StyleGAN2)}
\label{sec:5\_2\_refinements\_for\_quality\_\_and\_\_artifact\_reduction\_(stylegan2)}

The pursuit of generating highly realistic and controllable images with Generative Adversarial Networks (GANs) has been significantly advanced by the StyleGAN architecture. While the original StyleGAN \cite{Karras2019} revolutionized image synthesis by introducing a style-based generator that enabled intuitive control over visual features and disentangled latent representations, it also exhibited several persistent visual artifacts, such as 'water droplet' textures and an inconsistent mapping from latent codes to images. Addressing these limitations became crucial for further progress, leading to the development of StyleGAN2 \cite{Karras2020}.

The foundational StyleGAN model \cite{Karras2019} built upon the progressive growing methodology introduced in PGGAN \cite{karras2017raw}, which gradually increases network complexity and resolution during training to stabilize the process and achieve high-resolution outputs. StyleGAN enhanced this by incorporating a mapping network to transform latent codes into an intermediate style representation, which then modulated the feature maps of the generator via Adaptive Instance Normalization (AdaIN) layers. This design successfully disentangled different levels of visual features, allowing for fine-grained control. However, the reliance on progressive growing and the specific implementation of AdaIN introduced undesirable side effects. For instance, the progressive growing approach, while stabilizing training, could lead to resolution-specific artifacts, and the AdaIN layers were found to inadvertently contribute to signal-to-noise ratio issues and the characteristic 'water droplet' artifacts, where generated textures appeared to stick to specific coordinates.

StyleGAN2 \cite{Karras2020} meticulously identified and addressed these shortcomings through a series of architectural and regularization refinements, significantly boosting overall image quality and stability. One of the most prominent changes was the \textbf{removal of progressive growing}. Instead of gradually adding layers, StyleGAN2 trains a fixed-size generator from the outset, which helps to eliminate resolution-specific artifacts and simplifies the training pipeline. To compensate for the stability benefits lost by removing progressive growing, StyleGAN2 redesigned the normalization layers. The original AdaIN was replaced with a \textbf{demodulation mechanism} applied directly to the convolutional weights. This modification ensures that the magnitude of feature maps remains consistent across different styles, effectively preventing the 'water droplet' artifacts and improving the signal-to-noise ratio by avoiding the introduction of unwanted high-frequency noise.

Beyond architectural changes, StyleGAN2 introduced a novel \textbf{path length regularization} technique. This regularization encourages a more consistent and smooth mapping from the latent space to the image space. Specifically, it penalizes the deviation of the generator's output from the ideal behavior where a fixed-size step in latent space corresponds to a fixed-size change in the generated image. By ensuring that the magnitude of the Jacobian of the generator mapping is consistent, path length regularization improves latent space disentanglement and prevents abrupt changes in generated images for small latent code perturbations. This leads to a more predictable and interpretable latent space, which is critical for applications requiring fine-grained control over image attributes.

The cumulative effect of these refinements in StyleGAN2 \cite{Karras2020} was a substantial improvement in image fidelity, a marked reduction in visual artifacts, and enhanced training stability. These advancements further solidified the StyleGAN architecture as a benchmark for state-of-the-art image synthesis, demonstrating that careful analysis of architectural components and the introduction of targeted regularization can overcome persistent challenges in GAN training. Subsequent work, such as StyleGAN3 \cite{Karras2021}, continued this trajectory of refinement by addressing aliasing issues in the generator, highlighting the continuous effort to achieve truly alias-free and perfectly realistic image generation. Despite these significant strides, the computational cost and substantial data requirements for training such high-fidelity models remain a common limitation, posing ongoing challenges for broader accessibility and application.
\subsection{Alias-Free Synthesis and Equivariance (StyleGAN3)}
\label{sec:5\_3\_alias-free\_synthesis\_\_and\_\_equivariance\_(stylegan3)}

The pursuit of photorealistic image synthesis through Generative Adversarial Networks (GANs) has continuously pushed the boundaries of visual fidelity, yet subtle artifacts, often rooted in signal processing limitations, persisted as a challenge. The latest iteration in the StyleGAN series, commonly referred to as StyleGAN3, represents a profound theoretical and practical refinement, specifically tackling the fundamental issue of aliasing artifacts to achieve alias-free synthesis and enhanced equivariance.

The journey towards high-fidelity image generation was significantly advanced by methods like Progressive Growing of GANs (PGGAN) \cite{karras2017raw}, which enabled the training of GANs for unprecedented resolutions by gradually increasing network complexity during the training process. Building on this foundation, the StyleGAN architecture \cite{Karras2019} revolutionized generator design by introducing a style-based approach that decoupled latent space representations, allowing for intuitive and disentangled control over various visual features through adaptive instance normalization. While StyleGAN marked a substantial leap in controllability and perceptual quality, it exhibited certain visual artifacts.

These issues were subsequently addressed and refined in StyleGAN2 \cite{Karras2020}, which meticulously analyzed and mitigated several common artifacts, such as the "blob" artifact, and introduced path length regularization to improve the linearity and disentanglement of the latent space. This led to a notable improvement in image quality and consistency. However, even with these advancements, a new class of subtle signal processing artifacts became apparent, especially when generated images were subjected to geometric transformations like translation or rotation. These artifacts, often manifesting as "texture sticking" or "aliasing," indicated that the underlying network architecture was not truly operating on continuous signals.

StyleGAN3 \cite{Karras2021} emerged as a direct response to these persistent aliasing artifacts, approaching the problem from a rigorous signal processing perspective. The core innovation lies in its alias-free architecture, which fundamentally redesigns the generative process to operate on continuous signals rather than discrete pixel grids. This is achieved by incorporating carefully designed anti-aliasing filters directly into the upsampling path of the generator. By ensuring that each upsampling operation is properly band-limited, StyleGAN3 effectively prevents the introduction of high-frequency components that would otherwise lead to aliasing when the image is sampled or transformed.

This architectural shift has profound implications for the robustness of generated images to transformations. By operating on continuous representations, the model inherently gains a high degree of equivariance, meaning that transformations applied to the latent space correspond to predictable and smooth transformations in the generated image space. For instance, translating or rotating a generated image no longer reveals fixed, grid-like artifacts or "texture sticking" that were characteristic of previous models. Instead, the textures and details deform and move realistically, reflecting a more robust and coherent understanding of spatial relationships. This focus on equivariance and artifact reduction pushed the boundaries of perceptual quality, leading to even more realistic textures, finer details, and a heightened sense of visual coherence, particularly in challenging scenarios involving movement or changes in perspective.

In conclusion, StyleGAN3 represents a mature understanding of image synthesis, moving beyond mere pixel generation to address the underlying signal processing principles that govern visual realism. By tackling subtle yet pervasive aliasing issues through an alias-free architecture and continuous signal processing, it significantly enhanced the equivariance and robustness of generated content. While the computational cost and data requirements remain substantial, this deep theoretical and practical refinement sets a new standard for perceptual quality and paves the way for generative models that are not only capable of producing stunning static images but also robust to dynamic transformations.


\label{sec:practical_considerations_and_alternative_generative_paradigms}

\section{Practical Considerations and Alternative Generative Paradigms}
\label{sec:practical\_considerations\_\_and\_\_alternative\_generative\_paradigms}

\subsection{Data Augmentation for Limited Data Regimes}
\label{sec:6\_1\_data\_augmentation\_for\_limited\_data\_regimes}

Training Generative Adversarial Networks (GANs) effectively often requires vast amounts of data, a condition not always met in real-world applications. When faced with limited training data, the discriminator can quickly overfit to the small dataset, leading to unstable training, mode collapse, and poor generation quality. To address this critical challenge, recent research has focused on sophisticated data augmentation strategies that enable stable and high-quality GAN training even with scarce data, without fundamentally altering the core GAN objective or architecture.

A significant breakthrough in this area was the introduction of Differentiable Augmentation (DiffAugment) by \cite{Zhao2020}. This method proposes applying a consistent set of differentiable transformations (such as translation, cutout, or color jitter) to \textit{both} the real and fake images \textit{before} they are fed into the discriminator. By ensuring that the discriminator always processes augmented data, DiffAugment prevents it from learning to distinguish between real and fake samples based on the presence or absence of augmentation. This consistent application forces the discriminator to learn features that are invariant to the chosen augmentations, thereby mitigating overfitting to the limited real data and significantly improving training stability and generation quality in low-data regimes. The differentiability of these augmentations allows gradients to flow unimpeded through the augmentation process, making it seamlessly integrable into the GAN training loop.

Building upon the foundational concept of differentiable augmentation, \cite{Karras2020a} introduced Adaptive Discriminator Augmentation (ADA), a more sophisticated approach that dynamically adjusts the strength of augmentations. While DiffAugment applies augmentations with a fixed policy, ADA monitors the discriminator's overfitting heuristic during training. Specifically, it tracks the discriminator's performance on a validation set or measures the ratio of real-to-fake classification accuracy. If the discriminator shows signs of overfitting (e.g., its accuracy on real samples becomes too high compared to fake samples), ADA incrementally increases the probability of applying augmentations. Conversely, if the discriminator is underfitting, the augmentation probability is reduced. This adaptive control mechanism ensures that the optimal level of augmentation is applied throughout training, preventing both overfitting and underfitting, and achieving unprecedented results in data-scarce scenarios. ADA has demonstrated remarkable performance, enabling the training of high-fidelity GANs with as few as a few thousand images, a feat previously considered impossible.

These data augmentation techniques are crucial for expanding the applicability of GANs to domains where large datasets are impractical or impossible to acquire, such as medical imaging, specialized industrial applications, or rare event generation. They effectively address a major practical limitation by making GANs robust to data scarcity, thereby democratizing access to powerful generative models. However, the choice of augmentation types and the specific implementation of adaptive policies can still influence performance, and finding universally optimal strategies remains an area of ongoing research. Future work may explore more complex adaptive mechanisms, the integration of semantic-aware augmentations, or combinations with other data-efficient learning paradigms to further enhance GAN performance under extreme data limitations.
\subsection{Hybrid VAE-GAN Architectures for Structured Generation}
\label{sec:6\_2\_hybrid\_vae-gan\_architectures\_for\_structured\_generation}

The landscape of generative modeling has been significantly shaped by Variational Autoencoders (VAEs) \cite{Kingma2014} and Generative Adversarial Networks (GANs) \cite{Goodfellow2014}. While VAEs excel at learning structured, probabilistic latent spaces and generating diverse samples, they often produce outputs that lack sharpness and fine detail. Conversely, GANs are renowned for generating high-fidelity, perceptually sharp images but frequently suffer from training instability and mode collapse, where the generator fails to capture the full diversity of the data distribution. To overcome these inherent limitations, hybrid VAE-GAN models have emerged as a powerful synergistic integration, aiming to leverage the strengths of both paradigms for more robust, diverse, and high-quality generative capabilities \cite{Larsen2016, cai2024m9z}. This approach represents a distinct strategy for stabilization, moving beyond purely adversarial dynamics or reconstruction-based learning.

The foundational VAE-GAN architecture, pioneered by Larsen et al. \cite{Larsen2016}, integrates a VAE's encoder-decoder structure with a GAN's adversarial training mechanism. In this setup, the VAE's encoder maps real data into a structured latent space, which is then sampled to feed the VAE's decoder. This decoder simultaneously acts as the generator in the GAN framework, producing synthetic samples. The VAE component, with its probabilistic framework, ensures that the latent space is well-regularized (often to a Gaussian prior via a Kullback-Leibler divergence term), which inherently promotes sample diversity and helps mitigate mode collapse, a common pitfall in standalone GANs \cite{Larsen2016, cai2024m9z}. The GAN discriminator, in turn, evaluates the realism of the samples generated by the VAE decoder, pushing it to produce outputs that are perceptually indistinguishable from real data, thereby resolving the characteristic blurriness of traditional VAEs \cite{Larsen2016}.

A key technical innovation in VAE-GANs lies in their unique combined loss function, which orchestrates the interplay between the VAE and GAN components. Typically, this loss function incorporates three main terms:
\begin{enumerate}
    \item \textbf{KL Divergence Term ($\mathcal{L}\_{prior}$)}: This is the standard VAE regularization term, ensuring the latent distribution learned by the encoder adheres to a predefined prior (e.g., a standard normal distribution). This term is crucial for maintaining a structured and diverse latent space, enabling meaningful interpolation and preventing posterior collapse \cite{Kingma2014, Larsen2016}.
    \item \textbf{Adversarial Loss ($\mathcal{L}\_{GAN}$)}: This term drives the generator (VAE decoder) to produce realistic samples that can fool the discriminator, and simultaneously trains the discriminator to distinguish between real and fake data \cite{Goodfellow2014, Larsen2016}.
    \item \textbf{Feature-wise Reconstruction Loss ($\mathcal{L}\_{llikeDis\_l}$)}: Instead of a pixel-wise reconstruction loss, which often leads to blurry images, VAE-GANs often employ a perceptual or feature-wise reconstruction loss. This loss measures the similarity between the input image and its reconstruction in the feature space of an intermediate layer of the GAN discriminator \cite{Larsen2016, cai2024m9z}. By forcing the generated images to be perceptually similar in a high-level feature representation, this term significantly enhances output sharpness and realism, while still allowing for some variation beyond strict pixel identity.
\end{enumerate}
This sophisticated loss combination ensures that the generated samples are not only diverse and representative of the data distribution but also sharp and high-fidelity.

Beyond the seminal VAE-GAN, other hybrid approaches have further explored the synergy between autoencoding and adversarial principles. For instance, the Autoencoding Generative Adversarial Network (AEGAN) \cite{lazarou2020gu8} proposes a four-network model that learns a bijective mapping between a specified latent space and the sample space. AEGAN applies both adversarial and reconstruction losses to \textit{both} the generated images and the generated latent vectors, aiming for enhanced training stabilization, prevention of mode collapse, and improved direct interpolation between real samples. Similarly, other works have explored implicitly regularizing the GAN discriminator using representative features extracted from a pre-trained autoencoder \cite{bang2018ps8}. By influencing GAN training with both reverse and forward Kullback-Leibler divergence, such methods aim to simultaneously improve visual quality and image diversity, demonstrating a broader trend of leveraging autoencoder-derived insights to stabilize and enhance GAN performance.

While VAE-GANs offer a compelling solution for stable and high-quality generation, they are not without their complexities and limitations. The theoretical underpinnings of generative models, including VAEs and GANs, suggest an inherent trade-off between the ability to fit multimodal distributions and training stability \cite{salmona202283g}. Models that synthesize data by transforming a standard Gaussian (often termed "push-forward" models) require a large Lipschitz constant in their generative networks to accurately represent multimodal data. This requirement directly conflicts with common stabilization techniques that constrain Lipschitz constants (e.g., Spectral Normalization, WGAN-GP), leading to a provable tension between expressivity and stability \cite{salmona202283g}. VAE-GANs attempt to navigate this by providing structured latent spaces and reconstruction guidance, but they still operate within these fundamental theoretical constraints. Compared to purely gradient-penalty-based stabilization methods like WGAN-GP \cite{Gulrajani2017} or architectural regularization like Spectral Normalization \cite{Miyato2018}, VAE-GANs introduce a different kind of stability by imposing structure on the latent space and grounding generation with reconstruction. However, this often comes at the cost of increased model complexity and computational demands due to the multiple interacting networks and loss terms, making them potentially harder to tune and train than some simpler GAN variants \cite{cai2024m9z}.

Despite these challenges, VAE-GANs have demonstrated enhanced capabilities across various applications. In creative media, they enable the synthesis and manipulation of digital images and the mimicry of artistic styles \cite{cai2024m9z}. In medical imaging, they have been utilized for generating anatomically accurate synthetic images, including high-resolution representations of rare tumors, which can aid diagnostics and training in data-scarce environments \cite{cai2024m9z}. The ability to generate highly realistic data also raises ethical concerns regarding potential misuse, such as the creation of deepfakes and misinformation, underscoring the need for stringent ethical guidelines and regulatory frameworks \cite{cai2024m9z}. Future research aims to further enhance training stability through self-correcting mechanisms, reduce computational costs with lightweight models, and expand their applicability into new domains like augmented reality and personalized medicine, further democratizing and advancing generative AI \cite{cai2024m9z}.


\label{sec:applications_of_stabilized_generative_adversarial_networks}

\section{Applications of Stabilized Generative Adversarial Networks}
\label{sec:applications\_of\_stabilized\_generative\_adversarial\_networks}

\subsection{Medical Imaging and 3D Reconstruction}
\label{sec:7\_1\_medical\_imaging\_\_and\_\_3d\_reconstruction}

The sensitive domains of medical imaging and 3D reconstruction present unique challenges for generative models, primarily due to data scarcity, privacy concerns, and the critical need for high-fidelity, clinically relevant synthetic data. Stabilized Generative Adversarial Networks (GANs) have emerged as transformative tools, addressing these limitations by enabling robust data augmentation and sophisticated anatomical reconstruction, thereby enhancing the performance and applicability of downstream analytical models. The inherent stability mechanisms, such as those discussed in Sections 3, 4, and 5, are paramount for generating diverse and realistic medical data, preventing artifacts or mode collapse that could compromise diagnostic accuracy.

A primary application of stabilized GANs in medical imaging is data augmentation, particularly vital in areas like cancer gene classification and medical image analysis where real-world data acquisition is often constrained. For instance, \cite{wei2021qea} highlights the problem of inadequate cancer gene expression data, which typically leads to poor generalization in classification models. While the specific GAN architecture is not detailed, the success of their proposed Cancer Classification Model, which leverages GANs to synthesize artificial cancer data "highly similar to the real one" and demonstrably improve classification results, implicitly relies on the GAN's ability to maintain stability and fidelity during generation. Without stable training, the generated data would lack the diversity or realism required to effectively augment limited datasets.

Building on this, more advanced and explicitly stabilized GAN architectures have shown significant promise. \cite{deebani202549r} utilized a conditional Wasserstein GAN (cWGAN) for data augmentation in breast cancer diagnosis. By employing the Wasserstein distance and gradient penalty (as discussed in Section 3.2), cWGAN ensures a smoother loss landscape and more stable training, crucial for generating high-quality synthetic breast cancer images. This approach, combined with multi-scale transfer learning, achieved an impressive 99.2\\% accuracy for binary classification on the BreakHis dataset, significantly outperforming existing methods. The conditional aspect of cWGAN further enhances its utility by allowing targeted generation of specific cancer subtypes, a critical feature for augmenting minority classes in imbalanced medical datasets. Similarly, \cite{park2021v6f} demonstrated that GAN-based synthetic MRI images for glioblastoma ensured sufficient morphological variations, improving the diagnostic accuracy for isocitrate dehydrogenase (IDH)-mutant types from 84.1\\% to 90.9\\% in an independent validation set. The ability to generate morphologically variable yet realistic samples is a direct consequence of stable GAN training, preventing mode collapse where the generator might only produce a limited set of variations.

Beyond direct image synthesis, stabilized GANs are also employed for medical image standardization. \cite{liang2018axu} introduced GANai for standardizing CT images across non-standard imaging protocols. Their model incorporates an "alternative improvement training strategy" with phase-specific loss functions and training data, which significantly enhances the efficiency and stability of GAN training. This approach mitigates the "lack-of-detail problem" in CT image synthesis, ensuring that generated images maintain crucial radiomic features for consistent analysis. The stability provided by GANai's unique training strategy is essential for generating standardized images that are both realistic and diagnostically reliable. Furthermore, the application extends to biosignals, where \cite{munia20201u2} leveraged a Wasserstein GAN (WGAN) for oversampling electrocardiogram (ECG) signals. The inherent stability of WGAN (Section 3.1), with its loss function correlating with generated image quality, enabled the generation of synthetic ECG data that improved minority-class classification accuracy for imbalanced biomedical datasets, validated using the Fréchet Inception Distance (FID) score.

In the realm of complex 3D reconstruction, the enhanced stability and fidelity of modern GAN architectures are particularly impactful. Acquiring comprehensive 3D anatomical data is often challenging, invasive, or expensive. Advanced GANs, such as StyleGAN-2 (detailed in Section 5.2), have been successfully applied to these tasks. For instance, \cite{broll2024edy} utilized StyleGAN-2 for the partial reconstruction of individual human molar teeth. Their innovative approach involved training StyleGAN-2 on 3D mesh files of full dental crown restorations, learning the intricate morphology of teeth. A crucial aspect was a PCA-optimized 2D projection method to convert 3D data into 2D images for StyleGAN-2 input, followed by a downstream optimization process for reconstruction. The architectural refinements of StyleGAN-2, including its style-based generator and path length regularization (Section 5.2), were critical for generating highly realistic and diverse occlusal surfaces, achieving Root Mean Square Error (RMSE) values between 0.02mm and 0.18mm. Notably, dentists preferred the GAN-based restorations for 3 out of 4 inlay geometries compared to conventional CAD procedures, underscoring the clinical relevance of StyleGAN-2's high-fidelity output.

Another significant contribution to 3D medical generation comes from \cite{saqur2018oqp}, who proposed CapsGAN for generating 3D images with high geometric transformations. Their work explicitly tackles GAN training stability by experimenting with Wasserstein distance (including gradient clipping and penalty, as discussed in Sections 3.1 and 3.2) and Spectral Normalization (Section 4.1). These foundational stability mechanisms are crucial for CapsGAN's ability to generate robust 3D images, paving the way for applications in 3D medical imaging and potentially video generation where geometric consistency is paramount.

These diverse applications collectively underscore the pivotal role of stabilized GANs in advancing medical imaging and 3D reconstruction. The ability to generate realistic, diverse, and anatomically plausible synthetic data—whether for augmenting cancer gene datasets, standardizing CT images, or reconstructing complex dental structures—directly addresses the pervasive issue of data limitation in medical research and clinical practice. The explicit integration of stability mechanisms, from WGAN's gradient penalties to StyleGAN's architectural refinements, is not merely an improvement but a prerequisite for generating data that meets the stringent requirements of clinical validity and diagnostic utility.

Despite these significant strides, challenges remain in ensuring the absolute clinical validity, interpretability, and generalizability of GAN-generated medical data. Future research must focus on developing more robust quantitative validation metrics tailored for synthetic medical data, exploring explainable AI techniques for GANs to provide transparency in their generative process, and further enhancing the controllability and specificity of generative models to meet the stringent requirements of diverse medical applications. Addressing these challenges will be crucial for paving the way for their broader and more confident adoption in clinical settings, ultimately improving patient care and accelerating medical research.
\subsection{Signal Processing and Denoising}
\label{sec:7\_2\_signal\_processing\_\_and\_\_denoising}

The accurate analysis of sensitive biomedical signals, such as Electroencephalography (EEG), is frequently hampered by pervasive noise and artifacts that obscure critical information. These interferences, often nonlinear and dynamic, pose significant challenges for traditional denoising methods like linear filtering or wavelet-based thresholding, which often struggle to adapt to the complex, non-stationary nature of biological signals without distorting essential features \cite{tibermacine2025pye}. Consequently, there is a pressing need for more robust and adaptive approaches capable of distinguishing between noise and subtle physiological activity.

In this context, Generative Adversarial Networks (GANs), particularly their stabilized variants, have emerged as powerful tools for signal processing, demonstrating a unique capability to learn complex data distributions and generate realistic synthetic signals. This generative capacity can be leveraged for tasks such as data augmentation, super-sampling, and crucially, denoising and artifact removal \cite{hartmann2018h3s}. The inherent adversarial training mechanism allows GANs to learn a mapping from noisy to clean signal distributions, offering a distinct advantage over methods that rely on explicit noise models.

A significant area of research has focused on the utility of stabilized GANs in signal denoising, where the core challenge lies in balancing aggressive noise removal with the high-fidelity reconstruction of subtle signal features. For instance, \cite{tibermacine2025pye} conducted a comprehensive comparative analysis of a conventional GAN model and a Wasserstein GAN with Gradient Penalty (WGAN-GP) for the adversarial denoising of EEG signals. Their work was motivated by the limitations of existing methods against nonlinear EEG artifacts and the scarcity of direct, systematic comparisons between different stable GAN architectures under identical experimental conditions. The "conventional GAN" in this context typically refers to an architecture employing the original minimax objective function, often with convolutional layers (similar to DCGANs), which, as discussed in Section 2.1, is prone to training instabilities but can sometimes excel in specific aspects of generation if carefully tuned \cite{wang2019w53}. WGAN-GP, on the other hand, leverages the Wasserstein distance and gradient penalties to ensure a smoother loss landscape and more stable training, as detailed in Section 3.2.

The study by \cite{tibermacine2025pye} meticulously designed an adversarial pipeline, evaluating both GAN variants on two distinct EEG datasets (healthy and unhealthy recordings). They employed a comprehensive suite of quantitative metrics, including signal-to-noise ratio (SNR), peak signal-to-noise ratio (PSNR), correlation coefficient, mutual information, dynamic time warping (DTW) distance, and relative root mean squared error (RRMSE). The findings demonstrated that both adversarial frameworks significantly outperformed classical wavelet-based thresholding and linear filtering methods, showcasing their superior adaptability to nonlinear distortions prevalent in EEG data.

Critically, the comparison revealed a practical trade-off between the two GAN architectures. The WGAN-GP model exhibited greater training stability and achieved higher SNR values (up to 14.47dB compared to 12.37dB for the conventional GAN), alongside consistently lower RRMSE values. This indicates WGAN-GP's proficiency in more aggressive and robust noise suppression, making it suitable for scenarios where strong artifact reduction is paramount. Conversely, the conventional GAN model proved more adept at preserving finer signal details, reflected in a higher PSNR of 19.28dB and a correlation coefficient exceeding 0.90 in several recordings. This suggests that while potentially less stable in training, a conventional GAN, when successfully trained, might offer superior fidelity for subtle signal features, which is crucial in clinical settings where minute neural activity is diagnostically significant.

Beyond pure denoising, the principles of stabilized GANs have also been applied to related challenges in EEG signal processing. For instance, \cite{panwar2019psx} utilized a semi-supervised WGAN-GP (sWGAN-GP) for classifying driving fatigue from EEG signals. While their primary goal was classification, the sWGAN-GP framework was instrumental in augmenting limited labeled data with generated EEG samples, effectively addressing frequency artifacts and training instability. This highlights how the stability and generative power of WGAN-GP extend beyond direct denoising to improve the robustness of downstream analytical tasks by providing cleaner or augmented data. Similarly, \cite{hartmann2018h3s} explored EEG-GANs for generating naturalistic EEG signals, noting the importance of WGAN improvements for training stabilization and the potential for "restoration of corrupted data segments," directly linking generation capabilities to denoising applications.

In summary, the application of stabilized GANs to signal processing, particularly for denoising EEG, underscores their transformative potential. The comparative analysis by \cite{tibermacine2025pye} provides valuable insights into the performance trade-offs inherent in different stable GAN variants, demonstrating that "stabilization" is not merely about achieving convergence, but also about understanding the nuanced impact on output quality and fidelity. While WGAN-GP offers superior training stability and aggressive noise reduction, conventional GANs might be preferable for preserving subtle signal details. Future research should build upon such comparative analyses, exploring how novel stabilization methods (e.g., spectral normalization, progressive growing, or self-attention mechanisms adapted for time series, as discussed in Section 4) impact these application-specific trade-offs. Further investigation into hybrid models, real-time denoising capabilities, and generalization across diverse physiological signal types will be crucial for advancing the field and enhancing data quality for subsequent analysis in various real-world scenarios.


\label{sec:conclusion,_challenges,_and_future_directions}

\section{Conclusion, Challenges, and Future Directions}
\label{sec:conclusion,\_challenges,\_\_and\_\_future\_directions}

\subsection{Summary of Key Advancements}
\label{sec:8\_1\_summary\_of\_key\_advancements}

The journey of stabilizing Generative Adversarial Networks (GANs) has been a continuous quest to overcome inherent training instabilities and mode collapse, progressively evolving towards models capable of high-fidelity, controllable, and data-efficient image synthesis. Initial GAN formulations, while theoretically powerful, were notoriously difficult to train, often suffering from vanishing gradients, exploding gradients, and the generator collapsing to produce only a limited set of outputs (mode collapse) \cite{jabbar2020aj0, wang2019w53, wiatrak20194ib}. This section outlines the major milestones and intellectual shifts that transformed GANs into robust generative tools.

Early efforts primarily focused on fundamental training stability through robust loss functions and gradient regularization. A significant breakthrough was the introduction of Wasserstein GAN (WGAN) by \cite{arjovsky2017ze5}, which replaced the original Jensen-Shannon divergence with the Wasserstein-1 distance. This provided a smoother loss landscape, offering more meaningful gradient signals even when the distributions had non-overlapping support, thereby mitigating mode collapse. However, WGAN's reliance on weight clipping to enforce the Lipschitz constraint on the discriminator proved problematic, often leading to suboptimal capacity and pathological behavior. This limitation was elegantly addressed by \cite{gulrajani2017improved} with the proposal of Wasserstein GAN with Gradient Penalty (WGAN-GP). WGAN-GP enforced the Lipschitz constraint by penalizing the norm of the discriminator's gradients with respect to its input, offering a more stable and effective regularization that became a cornerstone for subsequent GAN research. Concurrently, \cite{mao2017ss0} introduced Least Squares Generative Adversarial Networks (LSGANs), which adopted a least squares loss function for the discriminator, providing more stable gradients and reducing the vanishing gradient problem compared to the original sigmoid cross-entropy loss. Another pivotal development in regularization was Spectral Normalization (SN) by \cite{miyato2018arc}, a computationally light method to enforce the Lipschitz constraint on the discriminator's weights directly. SN-GANs demonstrated superior performance and stability across various datasets, and spectral normalization quickly became a standard component in many high-performing GAN architectures. Further theoretical insights into GAN convergence and the generalization of gradient penalties were provided by \cite{mescheder2018which}, reinforcing the importance of Lipschitz continuity. Other regularization techniques, such as Unrolled GANs \cite{metz20169ir} and various forms of explicit regularization \cite{roth2017eui, chu2020zbv}, also contributed to stabilizing the training dynamics by providing more informative gradient signals or constraining the model's behavior. These foundational advancements collectively laid the essential groundwork, transforming GANs from unstable curiosities into reliably trainable systems.

Building upon this improved stability, the intellectual focus shifted towards achieving high-fidelity, high-resolution, and controllable image synthesis through architectural innovations and progressive training strategies. \cite{karras2017raw} introduced Progressive Growing of GANs (PGGANs), a novel training methodology that significantly enhanced resolution and stability. PGGANs progressively grew both the generator and discriminator networks, starting from low-resolution images and gradually adding layers to model increasingly fine details. This approach not only accelerated training but also stabilized it, enabling the generation of unprecedentedly high-quality images. The subsequent work by \cite{brock2019large} (BigGAN) pushed the boundaries of scale and fidelity, demonstrating that large models, coupled with techniques like self-attention and conditional batch normalization, could generate diverse and highly realistic images across a wide range of classes. BigGAN often integrated spectral normalization, showcasing how foundational stability techniques were combined with architectural advancements. The StyleGAN series marked a significant leap in controllable synthesis. \cite{karras2019stylebased} (StyleGAN) introduced a style-based generator architecture that disentangled latent space, allowing for intuitive, scale-specific control over visual features through adaptive instance normalization. This innovation enabled users to manipulate aspects like pose, identity, and texture independently. StyleGAN2 \cite{karras2020analyzing} further refined the architecture by addressing common artifacts, improving regularization with path length regularization, and enhancing overall image quality. The series culminated with StyleGAN3 \cite{karras2021aliasfree}, which tackled aliasing issues inherent in previous models by introducing an alias-free architecture, leading to even sharper and more consistent outputs. These architectural innovations, from progressive training to style-based generators, collectively transformed GANs into powerful and versatile tools for generating remarkably realistic and diverse outputs with fine-grained control.

More recently, the research community has turned its attention to practical challenges, particularly data efficiency, to make GANs more accessible and applicable in scenarios with limited training data. High-fidelity GANs typically require vast datasets, which are not always available. To address this, \cite{karras202039x} proposed an adaptive discriminator augmentation (ADA) mechanism. ADA dynamically applies various augmentations to the training data, preventing the discriminator from overfitting to a small dataset and thereby stabilizing training in limited data regimes. This allowed for good results with significantly fewer images, often matching StyleGAN2 performance with an order of magnitude less data. Following this, \cite{zhao2020xhy} introduced Differentiable Augmentation (DiffAugment), which applies differentiable augmentations to both real and fake samples. This approach further improved data efficiency and training stability across various GAN architectures and loss functions, enabling high-fidelity generation with very limited datasets. Efforts like \cite{liu20212c2} (FastGAN) also contributed to this direction, focusing on faster and stabilized few-shot image synthesis with minimal computational cost.

In summary, the evolution of GAN stabilization has followed a clear narrative arc: from tackling fundamental optimization hurdles with robust loss functions and gradient regularization to unlocking high-fidelity, controllable synthesis through sophisticated architectural designs, and finally, to addressing practical deployment challenges like data efficiency. Each stage built upon the preceding one, demonstrating a cumulative advancement where stable training mechanisms enabled more complex architectures, and these architectures, in turn, highlighted new practical limitations. While significant progress has been made, ongoing challenges include further improving theoretical understanding of complex game dynamics, balancing diversity and quality in extreme data scarcity, and reducing the computational footprint for broader applicability.
\subsection{Unresolved Tensions and Open Challenges}
\label{sec:8\_2\_unresolved\_tensions\_\_and\_\_open\_challenges}

Despite immense progress in generative adversarial networks (GANs), the quest for perfectly stable, efficient, and robust models remains an active area of research, with many fundamental and practical issues yet to be fully resolved. A persistent challenge lies in the inherent tension between simultaneously achieving high fidelity, diversity, and controllability in generated outputs, often compounded by significant computational costs and data requirements.

One of the most enduring difficulties in GAN training is mode collapse, where the generator fails to capture the full diversity of the real data distribution. Early attempts to mitigate this, such as Unrolled GANs \cite{metz20169ir}, aimed to stabilize training by considering future discriminator states in the generator's objective, thereby encouraging broader exploration. Similarly, Mode Regularized GANs \cite{che2016kho} introduced regularization techniques to prevent discriminators from developing pathological functional shapes that lead to mode collapse. However, these issues persisted, leading to more sophisticated solutions. Prescribed GANs (PresGANs) \cite{dieng2019rjn} tackled mode collapse by adding noise to the generator's output and optimizing an entropy-regularized adversarial loss, encouraging the capture of all data modes. The SGAN framework \cite{chavdarova20179w6} proposed training multiple adversarial pairs to improve mode coverage and stability. More recently, DuelGAN \cite{wei2021gla} introduced a second discriminator to explicitly discourage agreement between discriminators, thereby increasing sample diversity and alleviating early mode collapse. Twin Discriminator GANs (TWGAN) \cite{zhang2021ypi} also leveraged a dual discriminator setup, combining saturating and non-saturating losses to improve training stability and diversity. These ongoing efforts underscore that mode collapse, particularly in highly diverse datasets, remains a continuous struggle, as highlighted in comprehensive surveys \cite{jabbar2020aj0, wang2019w53, wiatrak20194ib}.

Beyond mode collapse, general training instability is a pervasive issue. The original GAN formulation suffered from vanishing gradients, which Least Squares GANs (LSGANs) \cite{mao2017ss0} addressed by adopting a least squares loss, demonstrating improved stability. Viewing GAN training as a zero-sum game, Chekhov GAN \cite{grnarova20171tc} proposed an online learning approach to achieve provable convergence for certain architectures, though extending this to deep, complex models remains challenging. Theoretical analyses, such as those by \cite{liang2018r52}, have illuminated the complex interaction terms in two-player games that can both aid and hinder convergence, suggesting that a universal understanding of deep GAN convergence is still elusive. Control theory has also been applied to model and stabilize GAN dynamics in function space \cite{xu2019uwg}, showing promise but requiring further generalization. Recent works continue to explore novel stabilization mechanisms, such as the learnable auxiliary module in \cite{gan202494y} to counter generator sensitivity, or the collaborative transfer learning approach in Collaborative-GAN \cite{megahed2024c23} to mitigate instability and mode collapse.

Another critical unresolved tension lies in balancing high fidelity, diversity, and controllability with practical constraints like computational cost and data requirements. While architectures like Progressive Growing of GANs (PGGAN) \cite{karras2017raw} and the StyleGAN family \cite{karras2019, karras2020, karras2021} have achieved unprecedented image quality and disentangled control, they demand significant computational resources and massive datasets for training. The theoretical work by \cite{salmona202283g} revealed a provable trade-off between a push-forward generative model's ability to fit multimodal distributions (diversity) and the Lipschitz constant of its generator (stability), implying that increasing diversity can inherently conflict with stability constraints. Efforts like InfoMax-GAN \cite{lee20205ue} attempt to address both catastrophic forgetting and mode collapse through mutual information maximization, aiming for improved performance across diverse datasets without excessive hyperparameter tuning. Similarly, methods like StackGAN \cite{zhang2016mm0} decompose the problem into multi-stage generation for high-resolution images, implicitly increasing complexity. The integration of genetic algorithms in MEvo-GAN \cite{fu20241mw} represents another approach to enhance image details, structural integrity, and training stability, suggesting that novel optimization paradigms are still needed to navigate these trade-offs efficiently.

Furthermore, GANs are notoriously sensitive to hyperparameters. The success of methods like WGAN-GP \cite{gulrajani2017} and Spectral Normalization \cite{miyato2018} in stabilizing training often hinges on careful tuning of regularization strengths and learning rates. Even in applications like polyphonic music generation, careful optimization of reinforcement learning signals is reported as crucial \cite{lee2017zsj}. Adaptive mechanisms, such as the Adaptive Weighted Discriminator \cite{zadorozhnyy20208ft}, which dynamically adjusts loss weights, offer a step towards reducing this sensitivity, but a truly robust, hyperparameter-agnostic GAN training framework remains elusive.

Finally, the lack of universal theoretical convergence guarantees for complex, deep GAN architectures persists. While works like \cite{mescheder2018} and \cite{chu2020zbv} provide valuable theoretical insights into convergence conditions and the role of smoothness and Lipschitz constraints, these often apply to simplified settings or provide conditions that are not universally met by state-of-the-art models. Constrained GANs (GAN-C) \cite{chao2021ynq} introduce explicit constraints on discriminator output to accelerate convergence to Nash equilibrium, yet the broader challenge of proving global convergence for arbitrary deep GANs in non-convex, non-cooperative settings remains an open problem.

In conclusion, despite the remarkable advancements, the field of GAN stabilization is characterized by ongoing research into fundamental and practical issues. The simultaneous pursuit of high fidelity, diversity, and controllability, while managing computational costs, hyperparameter sensitivity, and the elusive goal of universal theoretical convergence, continues to drive innovation. The quest for perfectly stable, efficient, and robust GANs that can generalize across diverse data modalities and applications remains a vibrant and challenging frontier.
\subsection{Ethical Considerations and Societal Impact}
\label{sec:8\_3\_ethical\_considerations\_\_and\_\_societal\_impact}

The remarkable advancements in Generative Adversarial Networks (GANs), characterized by enhanced stability, fidelity, and accessibility, have ushered in a new era of synthetic content generation with profound ethical implications and societal challenges. While the technical prowess of these models, extensively reviewed in previous sections, offers unprecedented opportunities across diverse domains, it simultaneously amplifies their potential for misuse and the perpetuation of harmful biases, demanding a robust and proactive approach to responsible AI development \cite{bhat202445j, goyal2024ufg}.

One of the most prominent ethical concerns stems from the capacity of advanced GAN architectures to synthesize highly realistic and convincing content. The high-fidelity image generation enabled by techniques like Progressive Growing of GANs \cite{karras2017raw}, StyleGAN series \cite{karras2019, karras2020, karras2021}, and the stable training facilitated by methods such as Spectral Normalization \cite{miyato2018arc} have made synthetic media virtually indistinguishable from real content. This capability directly underpins the rise of sophisticated deepfakes, which can be leveraged for misinformation campaigns, identity theft, and malicious impersonation, posing significant threats to individual privacy, public trust, and democratic processes \cite{jenkins2024qf5}. The ability to generate photo-realistic images from text descriptions, as demonstrated by models like StackGAN \cite{zhang2016mm0}, further exacerbates this risk by enabling targeted and deceptive content creation with minimal effort. The integration of Variational Autoencoders (VAEs) and GANs into hybrid models, while improving output diversity and fidelity, also contributes to the challenge of discerning authentic from synthetic media \cite{cai2024m9z}.

Beyond deliberate misuse, generative models pose substantial risks through the amplification of inherent biases present in their training data. GANs learn to mimic the statistical distribution of their input, meaning that if the training data is unrepresentative, skewed, or contains societal stereotypes, the generated content will reflect and even exacerbate these biases. This can lead to discriminatory outcomes, perpetuate harmful stereotypes, and reinforce existing societal inequalities. The development of data-efficient training methods, such as Adaptive Discriminator Augmentation (ADA) \cite{karras202039x} and Differentiable Augmentation \cite{zhao2020xhy}, while beneficial for scenarios with data scarcity (e.g., medical imaging \cite{wei2021qea}), compounds this issue. These methods enable high-quality generation from limited datasets, meaning that even small, potentially skewed data pools can lead to powerful generative models that entrench and spread biased representations \cite{goyal2024ufg}. For instance, if a GAN is trained on a dataset predominantly featuring specific demographics, it may struggle to generate diverse representations or even perpetuate stereotypes, leading to biased outputs that can have real-world consequences \cite{seon202526r}.

A critical, though often less discussed, ethical concern is the potential for privacy leakage from generative models. While GANs are designed to learn the data distribution, not memorize individual samples, research has shown that they can inadvertently leak sensitive information about their training data. Hayes et al. \cite{hayes201742g} demonstrated the feasibility of membership inference attacks, where an adversary can determine whether a specific data sample was part of the model's training set. This vulnerability is particularly alarming when GANs are trained on sensitive datasets, such as medical records or personal biometric data, raising serious privacy concerns for individuals whose data might have been used in training \cite{jenkins2024qf5}.

To counteract these multifaceted challenges, a concerted effort towards responsible AI development, transparency, and the creation of robust detection and mitigation mechanisms is paramount. The development of reliable and resilient detection technologies is crucial for distinguishing between authentic and AI-generated content, forming a continuous "cat-and-mouse" game as generative models evolve. Research in adversarial defense, such as the use of GANs themselves for intrusion detection in cybersecurity \cite{u2023m2y} or for generating evasion samples to harden classifiers in low-data regimes \cite{randhawa2021ksq}, highlights the dual nature of these technologies in both creating and combating threats. Furthermore, understanding the internal workings of GANs is a crucial step towards accountability. Tools like GAN Dissection \cite{bau2018n2x} provide a means to visualize and understand the semantic concepts learned by different layers of a GAN, offering vital insights into how images are generated. Such transparency is essential for identifying and mitigating biases embedded within the model and for developing strategies to control generated content.

Beyond technical solutions, the establishment of clear ethical guidelines, regulatory frameworks, and interdisciplinary collaboration is imperative \cite{bhat202445j, goyal2024ufg}. Addressing the ethical landscape of GANs also involves grappling with emerging issues such as intellectual property rights for AI-generated content and the potential for economic disruption in creative industries. As generative models become increasingly capable and accessible, the research community bears a significant responsibility to proactively address these ethical implications and societal impacts. This involves not only pushing the boundaries of generative quality and stability but also investing in transparency tools, robust bias mitigation strategies, and advanced detection mechanisms. Fostering public trust and ensuring the beneficial and responsible deployment of these powerful technologies across various applications hinges on a concerted and holistic effort to navigate these complex ethical landscapes, balancing innovation with foresight and accountability.


\newpage
\section*{References}
\addcontentsline{toc}{section}{References}

\begin{thebibliography}{194}

\bibitem{arjovsky2017ze5}
Martín Arjovsky, Soumith Chintala, and L. Bottou (2017). \textit{Wasserstein Generative Adversarial Networks}. International Conference on Machine Learning.

\bibitem{karras2017raw}
Tero Karras, Timo Aila, S. Laine, et al. (2017). \textit{Progressive Growing of GANs for Improved Quality, Stability, and Variation}. International Conference on Learning Representations.

\bibitem{miyato2018arc}
Takeru Miyato, Toshiki Kataoka, Masanori Koyama, et al. (2018). \textit{Spectral Normalization for Generative Adversarial Networks}. International Conference on Learning Representations.

\bibitem{karras202039x}
Tero Karras, M. Aittala, Janne Hellsten, et al. (2020). \textit{Training Generative Adversarial Networks with Limited Data}. Neural Information Processing Systems.

\bibitem{zhang2016mm0}
Han Zhang, Tao Xu, Hongsheng Li, et al. (2016). \textit{StackGAN: Text to Photo-Realistic Image Synthesis with Stacked Generative Adversarial Networks}. IEEE International Conference on Computer Vision.

\bibitem{shrivastava2016uym}
A. Shrivastava, Tomas Pfister, Oncel Tuzel, et al. (2016). \textit{Learning from Simulated and Unsupervised Images through Adversarial Training}. Computer Vision and Pattern Recognition.

\bibitem{zhao2020xhy}
Shengyu Zhao, Zhijian Liu, Ji Lin, et al. (2020). \textit{Differentiable Augmentation for Data-Efficient GAN Training}. Neural Information Processing Systems.

\bibitem{metz20169ir}
Luke Metz, Ben Poole, David Pfau, et al. (2016). \textit{Unrolled Generative Adversarial Networks}. International Conference on Learning Representations.

\bibitem{guo2020n4t}
Ye-cai Guo, Hanyu Li, and Peixian Zhuang (2020). \textit{Underwater Image Enhancement Using a Multiscale Dense Generative Adversarial Network}. IEEE Journal of Oceanic Engineering.

\bibitem{bau2018n2x}
David Bau, Jun-Yan Zhu, Hendrik Strobelt, et al. (2018). \textit{GAN Dissection: Visualizing and Understanding Generative Adversarial Networks}. International Conference on Learning Representations.

\bibitem{che2016kho}
Tong Che, Yanran Li, Athul Paul Jacob, et al. (2016). \textit{Mode Regularized Generative Adversarial Networks}. International Conference on Learning Representations.

\bibitem{liu20212c2}
Bingchen Liu, Yizhe Zhu, Kunpeng Song, et al. (2021). \textit{Towards Faster and Stabilized GAN Training for High-fidelity Few-shot Image Synthesis}. International Conference on Learning Representations.

\bibitem{jabbar2020aj0}
Abdul Jabbar, Xi Li, and Bourahla Omar (2020). \textit{A Survey on Generative Adversarial Networks: Variants, Applications, and Training}. ACM Computing Surveys.

\bibitem{roth2017eui}
Kevin Roth, Aurélien Lucchi, Sebastian Nowozin, et al. (2017). \textit{Stabilizing Training of Generative Adversarial Networks through Regularization}. Neural Information Processing Systems.

\bibitem{yang2018svo}
Liu Yang, Dongkun Zhang, and G. Karniadakis (2018). \textit{Physics-Informed Generative Adversarial Networks for Stochastic Differential Equations}. SIAM Journal on Scientific Computing.

\bibitem{zhang2019hjo}
Han Zhang, Zizhao Zhang, Augustus Odena, et al. (2019). \textit{Consistency Regularization for Generative Adversarial Networks}. International Conference on Learning Representations.

\bibitem{tseng2021m2s}
Hung-Yu Tseng, Lu Jiang, Ce Liu, et al. (2021). \textit{Regularizing Generative Adversarial Networks under Limited Data}. Computer Vision and Pattern Recognition.

\bibitem{mao20196tx}
Wentao Mao, Yamin Liu, Ling Ding, et al. (2019). \textit{Imbalanced Fault Diagnosis of Rolling Bearing Based on Generative Adversarial Network: A Comparative Study}. IEEE Access.

\bibitem{hartmann2018h3s}
K. Hartmann, R. Schirrmeister, and T. Ball (2018). \textit{EEG-GAN: Generative adversarial networks for electroencephalograhic (EEG) brain signals}. arXiv.org.

\bibitem{wang2019w53}
Zhengwei Wang, Qi She, and T. Ward (2019). \textit{Generative Adversarial Networks in Computer Vision}. ACM Computing Surveys.

\bibitem{luo2020aaj}
Jia Luo, Jinying Huang, and Hongmei Li (2020). \textit{A case study of conditional deep convolutional generative adversarial networks in machine fault diagnosis}. Journal of Intelligent Manufacturing.

\bibitem{liu2020jt0}
Ming-Yu Liu, Xun Huang, Jiahui Yu, et al. (2020). \textit{Generative Adversarial Networks for Image and Video Synthesis: Algorithms and Applications}. Proceedings of the IEEE.

\bibitem{liang2018r52}
Tengyuan Liang, and J. Stokes (2018). \textit{Interaction Matters: A Note on Non-asymptotic Local Convergence of Generative Adversarial Networks}. International Conference on Artificial Intelligence and Statistics.

\bibitem{ghafoorian2018fwh}
Mohsen Ghafoorian, C. Nugteren, N. Baka, et al. (2018). \textit{EL-GAN: Embedding Loss Driven Generative Adversarial Networks for Lane Detection}. ECCV Workshops.

\bibitem{guo2019414}
Xiaopeng Guo, Rencan Nie, Jinde Cao, et al. (2019). \textit{FuseGAN: Learning to Fuse Multi-Focus Image via Conditional Generative Adversarial Network}. IEEE transactions on multimedia.

\bibitem{liu2020kd1}
B. Liu, Cheng Tan, Shuqin Li, et al. (2020). \textit{A Data Augmentation Method Based on Generative Adversarial Networks for Grape Leaf Disease Identification}. IEEE Access.

\bibitem{hjelm2017iqg}
R. Devon Hjelm, Athul Paul Jacob, Tong Che, et al. (2017). \textit{Boundary-Seeking Generative Adversarial Networks}. International Conference on Learning Representations.

\bibitem{shahriar2020sm7}
Md Hasan Shahriar, Nur Imtiazul Haque, M. Rahman, et al. (2020). \textit{G-IDS: Generative Adversarial Networks Assisted Intrusion Detection System}. Annual International Computer Software and Applications Conference.

\bibitem{pfau2016v7o}
David Pfau, and O. Vinyals (2016). \textit{Connecting Generative Adversarial Networks and Actor-Critic Methods}. arXiv.org.

\bibitem{mao2017ss0}
Xudong Mao, Qing Li, Haoran Xie, et al. (2017). \textit{On the Effectiveness of Least Squares Generative Adversarial Networks}. IEEE Transactions on Pattern Analysis and Machine Intelligence.

\bibitem{fekri2019c1i}
Mohammad Navid Fekri, A. M. Ghosh, and Katarina Grolinger (2019). \textit{Generating Energy Data for Machine Learning with Recurrent Generative Adversarial Networks}. Energies.

\bibitem{chen2019ng2}
Xinyuan Chen, Chang Xu, Xiaokang Yang, et al. (2019). \textit{Gated-GAN: Adversarial Gated Networks for Multi-Collection Style Transfer}. IEEE Transactions on Image Processing.

\bibitem{baby2019h4h}
Deepak Baby, and S. Verhulst (2019). \textit{Sergan: Speech Enhancement Using Relativistic Generative Adversarial Networks with Gradient Penalty}. IEEE International Conference on Acoustics, Speech, and Signal Processing.

\bibitem{wiatrak20194ib}
Maciej Wiatrak, Stefano V. Albrecht, and A. Nystrom (2019). \textit{Stabilizing Generative Adversarial Networks: A Survey}. Unpublished manuscript.

\bibitem{salmona202283g}
Antoine Salmona, Valentin De Bortoli, J. Delon, et al. (2022). \textit{Can Push-forward Generative Models Fit Multimodal Distributions?}. Neural Information Processing Systems.

\bibitem{lee20205ue}
Kwot Sin Lee, Ngoc-Trung Tran, and Ngai-Man Cheung (2020). \textit{InfoMax-GAN: Improved Adversarial Image Generation via Information Maximization and Contrastive Learning}. IEEE Workshop/Winter Conference on Applications of Computer Vision.

\bibitem{herr20208x4}
Daniel Herr, B. Obert, and Matthias Rosenkranz (2020). \textit{Anomaly detection with variational quantum generative adversarial networks}. Quantum Science and Technology.

\bibitem{hayes201742g}
Jamie Hayes, Luca Melis, G. Danezis, et al. (2017). \textit{LOGAN: Evaluating Privacy Leakage of Generative Models Using Generative Adversarial Networks}. arXiv.org.

\bibitem{negi20208n9}
Anuja Negi, A. Noel, Joseph Raj, et al. (2020). \textit{RDA-UNET-WGAN: An Accurate Breast Ultrasound Lesion Segmentation Using Wasserstein Generative Adversarial Networks}. The Arabian journal for science and engineering.

\bibitem{meng2022you}
Zong Meng, Qian Li, De-gang Sun, et al. (2022). \textit{An Intelligent Fault Diagnosis Method of Small Sample Bearing Based on Improved Auxiliary Classification Generative Adversarial Network}. IEEE Sensors Journal.

\bibitem{liu2019sb7}
Yi Liu, Jialiang Peng, James J. Q. Yu, et al. (2019). \textit{PPGAN: Privacy-Preserving Generative Adversarial Network}. International Conference on Parallel and Distributed Systems.

\bibitem{yuan2020bt6}
Zhenmou Yuan, M. Jiang, Yaming Wang, et al. (2020). \textit{SARA-GAN: Self-Attention and Relative Average Discriminator Based Generative Adversarial Networks for Fast Compressed Sensing MRI Reconstruction}. Frontiers in Neuroinformatics.

\bibitem{agarwal2022p6d}
Aishwarya Agarwal, Biplab Banerjee, Fabio Cuzzolin, et al. (2022). \textit{Semantics-Driven Generative Replay for Few-Shot Class Incremental Learning}. ACM Multimedia.

\bibitem{grnarova20171tc}
Paulina Grnarova, K. Levy, Aurélien Lucchi, et al. (2017). \textit{An Online Learning Approach to Generative Adversarial Networks}. International Conference on Learning Representations.

\bibitem{liu2019oc8}
Zhiyue Liu, Jiahai Wang, and Zhiwei Liang (2019). \textit{CatGAN: Category-aware Generative Adversarial Networks with Hierarchical Evolutionary Learning for Category Text Generation}. AAAI Conference on Artificial Intelligence.

\bibitem{chung2022s9a}
Jihoon Chung, Bo Shen, and Zhen Kong (2022). \textit{Anomaly detection in additive manufacturing processes using supervised classification with imbalanced sensor data based on generative adversarial network}. Journal of Intelligent Manufacturing.

\bibitem{chu2020zbv}
Casey Chu, Kentaro Minami, and K. Fukumizu (2020). \textit{Smoothness and Stability in GANs}. International Conference on Learning Representations.

\bibitem{jenni2019339}
S. Jenni, and P. Favaro (2019). \textit{On Stabilizing Generative Adversarial Training With Noise}. Computer Vision and Pattern Recognition.

\bibitem{xiang20171at}
Sitao Xiang, and Hao Li (2017). \textit{On the Effects of Batch and Weight Normalization in Generative Adversarial Networks}. Unpublished manuscript.

\bibitem{neyshabur201713g}
Behnam Neyshabur, Srinadh Bhojanapalli, and Ayan Chakrabarti (2017). \textit{Stabilizing GAN Training with Multiple Random Projections}. arXiv.org.

\bibitem{bau20197hm}
David Bau, Jun-Yan Zhu, Hendrik Strobelt, et al. (2019). \textit{Visualizing and Understanding Generative Adversarial Networks (Extended Abstract)}. arXiv.org.

\bibitem{dieng2019rjn}
A. B. Dieng, Francisco J. R. Ruiz, D. Blei, et al. (2019). \textit{Prescribed Generative Adversarial Networks}. arXiv.org.

\bibitem{zhang2020376}
Hongliang Zhang, Rui Wang, Ruilin Pan, et al. (2020). \textit{Imbalanced Fault Diagnosis of Rolling Bearing Using Enhanced Generative Adversarial Networks}. IEEE Access.

\bibitem{yuan202257j}
Chao Yuan, Hongxia Wang, Peisong He, et al. (2022). \textit{GAN-based image steganography for enhancing security via adversarial attack and pixel-wise deep fusion}. Multimedia tools and applications.

\bibitem{iwai2020fp2}
Shoma Iwai, Tomo Miyazaki, Yoshihiro Sugaya, et al. (2020). \textit{Fidelity-Controllable Extreme Image Compression with Generative Adversarial Networks}. International Conference on Pattern Recognition.

\bibitem{kaneko2018jex}
Takuhiro Kaneko, Y. Ushiku, and T. Harada (2018). \textit{Label-Noise Robust Generative Adversarial Networks}. Computer Vision and Pattern Recognition.

\bibitem{khan20223o7}
Maleika Heenaye-Mamode Khan, N. Gooda Sahib-Kaudeer, Motean Dayalen, et al. (2022). \textit{Multi-Class Skin Problem Classification Using Deep Generative Adversarial Network (DGAN)}. Computational Intelligence and Neuroscience.

\bibitem{lin20224oj}
Qiuzhen Lin, Z. Fang, Yi Chen, et al. (2022). \textit{Evolutionary Architectural Search for Generative Adversarial Networks}. IEEE Transactions on Emerging Topics in Computational Intelligence.

\bibitem{tuan2018kbr}
Yi-Lin Tuan, and Hung-yi Lee (2018). \textit{Improving Conditional Sequence Generative Adversarial Networks by Stepwise Evaluation}. IEEE/ACM Transactions on Audio Speech and Language Processing.

\bibitem{wei2021qea}
Kaimin Wei, Tianqi Li, Feiran Huang, et al. (2021). \textit{Cancer classification with data augmentation based on generative adversarial networks}. Frontiers of Computer Science.

\bibitem{wang20178xf}
Ruohan Wang, Antoine Cully, H. Chang, et al. (2017). \textit{MAGAN: Margin Adaptation for Generative Adversarial Networks}. arXiv.org.

\bibitem{sage2017ywd}
Alexander Sage, E. Agustsson, R. Timofte, et al. (2017). \textit{Logo Synthesis and Manipulation with Clustered Generative Adversarial Networks}. 2018 IEEE/CVF Conference on Computer Vision and Pattern Recognition.

\bibitem{wu2020p8p}
Yue Wu, Pan Zhou, A. Wilson, et al. (2020). \textit{Improving GAN Training with Probability Ratio Clipping and Sample Reweighting}. Neural Information Processing Systems.

\bibitem{chavdarova20179w6}
Tatjana Chavdarova, and F. Fleuret (2017). \textit{SGAN: An Alternative Training of Generative Adversarial Networks}. 2018 IEEE/CVF Conference on Computer Vision and Pattern Recognition.

\bibitem{li2020muy}
Ziqiang Li, Pengfei Xia, Rentuo Tao, et al. (2020). \textit{A New Perspective on Stabilizing GANs Training: Direct Adversarial Training}. IEEE Transactions on Emerging Topics in Computational Intelligence.

\bibitem{goudarzi2020ymw}
Sobhan Goudarzi, A. Asif, and H. Rivaz (2020). \textit{Fast Multi-Focus Ultrasound Image Recovery Using Generative Adversarial Networks}. IEEE Transactions on Computational Imaging.

\bibitem{tao20219q2}
Yuechuan Tao, J. Qiu, and Shuying Lai (2021). \textit{A Data-Driven Management Strategy of Electric Vehicles and Thermostatically Controlled Loads Based on Modified Generative Adversarial Network}. IEEE Transactions on Transportation Electrification.

\bibitem{zhong2019opk}
Yue Zhong, Lizhuang Liu, Dan Zhao, et al. (2019). \textit{A generative adversarial network for image denoising}. Multimedia tools and applications.

\bibitem{yan2020889}
Peiyao Yan, Feng He, Yajie Yang, et al. (2020). \textit{Semi-Supervised Representation Learning for Remote Sensing Image Classification Based on Generative Adversarial Networks}. IEEE Access.

\bibitem{lee20203j4}
Shindong Lee, Bonggu Ko, Keonnyeong Lee, et al. (2020). \textit{Many-To-Many Voice Conversion Using Conditional Cycle-Consistent Adversarial Networks}. IEEE International Conference on Acoustics, Speech, and Signal Processing.

\bibitem{hu2021yk5}
Tianyu Hu, Yang Huang, Qiuming Zhu, et al. (2021). \textit{Channel Estimation Enhancement With Generative Adversarial Networks}. IEEE Transactions on Cognitive Communications and Networking.

\bibitem{chen2021n5h}
Tianlong Chen, Yu Cheng, Zhe Gan, et al. (2021). \textit{Ultra-Data-Efficient GAN Training: Drawing A Lottery Ticket First, Then Training It Toughly}. arXiv.org.

\bibitem{cai2019g1w}
Yali Cai, Xiaoru Wang, Zhihong Yu, et al. (2019). \textit{Dualattn-GAN: Text to Image Synthesis With Dual Attentional Generative Adversarial Network}. IEEE Access.

\bibitem{zhou20199sm}
Niyun Zhou, De Cai, Xiao Han, et al. (2019). \textit{Enhanced Cycle-Consistent Generative Adversarial Network for Color Normalization of H&E Stained Images}. International Conference on Medical Image Computing and Computer-Assisted Intervention.

\bibitem{tang2018iie}
Hao Tang, Dan Xu, Wei Wang, et al. (2018). \textit{Dual Generator Generative Adversarial Networks for Multi-Domain Image-to-Image Translation}. Asian Conference on Computer Vision.

\bibitem{tong2022lu4}
Q. Tong, Feiyu Lu, Ziwei Feng, et al. (2022). \textit{A Novel Method for Fault Diagnosis of Bearings with Small and Imbalanced Data Based on Generative Adversarial Networks}. Applied Sciences.

\bibitem{costa2019pj9}
Victor Costa, Nuno Lourenço, and P. Machado (2019). \textit{Coevolution of Generative Adversarial Networks}. EvoApplications.

\bibitem{tang2021c82}
Hongtao Tang, Shengbo Gao, Lei Wang, et al. (2021). \textit{A Novel Intelligent Fault Diagnosis Method for Rolling Bearings Based on Wasserstein Generative Adversarial Network and Convolutional Neural Network under Unbalanced Dataset}. Italian National Conference on Sensors.

\bibitem{yin2022izd}
Haitao Yin, and Jing Xiao (2022). \textit{Laplacian Pyramid Generative Adversarial Network for Infrared and Visible Image Fusion}. IEEE Signal Processing Letters.

\bibitem{xu2020pkq}
Kun Xu, Chongxuan Li, Huanshu Wei, et al. (2020). \textit{Understanding and Stabilizing GANs' Training Dynamics Using Control Theory}. International Conference on Machine Learning.

\bibitem{rahman2021wm8}
Taseef Rahman, Yuanqi Du, Liang Zhao, et al. (2021). \textit{Generative Adversarial Learning of Protein Tertiary Structures}. Molecules.

\bibitem{zhang2022ysl}
Zheng Zhang, Jingsong Yang, and Yang Du (2022). \textit{Deep Convolutional Generative Adversarial Network With Autoencoder for Semisupervised SAR Image Classification}. IEEE Geoscience and Remote Sensing Letters.

\bibitem{varshney2021954}
Sakshi Varshney, V. Verma, K. SrijithP., et al. (2021). \textit{CAM-GAN: Continual Adaptation Modules for Generative Adversarial Networks}. Neural Information Processing Systems.

\bibitem{creswell2016mol}
Antonia Creswell, and A. Bharath (2016). \textit{Adversarial Training for Sketch Retrieval}. ECCV Workshops.

\bibitem{bang2018ps8}
Duhyeon Bang, and Hyunjung Shim (2018). \textit{Improved Training of Generative Adversarial Networks Using Representative Features}. International Conference on Machine Learning.

\bibitem{wang202066v}
Dong Wang, Xiaoqian Qin, F. Song, et al. (2020). \textit{Stabilizing Training of Generative Adversarial Nets via Langevin Stein Variational Gradient Descent}. IEEE Transactions on Neural Networks and Learning Systems.

\bibitem{cai2020n2k}
Likun Cai, Yanjie Chen, Ning Cai, et al. (2020). \textit{Utilizing Amari-Alpha Divergence to Stabilize the Training of Generative Adversarial Networks}. Entropy.

\bibitem{wenzel20225g3}
Markus T. Wenzel (2022). \textit{Generative Adversarial Networks and Other Generative Models}. arXiv.org.

\bibitem{gidel2018pg0}
G. Gidel, Hugo Berard, Pascal Vincent, et al. (2018). \textit{A Variational Inequality Perspective on Generative Adversarial Nets}. arXiv.org.

\bibitem{grinblat2017cem}
G. Grinblat, Lucas C. Uzal, and P. Granitto (2017). \textit{Class-Splitting Generative Adversarial Networks}. arXiv.org.

\bibitem{zhang202263o}
Yingxue Zhang, Yanhua Li, Xun Zhou, et al. (2022). \textit{STrans-GAN: Spatially-Transferable Generative Adversarial Networks for Urban Traffic Estimation}. Industrial Conference on Data Mining.

\bibitem{shin2020169}
Hoo-Chang Shin, Alvin Ihsani, Ziyue Xu, et al. (2020). \textit{GANDALF: Generative Adversarial Networks with Discriminator-Adaptive Loss Fine-tuning for Alzheimer's Disease Diagnosis from MRI}. International Conference on Medical Image Computing and Computer-Assisted Intervention.

\bibitem{ham2020svv}
Hyung-Gi Ham, T. Jun, and Daeyoung Kim (2020). \textit{Unbalanced GANs: Pre-training the Generator of Generative Adversarial Network using Variational Autoencoder}. arXiv.org.

\bibitem{wang20182xz}
Chu Wang, Yanming Zhang, and Cheng-Lin Liu (2018). \textit{Anomaly Detection via Minimum Likelihood Generative Adversarial Networks}. International Conference on Pattern Recognition.

\bibitem{zhang2018oba}
Zhirui Zhang, Shujie Liu, Mu Li, et al. (2018). \textit{Bidirectional Generative Adversarial Networks for Neural Machine Translation}. Conference on Computational Natural Language Learning.

\bibitem{liang2018axu}
G. Liang, S. Fouladvand, Jie Zhang, et al. (2018). \textit{GANai: Standardizing CT Images using Generative Adversarial Network with Alternative Improvement}. bioRxiv.

\bibitem{wiatrak20194ae}
Maciej Wiatrak, and Stefano V. Albrecht (2019). \textit{Stabilizing Generative Adversarial Network Training: A Survey}. arXiv.org.

\bibitem{xue2022n0r}
Yu Xue, Weinan Tong, Ferrante Neri, et al. (2022). \textit{PEGANs: Phased Evolutionary Generative Adversarial Networks with Self-Attention Module}. Mathematics.

\bibitem{oeldorf2019kj7}
Cedric Oeldorf, and Gerasimos Spanakis (2019). \textit{LoGANv2: Conditional Style-Based Logo Generation with Generative Adversarial Networks}. International Conference on Machine Learning and Applications.

\bibitem{sajjadi2018w83}
Mehdi S. M. Sajjadi, and B. Scholkopf (2018). \textit{Tempered Adversarial Networks}. International Conference on Machine Learning.

\bibitem{park2021v6f}
J. E. Park, Da-in Eun, H. Kim, et al. (2021). \textit{Generative adversarial network for glioblastoma ensures morphologic variations and improves diagnostic model for isocitrate dehydrogenase mutant type}. Scientific Reports.

\bibitem{song2020mj8}
Xiaoning Song, Yao Chen, Zhenhua Feng, et al. (2020). \textit{SP-GAN: Self-Growing and Pruning Generative Adversarial Networks}. IEEE Transactions on Neural Networks and Learning Systems.

\bibitem{randhawa2021ksq}
Rizwan Hamid Randhawa, N. Aslam, Mohammad Alauthman, et al. (2021). \textit{Evasion Generative Adversarial Network for Low Data Regimes}. IEEE Transactions on Artificial Intelligence.

\bibitem{wang2020vbt}
Mengxue Wang, Zhenxue Chen, Q. M. J. Wu, et al. (2020). \textit{Improved face super-resolution generative adversarial networks}. Machine Vision and Applications.

\bibitem{saqur2018oqp}
Raeid Saqur, and Sal Vivona (2018). \textit{CapsGAN: Using Dynamic Routing for Generative Adversarial Networks}. Advances in Intelligent Systems and Computing.

\bibitem{gao2018d4g}
F. Gao, Fei Ma, Jun Wang, et al. (2018). \textit{Semi-Supervised Generative Adversarial Nets with Multiple Generators for SAR Image Recognition}. Italian National Conference on Sensors.

\bibitem{you2018a3m}
Haoran You, Yu Cheng, Tianheng Cheng, et al. (2018). \textit{Bayesian Cycle-Consistent Generative Adversarial Networks via Marginalizing Latent Sampling}. IEEE Transactions on Neural Networks and Learning Systems.

\bibitem{du2021bhg}
Biao Du, Lin Tang, Lin Liu, et al. (2021). \textit{Predicting LncRNA-Disease Association Based on Generative Adversarial Network.}. Current Gene Therapy.

\bibitem{wei2021gla}
Jiaheng Wei, Minghao Liu, Jiahao Luo, et al. (2021). \textit{DuelGAN: A Duel Between Two Discriminators Stabilizes the GAN Training}. European Conference on Computer Vision.

\bibitem{lazarou2020gu8}
Conor Lazarou (2020). \textit{Autoencoding Generative Adversarial Networks}. arXiv.org.

\bibitem{zhang2021ypi}
Zhaoyu Zhang, Mengyan Li, Haonian Xie, et al. (2021). \textit{TWGAN: Twin Discriminator Generative Adversarial Networks}. IEEE transactions on multimedia.

\bibitem{jiang2020e6i}
Yi Jiang, Jiajie Xu, Baoqing Yang, et al. (2020). \textit{Image Inpainting Based on Generative Adversarial Networks}. IEEE Access.

\bibitem{plakias2018h0x}
Spyridon Plakias, and Y. Boutalis (2018). \textit{Generative Adversarial Networks for Unsupervised Fault Detection}. European Control Conference.

\bibitem{chao2021ynq}
Xiaopeng Chao, Jiangzhong Cao, Yuqin Lu, et al. (2021). \textit{Constrained Generative Adversarial Networks}. IEEE Access.

\bibitem{zhang20182tk}
Jiacen Zhang, Nakamasa Inoue, and K. Shinoda (2018). \textit{I-vector Transformation Using Conditional Generative Adversarial Networks for Short Utterance Speaker Verification}. Interspeech.

\bibitem{cao20184y8}
Yanshuai Cao, G. Ding, Kry Yik-Chau Lui, et al. (2018). \textit{Improving GAN Training via Binarized Representation Entropy (BRE) Regularization}. International Conference on Learning Representations.

\bibitem{costa2020anu}
Victor Costa, Nuno Lourenço, João Correia, et al. (2020). \textit{Neuroevolution of Generative Adversarial Networks}. Deep Neural Evolution.

\bibitem{panwar2019psx}
Sharaj Panwar, P. Rad, J. Quarles, et al. (2019). \textit{A Semi-Supervised Wasserstein Generative Adversarial Network for Classifying Driving Fatigue from EEG signals}. IEEE International Conference on Systems, Man and Cybernetics.

\bibitem{wu20212vn}
Aming Wu, Juyong Shin, Jae-Kwang Ahn, et al. (2021). \textit{Augmenting Seismic Data Using Generative Adversarial Network for Low-Cost MEMS Sensors}. IEEE Access.

\bibitem{shou2020v6h}
Chunhui Shou, Ling Hong, Waner Ding, et al. (2020). \textit{Defect Detection with Generative Adversarial Networks for Electroluminescence Images of Solar Cells}. Youth Academic Annual Conference of Chinese Association of Automation.

\bibitem{liu2019v0x}
Jianfei Liu, Christine Shen, Tao Liu, et al. (2019). \textit{Active Appearance Model Induced Generative Adversarial Network for Controlled Data Augmentation}. International Conference on Medical Image Computing and Computer-Assisted Intervention.

\bibitem{farrell2019kjy}
S. Farrell, W. Bhimji, T. Kurth, et al. (2019). \textit{Next Generation Generative Neural Networks for HEP}. EPJ Web of Conferences.

\bibitem{wu2020n95}
Zhongze Wu, Chunmei He, Liwen Yang, et al. (2020). \textit{Attentive evolutionary generative adversarial network}. Applied intelligence (Boston).

\bibitem{majtner20192pi}
Tomás Majtner, Buda Bajić, Joakim Lindblad, et al. (2019). \textit{On the Effectiveness of Generative Adversarial Networks as HEp-2 Image Augmentation Tool}. Scandinavian Conference on Image Analysis.

\bibitem{zadorozhnyy20208ft}
Vasily Zadorozhnyy, Q. Cheng, and Q. Ye (2020). \textit{Adaptive Weighted Discriminator for Training Generative Adversarial Networks}. Computer Vision and Pattern Recognition.

\bibitem{munia20201u2}
M. Munia, M. Nourani, and Sammy Houari (2020). \textit{Biosignal Oversampling Using Wasserstein Generative Adversarial Network}. IEEE International Conference on Healthcare Informatics.

\bibitem{warner2020a5z}
J. Warner, Julian Cuevas, G. Bomarito, et al. (2020). \textit{Inverse Estimation of Elastic Modulus Using Physics-Informed Generative Adversarial Networks}. arXiv.org.

\bibitem{lee2017zsj}
Sang-gil Lee, Uiwon Hwang, Seonwoo Min, et al. (2017). \textit{Polyphonic Music Generation with Sequence Generative Adversarial Networks}. Journal of KIISE.

\bibitem{xu2019uwg}
Kun Xu, Chongxuan Li, Huanshu Wei, et al. (2019). \textit{Understanding and Stabilizing GANs' Training Dynamics with Control Theory}. arXiv.org.

\bibitem{zhang201996t}
Shufei Zhang, Zhuang Qian, Kaizhu Huang, et al. (2019). \textit{Robust generative adversarial network}. Machine-mediated learning.

\bibitem{pieters2018jh1}
Mathijs Pieters, and M. Wiering (2018). \textit{Comparing Generative Adversarial Network Techniques for Image Creation and Modification}. arXiv.org.

\bibitem{xiang2017cc9}
Sitao Xiang, and Hao Li (2017). \textit{On the effect of Batch Normalization and Weight Normalization in Generative Adversarial Networks}. arXiv.org.

\bibitem{xiong20243bt}
Hongqiang Xiong, Jing Li, Zhilian Li, et al. (2024). \textit{GPR-GAN: A Ground-Penetrating Radar Data Generative Adversarial Network}. IEEE Transactions on Geoscience and Remote Sensing.

\bibitem{xue2024e7i}
Yu Xue, Weinan Tong, Ferrante Neri, et al. (2024). \textit{Evolutionary Architecture Search for Generative Adversarial Networks Based on Weight Sharing}. IEEE Transactions on Evolutionary Computation.

\bibitem{xue20248md}
Yu Xue, Kun Chen, and Ferrante Neri (2024). \textit{Differentiable Architecture Search With Attention Mechanisms for Generative Adversarial Networks}. IEEE Transactions on Emerging Topics in Computational Intelligence.

\bibitem{jenkins2024qf5}
John Jenkins, and Kaushik Roy (2024). \textit{Exploring deep convolutional generative adversarial networks (DCGAN) in biometric systems: a survey study}. Discover Artificial Intelligence.

\bibitem{qiu2025hu0}
Shiqing Qiu, Yang Wang, Zong Ke, et al. (2025). \textit{A Generative Adversarial Network-Based Investor Sentiment Indicator: Superior Predictability for the Stock Market}. Mathematics.

\bibitem{boubrahimi2024kts}
Soukaina Filali Boubrahimi, Ashit Neema, Ayman Nassar, et al. (2024). \textit{Spatiotemporal Data Augmentation of MODIS‐Landsat Water Bodies Using Adversarial Networks}. Water Resources Research.

\bibitem{liu20232tr}
Naihao Liu, Youbo Lei, Yang Yang, et al. (2023). \textit{Self-supervised Time-Frequency Representation based on Generative Adversarial Networks}. Geophysics.

\bibitem{song20239hi}
Yihong Song, Haoyan Zhang, Jiaqi Li, et al. (2023). \textit{High-Accuracy Maize Disease Detection Based on Attention Generative Adversarial Network and Few-Shot Learning}. Plants.

\bibitem{pal2023147}
Debabrata Pal, Shirsha Bose, Biplab Banerjee, et al. (2023). \textit{MORGAN: Meta-Learning-based Few-Shot Open-Set Recognition via Generative Adversarial Network}. IEEE Workshop/Winter Conference on Applications of Computer Vision.

\bibitem{gan202494y}
Yan Gan, Chenxue Yang, Mao Ye, et al. (2024). \textit{Generative Adversarial Networks with Learnable Auxiliary Module for Image Synthesis}. ACM Trans. Multim. Comput. Commun. Appl..

\bibitem{eltehewy2023cj4}
Rokaya Eltehewy, A. Abouelfarag, and Sherine Nagy Saleh (2023). \textit{Efficient Classification of Imbalanced Natural Disasters Data Using Generative Adversarial Networks for Data Augmentation}. ISPRS Int. J. Geo Inf..

\bibitem{chen2023rrf}
Shiming Chen, Shuhuang Chen, W. Hou, et al. (2023). \textit{EGANS: Evolutionary Generative Adversarial Network Search for Zero-Shot Learning}. IEEE Transactions on Evolutionary Computation.

\bibitem{fu20241mw}
Feiran Fu, Peng Liu, Zhen Shao, et al. (2024). \textit{MEvo-GAN: A Multi-Scale Evolutionary Generative Adversarial Network for Underwater Image Enhancement}. Journal of Marine Science and Engineering.

\bibitem{soleymanzadeh202358z}
Raha Soleymanzadeh, and R. Kashef (2023). \textit{Efficient intrusion detection using multi-player generative adversarial networks (GANs): an ensemble-based deep learning architecture}. Neural computing & applications (Print).

\bibitem{fathallah20236k5}
Mohamed Fathallah, Mohamed Sakr, and Sherif Eletriby (2023). \textit{Stabilizing and Improving Training of Generative Adversarial Networks Through Identity Blocks and Modified Loss Function}. IEEE Access.

\bibitem{luo2024o1x}
Tianjiao Luo, Tim Pearce, Huayu Chen, et al. (2024). \textit{C-GAIL: Stabilizing Generative Adversarial Imitation Learning with Control Theory}. Neural Information Processing Systems.

\bibitem{li2024uae}
Wei Li, and Yongchuan Tang (2024). \textit{Soft Generative Adversarial Network: Combating Mode Collapse in Generative Adversarial Network Training via Dynamic Borderline Softening Mechanism}. Applied Sciences.

\bibitem{cai2024m9z}
Dongting Cai (2024). \textit{Enhancing capabilities of generative models through VAE-GAN integration: A review}. Applied and Computational Engineering.

\bibitem{u2023m2y}
K. U, T. S, T.V. Nidhin Prabhakar, et al. (2023). \textit{Adversarial Defense: A GAN-IF Based Cyber-security Model for Intrusion Detection in Software Piracy}. J. Wirel. Mob. Networks Ubiquitous Comput. Dependable Appl..

\bibitem{liu2023q2q}
Xiaobao Liu, Shuailin Su, Wenjuan Gu, et al. (2023). \textit{Super-Resolution Reconstruction of CT Images Based on Multi-scale Information Fused Generative Adversarial Networks}. Annals of Biomedical Engineering.

\bibitem{cheng2023t9b}
Shijie Cheng, Lingfeng Wang, M. Zhang, et al. (2023). \textit{SUGAN: A Stable U-Net Based Generative Adversarial Network}. Italian National Conference on Sensors.

\bibitem{luo2022rm1}
Xukang Luo, Ying Jiang, Enqiang Wang, et al. (2022). \textit{Anomaly detection by using a combination of generative adversarial networks and convolutional autoencoders}. EURASIP Journal on Advances in Signal Processing.

\bibitem{xu2022ss4}
Jialing Xu, Jingxing He, Jinqiang Gu, et al. (2022). \textit{Financial Time Series Prediction Based on XGBoost and Generative Adversarial Networks}. International Journal of Circuits, Systems and Signal Processing.

\bibitem{alshehri2022d1h}
Abeer Alshehri, Mounira Taileb, and Reem M. Alotaibi (2022). \textit{DeepAIA: An Automatic Image Annotation Model Based on Generative Adversarial Networks and Transfer Learning}. IEEE Access.

\bibitem{yeh2022yvr}
Yen-Tung Yeh, Bo-Yu Chen, and Yi-Hsuan Yang (2022). \textit{Exploiting Pre-trained Feature Networks for Generative Adversarial Networks in Audio-domain Loop Generation}. International Society for Music Information Retrieval Conference.

\bibitem{gonzlezprieto20214wh}
Ángel González-Prieto, Alberto Mozo, Edgar Talavera, et al. (2021). \textit{Dynamics of Fourier Modes in Torus Generative Adversarial Networks}. Mathematics.

\bibitem{huang2022zar}
Ying Huang, Wenhao Mei, Su Liu, et al. (2022). \textit{Asymmetric Training of Generative Adversarial Network for High Fidelity SAR Image Generation}. IEEE International Geoscience and Remote Sensing Symposium.

\bibitem{wang2020iia}
Chunzhi Wang, Pan Wu, Lingyu Yan, et al. (2020). \textit{Image classification based on principal component analysis optimized generative adversarial networks}. Multimedia tools and applications.

\bibitem{ma2021w69}
Ruixin Ma, and Junying Lou (2021). \textit{CPGAN : An Efficient Architecture Designing for Text-to-Image Generative Adversarial Networks Based on Canonical Polyadic Decomposition}. Scientific Programming.

\bibitem{baby2020e5n}
Deepak Baby (2020). \textit{iSEGAN: Improved Speech Enhancement Generative Adversarial Networks}. arXiv.org.

\bibitem{pasini2021ta3}
Massimiliano Lupo Pasini, and Junqi Yin (2021). \textit{Stable parallel training of Wasserstein conditional generative adversarial neural networks}. 2021 International Conference on Computational Science and Computational Intelligence (CSCI).

\bibitem{goyal2024ufg}
Mandeep Goyal, and Q. Mahmoud (2024). \textit{A Systematic Review of Synthetic Data Generation Techniques Using Generative AI}. Electronics.

\bibitem{wang2024v83}
Shuzhan Wang, Ruxue Jiang, Zhaoqi Wang, et al. (2024). \textit{Deep Learning-based Anomaly Detection and Log Analysis for Computer Networks}. arXiv.org.

\bibitem{liao20249ku}
Wenjie Liao, Like Wu, Shihui Xu, et al. (2024). \textit{A Novel Approach for Intelligent Fault Diagnosis in Bearing With Imbalanced Data Based on Cycle-Consistent GAN}. IEEE Transactions on Instrumentation and Measurement.

\bibitem{peng2024kkw}
Yingying Peng (2024). \textit{A Comparative Analysis Between GAN and Diffusion Models in Image Generation}. Transactions on Computer Science and Intelligent Systems Research.

\bibitem{luo2024znt}
Yihong Luo, Xiaolong Chen, Tianyang Hu, et al. (2024). \textit{You Only Sample Once: Taming One-Step Text-To-Image Synthesis by Self-Cooperative Diffusion GANs}. International Conference on Learning Representations.

\bibitem{chen2024ajr}
Xin Chen, Zaigang Chen, Shiqian Chen, et al. (2024). \textit{Unsupervised GAN With Fine-Tuning: A Novel Framework for Induction Motor Fault Diagnosis in Scarcely Labeled Sample Scenarios}. IEEE Transactions on Instrumentation and Measurement.

\bibitem{song2024htg}
Xiangjin Song, Zhicheng Liu, and Zhaowei Wang (2024). \textit{Rolling bearing fault diagnosis in electric motors based on IDIG-GAN under small sample conditions}. Measurement science and technology.

\bibitem{qin2024a4b}
Zhaohui Qin, Faguo Huang, Jiafang Pan, et al. (2024). \textit{Improved Generative Adversarial Network for Bearing Fault Diagnosis with a Small Number of Data and Unbalanced Data}. Symmetry.

\bibitem{tibermacine2025pye}
Imad Eddine Tibermacine, Samuele Russo, Francesco Citeroni, et al. (2025). \textit{Adversarial denoising of EEG signals: a comparative analysis of standard GAN and WGAN-GP approaches}. Frontiers in Human Neuroscience.

\bibitem{baoueb2024rlq}
Teysir Baoueb, Haocheng Liu, Mathieu Fontaine, et al. (2024). \textit{SpecDiff-GAN: A Spectrally-Shaped Noise Diffusion GAN for Speech and Music Synthesis}. IEEE International Conference on Acoustics, Speech, and Signal Processing.

\bibitem{broll2024edy}
Alexander Broll, M. Rosentritt, Thomas Schlegl, et al. (2024). \textit{A data-driven approach for the partial reconstruction of individual human molar teeth using generative deep learning}. Frontiers Artif. Intell..

\bibitem{wang20245dt}
Yumiao Wang, Chuanfei Zang, Bo Yu, et al. (2024). \textit{WTE-CGAN Based Signal Enhancement for Weak Target Detection}. IEEE Geoscience and Remote Sensing Letters.

\bibitem{megahed2024c23}
Mohammed Megahed, and Ammar Mohammed (2024). \textit{Collaborative-GAN: An Approach for Stabilizing the Training Process of Generative Adversarial Network}. IEEE Access.

\bibitem{zhang2024k8a}
Xiurong Zhang, Shaoqian Fan, and Daoliang Li (2024). \textit{Spectral normalization generative adversarial networks for photovoltaic power scenario generation}. IET Renewable Power Generation.

\bibitem{bhat202445j}
Ranjith Bhat, and Raghu Nanjundegowda (2024). \textit{A Review on Comparative Analysis of Generative Adversarial Networks’ Architectures and Applications}. Journal of Robotics and Control (JRC).

\bibitem{ler20248xg}
Fiete Lüer, and Christian Böhm (2024). \textit{Anomaly Detection using Generative Adversarial Networks Reviewing methodological progress and challenges}. SIGKDD Explorations.

\bibitem{purwono2025spz}
Purwono Purwono, Annastasya Nabila Elsa Wulandari, Alfian Ma’arif, et al. (2025). \textit{Understanding Generative Adversarial Networks (GANs): A Review}. Control Systems and Optimization Letters.

\bibitem{roy2024k91}
Arunava Roy, and Dipankar Dasgupta (2024). \textit{A Distributed Conditional Wasserstein Deep Convolutional Relativistic Loss Generative Adversarial Network With Improved Convergence}. IEEE Transactions on Artificial Intelligence.

\bibitem{seon202526r}
Joonho Seon, Seongwoo Lee, Youngghyu Sun, et al. (2025). \textit{Least Information Spectral GAN With Time-Series Data Augmentation for Industrial IoT}. IEEE Transactions on Emerging Topics in Computational Intelligence.

\bibitem{ni2024y70}
Yao Ni, and Piotr Koniusz (2024). \textit{$\bigcirc\!\!\!\!\bigcirc$ CHAIN: Enhancing Generalization in Data-Efficient GANs via LipsCHitz Continuity ConstrAIned Normalization}. Computer Vision and Pattern Recognition.

\bibitem{ye2024n41}
Ming Ye, Cunhua Pan, Yinfei Xu, et al. (2024). \textit{Generative Adversarial Networks-Based Channel Estimation for Intelligent Reflecting Surface Assisted mmWave MIMO Systems}. IEEE Transactions on Cognitive Communications and Networking.

\bibitem{pajuhanfard2024ult}
Mohammadsaleh Pajuhanfard, Rasoul Kiani, and Victor S. Sheng (2024). \textit{Survey of Quantum Generative Adversarial Networks (QGAN) to Generate Images}. Mathematics.

\bibitem{eskandarinasab202431h}
MohammadReza EskandariNasab, S. M. Hamdi, and S. F. Boubrahimi (2024). \textit{ChronoGAN: Supervised and Embedded Generative Adversarial Networks for Time Series Generation}. International Conference on Machine Learning and Applications.

\bibitem{deebani202549r}
Wejdan Deebani, Lubna Aziz, Arshad Aziz, et al. (2025). \textit{Synergistic transfer learning and adversarial networks for breast cancer diagnosis: benign vs. invasive classification}. Scientific Reports.

\bibitem{ali2024ks3}
Abid Ali, Muhammad Sharif, Muhammad Shahzad Faisal, et al. (2024). \textit{Brain Tumor Segmentation Using Generative Adversarial Networks}. IEEE Access.

\bibitem{ju2024uai}
Xiangui Ju, Chi-Ho Lin, Suan Lee, et al. (2024). \textit{Melanoma classification using generative adversarial network and proximal policy optimization}. Photochemistry and Photobiology.

\bibitem{xu2024u5a}
Chi Xu, Haozheng Xu, and S. Giannarou (2024). \textit{Distance Regression Enhanced With Temporal Information Fusion and Adversarial Training for Robot-Assisted Endomicroscopy}. IEEE Transactions on Medical Imaging.

\bibitem{elbaz2025wzb}
Mostafa Elbaz, Wael Said, G. Mahmoud, et al. (2025). \textit{A dual GAN with identity blocks and pancreas-inspired loss for renewable energy optimization}. Scientific Reports.

\bibitem{chang2024c0a}
Yuanhong Chang, Jinglong Chen, Rong Su, et al. (2024). \textit{Two-Phase Dual-Adversarial Agents With Multivariate Information for Unsupervised Anomaly Detection of IIoT-Edge Devices}. IEEE Internet of Things Journal.

\bibitem{guo2024y0l}
Pang Guo, and Yining Chen (2024). \textit{Enhanced Yield Prediction in Semiconductor Manufacturing: Innovative Strategies for Imbalanced Sample Management and Root Cause Analysis}. International Symposium on the Physical and Failure Analysis of Integrated Circuits.

\bibitem{peng2024crk}
Jun Peng, Kaiyi Chen, Yuqing Gong, et al. (2024). \textit{Cyclic Consistent Image Style Transformation: From Model to System}. Applied Sciences.

\end{thebibliography}

\end{document}