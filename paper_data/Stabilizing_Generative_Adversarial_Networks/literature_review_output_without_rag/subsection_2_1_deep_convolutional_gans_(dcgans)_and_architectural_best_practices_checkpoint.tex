\subsection*{Deep Convolutional GANs (DCGANs) and Architectural Best Practices}

Early Generative Adversarial Networks (GANs), as initially proposed by \cite{Goodfellow2014}, demonstrated the powerful concept of adversarial training but were notoriously difficult to stabilize, often suffering from issues like mode collapse and unstable convergence, particularly when relying on fully connected network architectures. This inherent instability limited their ability to generate high-quality and diverse images. A pivotal advancement in addressing these challenges came with the introduction of Deep Convolutional Generative Adversarial Networks (DCGANs) by \cite{Radford2015}, which established a set of crucial architectural guidelines that significantly improved GAN training stability and the visual fidelity of generated outputs.

DCGANs revolutionized GAN design by systematically integrating convolutional layers and adopting specific architectural choices that became standard practice for many subsequent generative models. The core innovation involved replacing traditional pooling layers with strided convolutions in the discriminator for downsampling and fractional-strided convolutions (also known as transposed convolutions) in the generator for upsampling. This allowed the networks to learn their own spatial downsampling and upsampling operations, rather than relying on fixed, non-learnable pooling functions. This architectural shift enabled the generator to produce images with higher spatial coherence and the discriminator to learn more robust features, leading to a substantial improvement in the quality and stability of generated images compared to earlier fully connected network designs.

Beyond the strategic use of convolutional layers, \cite{Radford2015} also highlighted the critical role of batch normalization. Batch normalization layers were strategically applied in both the generator and discriminator, with the notable exceptions of the generator's output layer and the discriminator's input layer. The primary benefit of batch normalization is its ability to stabilize learning by normalizing the input to each layer to have zero mean and unit variance. This prevents internal covariate shift, mitigates vanishing or exploding gradients, and allows for the training of deeper generative models. By ensuring a more consistent distribution of activations throughout the network, batch normalization significantly contributes to smoother training dynamics and more robust convergence.

The selection of activation functions was another key architectural recommendation from DCGANs. For the generator, the Rectified Linear Unit (ReLU) activation function was predominantly used in all layers, except for the output layer, which employed a Tanh activation. ReLU's sparse activation properties help in learning complex, non-linear mappings efficiently. In contrast, the discriminator utilized Leaky ReLU (LeakyReLU) activations for all layers. LeakyReLU addresses the "dying ReLU" problem by allowing a small, non-zero gradient when the unit is not active, ensuring that gradients can flow through the network even for negative inputs. This provides the discriminator with more robust gradient information, preventing saturation and aiding in the stability of the adversarial training process.

These architectural recommendations from \cite{Radford2015} quickly became an enduring foundation for achieving greater stability in generative models. By moving away from fully connected layers and embracing a fully convolutional architecture with specific normalization and activation strategies, DCGANs demonstrated that carefully designed network structures could dramatically improve GAN performance. While other contemporary research efforts also sought to improve GAN stability through different means, such as mode regularization \cite{che2016kho} or unrolled optimization objectives \cite{metz20169ir}, DCGANs provided a concrete, widely adopted architectural blueprint. Its contributions were pivotal in transforming GANs from a theoretically promising but practically challenging concept into a powerful tool capable of generating visually compelling images, setting the stage for future advancements in high-fidelity image synthesis. The principles established by DCGANs continue to influence the design of modern GAN architectures, underscoring their lasting impact on the field.