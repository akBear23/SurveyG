\subsection{3D-Aware Synthesis from 2D Latent Spaces (StyleGAN-NeRF)}

While Generative Adversarial Networks (GANs) have achieved remarkable success in synthesizing photorealistic 2D images, particularly with the advent of the StyleGAN series, their inherent lack of 3D understanding and multi-view consistency has posed a significant challenge for generating coherent 3D scenes and novel views. This subsection explores the innovative integration of highly stable and controllable 2D GANs, specifically the StyleGAN architecture, with 3D scene representation models like Neural Radiance Fields (NeRFs), enabling the generation of high-quality, controllable 3D scenes directly from disentangled 2D latent spaces.

The journey towards 3D-aware synthesis from 2D latent spaces is fundamentally rooted in the advancements of 2D generative models, particularly the StyleGAN family. The original StyleGAN \cite{Karras2019} revolutionized image synthesis by introducing a style-based generator architecture that leveraged adaptive instance normalization (AdaIN) to inject latent codes at multiple scales, leading to unprecedented control over visual attributes and highly disentangled latent spaces. This was further refined by StyleGAN2 \cite{Karras2020}, which addressed common artifacts and improved image quality through architectural modifications like removing the progressive growth and introducing path length regularization. The pinnacle of 2D image fidelity was arguably reached with StyleGAN3 \cite{Karras2021}, which focused on achieving alias-free synthesis, eliminating common texture stickiness and ensuring scale-invariance. These StyleGAN models established a robust framework for generating high-fidelity, semantically controllable 2D images, where traversing the latent space could smoothly interpolate between distinct visual features such as pose, expression, and lighting.

Despite their impressive capabilities in 2D, these StyleGAN models inherently operate in a 2D image domain, meaning they lack an explicit understanding of 3D geometry or scene structure. Generating consistent novel views of an object or scene, or directly manipulating 3D properties like camera pose or object shape, remained outside their direct scope. The challenge thus became how to leverage the rich, disentangled latent representations learned by StyleGANs for 2D images and extend them to coherent and consistent 3D representations. This problem, of bridging the gap between high-quality 2D synthesis and 3D scene understanding, became a critical frontier in generative modeling.

A pivotal methodological progression in this direction is presented by \cite{Chan2023}, which introduces a novel approach to integrate StyleGAN's latent space with Neural Radiance Fields (NeRFs). NeRFs \cite{Mildenhall2020} (implicitly referenced as the 3D representation model) represent 3D scenes as continuous volumetric functions that output color and density at any point in space, enabling photorealistic novel view synthesis. \cite{Chan2023} effectively bridges the gap by demonstrating how the highly disentangled and semantically rich latent codes from pre-trained 2D StyleGANs can be used to control the generation of 3D-aware scenes via NeRFs. The key innovation lies in mapping the StyleGAN latent space to parameters that define a NeRF, allowing for the generation of high-quality 3D-aware images and novel views. This integration leverages StyleGAN's established disentanglement for intuitive 3D control, meaning that manipulations in the 2D latent space now correspond to meaningful changes in the generated 3D scene, such as object identity, pose, or lighting. This approach effectively extends the high-fidelity and controllability of 2D StyleGANs into the third dimension, addressing the limitations of prior 2D-only generators by endowing them with multi-view consistency and 3D scene understanding.

The integration presented by \cite{Chan2023} represents a significant step towards pushing the boundaries of generative modeling into the third dimension. It offers new avenues for content creation and virtual environment generation by allowing users to synthesize complex 3D scenes with the same ease and control previously only available for 2D images. However, challenges remain, including the computational cost associated with NeRF rendering, the potential for artifacts in complex scene compositions, and the fidelity gap when compared to real-world 3D scans. Future directions may involve improving the efficiency of 3D representation, enhancing the realism of generated textures and materials, and extending the framework to enable more complex scene editing and interaction.