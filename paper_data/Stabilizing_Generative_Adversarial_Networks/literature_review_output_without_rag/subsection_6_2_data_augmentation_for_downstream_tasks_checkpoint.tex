\subsection{Data Augmentation for Downstream Tasks}

Data scarcity remains a pervasive challenge across numerous machine learning applications, particularly in sensitive domains where data collection is difficult or costly. Stabilized Generative Adversarial Networks (GANs) have emerged as a crucial tool to address this by generating high-quality synthetic data, significantly enhancing the performance, generalization, and robustness of downstream machine learning models.

The utility of GANs for data augmentation is fundamentally predicated on their training stability and ability to produce diverse, realistic samples. Early efforts to stabilize GAN training, such as \textcite{metz20169ir}'s introduction of unrolled GANs, were pivotal in mitigating mode collapse and increasing the diversity and coverage of generated data distributions. This foundational work enabled GANs to produce more varied synthetic samples, a critical requirement for effective data augmentation. Building upon these stability improvements, the Identity Generative Adversarial Network (IGAN) proposed by \textcite{fathallah20236k5} further enhanced training stability and image diversity. IGAN incorporated non-linear identity blocks, a modified loss function with label smoothing, and minibatch training, allowing the model to fit complex data types more efficiently and produce higher-quality, more varied images. This made GANs more robust for general data synthesis, laying groundwork for their application in data augmentation. Similarly, \textcite{pasini2021ta3} addressed scalability and mode collapse in large-scale data generation by proposing a stable, parallel approach for training Wasserstein Conditional GANs (W-CGANs). Their method utilized multiple generators concurrently trained on single data labels, reducing inter-process communications and leveraging the Wasserstein metric to stabilize training and enhance image quality.

Leveraging these advancements in GAN stability, their application for data augmentation has proven particularly impactful in domains facing severe data limitations. In medical imaging, for instance, \textcite{wei2021qea} demonstrated the efficacy of GAN-based data augmentation in improving cancer classification. By generating synthetic cancer data highly similar to real samples, their approach directly addressed the inadequacy of available datasets, leading to enhanced classification performance and generalization of models in this critical area. This application highlights how stabilized GANs can provide diverse and realistic samples necessary for robust model training in privacy-sensitive and data-scarce medical contexts.

Further pushing the boundaries of GAN-based data augmentation, \textcite{elbaz2025wzb} introduced Penca-GAN, a novel architecture specifically designed for renewable energy optimization. This work addresses the persistent challenges of mode collapse, vanishing gradients, and pixel integrity in data-scarce domains like solar and wind energy. Penca-GAN integrates identity blocks for training stabilization and smoother gradient flow, alongside a novel dual loss function and a unique pancreas-inspired metaheuristic loss. This biologically-inspired loss dynamically adapts to training data variations, ensuring pixel integrity and promoting diversity, thereby significantly mitigating mode collapse and improving the quality of synthetic data for downstream tasks like fault detection and energy prediction. The superior performance of Penca-GAN, achieving lower FID and higher IS scores across various renewable energy datasets, demonstrates its effectiveness in generating high-fidelity synthetic images that directly improve the accuracy of fault detection models. This exemplifies how advanced stabilization techniques are being tailored and integrated for real-world impact in critical scientific and engineering applications. Beyond direct data generation, GANs also contribute to controlled data transformation. \textcite{peng2024crk} proposed C3GAN, an image style transfer method based on the CycleGAN architecture, which leverages cyclic consistency to achieve stable and coherent style transformations. While focused on artistic creation, this approach can be viewed as a form of data augmentation by transforming existing data to new styles, thereby expanding dataset diversity for specific visual attributes.

The continuous evolution of GAN architectures and training methodologies, as comprehensively reviewed by \textcite{purwono2025spz}, underscores the ongoing efforts to overcome challenges such as mode collapse, training instability, and quality evaluation. The works discussed herein collectively demonstrate that stabilized GANs are not merely tools for generating realistic images, but powerful instruments for enhancing data availability and model resilience in real-world scenarios. Future research will likely focus on developing even more adaptive and domain-specific stabilization mechanisms, integrating multimodal data augmentation, and addressing the ethical implications of synthetic data generation to ensure responsible deployment across diverse applications.