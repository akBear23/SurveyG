\subsection*{Scope and Overview of Stabilization Strategies}

The inherent instability of Generative Adversarial Networks (GANs) during training, often manifesting as mode collapse or non-convergence, has necessitated the development of a diverse array of stabilization strategies. These approaches span fundamental modifications to objective functions, sophisticated regularization techniques, innovative architectural designs, and advanced training paradigms, collectively aiming to make GANs more robust, reliable, and capable of generating high-quality, diverse outputs. This section provides a comprehensive roadmap of these multi-faceted strategies, highlighting their contributions to advancing GAN performance across various domains.

Early efforts to stabilize GAN training primarily focused on refining the objective function and introducing regularization to improve the optimization landscape. The original GAN formulation suffered from vanishing gradients and mode collapse due to the Jensen-Shannon divergence. To address this, \textcite{arjovsky2017ze5} introduced Wasserstein GAN (WGAN), which utilizes the Earth-Mover (Wasserstein) distance as a loss function, providing a smoother gradient and a more meaningful metric of generator performance. Building on this, alternative loss functions were explored, such as the Least Squares Generative Adversarial Networks (LSGANs) proposed by \textcite{mao2017ss0}, which adopt a least squares loss to mitigate vanishing gradients and achieve more stable training and higher quality images compared to traditional GANs. Complementing these loss function changes, regularization techniques became crucial. \textcite{miyato2018arc} introduced Spectral Normalization (SN-GAN), a computationally light weight normalization technique that directly constrains the Lipschitz constant of the discriminator's layers, proving highly effective in stabilizing training and improving sample quality. Further theoretical insights into GAN convergence and the effectiveness of gradient penalties were provided by \textcite{mescheder2018}, which analyzed various training methods and their convergence properties. Other regularization methods include Mode Regularized GANs \textcite{che2016kho}, which regularize the objective to encourage a fairer distribution of probability mass across data modes, and the Binarized Representation Entropy (BRE) regularization by \textcite{cao20184y8}, which guides the discriminator to better allocate its model capacity by encouraging high joint entropy of activation patterns. The theoretical underpinnings for these techniques, clarifying the need for Lipschitz constraints and gradient penalties, were further elaborated by \textcite{chu2020zbv} and \textcite{xu2019uwg}, who applied control theory to model GAN dynamics and propose stabilization through L2 regularization on the discriminator output.

Beyond objective function and regularization, significant advancements have been made through innovative architectural designs and progressive training paradigms. A pivotal contribution in this area is the Progressive Growing of GANs (PGGAN) by \textcite{karras2017raw}, which introduced a training methodology where both the generator and discriminator progressively grow in resolution, starting from low-resolution images and gradually adding layers for finer details. This approach not only speeds up training but also significantly stabilizes it, enabling the generation of unprecedentedly high-quality images. Similarly, \textcite{zhang2016mm0} proposed StackGAN, a multi-stage architecture that decomposes the complex problem of text-to-image synthesis into manageable sub-problems, refining images progressively. To address the challenge of mode collapse and instability, \textcite{metz20169ir} introduced Unrolled Generative Adversarial Networks, where the generator's objective is defined with respect to an unrolled optimization of the discriminator, allowing for a more stable training process. More recently, architectural innovations have involved multiple discriminators: \textcite{wei2021gla} presented DuelGAN, which introduces a peer discriminator that "duels" with the primary discriminator to discourage agreement and enhance sample diversity, thereby mitigating early mode collapse. Following a similar vein, \textcite{zhang2021ypi} proposed Twin Discriminator Generative Adversarial Networks (TWGAN), combining saturating and non-saturating losses with two discriminators to exploit complementary statistical properties for improved training stability and sample diversity. Furthermore, \textcite{gan202494y} introduced a learnable auxiliary module with an auxiliary penalty and augmented discriminator to enhance the stability of both networks during noise-to-image synthesis.

Advanced training paradigms have also emerged to tackle specific challenges, such as data efficiency and fine-grained control. For scenarios with limited training data, which often lead to discriminator overfitting and training divergence, \textcite{zhao2020xhy} proposed Differentiable Augmentation (DiffAugment). This method applies various types of differentiable augmentations to both real and fake samples, effectively stabilizing training and achieving impressive results with fewer images. This concept was further refined by \textcite{karras202039x} with Adaptive Discriminator Augmentation (ADA), demonstrating that good results are possible with only a few thousand images. Other training strategies include InfoMax-GAN \textcite{lee20205ue}, which employs contrastive learning and mutual information maximization to simultaneously mitigate catastrophic forgetting in the discriminator and mode collapse in the generator. For faster and stabilized few-shot image synthesis, \textcite{liu20212c2} introduced FastGAN, a lightweight architecture leveraging skip-layer channel-wise excitation and a self-supervised discriminator. The concept of transfer learning between networks for stability was explored by \textcite{megahed2024c23} in Collaborative-GAN, where well-performing networks transfer learned weights to underperforming ones to mitigate instability and mode collapse. Theoretical work by \textcite{salmona202283g} has also shed light on the inherent trade-offs, demonstrating that constraining the Lipschitz constant (a common stabilization technique) can limit the ability of push-forward generative models to fit multimodal distributions, thus highlighting a tension between stability and diversity.

In conclusion, the evolution of GAN stabilization strategies reflects a continuous effort to overcome fundamental optimization challenges and push the boundaries of generative model capabilities. From foundational adjustments to loss functions and regularization, through innovative architectural designs like progressive growing and multi-discriminator setups, to sophisticated training paradigms for data efficiency and enhanced control, research has progressively built upon earlier insights. While significant strides have been made in achieving high-fidelity and diverse image generation, the field continues to grapple with balancing stability, quality, diversity, and computational efficiency, particularly in novel domains and under resource constraints. Future research will likely continue to explore these interdependencies, aiming for more robust, efficient, and universally applicable GAN training methodologies.