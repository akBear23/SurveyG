\subsection{Large-Scale Training (BigGAN) for Diversity and Fidelity}

Achieving both high fidelity and broad diversity in generated images has long been a central challenge in generative adversarial networks (GANs), with early models often struggling with training instability and mode collapse, where the generator produces only a limited variety of outputs \cite{peng2024kkw}. The advent of large-scale training, epitomized by BigGAN, marked a transformative period, demonstrating that with sufficient computational resources and meticulous engineering, GANs could synthesize images of unprecedented realism and variety across complex datasets.

The foundation for BigGAN's success was laid by significant advancements in stabilizing GAN training. Early efforts to mitigate mode collapse and improve stability included regularization techniques such as those proposed by \cite{che2016kho}, which aimed to regularize the objective to ensure a fairer distribution of probability mass across data modes. A pivotal shift occurred with the introduction of Wasserstein GAN (WGAN) \cite{Arjovsky2017}, which replaced the problematic Jensen-Shannon divergence with the Earth-Mover distance, providing a more stable and meaningful loss function. This was further refined by WGAN with Gradient Penalty (WGAN-GP) \cite{Gulrajani2017}, which introduced a robust gradient penalty to enforce the Lipschitz constraint on the discriminator, becoming a widely adopted technique for stable training. The importance of these foundational stability methods is underscored by their continued application in diverse fields, such as channel estimation in wireless communications, where improved Wasserstein GANs are integrated to enhance training stability and learning ability \cite{hu2021yk5, ye2024n41}.

Building upon these stability breakthroughs, Spectral Normalization (SN) emerged as a particularly effective and computationally efficient method for regularizing the discriminator. Introduced by \cite{Miyato2018}, SN directly controls the Lipschitz constant of each layer by normalizing the spectral norm of its weight matrices, offering a robust alternative to gradient penalties for ensuring stable discriminator training. This technique proved crucial for scaling GANs to larger architectures without encountering gradient explosion or vanishing issues.

The culmination of these advancements was the development of BigGAN by \cite{Brock2019}, which showcased the immense potential of scaling GANs for high-fidelity natural image synthesis. BigGAN meticulously combined existing stabilization techniques, most notably Spectral Normalization, with several architectural improvements and training strategies. Key architectural innovations included the integration of self-attention mechanisms, which allowed the generator and discriminator to model long-range dependencies across image regions, and shared embeddings for class-conditional generation, enabling the model to leverage class information more effectively. Furthermore, BigGAN leveraged significantly larger batch sizes and model capacities, pushing the boundaries of what was computationally feasible at the time.

A critical innovation introduced by BigGAN was the "truncation trick," a post-training technique that allowed for a flexible trade-off between image fidelity and diversity. By sampling latent vectors from a truncated normal distribution rather than a standard normal distribution, BigGAN could generate images of exceptionally high fidelity, albeit with slightly reduced diversity, or vice versa. This meticulous combination of robust regularization, architectural enhancements, massive computational scale, and the truncation trick enabled BigGAN to achieve unprecedented levels of image fidelity and diversity on vast, complex datasets like ImageNet. It demonstrated that with sufficient scale and careful engineering, GANs could generate highly realistic and varied images across a wide range of categories, setting a new benchmark for generative models.

BigGAN's success firmly established the paradigm that leveraging substantial computational resources, coupled with a deep understanding of GAN training dynamics and architectural design, could lead to dramatic improvements in generative performance. While BigGAN achieved remarkable fidelity and diversity, the computational demands for training such models remained substantial. Subsequent research, such as the StyleGAN series, would further refine generator architectures to enhance control and disentanglement, building upon the high-fidelity generation capabilities first demonstrated at scale by BigGAN.