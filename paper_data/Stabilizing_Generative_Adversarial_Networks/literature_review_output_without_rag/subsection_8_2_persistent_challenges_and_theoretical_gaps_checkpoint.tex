\subsection{Persistent Challenges and Theoretical Gaps}

Despite significant advancements in stabilizing Generative Adversarial Networks (GANs) and enhancing their generative capabilities, several fundamental challenges and theoretical limitations continue to impede their widespread and robust application. These issues underscore that GAN training remains a complex, non-convex optimization problem with active and pressing research needs.

One of the most persistent challenges in GAN research is achieving perfect mode coverage and ensuring the generator captures the entire diversity of the real data distribution. Early GANs were notoriously prone to mode collapse, where the generator produces only a limited subset of the true data distribution's modes \cite{che2016kho}. While foundational works like Wasserstein GANs with gradient penalty (WGAN-GP) \cite{gulrajani2017} significantly improved training stability and mitigated some forms of mode collapse compared to the original GAN formulation \cite{arjovsky2017}, the problem of comprehensive mode discovery remains. For instance, Mode Regularized GANs \cite{che2016kho} and Prescribed Generative Adversarial Networks (PresGANs) \cite{dieng2019rjn} were specifically designed to encourage broader mode coverage by regularizing the objective or adding noise, respectively. Similarly, InfoMax-GAN \cite{lee20205ue} and SGAN \cite{chavdarova20179w6} introduced mechanisms to improve diversity and prevent early mode collapse by leveraging mutual information maximization or ensemble training. More recently, DuelGAN \cite{wei2021gla} and TWGAN \cite{zhang2021ypi} employed multiple discriminators to discourage agreement and increase sample diversity. However, a theoretical limitation highlighted by \cite{salmona202283g} demonstrates a provable trade-off: push-forward generative models, which include many GAN architectures, require a large Lipschitz constant to fit multimodal distributions. This directly conflicts with common stabilization techniques that constrain the Lipschitz constant of the discriminator, implying an inherent tension between training stability and the ability to perfectly capture high-diversity, multimodal data. Even state-of-the-art architectures like the StyleGAN series \cite{karras2019, karras2020, karras2021}, while producing unprecedentedly high-fidelity images, often operate within specific domains (e.g., human faces) and are not explicitly designed to guarantee exhaustive mode coverage across arbitrary, highly diverse real-world distributions.

Another critical area of ongoing research concerns the struggle to obtain robust theoretical convergence guarantees for increasingly complex GAN architectures. The minimax game formulation of GANs is inherently challenging to optimize, and while empirical stability has improved, a comprehensive theoretical understanding of their convergence properties, especially for deep, non-convex networks, is still evolving. \cite{mescheder2018} provided a theoretical analysis of which training methods for GANs actually converge, but this often applies to specific formulations or under strong assumptions. \cite{liang2018r52} delved into the non-asymptotic local convergence of two-player games, revealing that the interaction term between generator and discriminator can be both beneficial and detrimental, complicating convergence analysis. Further, \cite{chu2020zbv} developed a principled theoretical framework for GAN stability, deriving conditions for generator stationarity, but noted that existing GAN variants only partially satisfy these conditions, indicating a theoretical gap between current practice and ideal stability. Attempts to model GAN dynamics using control theory \cite{xu2019uwg} also highlight the complexity, with practical stabilizing methods still being actively sought. While techniques like Spectral Normalization \cite{miyato2018} offer efficient ways to enforce Lipschitz constraints on the discriminator, contributing to empirical stability, global convergence guarantees for the entire system remain elusive, particularly as architectures become more intricate.

Furthermore, GANs continue to exhibit significant sensitivity to hyperparameter tuning, and researchers constantly grapple with inherent trade-offs between computational efficiency and generative quality. The introduction of regularization techniques, while beneficial for stability, often adds new hyperparameters that require careful tuning, such as the gradient penalty weight in WGAN-GP \cite{gulrajani2017}. Surveys on GANs consistently identify training instability and hyperparameter sensitivity as major obstacles \cite{jabbar2020aj0, wang2019w53, wiatrak20194ib}. Achieving high-fidelity synthesis, as demonstrated by models like BigGAN \cite{brock2019} and the StyleGAN family \cite{karras2019, karras2020, karras2021}, often comes at the cost of substantial computational resources, large batch sizes, and extensive datasets, making these models less accessible for researchers with limited computational budgets. This creates a practical trade-off between the desired quality and the feasibility of training. Efforts to mitigate this sensitivity include adaptive methods, such as the adaptive weighted discriminator proposed by \cite{zadorozhnyy20208ft} or the learnable auxiliary module in \cite{gan202494y}, which aim to reduce reliance on fixed hyperparameters by dynamically adjusting training components. Similarly, Collaborative-GAN \cite{megahed2024c23} proposes transfer learning between discriminators to stabilize training. Even in applications like polyphonic music generation, careful optimization of reinforcement learning signals is deemed crucial \cite{lee2017zsj}, underscoring the pervasive nature of hyperparameter sensitivity across different data modalities. Other approaches like Constrained GANs \cite{chao2021ynq} and Binarized Representation Entropy (BRE) Regularization \cite{cao20184y8} introduce new regularization terms or constraints to improve stability, which themselves may require careful tuning.

In conclusion, despite remarkable progress in generative quality and empirical stability, GAN training remains a profoundly complex optimization problem. The quest for perfect mode coverage, robust theoretical convergence guarantees, reduced hyperparameter sensitivity, and a better balance between computational efficiency and generative quality continues to drive active research. Addressing these persistent challenges will be crucial for unlocking the full potential of GANs across a broader spectrum of real-world applications.