\subsection{Motivation for Knowledge Graph Embedding}

The rapid growth and widespread adoption of knowledge graphs (KGs) across diverse domains, from scientific discovery to enterprise intelligence, have highlighted an urgent need for advanced methods to effectively represent, process, and reason with structured information. Traditional symbolic representations of KGs, typically expressed as Resource Description Framework (RDF) triples (e.g., (entity, relation, entity)), possess inherent limitations despite their interpretability and precision \cite{yan2022}. These limitations fundamentally impede their utility in modern, data-driven applications. Foremost among these is extreme data sparsity, where the vast majority of potential facts are unobserved, making robust statistical inference and pattern discovery challenging. Furthermore, symbolic reasoning, while logically sound, often suffers from computational inefficiency when applied to large-scale KGs, leading to intractable performance for complex queries or exhaustive inference tasks. Critically, symbolic representations pose a significant barrier to seamless integration with contemporary statistical machine learning (ML) models, which predominantly operate on continuous vector spaces rather than discrete symbols \cite{cao2022}.

Knowledge Graph Embedding (KGE) emerges as a pivotal paradigm designed to overcome these challenges by transforming discrete symbolic entities and relations into dense, low-dimensional continuous vector representations. This transformation acts as a vital bridge, enabling KGs to be leveraged effectively by advanced ML pipelines, thereby facilitating scalability, generalization, and seamless integration for a myriad of downstream tasks \cite{yan2022}. The core motivation for KGE lies in its ability to convert discrete symbolic knowledge into a format amenable to numerical computation and statistical learning. By embedding entities and relations into a continuous vector space, KGE models can capture latent semantic relationships and implicit patterns that are otherwise obscured or difficult to discern in sparse symbolic graphs. This dense representation significantly enhances computational efficiency, allowing for large-scale operations such as similarity calculations, approximate reasoning, and efficient graph traversal that would be intractable with purely symbolic methods \cite{do20184o2}. Moreover, KGEs provide a unified, flexible interface for integrating structured knowledge with diverse machine learning models, ranging from deep neural networks for natural language processing to recommendation systems, enabling the development of more intelligent and context-aware applications. This capability is particularly advantageous for fundamental tasks such as link prediction, where missing facts can be inferred by analyzing vector proximities; entity alignment, which identifies equivalent entities across different KGs; and question answering, where complex queries can be resolved by navigating the embedded knowledge space \cite{portisch20221rd}.

Beyond improving efficiency and ML integration, KGE is also motivated by the need for enhanced generalization and adaptability in dynamic environments. Traditional symbolic systems often struggle to infer facts about unseen entities or relations (inductive learning) without explicit rules or prior observations. KGE, by learning generalizable patterns in a continuous space, can inherently support inductive capabilities, allowing for the generation of embeddings for new entities and relations based on their local graph structure or associated textual information \cite{lee202380l}. This is crucial for real-world KGs that are constantly evolving with new facts, entities, and relationships. Furthermore, the inherent flexibility of continuous vector spaces allows KGE models to better represent the nuanced, often fuzzy, and temporal nature of real-world knowledge, where facts may have varying degrees of certainty or validity over time. While the initial KGE models provided static representations, the underlying principle of continuous embedding lays the groundwork for more sophisticated models that can capture these dynamic and uncertain aspects, moving beyond the rigid, static constraints of purely symbolic logic. The ability of KGE to abstract symbolic knowledge into a mathematically tractable form thus enables a richer, more robust, and adaptable representation of knowledge, which is indispensable for building advanced AI systems.

In conclusion, the motivation for Knowledge Graph Embedding is multifaceted, stemming from the fundamental limitations of symbolic representations in terms of sparsity, computational scalability, and difficulty in integrating with modern machine learning paradigms. KGE provides a powerful solution by transforming discrete knowledge into dense vector spaces, enabling efficient reasoning, enhanced generalization, and seamless integration with diverse AI applications. This foundational capacity to represent symbolic knowledge in a continuous space has given rise to a rich and evolving landscape of embedding models, which this review will now proceed to organize and explore, detailing their intellectual trajectory, diverse applications, and emerging trends.