\subsection{Temporal Knowledge Graph Embedding}

The integration of temporal dynamics into Knowledge Graph Embedding (KGE) has emerged as a critical area of research, addressing the limitations of static representations that fail to capture the evolving nature of knowledge. Temporal Knowledge Graphs (TKGs) encapsulate facts that are valid only within specific time frames, necessitating models that can effectively represent these temporal aspects while maintaining the expressiveness and interpretability of embeddings.

A foundational work in this domain is HyTE, which introduced a hyperplane-based approach to incorporate temporal information into KGE. By associating each timestamp with a hyperplane, HyTE enables temporally guided inference and the prediction of temporal scopes for relational facts, thereby addressing the challenge of static representations in KGs \cite{dasgupta2018}. However, while HyTE effectively models temporal validity, it may struggle with the complexities of non-linear temporal patterns and the scalability of hyperplane representations.

Building on the insights of HyTE, ATiSE proposed a novel framework based on additive time series decomposition to model the evolution of entity and relation representations as multi-dimensional Gaussian distributions. This approach explicitly accounts for temporal uncertainty, a significant advancement over deterministic models that assume fixed representations over time \cite{xu2019}. The incorporation of Gaussian distributions allows ATiSE to capture the inherent randomness in the evolution of knowledge, thus enhancing its applicability in dynamic environments. However, the reliance on Gaussian assumptions may limit its effectiveness in scenarios with highly volatile temporal dynamics.

Further extending the temporal embedding paradigm, TeRo introduced a time-aware KGE model that employs rotations in complex space to represent temporal evolution. This method allows for the modeling of diverse relation patterns and effectively handles varying time annotations, including time intervals \cite{xu2020}. By leveraging the geometric properties of rotations, TeRo overcomes some limitations of earlier models, such as HyTE and ATiSE, which may not adequately capture complex relational dynamics. Nevertheless, the complexity of the model may increase with higher dimensions, posing challenges for interpretability and computational efficiency.

ChronoR, another significant contribution, enhances the modeling of TKGs by integrating k-dimensional rotation transformations with temporal information. It introduces a novel scoring function based on inner products, which is more robust in high-dimensional spaces compared to traditional Euclidean distance metrics \cite{sadeghian2021}. This advancement allows ChronoR to effectively model complex interactions between temporal and relational characteristics, addressing some of the shortcomings of earlier models. However, the reliance on high-dimensional rotations may introduce additional computational overhead.

Recent developments, such as the TeAST model, leverage an Archimedean spiral timeline to represent temporal relations, transforming the quadruple completion problem into a 3rd-order tensor completion task \cite{li2023}. This innovative approach not only facilitates the handling of temporal dynamics but also enhances interpretability by structuring the representation of relations over time. Despite its advancements, the model's effectiveness in capturing highly complex temporal interactions remains to be fully explored.

In conclusion, while significant strides have been made in the realm of temporal knowledge graph embedding, challenges persist in addressing the complexities of dynamic knowledge representation. Future research directions may focus on developing hybrid models that combine the strengths of existing approaches, enhancing scalability, interpretability, and the ability to capture intricate temporal patterns. The ongoing evolution of TKGs underscores the need for robust embedding techniques that can adapt to the fluid nature of real-world knowledge.
```