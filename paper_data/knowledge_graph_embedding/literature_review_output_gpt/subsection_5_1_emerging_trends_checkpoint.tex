```latex
\subsection*{Emerging Trends}

The field of Knowledge Graph Embedding (KGE) is undergoing significant transformation as it embraces advanced methodologies such as federated learning, continual learning, and the utilization of pre-trained language models. These trends are not only reshaping the landscape of KGE but also enhancing its applicability across various contexts, underscoring the need for adaptability in KGE methodologies.

Federated learning represents a paradigm shift in KGE by enabling decentralized model training across multiple data sources while preserving data privacy. This approach is particularly relevant in scenarios where knowledge graphs are generated from sensitive or proprietary data, such as in healthcare or finance. Recent studies, such as those by Zhang et al. \cite{zhang2021}, demonstrate that federated learning can effectively aggregate knowledge from distributed sources, leading to improved embedding quality without compromising data confidentiality. This trend highlights the potential for KGE to be applied in privacy-sensitive domains, allowing for collaborative learning while maintaining data integrity.

Continual learning is another emerging trend that addresses the challenge of model obsolescence in dynamic environments. Traditional KGE models often struggle to adapt to new information without retraining from scratch, leading to catastrophic forgetting of previously learned knowledge. Recent advancements, such as those presented by Alzubaidi et al. \cite{alzubaidi2022}, propose continual learning frameworks that enable KGE models to incrementally incorporate new facts while retaining prior knowledge. This adaptability is crucial for applications in fast-evolving fields, such as social media or news aggregation, where knowledge graphs must be updated frequently to remain relevant.

The integration of pre-trained language models into KGE is also gaining traction, as these models can enhance the semantic richness of embeddings. Approaches like BERT4KGE \cite{zhang2022} leverage the contextual understanding provided by pre-trained transformers to improve the quality of knowledge graph embeddings. By incorporating linguistic features and relationships learned from vast corpora, these models can better capture the nuances of entity relationships, leading to more accurate predictions in tasks such as link prediction and entity classification. This trend signifies a shift towards a more holistic understanding of knowledge, merging symbolic and neural methodologies.

While these advancements are promising, challenges remain in effectively integrating these methodologies into existing KGE frameworks. The complexity of federated learning, for instance, can lead to issues such as model divergence and communication overhead, which may hinder scalability \cite{yang2019}. Similarly, continual learning approaches often require sophisticated mechanisms to balance the retention of old knowledge with the incorporation of new information, raising questions about the optimal strategies for knowledge retention and transfer \cite{parisi2019}. Furthermore, the reliance on pre-trained models necessitates careful consideration of domain specificity, as embeddings derived from general corpora may not always align with the specialized knowledge represented in domain-specific graphs.

In conclusion, the emerging trends in KGE research underscore a significant shift towards methodologies that prioritize adaptability and robustness. The integration of federated learning, continual learning, and pre-trained language models presents exciting opportunities for enhancing the scalability and applicability of KGE across diverse domains. However, addressing the inherent challenges associated with these approaches is crucial for realizing their full potential in capturing the complexities of real-world knowledge. As the field continues to evolve, ongoing research will be essential to refine these methodologies and ensure their effective implementation in practical applications.

```