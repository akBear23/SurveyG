\subsection{Multi-Modal and Uncertainty Models}

The integration of multi-modal data and the consideration of uncertainty in Knowledge Graph Embedding (KGE) models are increasingly recognized as essential for enhancing the robustness and expressiveness of embeddings. Traditional KGE models often struggle to encapsulate the dynamic and uncertain nature of real-world data. Recent advancements, such as FSTRE and various fuzzy logic approaches, have emerged to address these challenges by incorporating diverse data sources and modeling inherent uncertainties in knowledge representation.

The Fuzzy Spatiotemporal RDF Knowledge Graph Embedding (FSTRE) model proposed by Ji et al. \cite{ji2024} exemplifies this trend. FSTRE introduces a fine-grained fuzzy spatiotemporal RDF model that leverages projection for spatial information and rotation for temporal information within a complex vector space. This dual approach allows FSTRE to effectively embed uncertain and dynamic knowledge, capturing rich interactions between static and fuzzy spatiotemporal data. The model's use of anisotropic vectors to integrate fine-grained fuzziness represents a significant innovation, enabling it to handle complex knowledge representations that traditional KGE models cannot.

Building on the need for temporal dynamics, Xu et al. \cite{xu2019} developed the Additive Time Series Embedding (ATiSE) model, which models the evolution of entity and relation representations as multi-dimensional additive time series. By representing entities and relations as Gaussian distributions, ATiSE explicitly accounts for temporal uncertainty, a critical advancement over earlier models that often treated time as a deterministic vector. This approach allows for a more nuanced understanding of how knowledge evolves over time, addressing limitations in existing temporal KGE models that fail to capture the complexity of temporal relationships.

Furthermore, the work by Sadeghian et al. \cite{sadeghian2021} with ChronoR builds upon the foundation laid by earlier models by introducing a rotation-based embedding technique specifically designed for temporal knowledge graphs. ChronoR's inner product scoring function captures rich interactions between temporal and relational characteristics, enhancing the model's ability to predict missing facts in TKGs. This model effectively addresses the shortcomings of static KGE approaches by integrating temporal dynamics directly into the embedding process.

In contrast, the TeRo model by Xu et al. \cite{xu2020} also emphasizes the importance of temporal information but does so through a unique rotation mechanism in complex space. TeRo's ability to handle various time annotations, including time intervals, showcases its flexibility in modeling temporal relations, which is often overlooked in other models. This highlights a critical gap in earlier methodologies that primarily focused on either static representations or simplistic temporal embeddings.

The progression from foundational models to those addressing multi-modal data and uncertainty illustrates a growing recognition of the complexities inherent in real-world knowledge representation. While models like FSTRE and ATiSE have made significant strides in integrating uncertainty and temporal dynamics, challenges remain in effectively combining these dimensions with other forms of knowledge representation. Future research should explore the potential of hybrid models that can seamlessly integrate multiple modalities while robustly addressing uncertainties, thereby enhancing the expressiveness and applicability of KGE in complex scenarios.

In conclusion, the evolution of KGE models towards incorporating multi-modal data and uncertainty reflects a critical shift in the field, driven by the need to adapt to the complexities of real-world data. As demonstrated by the advancements in models like FSTRE, ATiSE, and ChronoR, there is a clear trajectory towards more sophisticated approaches that can better capture the dynamic and uncertain nature of knowledge, paving the way for future innovations in knowledge graph representation and reasoning.
```