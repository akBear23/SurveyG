\subsection{Ethical Considerations}

Knowledge graph embedding (KGE) technologies have gained significant traction in various applications, yet they raise critical ethical concerns that warrant thorough examination. This subsection explores the implications of KGE concerning data privacy, bias, and transparency, emphasizing the necessity for responsible AI practices in this domain.

One of the foremost ethical concerns in KGE is data privacy. Many KGE models rely on vast amounts of data, often sourced from public and private repositories. For instance, models like TransE \cite{wang2014} and RotatE \cite{sun2018} utilize extensive datasets that may inadvertently include sensitive information about individuals or organizations. The potential for misuse of such data underscores the need for robust privacy-preserving techniques. Researchers must prioritize the development of KGE methods that incorporate differential privacy or similar frameworks to ensure that sensitive information remains protected while still enabling effective knowledge representation.

Bias is another critical issue associated with KGE. The embedding models can inadvertently perpetuate or even exacerbate existing biases present in the training data. For example, the work on entity alignment by BootEA \cite{sun2018} reveals that alignment results can be skewed if the underlying knowledge graphs contain biased representations of entities. This bias can lead to unfair treatment or misrepresentation of certain groups when KGE outputs are used in downstream applications, such as recommendation systems or search engines. Addressing this challenge requires the integration of fairness-aware algorithms and continuous monitoring of KGE outputs to identify and mitigate biases.

Transparency in KGE models is equally vital. The complexity of models, such as those proposed in HyTE \cite{dasgupta2018} and ChronoR \cite{sadeghian2021}, often obscures the decision-making processes involved in generating embeddings. This lack of interpretability can hinder user trust and accountability, particularly in critical applications like healthcare or legal systems. To counter this, researchers should focus on developing explainable KGE frameworks that elucidate how embeddings are derived and the rationale behind specific predictions. Techniques such as attention mechanisms or feature importance scoring could enhance transparency and foster greater user confidence in KGE applications.

The progression of KGE research reflects an increasing awareness of these ethical considerations. Early models, such as TransE \cite{wang2014}, primarily focused on performance metrics without addressing ethical implications. However, recent advancements, including the integration of temporal dynamics in embeddings via models like TeRo \cite{xu2020} and FSTRE \cite{ji2024}, signal a shift towards more responsible AI practices. These models not only enhance the representational capabilities of KGE but also open avenues for embedding ethical considerations into their design and implementation.

In conclusion, while KGE technologies hold immense potential for advancing knowledge representation, they also present significant ethical challenges related to data privacy, bias, and transparency. Future research must prioritize the development of frameworks that ensure fairness and accountability in knowledge representation, thereby fostering responsible AI practices in KGE applications. As the field evolves, it is imperative to continuously assess and address these ethical dimensions to safeguard against potential misuse and promote equitable outcomes.
```