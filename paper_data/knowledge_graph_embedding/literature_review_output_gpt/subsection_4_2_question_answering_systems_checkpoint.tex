\subsection{Question Answering Systems}

Question answering (QA) systems leverage knowledge graphs (KGs) to provide precise answers to user queries. A critical challenge in this domain is effectively retrieving relevant information from KGs, particularly when the underlying data is complex and multi-relational. Knowledge graph embeddings (KGE) have emerged as a powerful technique to address this challenge by transforming entities and relations into low-dimensional vector spaces, facilitating efficient information retrieval and reasoning.

The foundational work in KGE, such as TransE \cite{wang2014}, set the stage by proposing a translation-based approach for embedding entities and relations. However, this method struggled with complex relational patterns, such as one-to-many or many-to-many relationships. This limitation prompted the development of more sophisticated models. For instance, TransH \cite{wang2014} introduced hyperplanes to allow for relation-specific embeddings, thus enhancing the expressiveness of the embeddings. Following this, RotatE \cite{sun2018} advanced the field by modeling relations as rotations in complex space, enabling the simultaneous capture of various relational patterns, including symmetry and inversion.

Building on these foundational models, Huang et al. \cite{huang2019} introduced the Knowledge Embedding based Question Answering (KEQA) framework, which specifically targets the QA-KG problem. KEQA innovatively aims to jointly recover the representations of head entities, predicates, and tail entities from natural language questions within the KGE space. This approach addresses the inherent ambiguity and variability in natural language queries by leveraging the structured representations provided by KGE, thus facilitating more accurate answer retrieval.

Further advancements in KGE for QA can be seen in BootEA \cite{sun2018}, which applies KGE techniques to the entity alignment problem, crucial for integrating heterogeneous KGs that often underpin QA systems. This method addresses the scarcity of prior alignment data by employing a bootstrapping approach to iteratively refine entity alignments, thereby enhancing the overall quality of the KG and, consequently, the QA system's performance.

Despite these advancements, challenges remain in effectively capturing the temporal dynamics of knowledge, which are often critical for accurate QA. For instance, HyTE \cite{dasgupta2018} and ATiSE \cite{xu2019} both attempt to incorporate temporal information into KGE, highlighting the need for models that can handle the evolving nature of knowledge. HyTE utilizes hyperplanes to represent temporal validity, while ATiSE employs additive time series decomposition to capture the uncertainty in temporal evolution.

Recent work, such as ChronoR \cite{sadeghian2021} and TeRo \cite{xu2020}, further explores the integration of temporal dynamics into KGE by employing rotation-based transformations. These models enhance the expressiveness of embeddings by allowing for the modeling of complex temporal relationships, thus improving the performance of QA systems that rely on temporally dynamic data.

In conclusion, while significant progress has been made in integrating KGE with QA systems, challenges related to the complexity of relational patterns, the incorporation of temporal dynamics, and the effective alignment of heterogeneous KGs persist. Future research should focus on developing more robust frameworks that can dynamically adapt to the evolving nature of knowledge while maintaining high accuracy in information retrieval and reasoning tasks.
```