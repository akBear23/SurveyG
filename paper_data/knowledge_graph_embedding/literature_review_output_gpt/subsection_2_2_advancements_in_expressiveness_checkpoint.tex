\subsection{Advancements in Expressiveness}

The expressiveness of Knowledge Graph Embedding (KGE) models has evolved significantly, addressing the complexities inherent in relational data representation. Traditional models like TransE \cite{wang2014} and its successors often struggled to capture intricate relational patterns such as symmetry, inversion, and composition. This limitation prompted the development of more sophisticated models that leverage geometric interpretations and algebraic properties to enhance expressiveness and performance across various tasks.

One of the pivotal advancements in this domain is the introduction of RotatE \cite{sun2018}, which conceptualizes relations as rotations in complex space. This model allows for the simultaneous representation of diverse relational patterns, effectively capturing symmetry, antisymmetry, inversion, and composition. By defining relations as element-wise rotations, RotatE provides a unified framework that overcomes the limitations of earlier models like DistMult and ComplEx, which could not model all three crucial relation patterns concurrently. The self-adversarial negative sampling technique introduced in RotatE further enhances its training efficiency, allowing for more informative negative samples during the learning process.

Following RotatE, the model ComplEx \cite{trouillon2016} extended the capabilities of DistMult by incorporating complex embeddings to handle inversion and symmetry but fell short in addressing composition. This gap was subsequently bridged by models like TeRo \cite{xu2020}, which not only utilized the rotation concept but also introduced a mechanism to handle time-awareness in embeddings. TeRo's dual relation embeddings for time intervals exemplify how temporal dynamics can be effectively integrated into KGE, enhancing the model's ability to manage complex temporal relationships.

Moreover, the development of HyTE \cite{dasgupta2018} marked a significant leap by explicitly incorporating temporal information into KGE through hyperplane representations. HyTE associates timestamps with hyperplanes, allowing it to predict the temporal validity of relational facts. This innovation demonstrates how temporal dynamics can be integrated without sacrificing the model's expressiveness, addressing the shortcomings of static models that fail to account for the time-varying nature of knowledge.

In the realm of uncertainty and dynamic knowledge, FSTRE \cite{ji2024} further advanced the expressiveness of KGE by integrating fuzzy and spatiotemporal elements into embeddings. By employing projection and rotation techniques within a complex vector space, FSTRE captures the inherent uncertainty in dynamic knowledge, which is often overlooked in traditional KGE frameworks. This model exemplifies the ongoing trend towards incorporating multi-dimensional aspects into embeddings, thereby enhancing their applicability to real-world scenarios.

Despite these advancements, challenges remain in fully capturing the complex interplay of temporal, spatial, and uncertain elements in knowledge graphs. Future research directions could focus on developing hybrid models that combine the strengths of existing approaches while addressing their limitations. For instance, integrating the temporal modeling capabilities of HyTE with the rotational dynamics of RotatE and the uncertainty handling of FSTRE could yield more robust embeddings capable of representing the intricate nature of knowledge graphs.

In conclusion, the evolution of KGE models towards enhanced expressiveness showcases a significant shift from simplistic representations to more sophisticated methodologies that can accommodate the complexities of real-world data. As the field progresses, the integration of diverse relational patterns, temporal dynamics, and uncertainty will be crucial in developing models that are not only expressive but also practical for various applications in artificial intelligence.
```