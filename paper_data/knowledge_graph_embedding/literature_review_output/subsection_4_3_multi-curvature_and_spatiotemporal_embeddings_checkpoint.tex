\subsection{Multi-Curvature and Spatiotemporal Embeddings}
The inherent complexity of dynamic knowledge graphs, characterized by diverse geometric structures, temporal evolution, spatial attributes, and inherent uncertainties, necessitates embedding models that transcend the limitations of single Euclidean spaces. This subsection explores cutting-edge approaches that leverage multi-curvature geometries, integrate spatiotemporal information, and incorporate fuzziness to capture the multifaceted nature of real-world knowledge.

Early efforts to introduce geometric awareness into temporal knowledge graph embeddings (TKGE) include \cite{dasgupta2018}, which proposed HyTE, a hyperplane-based model that associates each timestamp with a hyperplane to enable temporally-guided inference and predict temporal scopes. Building on the need to capture more nuanced temporal dynamics, \cite{xu2019} introduced ATiSE, which modeled entity and relation evolution as additive time series with Gaussian distributions, marking a significant step towards explicitly accounting for temporal uncertainty. Similarly, \cite{lin2020} offered a tensor decomposition method, representing facts as a fourth-order tensor to inherently capture time, providing a generalizable framework. More recently, \cite{li2023} TeAST introduced a novel Archimedean spiral timeline for relations, transforming the problem into 3rd-order tensor completion to avoid direct entity evolution and enhance interpretability.

A significant paradigm shift in temporal modeling emerged with rotation-based embeddings, which leverage complex spaces to elegantly capture temporal evolution. \cite{xu2020} TeRo pioneered this by introducing temporal rotation in complex space for entity evolution, effectively handling diverse relation patterns and time intervals. This approach offered superior expressiveness for various relation patterns and temporal dynamics. Extending this, \cite{sadeghian2021} ChronoR generalized rotation to k-dimensions, proposing an inner product scoring function that theoretically encompasses complex-domain models like ComplEx, and introduced advanced regularization for temporal smoothness. These rotation-based models laid the groundwork for integrating more complex dimensions.

Recognizing that real-world knowledge graphs exhibit a variety of geometric structures (e.g., hierarchical, ring-like, chain-like) that cannot be adequately captured by a single Euclidean space, recent research has moved towards multi-curvature adaptive embeddings. \cite{wang2024} MADE addresses this limitation by modeling temporal knowledge graphs in multiple curvature spaces, specifically Euclidean (zero curvature), hyperbolic (negative curvature), and hyperspherical (positive curvature) spaces. It employs a data-driven weighting mechanism to adaptively select and strengthen the most suitable geometry for different parts of the graph, alongside a quadruplet distributor and innovative temporal regularization for smoothness. Further refining this concept, \cite{wang2024} IME builds upon MADE by explicitly addressing the "spatial gap" and heterogeneity between different curvature spaces. IME integrates "space-shared" properties to capture commonalities across spaces and "space-specific" properties for unique features, complemented by an Adjustable Multi-curvature Pooling (AMP) mechanism that learns optimal pooling weights for adaptive information fusion and novel similarity, difference, and structure loss functions.

The integration of spatial and fuzzy information alongside temporal dynamics represents another critical advancement. \cite{ji2024} FSTRE (Fuzzy Spatiotemporal RDF Knowledge Graph Embedding) introduces a groundbreaking model that comprehensively integrates fuzziness, spatial, and temporal dimensions within a complex vector space. This model uniquely employs projection for spatial information and rotation for temporal embedding, leveraging anisotropic vectors to incorporate fine-grained fuzziness into spatiotemporal five-tuples. FSTRE addresses the insufficiency of traditional KGE models for uncertain and dynamic knowledge, demonstrating its capability to handle complex fuzzy spatiotemporal information. Extending this, \cite{ji2024} Multihop Fuzzy Spatiotemporal RDF Knowledge Graph Query further leverages quaternion-based embeddings to jointly represent spatiotemporal entities and enable multihop fuzzy spatiotemporal queries. Quaternions, with their non-commutative compositional patterns, allow for robust path construction and reasoning over uncertain and dynamic knowledge graphs, addressing the challenge of multihop querying in such complex environments. Complementary to these geometric approaches, \cite{xie2023} TARGAT offers a Graph Neural Network (GNN) based approach with a dynamic time-aware relational generator to explicitly capture multi-fact interactions across different timestamps, providing another avenue for modeling complex temporal dependencies.

In conclusion, the evolution of temporal KGE has progressed significantly from foundational temporal integration to sophisticated multi-curvature and spatiotemporal embeddings. While models like MADE and IME offer geometrically adaptive solutions for diverse TKG structures, FSTRE and its quaternion-based extension push the boundaries by comprehensively integrating spatial, fuzzy, and temporal dimensions, even enabling complex multihop queries. A shared challenge across these advanced models lies in managing the increased computational overhead associated with multi-space embeddings, complex GNN architectures, or quaternion operations, and enhancing the interpretability of how different curvature spaces or integrated dimensions contribute to the overall embedding quality. Future directions will likely focus on optimizing these complex models for scalability, developing more transparent mechanisms for adaptive geometry selection, and exploring their application in real-world scenarios demanding highly nuanced spatiotemporal and uncertain reasoning.