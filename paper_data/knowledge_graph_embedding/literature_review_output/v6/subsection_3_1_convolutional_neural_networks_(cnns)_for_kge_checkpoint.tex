Convolutional Neural Networks (CNNs) have emerged as a powerful paradigm in Knowledge Graph Embedding (KGE), offering a distinct advantage over purely geometric models by automatically extracting intricate local features and modeling complex, non-linear interactions between entity and relation embeddings. This approach marks a significant shift from predefined mathematical transformations, enabling models to discover patterns directly from data. As discussed in previous subsections, while translational and rotational models \cite{ji2015, ebisu2017, gao2020, li2022, ge2022, ge2023, liu2024, chen2025} provide elegant mathematical frameworks for specific relation types, they often struggle with the sheer diversity and complexity of real-world relational patterns, particularly N-to-1, 1-to-N, and N-to-N mappings. CNNs address this by treating concatenated entity and relation embeddings as a 2D input, over which convolutional filters can slide to learn hierarchical feature representations.

A core innovation of CNN-based KGE models lies in their ability to capture rich feature interactions through shared weights and local receptive fields. Unlike simple scoring functions that rely on dot products or Euclidean distances, CNNs can learn multiple interaction patterns simultaneously. For instance, the pioneering work of ConvE (not directly cited but foundational to many CNN KGEs) demonstrated how reshaping the concatenated head entity and relation embeddings into a 2D matrix and applying convolutional filters could significantly enhance expressiveness. This mechanism allows the model to identify specific patterns that indicate the plausibility of a triplet $(h, r, t)$, moving beyond global transformations to nuanced local feature detection.

One notable approach, \textbf{AcrE} (Knowledge Graph Embedding with Atrous Convolution and Residual Learning) \cite{ren2020}, aims to address the computational complexity and potential information loss in deep neural network-based KGE models. AcrE's core innovation lies in its use of *atrous convolutions* and *residual learning*. Atrous convolutions enable the model to effectively enlarge its receptive field without increasing the number of parameters or losing resolution, allowing it to capture broader contextual information around the entity-relation pair. This is crucial for understanding relations that might depend on slightly more distant features within the concatenated embedding. Furthermore, the integration of residual learning helps in mitigating the vanishing/exploding gradient problem and ensures that original information is not forgotten as features are processed through deeper layers. AcrE claims to be simpler and more parameter-efficient than existing state-of-the-art methods \cite{ren2020}, achieving superior results on diverse datasets. However, while atrous convolutions expand the receptive field, they still operate locally on the input, which might theoretically limit their ability to capture extremely long-range, non-contiguous dependencies compared to self-attention mechanisms found in Transformers \cite{wang2019, li2023, shi2025}. Practically, while parameter-efficient, the convolutional operations themselves can still be computationally intensive depending on the filter configurations and input dimensions.

Building on the idea of learning richer feature embeddings, the \textbf{Multi-Scale Dynamic Convolutional Network (M-DCN)} \cite{zhang2020} was proposed to specifically tackle the challenge of handling complex relation types (1-to-N, N-to-1, N-to-N) that simpler models often struggle with. M-DCN's core innovation is its use of *multi-scale dynamic filters*. Instead of fixed filters, M-DCN generates filter weights that are dynamically related to each specific relation. This allows the model to adapt its feature extraction process based on the unique characteristics of the relation in question, making it highly flexible and expressive for diverse relational patterns. Additionally, M-DCN composes subject entity and relation embeddings in an alternating pattern in the input layer to extract more feature interactions. This dynamic and multi-scale approach enables M-DCN to achieve state-of-the-art link prediction results on several benchmark datasets \cite{zhang2020}. The theoretical limitation of M-DCN, despite its power, lies in the increased complexity and potential opacity introduced by dynamic filter generation; understanding the precise rationale behind a filter's adaptation for a given relation can be challenging. Practically, the dynamic nature and multi-scale operations likely incur higher computational costs and parameter counts compared to more straightforward CNN architectures.

Further advancing the capabilities of CNNs, \textbf{ReInceptionE} (Relation-Aware Inception Network with Joint Local-Global Structural Information) \cite{xie2020} addresses the limitation of earlier CNN models, such as ConvE, which often lack explicit structural information in their embedding space. ReInceptionE's core innovation is a two-pronged approach: first, it leverages the *Inception network* architecture (popular in computer vision for its ability to capture features at multiple scales) to learn richer query embeddings, thereby increasing the number and diversity of interactions between head and relation embeddings. Second, and crucially, it integrates a *relation-aware attention mechanism* to enrich these query embeddings with *joint local neighborhood and global entity information*. This attention mechanism allows ReInceptionE to dynamically weigh the importance of neighboring entities and relations, effectively incorporating broader graph context into the CNN's local feature extraction process. This directly addresses the critique that CNNs, by their nature, are primarily local feature extractors. ReInceptionE demonstrates competitive performance on datasets like WN18RR and FB15k-237, which are known for their complex relational structures \cite{xie2020}. However, the theoretical limitation arises from the heuristic combination of Inception modules and attention; while effective, the precise interaction and contribution of each component can be difficult to disentangle, potentially leading to a more complex "black-box" model. From a practical standpoint, the architectural depth and complexity of Inception networks combined with attention mechanisms generally lead to increased computational demands and a larger number of parameters compared to simpler KGE models.

More recent works continue to refine CNN-based KGE. \textbf{CNN-ECFA} (Convolutional Neural Network-Based Entity-Specific Common Feature Aggregation) \cite{hu2024} focuses on an underexplored area: leveraging entity-specific common feature aggregation to enhance knowledge graph representation learning. Its innovation is a novel CNN-based strategy that aggregates common features specific to entities, which has proven valuable in other deep learning tasks like text classification. This approach aims to provide more discriminative and robust entity representations by focusing on shared, yet entity-specific, characteristics. CNN-ECFA outperforms state-of-the-art feature projection strategies on datasets like WN18RR, YAGO3-10, and NELL-995, demonstrating average improvements in MRR and Hits@1 \cite{hu2024}. The theoretical underpinning of "common features" requires careful definition in the KGE context, and its effectiveness might depend on the inherent commonalities present in the dataset. Practically, the aggregation strategy, while beneficial, could introduce additional computational overhead during training.

Similarly, \textbf{SEConv} (A Semantic Enhanced Knowledge Graph Embedding Model With AIGC Designed for Healthcare Prediction) \cite{yang2025} addresses the neglect of complex structural features in traditional KGE models, particularly in the context of healthcare prediction. SEConv's core innovation combines a *less resource-consuming self-attention mechanism* with a *multilayer convolutional neural network*. The self-attention component generates more expressive embedding representations, while the multi-layer CNN is adopted to learn deeper structural features from triplets. This dual approach aims for both efficiency (crucial for resource-limited consumer electronics in healthcare) and enhanced expressiveness. SEConv demonstrates substantial improvements on medical datasets like UMLS and DBpedia50, validating its utility for healthcare prediction tasks \cite{yang2025}. A theoretical limitation is ensuring the synergistic rather than redundant operation of self-attention and multi-layer CNNs. Practically, while claiming "less resource-consuming," deep neural networks inherently demand significant computational resources, and application-specific models like SEConv require rigorous validation for generalizability beyond their target domain.

\begin{table}[htbp]
    \centering
    \caption{Comparative Framework of CNN-based KGE Models}
    \label{tab:cnn_kge_comparison}
    \begin{tabularx}{\textwidth}{|l|X|X|X|X|}
        \hline
        \textbf{Model} & \textbf{Core Innovation} & \textbf{Problem Solved} & \textbf{Key Advantages} & \textbf{Limitations} \\
        \hline
        \textbf{AcrE \cite{ren2020}} & Atrous convolutions + Residual learning & Complexity, training time, feature interaction & Increased receptive field, parameter efficiency, mitigates gradient issues & Still local, potential for long-range dependency struggle \\
        \hline
        \textbf{M-DCN \cite{zhang2020}} & Multi-scale dynamic filters, alternating composition & Complex relations (1-to-N, N-to-N), limited expressiveness & Relation-adaptive feature extraction, rich interactions & Increased complexity, potential opacity of dynamic filters, higher computational cost \\
        \hline
        \textbf{ReInceptionE \cite{xie2020}} & Inception network + Relation-aware attention & Lack of structural info, limited interactions in ConvE-like models & Joint local-global structural info, multi-scale feature learning & High complexity (Inception + Attention), increased parameters/computation, interpretability challenges \\
        \hline
        \textbf{CNN-ECFA \cite{hu2024}} & Entity-specific common feature aggregation & Underexplored entity-specific feature learning & Enhanced entity representations, improved link prediction & Definition of "common features" can be vague, potential computational overhead \\
        \hline
        \textbf{SEConv \cite{yang2025}} & Multi-layer CNN + Resource-efficient self-attention & Neglect of complex structural features, healthcare prediction & Deeper structural features, expressive embeddings, efficiency for specific applications & "Less resource-consuming" is relative, application-specific generalizability \\
        \hline
    \end{tabularx}
\end{table}

The evolution of CNN-based KGE models demonstrates a clear intellectual trajectory. Early CNN applications in KGE primarily focused on leveraging their local feature extraction capabilities to move beyond the fixed operations of geometric models. AcrE \cite{ren2020} then introduced atrous convolutions to expand the receptive field while maintaining efficiency, marking a step towards capturing broader context. M-DCN \cite{zhang2020} pushed this further by introducing *dynamic filters*, allowing the CNN to adapt its feature learning to specific relation types, thereby addressing the challenge of diverse relation patterns. ReInceptionE \cite{xie2020} represents a significant convergence, integrating attention mechanisms with CNNs to explicitly incorporate *global structural information* alongside local features, directly addressing the limitation of CNNs being purely local. This exemplifies the field's shift from purely local pattern recognition to a more holistic understanding of graph structure. More recent works like CNN-ECFA \cite{hu2024} and SEConv \cite{yang2025} refine these techniques, focusing on specific feature aggregation strategies or application-driven efficiency, indicating a maturation and specialization of CNN applications in KGE.

A critical tension across these models is the trade-off between model expressiveness and computational cost. While CNNs are powerful in learning intricate, non-linear feature interactions, the introduction of multi-scale filters, dynamic mechanisms, and attention layers (as seen in M-DCN \cite{zhang2020} and ReInceptionE \cite{xie2020}) inevitably increases parameter counts and computational demands. This can pose scalability challenges for extremely large knowledge graphs, a concern that is partially addressed by models like AcrE \cite{ren2020} focusing on parameter efficiency or SEConv \cite{yang2025} aiming for "less resource-consuming" designs. Furthermore, most CNN-based KGE models implicitly assume that concatenating entity and relation embeddings and applying convolutions is the optimal way to represent their interaction. However, the precise spatial arrangement of these input embeddings (e.g., $(h, r)$ vs. $(r, h)$) and the optimal way to represent their interaction for convolutional filters are often unstated assumptions that can significantly impact performance. While these models achieve state-of-the-art results in link prediction, the interpretability of *why* a specific convolutional filter learns a particular pattern for a relation remains a challenge, often making them more "black-box" compared to the mathematically transparent geometric models discussed in Section 2. This lack of interpretability can be a practical limitation, especially in high-stakes applications like healthcare, as highlighted by SEConv \cite{yang2025}.

In summary, CNNs have profoundly impacted KGE by enabling the automatic discovery of complex, non-linear patterns that are difficult to model with fixed geometric transformations. From early efforts to extract local features to advanced architectures incorporating dynamic filters and attention mechanisms, CNNs have consistently pushed the boundaries of link prediction performance. However, the field continues to balance the pursuit of higher expressiveness with the practical constraints of computational efficiency, scalability, and model interpretability, paving the way for further innovations in this promising area.