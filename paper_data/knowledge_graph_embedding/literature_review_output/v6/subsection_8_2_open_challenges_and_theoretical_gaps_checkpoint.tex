\subsection*{Open Challenges and Theoretical Gaps}
Despite the significant advancements in knowledge graph embedding (KGE) research, as highlighted in the preceding sections, several critical open challenges and theoretical gaps persist, representing fertile ground for future investigation. These challenges often stem from the inherent complexity of knowledge graphs, the trade-offs in model design, and the practical demands of real-world applications.

One of the most fundamental and recurring tensions in KGE research is \textbf{balancing model expressiveness with computational complexity and interpretability}. As discussed in Section 2 and 3, the field has progressed from simpler translational models like TransH \cite{wang2014} to highly expressive rotational models such as RotatE \cite{sun2018} and advanced deep learning architectures including GNNs \cite{wang2020, di2023} and Transformers. While models like GoldE \cite{li2024} and Fully Hyperbolic Rotation \cite{liang2024} push the boundaries of expressiveness by capturing intricate logical patterns and hierarchical structures, they often introduce higher computational costs and increased parameter counts. For instance, deep learning models, while powerful in automatically extracting features and modeling non-linear interactions, are notoriously difficult to interpret. This "black-box" nature is a significant practical limitation, especially in high-stakes domains like medicine or finance, where understanding *why* a prediction is made is as crucial as the prediction itself. Although some efforts, such as SpherE \cite{li2024} and Contextualized KGE for Explainable Recommendation (CKGE) \cite{yang2023}, attempt to imbue models with interpretability through specific geometric designs or path-based explanations, these are often post-hoc or limited in scope. The challenge lies in developing models that are both highly expressive and inherently interpretable, without sacrificing efficiency. The work on knowledge distillation like DualDE \cite{zhu2020} offers a partial solution by creating smaller, faster student models from larger teachers, but the interpretability of the underlying teacher model often remains unaddressed.

A second major challenge revolves around \textbf{data quality, rule integration, and the 'true' negative distribution in training}. Real-world knowledge graphs are inherently incomplete, noisy, and constantly evolving. As explored in Section 4.2, integrating logical rules (e.g., RUGE \cite{guo2017}, RulE \cite{tang2022}) can inject prior knowledge and enhance reasoning. However, the efficient extraction of high-quality, consistent rules from massive and often noisy KGs remains a significant practical hurdle. Furthermore, balancing strict adherence to rules with the flexibility to capture exceptions is a delicate act. Theoretically, the most profound gap in training KGE models lies in resolving issues related to the 'true' negative distribution. Most KGE models rely on negative sampling strategies to create contrastive learning signals, as KGs typically only store positive facts. While methods like NSCaching \cite{zhang2018} and Modality-Aware Negative Sampling (MANS) \cite{zhang2023} have improved sampling efficiency and quality, and non-sampling approaches like NS-KGE \cite{li2021} attempt to circumvent the problem entirely, the fundamental issue persists: the true distribution of false facts is unknown \cite{qian2021, madushanka2024}. Current negative sampling methods are heuristics, implicitly assuming certain distributions of false facts, which may not hold in real-world scenarios. This can lead to suboptimal performance or models that are brittle to noise. For instance, Confidence-Aware Negative Sampling \cite{shan2018} attempts to mitigate noise, but its effectiveness is still bounded by the quality of confidence scores. The challenge is to develop more theoretically grounded and robust training paradigms that are less reliant on such uncertain heuristics, or to devise mechanisms that can reliably infer the 'true' negative distribution.

The \textbf{scalability for extremely large and dynamic knowledge graphs} presents another formidable obstacle. As KGs grow to web-scale, traditional KGE training and inference become computationally prohibitive. Section 6.1 highlighted efforts in efficiency and compression, such as parameter-efficient learning (EARL \cite{chen2023}), embedding compression \cite{sachan2020, wang2021}, and optimized systems like GE2 \cite{zheng2024}. However, these often involve trade-offs. For example, while CPa-WAC \cite{modak2024} uses graph partitioning to improve scalability for GNN-based KGEs, partitioning can inherently lead to a loss of global graph structure, potentially impacting the quality of embeddings that rely on long-range dependencies. The challenge is to devise methods that can scale to billions of entities and relations without sacrificing the richness of the learned representations. Moreover, real-world KGs are rarely static; new entities, relations, and facts emerge constantly. This necessitates the development of \textbf{truly generalizable inductive and continual models}. While meta-learning approaches like MorsE \cite{chen2021} and MetaHG \cite{sun2024} and parameter-efficient adaptation techniques like incremental LoRA (FastKGE \cite{liu2024}) have shown promise in handling unseen entities and efficiently updating models, they still grapple with the problem of catastrophic forgetting, where new knowledge acquisition degrades performance on previously learned information. A truly generalizable inductive model should be able to embed entirely novel entities or relations with minimal or no prior context, a capability that remains largely elusive. Furthermore, the complexities of Federated KGE (FKGE), as discussed in Section 5.3, introduce additional scalability and privacy-preserving challenges, such as communication efficiency \cite{zhang2024} and personalization across diverse client data \cite{zhang2024}, while also guarding against adversarial attacks \cite{zhou2024}.

Finally, the field continues to grapple with the \textbf{need for more robust evaluation metrics and improved reproducibility}. As highlighted in Section 6.3, studies like \cite{ali2020} and \cite{rossi2020} have exposed significant reproducibility failures and biases in standard evaluation practices, such as the over-representation of certain entities in benchmarks. The high sensitivity of KGE performance to hyperparameter tuning, which varies across datasets \cite{lloyd2022}, further complicates fair comparisons and hinders scientific progress. The current metrics (e.g., MRR, Hits@K) may not fully capture the nuances of complex relational patterns, the quality of explanations, or the utility in specific downstream applications (e.g., set retrieval \cite{li2024}, molecular docking for drug repurposing \cite{islam2023}). The challenge is twofold: first, to develop standardized, robust, and domain-aware evaluation protocols that accurately reflect model performance across diverse tasks and real-world scenarios; and second, to foster a culture of rigorous reproducibility by providing comprehensive code, configurations, and detailed experimental setups, as advocated by initiatives like LibKGE \cite{broscheit2020}.

In summary, future research must push the boundaries of KGE capabilities by addressing these multifaceted challenges. This includes developing hybrid architectures that intelligently balance expressiveness, efficiency, and interpretability; designing more theoretically sound training paradigms that overcome the limitations of heuristic negative sampling; creating truly scalable and adaptable models for dynamic, web-scale KGs; and establishing more rigorous and comprehensive evaluation frameworks to ensure reliable and trustworthy advancements in the field.