The effective representation and utilization of knowledge have long been central to the advancement of artificial intelligence. Knowledge Graphs (KGs) have emerged as a powerful paradigm for organizing vast amounts of structured information, representing real-world entities and their intricate relationships in a human-readable, symbolic format. However, despite their expressiveness, traditional symbolic KGs inherently face challenges such as data sparsity, computational inefficiency in large-scale reasoning, and difficulty in capturing nuanced semantic similarities or handling incompleteness. These limitations hinder their seamless integration with data-driven machine learning models.

To overcome these hurdles, Knowledge Graph Embedding (KGE) methods have revolutionized the field by transforming discrete, symbolic knowledge into continuous, low-dimensional vector representations. This paradigm shift enables entities and relations to be represented as points or vectors in a continuous space, where semantic relationships are captured through geometric or algebraic operations. Such embeddings provide a machine-understandable format, facilitating scalability, efficient computation, and the ability to infer latent relationships, thereby making KGs more actionable for a wide array of AI tasks, including link prediction, entity alignment, and question answering.

This introductory section lays the foundational context for understanding the significance and trajectory of KGE research. We begin by tracing the evolution of knowledge representation, introducing the fundamental concepts and inherent challenges of knowledge graphs. Subsequently, we delve into the core motivations driving the development of KGE methods, explaining how they address the limitations of symbolic representations and enable new capabilities for AI systems. Finally, this section delineates the comprehensive scope and organizational structure of this literature review, providing a roadmap for the detailed discussions that follow on foundational models, advanced architectures, practical considerations, and diverse applications of KGE.