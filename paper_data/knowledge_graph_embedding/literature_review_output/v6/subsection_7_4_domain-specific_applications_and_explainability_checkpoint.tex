\subsection*{Domain-Specific Applications and Explainability}
Building upon the discussion of Knowledge Graph Embeddings (KGEs) in general applications like Question Answering and Recommender Systems (Section 7.3), this subsection shifts focus to the specialized and often high-stakes domains where KGEs are not merely applied but meticulously tailored, validated, and, crucially, made interpretable. In fields such as biological systems, patent analysis, and drug discovery, the demand for verifiable and transparent solutions necessitates moving beyond generic performance metrics to deliver actionable insights and build trust. This represents a critical evolution in KGE research, where the emphasis is on practical utility and explainability rather than solely on abstract embedding quality.

\begin{table}[htbp]
    \centering
    \caption{Comparative Framework: KGE in Specialized Domains with Emphasis on Explainability and Validation}
    \label{tab:domain_specific_kge}
    \begin{tabularx}{\textwidth}{|p{0.15\textwidth}|p{0.2\textwidth}|p{0.2\textwidth}|p{0.2\textwidth}|p{0.15\textwidth}|}
        \hline
        \textbf{Domain/Paper} & \textbf{Problem Addressed} & \textbf{KGE Tailoring/Innovation} & \textbf{Domain-Specific Evaluation/Validation} & \textbf{Explainability Focus} \\
        \hline
        \textbf{Biological Systems} \cite{mohamed2020} & Scalability for complex biological systems, drug-target prediction, polypharmacy side effects. & Learning low-rank vector representations of biological entities/relations. & Predictive accuracy for drug-target interactions, polypharmacy side effects. & Implicit (preserving graph structure), but less explicit. \\
        \hline
        \textbf{Patent Metadata Analysis} \cite{li2022} & Measuring knowledge proximity, explaining domain expansion profiles of inventors/assignees. & Training KGE on patent metadata (citations, inventors, assignees, classifications). & Predicting target entities, explaining inventor/assignee domain expansion profiles. & Explaining domain expansion profiles. \\
        \hline
        \textbf{Drug Repurposing (COVID-19)} \cite{islam2023} & Urgent drug discovery for COVID-19, need for verifiable predictions. & Ensemble KGE, deep neural network for prediction. & Molecular docking (novel for KGE-based repurposing), retrieval of in-trial drugs. & Rules extracted from KG, explanatory paths (post-hoc). \\
        \hline
        \textbf{Specific Diseases (Multimodal)} \cite{zhu2022} & Discovering new reliable knowledge for specific diseases, multimodal reasoning. & Multimodal reasoning (structural, category, description embeddings), reverse-hyperplane projection. & Manual proofreading of predicted pairs (drug-gene, gene-disease, disease-drug), universality of embeddings. & Implicit through multimodal integration, manual proofreading for human-understandable validation. \\
        \hline
    \end{tabularx}
\end{table}

\subsubsection*{KGE in Biological Systems and Drug Discovery}
The complexity of biological systems, characterized by intricate networks of genes, proteins, drugs, and diseases, presents a formidable challenge for traditional analytical methods. Knowledge graphs naturally represent these interconnected biological entities, but their sheer scale often limits the scalability of graph exploratory approaches \cite{mohamed2020}. KGE models offer a solution by learning low-dimensional vector representations that preserve the graph's inherent structure, enabling efficient predictive and analytical tasks.

\begin{itemize}
    \item \textbf{General Biological Applications}
    \begin{enumerate}
        \item \textbf{Context}: Traditional graph exploration methods for biological KGs, while accurate, struggle with scalability due to time-consuming path exploration. The need for efficient, high-accuracy predictive models in drug discovery and understanding disease mechanisms motivated the adoption of KGEs.
        \item \textbf{Problem Solved}: \textsf{Biological applications of knowledge graph embedding models} \cite{mohamed2020} addresses the scalability limitations of traditional graph processing in biological systems and demonstrates KGE's utility for tasks like predicting drug-target interactions and polypharmacy side effects.
        \item \textbf{Core Innovation}: The core innovation lies in applying existing KGE models (e.g., \textsf{TransE}, \textsf{DistMult}) to biological knowledge graphs, leveraging their ability to learn low-rank vector representations of biological entities (drugs, proteins, diseases) and relations. This transforms complex graph structures into a computationally tractable embedding space.
        \item \textbf{Mechanism}: Biological entities and their relationships (e.g., "drug A targets protein B," "protein B is associated with disease C") are encoded into a KG. KGE models then embed these entities and relations, allowing for similarity-based predictions (e.g., if drug A is similar to drug D, and drug D targets protein E, then drug A might also target protein E).
        \item \textbf{Conditions for Success}: Requires well-curated biological KGs. The success hinges on the KGE model's ability to capture nuanced biological relationships accurately, which can be challenging given the heterogeneity and complexity of biological data.
        \item \textbf{Theoretical Limitations}: While KGEs improve scalability, the interpretability of the learned embeddings in a purely biological sense can be limited. It's often unclear *why* certain embeddings are close or far apart, making it difficult to derive novel biological hypotheses directly from the embedding space without further analysis.
        \item \textbf{Practical Limitations}: The quality and completeness of biological KGs are paramount. Data sparsity in specific biological pathways or drug-disease interactions can lead to unreliable embeddings.
        \item \textbf{Comparison}: KGE models offer superior scalability compared to traditional path-exploratory methods \cite{mohamed2020}. However, they often lack the direct, human-readable reasoning paths that rule-based systems (as discussed in Section 4.2) or symbolic reasoning approaches might provide, which is a significant drawback in high-stakes fields.
    \end{enumerate}

    \item \textbf{Molecular-evaluated and Explainable Drug Repurposing for COVID-19}
    \begin{enumerate}
        \item \textbf{Context}: The urgent need for effective treatments for diseases like COVID-19 highlights the value of drug repurposing. While KGEs can predict drug-disease associations, the high-stakes nature of medical applications demands not just predictions, but also verifiable evidence and clear explanations. This addresses a critical gap in generic KGE applications, which often rely on abstract metrics.
        \item \textbf{Problem Solved}: \textsf{Molecular-evaluated and explainable drug repurposing for COVID-19 using ensemble knowledge graph embedding} \cite{islam2023} tackles the challenge of identifying potential drug candidates for COVID-19 while providing molecular-level validation and human-interpretable explanations.
        \item \textbf{Core Innovation}: This work proposes an ensemble KGE approach to create robust latent representations, which are then fed into a deep neural network for drug discovery. Crucially, it introduces *molecular docking* as a novel, domain-specific evaluation metric for KGE-based drug repurposing, moving beyond standard link prediction metrics. Furthermore, it provides *explanations* through rules extracted from the KG and instantiated by explanatory paths, directly addressing the need for interpretability in high-stakes medical contexts.
        \item \textbf{Mechanism}: A COVID-19 centric KG is constructed. Ensemble embeddings are learned to capture diverse relational patterns, enhancing the robustness of representations. A deep neural network leverages these embeddings to predict drug-COVID-19 associations. Top predictions are then subjected to molecular docking simulations to assess their binding affinity to SARS-CoV-2 targets, providing a physical, verifiable validation. Explanatory paths are generated by tracing relevant relations in the KG that support the prediction, offering a transparent rationale.
        \item \textbf{Conditions for Success}: Requires a comprehensive and high-quality COVID-19 KG. The effectiveness of molecular docking depends on accurate protein structures and docking algorithms. The quality of extracted rules for explanation is also vital.
        \item \textbf{Theoretical Limitations}: While ensemble embeddings improve robustness, the underlying deep neural network remains a black box. The post-hoc nature of rule extraction for explanation, while valuable, may not fully capture the complex non-linear reasoning of the neural network. The generalizability of specific molecular docking results to clinical efficacy is also a known challenge in drug discovery.
        \item \textbf{Practical Limitations}: Molecular docking is computationally intensive. Constructing and maintaining a high-quality, up-to-date domain-specific KG is a significant effort.
        \item \textbf{Comparison}: This approach significantly advances beyond general biological KGE applications \cite{mohamed2020} by integrating a *verifiable, domain-specific evaluation metric* (molecular docking) and *explicit explainability mechanisms*. It directly addresses the need for trustworthiness in medical AI, a theme echoed in discussions on evaluation and reproducibility (Section 6.3). The emphasis on rules for explanation connects to the rule-based KGE methods discussed in Section 4.2, demonstrating their practical utility in providing interpretability.
    \end{enumerate}

    \item \textbf{Multimodal Reasoning for Specific Diseases}
    \begin{enumerate}
        \item \textbf{Context}: Biomedical KGs are rich but often incomplete. Leveraging multimodal information (e.g., textual descriptions, categories) alongside structural data can enhance knowledge discovery, especially for specific diseases. The challenge is effectively integrating these diverse modalities.
        \item \textbf{Problem Solved}: \textsf{Multimodal reasoning based on knowledge graph embedding for specific diseases} \cite{zhu2022} aims to discover new and reliable knowledge within Specific Disease Knowledge Graphs (SDKGs) by integrating structural, category, and description embeddings.
        \item \textbf{Core Innovation}: This work constructs SDKGs for various diseases and implements multimodal reasoning using reverse-hyperplane projection. It combines structural embeddings (from graph triples), category embeddings (from entity types), and description embeddings (from textual information) to enrich representations. Reliability is verified through *manual proofreading* of predicted drug-gene, gene-disease, and disease-drug pairs.
        \item \textbf{Mechanism}: SDKGs are built by extracting triplets, standardizing entities, and linking relations. KGEs are learned for each modality. Reverse-hyperplane projection is used to integrate these multimodal embeddings, allowing for more comprehensive reasoning. The predictions are then manually validated by domain experts, providing a human-centric layer of trustworthiness.
        \item \textbf{Conditions for Success}: Requires high-quality multimodal data (textual descriptions, categories) alongside structural data. The effectiveness of multimodal fusion is critical. Manual proofreading, while robust, is resource-intensive and relies on expert availability.
        \item \textbf{Theoretical Limitations}: The reverse-hyperplane projection, while effective, might not fully capture all complex interactions between modalities. The "universality" of embeddings, while demonstrated for biomolecular interaction classification, might not extend to all downstream tasks without further fine-tuning.
        \item \textbf{Practical Limitations}: Constructing and maintaining multimodal SDKGs is complex. Manual proofreading, while a strong validation, is not scalable for large-scale, continuous knowledge discovery.
        \item \textbf{Comparison}: This approach highlights the power of multimodal KGEs (as discussed in Section 4.3) in specialized domains. Similar to \cite{islam2023}, it emphasizes *verifiable reliability* through manual proofreading, reinforcing the need for human oversight in high-stakes applications. It shows that combining diverse information sources can lead to more robust and reliable knowledge discovery.
    \end{enumerate}

\subsubsection*{KGE in Patent Metadata Analysis}
Beyond biomedical applications, KGEs are proving invaluable in intellectual property and innovation analysis, where understanding the relationships between patents, inventors, assignees, and technological domains is crucial for strategic decision-making.

\begin{enumerate}
    \item \textbf{Embedding Knowledge Graph of Patent Metadata}
    \begin{enumerate}
        \item \textbf{Context}: Understanding "knowledge proximity" between entities (e.g., patents, inventors, companies) in large patent databases is essential for innovation management. Traditional methods often rely on simple co-occurrence or citation analysis, which may not capture nuanced semantic associations.
        \item \textbf{Problem Solved}: \textsf{Embedding knowledge graph of patent metadata to measure knowledge proximity} \cite{li2022} operationalizes knowledge proximity in the US Patent Database by building a knowledge graph (\textsf{PatNet}) from patent metadata and applying KGEs to measure associations.
        \item \textbf{Core Innovation}: The work constructs \textsf{PatNet} using patent metadata (citations, inventors, assignees, domain classifications) and trains various KGE models on it. Knowledge proximity is then defined as the cosine similarity between the learned embeddings of entities. The approach is validated by its ability to predict target entities and *explain domain expansion profiles* of inventors and assignees.
        \item \textbf{Mechanism}: Patent metadata is transformed into a KG where patents, inventors, assignees, and technology classifications are entities, and relationships like "inventor invents patent," "patent cites patent," "assignee owns patent" are relations. KGE models (e.g., \textsf{TransE}, \textsf{DistMult}) learn embeddings for these entities and relations. The proximity in the embedding space reflects knowledge proximity.
        \item \textbf{Conditions for Success}: Requires a comprehensive patent database with rich metadata. The quality of the constructed \textsf{PatNet} and the chosen KGE model are critical.
        \item \textbf{Theoretical Limitations}: While cosine similarity in embedding space is a common proxy for semantic proximity, its direct interpretation as "knowledge proximity" might be an oversimplification. The KGE models primarily capture structural patterns, and the richness of semantic proximity might require more sophisticated models or multimodal integration.
        \item \textbf{Practical Limitations}: Building and maintaining a large-scale patent KG is a significant data engineering task. The interpretation of "domain expansion profiles" from embeddings requires careful domain expertise.
        \item \textbf{Comparison}: This application demonstrates KGE's utility in a non-biomedical, high-value domain. Similar to the drug repurposing work, it moves beyond generic link prediction to provide *actionable insights* (knowledge proximity, domain expansion) that are directly relevant to industry problems. The focus on explaining domain expansion profiles aligns with the broader demand for explainability, albeit in a different context than medical applications.
    \end{enumerate}
\end{itemize}

\subsubsection*{Synthesis and Critical Analysis}
The application of KGEs in specialized domains marks a significant maturation of the field, shifting from theoretical model development to delivering verifiable and transparent solutions for complex, real-world challenges. A clear pattern emerges: the high-stakes nature of these fields (e.g., medicine, intellectual property) necessitates a departure from sole reliance on generic KGE performance metrics (like Hits@K or MRR), which can be misleading or insufficient for building trust \cite{ali2020, rossi2020}. Instead, there is a strong demand for *domain-specific evaluation* and *explainability*.

The evolution from general biological applications \cite{mohamed2020} to highly specific tasks like COVID-19 drug repurposing \cite{islam2023} illustrates this trend. While \cite{mohamed2020} highlights KGE's scalability advantage over traditional graph methods, it implicitly assumes that preserving graph structure is sufficient for biological insight. \cite{islam2023} directly questions this assumption by introducing *molecular docking* as a novel, independent, and physically verifiable evaluation metric. This is a crucial innovation, as it grounds abstract embedding predictions in concrete scientific evidence, addressing the methodological quality concern of evaluating KGEs solely on internal metrics. The integration of rule extraction and explanatory paths in \cite{islam2023} further exemplifies the growing demand for interpretable KGE models, moving beyond "what" is predicted to "why" it is predicted, which is paramount in fields where decisions have direct human impact. This directly connects to the rule-based KGE approaches discussed in Section 4.2, showcasing their practical utility in providing transparency.

Similarly, the multimodal reasoning for specific diseases \cite{zhu2022} and patent metadata analysis \cite{li2022} underscore the need for tailored KGEs. \cite{zhu2022} leverages multimodal information (structural, category, description embeddings) to enrich representations, aligning with the broader trend of multi-modal KGEs (Section 4.3). Its reliance on *manual proofreading* by experts for validation is a strong, albeit resource-intensive, commitment to trustworthiness. This highlights an unstated assumption in many KGE papers: that standard benchmark evaluations are sufficient. In contrast, these domain-specific applications demonstrate that human expert validation is often indispensable for high-stakes scenarios. \cite{li2022}'s focus on explaining "domain expansion profiles" in patents further emphasizes that explainability is not confined to medical applications but is a cross-domain requirement for actionable insights.

A critical tension exists between the desire for highly expressive and complex KGE models (often deep learning-based, as discussed in Section 3) and the imperative for interpretability and verifiable solutions in these high-stakes domains. While complex models might achieve higher predictive accuracy, their black-box nature can hinder adoption where trust and accountability are paramount. The approaches reviewed here attempt to bridge this gap through *post-hoc explanation mechanisms* (e.g., rule extraction, explanatory paths \cite{islam2023}) or *human-in-the-loop validation* (e.g., manual proofreading \cite{zhu2022}). However, the theoretical limitations of post-hoc explanations, which may not fully reflect the model's internal reasoning, remain an open challenge.

In conclusion, the application of KGEs in specialized domains represents a significant paradigm shift. It underscores the field's commitment to real-world impact by demanding models that are not only performant but also interpretable, verifiable, and tailored to specific industry problems. This necessitates a re-evaluation of standard KGE evaluation practices, pushing towards domain-specific metrics and robust explanation mechanisms to build trust and deliver transparent, actionable insights for complex, real-world challenges. This evolution aligns with the broader demand for trustworthy AI, a theme that will continue to shape future KGE research.