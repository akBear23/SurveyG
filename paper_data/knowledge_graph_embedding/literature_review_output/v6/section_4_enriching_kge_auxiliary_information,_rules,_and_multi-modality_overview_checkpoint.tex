\section{Enriching KGE: Auxiliary Information, Rules, and Multi-modality}
\label{sec:enriching_kge_auxiliary_information_rules_and_multi-modality}

While the preceding section demonstrated the profound impact of deep learning architectures, such as CNNs, GNNs, and Transformers, in extracting intricate structural patterns and learning context-aware representations directly from knowledge graph topology, real-world knowledge graphs often present challenges that purely structural models cannot fully address. These include inherent data sparsity, the need for deeper semantic understanding beyond connectivity, and the demand for explicit logical reasoning capabilities. To overcome these limitations and move towards a more comprehensive and nuanced representation of knowledge, this section explores advanced Knowledge Graph Embedding (KGE) approaches that integrate diverse external knowledge sources and logical constraints, thereby enriching the learned embeddings.

This section delves into three primary avenues for KGE enrichment. Firstly, we examine methods that incorporate auxiliary information, such as entity types, attributes, and hierarchical structures, to provide richer semantic context and guide the embedding process, making models more robust to incompleteness \cite{community_0, community_1}. Secondly, we discuss the integration of explicit logical rules and constraints, which inject prior knowledge into the learning process, enhancing reasoning capabilities and improving the interpretability and consistency of the learned embeddings \cite{community_3}. Finally, we explore multi-modal and cross-domain KGE techniques that leverage complementary information from various sources, including textual descriptions and visual features, to alleviate data sparsity and enable a more holistic understanding of entities and relations \cite{community_3, a6a735f8e218f772e5b9dac411fa4abea87fdb9c}. By moving beyond solely structural information, these enriched KGE models aim to achieve superior performance, enhance semantic understanding, improve reasoning, and foster greater robustness and interpretability in complex, real-world scenarios.