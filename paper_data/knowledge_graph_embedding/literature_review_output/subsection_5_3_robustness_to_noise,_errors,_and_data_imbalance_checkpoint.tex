\subsection*{Robustness to Noise, Errors, and Data Imbalance}

Real-world knowledge graphs (KGs) are inherently imperfect, frequently containing noisy data, erroneous triples, and skewed (long-tail) distributions, which significantly challenge the reliability and performance of Knowledge Graph Embedding (KGE) models. This subsection explores advanced techniques designed to enhance the robustness of KGE models against these pervasive data imperfections.

Early efforts to address data quality focused on mitigating noise during training. \cite{shan2018} introduced the concept of *negative triple confidence* and proposed a *confidence-aware negative sampling method* to improve training stability and prevent false detections in confidence-aware KGE models operating on noisy KGs. Building on the need for reliable predictions, \cite{tabacof2019} revealed that KGE models often produce *uncalibrated probabilities*, making their confidence scores unreliable. They proposed methods to *calibrate output probabilities* using synthetic negatives and a novel weighting scheme, ensuring trustworthy confidence scores crucial for downstream decision-making. To actively combat noise during the learning process, \cite{zhang2021} introduced a *multi-task reinforcement learning (MTRL)* framework. In this approach, RL agents make *hard decisions* to select high-quality triples, guided by *delayed rewards* from the KGE model and leveraging multi-task learning for semantically similar relations, moving beyond passive confidence scoring to active data filtering. Further addressing the issue of erroneous data, \cite{zhang2024} presented the *Error-Aware Knowledge Graph Embedding (AEKE)* framework. This framework leverages *entity attributes and hypergraphs* to compute *joint confidence scores* for triples, adaptively *down-weighting unreliable ones* during embedding learning, thereby providing a direct mechanism to mitigate the impact of errors using auxiliary information.

Beyond noise and errors, KGE models must contend with data imbalance. While not explicitly designed for imbalance, \cite{xiao2015} (TransA) introduced an *adaptive Mahalanobis distance with a relation-specific weight matrix* to model diverse and complex relation topologies, implicitly adapting to varying data characteristics and suppressing noise from irrelevant dimensions. This laid groundwork for adaptive metric learning. Tackling a specific form of semantic imbalance, \cite{lv2018} (TransC) differentiated between *concepts (represented as spheres) and instances (as vectors)*, modeling `isA` relations through spatial containment to capture hierarchical structures and transitivity, which are often skewed in real KGs. Directly addressing the pervasive *long-tail data imbalance*, \cite{zhang2023} (WeightE) introduced a novel *bilevel optimization framework* to adaptively assign *higher weights to infrequent entities and relations* during training. This ensures better representation for under-represented knowledge, providing a direct and flexible mechanism to mitigate the impact of skewed distributions.

General improvements in training robustness and efficiency also contribute to handling imperfect data. To ensure more robust and efficient training, \cite{zhang2018} (NSCaching) proposed a *cache-based negative sampling method* that directly stores and samples "hard" negative triplets, using *importance sampling* for dynamic updates. This approach effectively addresses the vanishing gradient problem more efficiently than complex GAN-based methods. Further enhancing training stability and accuracy, \cite{li2021} (NS-KGE) introduced an *efficient non-sampling framework* that mathematically re-derives the square loss to consider *all negative instances* without prohibitive computational cost, thereby eliminating the uncertainty inherent in sampling. A comprehensive *review of negative sampling methods* by \cite{madushanka2024} consolidates the diverse strategies, highlighting their impact on KGE robustness and efficiency, and underscoring the ongoing importance of this training component. Additionally, \cite{xiao2015} (TransG) addressed ambiguity, which can be seen as a form of inherent data imperfection, by utilizing a *Bayesian non-parametric infinite mixture model* to discover *multiple latent semantic components* for relations. This allows for adaptive selection of the most appropriate translation vector for each triple, reducing noise from unrelated semantics.

Despite significant progress, several challenges remain. The dynamic nature of noise and imbalance in evolving KGs requires continuous adaptation rather than one-time fixes. Generalizing these robustness techniques across diverse KGE architectures and seamlessly integrating multiple robustness mechanisms (e.g., error detection, imbalance handling, and calibrated outputs) within a single, end-to-end optimized framework remains an open research direction. Future work could also explore more sophisticated causal inference techniques to better understand and mitigate the root causes of data imperfections.