\label{sec:dynamic_inductive_and_distributed_kge}
\section{Dynamic, Inductive, and Distributed KGE}

Building upon the enriched KGE models discussed in Section \ref{sec:enriching_kge_auxiliary_information_rules_and_multi_modality}, which focused on leveraging diverse auxiliary information and logical constraints to deepen semantic understanding, this section shifts attention to the critical operational challenges of Knowledge Graph Embeddings in real-world environments. Traditional KGE models often assume static, complete, and centrally managed knowledge graphs, a paradigm increasingly at odds with the dynamic, evolving, and distributed nature of modern knowledge bases. This section addresses these fundamental limitations by exploring advanced KGE methodologies designed for adaptability, scalability, and security in complex, real-world operational settings, moving beyond static and centralized assumptions to meet the demands of modern knowledge management systems.

We delve into three interconnected areas. First, \textit{Temporal Knowledge Graph Embedding (TKGE)} tackles the inherent dynamism of real-world facts, where entities and relations change over time. These methods move beyond static representations to capture the fluidity and evolution of knowledge, crucial for tasks requiring reasoning over time. Second, \textit{Inductive and Continual KGE} addresses the challenge of unseen entities and the need for continuous model updates. This area explores how KGE models can efficiently learn embeddings for new entities without full retraining and adapt to a constant influx of new facts, mitigating catastrophic forgetting. Finally, \textit{Federated and Privacy-Preserving KGE} investigates collaborative learning paradigms for distributed knowledge graphs. This crucial direction enables multiple parties to jointly train robust KGE models while safeguarding sensitive data, addressing the growing demand for privacy-aware AI systems. Together, these advancements are pivotal for transitioning KGE from theoretical constructs to robust, secure, and continuously operational components within complex, real-world knowledge management systems.