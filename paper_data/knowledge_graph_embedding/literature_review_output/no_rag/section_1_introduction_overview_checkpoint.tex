The landscape of artificial intelligence has been profoundly shaped by advancements in knowledge representation, evolving from early symbolic systems to the sophisticated structures of modern knowledge graphs (KGs). These KGs, which organize world knowledge into networks of entities and relations, have become indispensable for various AI tasks, yet their inherent symbolic nature presents significant challenges. Issues such as data sparsity, computational inefficiency in large-scale reasoning, and difficulty in capturing nuanced semantic similarities often hinder their full potential \cite{68f34ed64fdf07bb1325097c93576658e061231e}. This introductory section establishes the foundational context for understanding Knowledge Graph Embeddings (KGEs), a paradigm-shifting approach that addresses these limitations by transforming discrete symbolic knowledge into continuous, low-dimensional vector spaces.

KGE methods have emerged as a central pillar in modern knowledge graph research, enabling machines to understand, reason with, and leverage complex relational data more effectively. By converting entities and relations into dense embeddings, KGEs facilitate seamless integration with advanced machine learning models, thereby enhancing capabilities in diverse AI applications such as link prediction, entity alignment, question answering, and recommender systems. This section begins by providing a comprehensive background on knowledge graphs, tracing their evolution and highlighting their structural characteristics and inherent challenges. Subsequently, it delves into the core motivations behind the development of knowledge graph embedding techniques, explaining how they overcome the limitations of traditional symbolic representations. Finally, this section delineates the scope and organizational structure of the entire literature review, offering readers a roadmap through the intricate landscape of KGE research. By laying this essential groundwork, we aim to underscore the significance and trajectory of KGEs in transforming symbolic knowledge into actionable, machine-understandable formats, thereby advancing the broader field of artificial intelligence.