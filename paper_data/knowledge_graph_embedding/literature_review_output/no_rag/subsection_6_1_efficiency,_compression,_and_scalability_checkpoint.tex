\subsection{Efficiency, Compression, and Scalability}
The practical deployment of Knowledge Graph Embedding (KGE) models for massive knowledge graphs (KGs) is often hampered by significant computational costs, extensive training times, and prohibitive memory footprints. Addressing these bottlenecks is crucial for transitioning KGEs from academic benchmarks to real-world applications, especially in resource-constrained environments. This area of research focuses on techniques spanning knowledge distillation, embedding compression, parameter-efficient learning, optimized system designs, and novel algorithms.

One prominent strategy for reducing the computational burden and memory footprint is \textit{knowledge distillation}. \cite{zhu2020} introduced DualDE, a method that distills knowledge from a high-dimensional, high-performing teacher KGE model into a low-dimensional student model. This approach significantly reduces embedding parameters (by 7-15x) and increases inference speed (by 2-6x) while retaining competitive performance. DualDE's generality allows its application across various KGE architectures, making it a versatile tool for efficiency. However, a common trade-off in distillation is a minor, albeit often acceptable, loss in performance, which is inherent to compressing information.

Complementary to distillation are direct \textit{embedding compression} techniques. \cite{sachan2020} proposed representing entities with discrete codes, achieving remarkable compression ratios (50-1000x) of the embedding layer with only a minor performance drop. Building on this, LightKG \cite{wang2021} introduced a lightweight framework that stores only a few codebooks and indices, drastically reducing storage and boosting inference efficiency through quick look-ups. LightKG also incorporates a dynamic negative sampling strategy, which further enhances performance. While these methods offer substantial gains in storage and inference speed, they rely on approximating the original embeddings, which might introduce subtle inaccuracies compared to full-precision representations.

\textit{Parameter-efficient learning} offers another avenue for scalability. Entity-Agnostic Representation Learning (EARL) \cite{chen2023} tackles the escalating parameter storage costs by learning embeddings only for a small set of "reserved entities." Embeddings for other entities are then derived from their context (e.g., connected relations, k-nearest reserved entities, multi-hop neighbors) using universal, entity-agnostic encoders. This approach results in a static and significantly lower parameter count, making it particularly beneficial for large and continuously growing KGs. A potential limitation of EARL is its reliance on contextual information; entities with sparse connections or those truly "unseen" without sufficient neighbors might have less expressive representations compared to directly learned embeddings.

Beyond model-specific optimizations, \textit{optimized system designs} are critical for large-scale KGE training. GE2 \cite{zheng2024} proposes a general and efficient KGE learning system that addresses long CPU times and high CPU-GPU communication overhead, especially in multi-GPU setups. GE2 offloads operations to GPUs and introduces COVER, a novel algorithm for managing data swap between CPU and multiple GPUs, achieving speedups of 2x to 7.5x across various models and datasets. This system-level innovation provides a foundational improvement for KGE research, enabling faster experimentation and deployment of complex models. While GE2 focuses on training efficiency, it doesn't directly reduce the size of the *final* embedding model, which is where distillation and compression techniques become relevant.

Novel algorithmic approaches also contribute significantly to efficiency. "Highly Efficient Knowledge Graph Embedding Learning with Orthogonal Procrustes Analysis" \cite{peng2021} introduces a fundamentally different paradigm by proposing a closed-form solution using Orthogonal Procrustes Analysis (OPA). This enables full-batch learning and non-negative sampling, reducing training time and carbon footprint by orders of magnitude while yielding competitive performance. The closed-form nature bypasses iterative optimization, which is a major source of computational cost in traditional KGEs. However, the inherent mathematical constraints of OPA might limit its expressiveness compared to more flexible, iteratively optimized models for certain complex relational patterns.

For Graph Neural Network (GNN)-based KGEs, which are known for their computational intensity, \textit{graph partitioning strategies} are essential. CPa-WAC \cite{modak2024} employs modularity maximization-based constellation partitioning to break down KGs into subgraphs. This allows for separate processing, reducing memory and training time for GNNs while aiming to retain prediction accuracy. CPa-WAC demonstrates up to a five-fold speedup, highlighting the effectiveness of distributed processing. Nevertheless, partitioning a graph can sometimes hinder the capture of global dependencies that span across different subgraphs, potentially impacting performance on tasks that require broader structural understanding.

Finally, a comprehensive understanding of \textit{parallelization techniques} is vital. \cite{kochsiek2021} provided a critical meta-study, re-implementing and comparing various parallelization strategies for KGE training. Their work revealed that naive parallelization can degrade embedding quality and proposed effective mitigations, such as a variation of the stratification technique. This study underscores that simply distributing computation does not guarantee efficiency or quality, emphasizing the need for careful technique selection.

In summary, the pursuit of efficiency and scalability in KGE involves a multi-faceted approach. Techniques like DualDE \cite{zhu2020}, \cite{sachan2020}, and LightKG \cite{wang2021} focus on compressing the model itself for storage and inference. EARL \cite{chen2023} addresses parameter growth for evolving KGs. Meanwhile, GE2 \cite{zheng2024}, OPA-based learning \cite{peng2021}, CPa-WAC \cite{modak2024}, and parallelization studies \cite{kochsiek2021} target the efficiency of the training process. These innovations collectively aim to overcome the practical bottlenecks of KGE, making them deployable in resource-constrained environments and capable of handling the ever-growing scale of real-world knowledge bases. The ongoing challenge lies in balancing the gains in efficiency and scalability with the preservation of model expressiveness and predictive accuracy.