\subsection*{Robustness and Training Optimization}
The efficacy of Knowledge Graph Embedding (KGE) models in real-world applications is profoundly dependent on their robustness against data imperfections and the optimization of their training processes. Knowledge graphs are inherently noisy, incomplete, and often suffer from imbalanced data distributions, necessitating sophisticated techniques to ensure that learned representations are accurate, reliable, and generalize well. This subsection delves into methods designed to enhance KGE model robustness and refine their training, particularly focusing on negative sampling strategies.

A critical aspect of model reliability is the trustworthiness of its predictions. \cite{tabacof2019} highlight that many popular KGE models, despite achieving high accuracy, produce uncalibrated probability estimates, meaning their predicted scores do not directly correspond to true probabilities. They propose post-hoc calibration methods like Platt scaling and isotonic regression, which are particularly valuable when ground truth negatives are scarce, a common scenario in KGs. While these methods offer a general solution applicable to various KGE models, their effectiveness relies on the quality of the calibration data and may introduce additional computational overhead during inference. This emphasizes that model performance metrics alone are insufficient; the reliability of output probabilities is equally crucial for practical deployment.

To address the pervasive issue of noisy data within KGs, which often arises from automatic knowledge construction, \cite{zhang2021} introduce a multi-task reinforcement learning framework. This innovative approach actively filters out noisy triples during training, allowing the KGE model to learn from a cleaner, more reliable subset of facts. By exploiting correlations among semantically similar relations through multi-task learning, their method aims to learn more robust representations. This proactive noise-filtering mechanism is a significant advancement over passive error handling, as it directly impacts the quality of the input data for embedding. However, the complexity introduced by a reinforcement learning agent within the training loop can increase computational cost and require careful hyperparameter tuning, posing scalability challenges for extremely large KGs.

Another common imperfection in KGs is data imbalance, where entities and relations follow a long-tail distribution, with a few occurring frequently and many appearing rarely. Traditional KGE methods often assign equal weights during training, leading to unreliable representations for infrequent (long-tail) entities and relations. To counteract this, \cite{zhang2023} propose WeightE, a weighted KGE model that employs a bilevel optimization scheme to assign differential weights. WeightE dynamically endows lower weights to frequent elements and higher weights to infrequent ones, ensuring that long-tail entities and relations receive adequate training attention. This flexible weighting technique can be applied to various existing KGE models, offering a practical solution to a widespread problem. While effective, the bilevel optimization adds a layer of complexity to the training process, which might require more computational resources or careful convergence monitoring compared to standard training.

Beyond handling data imperfections, optimizing the training process itself, particularly through effective negative sampling, is paramount. KGE models typically rely on contrastive learning, requiring both positive (observed) and negative (false) triples. The quality of generated negative samples profoundly impacts model performance \cite{qian2021, madushanka2024}. Early approaches to negative sampling often struggled in noisy environments. \cite{shan2018} address this with a confidence-aware negative sampling method for noisy KGE, introducing the concept of negative triple confidence to improve training stability and prevent issues like zero loss or false detection. This method acknowledges that not all negative samples are equally informative, especially in the presence of noise. Building on the idea of identifying "hard" negatives, \cite{zhang2018} propose NSCaching, an efficient caching strategy that tracks and samples challenging negative triplets. This approach, inspired by generative adversarial networks (GANs) but simpler, aims to distill the benefits of complex adversarial sampling into a more computationally efficient framework, demonstrating a trade-off between model complexity and training efficiency.

A more radical departure from traditional negative sampling is presented by \cite{li2021} with their Efficient Non-Sampling Knowledge Graph Embedding (NS-KGE). This method proposes to avoid negative sampling entirely by considering all negative instances. While this theoretically removes the uncertainty and potential instability inherent in sampling, it dramatically increases computational complexity. NS-KGE addresses this by leveraging mathematical derivations to reduce the complexity of the non-sampling loss function, aiming for both better efficiency and accuracy. This non-sampling paradigm offers a compelling alternative, particularly for models whose loss functions can be efficiently reformulated, but its generalizability across all KGE architectures and its practical scalability to extremely dense KGs remain areas of active research.

The increasing complexity of KGE models and the integration of diverse data modalities further complicate negative sampling. \cite{zhang2023} (Modality-Aware Negative Sampling) address this by proposing MANS, a method specifically designed for multi-modal KGE. MANS aligns structural and visual embeddings for entities, demonstrating that negative sampling strategies must adapt to the unique characteristics of multi-modal information to learn meaningful embeddings. This highlights a critical development direction where training optimization must evolve in tandem with model architectural advancements.

Overall, the advancements in robustness and training optimization reflect a maturing KGE field that is moving beyond purely theoretical model expressiveness towards practical utility. The systematic reviews of negative sampling \cite{qian2021, madushanka2024} underscore its foundational importance, even as models like \cite{chen2025} (ConQuatE) introduce sophisticated quaternion-based embeddings to handle polysemy. The continued focus on refining training mechanisms, such as negative sampling, alongside the development of advanced architectures, indicates a holistic approach where the effectiveness of complex KGE models is deeply intertwined with robust and efficient training methodologies. While significant progress has been made in handling noise, imbalance, and sampling, the challenge of efficiently and accurately identifying the "true" negative distribution in KGs, especially in dynamic and multi-modal settings, remains a theoretical gap and an active area of research.