\subsection*{Question Answering and Recommendation Systems}
Knowledge Graph Embeddings (KGEs) have emerged as a pivotal technology for bridging the semantic gap between natural language and structured knowledge, enabling more intelligent and interpretable interactions in diverse applications such as Question Answering (QA) and Recommender Systems. These applications leverage KGEs to transform complex symbolic reasoning into efficient vector space operations, thereby enhancing performance and user experience \cite{dai2020, cao2022}.

In the realm of Question Answering over Knowledge Graphs (QA-KG), KGEs facilitate the understanding of natural language queries by mapping them to entities and relations within the underlying knowledge graph. Early frameworks, such as Knowledge Embedding based Question Answering (KEQA), demonstrated the utility of KGEs by jointly recovering head entity, predicate, and tail entity representations in the embedding space to answer simple natural language questions \cite{huang2019}. KEQA's strength lies in its ability to address predicate variability and entity ambiguity by leveraging the semantic proximity captured by embeddings. However, its focus on "simple questions" highlights a limitation in handling more complex, multi-hop, or nuanced queries, which often require deeper integration with natural language processing (NLP) capabilities.

More advanced QA systems have evolved into hybrid architectures that seamlessly integrate KGEs with sophisticated NLP models. A prime example is the Marie and BERT system for chemistry, which showcases a comprehensive approach to domain-specific QA \cite{zhou2023}. This system employs hybrid KGEs, leveraging multiple embedding spaces to capture diverse relational patterns, and integrates a BERT-based entity-linking model to enhance robustness and accuracy in identifying entities from natural language queries. Furthermore, Marie and BERT addresses the complexities of deep ontologies by deriving implicit multi-hop relations and incorporates mechanisms for numerical filtering, demonstrating a significant leap in handling intricate, fact-oriented information retrieval in specialized domains. While such hybrid systems offer superior performance in complex scenarios, their domain specificity and the inherent complexity of integrating heterogeneous components (KGEs, BERT, semantic agents) can limit generalizability and increase development overhead. The methodological challenge lies in effectively harmonizing the continuous vector representations from KGEs with the discrete, symbolic reasoning often required for precise QA.

For recommender systems, KGEs provide a powerful mechanism to model user preferences and item characteristics by representing them as entities in a knowledge graph, thereby enabling more personalized and transparent suggestions. Recurrent Knowledge Graph Embedding (RKGE) was an early and influential approach that utilized a recurrent network to automatically learn semantic representations of paths between entities \cite{sun2018}. By fusing these path semantics into the recommendation process, RKGE not only improved recommendation accuracy but also offered meaningful explanations based on path saliency, a crucial step towards interpretable recommendations. The recurrent network architecture allowed RKGE to discriminate the saliency of different paths in characterizing user preferences, moving beyond traditional feature engineering that often requires extensive domain knowledge.

Building upon this foundation, recent research has pushed towards more contextualized and explainable approaches. Contextualized Knowledge Graph Embedding for Explainable Talent Training Course Recommendation (CKGE) represents a significant advancement in this direction \cite{yang2023}. CKGE constructs "meta-graphs" for talent-course pairs, incorporating contextualized neighbor semantics and high-order connections as "motivation-aware information." It then employs a novel KG-based Transformer, equipped with relational attention and structural encoding, to model the global dependencies of KG structured data. A key innovation in CKGE is its "local path mask prediction," which effectively reveals the importance of different paths, thereby offering precise and explainable recommendations that can discriminate the saliencies of meta-paths in characterizing corresponding preferences \cite{yang2023}.

Comparing RKGE and CKGE, the evolution highlights a shift from recurrent networks for path modeling to more sophisticated Transformer architectures that capture richer contextual information. While RKGE provided explanations based on path saliency, CKGE offers a more granular and motivation-aware interpretability by integrating contextualization and high-order connections. This progression, as noted in the development directions, reflects an acceleration in research driven by the demand for more sophisticated, robust, and interpretable recommendation solutions. However, the increased complexity of models like CKGE, with their meta-graph construction and Transformer integration, introduces trade-offs in computational cost and the interpretability of the underlying embedding space. The theoretical gap remains in developing truly intuitive and actionable explanations that are universally understandable to human users, rather than merely technical artifacts of the model.

In summary, KGEs have profoundly impacted QA and recommender systems by providing a robust framework to bridge natural language and structured knowledge. From foundational KEQA to hybrid systems like Marie and BERT, and from recurrent RKGEs to contextualized CKGEs, these advancements demonstrate KGE's utility in enabling more intelligent, personalized, and interpretable user interactions. The ongoing challenge lies in balancing model expressiveness with computational efficiency and ensuring that the generated explanations are genuinely transparent and actionable across diverse application contexts.