\subsection*{Background: Knowledge Graphs}

Knowledge Graphs (KGs) represent a fundamental paradigm for organizing and representing world knowledge in a structured, machine-readable format. At their core, KGs are directed graphs composed of entities (nodes) and relations (edges), forming a collection of factual triplets in the form of (head entity, relation, tail entity) \cite{ge2023, dai2020}. For instance, the triplet (Barack Obama, bornIn, Hawaii) explicitly states a factual relationship between two entities. This structured representation allows for explicit semantic connections, enabling machines to understand and process information in a manner closer to human cognition.

The historical trajectory of knowledge representation has seen a significant evolution, from early semantic networks and expert systems in artificial intelligence to the more formalized ontologies and the vision of the Semantic Web. These foundational efforts aimed to capture human knowledge in a symbolic form, providing a basis for logical reasoning and inference. With the advent of the internet and the explosion of digital information, the need for large-scale, interconnected knowledge bases became paramount. This led to the development of modern, expansive KGs such as Freebase (now largely integrated into Wikidata), DBpedia, and Wikidata itself \cite{wang2014, lv2018, zhang2018}. These prominent examples serve as crucial repositories, aggregating and organizing vast amounts of world knowledge from diverse sources like Wikipedia, enabling a wide array of intelligent systems, from search engines and question-answering systems to recommender platforms \cite{huang2019, sun2018}.

Despite their immense utility and structured nature, symbolic KGs inherently face several significant challenges that limit their scalability, efficiency, and ability to handle real-world complexities. Firstly, reasoning with symbolic KGs, particularly when involving complex logical rules or multi-hop inference, can be computationally inefficient and resource-intensive, often exhibiting exponential complexity \cite{ge2023, dai2020}. This makes real-time inference on large-scale KGs a formidable task. Secondly, KGs are almost always incomplete; real-world knowledge is vast and constantly evolving, making it practically impossible to explicitly represent every single fact. Symbolic methods struggle profoundly with this incompleteness, as they typically require explicit rules or complete data to infer missing links, leading to brittle and often inaccurate predictions in sparse environments. The "data sparsity" problem is a recurring theme, where many entities and relations have limited connections, hindering comprehensive analysis \cite{ge2023, dai2020}.

Furthermore, symbolic representations treat entities and relations as discrete, atomic tokens, which inherently limits their capacity to capture nuanced semantic similarities or implicit relationships. For example, while "car" and "automobile" are semantically very close, a purely symbolic KG would treat them as distinct, unrelated entities unless explicitly linked by a relation. This lack of inherent semantic fluidity makes it difficult to generalize knowledge or discover novel patterns based on underlying similarities. These limitations—computational inefficiency, difficulty in managing growing data, challenges in handling incompleteness, and the inability to capture implicit semantic similarities—collectively underscore the necessity for more advanced representation techniques. As highlighted in the broader context of knowledge graph embedding research, these issues motivate the fundamental shift towards embedding entities and relations into continuous, low-dimensional vector spaces, thereby transforming complex symbolic problems into more efficient vector operations and laying the "bedrock for representing complex relational data in a machine-understandable format" \cite{cao2022}. This transition to embedding techniques is crucial for unlocking the full potential of KGs in modern AI applications, providing a robust and scalable foundation for knowledge inference and fusion \cite{ge2023, dai2020}.