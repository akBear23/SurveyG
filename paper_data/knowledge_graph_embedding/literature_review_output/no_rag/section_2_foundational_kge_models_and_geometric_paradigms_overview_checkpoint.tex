\section{Foundational KGE Models and Geometric Paradigms}
\label{sec:foundational_kge_models_and_geometric_paradigms}

Building upon the motivation to overcome the limitations of sparse symbolic knowledge graphs, this section delves into the pioneering efforts that established the bedrock of knowledge graph embedding research. It explores the early and influential models that first translated entities and relations into continuous vector spaces, laying the theoretical and practical groundwork for subsequent advancements. The primary focus here is on geometric and algebraic paradigms, which conceptualize relations not merely as static links, but as dynamic transformations or interactions within these learned embedding spaces. This fundamental shift enabled the capture of implicit semantic similarities and relational patterns that were previously inaccessible.

The evolution began with simple yet powerful translational models, such as TransE and its extensions like TransH \cite{wang2014} and TransD \cite{ji2015}. These models represent relations as direct translations from head to tail entities, offering an efficient way to capture basic relational patterns. However, their inherent limitations in modeling complex properties like symmetry, antisymmetry, and composition spurred the development of more sophisticated approaches. This led to the emergence of rotational models, exemplified by RotatE \cite{sun2018}, which leverage rotations in complex or higher-dimensional spaces to capture richer and more diverse relational semantics. Further innovations explored other geometric and algebraic structures, including embeddings on Lie groups or using quaternions, to enhance expressiveness and address specific challenges. Collectively, these foundational models were instrumental in demonstrating how geometric operations could effectively capture intricate relational patterns, significantly improving the expressiveness and utility of KGEs and forming the essential basis for the field's rapid expansion into more advanced architectures.