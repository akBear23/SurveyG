\section{Deep Learning Architectures for Knowledge Graph Embedding}
\label{sec:deep_learning_architectures_for_knowledge_graph_embedding}

Building upon the foundational geometric and algebraic paradigms discussed in Section \ref{sec:foundational_kge_models_and_geometric_paradigms}, which established the initial framework for representing knowledge in continuous vector spaces, this section explores a significant paradigm shift in Knowledge Graph Embedding (KGE) research. While earlier geometric models, such as translational and rotational approaches, provided valuable insights into capturing specific relational patterns, their reliance on predefined transformations often limited their capacity to model the highly intricate, hierarchical, and non-linear relationships pervasive in real-world knowledge graphs \cite{rossi2020, cao2022}. The advent of deep learning has revolutionized this landscape, enabling the development of more expressive and context-aware KGE models by allowing them to automatically extract features and learn complex interactions directly from data.

This section delves into how advanced deep learning architectures have been adapted to overcome these limitations, pushing the boundaries of KGE performance. We will detail the application of Convolutional Neural Networks (CNNs), which are leveraged to capture local features and intricate structural patterns within triplets and their neighborhoods. Subsequently, we examine Graph Neural Networks (GNNs), including various attention mechanisms, which inherently excel at encoding rich structural information and neighborhood context through message passing, thereby learning more robust, context-dependent embeddings. Finally, we explore the emergence of Transformer models in KGE, demonstrating how their powerful self-attention mechanisms capture long-range dependencies and contextualized representations, enabling the modeling of both global and local semantic relationships. This architectural evolution marks a crucial advancement, facilitating the learning of nuanced, non-linear relationships and paving the way for significantly enhanced performance across a spectrum of downstream AI tasks \cite{dai2020, cao2022}.