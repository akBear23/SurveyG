\subsection{Domain-Specific Applications and Explainability}
The utility of Knowledge Graph Embedding (KGE) models extends significantly into specialized, high-stakes domains, where their ability to transform symbolic knowledge into actionable insights is paramount. This section highlights the application of KGE in fields such as biological systems, patent metadata analysis, and drug repurposing, emphasizing how these models are tailored, validated, and increasingly scrutinized for explainability to build trust and deliver verifiable solutions.

In biological and biomedical systems, KGE models offer powerful tools for discovery and analysis. \cite{mohamed2020} provides a comprehensive review of KGE applications in this domain, showcasing their predictive and analytical capabilities for tasks like drug-target interactions and polypharmacy side effects. The authors argue that KGEs are a natural fit for representing complex biological knowledge, overcoming the scalability limitations of traditional graph exploratory approaches. Building on this, \cite{zhu2022} demonstrates multimodal reasoning based on KGE for specific diseases. They construct Specific Disease Knowledge Graphs (SDKGs) and integrate structural, category, and description embeddings using reverse-hyperplane projection. This multimodal approach enhances the discovery of new, reliable knowledge, underscoring the value of tailoring KGEs to domain-specific knowledge structures and leveraging diverse data modalities to improve reasoning.

A particularly compelling example of KGE application in a critical domain is drug repurposing for diseases like COVID-19. \cite{islam2023} proposes an innovative approach that utilizes ensemble KGEs to generate robust latent representations, which are then fed into a deep neural network for identifying potential drug candidates. Crucially, this work moves beyond generic KGE evaluation metrics by incorporating \textit{molecular docking} to validate predictions, a domain-specific and verifiable method for assessing drug-target interactions. This integration of molecular-level evaluation is a significant step towards delivering transparent solutions, as it provides concrete, scientific validation for the abstract KGE predictions. Furthermore, \cite{islam2023} addresses the paramount need for explainability by providing explanations through rules extracted from the knowledge graph and instantiated by explanatory paths. This allows medical professionals and researchers to understand *why* a particular drug is predicted, fostering trust and enabling actionable insights in a field where decisions have direct human impact.

Beyond biomedicine, KGEs are also applied to analyze complex structured data like patent metadata. \cite{li2022} operationalizes knowledge proximity within the US Patent Database by training KGE models on a "PatNet" knowledge graph constructed from patent citations, inventors, assignees, and domain classifications. By using cosine similarity between learned embeddings, they measure knowledge proximity between homogeneous (e.g., patent-patent) and heterogeneous (e.g., inventor-assignee) entities. This application demonstrates how KGEs can be tailored to specific industry problems, providing quantitative measures for abstract concepts like "knowledge proximity" and enabling the analysis of domain expansion profiles for inventors and assignees. While this work primarily focuses on predictive performance, the inherent interpretability of proximity measures in the embedding space can offer insights into innovation landscapes.

The growing demand for interpretable KGE models in these high-stakes fields is a significant evolutionary trend. The explicit focus on explainability in \cite{islam2023} highlights a shift from merely achieving high performance on abstract metrics to delivering verifiable and transparent solutions. This aligns with broader research efforts in making KGEs more understandable. For instance, methods that integrate logical rules and constraints into the embedding process, such as \cite{guo2017}'s iterative guidance from soft rules (RUGE) or \cite{tang2022}'s RulE framework, which learns rule embeddings jointly with entity and relation embeddings, inherently contribute to explainability. By aligning embeddings with human-understandable logical patterns, these approaches can provide a basis for explaining model predictions. Similarly, \cite{ding2018} showed that even simple constraints like non-negativity on entity representations can improve model interpretability by structuring the embedding space.

However, achieving robust explainability in complex KGE models, especially those leveraging deep learning architectures, remains a challenge. The trade-off often lies between the high expressiveness and predictive power of complex models and the inherent difficulty in extracting clear, human-understandable explanations from their latent spaces. While \cite{islam2023} successfully combines ensemble KGEs with molecular docking and rule-based explanations, the generalizability of such multi-faceted explanation strategies across all domain-specific KGE applications requires further investigation. The theoretical gap in universally interpretable embedding spaces, particularly for highly non-linear models, prevents a straightforward solution to providing transparent insights for every prediction. Nevertheless, the explicit integration of domain-specific validation and explanation mechanisms, as exemplified by these works, marks a crucial step towards building trust and enabling the responsible deployment of KGE technologies in critical real-world scenarios.