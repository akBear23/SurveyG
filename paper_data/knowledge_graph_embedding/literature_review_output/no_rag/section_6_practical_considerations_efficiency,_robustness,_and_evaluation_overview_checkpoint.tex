\section{Practical Considerations: Efficiency, Robustness, and Evaluation}
\label{sec:practical_considerations_efficiency_robustness_and_evaluation}

Building upon the advancements in dynamic, inductive, and distributed knowledge graph embedding (KGE) models discussed in Section \ref{sec:dynamic_inductive_and_distributed_kge}, which focused on adapting KGEs to evolving and decentralized knowledge, this section shifts its focus to the critical operational challenges of deploying and evaluating these models in real-world scenarios. While previous sections explored theoretical foundations and architectural innovations, the true utility of KGEs hinges on their practical viability, encompassing computational efficiency, resilience to imperfect data, and the trustworthiness of their empirical validation. This section bridges the gap between theoretical progress and reliable real-world application, addressing the fundamental requirements for KGE models to move from research prototypes to robust, scalable, and trustworthy components of intelligent systems.

We delve into three interconnected areas vital for practical KGE deployment. First, we examine strategies for enhancing \textit{efficiency, compression, and scalability}, which are paramount for handling the immense size and complexity of modern knowledge graphs \cite{community_1, community_6}. This involves techniques to reduce computational cost, minimize memory footprint, and optimize training and inference processes, making KGE models viable for resource-constrained environments and large-scale applications. Second, we explore methods for improving \textit{robustness and training optimization}, crucial for mitigating the impact of noisy, incomplete, or imbalanced data inherent in real-world KGs \cite{community_2, community_3, 2a3f862199883ceff5e3c74126f0c80770653e05}. This includes advanced negative sampling techniques and noise filtering mechanisms that ensure models learn accurate and reliable representations. Finally, a significant portion is dedicated to the importance of rigorous \textit{evaluation, benchmarking, and reproducibility} \cite{community_0, community_6}. This area underscores the necessity for standardized metrics, fair comparisons, and transparent research practices to ensure that KGE advancements are scientifically sound and lead to reliable, trustworthy applications. Together, these practical considerations are indispensable for translating the rich theoretical landscape of KGE into impactful and dependable real-world solutions.