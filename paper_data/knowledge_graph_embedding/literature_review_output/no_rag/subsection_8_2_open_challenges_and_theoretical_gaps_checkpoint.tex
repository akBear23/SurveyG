\subsection{Open Challenges and Theoretical Gaps}
Despite significant advancements in Knowledge Graph Embedding (KGE) research, several critical open challenges and theoretical gaps persist, representing fertile ground for future investigation. These issues often stem from inherent trade-offs, the complexity of real-world knowledge graphs, and the limitations of current theoretical understandings.

One pervasive challenge lies in balancing model expressiveness with computational complexity. While models like RotatE \cite{sun2018} and the composition-closed HolmE \cite{zheng2024} have pushed the boundaries of capturing intricate relational patterns, their increased expressiveness often comes at the cost of higher computational demands for training and inference (Subgroup 1, Overall Perspective). Efforts in efficiency and compression, such as DualDE's distillation \cite{zhu2020} and LightKG's codebook-based storage \cite{wang2021}, aim to mitigate this. However, as noted by \cite{sachan2020}, these often entail a "minor loss in performance." The theoretical gap here is to devise architectures that are *inherently* expressive yet computationally lean, perhaps through novel mathematical formulations like the Orthogonal Procrustes Analysis in \cite{peng2021}, which offers a closed-form solution for efficiency, rather than relying on post-hoc compression or distillation.

Ensuring the interpretability of complex deep learning KGE models is another significant hurdle. As KGE increasingly leverages Graph Neural Networks (GNNs) and Transformer architectures (Subgroup 6, Advanced Model Design), their black-box nature becomes a concern, especially in high-stakes applications. While some models like SpherE \cite{li2024} claim interpretability through their geometric properties, this is often specific to the model's design and does not generalize to the complex reasoning paths learned by deep networks. Application-focused works, such as the explainable drug repurposing by \cite{islam2023} and contextualized recommendation by \cite{yang2023}, demonstrate the *need* for explanations, but typically rely on post-hoc rule extraction or path saliency. A theoretical gap exists in developing intrinsically interpretable deep KGE models that can transparently reveal their reasoning processes without sacrificing predictive power.

The efficient extraction and integration of high-quality logical rules remain a bottleneck. While methods like RUGE \cite{guo2017} and RulE \cite{tang2022} have shown the value of incorporating soft rules to enhance reasoning and consistency (Subgroup 3, Rule-based & Constraint-driven KGE), the process of obtaining these rules and balancing their adherence with the flexibility to capture exceptions is challenging. Automatically extracted rules often carry "uncertainties" \cite{guo2017}, and the scalability of integrating complex rule sets can be problematic \cite{guo2020}. The theoretical challenge lies in developing robust, automated rule induction systems that can generate high-fidelity rules from noisy KGs and seamlessly integrate them into embedding models without introducing significant computational overhead or compromising the data-driven learning of nuanced patterns.

Furthermore, resolving issues related to the 'true' negative distribution in training is fundamental. KGE models rely heavily on negative sampling for contrastive learning, yet the "true" negative distribution is inherently unknown (Subgroup 3, Negative Sampling & Training Optimization). While approaches like NSCaching \cite{zhang2018} and confidence-aware sampling \cite{shan2018} improve efficiency and robustness to noise, they are still heuristic approximations. The "Efficient Non-Sampling Knowledge Graph Embedding" \cite{li2021} attempts to bypass sampling entirely, but requires complex mathematical derivations to manage computational complexity. As highlighted by comprehensive reviews \cite{qian2021, madushanka2024}, this remains a persistent challenge, particularly when extending to multi-modal KGE \cite{zhang2023}. A theoretical breakthrough is needed to either accurately model the true negative distribution or develop training paradigms that are robust to its uncertainty without prohibitive computational costs.

The field also critically needs more robust and unbiased evaluation metrics. As revealed by the "Evaluation, Benchmarking, and Reproducibility" subgroup, standard metrics often suffer from biases, such as the over-representation of certain entities \cite{rossi2020}, and reported results can be sensitive to hyperparameter choices or even data leakage \cite{lloyd2022}. The lack of standardized frameworks and reproducibility issues \cite{ali2020, broscheit2020} further complicate fair comparisons. The theoretical gap is in developing evaluation protocols that are robust to dataset characteristics, sensitive to diverse relational patterns (e.g., long-tail, compositional), and truly reflect real-world application performance, moving beyond simple link prediction accuracy.

The challenges of scalability for extremely large and dynamic knowledge graphs are paramount. While progress has been made in efficiency \cite{peng2021, zheng2024} and compression \cite{sachan2020, wang2021}, handling KGs with billions of triples and continuous, real-time updates remains difficult. Dynamic KGE methods like FastKGE \cite{liu2024} and MetaHG \cite{sun2024} focus on efficient updates and mitigating catastrophic forgetting, but the sheer volume and velocity of changes in real-world KGs can still overwhelm these systems. Federated KGE introduces additional complexities related to communication efficiency \cite{zhang2024} and personalized aggregation for diverse client data \cite{zhang2024_personalized}. The theoretical challenge is to design truly elastic and adaptive KGE architectures that can scale horizontally, handle continuous streams of new information, and maintain global consistency and performance in highly distributed and dynamic environments without prohibitive computational or communication costs.

Finally, the development of truly generalizable inductive models remains an open problem. While neighborhood aggregation \cite{wang2018} and meta-learning approaches \cite{chen2021, sun2024} have enabled inductive capabilities for new entities (Subgroup 4, Dynamic, Inductive, and Continual KGE), they often rely on existing neighbors or transferable meta-knowledge. Truly novel entities with entirely new semantic or structural patterns, or isolated entities with sparse connections, still pose a significant challenge. A deeper theoretical understanding of "semantic evidence" for extrapolation \cite{li2021} is required to build models that can robustly infer representations for such unseen entities or entirely new relational types, pushing beyond mere interpolation to genuine generalization. These challenges collectively underscore the need for continued fundamental and applied research to unlock the full potential of KGE.