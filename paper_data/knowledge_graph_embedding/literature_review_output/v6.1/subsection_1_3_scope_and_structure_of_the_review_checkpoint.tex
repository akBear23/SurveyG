This literature review offers a comprehensive and systematic exploration of Knowledge Graph Embedding (KGE), charting the field's intellectual trajectory from its foundational theoretical underpinnings to its most advanced architectural innovations and diverse real-world applications. KGE models are designed to transform sparse, symbolic knowledge into dense, continuous vector representations, thereby enabling more efficient computation and sophisticated reasoning within AI systems \cite{choudhary2021, layer_1, community_1}. This transformation serves dual purposes: providing effective encodings for various data mining tasks and facilitating robust link prediction within knowledge graphs \cite{kge_purposes}. Our review's scope is deliberately broad, encompassing the evolution of KGE models, their practical considerations, and their diverse utility across domains. We adopt a pedagogical progression, beginning with core concepts and gradually advancing to cutting-edge developments, ensuring a coherent narrative that illuminates the motivations behind successive innovations and the continuous efforts to overcome inherent limitations. This structured approach provides a clear roadmap for understanding the complex and rapidly evolving landscape of KGE research, as highlighted by numerous comprehensive surveys in the field \cite{layer_1, community_1, community_2, community_3}.

The review commences by establishing the foundational context of knowledge graphs and the compelling motivations for their embedding into continuous vector spaces, addressing the inherent limitations of symbolic representations in terms of scalability, semantic richness, and handling incompleteness \cite{layer_1, community_1}. This sets the stage for Section 2, "Foundational KGE Models and Geometric Paradigms," which delves into the bedrock of the field. This section examines early geometric and algebraic models that laid the groundwork for representing entities and relations in continuous spaces \cite{community_0, community_1}. While these pioneering approaches significantly advanced the field by offering improved efficiency over purely symbolic methods, they often faced limitations in capturing complex relational patterns such as symmetry, antisymmetry, composition, or the nuances of weighted relationships \cite{weext_weighted_kge}. This critical evaluation of their expressiveness and computational trade-offs highlights the impetus for more sophisticated mathematical formulations.

Building upon these foundations, Section 3, "Deep Learning Architectures for Knowledge Graph Embedding," explores the significant paradigm shift towards leveraging advanced deep learning models. This section details how Convolutional Neural Networks (CNNs), Graph Neural Networks (GNNs), and Transformer models have been adapted to learn more expressive and context-aware representations \cite{community_0, community_2}. Unlike purely geometric models that rely on predefined transformations, these architectures automatically extract intricate structural patterns and model complex, non-linear interactions, pushing the boundaries of KGE performance. However, this enhanced expressiveness often introduces increased computational demands and requires larger datasets for effective training, presenting a new set of practical challenges that the field continues to address.

Recognizing that purely structural information can be insufficient for comprehensive knowledge representation, Section 4, "Enriching KGE: Auxiliary Information, Rules, and Multi-modality," investigates methods that integrate diverse external knowledge sources and logical constraints \cite{community_0, community_3}. This includes leveraging entity types and attributes, incorporating explicit logical rules, and fusing multi-modal data (e.g., text, images). This critical development addresses data sparsity, enhances semantic understanding, and improves reasoning capabilities, moving KGE models towards more robust and interpretable representations in complex, real-world scenarios. This integration of heterogeneous information is crucial for overcoming the limitations of relying solely on the graph's topological structure.

The review then addresses the dynamic, evolving, and often distributed nature of real-world knowledge graphs in Section 5, "Dynamic, Inductive, and Distributed KGE." This section covers methods for handling temporal changes, learning embeddings for unseen entities (inductive learning), and efficiently updating models with new facts (continual learning) \cite{community_1, community_4}. Furthermore, it explores emerging federated and privacy-preserving KGE approaches that enable collaborative learning across decentralized data sources \cite{community_1, 8c93f3cecf79bd9f8d021f589d095305e281dd2f}. These advancements are crucial for making KGE models adaptable, scalable, and secure in operational environments, moving beyond static and centralized assumptions to meet the demands of modern knowledge management systems.

Section 6, "Practical Considerations: Efficiency, Robustness, and Evaluation," shifts focus to the critical practical challenges in deploying and evaluating KGE models. Drawing insights from comprehensive comparative analyses and surveys \cite{madushanka2024}, this section examines strategies for improving computational efficiency, reducing memory footprint, and ensuring scalability for massive knowledge graphs \cite{community_1, community_6}. It delves into methods for enhancing model robustness against data imperfections and optimizing training processes, recognizing that factors beyond scoring functions, such as loss functions, hyperparameters, and negative sampling strategies, significantly impact model efficiency and accuracy \cite{kge_training_components}. A significant part is dedicated to the importance of rigorous evaluation, standardized benchmarking, and reproducibility, which are essential for translating theoretical advancements into reliable and trustworthy real-world applications \cite{community_0, community_6}. The varied performance reported across benchmarks underscores the need for robust evaluation protocols and transparent reporting practices.

Finally, Section 7, "Applications and Real-World Impact of KGE," showcases the diverse and significant real-world utility of KGE across various AI applications. This includes fundamental tasks like link prediction and knowledge graph completion \cite{community_0, community_1}, as well as more complex applications such as entity alignment \cite{layer_1, community_5}, question answering, and recommender systems \cite{layer_1, community_3}. This section demonstrates how KGE models are leveraged to solve complex problems, not only for traditional tasks but also for advanced data representation and analysis, enabling semantic queries and data exploration based on learned semantic structures \cite{kge_semantic_structures}. This illustrates their transformative potential in different industries and highlights the tangible benefits of embedding techniques in modern AI systems. The review concludes in Section 8 with a summary of key developments, a discussion of persistent open challenges, theoretical gaps, and emerging trends, including ethical considerations, charting a course for future research and development in the field \cite{community_1, community_6}. This structured approach provides a comprehensive roadmap for understanding the complex and rapidly evolving landscape of KGE research.