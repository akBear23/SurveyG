Convolutional Neural Networks (CNNs) have emerged as a powerful deep learning paradigm for Knowledge Graph Embedding (KGE), offering a distinct advantage in extracting local features and modeling intricate interactions between entity and relation embeddings. Unlike traditional geometric models that rely on predefined transformations, CNNs automatically learn complex, non-linear feature patterns, thereby enhancing the expressiveness of KGE models \cite{cao2022, ge2023}. This approach often treats the concatenated embeddings of head entities and relations as a 2D input, applying convolutional filters to capture diverse interaction patterns.

Early applications of CNNs in KGE, such as AcrE \cite{ren2020}, leveraged atrous convolutions and residual learning to effectively increase feature interactions while maintaining parameter efficiency. AcrE demonstrated that by using dilated convolutions, models could capture a wider receptive field without increasing computational cost, addressing the challenge of modeling complex relation patterns with a simpler yet effective architecture. Similarly, the Multi-Scale Dynamic Convolutional Network (M-DCN) \cite{zhang2020} introduced multi-scale dynamic filters to generate richer and more expressive feature embeddings. M-DCN specifically tackled complex relation types, including 1-to-N, N-to-1, and N-to-N, by dynamically adapting filter weights to each relation, thereby enabling a more nuanced understanding of diverse relational semantics. This dynamic filtering mechanism allows the model to learn relation-specific feature extractors, a significant improvement over static filters.

Further advancements integrated more sophisticated neural network components. ReInceptionE \cite{xie2020} combined the power of Inception networks with attention mechanisms to capture both local and global structural information. By employing an Inception network, ReInceptionE increased the interactions between head and relation embeddings, while its relation-aware attention mechanism enriched query embeddings with local neighborhood and global entity context. This hybrid approach showcased the potential of CNNs to not only learn intricate feature interactions but also to incorporate broader graph context, moving beyond purely local pattern recognition. This represents a critical step in bridging the gap between local feature extraction and global structural understanding, which is often a limitation of purely convolutional approaches.

More recent works continue to refine CNN-based KGE. CNN-ECFA \cite{hu2024} proposed a Convolutional Neural Network-based Entity-specific Common Feature Aggregation strategy, demonstrating that aggregating entity-specific common features can significantly improve knowledge graph representation learning. This model focuses on enhancing the feature projection strategies, achieving notable improvements in link prediction tasks by learning more effective representations. Similarly, SEConv \cite{yang2025} introduced a semantic-enhanced KGE model, employing a multi-layer convolutional neural network alongside a self-attention mechanism. SEConv's design aims to extract deeper structural features and generate more expressive embedding representations, particularly highlighting its potential for application in resource-limited consumer electronics and specialized domains like healthcare prediction. The integration of self-attention in SEConv further addresses the limitation of CNNs in capturing long-range dependencies, a common challenge for models relying solely on fixed-size convolutional kernels.

Overall, CNN-based KGE models demonstrate a clear intellectual trajectory towards learning intricate, non-linear feature interactions for improved link prediction. Their strength lies in their ability to automatically discover complex patterns from raw embedding inputs, often achieving state-of-the-art results on benchmark datasets. However, this expressiveness often comes with trade-offs. Compared to simpler geometric models like TransE or RotatE, CNN-based models typically have a higher computational cost and a larger number of parameters, which can impact scalability on extremely large knowledge graphs \cite{ge2023}. Furthermore, while they excel at local feature extraction, their capacity to capture global graph structures or complex logical reasoning patterns might be limited without explicit architectural additions like attention mechanisms or graph neural network components \cite{wu2021, li2023}. The interpretability of the learned features within deep CNN layers also remains a challenge, making it difficult to fully understand *why* certain predictions are made, a common critique across deep learning architectures as noted in the broader analysis of deep learning KGE models. Despite these challenges, the continuous refinement of CNN architectures, including dynamic filtering, multi-scale processing, and attention integration, underscores their enduring relevance and potential for advancing KGE research.