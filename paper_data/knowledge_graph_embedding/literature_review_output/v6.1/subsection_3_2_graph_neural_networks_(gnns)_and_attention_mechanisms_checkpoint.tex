\subsection{Graph Neural Networks (GNNs) and Attention Mechanisms}

The evolution of Knowledge Graph Embedding (KGE) has seen a significant shift from purely geometric and algebraic models to sophisticated deep learning architectures, with Graph Neural Networks (GNNs) and attention mechanisms emerging as particularly powerful tools. GNNs, through their inherent message passing and aggregation frameworks, are uniquely suited to capture the rich structural information and neighborhood context within knowledge graphs, moving beyond the limitations of simple triplet-based interactions. This capability is crucial for understanding complex relational patterns that span multiple hops and involve diverse entities.

Early forays into leveraging graph structures for KGE, particularly for inductive capabilities, can be seen in models like Logic Attention-based Neighborhood Aggregation \cite{wang2018}. This approach introduced the Logic Attention Network (LAN) to aggregate information from an entity's neighbors, enabling the embedding of unseen entities by considering both rule-based and network-based attention weights. This marked a significant step towards inductive KGE, addressing the challenge of dynamically evolving knowledge graphs where new entities frequently emerge \cite{chen2021, sun2024}. However, these early methods often faced limitations in fully capturing the complex, multi-faceted nature of relations and entity interactions, sometimes relying on simplistic aggregation functions.

The integration of explicit attention mechanisms has further enhanced GNNs in KGE, allowing models to dynamically weigh the importance of different neighbors and relational paths. Graph Attenuated Attention Networks (GAATs) \cite{wang2020}, for instance, incorporate an attenuated attention mechanism to assign varying weights to different relation paths and acquire information from neighborhoods. This allows entities and relations to be learned in a more nuanced, context-dependent manner, overcoming the limitation of assigning uniform weights to all neighbors or paths. While GAATs improve feature extraction, their attention mechanism might still be constrained by a single, aggregated representation.

A more advanced approach is seen in DisenKGAT \cite{wu2021}, which proposes a novel Disentangled Graph Attention Network for KGE. DisenKGAT addresses the challenge of accurately capturing complex relations by leveraging both micro-disentanglement (relation-aware aggregation for diverse component representations) and macro-disentanglement (mutual information regularization to enhance independence). This disentangled approach allows the model to generate more diverse and adaptive representations, which is critical for handling the polysemy and multifaceted nature of real-world relations. The disentanglement not only improves accuracy but also offers enhanced explainability, a growing demand in complex AI systems. Compared to simpler attention models, DisenKGAT's explicit disentanglement aims to prevent a single, monolithic representation from dominating, thereby capturing a broader spectrum of relational semantics.

Beyond specific attention mechanisms, the broader application of GNNs in KGE continues to evolve. Researchers are exploring how to optimize the fundamental message functions within GNNs to improve data adaptability and performance across various KG forms, including n-ary and hyper-relational data \cite{di2023}. This "message function search" aims to automatically discover optimal GNN architectures tailored to specific datasets, addressing the challenge of designing effective GNNs manually. Furthermore, understanding *how* GNN-based KGE models extrapolate to unseen data is a critical area of research. Models like SE-GNN \cite{li2021} explicitly model and merge "Semantic Evidences" at relation, entity, and triple levels through multi-layer aggregation, demonstrating how theoretically informed GNN designs can lead to better extrapolation abilities.

Despite their significant strengths in capturing complex structural information and enabling inductive learning, GNN-based KGE models are not without limitations. Their computational complexity can be substantial, especially with deep GNN layers or intricate attention mechanisms, posing scalability challenges for extremely large knowledge graphs. This is a common trade-off between model expressiveness and computational efficiency, a problem that other research areas in KGE, such as efficiency and compression \cite{modak2024}, actively address. Furthermore, while attention mechanisms provide a degree of interpretability by highlighting important paths or neighbors, the overall "black-box" nature of deep GNNs can still make it challenging to fully understand *why* a particular embedding or prediction is made. The ongoing research into message function search \cite{di2023} and understanding extrapolation \cite{li2021} represents efforts to mitigate these limitations by designing more adaptive, efficient, and interpretable GNN architectures for KGE.