Building upon the foundational geometric and algebraic paradigms discussed in the previous section, which primarily modeled relations as transformations in continuous vector spaces, this section marks a significant paradigm shift in Knowledge Graph Embedding (KGE) research. While models like TransE and RotatE provided crucial initial frameworks for representing entities and relations \cite{bordes2013, sun2018}, their inherent limitations in capturing highly intricate, non-linear, and hierarchical structural patterns within knowledge graphs became increasingly apparent. The emergence of advanced deep learning architectures has revolutionized KGE, enabling the learning of far more expressive and context-aware representations directly from data \cite{community_0, community_1, community_2}.

This section delves into how Convolutional Neural Networks (CNNs), Graph Neural Networks (GNNs), and Transformer models have been innovatively adapted to address these challenges. CNNs, initially renowned for image processing, are explored for their ability to extract local features and model intricate interactions between entity and relation embeddings, automatically discovering complex patterns that geometric models struggle with. Subsequently, we examine GNNs, which are inherently suited for graph-structured data. Through message passing and aggregation mechanisms, GNNs effectively capture rich structural information and neighborhood context, moving beyond simple triplet-based interactions to leverage the graph's full topology and relational paths. Finally, the discussion extends to Transformer-based KGE models, which harness powerful self-attention mechanisms to capture long-range dependencies and contextualized representations, pushing the state-of-the-art in modeling complex contextual information and multi-structural features \cite{8f096071a09701012c9c279aee2a88143a295935, d899e434a7f2eecf33a90053df84cf32842fbca9}. By detailing these architectural advancements, this section highlights how deep learning has pushed the boundaries of KGE performance, enabling the automatic extraction of features and the modeling of complex, non-linear relationships, thereby providing a more nuanced and powerful understanding of knowledge graphs.