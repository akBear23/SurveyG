\subsection*{Robustness and Training Optimization}
The effectiveness of Knowledge Graph Embedding (KGE) models in real-world applications is profoundly influenced by their robustness to data imperfections and the efficiency of their training processes. Knowledge graphs (KGs) are inherently noisy, incomplete, and often exhibit imbalanced data distributions, where a small fraction of entities and relations are frequent while the majority are long-tail \cite{zhang2023}. Addressing these challenges is crucial for ensuring that KGE models learn accurate, reliable representations and generalize well to unseen data. This subsection delves into methodologies designed to enhance KGE robustness and optimize training, with a particular focus on probability calibration, noise filtering, weighted training, and the critical role of negative sampling.

A fundamental aspect of model reliability is the trustworthiness of its predictions. While KGE models often achieve high accuracy in link prediction, their associated probability estimates can be uncalibrated, meaning that a predicted probability of 0.8 might not truly correspond to an 80\% chance of correctness. \cite{tabacof2019} highlights this overlooked problem, demonstrating that popular embedding models indeed produce uncalibrated probabilities. They propose a novel calibration method, applicable even when ground truth negatives are scarce, by employing techniques like Platt scaling and isotonic regression. This post-processing step significantly improves the reliability of KGE outputs, a vital consideration for high-stakes applications, though it does not address the underlying representational limitations of the KGE model itself. Isotonic regression, in particular, showed strong performance, albeit with potential trade-offs in computational complexity compared to simpler methods like Platt scaling.

To combat the pervasive issue of noisy data, which arises from automatic KG construction and updates, \cite{zhang2021} introduced a multi-task reinforcement learning (RL) framework. This framework aims to filter out noisy triples during training, thereby guiding the KGE model to learn more robust representations. By exploiting correlations among semantically similar relations through multi-task learning, the RL agent collectively selects high-quality triples. While effective in enhancing existing KGE models like TransE, DistMult, ConvE, and RotatE in noisy scenarios, this approach introduces additional complexity to the training process due to the RL component. The trade-off lies between the improved robustness gained from cleaner training data and the increased computational overhead and tuning effort required for the RL agent.

Another significant challenge in KGE training is data imbalance, where entities and relations follow a long-tail distribution. Traditional KGE methods often assign equal weights to all triples, leading to under-trained representations for infrequent entities and relations. To address this, \cite{zhang2023} proposed Weighted Knowledge Graph Embedding (WeightE), which employs a bilevel optimization scheme. The inner level focuses on learning reliable embeddings, while the outer level dynamically assigns higher weights to infrequent entities and relations and lower weights to frequent ones. This differential weighting ensures that long-tail components receive adequate attention during training, leading to more balanced and reliable representations across the entire KG. The flexibility of WeightE allows its application to various existing KGE models, offering a general solution to a common practical problem. However, the effectiveness of bilevel optimization can depend on careful hyperparameter tuning for the weighting mechanism.

A crucial and often complex aspect of KGE training optimization revolves around negative sampling. Since KGs typically store only positive facts, generating "negative" (false) triples is essential for contrastive learning, but the quality and efficiency of this process profoundly impact model performance \cite{qian2021, madushanka2024}. The challenge stems from the unknown true distribution of negative facts, making heuristic sampling strategies prone to generating "false negatives" (actual positive triples mistakenly labeled as negative) or "easy negatives" (obviously false triples that provide little learning signal).

Several strategies have emerged to refine negative sampling:
\begin{itemize}
    \item \textbf{Confidence-Aware Sampling:} In noisy KGs, uniform negative sampling can exacerbate issues. \cite{shan2018} introduced a confidence-aware negative sampling method that incorporates negative triple confidence. This approach aims to improve training in noisy environments by assigning a confidence score to negative triples, thereby mitigating the zero-loss and false detection problems associated with uniform sampling. However, the accuracy of this confidence estimation is critical; a flawed estimation could inadvertently introduce new biases or false negatives.
    \item \textbf{Caching Strategies:} To efficiently identify and leverage "hard" negative samples (those that are plausible but incorrect, providing strong learning signals), \cite{zhang2018} proposed NSCaching. Inspired by the observation that hard negatives are crucial but rare, NSCaching uses a cache to track and sample these challenging triples. This method aims to distill the benefits of more complex Generative Adversarial Network (GAN)-based negative sampling techniques into a simpler, more efficient framework, offering a good balance between exploration and exploitation of negative samples without the added complexity of GAN training.
    \item \textbf{Non-Sampling Approaches:} A more radical departure from traditional methods is the "Efficient Non-Sampling Knowledge Graph Embedding" (NS-KGE) framework by \cite{li2021}. This approach avoids negative sampling entirely by considering all negative instances. While this theoretically eliminates the uncertainty inherent in sampling, it significantly increases computational complexity. \cite{li2021} addresses this by leveraging mathematical derivations to reduce the complexity of the non-sampling loss function, aiming for more stable and accurate performance. The applicability of this method is primarily limited to square-loss based KGE models or those whose loss can be converted to a square loss.
    \item \textbf{Modality-Aware Negative Sampling:} As KGE increasingly integrates multi-modal information, negative sampling strategies must adapt. \cite{zhang2023} introduced Modality-Aware Negative Sampling (MANS) for multi-modal KGE. MANS aligns structural and visual embeddings for entities, generating meaningful embeddings for multi-modal KGE while remaining lightweight and efficient. This highlights that the definition of a "good" negative sample becomes more intricate when diverse data modalities are involved, requiring specialized sampling techniques.
    \item \textbf{Comprehensive Reviews:} The importance and diversity of negative sampling methods are underscored by systematic reviews such as \cite{qian2021} and \cite{madushanka2024}. These surveys categorize existing approaches (e.g., static, dynamic, custom cluster-based) and provide valuable insights into their advantages, disadvantages, and open research questions, guiding future advancements in this critical area.
\end{itemize}

In summary, the advancements in robustness and training optimization represent a crucial intellectual trajectory in KGE research, moving beyond purely architectural innovations to address the practical challenges of real-world data. While models like \cite{zhang2021} and \cite{zhang2023} enhance robustness against noise and imbalance, respectively, \cite{tabacof2019} ensures the reliability of model outputs. Simultaneously, the continuous refinement of negative sampling, from confidence-aware strategies \cite{shan2018} and caching \cite{zhang2018} to non-sampling paradigms \cite{li2021} and modality-aware adaptations \cite{zhang2023}, demonstrates a deep commitment to optimizing the fundamental learning process. These efforts collectively ensure that KGE models can learn accurate and generalizable representations even from imperfect data, leading to more reliable predictions and broader applicability across diverse AI tasks. The ongoing tension lies in balancing the increased model and training complexity with the gains in robustness and predictive reliability.