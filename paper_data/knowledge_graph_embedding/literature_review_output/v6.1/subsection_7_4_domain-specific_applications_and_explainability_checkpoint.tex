\subsection{Domain-Specific Applications and Explainability}
The utility of Knowledge Graph Embeddings (KGEs) extends far beyond generic link prediction benchmarks, finding critical applications in specialized, high-stakes domains where verifiable insights and interpretability are paramount. This subsection highlights the application of KGE in fields such as biological systems, patent metadata analysis, and drug repurposing, emphasizing how these models are tailored, validated with domain-specific metrics, and increasingly designed for explainability to build trust and provide actionable intelligence.

In the realm of biological and biomedical systems, KGE models offer powerful tools for understanding complex interactions. \cite{mohamed2020} provides a comprehensive review of KGE applications in this domain, showcasing their predictive and analytical capabilities for tasks like drug-target interactions and polypharmacy side effects. The authors emphasize that KGE models are a natural fit for representing intricate biological knowledge, which is inherently graph-structured. Building upon this, \cite{zhu2022} demonstrates the construction and multimodal reasoning capabilities of Specific Disease Knowledge Graphs (SDKGs). Their work integrates structural, category, and description embeddings to discover new, reliable knowledge for specific diseases, such as various cancers. This multimodal approach, while enhancing predictive power, also implicitly demands a higher degree of confidence in its outputs, given the critical nature of health-related predictions. However, while these works highlight the *application* of KGEs, the explicit mechanisms for *explaining* individual predictions often remain a challenge, as the interpretability of the underlying embedding space is not always transparent.

A compelling example of KGE in a high-stakes domain is drug repurposing for diseases like COVID-19. \cite{islam2023} proposes an ensemble KGE approach combined with a deep neural network to identify potential drug candidates. Crucially, this work moves beyond conventional KGE evaluation metrics by integrating *molecular docking* to assess the binding affinity of predicted drugs to viral targets. This provides a tangible, domain-specific validation, directly addressing the need for verifiable solutions in medical research. Furthermore, \cite{islam2023} places a strong emphasis on explainability, generating insights through rules extracted from the knowledge graph and instantiated by explanatory paths. This dual focus on molecular validation and interpretability is vital for building trust and providing actionable insights to researchers, moving beyond a black-box prediction to reveal *why* a particular drug might be effective. While the specific ensemble and DNN architecture might be highly tuned for the COVID-19 task, the methodology of integrating domain-specific validation and explanation is broadly applicable and sets a high standard for future KGE applications in medicine.

Beyond biology, KGEs are also applied to analyze complex socio-economic datasets. \cite{li2022} utilizes KGE models to embed patent metadata, thereby operationalizing and measuring "knowledge proximity" within the US Patent Database. By training models on entities like patents, inventors, and assignees, they demonstrate how KGEs can associate homogeneous and heterogeneous entities and explain domain expansion profiles. This application provides valuable insights into innovation ecosystems, where understanding connections and trends is crucial for policy-making and strategic planning. While the embeddings facilitate the *explanation* of observed phenomena, the inherent explainability of the KGE model itself is often a secondary consideration, with the emphasis placed on the utility of the derived proximity measures.

The growing demand for interpretable KGE models in these high-stakes fields reflects a broader shift in AI research. Generic performance metrics, such as Hit@10 or MRR, are insufficient when decisions impact human health or significant economic outcomes. This necessitates a move towards solutions that are not only performant but also transparent and verifiable. Rule-based KGE models, like those explored by \cite{guo2017} (RUGE) and \cite{tang2022} (RulE), inherently offer a degree of explainability by aligning learned embeddings with human-understandable logical rules. These approaches inject prior knowledge and enforce semantic consistency, allowing for a more traceable inference process. Even models primarily focused on efficiency, such as \cite{peng2021}, acknowledge that their design can lead to "highly interpretable" entity embeddings. Furthermore, research into understanding *how* KGE models extrapolate to unseen data, as investigated by \cite{li2021} through "Semantic Evidences," contributes to building a more transparent foundation for KGE predictions.

In summary, the application of KGE in specialized domains like drug discovery and patent analysis underscores a critical evolution in the field. The focus has expanded from merely achieving high accuracy on general benchmarks to delivering verifiable, transparent, and actionable solutions tailored to specific industry problems. This transition often involves the integration of domain-specific evaluation metrics (e.g., molecular docking) and a concerted effort to enhance model explainability, bridging the gap between abstract embedding spaces and concrete, trustworthy insights. The ongoing challenge lies in balancing the expressiveness and performance of complex KGE models with the imperative for clear, human-understandable explanations in sensitive application contexts.