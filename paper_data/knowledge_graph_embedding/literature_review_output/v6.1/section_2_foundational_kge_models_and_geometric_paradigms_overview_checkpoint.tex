Building upon the motivation for Knowledge Graph Embeddings (KGE) to overcome the limitations of sparse symbolic representations, this section delves into the foundational models that first translated entities and relations into continuous vector spaces. It explores the bedrock of KGE research, focusing on early and influential models that laid the theoretical and practical groundwork for representing knowledge in a machine-understandable, continuous format. Central to these pioneering efforts are geometric and algebraic paradigms, where relations are conceptualized as transformations or interactions within these learned embedding spaces. The journey began with simpler translational models, such as TransE and its extensions like TransH, TransR, and TransD, which effectively modeled relations as vector translations from head to tail entities \cite{bordes2013, wang2014, ji2015}. These initial breakthroughs significantly improved upon purely symbolic methods by offering enhanced efficiency and a basic level of semantic expressiveness, establishing a fundamental paradigm for KGE \cite{community_0, community_1}. However, the inherent limitations of simple translations in capturing complex relational patterns, such as symmetry, antisymmetry, inversion, or composition, quickly became apparent. This spurred the development of more sophisticated approaches, leading to the emergence of rotational and complex space embeddings. Models like RotatE, which define relations as rotations in complex vector spaces, demonstrated superior capabilities in modeling these intricate logical patterns, moving beyond the constraints of linear transformations \cite{sun2019}. Further innovations explored diverse geometric and algebraic structures, including embeddings on Lie groups, different metric choices, and advanced transformations like Householder parameterization, continuously refining the mathematical foundations of KGE for greater expressiveness and theoretical soundness \cite{community_2}. Collectively, these foundational geometric and algebraic models not only formed the initial theoretical and practical basis for KGE but also highlighted the continuous quest to capture diverse relational semantics, thereby setting the stage for subsequent advancements in the field.