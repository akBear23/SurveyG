\subsection{Rotational and Complex Space Embeddings}
While translational models like TransE \cite{wang2014} and its extensions such as TransH \cite{wang2014} and TransD \cite{ji2015} offered significant advancements in knowledge graph embedding (KGE), their inherent limitations in capturing complex relational patterns, particularly those involving symmetry, antisymmetry, inversion, and composition, became apparent. These simpler geometric operations often struggled to differentiate between various relation types or to model multi-hop reasoning effectively. This limitation spurred the development of a new class of KGE models that leverage rotations in complex or higher-dimensional spaces, offering more nuanced and powerful transformations.

A pivotal contribution in this area is RotatE \cite{sun2018}, which defines each relation as a rotation from the head entity to the tail entity in a complex vector space. This elegant formulation inherently allows RotatE to model and infer a rich set of relational patterns. For instance, symmetric relations can be represented by a rotation of 0 or $\pi$ (or multiples thereof), antisymmetric relations by a non-zero rotation, and inverse relations by a rotation that is the negative of the original. Crucially, composition of relations (e.g., $r_1 \circ r_2 = r_3$) naturally translates to the composition of rotations in the complex plane, making it highly effective for complex logical patterns. As highlighted in the thematic taxonomy, RotatE represents a significant step within the "Advanced Geometric and Temporal Models" group, moving beyond the simpler translational assumptions to capture richer semantics \cite{dasgupta2018}. Its success demonstrated that complex number spaces could provide a more expressive embedding environment without drastically increasing computational overhead compared to some projection-based methods.

Building upon the success of complex-space rotations, researchers explored extensions to higher-dimensional spaces. Rotate3D \cite{gao2020} maps entities into a three-dimensional Euclidean space and models relations as rotations within this space. A key advantage of Rotate3D lies in its ability to capture non-commutative composition, which is essential for accurate multi-hop reasoning where the order of relations matters. While RotatE effectively models composition in 2D complex space, Rotate3D's exploration of 3D rotations provides a richer algebraic structure, allowing for more intricate transformations. However, the increased dimensionality and complexity of 3D rotations can introduce challenges in optimization and parameter efficiency, a common trade-off between expressiveness and computational cost that many KGE models face \cite{sachan2020, wang2021}.

Further pushing the boundaries of algebraic structures, Contextualized Quaternion Embedding (ConQuatE) \cite{chen2025} leverages quaternions, a four-dimensional extension of complex numbers. Quaternions offer even greater expressive power for rotations and can capture more diverse relational contexts, specifically addressing the challenge of polysemy in knowledge graphs. Polysemy, where an entity can have different semantic characteristics depending on the relation it participates in, is a significant limitation for models that assign a single, static vector to each entity. ConQuatE enriches entity representations by incorporating contextual cues from various connected relations through efficient quaternion transformations, without requiring extra information beyond original triples. This represents a forward-looking development, as indicated by its 2025 publication, showcasing the continuous evolution towards more sophisticated algebraic structures to handle nuanced semantic problems.

These rotational and complex space embeddings are part of a broader trend within the "Geometric and Algebraic KGE Models for Complex Relations" subgroup, which continuously seeks to enhance model expressiveness \cite{cao2022, ge2023}. Other notable innovations include HousE \cite{li2022}, which employs Householder parameterization (a type of reflection and rotation) to achieve superior capacity in modeling relation patterns and mapping properties, and CompoundE \cite{ge2022} and CompoundE3D \cite{ge2023}, which generalize by combining translation, rotation, and scaling operations. These models collectively demonstrate a progression towards more intricate, cascaded geometric manipulations. The recent GoldE \cite{li2024} further generalizes orthogonal parameterization, aiming for a unified framework that captures both logical patterns and topological heterogeneity, extending the capabilities of these rotational approaches.

Despite their strengths, these models are not without limitations. While RotatE and its extensions significantly improve performance on complex patterns, they can still struggle with specific theoretical deficiencies. For instance, MQuinE \cite{liu2024} identifies a "Z-paradox" in some popular KGE models, where expressiveness is degraded, and proposes a new model to mitigate this, suggesting that even advanced geometric models may have subtle theoretical gaps. Furthermore, while models like Fully Hyperbolic Rotation \cite{liang2024} aim to fully exploit non-Euclidean spaces for hierarchical structures, they still need to demonstrate consistent superiority across all relation patterns, as some complex operations in hyperbolic space can be more involved. SpherE \cite{li2024} offers another innovative extension by representing entities as spheres within a rotational framework, specifically targeting many-to-many relations and set retrieval, indicating a diversification of geometric primitives to address specific challenges. The experimental setups for these models often focus on link prediction, which, while standard, may not fully capture the nuances of all the complex patterns they aim to model, especially for tasks like set retrieval or multi-hop logical reasoning, highlighting a potential gap in comprehensive evaluation.