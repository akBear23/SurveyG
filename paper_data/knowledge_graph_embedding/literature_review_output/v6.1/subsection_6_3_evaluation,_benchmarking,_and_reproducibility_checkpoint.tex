\subsection{Evaluation, Benchmarking, and Reproducibility}
The rapid proliferation of Knowledge Graph Embedding (KGE) models necessitates a strong emphasis on rigorous evaluation, standardized benchmarking, and reproducibility to ensure scientific integrity and foster reliable progress. Without these pillars, comparing novel approaches fairly, identifying true advancements, and deploying trustworthy models becomes exceedingly challenging. The KGE community has increasingly recognized this, leading to the development of unified frameworks and large-scale comparative studies that expose critical issues in current research practices.

A significant step towards enhancing reproducibility and facilitating comprehensive experimental studies is the introduction of standardized libraries. \cite{broscheit2020} presented \texttt{LibKGE}, an open-source PyTorch-based library designed for training, hyperparameter optimization, and evaluation of KGE models. Its key strengths lie in its configurability and the decoupling of individual components, allowing researchers to mix and match elements of training methods, model architectures, and evaluation techniques. This modularity is crucial for isolating the impact of specific design choices and ensuring that experiments can be fully reproduced using a single configuration file.

Complementing such frameworks, large-scale comparative studies have been instrumental in shedding light on the state of KGE research. \cite{ali2020}, in their seminal work "Bringing Light Into the Dark," undertook a massive re-implementation and evaluation of 21 KGE models within the unified \texttt{PyKEEN} framework. Their findings were stark: a significant number of previously published results could not be reproduced with their reported hyperparameters, often requiring extensive re-tuning or proving entirely irreproducible. This highlights a critical methodological limitation in the field, where inconsistent implementations, varying training protocols, and undisclosed hyperparameter choices hinder fair comparisons and obscure genuine progress. The study further provides invaluable insights into best practices, optimal configurations, and the crucial interplay between model architecture, training approach, loss function, and the explicit modeling of inverse relations. It underscores that performance is not solely dictated by architectural novelty but by a holistic and carefully configured experimental setup.

Further exposing biases in evaluation, \cite{rossi2020} conducted a comprehensive comparison of 18 state-of-the-art KGE methods for link prediction. Their analysis critically examined the effect of various design choices and, more importantly, highlighted biases inherent in standard evaluation practices. They found that many benchmarks suffer from an over-representation of certain entities in test sets, allowing models to achieve seemingly high scores by focusing on these frequently occurring entities while neglecting the vast majority of the knowledge graph. This phenomenon can lead to an inflated sense of model generalizability, as models might perform well on "easy" cases without truly capturing complex relational patterns across the entire graph. Such findings reveal a fundamental flaw in how experimental setups can inadvertently affect the generalizability of reported results, making it difficult to discern which models genuinely extrapolate well to unseen, diverse data.

The impact of hyperparameters on KGE quality is another critical aspect directly affecting reproducibility and model reliability. \cite{lloyd2022} conducted a Sobol sensitivity analysis to quantify the importance of different hyperparameters, revealing substantial variability in their effects across different knowledge graphs. This implies that hyperparameter configurations optimal for one dataset may be suboptimal or even detrimental for another, making universal "best practices" elusive. More alarmingly, their work identified potential data leakage via inverse relations in the widely used UMLS-43 benchmark, leading to the derivation of a leakage-robust variant. Such discoveries are crucial as data leakage can artificially inflate model performance, leading to unrealistic assumptions about a model's true capabilities and undermining the validity of comparisons. This underscores the need for continuous scrutiny of benchmark datasets themselves, not just the models evaluated on them.

Collectively, these studies underscore the urgent need for more robust practices in KGE research. The heterogeneity of implementations, the challenges in reproducing reported results, the biases in standard evaluation metrics, and the profound impact of hyperparameter choices all contribute to a landscape where fair comparisons and reliable scientific progress are often compromised. Moving forward, the field must embrace higher standards of empirical validation and transparency, leveraging unified frameworks like \texttt{LibKGE} and \texttt{PyKEEN}, adopting more rigorous and unbiased evaluation protocols, and thoroughly analyzing the sensitivity of models to hyperparameter variations. This shift is essential not only for advancing the theoretical understanding of KGE but also for ensuring the trustworthy and effective deployment of KGE models in real-world applications.