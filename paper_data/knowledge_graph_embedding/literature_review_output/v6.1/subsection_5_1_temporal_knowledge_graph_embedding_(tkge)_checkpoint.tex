\subsection{Temporal Knowledge Graph Embedding (TKGE)}
Temporal Knowledge Graph Embedding (TKGE) models are specifically designed to capture the dynamic evolution of facts within knowledge graphs, moving beyond static representations to understand how entities and relations change over time. This field is crucial for tasks requiring reasoning over time, predicting future events, and understanding the fluidity of real-world knowledge. The methodological evolution in TKGE has progressed from explicitly incorporating time as a distinct dimension to leveraging sophisticated geometric transformations and multi-curvature spaces, increasingly addressing the complexities of dynamic, spatiotemporal, and even fuzzy knowledge.

Early approaches to TKGE focused on explicitly modeling time within the embedding space. \cite{dasgupta2018} introduced HyTE, a hyperplane-based method that associates each timestamp with a corresponding hyperplane. This allows for temporal guidance during knowledge graph inference and the prediction of temporal scopes for facts with missing time annotations. While intuitive and an important early step, HyTE's hyperplane approach might struggle with highly complex, non-linear temporal dependencies inherent in rapidly evolving knowledge graphs, as it relies on a relatively simple geometric interpretation of time. Extending this, \cite{lin2020} proposed a tensor decomposition-based model that treats the entire fact set as a fourth-order tensor (head, relation, tail, time). This provides a robust mathematical framework for handling temporal data by generalizing static tensor-based KGEs. However, the computational cost associated with higher-order tensor operations can be a significant limitation, especially for very dense temporal datasets. \cite{xu2019} introduced ATiSE, which incorporates time using additive time series decomposition, mapping representations into multi-dimensional Gaussian distributions where the mean denotes the expected position and covariance captures temporal uncertainty. This probabilistic view is valuable for real-world noisy data, but the complexity of time series decomposition can also be computationally demanding. More recently, \cite{li2023} presented TeAST, a novel model that maps relations onto an Archimedean spiral timeline, transforming the quadruple completion problem into a 3rd-order tensor completion task. TeAST explicitly aims for interpretability by ensuring relations evolve orderly along the spiral, a distinct advantage over more abstract temporal representations.

A significant advancement in TKGE involves leveraging geometric transformations to model temporal dynamics. \cite{xu2020} proposed TeRo, which defines the temporal evolution of entity embeddings as rotations in a complex vector space. For facts involving time intervals, TeRo uses dual complex embeddings for the beginning and end of relations, effectively capturing dynamic interactions. Building on this, \cite{sadeghian2021} introduced ChronoR, employing k-dimensional rotation transformations parametrized by both relation and time to transform a head entity near its tail. Both TeRo and ChronoR demonstrate strong performance in temporal link prediction, but the interpretability of complex rotations in high-dimensional spaces can be challenging, and the computational complexity of learning these transformations can be substantial, particularly for large-scale KGs. More recent works have extended these geometric transformations to address additional complexities. \cite{ji2024} (FSTRE) utilizes projection and rotation in a complex vector space to embed spatial and temporal information, introducing fine-grained fuzziness through modal lengths of anisotropic vectors. This represents a crucial step towards handling uncertain and dynamic knowledge, reflecting the inherent messiness of real-world data. Further, \cite{ji2024} (Quaternion Embedding) leverages quaternions to jointly embed spatiotemporal entities, representing relations as rotations and exploiting the non-commutative compositional pattern of quaternions for multihop path reasoning and uncertainty modeling. While powerful, the increased algebraic complexity of quaternions can lead to higher model intricacy and training demands.

The latest frontier in TKGE research, particularly in 2024, has seen the emergence of multi-curvature space embeddings to address the limitations of single-space models in capturing intricate TKG structures. \cite{wang2024} (MADE) proposes modeling TKGs in multi-curvature spaces, including Euclidean, hyperbolic, and hyperspherical geometries. MADE introduces an adaptive weighting mechanism to assign different weights to these spaces in a data-driven manner, strengthening ideal spaces and weakening inappropriate ones, along with a temporal regularization for timestamp smoothness. Similarly, \cite{wang2024} (IME) integrates "space-shared" properties to learn commonalities across spaces and "space-specific" properties to capture characteristic features, also proposing an Adjustable Multi-curvature Pooling (AMP) approach. Both MADE and IME demonstrate state-of-the-art results by acknowledging that TKGs often contain interwoven hierarchical, ring, and chain structures that no single curvature space can optimally capture. A common limitation for these multi-curvature models is the increased complexity of optimizing embeddings across potentially disparate geometric spaces, and the computational overhead associated with managing these different curvatures. The interpretability of embeddings in such hybrid spaces also presents a significant challenge, as the theoretical advantages must translate into practical, understandable insights.

In summary, TKGE research has evolved from explicit, structured temporal modeling to sophisticated geometric transformations and adaptive multi-curvature embeddings. The field consistently grapples with the trade-off between increased model expressiveness (e.g., handling fuzziness, spatiotemporal data, multi-curvature geometries) and the associated computational complexity and interpretability challenges. While models like HyTE provided foundational temporal awareness, newer approaches like MADE and the quaternion-based embeddings push the boundaries by offering more nuanced and robust representations for the complex, dynamic, and often uncertain nature of real-world knowledge graphs. The ongoing challenge lies in developing scalable, efficient, and interpretable models that can seamlessly integrate all these facets to enable comprehensive reasoning over evolving knowledge.