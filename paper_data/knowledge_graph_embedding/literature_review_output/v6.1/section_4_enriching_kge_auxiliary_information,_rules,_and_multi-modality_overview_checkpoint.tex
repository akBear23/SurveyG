\section{Enriching KGE: Auxiliary Information, Rules, and Multi-modality}
\label{sec:enriching_kge:_auxiliary_information,_rules,_and_multi-modality}

While the preceding section demonstrated how advanced deep learning architectures have significantly enhanced Knowledge Graph Embedding (KGE) by capturing intricate structural patterns and complex interactions within knowledge graphs, a purely structural perspective often falls short in real-world scenarios. Even sophisticated Graph Neural Networks and Transformers, while powerful, can be limited by inherent data sparsity, struggle with explicit logical reasoning, or overlook rich semantic context available beyond the graph's topology. This section therefore shifts focus to advanced KGE approaches that move beyond purely structural information, exploring how diverse external knowledge sources and logical constraints can be integrated to create more robust, semantically rich, and interpretable embeddings.

We delve into three primary avenues for enrichment. First, we examine the incorporation of auxiliary information, such as entity types and attributes, which provides crucial semantic guidance to overcome data sparsity and enhance the discriminative power of embeddings \cite{community_0, community_1}. Second, the integration of explicit logical rules and constraints is explored, demonstrating how prior knowledge can improve reasoning capabilities, ensure consistency, and align learned representations with human-understandable patterns \cite{community_3}. Finally, we investigate multi-modal and cross-domain KGE models that leverage complementary information from various modalities like text and images. This multi-modal fusion not only addresses data incompleteness but also deepens semantic understanding, enabling more comprehensive and nuanced knowledge representation \cite{community_0, community_3, a6a735f8e218f772e5b9dac411fa4abea87fdb9c}. By detailing these innovative strategies, this section highlights the critical advancements in making KGE models more robust, interpretable, and capable of handling the complexities of real-world knowledge, ultimately providing a more holistic understanding of entities and relations.