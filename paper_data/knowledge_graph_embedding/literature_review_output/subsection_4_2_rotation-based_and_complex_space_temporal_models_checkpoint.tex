\subsection{Rotation-based and Complex Space Temporal Models}

Building on the success of static rotational models like RotatE \cite{sun2019rotate} and addressing limitations of earlier temporal KGE approaches that often relied on additive time-series or hyperplane projections \cite{li2017hyte, ma2020atise}, a sophisticated class of temporal models emerged. These models leverage rotations in complex or higher-dimensional geometric spaces to capture the intricate dynamics of evolving knowledge graphs, offering enhanced expressiveness for diverse temporal patterns. This paradigm shift provides more robust mechanisms for modeling how entities and relations evolve over time, moving beyond simpler linear transformations to capture more nuanced relational semantics.

A pivotal contribution in this direction is TeRo (Time-aware Knowledge Graph Embedding via Temporal Rotation) by \textcite{xu2020}, which introduced the concept of modeling temporal evolution as element-wise rotations in a complex vector space. TeRo addressed the limitations of earlier temporal KGE models, which often struggled with complex relation patterns like asymmetry and reflexivity, and were not robust in handling diverse time annotations such as intervals. By deriving time-specific entity embeddings through rotations from their initial states, TeRo effectively captures dynamic changes while preserving underlying structural properties. Furthermore, it innovated by employing dual relation embeddings for time intervals, allowing it to adapt to various temporal annotations and providing the first investigation into the effect of time granularity on performance \cite{xu2020}. While TeRo significantly advanced temporal modeling, its reliance on a 2-dimensional complex space per embedding component might limit its capacity to capture extremely high-dimensional or topologically complex temporal patterns.

Building upon the foundation laid by TeRo, \textcite{sadeghian2021} ChronoR (Rotation Based Temporal Knowledge Graph Embedding) generalized the rotation mechanism to $k$-dimensions, aiming to improve temporal link prediction while addressing issues of large parameter counts and the limitations of Euclidean distance in high-dimensional spaces. ChronoR introduced an innovative inner product scoring function, which theoretically encompasses complex-domain models like ComplEx as a special case when $k=2$, thereby providing a unifying theoretical framework \cite{sadeghian2021}. This generalization offers greater flexibility in choosing the embedding dimensionality, potentially allowing for the capture of more intricate temporal dynamics than a fixed complex space. The model further enhanced robustness and generalizability through novel tensor nuclear norm-inspired regularization and a temporal smoothness objective that explicitly encourages similar transformations for chronologically closer timestamps \cite{sadeghian2021}. Compared to TeRo's implicit temporal evolution through rotation, ChronoR's explicit smoothness objective provides a more direct mechanism for modeling gradual changes. However, the increased dimensionality and regularization complexity in ChronoR can lead to higher computational costs and potentially more challenging optimization landscapes.

Further extending the expressiveness of rotation-based temporal modeling, researchers have explored hypercomplex numbers and non-Euclidean geometries. \textcite{zhang2022muu} introduced TimeLine-Traced Knowledge Graph Embedding (TLT-KGE), which leverages complex or, more notably, *quaternion* vectors to embed entities and relations with timestamps. TLT-KGE uniquely models semantic information and temporal information as distinct axes within the quaternion space, allowing for a separation of these concerns while establishing a connection between them. Quaternions, as 4-dimensional hypercomplex numbers, offer a richer algebraic structure for representing rotations in 3D space, providing a more powerful mechanism than complex numbers for encoding complex, multi-faceted temporal dynamics. This approach addresses the challenge of distinguishing representations of the same entity or relation at different timestamps, which is often difficult for models that conflate semantic and temporal information.

Pushing the boundaries of geometric complexity even further, the 5EL model by \textcite{zhang2025ebv} proposes embedding temporal knowledge graphs into *projective geometric space*, specifically leveraging Möbius Group transformations on the Riemann Sphere. This advanced geometric approach is designed to overcome the limitations of single underlying embedding spaces (like Euclidean or standard complex spaces) that struggle to model intricate temporal patterns such as hierarchical and ring structures. By operating in a non-Euclidean space, 5EL can inherently capture these complex topological relationships. Furthermore, 5EL integrates Large Language Models (LLMs) to extract crucial temporal node information, employing a parameter-efficient fine-tuning strategy to align LLMs with specific task requirements. This hybrid approach enhances the model's ability to recognize structural information of key nodes in historical chains and enriches the representation of central entities, demonstrating a trend towards combining sophisticated geometric modeling with advanced neural architectures for temporal KGE.

These rotation-based and complex/higher-dimensional space models represent a significant advancement by providing highly expressive mechanisms for capturing the intricate dynamics of temporal knowledge graphs. Their ability to model diverse relation patterns (asymmetry, reflexivity), handle various time intervals, generalize to higher dimensions, and even integrate more complex factors like hierarchical or cyclic temporal structures, marks a robust evolution in temporal KGE. However, the increased dimensionality and integration of multiple factors, especially with hypercomplex numbers or non-Euclidean geometries, can lead to significantly higher model complexity, posing challenges for interpretability, computational efficiency, and the practical deployment of these models, particularly as the number of dimensions or integrated features grows. The choice of the most suitable geometric space often remains an empirical decision, heavily dependent on the specific characteristics and temporal patterns present in the knowledge graph.