\subsection*{Rotational and Complex Space Models}

The pursuit of richer relational semantics in Knowledge Graph Embedding (KGE) models has led to the exploration of more sophisticated mathematical operations beyond simple linear translations. This subsection examines models that leverage rotations in complex or higher-dimensional spaces, demonstrating enhanced expressiveness through intricate geometric transformations. These approaches move beyond the limitations of purely translational models by inherently capturing a wider array of relational patterns, such as symmetry, antisymmetry, and composition.

An early and influential step in this direction was the introduction of ComplEx \cite{trouillon2016}, which employs complex-valued embeddings for both entities and relations. ComplEx models a triple $(h, r, t)$ by defining a scoring function based on the Hermitian dot product of the complex embeddings of the head entity, relation, and the conjugate of the tail entity. This elegant formulation inherently allows ComplEx to effectively capture both symmetric and antisymmetric relations without imposing explicit constraints, offering a powerful alternative to the more parameter-heavy tensor factorization models and the geometric translational models that struggled with such patterns. However, while adept at symmetry and antisymmetry, ComplEx does not inherently model compositional patterns, limiting its ability to infer multi-step relationships.

Building upon the concept of complex-valued embeddings and addressing the limitations of prior models, including ComplEx, RotatE \cite{sun2018} emerged as a pivotal model. RotatE interprets relations as element-wise rotations in a complex vector space. Specifically, for a valid triple $(h, r, t)$, the model posits that the tail entity embedding $t$ can be approximated by the Hadamard (element-wise) product of the head entity embedding $h$ and the relation embedding $r$, i.e., $t \approx h \odot r$. A crucial constraint in RotatE is that the modulus of each element of the relation embedding $r$ is fixed to one, ensuring that $r$ acts purely as a rotation. This innovative geometric interpretation allows RotatE to elegantly and simultaneously model symmetric, antisymmetric, and compositional patterns. For instance, an inverse relation can be represented by a rotation in the opposite direction, while compositional relations are naturally captured by successive rotations. This unified capability to infer all three fundamental relation patterns marked a significant advancement, as no single prior state-of-the-art model could achieve this concurrently.

The shift from simple linear translations to rotational and complex space models, exemplified by ComplEx and RotatE, showcases a critical evolution in KGE research. ComplEx demonstrated the power of complex-valued embeddings for handling symmetric and antisymmetric relations efficiently. RotatE further generalized this by introducing relational rotations, providing a unified and mathematically elegant framework that inherently captures a broader spectrum of relational semantics, particularly compositional patterns, which ComplEx could not. These models highlight the enhanced expressiveness gained by moving beyond basic vector arithmetic to more intricate geometric transformations, paving the way for more robust and semantically rich knowledge graph representations. Future work continues to explore how these foundational geometric principles can be extended or integrated with other advanced techniques to capture even more nuanced and dynamic relational structures.