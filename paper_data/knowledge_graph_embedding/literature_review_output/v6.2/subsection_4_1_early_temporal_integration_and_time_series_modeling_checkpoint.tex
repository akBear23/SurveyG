\subsection{Early Temporal Integration and Time Series Modeling}
The initial foray into incorporating the crucial temporal dimension into knowledge graph embeddings marked a pivotal shift from static representations to dynamic models capable of reflecting the evolving nature of real-world knowledge. This foundational phase introduced diverse methodologies to explicitly model time, laying the groundwork for more sophisticated temporal reasoning.

One of the earliest and most influential methods was HyTE (\cite{dasgupta2018}), which pioneered a geometric approach to temporal knowledge graph embedding. HyTE associates each timestamp with a distinct hyperplane in the embedding space, allowing for a nuanced representation of temporal validity. This innovative design enables temporally-guided inference and facilitates the prediction of temporal scopes for facts with missing time annotations, a critical capability for incomplete knowledge graphs.

Building upon the need to capture more complex temporal properties and uncertainties, ATiSE (\cite{xu2019}) emerged as a pioneering model. Recognizing the limitations of simpler time representations, ATiSE conceptualizes entity and relation evolution as multi-dimensional additive time series. Crucially, it represents these evolving embeddings as Gaussian distributions, explicitly capturing temporal uncertainty through covariance matrices. This statistical approach provided a significant departure from deterministic models, establishing a novel connection between relational processes and time series analysis to model dynamic knowledge more robustly.

In parallel, tensor decomposition methods offered a generalizable framework for temporal modeling by inherently integrating time as a fourth dimension (\cite{lin2020}). These approaches conceptualize knowledge graph facts as a fourth-order tensor (head, relation, tail, time), allowing for direct incorporation of the temporal dimension into established tensor factorization techniques. This provided a powerful and flexible paradigm for learning embeddings that intrinsically account for time-dependent relationships.

More recently, TeAST (\cite{li2023}) introduced a novel structural mapping for temporal relations, further diversifying the foundational strategies for time integration. TeAST maps relations onto a unique Archimedean spiral timeline, transforming the quadruple completion problem into a 3rd-order tensor completion task. This innovative approach avoids direct entity evolution, ensuring that entity representations remain stable while relations dynamically evolve along the spiral, thereby enhancing interpretability and effectively modeling various relation patterns.

These early efforts collectively established the necessity and feasibility of temporal knowledge graph embeddings. While HyTE provided an intuitive geometric model, ATiSE introduced a sophisticated statistical view of temporal evolution and uncertainty, and tensor decomposition offered a generalizable algebraic framework. TeAST, though more recent, contributes a creative structural mapping for temporal relations. A common challenge across some of these foundational models, particularly the earlier ones, lies in their potential struggle with highly complex, non-linear temporal patterns or the scalability issues associated with explicit temporal indexing or large-scale tensor operations. Nevertheless, they successfully moved beyond static representations, paving the way for subsequent research into more expressive and dynamic temporal models.