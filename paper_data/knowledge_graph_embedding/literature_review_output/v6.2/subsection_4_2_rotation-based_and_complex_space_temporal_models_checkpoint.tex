\subsection{Rotation-based and Complex Space Temporal Models}

Capturing the dynamic evolution of entities and relations over time is a central challenge in Temporal Knowledge Graph Embedding (TKGE). A significant line of research has emerged, leveraging rotations in complex or higher-dimensional spaces to elegantly model these dynamic changes, often inspired by the success of their static counterparts. These models provide more expressive and robust mechanisms for understanding how knowledge evolves.

A pioneering effort in this direction is TeRo \cite{xu2020}, which introduced a novel approach to model temporal evolution as element-wise rotations in a complex vector space. TeRo addresses the limitations of earlier TKGE models, which often struggled with complex relation patterns (e.g., asymmetric, reflexive) and robustly handling diverse time annotations like intervals. By deriving time-specific entity embeddings through rotation from their initial, time-independent states, TeRo effectively captures dynamic changes. Furthermore, for facts with time intervals, TeRo innovatively employs a pair of dual complex relation embeddings, allowing it to adapt to various temporal annotations and providing the first investigation into the effect of time granularity on performance.

Building upon the foundation laid by TeRo, ChronoR \cite{sadeghian2021} generalizes the concept of rotation to k-dimensions, aiming to improve temporal link prediction while addressing issues like large parameter counts and the limitations of Euclidean distance in high-dimensional spaces. ChronoR proposes a novel inner product scoring function, demonstrating a significant theoretical insight by showing it generalizes commonly used scoring functions in complex-domain models like ComplEx \cite{trouillon2016complex} when $k=2$. To enhance model generalizability and ensure smooth entity evolution, ChronoR also introduces advanced regularization techniques, including a tensor nuclear norm-inspired regularization and a temporal smoothness objective based on the 4-norm, encouraging similar transformations for chronologically closer timestamps.

Further extending this paradigm, FSTRE \cite{ji2024} showcases the versatility of rotation-based models by integrating fuzziness and spatial information alongside temporal dynamics. Recognizing that real-world knowledge is often uncertain and dynamic, and that existing KGE models lack spatial modeling capabilities, FSTRE proposes a comprehensive framework within a complex vector space. It uniquely employs projection for spatial information and rotation for temporal information, while incorporating fine-grained fuzziness using the modal length of anisotropic vectors. This model represents a significant advancement by moving beyond crisp, static, and purely temporal KGs, demonstrating how geometric operations in complex spaces can effectively capture multifaceted dynamic knowledge.

These rotation-based and complex space models represent a significant advancement in TKGE by providing highly expressive mechanisms for modeling how entities and relations evolve over time. Their ability to capture diverse relation patterns, handle various time annotations, and even integrate additional dimensions like space and uncertainty, marks a crucial step towards more comprehensive and robust representations of dynamic knowledge. However, as the complexity of these models increases with higher dimensions or the integration of additional factors, computational efficiency and interpretability remain ongoing considerations for future research.