think
The user wants a LaTeX literature review subsection titled "Robustness to Noise, Errors, and Data Imbalance".
The subsection needs to be at least 400 words.
It must focus *only* on the specified topic: robustness to noisy data, erroneous triples, and skewed (long-tail) distributions, and techniques like confidence-aware negative sampling, adaptive weighting, and error-aware embedding.

I need to analyze the provided papers and identify which ones directly address these themes.

**SUBSECTION SPECIFIC FOCUS:**
*   **Problem:** Imperfect real-world knowledge graphs (noisy data, erroneous triples, skewed/long-tail distributions).
*   **Techniques:**
    *   Confidence-aware negative sampling for noisy KGs.
    *   Adaptive weighting mechanisms (e.g., bilevel optimization) for data imbalance.
    *   Error-aware embedding frameworks leveraging entity attributes to detect/down-weight unreliable triples.
*   **Goal:** Ensure reliable and well-performing embeddings despite imperfections.

Let's go through the papers and extract relevant information for this specific focus:

1.  **[xiao2015] ManifoldE:** Focuses on precise link prediction by using manifolds to represent relations, addressing ill-posed algebraic systems and over-strict geometric forms. *Not directly about noise, errors, or data imbalance in the way the prompt defines.*
2.  **[zhang2023] Modality-Aware Negative Sampling for Multi-modal Knowledge Graph Embedding:** Focuses on negative sampling for *multi-modal* KGE, specifically for modality alignment. While it's about negative sampling, its primary problem is multi-modality, not noise/imbalance in the general sense. *Might be tangentially relevant if I can frame negative sampling as a robustness mechanism, but the core problem is different.*
3.  **[wang2020] Knowledge Graph Embedding via Graph Attenuated Attention Networks (GAATs):** Focuses on improving GCN-based KGE by attenuated attention for better feature extraction. *Not directly about noise, errors, or data imbalance.*
4.  **[li2022] TranSHER:** Focuses on relaxing hyper-ellipsoidal restrictions in PairRE for better entity distribution and complex relation modeling. *Not directly about noise, errors, or data imbalance.*
5.  **[xiao2016] SSP: Semantic Space Projection for Knowledge Graph Embedding with Text Descriptions:** Focuses on incorporating textual descriptions to discover semantic relevance and precise expression. *Not directly about noise, errors, or data imbalance.*
6.  **[shan2018] Confidence-Aware Negative Sampling Method for Noisy Knowledge Graph Embedding:** **HIGHLY RELEVANT.** Directly addresses "noisy knowledge graph embedding" and proposes "confidence-aware negative sampling" and "negative triple confidence." This fits perfectly.
7.  **[guo2015] Semantically Smooth Knowledge Graph Embedding (SSE):** Focuses on enforcing semantic smoothness using external categories via manifold learning. *Not directly about noise, errors, or data imbalance.*
8.  **[shen2022] Joint Language Semantic and Structure Embedding for Knowledge Graph Completion (LASS):** Focuses on jointly embedding language semantics and structural information using LMs. *Not directly about noise, errors, or data imbalance.*
9.  **[jia2017] Knowledge Graph Embedding (TransA - Adaptive margin, incremental):** Proposes TransA with an adaptive margin and iTransA for incremental learning. The "adaptive margin" could be seen as a form of robustness to varying data characteristics, but the paper's focus is on general performance improvement and dynamic KGs, not specifically noise/errors/imbalance. *Potentially relevant for adaptive mechanisms, but less direct than others.*
10. **[zhang2018] NSCaching: Simple and Efficient Negative Sampling for Knowledge Graph Embedding:** Focuses on efficient negative sampling to avoid vanishing gradients by caching "hard" negatives. While negative sampling is a training technique, its primary goal here is efficiency and avoiding vanishing gradients, not explicitly robustness to noise or imbalance. *Less direct fit.*
11. **[he2023] A type-augmented knowledge graph embedding framework for knowledge graph completion (TaKE):** Focuses on leveraging *implicit* entity type information for KGE, including a type-constrained negative sampling strategy. While it uses negative sampling, the core problem is leveraging types, not noise/imbalance. *Less direct fit.*
12. **[tabacof2019] Probability Calibration for Knowledge Graph Embedding Models:** **HIGHLY RELEVANT.** Addresses "uncalibrated" predictions, which means unreliable probabilities. This directly relates to the reliability aspect of embeddings in the presence of imperfections. Proposes calibration techniques, including for synthetic negatives when ground truth negatives are absent. This is about making predictions *reliable* despite potential imperfections in the model's confidence.
13. **[choi2020] An Approach to Knowledge Base Completion by a Committee-Based Knowledge Graph Embedding:** Focuses on ensemble methods for KBC. Ensemble methods can improve robustness, but the paper doesn't explicitly frame it in terms of noise, errors, or imbalance. *General robustness, but not specific to the prompt's focus.*
14. **[guo2017] Knowledge Graph Embedding with Iterative Guidance from Soft Rules (RUGE):** Focuses on integrating soft rules iteratively. Soft rules might deal with uncertainty, but the paper's primary focus is rule integration, not robustness to data imperfections like noise or imbalance. *Less direct fit.*
15. **[madushanka2024] Negative Sampling in Knowledge Graph Representation Learning: A Review:** A survey on negative sampling. Could be used to contextualize other negative sampling papers, but not a primary method itself.
16. **[zhou2023] Marie and BERT—A Knowledge Graph Embedding Based Question Answering System for Chemistry:** Application-focused. *Not relevant.*
17. **[yoon2016] A Translation-Based Knowledge Graph Embedding Preserving Logical Property of Relations (lppTransE/R/D):** Focuses on preserving logical properties like transitivity. *Not relevant.*
18. **[tang2022] RulE: Knowledge Graph Reasoning with Rule Embedding:** Focuses on learning rule embeddings for reasoning. *Not directly about noise, errors, or data imbalance.*
19. **[xiao2015] TransG : A Generative Model for Knowledge Graph Embedding:** Focuses on multiple relation semantics using a generative model. *Not directly about noise, errors, or data imbalance.*
20. **[zhang2021] Towards Robust Knowledge Graph Embedding via Multi-Task Reinforcement Learning:** **HIGHLY RELEVANT.** Directly addresses "noisy and conflicts" in KGs and proposes RL agents to "filter out noisy ones" and MTL for similar relations. This fits perfectly.
21. **[yuan2019] TransGate: Knowledge Graph Embedding with Shared Gate Structure:** Focuses on parameter efficiency and expressivity using shared gates. *Not directly about noise, errors, or data imbalance.*
22. **[wang2021] TransET: Knowledge Graph Embedding with Entity Types:** Focuses on incorporating entity types via circle convolution. *Not directly about noise, errors, or data imbalance.*
23. **[zhang2018] Knowledge Graph Embedding with Hierarchical Relation Structure:** Focuses on leveraging hierarchical relation structure. *Not directly about noise, errors, or data imbalance.*
24. **[xiao2015] TransA: An Adaptive Approach for Knowledge Graph Embedding (Adaptive Mahalanobis distance):** Focuses on an adaptive Mahalanobis distance to model complex relation topologies and suppress noise from unrelated dimensions. This is about *noise suppression* and *adaptive metrics* for diverse relations. **HIGHLY RELEVANT.**
25. **[huang2019] Knowledge Graph Embedding Based Question Answering:** Application-focused. *Not relevant.*
26. **[dai2020] A Survey on Knowledge Graph Embedding: Approaches, Applications and Benchmarks:** A general survey. Could be used for context.
27. **[li2021] Efficient Non-Sampling Knowledge Graph Embedding (NS-KGE):** Focuses on non-sampling for stability and accuracy, avoiding negative sampling. While it's about training stability, its primary problem is the *uncertainty of sampling*, not explicitly noise/errors/imbalance in the data itself. *Less direct fit.*
28. **[zhu2022] Multimodal reasoning based on knowledge graph embedding for specific diseases:** Application-focused. *Not relevant.*
29. **[lv2018] Differentiating Concepts and Instances for Knowledge Graph Embedding (TransC):** Focuses on differentiating concepts and instances (spheres vs. vectors). *Not relevant.*
30. **[guo2020] Knowledge Graph Embedding Preserving Soft Logical Regularity:** Focuses on integrating soft logical rules. Similar to [guo2017], deals with uncertainty but not directly noise/errors/imbalance. *Less direct fit.*
31. **[li2022] Embedding knowledge graph of patent metadata to measure knowledge proximity:** Application-focused. *Not relevant.*
32. **[rossi2020] Knowledge Graph Embedding for Link Prediction:** A general survey. Could be used for context.
33. **[sun2018] RotatE: Knowledge Graph Embedding by Relational Rotation in Complex Space:** Focuses on modeling relation patterns (symmetry, inversion, composition). *Not directly about noise, errors, or data imbalance.*
34. **[zheng2024] Knowledge graph embedding closed under composition (HolmE):** Focuses on theoretical closure under composition. *Not relevant.*
35. **[shi2025] TGformer: A Graph Transformer Framework for Knowledge Graph Embedding:** Focuses on graph transformers for multi-structural features. *Not relevant.*
36. **[liu2024] Fast and Continual Knowledge Graph Embedding via Incremental LoRA:** Focuses on continual learning and efficiency. *Not relevant.*
37. **[zhang2024] Integrating Entity Attributes for Error-Aware Knowledge Graph Embedding:** **HIGHLY RELEVANT.** Directly addresses "erroneous triples" and proposes an "error-aware embedding framework" using "entity attributes" to "down-weight unreliable triples." This fits perfectly.

**Selected Papers for the Subsection:**

*   **Noise/Errors:**
    *   [shan2018] Confidence-Aware Negative Sampling Method for Noisy Knowledge Graph Embedding
    *   [zhang2021] Towards Robust Knowledge Graph Embedding via Multi-Task Reinforcement Learning
    *   [zhang2024] Integrating Entity Attributes for Error-Aware Knowledge Graph Embedding
    *   [tabacof2019] Probability Calibration for Knowledge Graph Embedding Models (reliability of predictions)
*   **Data Imbalance:**
    *   [zhang2023] Weighted Knowledge Graph Embedding (This paper was in my original plan for the overall review, but not in the provided list. I need to check if it's in the list of papers I'm allowed to cite. *Self-correction: It's not in the provided list of papers to reference for this specific prompt, so I cannot use it.* This is a critical point. I must only use the provided list.)
    *   *Re-evaluating for data imbalance from the provided list:*
        *   [xiao2015] TransA: An Adaptive Approach for Knowledge Graph Embedding (Adaptive Mahalanobis distance) - "incompetent to model diverse and complex relation topologies (e.g., one-to-many, many-to-one, many-to-many relations)" and "identical treatment of dimensions." This can be framed as robustness to diverse relation types, which often correlates with imbalance (complex relations might be rarer or have more varied instances). The adaptive metric helps.

**Revised Selection:**

1.  **[shan2018] Confidence-Aware Negative Sampling Method for Noisy Knowledge Graph Embedding:** Direct hit on noisy KGs and confidence-aware negative sampling.
2.  **[zhang2021] Towards Robust Knowledge Graph Embedding via Multi-Task Reinforcement Learning:** Direct hit on noisy KGs and filtering noisy triples.
3.  **[zhang2024] Integrating Entity Attributes for Error-Aware Knowledge Graph Embedding:** Direct hit on erroneous triples and error-aware embedding using attributes.
4.  **[tabacof2019] Probability Calibration for Knowledge Graph Embedding Models:** Direct hit on reliability of predictions despite imperfections.
5.  **[xiao2015] TransA: An Adaptive Approach for Knowledge Graph Embedding:** Addresses "diverse and complex relation topologies" and "identical treatment of dimensions" by an adaptive metric. This can be interpreted as robustness to varying relational characteristics, which often includes handling diverse (and potentially imbalanced) relation types more effectively than a fixed metric.

**Connecting the papers (Progression Chain):**

*   **Initial Problem:** KGE models assume perfect data, but real-world KGs are noisy and complex.
*   **Early attempt at noise:** `\cite{shan2018}` directly tackles noisy KGs by introducing *negative triple confidence* and a *confidence-aware negative sampling* method to improve the training of confidence-aware KGE models like CKRL. This focuses on improving the sampling process to handle noise.
*   **Broader noise handling & adaptive metrics:** Building on the need for models to adapt to complex data, `\cite{xiao2015}` (TransA) addresses the limitations of oversimplified metrics (like Euclidean distance) that treat all dimensions equally and struggle with diverse relation topologies (e.g., one-to-many, many-to-many). TransA introduces an *adaptive Mahalanobis distance* with a relation-specific weight matrix to better model complex relations and suppress noise from irrelevant dimensions, enhancing robustness to varied data characteristics.
*   **Systematic noise filtering:** Moving beyond just sampling or adaptive metrics, `\cite{zhang2021}` proposes a more proactive approach to noise. Their *multi-task reinforcement learning (MTRL)* framework explicitly trains RL agents to *filter out noisy triples* during training, providing a "hard decision" for triple selection. This is a significant step towards actively cleaning the training data for robustness.
*   **Error-aware embedding with external information:** While `\cite{zhang2021}` focuses on internal graph structure for noise filtering, `\cite{zhang2024}` introduces an *error-aware embedding framework (AEKE)* that leverages *external entity attributes* and hypergraphs to calculate joint confidence scores for triples. This allows for adaptively *down-weighting unreliable triples*, providing a more nuanced handling of errors by incorporating rich semantic context.
*   **Reliability of predictions:** Even with robust training and error handling, the *reliability of model predictions* is crucial. `\cite{tabacof2019}` addresses this by demonstrating that KGE models are often *uncalibrated*. They propose *probability calibration techniques*, including a novel heuristic for handling synthetic negatives, to ensure that the predicted probabilities of triples are reliable and trustworthy, which is essential for high-stakes applications.

**Narrative Flow:**
Start with the general problem of imperfect KGs.
1.  Introduce `\cite{shan2018}` as an early, direct approach to handling noisy KGs via confidence-aware negative sampling.
2.  Transition to `\cite{xiao2015}` (TransA) which, while not explicitly about "noise" in the data quality sense, introduces an *adaptive metric* to handle "diverse and complex relation topologies" and "suppress noise from unrelated dimensions," making embeddings more robust to varied data characteristics, which can include aspects of imbalance or complexity. This shows an adaptive approach to data characteristics.
3.  Then, discuss `\cite{zhang2021}` (MTRL), which takes a more active stance by using RL to *filter out noisy triples* directly, thereby improving robustness by cleaning the training data itself. This is a more systematic approach to noise.
4.  Follow with `\cite{zhang2024}` (AEKE), which extends noise handling by leveraging *entity attributes* for *error-aware embedding* and *down-weighting unreliable triples*. This introduces external semantic information for finer-grained error detection.
5.  Conclude with `\cite{tabacof2019}` on *probability calibration*, emphasizing that even with robust training, the *reliability of predicted confidences* is paramount for real-world applications, ensuring trustworthiness.

This flow covers noise, errors, and adaptive mechanisms, and touches upon data imbalance implicitly through "diverse relation topologies" and "complex relations" where adaptive metrics are beneficial. The word count should be achievable.

**Constraint Checklist & Confidence Score:**
1. LaTeX format: Yes
2. Proper citations: Yes
3. At least 400 words: Will aim for it.
4. No numbering in subsection titles: Yes
5. Focus ONLY on SUBSECTION SPECIFIC FOCUS: Yes
6. Specific examples with citations: Yes
7. Opening 1-2 sentences: Yes
8. Each paper 2-3 sentences with technical contributions: Yes
9. Papers connected by addressing limitations: Yes
10. Narrative follows progression: Yes
11. Closing: Yes

Confidence Score: 5/5

Strategizing complete. I will now proceed with generating the response following these steps.
\subsection{Robustness to Noise, Errors, and Data Imbalance}
Real-world knowledge graphs (KGs) are inherently imperfect, often plagued by noisy data, erroneous triples, and skewed distributions, which can severely degrade the quality and reliability of learned knowledge graph embeddings (KGEs). Addressing these imperfections is crucial to ensure that KGE models produce reliable representations that perform well in practical applications.

Early efforts to bolster KGE robustness directly confronted the issue of noisy data during training. \cite{shan2018} introduced a confidence-aware negative sampling method specifically designed for noisy KGE. This approach leverages the concept of *negative triple confidence* to enhance the training of confidence-aware KGE models, mitigating issues like zero loss and false detection that arise from assuming uniform confidence in noisy KGs. Complementing this, \cite{xiao2015} (TransA) tackled the limitations of oversimplified Euclidean distance metrics in translation-based models, which struggle with diverse and complex relation topologies (e.g., one-to-many, many-to-many) and treat all embedding dimensions equally. TransA proposed an *adaptive Mahalanobis distance* with a relation-specific weight matrix to flexibly model these complex relations and suppress noise from irrelevant dimensions, thereby improving robustness to varied data characteristics.

Building on the need for more systematic noise handling, \cite{zhang2021} presented a multi-task reinforcement learning (MTRL) framework for robust KGE. This framework employs policy-based reinforcement learning agents to make *hard decisions* on selecting high-quality triples while filtering out noisy ones during training. By also incorporating multi-task learning for semantically similar relations, this method provides a proactive and adaptive mechanism to clean the training data, significantly enhancing the robustness of various KGE models against noise and conflicts. Further advancing error-aware learning, \cite{zhang2024} introduced an error-aware embedding framework (AEKE) that integrates *entity attributes* and hypergraphs. This framework calculates joint confidence scores for triples, allowing the model to adaptively *down-weight unreliable triples* and thus mitigate the impact of erroneous data by leveraging rich semantic context beyond the structural triples.

Finally, even with advanced noise filtering and error-aware embedding, the ultimate utility of KGE models hinges on the reliability of their predictions. \cite{tabacof2019} highlighted a critical oversight in KGE research: models are often *uncalibrated*, meaning their predicted probabilities are unreliable. To address this, they proposed novel *probability calibration techniques*, including a heuristic for scenarios lacking ground truth negatives. By ensuring that predicted confidence scores accurately reflect the likelihood of a triple being true, this work significantly enhances the trustworthiness of KGE models, which is paramount for high-stakes applications where reliable decision-making is essential.

Despite these advancements, challenges remain in developing KGE models that can seamlessly adapt to dynamic changes in data quality and distribution over time, particularly in continually evolving KGs. Future research could explore more sophisticated adaptive weighting mechanisms that dynamically adjust to both data imbalance and evolving noise patterns, potentially integrating meta-learning approaches to learn optimal robustness strategies.