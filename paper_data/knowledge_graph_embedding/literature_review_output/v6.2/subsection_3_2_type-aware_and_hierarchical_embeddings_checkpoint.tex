\subsection{Type-Aware and Hierarchical Embeddings}

Knowledge graph embedding (KGE) models often gain significant expressive power and logical consistency by explicitly incorporating semantic type information or hierarchical structures inherent within knowledge graphs. These approaches move beyond simple triplet-level interactions to ensure that learned representations respect the inherent semantic categories and structural relationships between entities, thereby enhancing accuracy, particularly for KGs with rich ontological schemas.

Early efforts to infuse semantic information into KGE models include \cite{guo2015}, which proposed Semantically Smooth Embedding (SSE). SSE leverages external semantic categories (e.g., entity types) to enforce a "semantically smooth" embedding space, where entities of the same category are constrained to be close, improving embeddings by incorporating intrinsic geometric structure. Building on the idea of differentiated representations, \cite{lv2018} introduced TransC, which explicitly distinguishes between concepts and instances. TransC models concepts as spheres and instances as vectors, naturally capturing `isA` (instanceOf, subClassOf) transitivity by requiring instance vectors to lie within concept spheres, and sub-concept spheres within super-concept spheres. This geometric approach provides a more nuanced representation of hierarchical relationships than uniform vector embeddings.

Further extending the modeling of hierarchical structures, \cite{zhang2018} proposed a method that leverages a three-layer hierarchical relation structure (HRS) to learn knowledge representations. This HRS categorizes relations into clusters, individual relations, and fine-grained sub-relations, enriching the embedding process by integrating this structural information into existing KGE models like TransE and DistMult. Addressing the direct integration of entity types, \cite{wang2021} developed TransET, which employs circle convolution based on entity and type embeddings to generate type-specific representations for entities. This allows for more semantic features to be learned, moving beyond simple triple facts and better representing complex relations.

A more flexible approach to type integration was presented by \cite{he2023} with the Type-augmented Knowledge graph Embedding (TaKE) framework. TaKE is model-agnostic and learns implicit type features without requiring explicit type supervision, a common limitation in real-world KGs. It innovatively uses a relation-specific hyperplane mechanism to project an entity's type representation onto different hyperplanes, capturing the diverse type features relevant to each relation. In a similar vein, \cite{zhu2022} proposed a multimodal reasoning approach for disease-specific knowledge graphs that integrates category (type) embeddings via a novel reverse-hyperplane projection method. This demonstrates how type-aware projections can be specialized for domain-specific knowledge discovery, aligning with the principles of relation-specific type feature emphasis.

Beyond explicit type and hierarchical structures, logical rules often implicitly encode such semantic constraints. \cite{guo2017} introduced RUGE, a novel paradigm for KG embedding with iterative guidance from soft rules. RUGE enables an embedding model to learn simultaneously from labeled triples, unlabeled triples with predicted soft labels, and automatically extracted soft rules with varying confidence levels, allowing rules to iteratively refine embeddings and predictions. Expanding on rule integration, \cite{tang2022} proposed RulE, which learns explicit rule embeddings by jointly representing entities, relations, and first-order logical rules in a unified continuous space. RulE calculates rule confidence scores and uses a soft rule reasoning mechanism to address the brittleness of traditional logical inference, providing a balanced approach to reasoning. For scalable integration of soft rules, \cite{guo2020} developed Soft Logical Rule Embedding (SLRE), which directly regularizes relation representations to satisfy soft logical regularities, making the complexity of rule learning independent of the entity set size.

Finally, the inherent hierarchical nature of many KGs has led to the exploration of non-Euclidean geometries. \cite{pan2021} introduced a hyperbolic hierarchy-aware KGE model that leverages an extended Poincaré Ball and a polar coordinate system. This approach is designed to simultaneously capture both the complex hierarchical relationships and logical patterns within knowledge graphs, offering state-of-the-art results on specific link prediction tasks by exploiting the natural fit of hyperbolic space for hierarchical data.

Despite these advancements, challenges remain in seamlessly integrating diverse forms of type and hierarchical information. While models like TaKE and TransET effectively incorporate entity types, and TransC differentiates concepts, a universally adaptive framework that can dynamically infer and leverage optimal hierarchical structures (e.g., from flat KGs) for *any* given entity and relation context is still an open area. Furthermore, balancing the expressiveness gained from complex type-aware and hierarchical models with computational efficiency, especially for extremely large and dynamic knowledge graphs, remains a critical unresolved issue. Future research could focus on more adaptive and automated methods for discovering and integrating latent hierarchical and type structures, potentially through meta-learning or self-supervised approaches, to further enhance the logical consistency and accuracy of KGE models.