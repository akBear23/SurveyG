<think>
The user wants a LaTeX literature review subsection titled "Graph Neural Networks and Transformers for KGE".
The subsection needs to be at least 400 words and strictly adhere to the "SUBSECTION SPECIFIC FOCUS".
I need to use the provided community summaries, development directions, and paper summaries to construct the narrative.

**SUBSECTION SPECIFIC FOCUS:**
- Application of GNNs and Transformers to KGE.
- How GNNs capture multi-hop structural context and neighborhood information via message passing/aggregation.
- Adaptation of Transformer architectures using self-attention for graph structures.
- These models move beyond predefined scoring functions to learn complex, non-linear features, boosting representational power and addressing limitations of earlier geometric models.

**Constraint Checklist & Confidence Score:**
1. LaTeX format: Yes
2. Proper citations: Yes
3. At least 400 words: Yes (will aim for this)
4. No numbering in subsection title: Yes
5. Focus ONLY on SUBSECTION SPECIFIC FOCUS: Yes
6. Specific examples with citations: Yes
7. Structured (Opening, Body, Closing): Yes
8. Each paper 2-3 sentences with specific technical contributions: Yes
9. Papers connected by addressing limitations: Yes
10. Overall narrative follows progression: Yes
11. Closing about unresolved issues/future directions: Yes
12. Avoid sequential listing: Yes
13. Avoid vague critique: Yes
14. Avoid unsupported claims: Yes
15. Avoid missing comparisons: Yes
16. Avoid pure description: Yes
17. Avoid ignoring contradictions: N/A (no explicit contradictions in provided summaries for this subgroup)
18. Avoid cherry-picking: N/A (using all relevant papers)

Confidence Score: 5/5

**Mental Sandbox Simulation:**

*   **Initial thought:** Start with GNNs, then transition to Transformers. Connect them by showing how they both move beyond simpler models and offer more expressive, contextual learning.
*   **Community Summaries Analysis (Subgroup 2: Contextual & Structural Learning with Neural Architectures):**
    *   This subgroup directly aligns with the focus.
    *   Key themes: richer, multi-hop contextual/structural info, inductive learning, beyond triplet-level interactions, nuanced representations.
    *   Papers: [wang2018] LAN (inductive, attention, neighborhood aggregation), [xie2020] ReInceptionE (CNN, local-global structural info), [zhang2020] M-DCN (CNN, multi-scale dynamic filters), [li2021] SE-GNN (semantic evidence, GNN), [chen2021] MorsE (meta-knowledge transfer, inductive, GNN), [shi2025] TGformer (Graph Transformer, multi-structural, contextual), [yang2025] SEConv (self-attention, CNN, healthcare), [shang2024] MGTCA (mixed geometry message, trainable convolutional attention network).
    *   This confirms the GNN and Transformer focus.

*   **Development Directions Analysis:**
    *   The development direction specifically traces [yan2022] -> [li2023] -> [shi2025].
    *   [li2023] Position-Aware Relational Transformer for Knowledge Graph Embedding (2023) is a crucial Transformer paper. It addresses the order-invariance problem in vanilla Transformers for KGs. *This paper is NOT in the provided "Papers to reference" list for the subsection, but it's central to the development direction for Transformers. I need to be careful here. The prompt says "Papers to reference (sorted chronologically)" and then provides summaries. If [li2023] is not in the list of papers to reference, I cannot cite it directly. The prompt also says "Each paper is described in 2-3 sentences with specific technical contributions, using citation in Available Citations." This means I can only use the provided citations.*
    *   **Revised plan:** I need to focus on the papers *provided in the "Papers to reference" list*. The development direction section is *general* and might include papers not explicitly provided for *this specific subsection*. I will use the *themes* from the development direction (e.g., "specialized adaptation of Transformer architectures for relational semantics") but apply them to the *available* Transformer-related papers.

*   **Mapping Papers to Focus & Connecting them:**

    *   **Opening:** Introduce the shift from simpler geometric models to advanced neural architectures (GNNs, Transformers) for capturing complex, non-linear features and multi-hop context in KGE. Mention the limitations of earlier models (e.g., geometric models like TransE, TransD, TransH, RotatE, CompoundE, HousE, SpherE, TorusE, TransMS, HolmE, FHRE, which are mostly from Subgroup 1 and some from Subgroup 3, but the prompt specifically asks to address "limitations of earlier geometric models").

    *   **GNNs for Multi-hop Context:**
        *   Start with GNNs' ability to aggregate neighborhood information.
        *   \cite{wang2018} LAN: Early inductive KGE using logic attention for neighborhood aggregation, addressing permutation invariance and query awareness. *This is a good starting point for GNNs.*
        *   \cite{li2021} SE-GNN: Further explores how KGE extrapolates by identifying "Semantic Evidence" and proposes SE-GNN to explicitly model these factors through GNN mechanisms. This builds on the idea of structured aggregation for better generalization.
        *   \cite{shang2024} MGTCA: Advances GNNs by addressing data dependence (GCN vs. GAT) and message limitations. It introduces a *Mixed Geometry Message Function* (hyperbolic, hypersphere, Euclidean) and a *Trainable Convolutional Attention Network* (TCAN) for adaptive GNN switching, showing how GNNs evolve to capture richer, multi-geometric structural information. This directly addresses the "message passing and aggregation mechanisms" part of the focus.
        *   \cite{modak2024} CPa-WAC: While primarily about scalability, its "Weighted Aggregation Composition (WAC) Convolution" is an improved compositional message-passing GCN. It also uses graph attention layers. I can frame this as a GNN approach that also considers scalability for large KGs.

    *   **CNNs for Feature Learning (as a bridge/related neural architecture):**
        *   \cite{zhang2020} M-DCN: Uses multi-scale dynamic convolutional networks to learn expressive features and handle complex relations, showing how CNNs contribute to learning non-linear features.
        *   \cite{xie2020} ReInceptionE: Combines CNNs (Inception network) for interaction learning with a relation-aware attention mechanism to incorporate local-global structural information. This bridges CNNs with attention and structural awareness.
        *   \cite{yang2025} SEConv: Applies self-attention and multi-layer CNNs for deeper structural features, specifically in healthcare. This shows the combination of attention and CNNs.

    *   **Transformers for Graph Structures:**
        *   The core idea of Transformers is self-attention.
        *   \cite{yang2023} CKGE: Introduces a KG-based Transformer with relational attention and structural encoding to model global dependencies for explainable recommendations. This shows adaptation of Transformers to KG structures.
        *   \cite{shi2025} TGformer: A direct application of a Graph Transformer framework for KGE. It constructs context-level subgraphs and uses a Knowledge Graph Transformer Network (KGTN) to explore multi-structural features (triplet-level and graph-level) across static and temporal KGs. This is a strong example of a Transformer adapted for KGE, capturing contextual and multi-structural information.

    *   **Connecting to "Limitations of earlier geometric models":**
        *   The neural models (GNNs, CNNs, Transformers) inherently move beyond the fixed, predefined scoring functions and geometric operations of models like TransE \cite{ji2015}, TransH \cite{wang2014}, TransMS \cite{yang2019}, TorusE \cite{ebisu2017}, CompoundE \cite{ge2022}, HousE \cite{li2022}, SpherE \cite{li2024}, FHRE \cite{liang2024}, and HolmE \cite{zheng2024}. These earlier models, while mathematically elegant, often struggled with complex, non-linear interactions, multi-hop reasoning, and inductive settings. The neural models learn these features directly from data, offering greater expressiveness and often inductive capabilities. For example, TransE and its variants are limited by their assumption of simple translational relationships, which struggle with 1-N, N-1, N-N relations and multi-hop reasoning. RotatE and its complex-space counterparts improve on this but still rely on fixed rotational operations. Hyperbolic models like FHRE and Pan et al.'s work are good for hierarchies but still operate within a geometric paradigm. The neural models offer a more flexible, data-driven approach to feature learning.

    *   **Closing:** Summarize the benefits (expressiveness, inductive capabilities, contextual understanding) and potential future directions (e.g., combining strengths, interpretability of complex models, scalability for very large KGs, dynamic KGs).

*   **Self-correction on [li2023]:** Since [li2023] is not in the "Papers to reference" list, I cannot cite it. However, the *problem it solves* (order-invariance in Transformers for KGs) is a general challenge. I can discuss how Transformer-based models *address* such challenges, and then use [shi2025] and [yang2023] as examples of *how* Transformers are adapted to KGs. [shi2025] explicitly mentions "multi-structural features" and "context-level subgraph construction," which implies handling relational semantics beyond simple order-invariance. [yang2023] also talks about "relational attention and structural encoding" for global dependencies. This allows me to discuss the *concept* without citing the unavailable paper.

*   **Word Count Check:** With 2-3 sentences per paper and connections, plus opening/closing, I should easily hit 400 words.

**Final Structure Outline:**

1.  **Introduction:** Shift to GNNs/Transformers for KGE, addressing limitations of earlier geometric models (fixed scoring, limited context).
2.  **Graph Neural Networks for Contextual Learning:**
    *   Explain GNNs (message passing, aggregation).
    *   \cite{wang2018} LAN: Inductive KGE, logic attention, neighborhood aggregation.
    *   \cite{li2021} SE-GNN: Semantic Evidence, GNNs for extrapolation.
    *   \cite{modak2024} CPa-WAC: Scalable GNN with weighted aggregation composition.
    *   \cite{shang2024} MGTCA: Mixed geometry messages, trainable convolutional attention for adaptive GNNs.
3.  **Convolutional Neural Networks (as related neural architectures):**
    *   \cite{zhang2020} M-DCN: Multi-scale dynamic CNNs for expressive features.
    *   \cite{xie2020} ReInceptionE: Inception network + attention for local-global structural info.
    *   \cite{yang2025} SEConv: Self-attention + CNNs for deeper features (healthcare).
4.  **Transformers for Graph Structures and Relational Semantics:**
    *   Explain Transformers (self-attention) adapted for graphs.
    *   \cite{yang2023} CKGE: KG-based Transformer, relational attention, structural encoding for global dependencies.
    *   \cite{shi2025} TGformer: Graph Transformer framework, context-level subgraphs, multi-structural features (static/temporal KGs).
5.  **Addressing Limitations of Earlier Models:** Explicitly state how these neural models overcome the limitations of geometric models (e.g., fixed operations, limited expressiveness, transductivity).
6.  **Conclusion:** Summarize advantages (expressiveness, inductive, contextual) and future directions (scalability, interpretability, dynamic KGs).

This plan seems robust and covers all requirements.
\subsection{Graph Neural Networks and Transformers for KGE}
The evolution of Knowledge Graph Embedding (KGE) has seen a significant shift from models relying on predefined geometric transformations to advanced neural architectures capable of learning complex, non-linear features directly from the graph structure. This subsection focuses on the application of Graph Neural Networks (GNNs) and Transformers, which move beyond the limitations of earlier geometric models by effectively capturing multi-hop structural context, neighborhood information, and global dependencies, leading to more expressive and often inductive embeddings.

Graph Neural Networks (GNNs) leverage message passing and aggregation mechanisms to capture the rich, multi-hop structural context inherent in knowledge graphs. These models allow entities to learn representations by iteratively aggregating information from their neighbors, thereby incorporating local and global graph structure. An early example of this paradigm is the Logic Attention Network (LAN) proposed by \cite{wang2018}, which pioneered inductive KGE by using a logic attention mechanism for neighborhood aggregation. This approach specifically addresses permutation invariance and query awareness, enabling the model to effectively embed emerging entities based on their local connections. Building on the understanding of how KGE models generalize, \cite{li2021} introduced Semantic Evidence (SE) and proposed SE-GNN, a GNN-based model that explicitly captures relation-level, entity-level, and triple-level semantic evidence through distinct neighbor aggregation patterns, leading to superior extrapolation abilities. For large-scale KGs, \cite{modak2024} developed CPa-WAC, a scalable GNN-based KGE model that incorporates a Weighted Aggregation Composition (WAC) convolution, an improved compositional message-passing GCN that uses graph attention layers to efficiently learn embeddings from partitioned subgraphs. Further advancing GNN capabilities, \cite{shang2024} introduced MGTCA, a model that addresses the data dependence of GNNs and the limitations of Euclidean-only message functions. MGTCA integrates a Mixed Geometry Message Function (combining hyperbolic, hypersphere, and Euclidean spaces) with a Trainable Convolutional Attention Network (TCAN) that adaptively switches between GNN types (GCN, GAT, and a novel KGCAT), allowing for richer, multi-geometric structural information capture and robust embedding learning.

Complementary to GNNs, Convolutional Neural Networks (CNNs) have also been adapted to learn intricate features from concatenated entity and relation embeddings. \cite{zhang2020} proposed the Multi-Scale Dynamic Convolutional Network (M-DCN), which employs multi-scale filters with dynamic, relation-specific weights to learn expressive features and effectively handle complex relation types (1-to-N, N-to-1, N-to-N). This moves beyond fixed interaction patterns by allowing filters to adapt to specific relations. Similarly, \cite{xie2020} introduced ReInceptionE, which combines an Inception network for enhanced interaction learning with a relation-aware attention mechanism. This dual approach enriches query embeddings by incorporating both local neighborhood and global entity information, thereby capturing comprehensive structural context. In a specialized application, \cite{yang2025} developed SEConv for healthcare prediction, integrating a resource-efficient self-attention mechanism with a multi-layer CNN to learn deeper and more informative structural features from medical knowledge graph triplets.

The advent of Transformer architectures, with their powerful self-attention mechanisms, has further pushed the boundaries of KGE by enabling the modeling of global dependencies and complex relational semantics. These models are particularly adept at processing sequential data, and their adaptation to graph structures allows for a flexible, data-driven approach to feature learning. For instance, \cite{yang2023} presented CKGE, a Contextualized Knowledge Graph Embedding model for explainable recommendations. This model constructs meta-graphs for talent-course pairs and employs a specialized KG-based Transformer that leverages relational attention and structural encoding to capture global dependencies and motivation-aware context, providing explicit explanations through local path mask prediction. A more direct application of the Transformer paradigm to KGE is the TGformer framework by \cite{shi2025}. TGformer is a novel Graph Transformer that constructs context-level subgraphs for predicted triplets, explicitly modeling inter-triplet relationships. It utilizes a Knowledge Graph Transformer Network (KGTN) to comprehensively explore multi-structural features, encompassing both fine-grained triplet-level and broader graph-level information, and extends its applicability to both static and temporal knowledge graphs.

These advanced neural architectures significantly boost representational power by moving beyond the predefined scoring functions and rigid geometric operations of earlier models like TransE \cite{ji2015}, TransH \cite{wang2014}, TransMS \cite{yang2019}, TorusE \cite{ebisu2017}, CompoundE \cite{ge2022}, HousE \cite{li2022}, SpherE \cite{li2024}, FHRE \cite{liang2024}, and HolmE \cite{zheng2024}. While geometric models offer interpretability and efficiency for certain relation types, they often struggle with highly complex, non-linear interactions, multi-hop reasoning, and inductive settings. For example, translation-based models are inherently limited in handling 1-N, N-1, or N-N relations, and even more advanced rotation-based or hyperbolic models, while excelling at symmetry or hierarchy, still operate within fixed mathematical frameworks. GNNs and Transformers, by learning features directly from data and dynamically aggregating information, offer greater flexibility, expressiveness, and the ability to generalize to unseen entities and complex relational patterns, thereby addressing critical limitations of their predecessors.

In conclusion, the integration of Graph Neural Networks and Transformers into KGE represents a significant advancement, enabling models to learn complex, non-linear features and capture rich contextual and multi-structural information. While these models offer superior expressiveness and inductive capabilities, challenges remain in optimizing their computational efficiency for extremely large-scale KGs, enhancing their interpretability, and further exploring their application to dynamic and heterogeneous knowledge graphs. Future research will likely focus on hybrid architectures that combine the strengths of different neural components, develop more efficient attention mechanisms, and design models that can adaptively leverage diverse structural and semantic cues for robust knowledge representation.