\subsection{Background: Knowledge Graphs}

Knowledge Graphs (KGs) represent a cornerstone paradigm for organizing and representing factual information about the world in a structured, machine-readable format. Fundamentally, a KG is a network composed of entities (nodes) and relations (edges), typically instantiated as a collection of (subject, predicate, object) triples. This explicit, structured representation allows for the precise encoding of semantic relationships, making complex information readily accessible and interpretable for intelligent systems. Each triple asserts a specific fact, such as (Barack Obama, bornIn, Honolulu), where "Barack Obama" and "Honolulu" are entities, and "bornIn" is a relation linking them.

The historical trajectory of knowledge graphs is deeply rooted in early efforts to formalize human knowledge. Its lineage can be traced back to semantic networks of the 1960s and 1970s, which used graphical structures to represent concepts and their interconnections, and expert systems of the 1980s, which aimed to capture domain-specific knowledge through symbolic rules and ontologies. Early projects like Cyc sought to build a comprehensive common-sense knowledge base through extensive manual curation. However, these early symbolic systems often struggled with inherent limitations such as scalability, the brittleness of their rule sets, and high maintenance costs when confronted with the vast, dynamic, and often inconsistent nature of real-world knowledge. A pivotal shift occurred with the advent of the World Wide Web and the rise of Linked Data principles, which emphasized large-scale, interconnected, and often automatically extracted knowledge. This evolution moved from small, manually crafted ontologies to vast, data-driven knowledge bases, leveraging advancements in information extraction, natural language processing, and data integration \cite{yan2022}. The growing need for domain-specific KGs, particularly those constructed from mixed-quality resources, further highlights this evolution towards more diverse and specialized knowledge representation tailored to specific contexts \cite{additional_paper_3}.

Modern KGs exemplify this shift, serving as expansive repositories of world knowledge. Prominent examples include Freebase, a large collaborative knowledge base acquired by Google, which aggregated structured data from various sources; DBpedia, which systematically extracts structured information from the semi-structured content of Wikipedia infoboxes; and Wikidata, a free and open knowledge base that acts as a central, multilingual storage for the structured data of its Wikimedia sister projects. These KGs collectively organize immense volumes of diverse facts, spanning biographical details, geographical locations, scientific concepts, and cultural artifacts, thereby forming a crucial backbone for numerous AI-driven applications.

The primary role of knowledge graphs is to provide a comprehensive and structured representation of facts, enabling machines to understand and reason with information that would otherwise be locked in unstructured text or disparate databases. This machine-readability facilitates a wide array of artificial intelligence tasks. For instance, KGs are instrumental in enhancing semantic search capabilities, allowing systems to interpret the underlying intent of a user's query rather than merely performing keyword matching. They are indispensable for advanced question answering systems, where structured facts enable precise and factual responses to natural language queries by mapping linguistic expressions to graph patterns \cite{huang2019}. Furthermore, KGs underpin sophisticated recommendation systems by modeling intricate relationships between users, items, and their attributes, leading to more relevant and explainable suggestions. This includes methods ranging from path-based approaches \cite{sun2018} to more recent graph neural network architectures \cite{liu2023, yang2023}. Beyond these, KGs are increasingly utilized for data representation and analysis, enabling complex semantic queries and data exploration tasks that leverage the rich structure of the embedded space \cite{additional_paper_1}.

Despite their explicit and interpretable nature, the discrete, symbolic structure of knowledge graphs presents inherent challenges for direct processing by many modern machine learning models, motivating the development of alternative representations.