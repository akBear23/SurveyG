\subsection{Theoretical Foundations and Expressiveness Guarantees}

The advancement of Knowledge Graph Embedding (KGE) models necessitates a profound theoretical understanding that transcends mere empirical performance. This pursuit focuses on elucidating their inherent representational capabilities and limitations, formalizing algebraic properties, and resolving fundamental expressiveness paradoxes to guide the design of mathematically sound and logically consistent architectures.

Early KGE models, while demonstrating impressive empirical performance, often lacked explicit theoretical guarantees regarding their ability to capture complex relational patterns. For instance, foundational translational models like TransE \cite{bordes2013} struggled with modeling symmetric relations or complex compositional patterns without specific architectural modifications or heuristic rules. Building upon these initial observations, models such as RotatE \cite{sun2019}, as discussed in Section 2.2, made significant strides by implicitly capturing diverse relational patterns through element-wise rotations in complex vector space. RotatE's inherent capacity to infer symmetry, antisymmetry, inversion, and composition demonstrated a higher degree of expressiveness. However, this capability was largely an emergent property of its design, highlighting a critical need for a more generalized and formal framework to define and guarantee such properties across the broader spectrum of KGE architectures.

A significant line of theoretical inquiry has focused on formalizing the logical expressiveness of KGE models, often by drawing connections to fragments of first-order logic (FOL). Researchers have analyzed which logical rules (e.g., symmetry, transitivity, inversion, composition) different KGE models can inherently represent \cite{wang2018, guo2018, chen2020}. This work revealed that many popular models, despite their empirical success, are fundamentally limited in their ability to capture even basic logical inferences due to their underlying geometric or algebraic structures. For example, simple vector addition in TransE, while elegant, inherently restricts its capacity to model certain complex logical relationships. These analyses laid the groundwork for understanding *why* certain models succeed or fail at specific reasoning tasks, moving beyond black-box performance metrics and providing a critical lens for evaluating their theoretical soundness.

More recently, efforts have intensified to formalize algebraic properties that ensure robust modeling of complex relational patterns. A critical property in this regard is "closure under composition," which ensures that if relations $r_1$ and $r_2$ can be composed to form $r_3$ (i.e., $h \xrightarrow{r_1} m \xrightarrow{r_2} t \implies h \xrightarrow{r_3} t$), the model can consistently represent this composition. \textcite{zheng2024} introduced HolmE, a Riemannian KGE model specifically designed to achieve this property. By embedding entities and relations into a Riemannian manifold, HolmE offers a flexible geometric framework that inherently supports compositional reasoning, even for complex or long-tail relational patterns. Their work provides a unifying theoretical framework, demonstrating that prominent models like TransE and RotatE can be interpreted as special cases of HolmE under specific geometric assumptions. This formalization represents a significant step towards providing mathematical guarantees for relational inference, moving beyond ad-hoc pattern recognition to principled design.

Concurrently, researchers have identified and addressed fundamental expressiveness paradoxes that expose inherent limitations in KGE model capabilities. A notable example is the "Z-paradox," which describes a specific four-entity inference pattern where two distinct paths (e.g., $A \xrightarrow{r_1} B \xrightarrow{r_2} D$ and $A \xrightarrow{r_3} C \xrightarrow{r_4} D$) should logically imply the same relationship between $A$ and $D$, but many KGE models, due to their inherent structural limitations (e.g., reliance on simple vector addition or fixed transformations), are structurally incapable of representing this equivalence. This paradox highlights a critical gap between empirical performance and logical consistency. \textcite{liu2024} addressed this by proposing MQuinE, a novel KGE model designed to explicitly cure the "Z-paradox." MQuinE achieves this by employing a richer algebraic structure based on quaternions, which provides greater flexibility in modeling complex transformations and interactions between entities and relations, thereby enabling the model to capture the nuanced equivalences inherent in the Z-paradox. By identifying and providing a mathematically sound resolution to such fundamental limitations, MQuinE contributes to the development of models that are not only empirically strong but also logically consistent and theoretically robust.

These theoretical advancements collectively aim to provide unifying frameworks that explain the underlying principles, capabilities, and limitations of diverse KGE architectures. While significant progress has been made in formalizing properties like closure under composition and resolving expressiveness paradoxes, critical challenges remain. The ongoing quest involves extending these theoretical guarantees to more complex scenarios, such as the dynamic and temporal KGs discussed in Section 4, or multi-modal KGs (Section 7.3), where the interplay of time, context, and diverse data modalities introduces new layers of complexity. Furthermore, developing frameworks that can dynamically adapt their expressiveness based on the specific relational patterns present in a given knowledge graph, rather than relying on a fixed expressive capacity, represents a promising future research avenue. This continuous drive for deeper theoretical understanding is crucial for guiding the design of more robust, mathematically sound, and logically consistent KGE models capable of sophisticated reasoning.