\subsection*{Background: Knowledge Graphs}

The increasing volume and complexity of digital information necessitate robust mechanisms for organizing, representing, and reasoning with vast amounts of data. Knowledge Graphs (KGs) have emerged as a powerful paradigm to address this challenge, offering a structured and machine-readable representation of world knowledge \cite{semantic_web_foundations}. Fundamentally, a knowledge graph is a network of real-world entities (e.g., people, places, concepts) and their semantic relationships, typically expressed as subject-predicate-object triples. For instance, the triple (\textit{Barack Obama}, \textit{bornIn}, \textit{Hawaii}) explicitly states a factual relationship between two entities. This structured format distinguishes KGs from unstructured text or traditional relational databases by embedding rich semantic meaning directly into their graph structure, making complex information actionable for intelligent systems.

The historical development of knowledge graphs can be traced back to early artificial intelligence research on semantic networks in the 1960s, which aimed to represent human knowledge in a graph-like structure for machine understanding \cite{semantic_networks}. This foundational work evolved into the vision of the Semantic Web in the early 2000s, advocating for a "web of data" where information is given well-defined meaning, enabling computers and people to work in cooperation \cite{semantic_web}. Technologies like the Resource Description Framework (RDF) and Web Ontology Language (OWL) became cornerstones for building this structured web. Modern large-scale knowledge bases, often referred to as knowledge graphs, are a direct realization of this vision, moving beyond domain-specific ontologies to encompass vast amounts of general world knowledge, providing a backbone for many AI applications.

Several prominent examples illustrate the scale and impact of modern knowledge graphs. \textit{Freebase}, an early collaborative knowledge base acquired by Google, was instrumental in demonstrating the potential of large-scale, open-domain KGs to organize factual information \cite{freebase_origin}. It provided a structured repository of millions of entities and their relationships, significantly influencing the development of Google's Knowledge Graph used in search. Similarly, \textit{DBpedia} automatically extracts structured information from Wikipedia infoboxes, creating a vast cross-domain knowledge graph that serves as a central hub for linking various datasets on the Semantic Web \cite{dbpedia_creation}. More recently, \textit{Wikidata} has become a crucial, collaboratively edited, and multilingual knowledge base that provides structured data for Wikipedia, Wikimedia Commons, and other projects, acting as a central repository for facts that can be queried and reused globally \cite{wikidata_overview}. These examples highlight how KGs serve as invaluable resources for aggregating and organizing diverse world knowledge, making it accessible and interoperable.

The structured nature of knowledge graphs makes complex information inherently machine-readable, facilitating a wide array of intelligent systems and artificial intelligence tasks. By providing explicit representations of facts and their interconnections, KGs enable more sophisticated forms of information processing than keyword-based approaches. For instance, in \textit{semantic search}, KGs allow systems to understand the intent behind a query and retrieve relevant entities and relationships, rather than just matching keywords. \textit{Question Answering (QA)} systems leverage KGs to directly find answers to natural language questions by traversing the graph to locate specific facts. Furthermore, KGs enhance \textit{recommendation systems} by providing rich contextual information about items and users, enabling the discovery of nuanced preferences and relationships that go beyond simple co-occurrence. They also support various forms of \textit{reasoning}, allowing AI systems to infer new facts from existing ones, thereby expanding the knowledge base. This foundational understanding of knowledge graphs, as rich, structured repositories of interconnected facts, is paramount for appreciating the subsequent need for advanced techniques, such as knowledge graph embedding, to effectively leverage these intricate structures for machine learning tasks.

% Note: The provided "Papers to reference" and "Community summaries" were focused on Knowledge Graph Embedding (KGE) models, not the foundational "Background: Knowledge Graphs" itself. Therefore, generic placeholder citations (e.g., \cite{semantic_web_foundations}) have been used for the fundamental concepts and examples of Knowledge Graphs, as no specific papers on KG background were available in the provided list.