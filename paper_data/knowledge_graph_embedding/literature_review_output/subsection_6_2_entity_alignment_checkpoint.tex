\subsection{Entity Alignment}

Entity alignment (EA) is a critical task for integrating heterogeneous knowledge graphs (KGs), aiming to identify equivalent entities across different knowledge bases to enable interoperability and enrich semantic understanding. Knowledge Graph Embedding (KGE) techniques, which represent entities and relations as low-dimensional vectors, have proven instrumental in addressing this challenge by leveraging learned entity representations to match corresponding entities. The foundation for such methods lies in robust KGE models like TransH \cite{wang2014}, which improved upon earlier translational models by projecting entities onto relation-specific hyperplanes to better capture complex relation types, and TransD \cite{ji2015}, which further refined entity and relation diversity through dynamic mapping matrices. More advanced models like RotatE \cite{sun2018} introduced relational rotations in complex space, capable of simultaneously modeling diverse relational patterns such as symmetry, inversion, and composition, thereby generating richer entity embeddings crucial for discerning subtle equivalences in EA.

Early embedding-based entity alignment approaches faced significant hurdles, particularly the scarcity of labeled training data (prior alignments), which limited their precision. To mitigate this, \cite{sun2018} introduced BootEA, a pioneering bootstrapping approach that iteratively labels likely entity alignments to expand training data. BootEA further incorporated an alignment editing method and a global optimization strategy based on max-weighted matching to reduce error accumulation and ensure one-to-one alignment, marking a substantial advancement in semi-supervised EA. Building upon the need for robust semi-supervised learning, \cite{pei2019} proposed Semi-supervised Entity Alignment (SEA), which enhanced KGE by explicitly incorporating awareness of entity degree differences through adversarial training. This approach improved alignment accuracy and robustness, particularly for entities with varying frequencies, by addressing a limitation where high-frequency entities might dominate the embedding space.

While these methods effectively addressed data scarcity, they often relied primarily on relational structures, overlooking other rich entity features. Recognizing this limitation, \cite{zhang2019} introduced MultiKE, a novel framework that unifies multiple, complementary "views" of entities (name, relation, and attribute) to learn more comprehensive embeddings for alignment. MultiKE innovated by designing view-specific embedding models and, crucially, a "soft alignment" mechanism for relations and attributes that automatically identifies and updates alignment information during training, thereby reducing the heavy dependency on pre-existing seed alignments for these components. This multi-view approach significantly enhanced the accuracy and robustness of entity alignment by leveraging a broader spectrum of entity characteristics.

Despite advancements in integrating diverse features, a critical source of error, termed "class conflicts," persisted due to the neglect of ontological schema (classes, hierarchies, and logical constraints) in existing EA methods. To address this, \cite{xiang2021} presented OntoEA, the first comprehensive framework to integrate ontological knowledge into joint KG-ontology embedding. OntoEA introduced a novel Class Conflict Matrix (CCM) to model inter-class conflicts and employed non-linear ontology and membership embedding modules to ensure semantically consistent alignments, preventing false positives that arise from aligning entities belonging to incompatible classes. This marked a significant step towards ensuring the semantic integrity of aligned knowledge graphs.

As the field of KGE-based entity alignment matured, the need for systematic analysis and evaluation became paramount. General KGE surveys, such as that by \cite{yan2022}, provided foundational classifications of embedding models and their applications. More specifically for EA, \cite{fanourakis2022} conducted an experimental review, offering a meta-level analysis of state-of-the-art EA methods. This study provided statistically significant rankings of methods and identified correlations between their performance and various KG meta-features, offering empirical guidance for method selection. Most recently, \cite{zhu2024} provided a specialized and up-to-date survey on embedding-based EA, proposing a novel three-module framework (Information Aggregation, Alignment, and Post-Alignment) and identifying critical future directions, including multimodal EA and handling dynamic KGs.

In conclusion, the application of KGE to entity alignment has evolved from foundational embedding models to sophisticated, multi-faceted solutions. Initial efforts focused on overcoming data scarcity through semi-supervised techniques like bootstrapping \cite{sun2018} and addressing entity heterogeneity \cite{pei2019}. Subsequent research expanded the scope of information leveraged, incorporating diverse entity features through multi-view embeddings \cite{zhang2019} and integrating higher-level ontological semantics to ensure semantic consistency \cite{xiang2021}. While significant progress has been made in resolving data heterogeneity and enabling interoperability, challenges remain, particularly concerning scalability to extremely large and dynamic KGs, the integration of multimodal data, and the continuous need to balance rich information exploitation with computational efficiency. These areas represent promising avenues for future research in entity alignment.