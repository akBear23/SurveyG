\subsection{Domain-Specific Applications}
Knowledge graph embeddings (KGEs) have proven instrumental in transcending general-purpose knowledge representation, demonstrating remarkable versatility and impact when tailored for specialized, complex domains. These applications leverage KGEs for intricate knowledge discovery, nuanced analytical tasks, and highly specialized question answering, often integrating multimodal data and domain-specific reasoning mechanisms.

In the biomedical domain, KGEs are pivotal for accelerating knowledge discovery and drug repurposing. \cite{zhu2022} addressed the lack of disease-specific focus by pioneering the construction of Specific Disease Knowledge Graphs (SDKGs) and introducing a novel multimodal reasoning approach. This method effectively integrates structural, categorical, and descriptive embeddings via reverse-hyperplane projection to uncover new, reliable knowledge, with manual proofreading and molecular-level validation confirming its efficacy for specific diseases. Building on the need for deeper insights and practical deployment, \cite{yang2025} developed SEConv, a semantic-enhanced KGE model for healthcare prediction. SEConv leverages a resource-efficient self-attention mechanism to generate expressive embeddings and employs multi-layer convolutional neural networks to learn complex, deeper structural features from medical KGs, demonstrating its applicability for deployment on resource-limited consumer electronics and enhancing medical decision-making.

Beyond life sciences, KGEs offer powerful tools for innovation analysis within vast, heterogeneous datasets like patent metadata. \cite{li2022} tackled the challenge of measuring complex, heterogeneous knowledge proximity by constructing 'PatNet', a large-scale knowledge graph from US patent metadata (1976-2020). By applying various KGE models, including TransE\_l2, to PatNet, their work operationalized heterogeneous knowledge proximity (e.g., between inventors and technological domains) using cosine similarity of embeddings, providing a unified framework for analyzing innovation dynamics that surpassed limitations of prior measures restricted to homogeneous entity pairs. This approach demonstrated strong performance in explaining real-world domain expansion profiles.

For highly specialized scientific fields, KGEs underpin advanced question answering systems capable of handling deep domain ontologies and intricate mechanisms. \cite{zhou2023} presented "Marie and BERT," a sophisticated KGQA system specifically designed for chemistry. This system integrates hybrid multi-embedding spaces, parallel querying, a BERT-based bidirectional entity-linking model, and a joint numerical embedding model to efficiently answer questions requiring numerical filtering and understanding of complex chemical reaction mechanisms, thereby navigating deep chemical ontologies with high precision and robustness. The system's modular design allows for specialized handling of diverse chemical knowledge.

These diverse applications collectively underscore the transformative potential of KGEs in specialized domains. They highlight a critical evolution from generic link prediction to context-aware, multimodal, and domain-specific reasoning, often integrating advanced neural architectures and validation mechanisms. The progression from foundational multimodal reasoning to efficient deep structural feature learning, and from heterogeneous proximity measurement to highly specialized QA, showcases KGEs' adaptability. Future research will likely focus on further enhancing the explainability of these complex models, adapting to real-time knowledge evolution, and developing more robust multimodal integration strategies to unlock even deeper insights in these critical fields.