\subsection{Distributed, Secure, and Lifelong Learning for KGE}
The dynamic, sensitive, and decentralized nature of real-world knowledge graphs (KGs) necessitates advanced learning paradigms that enable models to continuously adapt, operate across distributed data sources, and maintain robust security and privacy. This section explores recent advancements in continual learning and federated learning for Knowledge Graph Embedding (KGE), addressing the challenges of evolving KGs and sensitive data.

A fundamental challenge for KGE in dynamic environments is the ability to adapt to new entities and relations without forgetting previously learned information, a concept central to lifelong or continual learning. Early efforts to enable KGE models to handle emerging entities focused on inductive approaches. \cite{wang2018} introduced a Logic Attention Network (LAN) for inductive KGE, which aggregates neighborhood information to embed unseen entities by satisfying properties like permutation invariance and query relation awareness. Building upon this, \cite{chen2021} proposed MorsE, a meta-learning framework that learns "meta-knowledge" to produce general entity embeddings for entirely new entities in new KGs, moving beyond mere inductive relation prediction. This meta-knowledge, captured through an entity initializer and a GNN modulator, allows for the transfer of structural patterns across different KGs. Further advancing this direction, \cite{sun2024} developed MetaHG, a meta-learning strategy specifically for dynamic KGE updates in evolving service ecosystems. MetaHG addresses the inefficiency of updating incremental knowledge by integrating both local and global structural information from KG snapshots, mitigating issues like spatial deformation and enhancing the representation of emerging entities.

Beyond evolving KGs, the distributed nature of data and increasing privacy concerns have spurred the development of federated learning approaches for KGE. Federated Knowledge Graph Embedding (FKGE) allows multiple clients to collaboratively train models without sharing their raw data. Addressing the inherent semantic disparities among clients in such settings, \cite{zhang2024} introduced PFedEG, a personalized federated KGE framework. PFedEG generates personalized supplementary knowledge for each client by aggregating entity embeddings from "neighboring" clients based on a learned client-wise relation graph, thus preventing shared knowledge from being inundated with noise. A critical practical challenge in FKGE is communication efficiency, especially given the large parameter sizes of KGE models and numerous communication rounds. To this end, \cite{zhang2024} proposed FedS, a bidirectional communication-efficient framework that employs an entity-wise Top-K sparsification strategy. This method dynamically identifies and transmits only the most significantly changed entity embeddings, drastically reducing communication overhead while maintaining embedding precision and overall performance.

The deployment of KGE models in distributed and resource-constrained environments further highlights the need for efficient model designs. Lightweight KGE models and knowledge distillation techniques are crucial for enabling federated learning on edge devices or mobile platforms. For instance, \cite{zhu2020} introduced DualDE, a knowledge distillation framework that constructs low-dimensional student KGEs from high-dimensional teacher models. DualDE adaptively weights soft labels and employs a two-stage distillation to achieve faster and cheaper reasoning with minimal accuracy loss, making KGE more deployable in resource-limited settings. Similarly, \cite{wang2021} presented LightKG, a lightweight KGE framework that significantly reduces storage and inference time by using codebooks and codewords instead of continuous vectors. Such efficient designs are vital for the practical implementation of FKGE, where individual clients may have limited computational resources.

However, distributed learning environments also introduce new security vulnerabilities. \cite{zhou2024} systematically investigated poisoning attacks on Federated Knowledge Graph Embedding, demonstrating how malicious clients can inject corrupted data to degrade the global model's performance. Their work highlights the critical need for robust and adaptive learning systems that can detect and mitigate such adversarial behaviors in distributed KGE environments.

In conclusion, the trajectory of KGE research is increasingly focused on developing models that are not only expressive but also adaptable to dynamic knowledge, privacy-preserving in distributed settings, and resilient against adversarial attacks. While significant progress has been made in continual learning for evolving KGs and the foundational aspects of federated KGE, challenges remain in ensuring robust security, seamless personalization across highly heterogeneous clients, and further optimizing communication efficiency for truly scalable and trustworthy distributed KGE systems.