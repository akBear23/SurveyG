\subsection{Deep Theoretical Analyses of KGE Expressiveness}

The evolution of Knowledge Graph Embedding (KGE) research has progressed significantly beyond empirical performance benchmarks, increasingly focusing on rigorous theoretical analyses of model expressiveness. This maturation aims to understand the fundamental mathematical capabilities and inherent limitations that govern KGE models' ability to capture complex relational patterns and perform logical inferences.

Early models, while achieving notable empirical success, often suffered from implicit theoretical deficiencies. A prominent example is the "Z-paradox," identified by \cite{liu2024}. This paradox reveals a fundamental expressiveness bottleneck where many popular KGE models, including translation-based (e.g., TransE, RotatE) and certain bilinear models, incorrectly infer relationships based on a specific graph pattern, leading to false positives. The Z-paradox significantly degrades performance on affected test facts (e.g., ~35\% in FB15k-237, causing over 20\% accuracy drops), highlighting a critical need for models with stronger theoretical guarantees. To address this, \cite{liu2024} introduced MQuinE, a novel matrix-based KGE model designed to inherently circumvent the Z-paradox through a unique score function incorporating a cross-term, while simultaneously preserving the ability to model diverse relation patterns like symmetry, inversion, and composition.

Another foundational limitation was observed in the regularization strategies of early models. For instance, \cite{ebisu2017} formally analyzed TransE's inherent conflict between its translation principle ($h+r=t$) and its regularization, which forces entity embeddings onto a unit sphere in Euclidean space. This conflict warps embeddings and adversely affects accuracy. To resolve this, \cite{ebisu2017} proposed TorusE, which embeds entities and relations on a compact Abelian Lie group (a torus). By leveraging the torus's compactness, TorusE eliminates the need for explicit regularization, thereby resolving the conflict and leading to more accurate and less warped embeddings, demonstrating the power of choosing an appropriate mathematical space.

While models like RotatE \cite{sun2018}, discussed in Section 2.3, demonstrated significant progress by modeling relations as element-wise rotations in complex vector space to inherently capture symmetry, antisymmetry, inversion, and composition, the field initially lacked a unifying algebraic framework to explain *why* certain models succeeded or failed, or how their expressiveness was fundamentally constrained. This gap has been substantially addressed by recent theoretical contributions that delve into the formal properties of KGE models.

A key development is the formalization of properties like 'closure under composition,' which provides robust theoretical guarantees for modeling complex relational patterns, especially multi-hop reasoning. \cite{zheng2024} introduced HolmE, a Riemannian KGE model explicitly designed to be closed under composition. This property ensures that if a model can represent relations $r_1$ and $r_2$, it can also accurately represent their composition $r_1 \circ r_2$, even for under-represented (long-tail) relations. Crucially, \cite{zheng2024} theoretically demonstrated that prominent existing models like TransE and RotatE can be unified as special cases of HolmE. This unification is achieved by viewing them as specific transformations on a Riemannian manifold derived from Lie group theory, where translation and rotation are specific instances. This provides a broader algebraic framework that explains their respective capabilities and limitations from a foundational perspective, highlighting the mathematical elegance and theoretical soundness of HolmE.

Beyond composition, theoretical analyses have broadened to encompass other forms of logical expressiveness and generalized geometric structures. \cite{he2024y6o} introduced AConE, a query embedding method that explains learned knowledge in the form of $SROI^-$ description logic axioms. AConE embeds $SROI^-$ concepts as cones in complex vector space and relations as transformations that rotate and scale these cones, defining an algebra whose operations correspond to $SROI^-$ description logic concept constructs. This work bridges KGE with formal logic, offering a principled way to represent and reason with ontological knowledge. Similarly, \cite{fatemi2018e6v} investigated the ability of KGE models to respect background taxonomic information (subclasses and subproperties), proving that existing fully expressive models often cannot provably respect such axioms without modifications, highlighting another area where theoretical expressiveness falls short.

The exploration of diverse mathematical spaces has led to more generalized and expressive architectures. \cite{li2024} proposed GoldE, a framework that generalizes KGE approaches in *both dimension and geometry* of orthogonal relation transformations. GoldE introduces a universal orthogonal parameterization based on a generalized Householder reflection, unifying reflections across Euclidean, elliptic, and hyperbolic geometries. This allows for mixed orthogonal parameterization within a product manifold, enabling GoldE to simultaneously capture both cyclical and hierarchical structures inherent in topologically heterogeneous KGs, overcoming the limitations of prior models restricted to homogeneous geometries or lower dimensions. Further, \cite{gebhart2021gtp} introduced Knowledge Sheaves, a novel sheaf-theoretic framework that describes KGE as an approximate global section of a "knowledge sheaf" over the graph, offering a generalized framework for reasoning about KGE models and incorporating consistency constraints from graph schemas.

These deep theoretical analyses, as systematically reviewed by \cite{cao2022} from the perspective of representation spaces (algebraic, geometric, analytical), move beyond mere empirical performance. They offer foundational insights into the mathematical capabilities and limitations of KGE models, guiding the design of more robust and expressive architectures. By formally characterizing properties such as the absence of the Z-paradox, closure under composition, and the ability to represent description logic axioms, the field gains a principled understanding of what makes an embedding space suitable for complex logical inferences. While significant progress has been made in formalizing composition and addressing specific paradoxes, the full theoretical characterization of KGE expressiveness across all types of logical inferences, including intersection, union, and negation, remains an active and critical area of research, paving the way for KGE models with even stronger theoretical guarantees for complex reasoning tasks.