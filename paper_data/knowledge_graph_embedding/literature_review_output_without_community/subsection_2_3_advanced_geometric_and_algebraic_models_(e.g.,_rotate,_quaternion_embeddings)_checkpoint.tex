\subsection{Advanced Geometric and Algebraic Models (e.g., RotatE, Quaternion Embeddings)}

The pursuit of more expressive Knowledge Graph Embeddings (KGEs) has driven research beyond simple Euclidean vector spaces, leading to the exploration of sophisticated mathematical spaces and algebraic structures. This paradigm shift, as systematically classified by surveys focusing on representation spaces \cite{cao2022}, aims to inherently capture complex relational patterns and multi-faceted entity semantics through the fundamental properties of the embedding space itself, thereby enhancing model expressiveness and inferential capabilities.

Early explorations into advanced geometric forms moved beyond point-wise representations. For instance, \textit{ManifoldE} \cite{xiao2015}, introduced in 2015, proposed a manifold-based embedding principle that mapped true triples not to single points, but to *manifolds* (e.g., high-dimensional spheres or hyperplanes). This approach addressed the "ill-posed algebraic system" and "over-strict geometric form" limitations prevalent in earlier point-wise models like TransE, offering a more flexible representation for complex relations (e.g., one-to-many, many-to-many) by allowing entities to reside on a manifold defined by the head and relation. ManifoldE thus laid groundwork for more nuanced geometric modeling by providing a richer geometric context for entity-relation interactions.

A foundational advancement that established a new paradigm was the introduction of \textit{RotatE} \cite{sun2018} in 2018, which revolutionized the modeling of relational patterns by embedding entities and relations into a complex vector space. In RotatE, a relation is conceptualized as an element-wise rotation from a head entity to a tail entity, leveraging the inherent properties of complex numbers to represent geometric transformations. The modulus of each element of the relation embedding is constrained to one, ensuring a pure rotational effect. This elegant formulation allowed RotatE to simultaneously capture and infer diverse relational patterns, including symmetry (rotation by $0$ or $\pi$), antisymmetry (rotation by $\pi$), inversion (rotation by $-\theta$), and composition (sequential rotations). This capability significantly surpassed many prior models like TransE, DistMult, and ComplEx \cite{sun2018}, providing a unified and mathematically sound framework for encoding intricate relational logic.

Despite its strengths, RotatE's strict rotational mechanism in complex space presented certain limitations, particularly in modeling transitivity without forcing identical entity embeddings in a transitive chain, which can limit expressiveness \cite{song2021}. To address this, \textit{Rot-Pro} \cite{song2021} extended RotatE by combining relational rotation with *projection* in complex space. Rot-Pro theoretically demonstrated that transitive relations could be modeled using idempotent transformations (projections), allowing entities in a transitive chain to have distinct embeddings while sharing a common projected vector. This innovation enabled Rot-Pro to model all five major relation patterns (symmetry, antisymmetry, inversion, composition, and transitivity) within a unified framework, showcasing how the complex space could be further refined to overcome specific pattern limitations.

Building upon the success of RotatE's 2D rotational mechanism, subsequent research explored extensions to higher-dimensional geometric transformations. \textit{Rotate3D} \cite{gao2020} proposed representing entities in a three-dimensional (3D) space and modeling relations as rotations within this space. This approach specifically aimed to address the challenge of modeling non-commutative relation composition, a crucial aspect for robust multi-hop reasoning where the order of relations significantly impacts the outcome \cite{gao2020}. By exploiting the mathematical properties of 3D rotations, Rotate3D offered a more natural and effective mechanism for preserving the order of relation composition compared to 2D complex rotations.

Extending this trajectory towards even higher-dimensional algebraic structures, the concept of \textit{quaternion embeddings} has emerged as a particularly promising avenue for further enriching KGE models. While complex numbers facilitate 2D rotations, quaternions, a non-commutative extension of complex numbers, naturally model 3D or 4D rotations, providing a more powerful algebraic framework. Early work, such as the multi-embedding model proposed by \cite{tran20195x3}, demonstrated the potential of quaternion algebra to unite and generalize various KGE models, while also introducing a novel quaternion-based model that achieved promising results on benchmark datasets. This indicated that quaternions could offer a richer representational capacity than complex numbers, enabling the capture of more intricate inter-dimensional dependencies. A concrete example is \textit{Contextualized Quaternion Embedding (ConQuatE)} \cite{chen2025}, which leverages quaternion rotation to efficiently incorporate contextual cues from various connected relations. ConQuatE enriches original entity representations through efficient vector transformations in quaternion space, allowing it to model multiple semantic dimensions for entities without requiring external information. This capability is particularly beneficial for addressing issues like entity polysemy, where entities exhibit different semantic characteristics depending on their relational context. However, the adoption of higher-dimensional algebras like quaternions also introduces increased model complexity and a larger parameter space, which can lead to computational overhead and a higher risk of overfitting, necessitating careful regularization and optimization strategies.

In conclusion, the evolution of KGE models from Euclidean spaces to advanced geometric and algebraic structures, exemplified by early manifold-based approaches like ManifoldE, complex vector space embeddings like RotatE and its extensions (e.g., Rot-Pro), 3D rotational models like Rotate3D, and the subsequent introduction of quaternion embeddings (e.g., ConQuatE), underscores a continuous drive to enhance expressiveness through fundamental mathematical principles. This progression from flexible geometric forms to 2D, 3D, and 4D algebraic structures has enabled KGE models to capture increasingly complex relational patterns, including non-commutative composition and more nuanced entity semantics, by leveraging the inherent rotational and compositional properties of these advanced mathematical spaces. This focus on enriching the fundamental mathematical basis of embeddings remains a vital direction for developing more powerful and theoretically sound KGE models.