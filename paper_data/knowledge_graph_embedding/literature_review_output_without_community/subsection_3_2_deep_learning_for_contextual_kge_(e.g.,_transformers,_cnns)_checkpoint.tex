\subsection{Deep Learning for Contextual KGE (e.g., Transformers, CNNs)}

Traditional Knowledge Graph Embedding (KGE) models often represent entities and relations as isolated triples, thereby overlooking the rich contextual information embedded within the broader graph structure and associated textual data. To address this limitation, deep learning architectures are increasingly employed to infuse contextual information into KGEs, moving beyond simplistic triple representations to capture richer semantics. As highlighted in general KGE surveys \cite{yan2022}, neural network-based approaches have become pivotal in this shift, enabling models to learn more nuanced and context-aware embeddings. This section explores the evolution of deep learning for contextual KGE, from early Graph Neural Networks (GNNs) and Convolutional Neural Networks (CNNs) to advanced Transformer-based architectures.

Early efforts to incorporate broader context into KGEs leveraged Graph Neural Networks (GNNs) due to their inherent ability to aggregate information from local neighborhoods. GNNs, such as the foundational Relational Graph Convolutional Networks (R-GCNs) \cite{schlichtkrull2018modeling} (though not directly from the provided papers, it's a key precursor), learn entity representations by iteratively aggregating messages from their neighbors, thereby capturing structural context. Building upon this, more sophisticated GNNs integrated attention mechanisms to discern the importance of different neighbors and relation paths. For instance, Graph Attenuated Attention networks (GAATs) \cite{wang2020} address the limitation of uniform weighting in traditional GCNs by introducing an attenuated attention mechanism. This mechanism dynamically assigns different weights to relation paths and actively acquires information from neighboring nodes, leading to more nuanced entity and relation embeddings. Similarly, \cite{li2021qr0} proposed a novel heterogeneous GNN framework designed to handle the intrinsic heterogeneity of knowledge graphs. This method aggregates neighbor features under various relation-paths and then learns the importance of these different paths through an attention mechanism, selectively focusing on informative semantic aspects. More recently, the Mixed Geometry Message and Trainable Convolutional Attention Network (MGTCA) \cite{shang2024} further refines GNN-based contextualization by integrating a Trainable Convolutional Attention Network (TCAN) that adaptively combines different GNN types (GCN, GAT, and a novel KGCAT) and learns attention weights for local structures, addressing data dependence and enhancing neighbor message aggregation by generating messages in mixed geometric spaces.

Beyond GNNs, Convolutional Neural Networks (CNNs) offer another powerful paradigm for extracting contextual features, particularly by treating entity-relation pairs as structured inputs. A seminal work in this area is ConvE \cite{dettmers2018conve}, which reshapes entity and relation embeddings into 2D matrices, applies convolutional filters to capture local interaction patterns, and then projects the result for scoring. This approach effectively extracts rich features from the triple structure. Building on this, PConvKB \cite{jia20207dd} improved KGE by incorporating relation paths, both locally and globally, into a convolutional framework. It uses an attention mechanism to measure the local importance of relation paths and a novel measure (DIPF) for global importance, demonstrating how CNNs can be augmented with attention to capture richer path-based context. Furthermore, in the context of scalable KGE, CPa-WAC \cite{modak2024} utilizes a Weighted Aggregation Composition (WAC) convolution within its GNN framework and a 1D Convolutional Neural Network for decoding. This highlights the versatility of CNNs not just for initial feature extraction but also for robust embedding learning and decoding in complex, partitioned graph structures.

The advent of Transformer architectures revolutionized sequence modeling, and their adaptation to KGE marked a significant leap in contextualization. Early work like **CoKE (Contextualized Knowledge Graph Embedding) \cite{wang2019}** pioneered this by modeling graph contexts (edges and multi-hop paths) as sequences of entities and relations. CoKE leverages Transformer encoder blocks to learn dynamic, context-dependent representations, where an entity's or relation's embedding adapts to its specific input sequence, moving beyond static global embeddings. However, applying vanilla Transformers to knowledge graphs presents a fundamental challenge: their self-attention mechanism is inherently order-invariant, struggling to distinguish the distinct roles (e.g., subject vs. object) of entities within a relational triple. This architectural mismatch can lead to semantically incorrect representations and inconsistent training. To overcome this, **Knowformer \cite{li2023}** introduced "relational compositions" into entity representations. These compositions explicitly inject semantics and capture an entity's role based on its position within a relation triple, effectively making the self-attention mechanism position-aware for knowledge graphs and correctly interpreting relational directionality. Moving towards more holistic contextualization, **TGformer \cite{shi2025}**, a comprehensive Graph Transformer framework (noted as a forthcoming work), extends this to both static and temporal knowledge graphs. TGformer innovates by constructing "context-level subgraphs" for each predicted triplet, explicitly modeling inter-triplet relationships. Its core **Knowledge Graph Transformer Network (KGTN)** explores multi-structural features, encompassing fine-grained triplet-level, broader graph-level, and contextual cues, representing a significant stride in integrating diverse structural and temporal features for a more holistic contextual understanding.

In summary, the integration of deep learning architectures for contextual KGE demonstrates a clear progression. GNNs excel at aggregating local neighborhood information, providing structural context by propagating messages across the graph. CNNs are adept at extracting local patterns and features from structured representations of triples or paths. Transformers, on the other hand, are powerful for capturing long-range dependencies and dynamic semantic contexts by treating graph structures as sequences. While GNNs and CNNs primarily focus on local or structured context, Transformer-based models like CoKE and Knowformer aim for more dynamic and position-aware contextualization, addressing the inherent limitations of static embeddings. Frameworks like TGformer further push the boundaries by integrating multi-structural and temporal features, moving towards a truly holistic understanding of context. Despite these advancements, challenges remain in effectively combining diverse types of context (structural, textual, temporal) in a unified, scalable, and computationally efficient manner, especially for extremely large and dynamic knowledge graphs. The ongoing research highlights a continuous effort to move beyond isolated triple representations and fully capture the rich, often implicit, contextual information within knowledge graphs.