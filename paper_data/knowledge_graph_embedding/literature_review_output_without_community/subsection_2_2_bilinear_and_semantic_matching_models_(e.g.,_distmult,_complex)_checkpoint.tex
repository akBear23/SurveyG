\subsection{Bilinear and Semantic Matching Models (e.g., DistMult, ComplEx)}

Semantic matching models constitute a pivotal paradigm in Knowledge Graph Embedding (KGE), diverging from the geometric intuition of translational models by directly measuring the plausibility of a triple $(h, r, t)$ through various scoring functions \cite{ge2023, choudhary2021}. These models embed entities and relations into a continuous vector space, where the interaction between their embeddings directly quantifies the semantic compatibility or match of a given fact. This approach, often rooted in bilinear forms or dot products, represents a key alternative in the early development of KGE, focusing on capturing richer semantic interactions beyond simple vector translations \cite{cao2022}.

One of the earliest and most foundational models in this category is RESCAL (Relation-based Semantic Matching) \cite{nickel2011three}. RESCAL models relations as full matrices, $\mathbf{M}_r$, and scores a triple using a bilinear form: $f(h, r, t) = \mathbf{h}^\top \mathbf{M}_r \mathbf{t}$. This general tensor factorization approach allows RESCAL to be highly expressive, theoretically capable of capturing a wide array of relational patterns, including symmetric, antisymmetric, and compositional structures. However, its expressiveness comes at a significant computational cost, as each relation requires a full matrix, leading to a large number of parameters and a propensity for overfitting, especially on sparse knowledge graphs \cite{choudhary2021}. The computational burden and overfitting issues of RESCAL motivated the search for more efficient yet still expressive semantic matching models.

Building upon the bilinear framework, DistMult \cite{yang2014embedding} emerged as a more efficient alternative. DistMult simplifies RESCAL by constraining the relation matrices $\mathbf{M}_r$ to be diagonal. The scoring function then becomes $f(h, r, t) = \sum_k \mathbf{h}_k \mathbf{r}_k \mathbf{t}_k$, where $\mathbf{r}$ is a vector representing the diagonal entries of $\mathbf{M}_r$. This diagonal constraint significantly reduces the number of parameters, making DistMult more scalable and less prone to overfitting than RESCAL. Consequently, DistMult excels at modeling symmetric relations, where $f(h, r, t) = f(t, r, h)$ holds naturally. However, this very constraint limits its ability to effectively capture antisymmetric relations (e.g., "is_parent_of" vs. "is_child_of") or inversion patterns, as the diagonal matrix cannot distinguish between the head and tail entities in an asymmetric manner \cite{choudhary2021}.

To overcome DistMult's limitation with antisymmetric relations, ComplEx \cite{trouillon2016complex} extended the semantic matching paradigm by embedding entities and relations into a complex vector space. In ComplEx, entities and relations are represented by complex vectors, allowing them to have both real and imaginary components. The scoring function for a triple $(h, r, t)$ is defined using a Hermitian dot product: $f(h, r, t) = \text{Re}(\langle \mathbf{h}, \mathbf{r}, \bar{\mathbf{t}} \rangle)$, where $\bar{\mathbf{t}}$ is the conjugate of $\mathbf{t}$. By leveraging complex-valued embeddings, ComplEx can naturally capture both symmetric and antisymmetric relations. For instance, if a relation is symmetric, its imaginary component can be zero, resembling DistMult. For antisymmetric relations, the imaginary components allow for distinct scores for $(h, r, t)$ and $(t, r, h)$, effectively modeling inversion patterns. This made ComplEx a significant advancement, offering a more expressive framework for diverse relational patterns compared to its real-valued bilinear predecessors \cite{choudhary2021}.

Another notable model within this family that predates or is contemporary with ComplEx and also aims for improved expressiveness is HolE (Holographic Embeddings) \cite{nickel2016holographic}. HolE introduces circular correlation to model interactions, where the relation embedding $\mathbf{r}$ acts as a composition operator on the head and tail entity embeddings. Its scoring function is typically $f(h, r, t) = \mathbf{r}^\top (\mathbf{h} \ast \mathbf{t})$, where $\ast$ denotes circular correlation. HolE offers a more compact and efficient way to capture interactions than full tensor factorization (RESCAL) while being more expressive than DistMult. It can model asymmetric relations and is conceptually linked to ComplEx, as the circular correlation can be viewed as a special case of a complex-valued dot product in the Fourier domain. However, like ComplEx, HolE also faced challenges in robustly inferring complex compositional patterns, which require more sophisticated algebraic or geometric transformations.

In summary, bilinear and semantic matching models, from the foundational RESCAL to the more refined DistMult, HolE, and ComplEx, established a crucial alternative to translational models in early KGE research. They shifted the focus to direct compatibility scores, leveraging tensor factorization and bilinear forms to capture semantic interactions. While RESCAL offered high expressiveness at a cost, DistMult provided efficiency for symmetric relations. HolE and ComplEx further advanced the paradigm by introducing circular correlation and complex vector spaces, respectively, significantly enhancing the ability to model antisymmetric relations and inversion patterns. Despite these advancements, a common limitation across these foundational semantic matching models was their inherent difficulty in robustly capturing complex compositional patterns, which subsequently motivated the exploration of more advanced geometric and algebraic structures in KGE, as discussed in the following subsections.