\subsection{Domain-Specific Knowledge Discovery}

Knowledge Graph Embeddings (KGEs) have emerged as a powerful paradigm for facilitating profound knowledge discovery and reasoning within specialized domains, where the intricate interplay of heterogeneous data and complex relationships often obscures valuable insights. Unlike traditional link prediction, which focuses on completing existing graph structures, KGEs applied to knowledge discovery aim to uncover novel, non-obvious relationships, generate testable hypotheses, and measure abstract conceptual proximities that drive scientific and industrial advancements \cite{portisch20221rd}. These applications demonstrate how KGEs, often integrated with other artificial intelligence (AI) techniques, can transform raw data into actionable intelligence, supporting decision-making in complex, domain-specific contexts.

The biomedical domain stands out as a prime area for KGE-driven knowledge discovery due to its vast, heterogeneous, and constantly evolving data landscape \cite{mohamed2020}. Early efforts in drug discovery leveraged KGEs to systematically identify potential drug-disease associations by mining biomedical literature. For instance, \cite{sosa2019ih0} applied KGEs to the Global Network of Biomedical Relationships (GNBR) to identify drug repurposing opportunities for rare diseases, explicitly modeling the uncertainty inherent in literature-derived relationships and achieving high performance in predicting known drug indications. Similarly, \cite{sang2019gjl} introduced GrEDeL, a KGE-based recurrent neural network method that not only discovers candidate drugs from biomedical literature but also provides corresponding mechanisms of action, thereby offering explainable insights into drug efficacy.

A significant advancement in this area was pioneered by \cite{zhu2022}, who constructed Specific Disease Knowledge Graphs (SDKGs) to enhance knowledge discovery for particular diseases. Their novel multimodal reasoning approach, employing reverse-hyperplane projection, effectively integrated structural, categorical, and descriptive embeddings. This allowed for a more comprehensive understanding of disease mechanisms and facilitated drug repurposing by uncovering latent connections between drugs, targets, and disease pathways that would be difficult to discern through unimodal analysis. Building upon this, \cite{islam2023} addressed the critical issues of high false positive rates and the lack of molecular-level validation in traditional drug repurposing. They proposed a novel ensemble KGE approach for COVID-19 that combined multiple complementary models for robust representations. Crucially, their method integrated molecular docking and ligand structural similarity for molecular-level validation, alongside providing rule-based explanations extracted from the KG, significantly enhancing the transparency and reliability of drug recommendations in a high-stakes clinical context. These biomedical applications collectively illustrate a progression from identifying simple associations to generating validated, explainable hypotheses, crucial for therapeutic development.

Beyond biomedicine, KGEs have proven instrumental in understanding complex innovation ecosystems. In the patent domain, \cite{li2022} constructed 'PatNet', a large-scale heterogeneous knowledge graph from US patent metadata comprising millions of entities (patents, inventors, assignees, technological groups, and subsections) and over a hundred million links. By applying various KGE models (e.g., TransE, RESCAL, ComplEx) to PatNet, they were able to operationalize and measure complex, heterogeneous knowledge proximity. This goes beyond simple link prediction, allowing for a nuanced understanding of technological landscapes, innovation trends, and the strategic positioning of entities within the patent ecosystem. The ability to embed diverse entities into a unified vector space enabled the identification of latent connections between disparate technological domains and the prediction of future technological trajectories, which is crucial for R\&D strategy and policy-making.

The utility of KGEs for domain-specific discovery extends to environmental science, exemplified by ecotoxicology. \cite{myklebust201941l} explored KGEs for ecotoxicological effect prediction, a task that traditionally requires extensive experimental effort. They constructed a knowledge graph integrating species taxonomy, chemical classification, chemical similarity, and publicly available effect data using ontology alignment. By applying KGEs, they were able to predict novel ecotoxicological effects of chemical compounds on species, demonstrating an improvement over selected baselines. This highlights KGEs' potential to reduce experimental burden and accelerate environmental risk assessment by inferring hidden relationships from diverse, structured data.

In summary, the evolution of KGE applications in domain-specific knowledge discovery showcases a clear trajectory towards highly specialized, multimodal, and explainable reasoning systems. These works collectively demonstrate KGE's profound utility in integrating diverse data modalities, handling complex relational patterns, and providing actionable insights. The underlying principle across these disparate domains is the KGE's ability to learn dense, low-dimensional representations that capture the latent semantic structure of knowledge graphs, thereby enabling the inference of novel relationships, the measurement of conceptual distances, and the generation of testable hypotheses. Future directions will likely involve further integration of KGEs with advanced AI paradigms, including large language models, to handle even more dynamic and complex domain knowledge, pushing the boundaries of automated scientific discovery and decision support.