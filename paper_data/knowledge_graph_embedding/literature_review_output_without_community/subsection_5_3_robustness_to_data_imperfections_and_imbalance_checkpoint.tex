\subsection{Robustness to Data Imperfections and Imbalance}

Real-world Knowledge Graphs (KGs) are inherently imperfect, characterized by long-tail distributions, missing information, and erroneous triples, posing significant challenges for Knowledge Graph Embedding (KGE) models. Addressing these data imperfections is critical for the reliability and generalization capabilities of KGEs. Early efforts to enhance robustness primarily focused on refining the training process, particularly through negative sampling strategies. For instance, \cite{wang2014} introduced **Bernoulli Negative Sampling** within the TransH framework, which adaptively corrupts head or tail entities based on relation mapping properties (e.g., one-to-many), thereby reducing the generation of false negative samples and improving training stability against KG incompleteness. Building on this, \cite{sun2018} further advanced the training robustness with **Self-Adversarial Negative Sampling** in their RotatE model, generating more informative and challenging negative examples by sampling based on the current model's scores. This technique implicitly helps models learn more robust representations, especially for less frequent entities or relations. The importance and evolution of such training-time robustness mechanisms are comprehensively reviewed by \cite{madushanka2024}, which systematically categorizes and analyzes various negative sampling methods, underscoring their foundational role in mitigating the impact of incomplete data.

While negative sampling addresses the implicit incompleteness of KGs, other imperfections like severe data imbalance and explicit errors require more targeted solutions. A pervasive issue in real-world KGs is the long-tail distribution, where a small number of entities and relations are highly frequent, while the vast majority are infrequent, leading to undertrained embeddings for the latter. To tackle this, \cite{zhang2023} proposed **Weighted KGE (WeightE)**, a novel approach that employs a bilevel optimization framework to adaptively assign higher weights to infrequent entities and relations during training. By dynamically adjusting the contribution of each triple, WeightE effectively mitigates the data imbalance problem, ensuring that long-tail elements receive sufficient learning signals and thus more reliable representations. This adaptive weighting strategy represents a significant step towards making KGE models robust to skewed data distributions.

Beyond structural and frequency-based imperfections, the semantic richness of KGs can also be underutilized, especially for entities with sparse connections. The **TaKE framework** by \cite{he2023} addresses this by integrating entity type information to implicitly capture semantic features and improve robustness. TaKE is a model-agnostic framework that learns type features automatically from the graph structure and projects an entity's type representation onto different relation-specific hyperplanes, capturing diverse type features without requiring explicit type supervision. This implicit semantic enrichment enhances the representation quality, particularly for entities that might otherwise suffer from limited structural context, thereby improving overall robustness. Furthermore, TaKE introduces a new type-constrained negative sampling strategy, which generates more effective negative samples by leveraging this implicit type knowledge, further contributing to a more robust training process.

Finally, the presence of inherent errors or noisy triples in KGs is a critical challenge that can severely degrade KGE performance. To address this, \cite{zhang2024} introduced **Error-Aware Knowledge Graph Embedding (AEKE)**, a framework designed to explicitly mitigate the impact of erroneous data. AEKE leverages entity attributes and hypergraphs to calculate joint confidence scores for triples. These confidence scores are then adaptively used to weight the contribution of each triple during the embedding learning process. By assigning lower weights to potentially erroneous triples, AEKE effectively reduces their detrimental influence on the learned embeddings, leading to more robust and accurate representations. This approach marks a crucial advancement, moving KGE models from passively accepting data as-is to actively discerning and down-weighting unreliable information.

In conclusion, the evolution of KGE research demonstrates a clear trajectory towards building models that are increasingly robust to the multifaceted imperfections of real-world data. From foundational advancements in negative sampling to sophisticated frameworks like WeightE, TaKE, and AEKE, the field has developed diverse strategies to counteract long-tail distributions, leverage implicit semantic cues, and explicitly handle erroneous information. Despite these significant strides, challenges remain in developing unified frameworks that can simultaneously address all forms of data imperfections in a scalable and efficient manner, especially in dynamic and evolving KGs. Future research may focus on integrating these robustness mechanisms into a more holistic and adaptive learning paradigm, potentially leveraging meta-learning or self-supervised approaches to continually refine data quality and model resilience.