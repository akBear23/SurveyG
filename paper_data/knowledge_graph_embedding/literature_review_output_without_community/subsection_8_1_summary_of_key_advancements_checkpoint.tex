\subsection*{Summary of Key Advancements}

The evolution of knowledge graph embedding (KGE) research reflects a profound journey from rudimentary structural models to highly sophisticated architectures, driven by a continuous pursuit of enhanced expressiveness, scalability, and robustness. This subsection synthesizes the most significant conceptual and methodological advancements, illustrating how the field has matured to address the increasingly complex challenges inherent in knowledge representation and reasoning.

Initially, KGE research established foundational geometric intuitions, primarily through translational models such as TransE \cite{TransE}. While groundbreaking for their simplicity, these models quickly encountered limitations in capturing the diverse and intricate nature of real-world relations, particularly symmetric, antisymmetric, and compositional patterns, as highlighted by comparative analyses \cite{rossi2020}. This inherent expressiveness bottleneck necessitated a paradigm shift towards more intricate geometric and algebraic transformations. Semantic matching models like ComplEx \cite{ComplEx} extended representational capacity by utilizing complex-valued embeddings, effectively distinguishing between asymmetric relations. Further, RotatE \cite{RotatE} introduced the elegant concept of relations as rotations in a complex vector space, providing a unified framework to model a wide spectrum of relational patterns with theoretical elegance and empirical performance. The field also explored higher-dimensional algebraic structures, such as quaternion embeddings (e.g., ConQuatE, discussed in Sections 2.3 and 3.3), offering richer representational power for multi-structural and polysemous entities. This progression aligns with the broader trend identified by \cite{ge2023} of combining various geometric transformations to enhance KGE model performance and capture complex relational patterns, a development that moved KGE models beyond simple Euclidean spaces into more adaptable "Geometric Models" as categorized by \cite{rossi2020}. Deep theoretical analyses (Section 3.5) further underpinned these advancements, providing a rigorous understanding of KGE expressiveness and guiding the design of more robust architectures.

A pivotal advancement has been the integration of deep learning architectures, fundamentally transforming how KGE models leverage contextual information and discern local graph patterns. The limitations of models that treated triples in isolation became apparent, leading to the adoption of techniques that could aggregate information from an entity's neighborhood. Convolutional Neural Networks (CNNs), exemplified by ConvE \cite{ConvE}, marked an early step by applying filters to reshaped embeddings, extracting richer interaction features beyond simple distance calculations. However, the true conceptual leap came with Graph Neural Networks (GNNs), such as Relational Graph Convolutional Networks (R-GCNs) \cite{R-GCN}. R-GCNs enabled entities to learn context-aware representations by aggregating information from their multi-hop relational neighborhoods, applying distinct transformations for each relation type. This marked a critical departure from modeling isolated triples, allowing representations to be dynamically informed by their local graph structure and significantly improving performance on reasoning tasks requiring broader evidence \cite{rossi2020}. This class of "Deep Learning Models" \cite{rossi2020} continued to evolve with the adaptation of Transformer architectures (e.g., Knowformer, TGformer from Section 3.2) to capture more global contextual dependencies and multi-structural features. Furthermore, the development of automated message function search for GNNs (from Section 5.2) has enabled models to dynamically adapt their aggregation mechanisms to diverse KG forms (e.g., n-ary, hyper-relational data), reducing manual engineering and enhancing data adaptability. This contextual awareness has been instrumental in enabling KGE for complex applications like question answering (Section 6.2) and recommendation systems (Section 6.3).

Beyond static and well-structured data, the field has demonstrated significant maturity in addressing the complexities of dynamic, spatiotemporal, and imperfect real-world knowledge. For dynamic knowledge graphs, models have progressed from basic temporal awareness, as seen in early approaches like HyTE (from Section 4.1) which explicitly incorporated timestamps, to sophisticated methods such as ATiSE (from Section 4.1) that model entity and relation evolution as multi-dimensional time series with Gaussian distributions, capturing temporal uncertainty. TeRo and ChronoR (from Section 4.1) further refined temporal modeling through element-wise rotations in complex space. Crucially, the challenge of continual learning in evolving KGs has been addressed by innovations like FastKGE with its Incremental Low-Rank Adapter (IncLoRA) mechanism (from Section 4.3), enabling efficient acquisition of new knowledge without catastrophic forgetting. Similarly, the integration of spatial and fuzzy dimensions has led to models like FSTRE (from Section 4.2), which comprehensively embed fuzziness, spatial, and temporal attributes within complex vector spaces, enabling more accurate predictions in uncertain, geographically relevant contexts. Robustness against data imperfections, such as long-tail distributions and errors, has been tackled by models like WeightE (from Section 5.3), which uses bilevel optimization to adaptively weight infrequent entities, and error-aware frameworks (AEKE from Section 5.3) that leverage entity attributes to mitigate erroneous data.

Furthermore, significant strides have been made in enhancing the practicality and scalability of KGE models for real-world deployment. This includes optimizing training efficiency through non-sampling frameworks (NS-KGE from Section 5.1) to avoid the instability of negative sampling, and developing scalable architectures that leverage parallel training techniques and efficient multi-GPU learning systems (e.g., GE2 from Section 5.2) to handle massive knowledge graphs. The advent of Federated Knowledge Graph Embedding (FKGE, from Section 5.4) represents a major conceptual shift towards privacy-preserving, distributed learning, addressing challenges of communication overhead and semantic disparities across clients. These operational advancements have been critical in transitioning KGE from theoretical models to reliable, high-performance solutions capable of handling massive, diverse, and imperfect knowledge graphs, thereby broadening their applicability across various domains (Section 6).

In summary, the trajectory of KGE has been characterized by a relentless drive towards greater sophistication and real-world applicability. From foundational geometric models, the field has embraced complex algebraic transformations and deep theoretical analyses to capture diverse relational semantics. The integration of deep learning architectures has infused KGE with contextual awareness, enabling models to learn intricate patterns from local and global graph structures. Moreover, the development of methods for dynamic, spatiotemporal, and imperfect data, alongside advancements in scalability and federated learning, signifies the field's profound maturation in addressing the multifaceted challenges of knowledge representation and reasoning, paving the way for more intelligent, robust, and ethical AI systems.