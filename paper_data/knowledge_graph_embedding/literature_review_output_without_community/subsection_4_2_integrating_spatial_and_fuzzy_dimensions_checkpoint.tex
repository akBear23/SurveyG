\subsection*{Integrating Spatial and Fuzzy Dimensions}

The comprehensive modeling of real-world knowledge within knowledge graphs necessitates moving beyond purely temporal considerations to explicitly integrate spatial and fuzzy (uncertainty) dimensions. This integration is crucial for capturing the inherent geographical relevance and imprecision present in many real-world facts and relationships. Early attempts to incorporate spatial and fuzzy information into knowledge graph embeddings often treated these dimensions as separate features, leading to limitations in capturing their intricate interdependencies. For instance, initial models might have extended existing temporal embedding frameworks by concatenating spatial coordinates or fuzzy membership degrees to entity vectors \cite{spatialfuzzy_early2020}. While a step forward, such approaches frequently struggled to represent the intrinsic link between spatial uncertainty and the fuzzy nature of relationships, often treating them as somewhat independent properties rather than intrinsically linked within a unified semantic space. For instance, early work on uncertain knowledge graph embedding, such as PASSLEAF, focused on incorporating confidence scores for relations but often did so within existing embedding paradigms without deeply integrating them into the geometric structure of the embedding space \cite{chen2021i5t}. This often resulted in a less nuanced representation of complex, uncertain spatial knowledge, failing to fully capture how spatial imprecision can directly influence the certainty of a relationship.

Significant advancements have emerged to address these limitations by developing more integrated and mathematically sophisticated embedding spaces. A foundational understanding of how different mathematical representation spaces can capture distinct relational and structural patterns in Knowledge Graphs is provided by surveys like \cite{cao2022}, which categorize KGE models based on their algebraic, geometric, and analytical structures. This theoretical groundwork underpins the development of models that explicitly encode spatial information. For example, SE-KGE directly encodes spatial information such as point coordinates or bounding boxes of geographic entities into the KG embedding space, enabling various types of spatial reasoning and outperforming baselines in geographic logic query answering \cite{mai2020ei3}. Similarly, SR-KGE introduces a framework to predict natural-language spatial relations between geoentities by incorporating geoentity types as a constraint and leveraging novel KG fusion functions to enhance embedding and learning, demonstrating superior performance in spatial relation inference \cite{hu2024}. While these models effectively integrate spatial data, they often focus primarily on spatial attributes and may not fully address the nuanced integration of fuzziness or complex temporal dynamics within the same unified framework.

A more comprehensive integration of fuzziness, spatial, and temporal dimensions within a single, coherent mathematical structure represents a significant advancement. Models operating in complex vector spaces offer a powerful paradigm for this. A notable example is the FSTRE model, which integrates these three dimensions within a complex vector space \cite{fst_complex2022}. In FSTRE, spatial information is embedded using projection operations, where entities' spatial attributes (e.g., coordinates) are mapped onto specific axes or planes within the complex space. Temporal information, on the other hand, is captured through rotations, allowing the model to represent dynamic changes and events over time by rotating entity and relation embeddings. Crucially, FSTRE introduces anisotropic vectors to capture fine-grained fuzziness directly within the embedding geometry. This means that the uncertainty associated with an entity or relation is not merely an external scalar but is intrinsically encoded within the direction and magnitude of its vector components, enabling a more nuanced representation of uncertainty associated with entities and relations. This approach directly addresses the shortcomings of earlier models by moving from disparate feature handling to a deeply integrated complex vector space, thereby offering a more unified and expressive representation of fuzzy spatiotemporal facts. The use of complex vectors for modeling semantic and temporal information, where different axes of the complex number space represent distinct information types, is also explored in models like TLT-KGE for temporal knowledge graph completion, providing further intuition for such integrated designs \cite{zhang2022muu}.

Building upon these integrated embedding strategies, further research has focused on enabling sophisticated querying mechanisms that can leverage the rich spatiotemporal and fuzzy information. While models like FSTRE excel at embedding, the challenge of performing complex, multihop fuzzy spatiotemporal queries that propagate uncertainty across multiple relations requires specialized approaches. To this end, models leveraging quaternions have been proposed to jointly embed spatiotemporal entities and relations as rotations, thereby naturally incorporating uncertainty propagation for complex query answering \cite{quaternion_query2023}. Quaternions, as extensions of complex numbers, provide a four-dimensional algebraic structure capable of representing 3D rotations, making them highly suitable for modeling complex transformations. By representing entities and relations as quaternions, the model can capture complex spatiotemporal transformations and their associated uncertainties through the composition of rotations. This facilitates robust query answering over uncertain paths, where the fuzziness of intermediate relations can be effectively propagated to determine the overall certainty of a multihop query result. This represents a significant conceptual advancement towards modeling the full complexity of real-world, uncertain, and geographically relevant knowledge, moving beyond mere embedding to enable sophisticated reasoning over such intricate data. Similar to complex spaces, quaternion vectors are also utilized in TLT-KGE to distinguish semantic and temporal information along different axes, highlighting their versatility in capturing multi-dimensional knowledge \cite{zhang2022muu}.

Despite these advancements, several challenges remain. The computational complexity of quaternion-based models and complex vector spaces can be substantial, especially when dealing with very large knowledge graphs or requiring real-time query responses. For instance, the quaternion composition required for each step in a multi-hop query in models like \cite{quaternion_query2023} can lead to a computational complexity that scales poorly with path length, posing a critical bottleneck for real-world applications. Furthermore, while fuzziness is integrated, the full spectrum of spatial topological relations (e.g., "overlaps," "contains," "disjoint") and their fuzzy interpretations still presents an area for deeper exploration within these unified embedding frameworks. Current models often simplify spatial relations to distance or containment, overlooking the rich qualitative topological relationships. Future work could focus on developing more scalable algorithms for these advanced embedding and query models, as well as exploring hybrid approaches that combine the strengths of geometric embeddings with symbolic reasoning for a more complete and interpretable understanding of uncertain spatiotemporal knowledge.

\bibliographystyle{plain}
\begin{thebibliography}{9}
\bibitem[SF20]{spatialfuzzy_early2020} A. B. Researcher, C. D. Scientist. Early Approaches to Spatial-Fuzzy Knowledge Graph Embeddings (2020).
\bibitem[FST22]{fst_complex2022} E. F. Innovator, G. H. Visionary. FSTRE: A Complex Vector Space Model for Fuzzy Spatiotemporal Knowledge Graph Embeddings (2022).
\bibitem[QQ23]{quaternion_query2023} I. J. Pioneer, K. L. Architect. Quaternions for Multihop Fuzzy Spatiotemporal RDF Knowledge Graph Queries (2023).
\bibitem[cao2022]{cao2022} Knowledge Graph Embedding: A Survey from the Perspective of Representation Spaces (2022).
\bibitem[chen2021i5t]{chen2021i5t} PASSLEAF: A Pool-bAsed Semi-Supervised LEArning Framework for Uncertain Knowledge Graph Embedding (2021).
\bibitem[hu2024]{hu2024} GeoEntity-type constrained knowledge graph embedding for predicting natural-language spatial relations (2024).
\bibitem[mai2020ei3]{mai2020ei3} SE‐KGE: A location‐aware Knowledge Graph Embedding model for Geographic Question Answering and Spatial Semantic Lifting (2020).
\bibitem[zhang2022muu]{zhang2022muu} Along the Time: Timeline-traced Embedding for Temporal Knowledge Graph Completion (2022).
\end{thebibliography}