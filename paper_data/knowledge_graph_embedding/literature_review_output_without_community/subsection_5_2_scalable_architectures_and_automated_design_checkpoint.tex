\subsection{Scalable Architectures and Automated Design}
The escalating scale and inherent diversity of real-world knowledge graphs (KGs) present formidable challenges for Knowledge Graph Embedding (KGE) systems, necessitating continuous innovation in scalable architectures and automated design paradigms. While foundational frameworks like PyTorch-BigGraph and DGL-KE initially enabled KGE training on billion-edge graphs, recent research has focused on more fine-grained optimizations in parallelization strategies, automated architecture design, and specialized GNN scaling to push performance boundaries further.

A fundamental aspect of handling large KGs is efficient training. \cite{kochsiek2021} empirically investigated and re-implemented various parallelization techniques, proposing a simple yet effective variation of stratification to mitigate negative impacts on embedding quality. Their work demonstrated that basic random partitioning, when coupled with suitable sampling strategies, can be highly effective for accelerating KGE training. However, while simple and efficient, such data-agnostic partitioning might struggle with highly skewed or heterogeneous KGs, where structural dependencies are critical. This highlights a trade-off between computational simplicity and the potential for information loss or suboptimal performance on complex graph structures. Complementing this, \cite{mohamed2021dwg} underscored that beyond model architecture, factors like loss functions, hyperparameters, and negative sampling strategies significantly impact model efficiency and scalability, an area often overlooked. This suggests that even with robust parallelization, sub-optimal training components can limit overall performance.

Addressing the challenge of adapting KGE models to the diverse structural forms of KGs, \cite{di2023} introduced a significant innovation: automated message function search for Graph Neural Networks (GNNs). This approach moves beyond fixed message function designs by allowing *both structures and operators* to be searched within a unified framework. This enables GNN-based KGE models to adapt to various KG forms, including n-ary and hyper-relational graphs, without extensive manual engineering. While demonstrating leading performance across diverse benchmarks, the computational overhead associated with automated architecture search remains a practical consideration, often requiring substantial computational resources for the search phase itself. This trade-off between automation and resource intensity is a common challenge in AutoML. Furthering this adaptive design, \cite{shang2024} proposed MGTCA, which integrates a Trainable Convolutional Attention Network (TCAN) that autonomously switches between GNN types (GCN, GAT, and a novel KGCAT) and learns attention for local structures. This adaptive GNN design, combined with a mixed geometry message function, offers another avenue for models to inherently adapt to varying local KG structures, reducing the need for manual pre-validation. In a related vein, \cite{zhang2022fpm} tackled the efficiency of hyper-parameter search for KGE, proposing KGTuner, a two-stage algorithm that efficiently explores configurations on subgraphs before fine-tuning on the full graph. This work demonstrates that automating and optimizing the search for crucial training parameters can yield significant performance improvements within practical time budgets, reinforcing the broader trend towards automated design.

For scalable GNN-based KGE, which often incurs significant computational and memory costs due to inherent data dependencies, specialized partitioning and system-level optimizations are crucial. \cite{modak2024} proposed CPa-WAC, a framework designed to overcome the accuracy reduction typically associated with KG partitioning. Their novel topology-preserving partitioning algorithm, CPa, minimizes cross-cluster links, while WAC, an improved compositional GCN, processes these partitions. A Global Decoder then effectively combines cluster-specific embeddings for global inference, achieving substantial speed-ups without accuracy loss. However, the complexity introduced by topology-preserving partitioning and the global decoder might present new bottlenecks or approximations for extremely large or dynamic graphs. A comparative approach by \cite{sheikh202245c} also addressed scaling GNN-based KGE for link prediction, proposing "self-sufficient partitions" and "constraint-based negative sampling" to achieve significant speedups (up to 16x) while maintaining performance. This highlights that different partitioning strategies and sampling techniques offer varying trade-offs between computational efficiency, topological preservation, and implementation complexity.

Complementing architectural and partitioning advancements, \cite{zheng2024} introduced GE2, a general and efficient KGE learning system specifically designed for multi-GPU environments. GE2 addresses long CPU times and high CPU-GPU communication overhead by offloading computationally intensive operations to the GPU and proposing the novel COVER algorithm for efficient multi-GPU data swap. This significantly optimizes negative sampling and data management, achieving substantial training speedups (2x to 7.5x) while maintaining model quality. While highly effective, GE2's performance is inherently tied to specific hardware configurations and interconnect speeds, a factor that warrants further investigation for broader applicability. The emphasis on optimizing negative sampling in GE2 aligns with \cite{mohamed2021dwg}'s findings on its critical impact. Furthermore, efforts to create lightweight and efficient KGE architectures, such as LightCAKE \cite{ning20219et} and AcrE \cite{ren2020}, demonstrate alternative approaches to scalability by focusing on parameter efficiency and optimized convolutional operations, rather than solely on distributed training.

In summary, the field is rapidly advancing towards KGE solutions that can robustly handle the immense scale and structural complexity of real-world knowledge graphs. This progression involves a multi-faceted approach, encompassing foundational parallel training strategies \cite{kochsiek2021}, automated model and hyperparameter design for diverse KG structures \cite{di2023, zhang2022fpm, shang2024}, specialized partitioning algorithms for GNNs \cite{modak2024, sheikh202245c}, and highly optimized multi-GPU systems \cite{zheng2024}. Future directions will likely focus on integrating these automated design principles with dynamic and streaming KGs, further enhancing the efficiency and adaptability of KGE systems in ever-evolving knowledge environments, while critically evaluating the computational costs and generalizability of these advanced techniques.