\subsection{Recommendation Systems}
Recommendation systems are pivotal in navigating the vast landscape of available information, guiding users to pertinent items, services, or content. Knowledge Graph Embeddings (KGE) have emerged as a transformative paradigm to enhance these systems, offering a robust solution to long-standing challenges such as data sparsity, cold-start problems, and the inherent lack of interpretability in traditional models \cite{liu2019e1u, kartheek2021aj7}. By enriching item and user representations with explicit semantic information derived from knowledge graphs (KGs), KGE-based approaches move beyond laborious manual feature engineering towards automatically learned, dense, low-dimensional vectors that capture complex semantic relationships.

Early applications of KGE in recommendation systems primarily focused on leveraging the rich semantic associations within KGs to improve the quality of item and user representations. These methods aimed to automatically learn embeddings that encapsulated the underlying meaning of entities and relations, thereby providing a more nuanced understanding of user preferences and item characteristics. For instance, \textcite{kartheek2021aj7} demonstrated the efficacy of KGE for constructing semantic-based recommenders by framing the task as link prediction within a knowledge graph. Their work highlighted how factorization-based scoring functions, such as HolE and DistMult, could generate more explicable recommendations and effectively alleviate cold-start and sparsity issues, which are common pitfalls for collaborative filtering methods. Similarly, \textcite{ni2020ruj} proposed a layered graph embedding framework for entity recommendation, utilizing Wikipedia's structured information to learn complementary entity representations from both topological and content-based features. This approach underscored the power of KGE in capturing diverse semantic features for recommending related entities, leading to improved quality and user engagement. These foundational efforts established KGE's capability to enrich profiles and yield more accurate, semantically informed recommendations.

The field subsequently advanced by deeply leveraging the structural information and relational paths within KGs, moving beyond mere entity embeddings to model complex interactions and multi-hop relationships. Early conceptual innovations, such as recurrent KGE (RKGE), sought to automate the learning of path semantics from KGs, thereby capturing the meaning of sequences of relations between entities without manual intervention. This paved the way for more sophisticated graph-based neural architectures. Graph Neural Networks (GNNs) have since become a dominant paradigm, effectively realizing the potential of path-based reasoning by propagating information across multi-hop connections. GNN-based models, such as Graph Convolutional Networks (GCNs), learn embeddings by iteratively aggregating information from an entity's neighbors, allowing them to capture higher-order structural patterns that are crucial for understanding user-item interactions. For example, \textcite{pham20243mh} proposed IDGCN, a knowledge graph embedding model integrated with a Graph Convolutional Network for context-aware recommendation systems. This model explicitly considers all user and item-based relationships to detect intricate connections, demonstrating improved efficacy in personalizing recommendations by modeling the relationships between entities more comprehensively than traditional KGEs. This class of models significantly enhanced the ability to capture the nuanced, multi-relational context that influences user preferences.

Further advancements integrated contextual and temporal dynamics into KGE for recommendations. \textcite{mezni20218ml} introduced a context-aware service recommendation system that utilized Dilated Recurrent Neural Networks to embed a Context-Aware Service Knowledge Graph (C-SKG). This method effectively captured multi-relational interactions between users and services within varying contexts, based on first-order and subgraph-aware proximity, demonstrating notable improvements in accuracy and scalability. Addressing the evolving nature of user preferences and item popularity, \textcite{mezni2021ezn} proposed a temporal KGE approach for service recommendation. By constructing a Temporal Service Knowledge Graph (TSKG) and employing Convolutional Neural Networks (CNNs) for embedding, their model successfully captured the dynamic changes in user tastes and service popularity over time, outperforming static and time-unaware KG-based systems. These developments highlighted the increasing sophistication in leveraging the full structural, contextual, and temporal richness of KGs.

A particularly significant advancement has been the development of contextualized KGE for explainable recommendations, driven by the growing demand for transparency in AI systems. These models aim to provide not only accurate predictions but also clear, interpretable reasons for their outputs. \textcite{yang2023} introduced a Contextualized Knowledge Graph Embedding (CKGE) framework specifically designed for explainable talent training course recommendations. This framework addresses the critical need for transparency and the challenge of accounting for diverse user learning motivations. \textcite{yang2023} developed a novel KG-based Transformer that captures motivation-aware information by processing serialized KG structures, incorporating relational attention and structural encoding. A key innovation is its local path mask prediction mechanism, which quantifies and highlights the saliency of meta-paths within the knowledge graph, thereby providing explicit, path-based explanations for recommendations. This approach moves beyond simple similarity-based recommendations to offer transparent, motivation-driven insights, significantly enhancing user trust and system utility.

Furthermore, KGE has proven invaluable in tackling the complex challenge of cross-domain recommendation and mitigating cold-start problems. Traditional recommendation systems often falter when faced with new users or items (cold-start) or when attempting to transfer knowledge across disparate domains. By providing rich, generalizable item and user embeddings, KGE techniques inherently broaden the applicability of recommendation systems to scenarios with sparse data or new items. A notable contribution in this area is the work by \textcite{liu2023}, who proposed a multi-domain item-item (I2I) recommendation approach based on cross-domain knowledge graph embedding. This method explicitly analyzes both homo-domain item associations and hetero-domain item interactions within a rich knowledge graph. They introduced a novel "cross-domain knowledge graph chiasmal embedding approach" to efficiently interact all items across multiple domains and a "binding rule" to facilitate both homo-domain and hetero-domain item embeddings. By framing multi-domain I2I recommendation as a link prediction problem within the knowledge graph, \textcite{liu2023} demonstrated superior performance in both link prediction and multi-domain recommendation results, effectively addressing the cross-domain cold-start problem and enabling comprehensive multi-domain recommendations.

In conclusion, KGE has profoundly transformed recommendation systems by enabling the automatic learning of rich, semantic representations from knowledge graphs, moving beyond the limitations of manual feature engineering and addressing critical issues like sparsity and cold-start. The evolution spans from early applications that enriched item features to sophisticated models that leverage complex relational paths through GNNs and capture temporal dynamics. The development of contextualized KGE frameworks, particularly those providing explainable, motivation-aware recommendations \cite{yang2023}, marks a significant step towards more transparent and trustworthy AI. Crucially, KGE has also provided robust solutions for complex multi-domain scenarios and cold-start problems \cite{liu2023}, significantly broadening the applicability and effectiveness of recommendation systems. The trajectory of KGE in recommendation systems reflects a continuous drive towards more intelligent, adaptable, and user-centric recommendation engines, with future research likely focusing on deeper integration of multimodal contextual information, more advanced graph neural architectures, and further enhancements in explainability and cross-domain transferability.