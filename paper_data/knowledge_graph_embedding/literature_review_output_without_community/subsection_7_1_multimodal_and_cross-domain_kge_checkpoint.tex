\subsection{Multimodal and Cross-Domain KGE}

Traditional Knowledge Graph Embedding (KGE) models, predominantly relying on structured triples, face inherent limitations in capturing the multifaceted complexity of real-world knowledge, which is often expressed through diverse modalities and spans multiple domains. This necessitates a crucial future direction for KGE research: the integration of heterogeneous data modalities, such as text, images, video, and audio, to construct richer and more comprehensive entity representations. Concurrently, leveraging knowledge graphs across different domains is paramount for alleviating cold-start problems, enhancing knowledge transfer, and enabling sophisticated applications like multi-domain recommendations. This trend aims to overcome the limitations of unimodal or single-domain KGEs, moving towards a more holistic understanding of knowledge that mirrors the complexity and interconnectedness of real-world information.

\subsubsection{Advancements and Open Challenges in Multimodal Knowledge Graph Embedding}
The core challenge in multimodal KGE (MMKGE) lies in effectively fusing heterogeneous data types—each with its own representation space and semantic characteristics—into a unified, coherent embedding space. Early approaches often treated different data types as mere additional attributes, but the field is progressing towards a deeper, integrated understanding that captures cross-modal interactions.

Significant methodological advancements include the development of Graph Neural Network (GNN)-based architectures specifically designed for multimodal KGs. For instance, \cite{liang202338l} introduced the Hyper-node Relational Graph Attention Network (HRGAT), a novel GNN that explicitly combines diverse modal information with graph structural information. HRGAT treats entities as "hyper-nodes" that can aggregate features from various modalities (e.g., text descriptions, image features) alongside their relational context, enabling the GNN's message-passing mechanism to incorporate and reconcile information from different sources. This leads to richer and more contextually aware entity embeddings. Another approach, demonstrated by \cite{zhu2022} in the biomedical domain, employs a novel *reverse-hyperplane projection* method for Specific Disease Knowledge Graphs (SDKGs), integrating structural, categorical, and descriptive embeddings to enhance knowledge inference. Such methods move beyond simple feature concatenation by designing sophisticated fusion mechanisms that learn cross-modal interactions.

However, MMKGE presents several open challenges. One critical area is the adaptation of fundamental KGE components, such as negative sampling, to multimodal settings. \cite{zhang2023} highlights that traditional negative sampling (NS) methods are often unsuitable and inefficient for MMKGE, as they typically perform "entity-level" replacement without explicitly considering the alignment of distinct modal embeddings (e.g., structural and visual). To address this, they propose Modality-Aware Negative Sampling (MANS), which includes a Visual Negative Sampling (MANS-V) component that samples only negative visual embeddings to explicitly guide modality alignment. This work underscores that even basic training strategies require re-evaluation for MMKGE, indicating the field's ongoing maturation.

Further challenges in MMKGE include:
\begin{itemize}
    \item \textbf{Robust and Scalable Fusion}: Developing fusion strategies that can effectively integrate an increasing number of modalities (e.g., video, audio) while handling their inherent noise, sparsity, and semantic gaps.
    \item \textbf{Asynchronous and Streaming Data}: Managing multimodal data that arrives asynchronously or in a streaming fashion, requiring adaptive and efficient update mechanisms.
    \item \textbf{Theoretical Guarantees}: Establishing theoretical frameworks to understand the information gain and potential biases introduced by cross-modal fusion, and to provide guarantees for the quality of fused representations.
    \item \textbf{Benchmark Datasets}: The scarcity of large-scale, high-quality multimodal KGs with diverse modalities remains a bottleneck for comprehensive evaluation and comparison of MMKGE models.
\item \textbf{Connecting to Grounded AI}: Leveraging MMKGE to advance grounded language understanding, where symbolic knowledge is firmly connected to perceptual experiences, enabling more human-like reasoning.
\end{itemize}

\subsubsection{Enabling Cross-Domain Knowledge Graph Embedding}
The "cross-domain" aspect of KGE focuses on leveraging knowledge from one domain to enhance understanding or performance in another, often disparate, domain. This is particularly crucial for addressing cold-start problems in new domains where data is scarce and for enabling sophisticated multi-domain applications, such as the cross-domain recommendation systems discussed in Section 6.3. The primary technical challenges include schema heterogeneity (different entity and relation types across KGs), entity/relation divergence (same concept, different representation), and the effective transfer of learned knowledge.

A foundational step towards cross-domain KGE is robust entity alignment, which identifies equivalent entities across different knowledge graphs (as detailed in Section 6.1). Methodologies like the semi-supervised entity alignment (SEA) by \cite{pei2019}, which accounts for entity degree differences through adversarial training, are instrumental in bridging semantic gaps between distinct KGs. However, entity alignment alone is often insufficient for comprehensive cross-domain knowledge transfer.

Beyond mere alignment, the future direction lies in developing more sophisticated domain adaptation and transfer learning strategies for KGE. \cite{eyharabide2021wx4} presents a novel method for domain adaptation based on KGE for musical instrument recognition, particularly in data-scarce cultural heritage collections. Their approach incorporates semantic vector spaces from KGE as a key ingredient to guide the domain adaptation process, combining them with visual embeddings to train a neural network. This demonstrates how KGE can serve as a powerful tool for transferring knowledge and adapting models to new domains, especially when direct data overlap is limited.

Despite these advancements, several critical challenges remain for truly effective cross-domain KGE:
\begin{itemize}
    \item \textbf{Generalizable Transfer Frameworks}: Developing frameworks that can efficiently transfer knowledge across *arbitrary* domains without extensive retraining or fine-tuning for each new domain pair.
    \item \textbf{Semantic Drift and Polysemy}: Effectively handling the nuanced changes in meaning (semantic drift) or multiple meanings (polysemy) of entities and relations when transferred between vastly different domains.
    \item \textbf{Zero-Shot and Few-Shot Adaptation}: Enabling knowledge transfer to entirely new domains with minimal or no labeled data, especially when entity or relation overlap is negligible. This requires models to infer mappings or commonalities from high-level semantic structures.
    \item \textbf{Transfer of Reasoning Capabilities}: Moving beyond transferring just entity/relation embeddings to transferring complex reasoning paths, logical rules, or inference patterns across domains.
    \item \textbf{Addressing Cold-Start in Complex Scenarios}: While progress has been made (e.g., in recommendations), generalizing cold-start solutions to more complex reasoning tasks across domains remains an open problem.
    \item \textbf{Connection to Lifelong Learning}: Integrating cross-domain KGE with lifelong learning paradigms, allowing AI systems to continually acquire and transfer knowledge across new tasks and domains without forgetting previously learned information.
\end{itemize}

In summary, the future of KGE increasingly lies in its ability to transcend the limitations of single-modality and single-domain data. By developing sophisticated methodologies for multimodal fusion and cross-domain knowledge transfer, researchers are moving towards creating KGEs that can capture a more holistic, interconnected, and nuanced understanding of real-world information. This shift promises to unlock new capabilities for AI systems, enabling them to operate more intelligently and robustly across complex, diverse, and data-rich environments.