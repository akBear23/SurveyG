\subsection*{Open Challenges and Ethical Considerations}

Despite remarkable progress in Knowledge Graph Embedding (KGE) research, the field continues to grapple with fundamental, persistent challenges that impede the robust, scalable, and ethically responsible deployment of KGE-powered AI systems. This subsection critically assesses these unresolved technical hurdles and delves into the crucial ethical implications, outlining essential areas for future investigation to ensure KGE techniques align with societal values.

A significant, overarching challenge remains **extreme scalability and continuous adaptation for dynamic Knowledge Graphs (KGs)**. While Section 6.3 discusses various solutions for improving KGE efficiency and handling dynamic updates, these often address specific facets rather than the holistic problem of KGs that are simultaneously massive, highly dynamic, and non-stationary. For instance, lightweight frameworks like LightKG \cite{wang2021} reduce storage and inference costs, and knowledge distillation methods like DualDE \cite{zhu2020} enable faster reasoning. Similarly, graph partitioning strategies such as CPa-WAC \cite{modak2024} enhance the scalability of GNN-based KGEs. However, these solutions frequently involve trade-offs, such as simplified representations or reliance on discrete updates, which may not suffice for KGs where topology and semantics evolve continuously and rapidly. Meta-learning approaches, exemplified by MetaHG \cite{sun2024} and \cite{mao2024v2s}, offer promise for adapting to incremental knowledge and emerging entities. Yet, the challenge of maintaining high-quality embeddings and efficient inference for KGs experiencing continuous, high-velocity changes across billions of entities and relations remains a grand research problem. The entire training pipeline, including loss functions, hyperparameters, and negative sampling strategies, requires holistic optimization to achieve true scalability and accuracy \cite{mohamed2021dwg}, rather than focusing solely on scoring functions.

Another critical area is the **robust handling of inherent uncertainty, incompleteness, and adversarial threats in real-world data**. KGs are intrinsically noisy, incomplete, and can be subject to manipulation. While methods like confidence-aware negative sampling \cite{shan2018} and reinforcement learning frameworks \cite{zhang2021} improve robustness against noise, and techniques like committee-based models \cite{choi2020} or logical rule integration \cite{guo2017, guo2020} address incompleteness, fundamental limitations persist. Real-world knowledge often contains fuzzy, conflicting, or inherently uncertain facts, as seen in complex spatiotemporal KGs \cite{ji2024}. Current KGE models frequently struggle to reliably quantify the certainty of their predictions, with studies showing that KGE models are often uncalibrated \cite{tabacof2019}. This lack of reliable confidence scores undermines the trustworthiness of KGE-powered systems in critical applications. Furthermore, KGE models are vulnerable to *data poisoning attacks* \cite{zhang20190zu}, where malicious actors can manipulate the plausibility of facts by injecting or deleting triples, posing a significant threat to the integrity and reliability of learned representations and downstream tasks. Developing KGEs that are inherently resilient to such adversarial manipulations and can robustly model epistemic uncertainty remains a profound challenge.

Beyond these technical limitations, KGE research faces critical **ethical implications** that demand proactive attention.
\begin{enumerate*}[label=(\roman*)]
    \item \textbf{Bias and Fairness in Learned Representations}: KGE models learn from existing KGs, which are often constructed from real-world data reflecting historical, societal, or data collection biases. If these biases are embedded in learned representations, they can be amplified when deployed in sensitive applications, leading to unfair or discriminatory outcomes. For instance, KGEs used in healthcare (e.g., SEConv \cite{yang2025}, multimodal reasoning for diseases \cite{zhu2022}) could perpetuate biases in medical records, affecting diagnoses or treatment recommendations for specific demographic groups. A specific technical challenge is *degree bias*, where entities with fewer connections (low-degree nodes) receive poorer representations, leading to performance disparities \cite{shomer2023imo}. While solutions like KG-Mixup \cite{shomer2023imo} mitigate degree bias in static KGs and FairDGE \cite{li2024wyh} addresses *structural fairness* in dynamic graphs, the broader challenge lies in developing comprehensive, domain-agnostic debiasing strategies that account for intersectional biases without compromising model utility. The field requires more robust methods to detect, quantify, and mitigate various forms of bias, ensuring equitable performance across diverse entity groups.
    \item \textbf{Data Privacy and Confidentiality Concerns}: Training KGE models often requires access to vast amounts of data, some of which may be sensitive or proprietary. Sharing such data or models trained on it can pose significant privacy risks. Federated KGE (FKGE) offers a promising direction by enabling collaborative learning from distributed KGs without centralizing raw data, as seen in FedS \cite{zhang2024}, personalized FKGE (PFedEG) \cite{zhang2024}, and cross-domain FKGE (FedCKE) \cite{huang2023grx}. However, FKGE is not immune to privacy threats. *Inference attacks* and *model inversion attacks* can still reveal sensitive information about clients' private KGs, even with distributed training \cite{hu20230kr}. Advanced defenses, such as differentially private FKGE (DP-Flames) \cite{hu20230kr}, aim to provide stronger privacy guarantees but often come with inherent *privacy-utility trade-offs*. The ongoing challenge is to develop robust privacy-preserving mechanisms that can effectively balance strong confidentiality with high model performance, especially in heterogeneous and dynamic federated environments.
    \item \textbf{Responsible Deployment of KGE-powered AI Systems}: KGEs are increasingly integrated into critical AI systems, from question answering in chemistry \cite{zhou2023} to fault diagnosis in aviation assembly \cite{liu2024q3q}. In such high-stakes applications, ensuring fairness, accountability, and transparency (FAT) is paramount. The aforementioned issues of uncalibrated predictions \cite{tabacof2019}, vulnerability to adversarial attacks \cite{zhang20190zu}, and embedded biases directly undermine the trustworthiness and safety of these systems. Responsible deployment necessitates moving beyond purely performance-driven metrics to a holistic consideration of ethical dimensions. This includes developing comprehensive frameworks for auditing KGE systems for bias, ensuring robust defenses against adversarial manipulation, and providing reliable confidence scores for predictions. The field must prioritize the development of KGEs that are not only powerful but also inherently trustworthy, fair, and beneficial for society, requiring interdisciplinary collaboration to establish ethical guidelines and regulatory standards.
\end{enumerate*}

In conclusion, while KGE research has made remarkable strides, significant challenges persist in achieving extreme scalability for dynamic KGs, ensuring robustness against inherent data imperfections and adversarial threats, and critically, addressing the profound ethical implications of bias, privacy, and responsible deployment. Future investigations must prioritize the development of KGE techniques that are not only performant but also inherently robust, fair, and interpretable (as discussed in Section 7.2), aligning with societal values and ensuring the trustworthy application of KGE-powered AI systems in an increasingly complex world.