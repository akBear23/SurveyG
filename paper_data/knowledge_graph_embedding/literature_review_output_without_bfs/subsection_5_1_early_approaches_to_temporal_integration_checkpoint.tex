\subsection{Early Approaches to Temporal Integration}

The dynamic nature of real-world knowledge necessitates that Knowledge Graph Embedding (KGE) models move beyond static representations to explicitly incorporate temporal information. Early research in temporal integration laid the foundational groundwork, exploring diverse methodologies to embed the temporal dimension directly into knowledge graphs, thereby enabling time-aware reasoning and prediction.

One of the pioneering geometric approaches was HyTE \cite{dasgupta2018}, which introduced a novel method for embedding temporal knowledge by associating each timestamp with a distinct hyperplane in the embedding space. This allowed HyTE to perform temporally-guided inference and predict the temporal scopes for relational facts, a crucial capability for incomplete knowledge graphs where temporal validity might be missing. In a different vein, tensor decomposition methods emerged as a powerful paradigm for inherently capturing time within KGEs \cite{lin2020}. This approach represented facts as higher-order tensors, explicitly including the time dimension alongside entities and relations, offering a generalizable framework for modeling dynamic knowledge. While these models provided initial explicit temporal handling, they often treated temporal evolution deterministically.

A significant departure from deterministic temporal modeling was introduced by ATiSE (Additive Time Series Embedding) \cite{xu2019}. ATiSE innovated by modeling the evolution of each entity and relation representation as a multi-dimensional additive time series, composed of trend, seasonal, and random components. Crucially, it represented entities and relations as multi-dimensional Gaussian distributions at each time step, explicitly accounting for temporal uncertainty during their evolution, a feature largely overlooked by prior models. This statistical perspective provided a more nuanced understanding of how knowledge changes over time.

More recently, TeAST (Temporal Knowledge Graph Embedding via Archimedean Spiral Timeline) \cite{li2023} presented a novel structural mapping for time. TeAST introduced the concept of an Archimedean spiral timeline for relations, which ensures that relations occurring simultaneously are placed on the same timeline and that all relations evolve in a structured manner. This approach transformed the quadruple completion problem into a 3rd-order tensor completion task, specifically designed to avoid direct entity evolution and enhance interpretability, building upon the idea of structured temporal representation with a creative geometric twist.

These early models collectively established diverse foundational methodologies for integrating temporal information into KGEs, ranging from geometric interpretations and tensor-based structural embeddings to time series analysis for uncertainty and novel timeline mappings. While they successfully moved beyond static representations, a shared limitation across some of these initial approaches, particularly the earlier ones, was their potential struggle with highly complex, non-linear temporal patterns or the scalability challenges associated with explicit temporal indexing or extensive tensor operations. These challenges highlighted the need for more sophisticated and efficient temporal modeling techniques, paving the way for subsequent advancements in the field.