\subsection{Spherical and Quaternion Embeddings for Diverse Semantics}

Traditional Knowledge Graph Embedding (KGE) models often represent entities as single points in Euclidean space, a simplification that can limit their capacity to capture the rich, multi-faceted semantic properties inherent in real-world knowledge graphs. This limitation becomes particularly apparent when modeling complex relations, entity extents, or multi-dimensional semantic relationships. To overcome these challenges, recent research has explored alternative, more expressive embedding spaces, notably spherical and quaternion representations, which leverage distinct mathematical properties to enhance KGE expressiveness and applicability to specialized tasks \cite{cao2022}.

The evolution of KGE models has seen a progression from simple vector operations to more sophisticated geometric and algebraic structures. Building upon the success of rotational models like RotatE \cite{sun2019}, which interprets relations as 2D rotations in complex vector space, quaternion embeddings offer a natural extension to higher dimensions. Quaternions, a 4D number system, provide a powerful algebraic framework for representing rotations in 3D and 4D space. The seminal work in this area is \textbf{QuatE} \cite{zhang2019quat}, which extends the concept of complex-valued embeddings by representing entities and relations as quaternions. QuatE models relations as rotations using the Hamilton product, allowing it to capture a wider array of relational patterns, including symmetry, anti-symmetry, inversion, and composition, within a unified 4D space. This higher-dimensional algebraic structure inherently provides greater expressiveness for intricate relational semantics compared to its 2D complex-valued counterparts.

Further advancements in quaternion embeddings demonstrate their versatility for diverse semantic challenges. \textbf{ConQuatE} \cite{chen2025} leverages quaternion rotations to address the critical "polysemy issue," where entities exhibit different semantic characteristics depending on their relational contexts. By efficiently incorporating contextual cues from various connected relations through quaternion transformations, ConQuatE enriches entity representations across multiple semantic dimensions without requiring auxiliary information, thereby improving link prediction performance. Similarly, for dynamic knowledge, \textbf{TimeLine-Traced KGE (TLT-KGE)} \cite{zhang2022muu} utilizes quaternion vectors to embed entities and relations with timestamps. TLT-KGE innovatively models semantic and temporal information as distinct axes within the quaternion space, enabling it to distinguish and connect these independent yet related aspects of temporal facts. This approach effectively addresses the challenge of representing evolving knowledge, outperforming state-of-the-art competitors in temporal knowledge graph completion. The broader utility of quaternion algebra in KGE is also highlighted by analyses that propose new multi-embedding models based on this framework \cite{tran20195x3}.

In a conceptually distinct but equally impactful direction, spherical embeddings move beyond point-based representations to model entities as geometric volumes. \textbf{SpherE} \cite{li2024} proposes embedding entities not as single points, but as \textit{spheres} in Euclidean space, each defined by a center vector and a radius, while relations are modeled as rotations. This novel representation allows SpherE to explicitly capture entity \textit{extents} and \textit{overlaps}, which is crucial for tasks like knowledge graph set retrieval where precise sets of answers are required. By representing entities with a "spread" rather than a singular location, SpherE robustly models one-to-many, many-to-one, and especially many-to-many relations, a long-standing challenge for traditional point-based models. The interpretability of the sphere's radius, which correlates with an entity's universality, further enhances its utility. Theoretically, SpherE demonstrates high expressiveness for various relation patterns and mapping properties by checking sphere intersection after rotation, providing a powerful mechanism to understand how entities relate in terms of their scope and potential intersections.

These advancements in spherical and quaternion embedding spaces represent a significant leap in Knowledge Graph Embedding expressiveness, moving beyond the inherent limitations of traditional vector representations. Quaternion embeddings, by extending rotational models to 4D algebraic structures, offer enhanced capabilities for capturing complex relational patterns and contextual semantics, particularly useful for polysemy and temporal dynamics. Spherical embeddings, on the other hand, introduce volumetric representations that explicitly model entity extents and overlaps, proving highly effective for set-based retrieval and intricate many-to-many relationships. As highlighted by \cite{cao2022}, the choice of representation space profoundly influences the types of KG properties that can be effectively modeled. By leveraging the unique algebraic properties of quaternions and the geometric advantages of spheres, these models provide novel and powerful ways to capture complex entity and relation properties, pushing the boundaries of KGE expressiveness and expanding their applicability to specialized tasks in diverse real-world applications. Future research may explore the integration of these advanced geometric and algebraic spaces with neural architectures, or develop adaptive strategies for selecting optimal embedding spaces based on specific KG characteristics and task requirements.