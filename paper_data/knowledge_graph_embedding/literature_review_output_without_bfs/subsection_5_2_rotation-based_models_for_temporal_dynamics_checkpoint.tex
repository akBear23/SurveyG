\subsection{Rotation-based Models for Temporal Dynamics}

Building on the success of rotation-based embedding models like RotatE in capturing diverse relation patterns within static knowledge graphs, a significant line of research has emerged that leverages rotations in complex or k-dimensional spaces to elegantly model temporal evolution. These models offer a highly expressive framework for representing time-dependent changes and their intricate interactions.

A foundational contribution in this area is \textcite{xu2020} TeRo (Temporal Rotation), which introduced the concept of temporal rotation in a complex vector space to model entity evolution. TeRo represents time-specific entity embeddings as element-wise rotations of their time-independent counterparts, allowing it to effectively capture diverse relation patterns, including temporary, asymmetric, and reflexive relations, and robustly handle various time annotations such as discrete time points and continuous time intervals through dual relation embeddings.

Building upon TeRo's pioneering work, \textcite{sadeghian2021} ChronoR (Chronological Rotation) generalized the rotation mechanism to k-dimensional spaces, offering a more flexible and powerful framework for temporal knowledge graph embedding. ChronoR proposed an inner product scoring function, which it theoretically demonstrated to generalize complex-domain models like ComplEx, thereby offering a robust alternative to Euclidean distance in high-dimensional spaces. Furthermore, ChronoR introduced advanced regularization techniques, including a tensor nuclear norm-inspired regularization and a 4-norm based temporal smoothness objective, to encourage consistent transformations for chronologically closer timestamps, enhancing model generalizability and capturing the smooth evolution of entities over time.

More recently, \textcite{ji2024} FSTRE (Fuzzy Spatiotemporal RDF Embedding) further extended this rotation-based paradigm by integrating fuzziness and spatial information alongside temporal rotation within a complex vector space. FSTRE addresses the challenge of modeling uncertain and dynamic knowledge by uniquely employing projection for spatial embedding and rotation for temporal embedding, while incorporating fine-grained fuzziness through the modal length of anisotropic vectors. This comprehensive approach demonstrates the versatility of geometric operations in complex space to represent multifaceted dynamics beyond just temporal evolution.

Collectively, these rotation-based models showcase high expressiveness for diverse relation patterns and temporal dynamics, providing an elegant and theoretically grounded method to capture time-dependent changes and their interactions within knowledge graphs. While their inherent complexity can increase with higher dimensions or the integration of additional factors like fuzziness and spatial information, they represent a powerful advancement in modeling the evolving nature of real-world knowledge. Future research could explore more adaptive rotation mechanisms, investigate their applicability to forecasting tasks, or integrate them with other advanced temporal modeling techniques to handle even more intricate forms of uncertainty and continuous time.