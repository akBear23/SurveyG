\subsection{Hierarchical and Multi-Scale Architectures}

The emergence of Vision Transformers (ViTs) has revolutionized computer vision, yet their application in dense prediction tasks has been hindered by challenges in computational efficiency and the ability to capture multi-scale features. Traditional ViTs often struggle with high computational costs and inadequate feature representation at varying scales, making them less suitable for tasks that require dense predictions, such as segmentation. This subsection reviews two significant advancements in hierarchical and multi-scale architectures: the Swin Transformer and the Pyramid Vision Transformer (PVT), highlighting their contributions to overcoming the limitations of conventional ViTs.

The Swin Transformer introduces a hierarchical structure combined with a novel shifted window attention mechanism, which effectively reduces the computational complexity to linear scale with respect to the input size. This architecture allows for the extraction of multi-scale features by progressively merging patches, facilitating the capture of both local and global context in images. The shifted window approach enables the model to maintain high-resolution representations while significantly lowering the computational burden, making it a versatile backbone for various vision tasks, including object detection and segmentation \cite{swin}. The Swin Transformer's ability to adaptively learn from different scales addresses the limitations of traditional ViTs, which often rely on fixed-size patches that may overlook critical contextual information.

Building on the need for efficient multi-scale feature extraction, the Pyramid Vision Transformer (PVT) presents a pyramid-like structure that generates multi-scale features without the use of convolutions. PVT employs a series of down-sampling layers that progressively reduce the spatial dimensions while increasing the feature depth, allowing the model to capture rich contextual information at multiple scales \cite{PVT}. This architecture is particularly adept at replacing convolutional neural networks (CNNs) in dense prediction tasks, as it retains the ability to model long-range dependencies while being computationally efficient. The PVT's design addresses the challenge of integrating multi-scale information, which is crucial for tasks such as semantic segmentation, where understanding both fine details and broader context is essential.

Further advancements in hierarchical architectures are evident in works that leverage the strengths of both CNNs and ViTs. For instance, the Swin Unet3D model exemplifies this hybrid approach by integrating 3D convolutional blocks with Swin Transformer blocks in a parallel architecture, allowing for simultaneous learning of global and local features in 3D medical images \cite{cai2023hji}. This model effectively mitigates the limitations of pure CNNs, which struggle with long-range dependencies, and pure ViTs, which may not capture local details effectively. The parallel integration of these two architectures enhances performance in challenging tasks like brain tumor segmentation, demonstrating the potential of hybrid models in addressing the limitations of traditional architectures.

Despite these advancements, unresolved issues remain in the pursuit of optimal performance and efficiency in Vision Transformers. Future research directions may focus on refining the balance between model complexity and accuracy, particularly in resource-constrained environments. Additionally, exploring novel training paradigms and quantization techniques could further enhance the practicality of hierarchical and multi-scale architectures in real-world applications. The ongoing evolution of Vision Transformers reflects a dynamic interplay between architectural innovation and the demands of diverse vision tasks, paving the way for more robust and efficient models in the future.
```