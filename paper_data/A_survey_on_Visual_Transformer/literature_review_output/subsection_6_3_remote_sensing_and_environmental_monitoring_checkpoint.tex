\subsection{Remote Sensing and Environmental Monitoring}

Remote sensing is indispensable for environmental monitoring, providing critical data for observing Earth's surface across vast geographical scales. Vision Transformers (ViTs) have emerged as a transformative technology in this domain, offering robust solutions for analyzing complex satellite and radar imagery, particularly for tasks like land use and land cover (LULC) classification and specialized object detection under challenging conditions. Their inherent ability to capture global contextual information across large geographical areas, coupled with their capacity to model long-range dependencies, makes them particularly well-suited for these tasks, often surpassing traditional convolutional neural network (CNN) methods in accuracy and robustness.

One significant application area is **object detection in Synthetic Aperture Radar (SAR) imagery**, which presents unique challenges due to speckle noise, strong scattering, and multi-scale objects against complex backgrounds. Traditional methods often struggle to maintain performance in such cluttered environments. To address this, \cite{zhao2023rle} introduced ST-YOLOA, a hybrid model that integrates the global context modeling capabilities of the Swin Transformer with the efficient YOLOX framework for real-time SAR ship detection. This approach leverages a novel STCNet backbone, an enhanced PANet with SE and CBAM attention for multi-scale feature fusion, and a decoupled detection head, demonstrating a notable accuracy improvement (e.g., 4.83\% over YOLOX on complex datasets) while maintaining real-time performance. Further enhancing SAR ship detection, \cite{li2022th8} proposed ESTDNet, an enhancement Swin Transformer detection network specifically designed to handle the strong scattering and multi-scale nature of ships in SAR images. ESTDNet employs a Feature Enhancement Swin Transformer (FESwin) module, which aggregates contextual information using CNNs before and after the Swin Transformer, and an Adjacent Feature Fusion (AFF) module to optimize feature pyramids, thereby improving spatial-to-channel feature expression and enhancing recognition and localization capabilities. These specialized ViT adaptations highlight their effectiveness in overcoming the inherent difficulties of SAR data.

In the realm of **Land Use and Land Cover (LULC) classification and semantic segmentation**, ViTs are particularly adept at processing multimodal remote sensing data and capturing intricate spatial patterns. The Swin Transformer, with its hierarchical structure and shifted window attention, has proven highly effective for general LULC tasks, outperforming state-of-the-art CNNs on datasets like EuroSat and NWPU-RESISC45 \cite{jannat20228u6}. Building on this, \cite{meng2022t3x} proposed the Class-Guided Swin Transformer (CG-Swin) for semantic segmentation of remote sensing images, leveraging the Swin Transformer as an encoder and designing a class-guided Transformer block for the decoder. This architecture effectively captures long-range dependencies crucial for accurate pixel-level classification in complex scenes, achieving significant breakthroughs on datasets such as ISPRS Vaihingen and Potsdam. For multimodal LULC classification, which often involves combining hyperspectral (HS), LiDAR, and SAR data, \cite{yao2023sax} presented ExViT, an Extended Vision Transformer framework. ExViT utilizes parallel ViT branches for each modality and a cross-modality attention (CMA) module to facilitate information exchange, demonstrating superior discriminative ability and classification performance compared to traditional methods.

Addressing the unique characteristics of **hyperspectral image classification (HSIC)**, \cite{chen2021d1q} introduced a multi-stage Vision Transformer with stacked samples. This approach tackles the challenge of ViTs sometimes ignoring local characteristics while focusing on global information, and mitigates their data hunger through an innovative data augmentation method. Similarly, for **Polarimetric SAR (PolSAR) land cover classification**, where labeled samples are often scarce, \cite{wang2022n7h} proposed a ViT-based method. This model leverages the ViT's powerful global feature representation and employs a Masked Autoencoder (MAE) for pre-training with unlabeled data, effectively overcoming the data scarcity limitation and demonstrating superior performance on datasets like Flevoland and Hainan. The integration of CNN-like inductive biases with ViTs also proves beneficial for remote sensing image classification (RSIC). \cite{wang202338i} developed P2FEViT, a Plug-and-Play CNN Feature Embedded Hybrid Vision Transformer. P2FEViT integrates CNN features into the ViT architecture to synchronously capture and fuse global context with discriminative local feature representations, thereby improving classification capability, reducing ViT's dependence on large-scale pre-training data, and accelerating convergence.

Beyond LULC and SAR, ViTs are also making strides in **agricultural monitoring**, where fine-grained detection and classification are essential. For instance, \cite{wang20215ra} developed SwinGD, a Swin Transformer-based model for robust grape bunch detection in complex vineyard environments, effectively handling irregularly shaped and dense objects. Similarly, \cite{p2024nbn} proposed GNViT, an enhanced Vision Transformer model for groundnut pest classification, achieving near-perfect accuracy (99.95\%) through robust data augmentation and transfer learning. These examples underscore ViTs' ability to capture global contextual information for precise identification tasks in diverse agricultural settings.

In conclusion, Vision Transformers have rapidly become a cornerstone in remote sensing and environmental monitoring. Their unparalleled ability to model global contextual information, often enhanced by specialized attention mechanisms and integrated into efficient or hybrid frameworks, provides powerful solutions for complex tasks. From improving SAR ship detection in cluttered maritime environments to enabling highly accurate multimodal LULC classification and fine-grained agricultural monitoring, ViTs offer significant advancements in accuracy and robustness. Future research in this domain will likely focus on developing more efficient ViT architectures tailored for real-time processing on edge devices, enhancing their robustness against environmental noise and adversarial attacks, and exploring their integration into broader geospatial foundation models for more comprehensive and intelligent environmental insights.