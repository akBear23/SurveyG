\subsection*{Modernizing CNNs with Vision Transformer Design Principles}

The emergence of Vision Transformers (ViTs) initially presented a significant challenge to the long-standing dominance of Convolutional Neural Networks (CNNs) in computer vision, primarily due to their superior global context modeling capabilities. However, this perceived architectural dichotomy has evolved into a profound convergence, where insights and design principles from ViTs are now being strategically applied to modernize and revitalize traditional CNNs. This trend effectively blurs the lines, fostering a new generation of powerful convolutional networks that benefit from lessons learned in the Transformer era, thereby challenging the absolute necessity of self-attention for state-of-the-art performance.

A crucial precursor to this modernization was the \textit{Swin Transformer} \cite{liu2021ljs}. While fundamentally a Transformer, Swin's hierarchical architecture, employing shifted window-based attention, introduced a more CNN-like inductive bias by processing local regions and progressively building global interactions. This design demonstrated that Transformers could achieve efficiency and scalability comparable to CNNs, inspiring researchers to investigate whether the performance gains of ViTs stemmed primarily from their attention mechanism or from their broader architectural structure and training methodologies. The Swin Transformer thus served as a blueprint, prompting a re-evaluation of CNN design principles.

The most seminal work exemplifying this modernization is the \textit{ConvNeXt} architecture \cite{liu2022convnext}. The authors embarked on a systematic 'metamorphosis' of a standard ResNet, incrementally incorporating design choices prevalent in state-of-the-art Vision Transformers, particularly those observed in Swin Transformer. This meticulous process involved several key modifications:
\begin{itemize}
    \item \textbf{Macro Design:} The overall stage-wise downsampling and channel ratios were adjusted to align with those of Swin Transformer, moving from a large stem to a patchify stem and adopting similar block repetition patterns.
    \item \textbf{ResNeXt-ification:} Efficient depthwise convolutions and inverted bottleneck designs, characteristic of efficient Transformer blocks, were integrated. This allowed for increased channel capacity within blocks while maintaining computational efficiency.
    \item \textbf{Larger Kernel Sizes:} The kernel size of depthwise convolutions was significantly increased (e.g., from 3x3 to 7x7). This allowed convolutional layers to capture a larger receptive field, mimicking the broader context captured by attention mechanisms without incurring their quadratic computational cost.
    \item \textbf{Layer Normalization:} Batch Normalization layers were replaced with Layer Normalization, a standard practice in Transformers that stabilizes training, especially with smaller batch sizes, and improves generalization.
    \item \textbf{Activation Functions:} The ReLU activation function was substituted with GELU, another activation function widely used in Transformers, contributing to improved performance.
    \item \textbf{Downsampling and Pooling:} Adjustments were made to downsampling layers, and the final classification head adopted a single global average pooling layer, streamlining the network's output stage.
\end{itemize}
By systematically applying these ViT-inspired design principles, ConvNeXt demonstrated that a pure convolutional network could achieve competitive or even superior performance to leading Transformers across various benchmarks, including ImageNet classification, COCO object detection, and ADE20K semantic segmentation. This work critically challenged the prevailing notion that self-attention was indispensable for achieving state-of-the-art results, instead highlighting the profound importance of macro-architectural design and training strategies.

Beyond architectural modifications, the influence of Vision Transformers extends to training methodologies. Transformer-inspired self-supervised learning strategies, particularly Masked Image Modeling (MIM), have proven highly effective for pre-training convolutional networks, further blurring the methodological distinctions. \textit{ConvMAE} \cite{gao2022convmae} directly applied the Masked Autoencoder (MAE) paradigm, originally developed for ViTs, to pure CNN architectures. It demonstrated that CNNs, like Transformers, could learn powerful visual representations by reconstructing masked image patches, achieving significant performance gains on downstream tasks. This showed that the effectiveness of MIM was not exclusive to attention-based models but could also benefit architectures with strong inductive biases like convolutions. Building on this, \textit{ConvNeXt V2} \cite{woo2023convnextv2} further showcased this synergy by pre-training the modernized ConvNeXt architecture using a fully convolutional masked autoencoder (FCMAE). This approach not only boosted ConvNeXt's performance but also solidified the argument that the benefits of Transformer-era training techniques are transferable to well-designed CNNs, leading to more robust and data-efficient convolutional backbones.

The revitalization of ConvNets through ViT design principles has found application in various domains. For instance, \textit{GenConViT} \cite{deressa2023lrl} for deepfake detection leverages both ConvNeXt and Swin Transformer models for robust feature extraction, illustrating how modernized CNNs can synergistically contribute to complex vision tasks alongside hierarchical ViTs. This ongoing convergence suggests that the future of powerful vision backbones will likely be a synthesis of the best elements from both worlds, leading to more versatile and robust models that effectively balance global context modeling with local inductive biases. The optimal balance and specific architectural configurations remain an active area of research, continually pushing the boundaries of what pure convolutional networks can achieve when inspired by Transformer innovations.