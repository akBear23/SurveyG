\subsection*{Specialized and Multimodal Applications}

Beyond their foundational success in standard image classification and object detection, Vision Transformers (ViTs) have demonstrated remarkable versatility and adaptability across a diverse array of specialized and complex application areas. This subsection highlights their profound utility in challenging real-world scenarios, ranging from low-level image processing to the analysis of high-dimensional and multimodal data, showcasing their capacity to transcend traditional benchmarks and open new avenues for research and deployment.

One significant domain where ViTs have excelled is image restoration, a crucial low-level vision task demanding precise pixel-level manipulation and global consistency. Traditional convolutional neural networks (CNNs) often struggle with capturing long-range dependencies essential for coherent restoration across an entire image. \textcite{liang2021v6x} addressed this by introducing SwinIR, a robust baseline for tasks such as super-resolution, denoising, and JPEG compression artifact reduction. SwinIR leverages the hierarchical Swin Transformer, which, through its shifted window attention, efficiently captures both local details and global structural information. This hierarchical design allows SwinIR to effectively model non-local correlations within images, crucial for hallucinating high-frequency details in super-resolution or removing noise while preserving fine textures, ultimately achieving superior performance over state-of-the-art CNNs with a reduced parameter count. Similarly, for visual saliency detection, which requires identifying the most visually prominent regions at a pixel level, \textcite{liu2021jpu} proposed the Visual Saliency Transformer (VST). VST re-conceptualizes saliency detection as a convolution-free sequence-to-sequence prediction problem. By employing multi-level token fusion and a novel token upsampling method, VST effectively processes both RGB and RGB-D inputs, demonstrating ViT's inherent capability for fine-grained pixel-level understanding and its adaptability to multimodal inputs through a token-based multi-task decoder. The global attention mechanism of VST allows it to integrate contextual information across the entire image, which is vital for distinguishing salient objects from complex backgrounds, a task where local CNN receptive fields might fall short.

The ability of ViTs to process and fuse information from diverse and high-dimensional data modalities further underscores their adaptability. For hyperspectral image classification (HSIC), which involves analyzing data cubes with hundreds of spectral bands alongside spatial information, ViTs offer a powerful alternative to traditional methods. HSIC presents unique challenges due to its high dimensionality, spectral redundancy, and often limited labeled samples. \textcite{zhao2024671} introduced the Groupwise Separable Convolutional Vision Transformer (GSC-ViT) to address these issues. GSC-ViT integrates a Groupwise Separable Convolution (GSC) module to efficiently extract local spectral-spatial features, mitigating the parameter burden of pure ViTs and enhancing local representation. Concurrently, a Groupwise Separable Multihead Self-Attention (GSSA) module captures both local and global spatial feature dependencies, allowing the model to effectively learn intricate relationships across the spectral and spatial dimensions. This hybrid approach demonstrates surprising classification performance even with fewer training samples, highlighting the ViT's capacity to model complex, high-dimensional data by judiciously combining inductive biases.

Beyond visual imagery, ViTs have proven effective in processing non-standard data types, such as radar signals for human activity recognition (HAR). HAR from radar data is challenging due to the abstract nature of micro-Doppler signatures, which represent subtle motion patterns. \textcite{huan202345b} proposed a Lightweight Hybrid Vision Transformer (LH-ViT) network for radar-based HAR. While the architectural details of LH-ViT are discussed in Section 5.2, its application here demonstrates how ViTs can effectively capture global temporal patterns within micro-Doppler spectrograms, which are crucial for distinguishing different human activities. The global attention mechanism allows the model to correlate features across the entire time-frequency representation, providing a more holistic understanding of the activity compared to local processing methods. The emphasis on a lightweight design further underscores the practical considerations for deploying such models in real-time or embedded radar systems.

Furthermore, ViTs are increasingly pivotal in multimodal data fusion, where information from heterogeneous sources must be synergistically combined for robust understanding. For multimodal land use and land cover (LULC) classification, which often involves integrating optical, hyperspectral, LiDAR, and Synthetic Aperture Radar (SAR) data, \textcite{yao2023sax} developed the Extended Vision Transformer (ExViT). This framework extends conventional ViTs with parallel branches, each tailored for a specific modality (e.g., hyperspectral and LiDAR/SAR data), utilizing position-shared ViTs and separable convolutions for efficient feature extraction. Critically, ExViT employs a cross-modality attention (CMA) module to facilitate dynamic information exchange and alignment between these heterogeneous modalities. This deep fusion mechanism, culminating in a robust decision-level fusion, leverages the ViT's token-based architecture to seamlessly integrate diverse data representations, leading to significantly improved classification accuracy and robustness compared to single-modality or simpler fusion approaches.

Collectively, these applications underscore the remarkable versatility of Vision Transformers. They illustrate how ViTs can be meticulously engineered, often through hybrid architectures, specialized attention mechanisms, and novel tokenization strategies, to effectively process diverse data types—from standard images and spectral cubes to radar micro-Doppler maps—and tackle complex tasks like restoration, saliency detection, high-dimensional classification, and multimodal fusion. The continuous drive to optimize ViTs for efficiency and integrate them with domain-specific inductive biases (e.g., convolutions for local features) highlights a key development direction. Future research will likely focus on further enhancing their efficiency for edge deployment, exploring more sophisticated multimodal fusion strategies, and adapting them to novel data modalities and real-time inference challenges in increasingly complex environments, thereby expanding their utility across an even broader spectrum of scientific and industrial applications.