\subsection{Data-Efficient Training and Tokenization Strategies}

Early Vision Transformers (ViTs) demonstrated impressive capabilities in image recognition, yet their reliance on massive proprietary datasets like JFT-300M for pre-training posed a significant barrier to their widespread adoption and practical application \cite{Dosovitskiy2021}. This section explores key innovations that addressed this limitation, focusing on strategies for data-efficient training and improved tokenization, thereby making ViTs more accessible and performant on standard, smaller datasets.

One of the most impactful approaches to mitigate ViTs' data hunger is knowledge distillation. The work by \cite{Touvron2021} introduced Data-efficient image Transformers (DeiT), which enabled ViTs to be trained effectively on ImageNet-1K, a significantly smaller dataset, without requiring external data. DeiT achieved this by employing a distillation strategy where a ViT student model learns from a powerful convolutional neural network (CNN) teacher. A crucial innovation was the "distillation token," an additional token that interacts with the class token through self-attention and is trained to match the teacher's output, effectively transferring the teacher's knowledge and enabling the ViT to achieve competitive performance with much less training data.

While knowledge distillation provided a powerful mechanism for data-efficient training, another line of research focused on enhancing the initial representation of image patches, thereby reducing the inherent data requirements for learning robust visual features. The original ViT treats non-overlapping image patches as independent tokens, which can lose fine-grained local structural information. To address this, \cite{Yuan2021} proposed Tokens-to-Token ViT (T2T-ViT), which introduced an improved tokenization strategy. Instead of a single-step patch embedding, T2T-ViT employs a multi-stage process where neighboring pixels are progressively aggregated into tokens through a series of self-attention layers. This hierarchical tokenization module effectively captures local structural details and preserves more information from the original image, leading to better performance from scratch on standard benchmarks like ImageNet-1K without relying on massive pre-training datasets or external teacher models.

The advancements in data-efficient training through distillation, as exemplified by DeiT, and improved tokenization strategies, such as T2T-ViT, collectively transformed the landscape of Vision Transformers. DeiT demonstrated that ViTs could achieve state-of-the-art results on ImageNet-1K by effectively mimicking a larger teacher, while T2T-ViT showed that a more sophisticated initial tokenization could inherently boost ViT performance on smaller datasets by better encoding local visual information. These innovations were critical in making ViTs more practical and accessible, moving them beyond the realm of models requiring immense computational resources and proprietary datasets. However, challenges remain in further reducing the data requirements for highly specialized tasks and in developing unified frameworks that seamlessly integrate the benefits of both distillation and advanced tokenization for optimal efficiency across diverse applications.