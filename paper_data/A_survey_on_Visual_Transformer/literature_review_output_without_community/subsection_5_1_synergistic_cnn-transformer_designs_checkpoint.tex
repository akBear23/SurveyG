\subsection{Synergistic CNN-Transformer Designs}

The evolution of computer vision architectures has increasingly moved towards a synergistic paradigm, strategically combining the inherent strengths of Convolutional Neural Networks (CNNs) with the global context modeling capabilities of Transformers. This convergence is driven by the recognition that while CNNs excel at local feature extraction and possess strong inductive biases like locality and translation equivariance, pure Vision Transformers (ViTs) \cite{dosovitskiy2021image} offer unparalleled global receptive fields and flexibility in capturing long-range dependencies. Hybrid designs aim to mitigate the limitations of each, such as ViTs' high data requirements and quadratic computational complexity, or CNNs' struggle with modeling global relationships, thereby fostering more robust, efficient, and versatile vision backbones.

One prominent direction in this synergy involves modernizing traditional CNN architectures by incorporating Transformer-inspired methodologies and training paradigms. A seminal work in this category is \textit{ConvNeXt V2} \cite{iclr2023}. Building upon the architectural principles of ConvNeXt, which re-examined and optimized CNN designs to resemble Transformers (e.g., using large kernel depthwise convolutions and inverted bottleneck structures), ConvNeXt V2 crucially leveraged the Masked Autoencoder (MAE) pre-training strategy \cite{he2022masked}. This integration demonstrated that CNNs could benefit immensely from self-supervised learning techniques originally developed for Transformers, achieving state-of-the-art performance and effectively bridging the perceived architectural gap. ConvNeXt V2 showcased that the inductive biases of CNNs, when combined with powerful, scalable pre-training, remain highly competitive and efficient.

Conversely, another significant approach integrates powerful CNN components directly into Transformer-like frameworks. \textit{InternImage} \cite{cvpr2023} exemplifies this by embedding deformable convolutions within a Transformer-inspired architecture. Deformable convolutions, a hallmark of advanced CNNs, enable adaptive receptive fields and flexible spatial sampling, which are crucial for handling geometric variations and object deformations effectively. By incorporating this mechanism, InternImage creates robust "Vision Foundation Models" that combine the global context understanding of Transformers with the precise, adaptive local feature extraction characteristic of sophisticated convolutions. This allows the model to dynamically focus on relevant regions, enhancing its ability to capture fine-grained details and adapt to varying object shapes and scales, a limitation often observed in standard fixed-grid attention mechanisms. A related approach is seen in \textit{Deformable Attention Transformer (DAT)} and its enhanced version \textit{DAT++} \cite{xia2022qga, xia2023bp7}. These models integrate a novel deformable multi-head attention module where key and value positions are adaptively allocated in a data-dependent manner, effectively bringing the adaptive sampling power of deformable convolutions directly into the self-attention mechanism, thereby enhancing spatial awareness and reducing computational overhead compared to dense global attention.

Beyond integrating components or pre-training strategies, some hybrid architectures explicitly interleave CNN and Transformer blocks to capitalize on their respective strengths at different stages of feature extraction. \textit{Next-ViT} \cite{li2022a4u} proposes a "Next Hybrid Strategy" (NHS) that stacks "Next Convolution Blocks" (NCB) for local information capture and "Next Transformer Blocks" (NTB) for global information. This design is optimized for efficient deployment in realistic industrial scenarios, addressing the challenge of achieving both high performance and low latency on hardware accelerators like TensorRT. Similarly, \textit{TRT-ViT} \cite{xia2022dnj} provides practical guidelines for TensorRT-oriented network design, suggesting an "early CNN and late Transformer at stage-level" and "early Transformer and late CNN at block-level" strategy. This highlights a nuanced understanding of where and how to best deploy each architectural component for optimal hardware efficiency and performance. These block-level hybrid models demonstrate that a careful arrangement of convolutional and self-attention layers can yield superior latency-accuracy trade-offs across various vision tasks.

Another innovative direction involves fusing local and global processing within the Transformer block itself, drawing inspiration from CNNs' local inductive biases without necessarily using explicit convolutional layers. \textit{PLG-ViT} \cite{ebert202377v} introduces a "Parallel Local-Global Vision Transformer" that merges local window self-attention with global self-attention. This design efficiently represents short- and long-range spatial interactions, bypassing the need for computationally expensive operations like shifted windows while still achieving strong performance in image classification and dense prediction tasks. This approach reflects a deeper integration, where the *concept* of locality, traditionally associated with CNNs, is re-imagined and implemented within the Transformer's attention mechanism.

In summary, the landscape of synergistic CNN-Transformer designs is rich and diverse, reflecting various philosophies of integration. ConvNeXt V2 represents a CNN-centric approach enhanced by Transformer pre-training. InternImage and DAT/DAT++ showcase embedding CNN-inspired adaptive mechanisms into Transformer backbones. Next-ViT and TRT-ViT demonstrate strategic interleaving of CNN and Transformer blocks for deployment efficiency. PLG-ViT illustrates the internal fusion of local and global attention within a Transformer, mimicking CNN-like inductive biases. These models collectively demonstrate that moving beyond a strict CNN-Transformer dichotomy towards a complementary approach unlocks new levels of performance, efficiency, and versatility, pushing the boundaries of architectural design for a wide array of computer vision applications. Future research will likely explore even more sophisticated integration strategies, novel unified pre-training objectives, and dynamic architectural adaptations to further harness the combined power of these two foundational paradigms.