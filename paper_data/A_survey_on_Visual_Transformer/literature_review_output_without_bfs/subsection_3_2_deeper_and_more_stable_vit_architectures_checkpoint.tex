\subsection{Deeper and More Stable ViT Architectures}
The initial success of Vision Transformers (ViTs) was often constrained by the inherent challenges of training very deep Transformer models, such as vanishing gradients, exploding activations, and general training instability. Overcoming these hurdles was crucial to unlock the full representational capacity of ViTs and enable the development of much deeper and more complex architectures. This subsection examines the architectural and training advancements that directly addressed these stability issues, facilitating the scaling of ViTs to unprecedented depths.

The foundational work by \cite{ViT} demonstrated the potential of a pure Transformer encoder for image recognition, but implicitly highlighted the difficulties in training such models from scratch, particularly at greater depths, often necessitating massive pre-training datasets. To directly tackle the instability encountered when scaling ViTs, \cite{CaiT} introduced several pivotal innovations. Their work, titled "Going deeper with Image Transformers," specifically focused on enabling the training of significantly deeper ViTs. A key contribution was **LayerScale**, a mechanism that adaptively re-scales residual connections within each Transformer block. This adaptive re-scaling helps to stabilize the training process by preventing activations from growing too large or vanishing too quickly, thereby mitigating gradient issues and allowing for the successful optimization of models with many more layers. Furthermore, \cite{CaiT} proposed **class-attention layers**, which refine the global representation by allowing the class token to interact with image tokens in a more nuanced way, further enhancing the model's ability to learn robust features in deep architectures. These techniques were instrumental in pushing ViT depth beyond previous limits, demonstrating the viability of very deep Transformer-based vision models.

Building upon these foundational insights, other works contributed to the overall robustness and scalability of ViTs, indirectly supporting the development of deeper models. For instance, \cite{DeiT} addressed the significant data requirements of early ViTs through knowledge distillation. While not directly an architectural stabilization technique, making ViTs trainable on smaller datasets broadened their applicability and made the exploration of deeper models more practical by reducing the computational burden of pre-training on colossal datasets. Similarly, advancements in initial tokenization by \cite{T2T-ViT} and more flexible positional encodings by \cite{CPVT} provided better initial representations and adaptability, which can contribute to the overall stability and performance of deeper networks.

The principles established by \cite{CaiT} for stabilizing deep Transformers have had a lasting impact, influencing the design of subsequent, more complex ViT variants. For example, hierarchical Vision Transformers like the Swin Transformer \cite{Swin} and Pyramid Vision Transformer \cite{PVT} introduced architectural modifications to handle multi-scale features and reduce quadratic complexity, but their ability to scale to significant depths also relies on robust training practices and implicitly benefits from the understanding of stability mechanisms. More recent hierarchical designs, such as Hiera \cite{ryali202339q} and HiViT \cite{zhang2022msa}, continue to explore efficient and deep architectures, often leveraging strong self-supervised pre-training (like Masked Autoencoders) to simplify architectural "bells-and-whistles" while maintaining performance. These models demonstrate a continued evolution, where the initial focus on stabilizing the core Transformer block has matured into designing entire deep, efficient architectures that are robust to training.

In conclusion, the journey towards deeper and more stable ViT architectures has been marked by critical innovations that directly address the inherent challenges of training very deep Transformers. Techniques like LayerScale have been instrumental in stabilizing the optimization landscape, allowing ViTs to scale to unprecedented depths and unlock greater representational power. While the field continues to explore various architectural modifications and training paradigms, the fundamental understanding of how to maintain training stability remains a cornerstone for pushing the boundaries of ViT performance and enabling the development of even larger and more complex models for diverse computer vision tasks. The ongoing challenge lies in balancing this architectural depth and representational capacity with computational efficiency and the continued search for robust, generalizable training methodologies.