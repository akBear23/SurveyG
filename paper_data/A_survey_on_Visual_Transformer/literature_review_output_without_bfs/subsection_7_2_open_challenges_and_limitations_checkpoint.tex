\subsection*{Open Challenges and Limitations}

Despite the remarkable success of Vision Transformers (ViTs) across numerous computer vision tasks, several open challenges and inherent limitations persist, hindering their widespread and efficient deployment in all scenarios. These issues primarily revolve around their high computational demands, reliance on extensive datasets, questions of interpretability, robustness, and the ongoing debate regarding optimal architectural inductive biases.

A primary limitation stems from the high computational cost associated with the global self-attention mechanism, particularly when processing high-resolution inputs. The quadratic complexity of vanilla Transformer architectures, as introduced by foundational works, becomes prohibitive for large images or video streams. Early efforts to mitigate this, such as the Swin Transformer \cite{Swin} and Pyramid Vision Transformer (PVT) \cite{PVT}, introduced hierarchical structures and localized attention (e.g., shifted windows) to achieve linear complexity with respect to image size, making ViTs more viable for dense prediction tasks. Building on this, works like Next-ViT \cite{li2022a4u} and TRT-ViT \cite{xia2022dnj} further emphasize practical deployment efficiency, optimizing block designs and hybrid strategies to achieve superior latency-accuracy trade-offs on industrial hardware like TensorRT, recognizing that theoretical FLOPs do not always translate to real-world inference speed. More recently, MambaVision \cite{hatamizadeh2024xr6} explores state-space models (Mamba) as an alternative to self-attention, proposing a hybrid Mamba-Transformer backbone that aims to capture long-range dependencies efficiently, potentially offering a new paradigm for linear complexity in visual modeling. For video applications, specific architectural adaptations like TP-ViT \cite{jing2022nkb} and ViT-Shift \cite{zhang2024g0m} introduce multi-pathway designs and temporal shift modules to efficiently model spatio-temporal information, addressing the unique computational challenges of video processing.

Another significant challenge for ViTs is their continued reliance on large datasets for optimal performance, a stark contrast to the data efficiency often exhibited by Convolutional Neural Networks (CNNs due to their strong inductive biases). While self-supervised learning methods like DINO \cite{DINO} and MAE \cite{MAE} have significantly reduced the need for labeled data, enabling ViTs to learn powerful representations from unlabeled images, the challenge of training ViT-based models from scratch without extensive pre-training remains. Hong et al. \cite{hong2022ks6} empirically studied training Transformer-based object detectors from scratch, finding that architectural modifications and longer training schedules are crucial to achieve competitive performance without large-scale pre-training. This issue is particularly pronounced in specialized domains like remote sensing, where labeled data can be scarce. To address this, QAGA-Net \cite{song202479c}, ODDL-Net \cite{song2025idg}, and the quantitative regularization (QR) approach \cite{song2024fx9} propose novel data augmentation and regularization strategies specifically tailored for remote sensing images, demonstrating that ViTs can achieve robust performance even with limited training samples by optimizing data distribution learning. Similarly, P2FEViT \cite{wang202338i} aims to reduce ViT's data dependency by embedding CNN features, facilitating faster convergence with less data.

The debate about the optimal balance between pure attention-based designs and the integration of convolutional priors for various tasks and deployment environments is ongoing, reflecting the inherent trade-offs between architectural complexity, efficiency, and performance. Pure ViTs, lacking the strong inductive biases of CNNs (e.g., locality, translation equivariance), often require more data and computation. Consequently, many works explore hybrid architectures that combine the strengths of both. For instance, CTNet \cite{deng2021man} and LDBST \cite{zheng202325h} propose dual-branch frameworks for remote sensing scene classification, leveraging CNNs for local structural features and ViTs for global semantic context. P2FEViT \cite{wang202338i} further exemplifies this by embedding plug-and-play CNN features into ViT architectures to enhance local feature representation and accelerate convergence. This convergence of ideas is also evident in works like ConvNeXt \cite{ConvNeXt}, which demonstrates that modern CNNs, by adopting ViT design principles, can achieve competitive performance, blurring the lines between the two paradigms. Conversely, PoolFormer \cite{PoolFormer} provocatively suggests that the meta-architecture of Transformers, rather than complex self-attention, might be the key, achieving strong results with simple pooling operations, challenging the necessity of attention itself for certain tasks.

Beyond these architectural and data-related challenges, issues of interpretability and robustness to adversarial attacks remain significant open problems for ViTs. The complex, global interactions modeled by self-attention layers make it difficult to ascertain *why* a ViT makes a particular decision, hindering trust and deployment in critical applications. While some methods attempt to visualize attention maps, these often provide only a partial view of the model's reasoning. Similarly, ViTs have shown vulnerability to adversarial attacks, a concern that needs more dedicated research to ensure their reliability in real-world scenarios. The theoretical understanding of their inductive biases, or the lack thereof, also remains less mature compared to CNNs, complicating systematic improvements and robust design principles.

In conclusion, while Vision Transformers have revolutionized computer vision, their journey is far from complete. The field continues to grapple with fundamental trade-offs between computational cost and performance, the persistent need for large-scale data, and the optimal integration of local and global feature learning mechanisms. Future research will likely focus on developing more efficient, data-agnostic, inherently interpretable, and robust ViT architectures, potentially through novel attention mechanisms, alternative long-range dependency modeling (e.g., state-space models), or more sophisticated hybrid designs that judiciously combine the strengths of various neural network components.