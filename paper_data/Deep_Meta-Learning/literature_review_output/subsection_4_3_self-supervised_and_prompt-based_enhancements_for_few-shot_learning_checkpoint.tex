\subsection{Self-Supervised and Prompt-Based Enhancements for Few-Shot Learning}

Few-shot learning, a critical paradigm for enabling models to generalize from minimal examples, has seen significant advancements through the strategic integration of meta-learning with auxiliary techniques such as self-supervised learning (SSL) and prompt-based adaptation for large pre-trained models. These innovations aim to enrich learned feature representations and improve generalization to novel classes by leveraging broader data sources and sophisticated adaptation strategies. Within the context of metric and relation-based meta-learning, these enhancements are particularly impactful as they directly improve the quality of the embedding space and the comparison mechanisms that are central to these approaches.

The core challenge in few-shot learning is to learn robust, generalizable representations from extremely limited labeled data. Self-supervised learning offers a powerful solution by exploiting vast amounts of unlabeled data to pre-train feature encoders, thereby acquiring general knowledge and richer embeddings that are more amenable to subsequent few-shot adaptation. This approach aligns well with the principles of metric-based meta-learning, where the quality of the learned embedding space directly dictates the effectiveness of similarity comparisons. For instance, early work recognized the importance of learning in a "concept space" where instances are represented by high-level concepts, allowing meta-learners to perform few-shot learning more effectively \cite{zhou20188lr}. Similarly, \cite{rajasegaran2020glw} demonstrated that self-supervised knowledge distillation can significantly improve the representation capacity of deep neural networks for few-shot learning by maximizing feature embedding entropy and then constraining it via student-teacher distillation. Such robust representations are crucial for preventing meta-learners from merely memorizing training tasks and instead encouraging data-driven adaptation to novel tasks \cite{yin2019cct}.

A prime example of this integration in text classification is Self-supervised Information Enhanced Meta-Learning (SEML) \cite{li2023zn0}. SEML addresses the limitation of relying solely on labeled few-shot examples by integrating self-supervised learning from unlabeled data. It employs a two-stage pipeline: first, a feature encoder (e.g., BERT) is trained using self-guided contrastive learning on unlabeled samples to acquire general knowledge and robust representations. Subsequently, a novel knowledge distillation method is introduced to expand and enrich the meta-learning representation with this self-supervised information. This is coupled with a graph aggregation method that generates more discriminative feature representations within each task by facilitating effective interaction between support and query sets \cite{li2023zn0}. This approach effectively fuses self-supervised knowledge into a unified distribution, yielding a meta-network with superior generalization capabilities for few-shot text classification, by providing a more informed and robust embedding space for metric-based comparisons.

While self-supervised methods like SEML primarily focus on building better feature extractors from the ground up using unlabeled data, a parallel and increasingly vital approach for massive pre-trained models, such as Vision-Language Models (VLMs) and Large Language Models (LLMs), focuses on efficiently steering their existing, powerful representations. This is where prompt-based enhancements, informed by meta-learning, come into play. The advent of these large foundation models has opened new avenues for few-shot learning, as their vast pre-training enables strong zero-shot and few-shot capabilities \cite{peng20209of}. Empirical observations suggest that larger backbone networks often correlate with better few-shot performance \cite{huisman2020b7w}, and even simple fine-tuning can be highly effective \cite{baz2022n78}. However, efficiently adapting these massive models to novel tasks with limited data without overfitting remains a significant hurdle.

Prompt tuning, which involves learning small, task-specific context vectors to guide VLMs or LLMs, offers a parameter-efficient adaptation strategy. Yet, existing prompt tuning methods, such as Context Optimization (CoOp), often suffer from severe overfitting to base classes and exhibit poor generalization to novel classes \cite{wang2024dai}. This limitation arises because the limited capacity of adaptable prompt vectors can lead to text knowledge being overly tailored to base classes, hindering transferability.

To mitigate these issues, meta-learning has been strategically applied to prompt tuning. The "Learning to Learn" (LoL) method \cite{wang2024dai} proposes a meta-learning-informed prompt tuning approach that treats the process as a meta-learning problem. By adopting an N-way K-shot episodic training framework for learning prompts, LoL explicitly addresses the base-to-new generalization problem. This two-stage strategy, combining initial CoOp-based prompt tuning with a subsequent meta-learning stage on base classes, effectively organizes input data during training. This enables the limited capacity of adaptable prompt vectors to better absorb and transfer knowledge to novel classes by minimizing overfitting, thereby improving the quality of the task-specific representations generated by the VLM for few-shot classification \cite{wang2024dai}. Such meta-training frameworks are also crucial for improving generalization in cross-domain few-shot learning, where tasks come from unseen domains \cite{tian2023iyh}.

In essence, both self-supervised and prompt-based enhancements leverage meta-learning to push the boundaries of few-shot performance, albeit through different mechanisms. Self-supervised learning, exemplified by SEML, focuses on building more robust and discriminative feature representations from unlabeled data, which directly benefits metric-based meta-learning by providing a superior embedding space for similarity comparisons. Prompt-based methods, such as LoL, on the other hand, focus on efficiently adapting the powerful, pre-existing representations of large foundation models. Here, meta-learning provides the "learning to learn" framework \cite{finn2017vrt} that enables the prompts to generalize effectively to novel tasks, rather than overfitting to base classes. The synergy between these paradigms represents a powerful direction for few-shot learning, enriching feature representations through broader data sources and enabling more effective, less overfitting adaptation of large pre-trained models. Future research will likely focus on optimizing the computational overhead of these integrated systems, developing more robust meta-objectives for highly diverse tasks, and exploring novel ways to combine the vast knowledge of foundation models with the rapid adaptation capabilities of meta-learning.