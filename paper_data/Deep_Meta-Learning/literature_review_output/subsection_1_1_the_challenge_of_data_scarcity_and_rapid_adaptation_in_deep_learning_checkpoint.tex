\subsection{The Challenge of Data Scarcity and Rapid Adaptation in Deep Learning}

Conventional deep learning models, despite their remarkable successes across diverse domains, are fundamentally constrained by their inherent demand for extensive labeled datasets and often lengthy, computationally intensive training regimes \cite{huisman2020b7w, hospedales2020m37}. This reliance on vast, task-specific data and static, pre-trained models presents a significant bottleneck for real-world artificial intelligence (AI) systems, which frequently encounter scenarios characterized by data scarcity, dynamic environments, and an imperative for rapid, flexible adaptation \cite{beck2023x24}. The inability of traditional deep learning to efficiently generalize with minimal new data, or to quickly acquire novel skills, severely limits its practical applicability and hinders progress towards truly versatile and efficient AI.

One of the most pressing challenges is **data scarcity**. In many critical applications, obtaining large, high-quality labeled datasets is either prohibitively expensive, time-consuming, or simply impossible. For instance, in specialized fields like medical imaging, rare disease diagnosis, or personalized healthcare, annotated data is inherently limited due to privacy concerns, the rarity of conditions, or the expertise required for annotation \cite{dufumier2021ec1}. Similarly, in scientific discovery, environmental monitoring, or industrial fault detection, novel events or anomalies often occur infrequently, yielding insufficient examples for conventional supervised learning. This problem is exacerbated by the fact that deep neural networks, particularly larger architectures, tend to overfit when trained on small datasets, leading to poor generalization \cite{huisman2020b7w}. The cost associated with human annotation, especially for complex tasks, further compounds this issue, making it impractical to scale deep learning solutions to a multitude of specialized tasks.

Beyond data volume, the challenge of **rapid adaptation and generalization** is equally critical. Real-world environments are rarely static; AI systems must frequently encounter novel tasks, new object categories, or unforeseen environmental conditions. Traditional deep learning approaches typically require retraining or extensive fine-tuning for each new task, which is computationally expensive and time-consuming. While transfer learning offers some mitigation by leveraging pre-trained features, it often struggles when the target task or domain significantly deviates from the source domain, or when only a handful of examples are available for adaptation \cite{hospedales2020m37}. This limitation is particularly pronounced in sequential decision-making settings, such as reinforcement learning (RL) and robotics, where agents must quickly adapt to new environments, reward functions, or physical dynamics with minimal interactions \cite{beck2023x24}. For example, enabling a robot to acquire a wide variety of complex skills from raw sensory inputs in unstructured environments demands rapid adaptation from few demonstrations, a feat traditional methods struggle with due to their data inefficiency per task \cite{finn20174c4}. The goal is to move beyond task-specific models that learn in isolation, towards systems capable of human-like learning, where prior experience across diverse tasks informs and accelerates the acquisition of new skills \cite{huisman2020b7w}.

These fundamental limitations—the insatiable data hunger and the slow, task-specific adaptation of conventional deep learning—represent a core bottleneck in the development of general artificial intelligence. They highlight a critical gap between the impressive performance of deep learning on well-defined, data-rich tasks and the flexibility, efficiency, and versatility required for AI systems to operate effectively in dynamic, open-ended real-world scenarios. Addressing this gap necessitates a paradigm shift: instead of designing models that learn a single task, the focus must move towards developing systems that can "learn to learn" \cite{hospedales2020m37}. This foundational need for models that can leverage experience across a distribution of tasks to adapt rapidly and efficiently to new, unseen tasks with minimal data is precisely what motivated the emergence of meta-learning, paving the way for more versatile and efficient AI systems \cite{baz2022n78}.