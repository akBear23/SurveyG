\documentclass[12pt,a4paper]{article}
    \usepackage[utf8]{inputenc}
    \usepackage[T1]{fontenc}
    \usepackage{amsmath,amsfonts,amssymb}
    \usepackage{graphicx}
    \usepackage[margin=2.5cm]{geometry}
    \usepackage{setspace}
    \usepackage{natbib}
    \usepackage{url}
    \usepackage{hyperref}
    \usepackage{booktabs}
    \usepackage{longtable}
    \usepackage{array}
    \usepackage{multirow}
    \usepackage{wrapfig}
    \usepackage{float}
    \usepackage{colortbl}
    \usepackage{pdflscape}
    \usepackage{tabu}
    \usepackage{threeparttable}
    \usepackage{threeparttablex}
    \usepackage[normalem]{ulem}
    \usepackage{makecell}
    \usepackage{xcolor}

    % Set line spacing
    \doublespacing

    % Configure hyperref
    \hypersetup{
        colorlinks=true,
        linkcolor=blue,
        filecolor=magenta,      
        urlcolor=cyan,
        citecolor=red,
    }

    % Title and author information
    \title{A Comprehensive Literature Review with Self-Reflection}
    \author{Literature Review}
    \date{\today}

    \begin{document}

    \maketitle

    % Abstract (optional)
    \begin{abstract}
    This literature review provides a comprehensive analysis of recent research in the field. The review synthesizes findings from 285 research papers, identifying key themes, methodological approaches, and future research directions.
    \end{abstract}

    \newpage
    \tableofcontents
    \newpage

    \label{sec:introduction_to_deep_meta-learning}

\section{Introduction to Deep Meta-Learning}
\label{sec:introduction\_to\_deep\_meta-learning}

\subsection{The Challenge of Data Scarcity and Rapid Adaptation in Deep Learning}
\label{sec:1\_1\_the\_challenge\_of\_data\_scarcity\_\_and\_\_rapid\_adaptation\_in\_deep\_learning}

Conventional deep learning models, despite their remarkable successes across diverse domains, are fundamentally constrained by their inherent demand for extensive labeled datasets and often lengthy, computationally intensive training regimes \cite{huisman2020b7w, hospedales2020m37}. This reliance on vast, task-specific data and static, pre-trained models presents a significant bottleneck for real-world artificial intelligence (AI) systems, which frequently encounter scenarios characterized by data scarcity, dynamic environments, and an imperative for rapid, flexible adaptation \cite{beck2023x24}. The inability of traditional deep learning to efficiently generalize with minimal new data, or to quickly acquire novel skills, severely limits its practical applicability and hinders progress towards truly versatile and efficient AI.

One of the most pressing challenges is \textbf{data scarcity}. In many critical applications, obtaining large, high-quality labeled datasets is either prohibitively expensive, time-consuming, or simply impossible. For instance, in specialized fields like medical imaging, rare disease diagnosis, or personalized healthcare, annotated data is inherently limited due to privacy concerns, the rarity of conditions, or the expertise required for annotation \cite{dufumier2021ec1}. Similarly, in scientific discovery, environmental monitoring, or industrial fault detection, novel events or anomalies often occur infrequently, yielding insufficient examples for conventional supervised learning. This problem is exacerbated by the fact that deep neural networks, particularly larger architectures, tend to overfit when trained on small datasets, leading to poor generalization \cite{huisman2020b7w}. The cost associated with human annotation, especially for complex tasks, further compounds this issue, making it impractical to scale deep learning solutions to a multitude of specialized tasks.

Beyond data volume, the challenge of \textbf{rapid adaptation and generalization} is equally critical. Real-world environments are rarely static; AI systems must frequently encounter novel tasks, new object categories, or unforeseen environmental conditions. Traditional deep learning approaches typically require retraining or extensive fine-tuning for each new task, which is computationally expensive and time-consuming. While transfer learning offers some mitigation by leveraging pre-trained features, it often struggles when the target task or domain significantly deviates from the source domain, or when only a handful of examples are available for adaptation \cite{hospedales2020m37}. This limitation is particularly pronounced in sequential decision-making settings, such as reinforcement learning (RL) and robotics, where agents must quickly adapt to new environments, reward functions, or physical dynamics with minimal interactions \cite{beck2023x24}. For example, enabling a robot to acquire a wide variety of complex skills from raw sensory inputs in unstructured environments demands rapid adaptation from few demonstrations, a feat traditional methods struggle with due to their data inefficiency per task \cite{finn20174c4}. The goal is to move beyond task-specific models that learn in isolation, towards systems capable of human-like learning, where prior experience across diverse tasks informs and accelerates the acquisition of new skills \cite{huisman2020b7w}.

These fundamental limitations—the insatiable data hunger and the slow, task-specific adaptation of conventional deep learning—represent a core bottleneck in the development of general artificial intelligence. They highlight a critical gap between the impressive performance of deep learning on well-defined, data-rich tasks and the flexibility, efficiency, and versatility required for AI systems to operate effectively in dynamic, open-ended real-world scenarios. Addressing this gap necessitates a paradigm shift: instead of designing models that learn a single task, the focus must move towards developing systems that can "learn to learn" \cite{hospedales2020m37}. This foundational need for models that can leverage experience across a distribution of tasks to adapt rapidly and efficiently to new, unseen tasks with minimal data is precisely what motivated the emergence of meta-learning, paving the way for more versatile and efficient AI systems \cite{baz2022n78}.
\subsection{Defining Deep Meta-Learning: The Learning to Learn Paradigm}
\label{sec:1\_2\_defining\_deep\_meta-learning:\_the\_learning\_to\_learn\_paradigm}

Deep Meta-Learning signifies a fundamental shift in the pursuit of artificial intelligence, moving beyond systems that are merely proficient at solving a single, predefined task to those capable of acquiring new skills and adapting to novel tasks with remarkable efficiency \cite{hospedales2020m37, son2023lda}. This paradigm is often encapsulated by the phrase "learning to learn," a concept that posits intelligence not just as the ability to perform a task, but as the capacity to improve one's own learning process \cite{thrun1998learning, schmidhuber1987evolutionary}. Unlike traditional deep learning, which typically requires vast amounts of task-specific data and extensive training for each new problem, deep meta-learning aims to extract transferable knowledge from a distribution of diverse tasks, enabling rapid generalization and adaptation to previously unseen tasks with minimal new data \cite{hospedales2020m37}.

At its core, deep meta-learning involves a bi-level optimization problem where an "outer-loop" or meta-learner processes a collection of distinct but related tasks. Each task $T\_i$ is typically presented with a small "support set" of labeled examples for adaptation and a "query set" for evaluation. The meta-learner's objective is not to master any single task, but to optimize a set of meta-parameters that govern the learning process itself, such that a base learner can quickly and effectively adapt to any new task drawn from the same underlying task distribution \cite{hospedales2020m37, finn2017vrt}. This contrasts sharply with conventional approaches where model parameters are optimized directly for a single task, and hyperparameters are typically tuned manually or via grid search. In meta-learning, the meta-parameters are learned automatically from experience across multiple learning episodes, thereby automating and optimizing the higher-level learning process \cite{sutton2022jss}.

The "learning to learn" mechanism manifests through the optimization of various components of the learning process, rather than just the final model parameters. These components can include:
\begin{enumerate}
    \item \textbf{Initialization Strategies}: Learning an optimal starting point for model parameters that can be rapidly fine-tuned with a few gradient steps on a new task. This is a cornerstone of optimization-based meta-learning, where the meta-learner seeks an initialization that is maximally amenable to quick adaptation \cite{hospedales2020m37}.
    \item \textbf{Optimization Rules}: Meta-learning can involve learning the dynamics of the optimization process itself, such as task-specific learning rates, update directions, or even entire gradient descent algorithms. This allows the meta-learner to discover more efficient ways to update parameters for novel tasks \cite{sutton2022jss}.
    \item \textbf{Comparison Mechanisms and Representations}: In scenarios like few-shot classification, the meta-learner can learn an embedding space and a robust similarity metric or relation function. This enables effective comparison between new examples and a small set of known examples, facilitating rapid categorization or decision-making \cite{hospedales2020m37}.
    \item \textbf{Model Architectures and Memory}: Some meta-learning approaches focus on designing neural network architectures that are intrinsically capable of rapid adaptation or explicit storage and retrieval of task-specific information. These models might incorporate recurrent mechanisms or external memory modules to process and adapt to new task contexts without explicit gradient updates \cite{hospedales2020m37}.
\end{enumerate}

The overarching goal is to enable models to generalize quickly and effectively from limited data, a critical capability for achieving human-like cognitive flexibility and addressing the challenges of data scarcity and slow adaptation inherent in traditional deep learning \cite{hospedales2020m37, son2023lda}. This involves not only empirical performance but also a deeper theoretical understanding of how meta-learning achieves its generalization capabilities. For instance, research has explored the universality of gradient-based meta-learning, demonstrating its capacity to approximate a wide range of learning algorithms \cite{finn2017vrt}, and has sought to derive generalization bounds that provide theoretical guarantees for its effectiveness in few-shot learning scenarios \cite{chen2021j5t}.

In essence, Deep Meta-Learning is about building intelligent systems that can continually improve their own learning capabilities by observing and learning from a diverse set of learning experiences. This foundational concept underpins various algorithmic families—optimization-based, metric-based, and model-based approaches—each offering distinct mechanisms to achieve this "learning to learn" objective, which will be explored in detail in subsequent sections.
\subsection{Scope and Structure of the Review}
\label{sec:1\_3\_scope\_\_and\_\_structure\_of\_the\_review}

This literature review offers a comprehensive and structured survey of Deep Meta-Learning, a rapidly evolving field focused on enabling AI systems to "learn to learn" and adapt efficiently to novel tasks with limited data \cite{lee2021jou}. Our primary objective is to provide a pedagogical understanding of the field's intellectual trajectory, meticulously detailing its evolution from foundational concepts and core algorithmic paradigms to advanced topics, real-world applications, and emerging research frontiers. By mapping the landscape of approaches, from fundamental principles to cutting-edge developments \cite{farrell2024mpy}, this review aims to present a coherent and insightful narrative for researchers and practitioners interested in the frontiers of adaptable AI, highlighting key methodological advancements, problem-solving capabilities, and future research directions.

The review commences with an essential introduction in \textbf{Section 1}, establishing the fundamental motivations for Deep Meta-Learning, particularly in addressing the limitations of traditional deep learning concerning data scarcity and slow adaptation. It formally defines the "learning to learn" paradigm, setting the conceptual groundwork for the subsequent exploration. Building upon this, \textbf{Section 2} introduces the three primary algorithmic families that underpin Deep Meta-Learning: optimization-based, metric-based, and model-based approaches. This foundational categorization is crucial for understanding the diverse methodological landscape and serves as a high-level overview before delving into the specifics of each paradigm.

Following the introduction of these core paradigms, the review dedicates three extensive sections to a deep dive into each methodological family. \textbf{Section 3} provides a detailed examination of optimization-based meta-learning, exploring techniques that learn adaptable initializations and explicit optimizers for rapid fine-tuning. This section covers seminal works and their subsequent enhancements, demonstrating how models can "learn to optimize" effectively. Subsequently, \textbf{Section 4} delves into metric and relation-based meta-learning, focusing on methods that learn discriminative embedding spaces and sophisticated comparison functions for effective few-shot classification. It traces the evolution from simple distance metrics to complex, learned relational functions. Concluding the methodological exploration, \textbf{Section 5} investigates model-based meta-learning, highlighting architectural innovations, including recurrent neural networks, memory-augmented networks, and transformer-based designs, which intrinsically enable rapid adaptation and in-context learning. This structured progression allows for a thorough understanding of the distinct mechanisms employed across the field, from gradient-based adaptation to architectural solutions.

With a solid understanding of the core methodologies established, the review transitions in \textbf{Section 6} to address advanced challenges and inherent properties crucial for developing robust and reliable meta-learning systems. This section explores probabilistic meta-learning for uncertainty quantification, meta-learning's role in enabling continual and lifelong learning to mitigate catastrophic forgetting, and strategies for improving domain generalization and out-of-distribution robustness. By examining these critical aspects, the review demonstrates how meta-learning extends beyond basic few-shot learning to tackle complex issues vital for real-world deployment, fostering more resilient and trustworthy AI.

The focus then shifts to the practical impact and application of Deep Meta-Learning in \textbf{Section 7}, specifically within sequential decision-making and control. This section highlights its utility in areas such as learning to reinforcement learn, one-shot imitation, and the development of efficient and safe meta-reinforcement learning algorithms. Furthermore, it explores meta-learning's contribution to adaptive control in embodied AI, particularly in robotics, showcasing its power in fostering robust and stable autonomy in dynamic physical environments. This section bridges the gap between theoretical advancements and tangible real-world problem-solving across diverse domains.

Finally, the review culminates by exploring modern frontiers and future directions. \textbf{Section 8} examines cutting-edge developments, including the integration of meta-learning with large foundation models for efficient adaptation, its increasing emphasis on robustness, safety, and trustworthy AI systems, and fundamental theoretical refinements aimed at improving generalization. This section highlights the field's progression towards building more capable, reliable, and responsible intelligent systems. Concluding the review, \textbf{Section 9} synthesizes the major advancements, identifies key remaining challenges and open questions, and discusses the broader ethical considerations and societal implications of highly adaptive AI, thereby guiding future research and responsible innovation in this dynamic domain. This comprehensive structure ensures a logical flow, enabling readers to progressively build their understanding of Deep Meta-Learning's evolution, current state, and future potential.


\label{sec:foundational_algorithmic_paradigms:_core_methodological_families}

\section{Foundational Algorithmic Paradigms: Core Methodological Families}
\label{sec:foundational\_algorithmic\_paradigms:\_core\_methodological\_families}

\subsection{Optimization-Based Meta-Learning: Learning an Adaptable Initialization}
\label{sec:2\_1\_optimization-based\_meta-learning:\_learning\_an\_adaptable\_initialization}

Optimization-based meta-learning represents a fundamental paradigm for developing intelligent systems capable of rapid adaptation and efficient knowledge transfer \cite{hospedales2020m37, sutton2022jss}. At its core, this approach seeks to "learn to learn" by acquiring an optimal set of initial model parameters, often referred to as meta-parameters. The objective is to ensure that a model, initialized with these meta-parameters, can be swiftly fine-tuned to novel, unseen tasks with only a minimal number of gradient descent steps \cite{community\_5, community\_7}. This makes models inherently "learnable" by discovering a starting point that is maximally amenable to quick generalization, a capability particularly critical for few-shot learning scenarios where task-specific data is extremely scarce.

This paradigm conceptualizes the meta-learning challenge as a sophisticated bi-level optimization problem. The structure consists of two interdependent optimization loops:
\begin{enumerate}
    \item \textbf{Inner Loop (Task-Specific Adaptation):} This loop focuses on adapting the model's parameters to a specific task $T\_i$ drawn from a distribution of tasks $\mathcal{P}(T)$. Given the meta-learned initialization $\theta$, the model parameters $\phi\_i$ are updated using a few steps of gradient descent (or another optimization algorithm) on the task's support set. The goal here is to achieve good performance on that particular task.
    \item \textbf{Outer Loop (Meta-Optimization):} This higher-level loop optimizes the initial model parameters $\theta$ (the meta-parameters) across a diverse distribution of tasks $\mathcal{P}(T)$. The objective is to find an initialization $\theta$ such that, \textit{after} the inner-loop adaptation, the model performs optimally on the query set of \textit{all} tasks in the distribution. This involves computing gradients with respect to $\theta$ that account for the subsequent inner-loop adaptation process.
\end{enumerate}
This bi-level formulation ensures that the meta-learner is not just optimizing for performance on a single task, but rather for the \textit{ability to adapt quickly} to a wide range of tasks. The meta-parameters thus encode broad, transferable knowledge about the structure of tasks within a given distribution, allowing the model to quickly navigate to effective solutions for diverse new tasks.

The Model-Agnostic Meta-Learning (MAML) algorithm \cite{Finn2017} stands as the seminal work that concretized this optimization-based paradigm. MAML introduced the concept of learning an initialization that is exquisitely sensitive to changes in task-specific parameters. This sensitivity ensures that even a few gradient updates on a new task yield substantial performance improvements. MAML's profound strength lies in its model-agnostic nature, making it applicable to virtually any model architecture that can be trained with gradient descent. It optimizes for parameters that, when updated by one or more gradient steps on a new task, result in maximum performance on that task. This effectively means the meta-learner is learning \textit{how to learn} efficiently, rather than just learning a specific task.

While MAML primarily focuses on learning an adaptable initialization, the optimization-based meta-learning paradigm is broader. It encompasses a family of methods that aim to learn various components of the optimization process itself, not just the starting point. This includes approaches that learn explicit optimization algorithms, such as per-parameter learning rates or update directions, or even entire recurrent neural networks that generate updates \cite{community\_3, community\_8}. These variations demonstrate the flexibility of the bi-level optimization framework, allowing meta-learners to gain finer-grained control over the adaptation process and potentially achieve more robust and efficient learning.

The significance of this paradigm lies in its capacity to address critical challenges in deep learning, particularly data scarcity and slow adaptation. By providing a "good" starting point, optimization-based meta-learning drastically reduces the amount of task-specific data and computational resources required for a model to achieve high performance on new tasks. This capability is paramount for real-world applications where data collection is expensive or time-consuming, and rapid deployment of adaptive AI systems is necessary. The focus on an adaptable initialization enables efficient knowledge transfer across diverse learning scenarios, fostering models that are more flexible and robust to new data and environments, thereby marking a crucial step towards more general artificial intelligence.

In conclusion, optimization-based meta-learning, through its bi-level optimization structure and emphasis on learning an adaptable initialization, offers a powerful and generalizable framework for achieving fast adaptation and few-shot learning. It shifts the focus from merely training a model for a single task to training a model that is itself an efficient learner. While foundational methods like MAML established the core principles, the paradigm continues to evolve, exploring various aspects of the learning process to be meta-learned. The challenges inherent in this paradigm—such as balancing computational efficiency with the expressiveness of the meta-learned initialization, ensuring robustness across highly diverse task distributions, and effectively formulating the inner and outer optimization loops—form the basis for the more detailed algorithmic explorations presented in the subsequent section.
\subsection{Metric-Based Meta-Learning: Learning to Compare and Embed}
\label{sec:2\_2\_metric-based\_meta-learning:\_learning\_to\_compare\_\_and\_\_embed}

Metric-based meta-learning represents a foundational paradigm within deep meta-learning, primarily designed to tackle few-shot learning challenges by enabling models to "learn to compare" \cite{hospedales2020m37}. The core philosophy revolves around acquiring an effective embedding space where novel tasks can be efficiently solved by measuring similarity or distance between examples. Unlike optimization-based methods that focus on learning adaptable model parameters, metric-based approaches train deep neural networks to map raw inputs into a robust feature space. Within this space, a simple distance metric (e.g., Euclidean distance) or a more complex learned comparison function can accurately classify or relate new examples to a small support set. This paradigm emphasizes robust representation learning and efficient comparison mechanisms, making it a powerful tool for rapid categorization in data-scarce scenarios.

The intellectual lineage of metric-based meta-learning can be traced to early work on learning similarity functions, such as Siamese Networks \cite{Bromley1993}. These networks were designed to learn an embedding such that similar inputs were mapped close together, and dissimilar inputs far apart, laying the groundwork for comparison-based learning. Building upon this, foundational meta-learning methods explicitly focused on learning discriminative embedding spaces for few-shot classification. Matching Networks \cite{Vinyals2016} introduced an end-to-end differentiable neural network that learned both an embedding function and an attention mechanism. This attention dynamically weighed the contributions of support examples to classify a query, effectively performing a non-parametric nearest-neighbor classification in a learned metric space. This demonstrated the potential of learning a comparison strategy directly from data.

A subsequent influential contribution was Prototypical Networks \cite{Snell2017}. This method simplified the comparison mechanism by learning an embedding space where each class is represented by a single prototype, computed as the mean of its support examples. Classification of new instances is then performed by assigning them to the class whose prototype is closest in the learned embedding space, typically using Euclidean distance. Prototypical Networks offered improved robustness, interpretability (as prototypes are explicit class representations), and computational efficiency compared to Matching Networks, establishing a highly influential baseline for the paradigm.

A significant conceptual advancement in metric-based meta-learning arrived with Relation Networks \cite{sung2017nc5}. Moving beyond fixed distance metrics or simple attention mechanisms, Relation Networks introduced the idea of learning a deep, non-linear "relation function" to explicitly compute similarity scores between embedded query and support examples. This relation function, often implemented as a convolutional neural network, takes the concatenation of embedded query and support features and outputs a scalar similarity score. This innovation allowed for more flexible and expressive comparisons, enabling the model to capture complex, non-linear relationships between instances. The ability to learn the comparison mechanism itself, rather than just the embedding, marked a crucial step towards more adaptable metric-based meta-learners, demonstrating that the "metric" could be a sophisticated, learned function.

The core advantage of metric-based meta-learning lies in its inference efficiency and direct focus on learning transferable representations. Once the embedding function and comparison mechanism are meta-trained, new tasks can be solved rapidly with minimal computational overhead during inference, often requiring only a single forward pass to embed examples and compute similarities. This makes them particularly well-suited for scenarios demanding quick adaptation without extensive fine-tuning. Furthermore, the explicit nature of similarity comparison often lends a degree of interpretability to their decisions, especially in prototype-based methods. However, a key challenge for these methods is that the quality and generalizability of the learned embedding space are paramount. If the embedding space is not sufficiently discriminative or robust across a diverse distribution of tasks, the performance can degrade. Moreover, while learning a relation function offers more flexibility than fixed metrics, it still relies on the learned feature space, and its expressiveness might be limited for extremely complex or novel task relationships. These considerations highlight the trade-offs between inference-time efficiency and the capacity to model highly intricate task structures, which motivates further advancements in the field.

In summary, metric-based meta-learning provides a powerful framework for few-shot learning by focusing on learning effective embedding spaces and comparison mechanisms. From the early concepts of similarity learning to the development of learned prototypes and adaptable relation functions, this paradigm has established itself as a cornerstone for rapid, data-efficient classification. While foundational, the ongoing research continues to refine these methods, exploring ways to enhance representation quality, address limitations of embedding spaces, and develop more sophisticated, context-aware comparison strategies, which will be explored in greater detail in subsequent sections.
\subsection{Model-Based Meta-Learning: Learning with Memory and Explicit Architectures}
\label{sec:2\_3\_model-based\_meta-learning:\_learning\_with\_memory\_\_and\_\_explicit\_architectures}

Model-based meta-learning represents a distinct paradigm that focuses on designing specific network architectures intrinsically capable of rapid adaptation or processing new task information, often without explicit gradient updates in an inner loop. This approach contrasts with optimization-based methods by embedding the meta-learning capability directly into the model's structure, leveraging mechanisms like external memory modules or integrated attention and temporal convolutions for efficient in-context learning.

One of the pioneering works in this area is the Memory-Augmented Neural Network (MANN) \cite{santoro2016323}. MANN introduced the concept of augmenting a neural network with an external memory module, inspired by Neural Turing Machines, to store and retrieve task-specific information. This architecture learns to read from and write to its memory based on the current task context, enabling it to rapidly adapt to new tasks and perform one-shot learning by directly recalling relevant experiences rather than through parameter updates. While groundbreaking in demonstrating the power of external memory for meta-learning, the complexity of training and the scalability of these explicit memory access mechanisms can be a significant challenge.

Addressing some of the complexities associated with explicit external memory, \cite{Mishra2018} proposed A Simple Neural Attentive Meta-Learner (SNAIL). SNAIL leverages a combination of temporal convolutions and attention mechanisms to process sequences of experience, allowing for efficient in-context learning. By integrating these standard deep learning components, SNAIL's architecture can directly process a support set and produce predictions for query examples, effectively learning to adapt by attending to relevant past information and aggregating temporal context without requiring explicit gradient-based fine-tuning. This approach offers a more streamlined and often more practical alternative for achieving rapid adaptation compared to the more intricate control mechanisms of memory-augmented networks.

Another significant architectural innovation within model-based meta-learning is the Conditional Neural Process (CNP) \cite{Garnelo2018}, which introduces a probabilistic framework for meta-learning. CNPs are designed to learn a distribution over functions, enabling them to not only make predictions but also quantify uncertainty, a crucial aspect for robust decision-making. The architecture consists of an encoder that processes a set of context points (support set) to produce a latent representation, and a decoder that uses this representation to predict the distribution of target points (query set). This allows the model to directly infer properties of a function from a few examples, effectively "processing new task information" through its learned function approximation capabilities. While differing in its probabilistic objective, CNP exemplifies a model-based approach by using a specialized architecture to directly learn and represent functions from data, providing a distinct path to meta-learning that offers valuable uncertainty estimates not typically available in MANN or SNAIL.

In summary, model-based meta-learning has evolved through architectural innovations that empower models to adapt rapidly without relying on inner-loop gradient optimization. From the pioneering use of external memory in MANN \cite{santoro2016323} to the integration of temporal convolutions and attention in SNAIL \cite{Mishra2018} for efficient in-context learning, and the probabilistic function approximation of CNPs \cite{Garnelo2018}, this paradigm explores how intrinsic architectural designs can confer meta-learning capabilities. A common challenge across these approaches lies in balancing the architectural complexity with scalability to very complex tasks and large datasets, and ensuring that the learned architectures are sufficiently generalizable to truly novel tasks. Future directions may involve hybrid models that combine the strengths of these architectural innovations with elements from other meta-learning paradigms to achieve even more robust and versatile meta-learners.


\label{sec:deep_dive_into_optimization-based_meta-learning}

\section{Deep Dive into Optimization-Based Meta-Learning}
\label{sec:deep\_dive\_into\_optimization-based\_meta-learning}

\subsection{Model-Agnostic Meta-Learning (MAML) and its Core Principles}
\label{sec:3\_1\_model-agnostic\_meta-learning\_(maml)\_\_and\_\_its\_core\_principles}

Deep learning models typically require extensive data and training to achieve high performance on new tasks, limiting their applicability in few-shot scenarios. Model-Agnostic Meta-Learning (MAML) \cite{finn2017model} emerged as a foundational contribution to address this challenge by enabling deep networks to rapidly adapt to novel tasks with minimal data. Its core idea revolves around learning a set of initial model parameters that are highly adaptable, meaning that only a few gradient steps on a new task will lead to significant performance improvement.

The fundamental strength of MAML lies in its bi-level optimization structure, which explicitly trains a model to be easily fine-tuned. The process involves an inner loop and an outer loop \cite{finn2017model}. In the inner loop, the model's parameters are updated for a specific task using a small amount of task-specific data (the support set). This adaptation step typically involves one or a few gradient descent updates. The key insight is that the parameters resulting from this inner-loop adaptation are then evaluated on a separate query set for the \textit{same task}. The outer loop then optimizes the initial meta-parameters (the starting point for adaptation) by minimizing the loss on these query sets, averaged across a distribution of diverse tasks. This meta-optimization effectively teaches the model "how to learn" by finding an initialization that is maximally sensitive to small changes, allowing for quick and effective adaptation to unseen tasks.

A defining characteristic of MAML is its model-agnostic nature \cite{finn2017model}. This means the algorithm can be applied to virtually any model architecture that is differentiable, including convolutional neural networks for image classification, recurrent neural networks for sequence prediction, or policy networks for reinforcement learning. This generality is a significant advantage, as it decouples the meta-learning algorithm from specific model designs, making it widely applicable. Furthermore, MAML's principles extend across various task types, from few-shot image classification and regression to more complex domains like reinforcement learning, where agents can learn new skills or adapt to new environments quickly.

The profound influence of MAML's core principles—learning an adaptable initialization through bi-level optimization—is evident in subsequent research that leverages meta-learning for rapid adaptation across diverse applications. For instance, in deep metric learning, where the goal is to learn embedding functions that map similar items close together and dissimilar items far apart, MAML-inspired approaches have been developed to learn adaptive strategies. \textcite{chen2019oep} introduced Deep Meta Metric Learning (DMML), which formulates metric learning in a meta way by sampling subsets and learning set-based distances. This approach echoes MAML's bi-level structure by learning metrics across different sub-tasks, thereby enhancing generalization for visual recognition tasks like person re-identification.

Building on this, \textcite{zheng20200ig} proposed Deep Metric Learning via Adaptive Learnable Assessment (DML-ALA), which formulates the learning of a sample assessor as a meta-learning problem. This method employs an episode-based training scheme, updating the assessor at each iteration to adapt to the current model status, using the performance of a one-gradient-updated metric on a validation subset as the meta-objective. This directly reflects MAML's inner-outer loop paradigm for learning an adaptive component (the assessor) that guides the training process. Further extending these ideas, \textcite{jiang20220tg} introduced a meta-mining strategy with semiglobal information (MMSI) for deep metric learning. This method applies meta-learning to learn to weight samples throughout training, incorporating richer information from validation sets and a memory dictionary, showcasing how MAML's adaptive learning framework can be scaled and refined for more robust sample selection.

Beyond supervised learning, MAML's paradigm for rapid adaptation has also found utility in control systems and robotics. For example, \textcite{oconnell2022twd} developed Neural-Fly, which uses a meta-learning algorithm called domain adversarially invariant meta-learning (DAIML) for rapid online adaptation in uninhabited aerial vehicles (UAVs) operating in strong winds. This approach learns a shared representation that allows for quick adaptation to specific wind conditions by updating a small set of linear coefficients, demonstrating MAML's principle of learning an adaptable basis for fast, real-time control.

In conclusion, MAML established a powerful and generalizable framework for meta-learning by focusing on the optimization of an initial parameter state that facilitates rapid adaptation to new tasks. Its bi-level optimization structure and model-agnostic nature have made it a cornerstone algorithm, inspiring numerous extensions and applications across various domains, from metric learning to robotic control. While MAML's computational cost due to second-order gradients was an initial limitation, its core principle of learning an adaptable initialization remains a central theme in the pursuit of more flexible and efficient learning systems.
\subsection{Enhancements for Efficiency and Scalability: First-Order Approximations}
\label{sec:3\_2\_enhancements\_for\_efficiency\_\_and\_\_scalability:\_first-order\_approximations}

The widespread adoption of Model-Agnostic Meta-Learning (MAML) \cite{Finn\_MAML\_2017} in deep learning was initially hampered by significant computational and memory demands. As discussed in Section 3.1, MAML's core principle of learning an adaptable initialization relies on a bi-level optimization problem, where the outer loop meta-optimizes the initial parameters by propagating gradients through the inner-loop task-specific adaptation. This process inherently requires the computation of second-order derivatives (Hessian-vector products or full Hessians), leading to substantial computational cost, high memory consumption, and increased implementation complexity, particularly for large models and datasets \cite{franceschi2018u1q}. Addressing these practical limitations became a critical frontier for making gradient-based meta-learning more scalable and accessible.

The most direct approach to mitigate MAML's computational burden was the introduction of \textbf{First-Order MAML (FOMAML)}, proposed within the original MAML paper \cite{Finn\_MAML\_2017}. FOMAML simplifies the meta-gradient computation by simply ignoring the second-order derivative terms, effectively treating the inner-loop adaptation as a fixed operation when computing the outer-loop gradient. While this approximation significantly reduces computational overhead and memory footprint, it can lead to a less accurate meta-gradient, potentially affecting the quality of the learned initialization and overall meta-performance. Despite this trade-off, FOMAML offered an immediate and practical solution, demonstrating that even a simplified gradient signal could yield substantial benefits in meta-learning, especially when full second-order computations were infeasible.

Building upon the need for efficient first-order approximations, \textbf{Reptile} \cite{Nichol\_Reptile\_2018} emerged as a pivotal advancement, offering an alternative and highly effective first-order approach to learning an adaptable initialization. Reptile operates by repeatedly sampling a task, performing several gradient steps on that task to obtain task-specific parameters, and then updating the global meta-parameters (the initialization) by moving them \textit{towards} these task-adapted parameters. Specifically, if $\theta$ are the meta-parameters and $\theta\_i'$ are the parameters adapted to task $i$, the meta-update is $\theta \leftarrow \theta + \alpha (\theta\_i' - \theta)$, where $\alpha$ is the meta-learning rate. This update rule, which can be implemented using only first-order gradients, dramatically improves efficiency and scalability compared to full MAML.

Reptile's elegance lies in its simplicity and its insightful interpretation. It can be viewed as an implicit optimization of a loss surface that encourages the meta-parameters to be a good average starting point for adaptation across tasks. Furthermore, its update rule bears a striking resemblance to algorithms used in federated learning, such as Federated Averaging (FedAvg), where a global model is updated by averaging local model updates \cite{fernando2018lt5}. This connection suggests that Reptile implicitly optimizes for a form of "joint training" across tasks, where the meta-parameters are pulled towards the optima of individual tasks. By eschewing complex second-order derivative computations, Reptile significantly reduces the computational overhead and memory footprint, making meta-learning more accessible and deployable in real-world deep learning scenarios without sacrificing substantial performance.

Beyond FOMAML and Reptile, other strategies have been explored to circumvent the computational cost of second-order derivatives. For instance, \textbf{HyperMAML} \cite{przewiezlikowski2022d4y} proposes replacing MAML's gradient-based inner-loop adaptation with a trainable hypernetwork. This hypernetwork directly generates weight updates based on the support set, effectively learning a more powerful and flexible adaptation mechanism in a single step, without requiring loss calculation or gradient backpropagation during the update phase. This approach offers a different pathway to computational efficiency by learning an explicit update function rather than approximating a gradient. The historical context of meta-gradient methods, as reviewed by \cite{sutton2022jss}, further highlights the long-standing interest in learning meta-parameters efficiently, with MAML and its first-order approximations representing a significant modern chapter in this trajectory.

Despite the significant strides made by first-order approximations like FOMAML and Reptile in enhancing efficiency, the trade-offs between approximation quality and computational savings remain an active area of research. While Reptile offers a compelling balance, its implicit assumptions about the loss landscape and the nature of task distributions warrant further theoretical investigation. For example, recent theoretical work by \cite{bernacchia20211r0} provides exact algebraic expressions for MAML's test loss, revealing that the optimal inner-loop learning rate during meta-training can even be \textit{negative} in certain overparameterized regimes. Such insights challenge conventional assumptions and suggest that the simple positive learning rates used in first-order approximations might not always be optimal. Future research needs to develop more robust and universally applicable first-order meta-learning strategies that can seamlessly scale to the ever-increasing complexity of modern deep learning architectures and diverse task distributions, particularly in scenarios demanding extreme data efficiency and rapid adaptation, while also incorporating deeper theoretical understandings of optimal meta-optimization.
\subsection{Learning Explicit Optimizers and Differentiable Solvers}
\label{sec:3\_3\_learning\_explicit\_optimizers\_\_and\_\_differentiable\_solvers}

Beyond learning a mere initialization, advanced meta-learning techniques focus on explicitly learning the optimization process itself or integrating differentiable solvers as adaptable base learners. These approaches aim to achieve greater flexibility and fine-grained control over the inner-loop adaptation, thereby enhancing the meta-learner's ability to optimize effectively for novel tasks.

The concept of learning an optimization algorithm itself, rather than relying on fixed gradient descent rules, represents a significant leap in meta-learning. This idea is rooted in the theoretical finding by \cite{finn2017vrt} that deep representations combined with standard gradient descent can approximate any learning algorithm, providing a strong universality argument for trainable optimizers. Historically, the broader field of meta-gradient methods, which learn meta-parameters like step sizes, has been explored for years \cite{sutton2022jss}. A seminal contribution to explicitly learning optimizers came from \cite{andrychowicz2016learning}, who proposed using a recurrent neural network (RNN), specifically an LSTM, as a learned optimizer. This LSTM takes gradients and previous hidden states as input and outputs parameter updates, effectively learning an update rule from data. Building on this, \cite{ravi2017optimization} introduced the Meta-Learner LSTM, applying this concept to few-shot learning by training an LSTM to generate updates for a base learner, demonstrating rapid adaptation to new tasks. Another key advancement, Meta-SGD \cite{li2017meta}, further refined this by meta-learning not only the initialization but also per-parameter learning rates and update directions, offering fine-grained control over the inner-loop adaptation process. These early works established the viability of replacing hand-designed optimizers with learned, data-driven counterparts, moving beyond the limitations of fixed update rules.

Further advancements have explored more sophisticated learned optimization mechanisms. \cite{franceschi2018u1q} provided a unifying framework based on bilevel programming, demonstrating how gradient-based hyperparameter optimization and meta-learning can be framed to explicitly account for inner-objective optimization dynamics, which is crucial for understanding learned optimizers. HyperMAML \cite{przewiezlikowski2022d4y} addresses limitations of traditional MAML by replacing its gradient-based inner-loop adaptation with a trainable hypernetwork. This hypernetwork directly generates weight updates in a single step, offering more substantial and efficient modifications without the computational overhead of second-order optimization, and allowing for more flexible, non-gradient-based adaptation. Learned optimizers have proven particularly valuable in challenging domains like continual learning, where mitigating catastrophic forgetting is paramount. \cite{vuorio2018gwb} proposed a meta-continual learning approach where a neural network predicts parameter update steps specifically designed to prevent forgetting. Similarly, \cite{gu2019tvc} explored meta-learning biologically plausible synaptic update rules that integrate neuron-local information, moving towards more interpretable and biologically inspired learning mechanisms. The Feedback and Local Plasticity (FLP) framework introduced by \cite{lindsey202075a} further advanced this by meta-learning decoupled feedback weights and local plasticity rules, demonstrating effective deep credit assignment and superior performance in continual learning compared to gradient-based baselines. For incremental learning, \cite{rajasegaran2020llk} proposed iTAML, a task-agnostic meta-learning approach that introduces a novel meta-update rule to maintain equilibrium across encountered tasks and prevent catastrophic forgetting. While not a learned optimizer itself, the theoretical analysis of MAML's inner-loop learning rate by \cite{bernacchia20211r0}, revealing the counter-intuitive finding that an optimal \textit{negative} learning rate can exist during meta-training in overparameterized models, highlights the complexities of fixed optimization schedules and implicitly motivates the need for adaptive, learned update rules that can transcend such limitations.

An alternative yet complementary direction involves integrating differentiable closed-form or iterative solvers as the inner-loop adaptation mechanism. This approach leverages the stability and efficiency of classical machine learning algorithms while allowing for end-to-end meta-learning. \cite{bertinetto2018ur2} pioneered this by proposing a meta-learning framework where the base learner is a differentiable closed-form solver, such as Ridge Regression or Logistic Regression. By efficiently backpropagating through these solvers using techniques like the Woodbury identity, their method achieved competitive performance in few-shot learning while offering greater flexibility than similarity-based methods and faster convergence than gradient-based ones. Extending this concept to more complex models, \cite{wistuba2021wha} developed Few-Shot Bayesian Optimization with Deep Kernel Surrogates, where a deep kernel network for Gaussian Processes is meta-learned to rapidly adapt to new hyperparameter optimization tasks with limited evaluations. This allows for robust predictions with well-calibrated uncertainty. Building on this, \cite{chen2022z45} introduced Adaptive Deep Kernel Fitting with Implicit Function Theorem (ADKF-IFT) for Deep Kernel Gaussian Processes, which unifies previous approaches by meta-learning a subset of parameters while adapting others per task using exact hypergradients computed via the Implicit Function Theorem. This provides fine-grained control over the adaptation process, balancing generalization and task-specificity effectively for molecular property prediction. Most recently, \cite{lee2024snq} proposed Sequential Bayesian Meta-Continual Learning (SB-MCL), a novel framework that meta-learns neural networks to bridge raw data to simple statistical models that perform \textit{exact} sequential Bayesian updates. This innovative approach inherently prevents catastrophic forgetting in continual learning by offloading sequential knowledge integration to robust statistical models, representing a sophisticated form of learned, differentiable adaptation.

In summary, the progression from learning initializations to explicitly learning optimization processes and integrating differentiable solvers marks a significant evolution in meta-learning. Learned optimizers, from foundational RNN-based approaches to hypernetworks, offer enhanced flexibility and fine-grained control over parameter updates, particularly beneficial in scenarios requiring complex adaptation or resistance to forgetting. Differentiable solvers, on the other hand, provide stability and efficiency by leveraging well-understood classical algorithms within an end-to-end meta-learning framework. These methods collectively represent a powerful paradigm for designing meta-learners that can adapt more effectively and robustly than those relying solely on initial parameter learning. Future research will likely explore hybrid approaches that combine the strengths of learned optimizers with the guarantees of differentiable solvers, extend these techniques to even more complex and safety-critical domains, and further deepen the theoretical understanding of the meta-optimization landscape to unlock new capabilities in adaptable AI systems.


\label{sec:deep_dive_into_metric_and_relation-based_meta-learning}

\section{Deep Dive into Metric and Relation-Based Meta-Learning}
\label{sec:deep\_dive\_into\_metric\_\_and\_\_relation-based\_meta-learning}

\subsection{Learning Discriminative Embeddings and Prototypes}
\label{sec:4\_1\_learning\_discriminative\_embeddings\_\_and\_\_prototypes}

Few-shot learning presents a significant challenge where models must generalize to novel categories with minimal labeled examples, a core problem that metric-based meta-learning aims to address \cite{huisman2020b7w, peng20209of}. This paradigm focuses on learning an effective embedding space where examples from the same class are close, and examples from different classes are far apart, coupled with robust comparison mechanisms. These methods are crucial for few-shot classification as they prioritize learning representations that generalize well to unseen classes, moving beyond rote memorization to capture transferable semantic properties and enable rapid adaptation to new tasks.

A foundational approach in this domain is \textit{Matching Networks} \cite{Vinyals2016}, which introduced an end-to-end differentiable neural network capable of performing one-shot learning. This method learns an embedding function that maps input examples into a feature space. Crucially, it employs an attention mechanism to compare a query example with all examples in a support set. The classification of the query is then determined by a weighted sum of the labels of the support examples, where weights are derived from the learned similarity scores. This effectively functions as a non-parametric classifier, adapting its decision boundary based on the support set. While innovative for its direct, end-to-end learning of a comparison mechanism, the attention over all support examples can become computationally intensive and less efficient for larger support sets, limiting its scalability.

Building upon the concept of learning a discriminative embedding space, \textit{Prototypical Networks} \cite{Snell2017} offered a significant simplification and improvement, directly addressing some of the complexities of Matching Networks. Instead of comparing a query to every individual support example, Prototypical Networks learn an embedding space where each class is represented by a single "prototype." This prototype is typically computed as the centroid (mean) of its support examples in the learned embedding space. Classification then becomes a straightforward nearest-neighbor problem, assigning a query example to the class whose prototype is closest in the embedding space. This approach demonstrated improved robustness and interpretability due to its simpler comparison mechanism and often achieved competitive or superior performance with greater efficiency, making it a highly influential method in few-shot learning.

However, this reliance on a single centroid per class is a strong inductive bias that can be detrimental when representing classes with complex, non-convex, or multi-modal feature distributions in the embedding space. For instance, if a class has two distinct sub-categories, a single centroid might not accurately represent either, leading to misclassifications. To address this single-centroid limitation, subsequent work has explored representing classes with more flexible distributions, such as Gaussian prototypes, thereby capturing intra-class variance more effectively. Furthermore, nearest-neighbor classification, as employed by Prototypical Networks, is susceptible to the "hubness problem" in high-dimensional embedding spaces, where certain prototypes become nearest neighbors to a disproportionately large number of query examples regardless of their true class \cite{fei20211x6}. This phenomenon can degrade classification performance by creating "hubs" that attract many unrelated samples, reducing the discriminative power of the learned embeddings.

Recognizing that the quality of the learned embedding space is paramount for these methods, subsequent research has focused on enhancing the feature extraction process itself. For example, \textit{Deep Meta-Learning: Learning to Learn in the Concept Space} \cite{zhou20188lr} improves upon Matching Networks and other meta-learners by introducing a "concept generator" that extracts high-level representations, allowing the meta-learner to operate in a more abstract and discriminative concept space. Similarly, \textit{Self-supervised Knowledge Distillation for Few-shot Learning} \cite{rajasegaran2020glw} proposes a two-stage process to improve representation capacity, first maximizing entropy of the feature embedding with a self-supervised auxiliary loss, then minimizing it via distillation, demonstrating that robust feature learning alone can significantly boost few-shot performance. The importance of the backbone network's complexity in achieving high performance in few-shot classification has also been empirically noted, suggesting that larger, more powerful feature extractors often lead to better embeddings \cite{huisman2020b7w}.

The effectiveness of these prototype-based and attention-based methods is highly dependent not only on the quality of the embedding function but also on the selection of support examples and the robustness of the metric learning process itself. Consequently, a significant line of research has focused on meta-learning the metric learning process to be robust to sub-optimal sampling and to adaptively refine the comparison mechanism. \textit{Deep Meta Metric Learning (DMML)} \cite{chen2019oep} formulates metric learning in a meta-way, sampling subsets from the training data to learn robust metrics across different sub-tasks. It introduces the concept of set-based distance and hard sample mining for these sets, demonstrating that common loss functions can be made consistent in a meta-space. This approach aims to learn a metric that is inherently more robust and generalizable by optimizing the metric learning process itself across diverse meta-tasks.

Addressing the challenge of sub-optimal sample selection, which is critical for effective metric learning, \textit{Deep Metric Learning via Adaptive Learnable Assessment (DML-ALA)} \cite{zheng20200ig} proposes a meta-learning approach to learn an adaptive sample assessment strategy. This method employs a sequence-aware learnable assessor that re-weights each training example to maximize the generalization of the trained metric. By adapting to the current model status through an episode-based training scheme, DML-ALA ensures that the most informative samples contribute more to the metric learning process, thereby improving the discriminative power of the learned embeddings. Expanding on this, \textit{Deep Metric Learning Based on Meta-Mining Strategy With Semiglobal Information (MMSI)} \cite{jiang20220tg} further refines the meta-mining strategy. It incorporates richer "semiglobal information" from a validation set and a memory dictionary, allowing for a more comprehensive and adaptive sample weighting during training. This approach aims to overcome the limitations of relying solely on information from a single training batch, leading to state-of-the-art performance by providing a more informed and stable meta-learning signal for sample selection.

The utility of learning discriminative embeddings and prototypes extends to more complex and challenging tasks beyond simple image classification. For instance, \textit{Few-Shot Cross-Domain Object Detection With Instance-Level Prototype-Based Meta-Learning (IPNet)} \cite{zhang2024mf0} applies instance-level prototype learning in a challenging cross-domain object detection scenario. IPNet fuses cropped instances from both source and target domains to learn representative prototypes for each class, demonstrating how prototype-based methods can be effectively adapted to compensate for data deficiency and mitigate domain shifts. Similarly, for few-shot short utterance speaker verification, a meta-learning approach integrating an Emphasized Channel Attention, Propagation and Aggregation in TDNN (ECAPA-TDNN) within a Prototypical Network (ETP) has shown superior performance by combining a robust feature extractor with an episodic training strategy that includes a global classification objective \cite{wang2023x5w}. These applications highlight the versatility of the core concept in pushing the boundaries of few-shot learning into more complex recognition and verification tasks.

In conclusion, the evolution of learning discriminative embeddings and prototypes has progressed from foundational attention-based comparison mechanisms in Matching Networks to simplified, efficient, and interpretable prototype-based representations in Prototypical Networks. While these methods established the efficacy of learning robust embedding spaces, critical analysis reveals limitations such as the strong inductive bias of single centroids and the hubness problem inherent in nearest-neighbor classification. Subsequent advancements have focused on enhancing the quality of the embedding function itself and, more recently, on meta-learning the metric learning process through adaptive sample assessment and mining strategies (e.g., DMML, DML-ALA, MMSI). While these methods have significantly advanced few-shot classification by focusing on robust representations and efficient comparison, challenges remain in scaling to extremely diverse tasks, balancing the simplicity and expressiveness of comparison mechanisms, and developing even more adaptive metric learning strategies that can generalize across a broader spectrum of data distributions and task complexities, particularly in scenarios with significant domain shifts or highly imbalanced data.
\subsection{Learning Non-Linear Relation Functions for Comparison}
\label{sec:4\_2\_learning\_non-linear\_relation\_functions\_for\_comparison}

The effectiveness of metric-based meta-learning in few-shot scenarios critically depends on its capacity to accurately quantify similarity between examples. While foundational methods like Prototypical Networks \cite{community\_8} and Matching Networks \cite{community\_5} learn robust embedding spaces, they often rely on fixed distance metrics (e.g., Euclidean or cosine similarity) or implicit attention mechanisms for comparison. This reliance can be a significant limitation, as complex, non-linear relationships inherent in diverse data may not be adequately captured by simple, pre-defined metrics \cite{hospedales2020m37, peng20209of}. This challenge spurred a pivotal advancement in the field: the development of methods that explicitly learn deep, non-linear "relation functions" to compute similarity scores, thereby enabling more flexible and expressive comparisons. This approach fundamentally shifts the focus from merely learning better representations for a fixed metric to learning the comparison mechanism itself.

A seminal contribution in this area is the \textbf{Relation Network (RN)}, proposed by \cite{sung2017nc5}. This work introduced an end-to-end meta-learning framework designed to move beyond fixed or linear similarity measures. The RN architecture comprises two main components: an embedding module ($f\_\phi$) and a relation module ($g\_\psi$). The embedding module extracts feature maps from both query and support examples. Crucially, these feature maps are then concatenated and fed into the relation module ($g\_\psi$), which is typically a deep convolutional neural network (CNN) or a multi-layer perceptron (MLP). This $g\_\psi$ learns to output a scalar "relation score" (between 0 and 1) that directly indicates the similarity between the input pair. The network is meta-trained with a Mean Squared Error (MSE) loss, regressing scores to 1 for matching pairs and 0 for non-matching pairs. This explicit learning of a non-linear comparison function allows the model to discover intricate relational patterns that simple Euclidean distances cannot, demonstrating superior performance in few-shot and zero-shot classification by adapting the comparison mechanism itself \cite{sung2017nc5}. Unlike Matching Networks, which use an attention mechanism to weigh support examples, the RN explicitly learns a \textit{function} to compute a similarity score for \textit{any given pair} of embedded examples, offering a more direct and adaptable comparison.

The core principle of learning an explicit, non-linear relation function has been successfully extended and adapted across various domains and data types. For instance, \cite{gao2020h75} applied the Relation Network paradigm to hyperspectral image few-shot classification. Their method leverages the spatial-spectral information inherent in hyperspectral images, where the embedding module extracts rich features, and the subsequent relation learning module performs comparisons. By meta-training with a task-based learning strategy, their approach demonstrates excellent generalization ability, achieving satisfactory classification results with limited labeled samples. This highlights the transferability of the RN's core idea to specialized data modalities, where the learned relation function can capture complex spectral and spatial similarities.

Further advancing the sophistication of learned relation functions, \cite{zhang2024ycr} introduced PG-DERN for few-shot molecular property prediction. This method employs a dual-view encoder for molecular representation and, more relevantly to this discussion, a novel \textbf{relation graph learning module}. Instead of simply concatenating feature vectors, this module constructs a "relation graph" based on learned similarities between molecules. This allows for more efficient information propagation and a richer understanding of complex, multi-faceted relationships between molecular structures, which are crucial for accurate property prediction. While PG-DERN also incorporates a MAML-based meta-learning strategy, its innovation in learning a structured, graph-based relation function for comparison represents a significant evolution from the simpler CNN/MLP relation modules of earlier RNs, showcasing how the concept of "relation learning" can be extended to model intricate, non-Euclidean structures between instances.

The evolution from fixed distance metrics to learned, non-linear relation functions marks a critical paradigm shift in metric-based meta-learning. The foundational Relation Networks \cite{sung2017nc5} provided a robust framework for explicitly learning similarity scores, demonstrating that a deeper, learned comparator offers greater generalizability than pre-defined metrics. Subsequent works have built upon this by adapting the relation learning module to diverse data types, such as hyperspectral images \cite{gao2020h75}, and by introducing more sophisticated, structured relation learning mechanisms, like graph-based approaches for molecular data \cite{zhang2024ycr}. This trajectory underscores a continuous drive towards more flexible, expressive, and domain-aware comparison mechanisms, enabling meta-learning models to capture increasingly intricate patterns and adapt effectively to novel tasks and domains. Future research in this area may explore even more complex, hierarchical relation structures, integrate attention mechanisms more deeply within the relation module, or investigate how these learned comparison functions can be made more interpretable.
\subsection{Self-Supervised and Prompt-Based Enhancements for Few-Shot Learning}
\label{sec:4\_3\_self-supervised\_\_and\_\_prompt-based\_enhancements\_for\_few-shot\_learning}

Few-shot learning, a critical paradigm for enabling models to generalize from minimal examples, has seen significant advancements through the strategic integration of meta-learning with auxiliary techniques such as self-supervised learning (SSL) and prompt-based adaptation for large pre-trained models. These innovations aim to enrich learned feature representations and improve generalization to novel classes by leveraging broader data sources and sophisticated adaptation strategies. Within the context of metric and relation-based meta-learning, these enhancements are particularly impactful as they directly improve the quality of the embedding space and the comparison mechanisms that are central to these approaches.

The core challenge in few-shot learning is to learn robust, generalizable representations from extremely limited labeled data. Self-supervised learning offers a powerful solution by exploiting vast amounts of unlabeled data to pre-train feature encoders, thereby acquiring general knowledge and richer embeddings that are more amenable to subsequent few-shot adaptation. This approach aligns well with the principles of metric-based meta-learning, where the quality of the learned embedding space directly dictates the effectiveness of similarity comparisons. For instance, early work recognized the importance of learning in a "concept space" where instances are represented by high-level concepts, allowing meta-learners to perform few-shot learning more effectively \cite{zhou20188lr}. Similarly, \cite{rajasegaran2020glw} demonstrated that self-supervised knowledge distillation can significantly improve the representation capacity of deep neural networks for few-shot learning by maximizing feature embedding entropy and then constraining it via student-teacher distillation. Such robust representations are crucial for preventing meta-learners from merely memorizing training tasks and instead encouraging data-driven adaptation to novel tasks \cite{yin2019cct}.

A prime example of this integration in text classification is Self-supervised Information Enhanced Meta-Learning (SEML) \cite{li2023zn0}. SEML addresses the limitation of relying solely on labeled few-shot examples by integrating self-supervised learning from unlabeled data. It employs a two-stage pipeline: first, a feature encoder (e.g., BERT) is trained using self-guided contrastive learning on unlabeled samples to acquire general knowledge and robust representations. Subsequently, a novel knowledge distillation method is introduced to expand and enrich the meta-learning representation with this self-supervised information. This is coupled with a graph aggregation method that generates more discriminative feature representations within each task by facilitating effective interaction between support and query sets \cite{li2023zn0}. This approach effectively fuses self-supervised knowledge into a unified distribution, yielding a meta-network with superior generalization capabilities for few-shot text classification, by providing a more informed and robust embedding space for metric-based comparisons.

While self-supervised methods like SEML primarily focus on building better feature extractors from the ground up using unlabeled data, a parallel and increasingly vital approach for massive pre-trained models, such as Vision-Language Models (VLMs) and Large Language Models (LLMs), focuses on efficiently steering their existing, powerful representations. This is where prompt-based enhancements, informed by meta-learning, come into play. The advent of these large foundation models has opened new avenues for few-shot learning, as their vast pre-training enables strong zero-shot and few-shot capabilities \cite{peng20209of}. Empirical observations suggest that larger backbone networks often correlate with better few-shot performance \cite{huisman2020b7w}, and even simple fine-tuning can be highly effective \cite{baz2022n78}. However, efficiently adapting these massive models to novel tasks with limited data without overfitting remains a significant hurdle.

Prompt tuning, which involves learning small, task-specific context vectors to guide VLMs or LLMs, offers a parameter-efficient adaptation strategy. Yet, existing prompt tuning methods, such as Context Optimization (CoOp), often suffer from severe overfitting to base classes and exhibit poor generalization to novel classes \cite{wang2024dai}. This limitation arises because the limited capacity of adaptable prompt vectors can lead to text knowledge being overly tailored to base classes, hindering transferability.

To mitigate these issues, meta-learning has been strategically applied to prompt tuning. The "Learning to Learn" (LoL) method \cite{wang2024dai} proposes a meta-learning-informed prompt tuning approach that treats the process as a meta-learning problem. By adopting an N-way K-shot episodic training framework for learning prompts, LoL explicitly addresses the base-to-new generalization problem. This two-stage strategy, combining initial CoOp-based prompt tuning with a subsequent meta-learning stage on base classes, effectively organizes input data during training. This enables the limited capacity of adaptable prompt vectors to better absorb and transfer knowledge to novel classes by minimizing overfitting, thereby improving the quality of the task-specific representations generated by the VLM for few-shot classification \cite{wang2024dai}. Such meta-training frameworks are also crucial for improving generalization in cross-domain few-shot learning, where tasks come from unseen domains \cite{tian2023iyh}.

In essence, both self-supervised and prompt-based enhancements leverage meta-learning to push the boundaries of few-shot performance, albeit through different mechanisms. Self-supervised learning, exemplified by SEML, focuses on building more robust and discriminative feature representations from unlabeled data, which directly benefits metric-based meta-learning by providing a superior embedding space for similarity comparisons. Prompt-based methods, such as LoL, on the other hand, focus on efficiently adapting the powerful, pre-existing representations of large foundation models. Here, meta-learning provides the "learning to learn" framework \cite{finn2017vrt} that enables the prompts to generalize effectively to novel tasks, rather than overfitting to base classes. The synergy between these paradigms represents a powerful direction for few-shot learning, enriching feature representations through broader data sources and enabling more effective, less overfitting adaptation of large pre-trained models. Future research will likely focus on optimizing the computational overhead of these integrated systems, developing more robust meta-objectives for highly diverse tasks, and exploring novel ways to combine the vast knowledge of foundation models with the rapid adaptation capabilities of meta-learning.


\label{sec:deep_dive_into_model-based_meta-learning:_architectures_for_rapid_adaptation}

\section{Deep Dive into Model-Based Meta-Learning: Architectures for Rapid Adaptation}
\label{sec:deep\_dive\_into\_model-based\_meta-learning:\_architectures\_for\_rapid\_adaptation}

\subsection{Recurrent Neural Network (RNN) Based Meta-Learners}
\label{sec:5\_1\_recurrent\_neural\_network\_(rnn)\_based\_meta-learners}

Recurrent Neural Networks (RNNs), particularly architectures like Long Short-Term Memory (LSTM) networks and Gated Recurrent Units (GRUs), represent an early and distinct paradigm within model-based meta-learning. These approaches leverage the inherent sequential processing capabilities and internal memory of RNNs to achieve rapid adaptation to new tasks without relying on explicit gradient-based inner loops. The core idea is that an RNN can maintain a dynamic internal state that implicitly encodes task-specific information, allowing it to adapt its behavior or generate task-specific parameters by simply processing a few examples sequentially \cite{community\_9, community\_10, layer\_1}. This enables RNNs to function as meta-optimizers, learning to generate model parameters or update rules, or as meta-generators of flexible behavior, offering a unique pathway to meta-learning and flexible behavior generation \cite{hospedales2020m37, huisman2020b7w}.

A foundational contribution to using RNNs as meta-optimizers is the work by \cite{andrychowicz2016learning}. This seminal paper demonstrated that an LSTM network could be trained to learn an optimization algorithm itself. In this framework, the LSTM takes as input the gradients of a base learner's parameters with respect to its loss function, along with its current parameters, and outputs the updated parameters. By processing this sequence of gradient information, the LSTM implicitly learns an effective update rule that can generalize across different optimization problems. This approach frames the optimization process as a sequential decision-making task, where the RNN's internal state dynamically captures the optimization history and guides subsequent parameter updates, effectively replacing hand-designed optimizers like Adam or SGD with a learned one.

Building on this concept, the Meta-Learner LSTM \cite{Ravi2017} specifically applied the idea of a learned optimizer to the few-shot classification setting. Here, an LSTM network is trained to process a small support set for a new task and then generate the parameter updates for a separate "learner" network. The LSTM's internal state accumulates information about the current task, allowing it to produce effective parameter adjustments for the learner network after only a few examples. This method bypasses the need for explicit second-order derivatives or extensive inner-loop gradient steps, as the adaptation logic is entirely encapsulated within the meta-learner LSTM's learned dynamics. The meta-learner implicitly learns how to initialize and update the learner network's weights to rapidly converge on new tasks.

Beyond parameter optimization, RNNs have also been shown to "learn to reinforcement learn" \cite{wang20167px}. In this deep meta-reinforcement learning framework, an LSTM-based agent is trained across a distribution of related reinforcement learning tasks. The LSTM processes sequences of previous actions, rewards, and observations as inputs. Through this meta-training, the RNN's recurrent dynamics implicitly learn an entire reinforcement learning algorithm, including exploration strategies and policy update rules. The agent's internal state dynamically encodes the current task context and accumulated experience, enabling rapid adaptation to new tasks within an episode without requiring explicit weight changes of the policy network during adaptation. This showcases the RNN's remarkable capacity to implicitly encode and execute complex learning procedures, effectively becoming a meta-policy that adapts its behavior based on in-context information.

More recently, RNN-based meta-learning has found applications in complex spatio-temporal prediction tasks, demonstrating their utility in adapting to dynamic real-world data. For instance, ST-MetaNet \cite{pan2019pue} for urban traffic prediction utilizes a sequence-to-sequence architecture where both the encoder and decoder incorporate "meta recurrent neural networks" alongside meta graph attention. These meta RNN components are specifically designed to capture and adapt to diverse temporal correlations present in traffic data. By processing historical traffic information and making step-by-step predictions, the RNNs enable the model to dynamically account for varying spatio-temporal patterns across different locations and times, demonstrating their capacity to generate context-specific predictions and adapt to complex, evolving dynamics.

While RNN-based meta-learners offer a powerful and distinct pathway to meta-learning, they come with specific challenges. A primary concern is the "black-box" nature of the learned adaptation rule \cite{hospedales2020m37}. Unlike explicit gradient-based methods, it is often difficult to interpret \textit{what} optimization strategy or learning algorithm the RNN has implicitly learned, making analysis, debugging, and theoretical guarantees challenging. Furthermore, training these meta-architectures can be computationally intensive, especially when backpropagating through long unrolled optimization paths or extensive sequences of task data. The learned optimizers or adaptation strategies may also exhibit instability or poor generalization when faced with tasks significantly different from those seen during meta-training, leading to potential optimization instability on out-of-distribution tasks \cite{huisman2020b7w}. Comparative studies, such as \cite{finn2017vrt}, have also suggested that while recurrent models are expressive, gradient-based meta-learning can sometimes lead to learning strategies that generalize more widely, prompting a deeper understanding of the expressive power and generalization capabilities of each paradigm.

In conclusion, RNN-based meta-learners provide a compelling approach to meta-learning by leveraging their inherent ability to process sequential information and maintain an internal state for implicit adaptation. They have demonstrated success in learning optimization algorithms \cite{andrychowicz2016learning, Ravi2017} and entire reinforcement learning procedures \cite{wang20167px}. However, challenges related to interpretability, computational cost, and robust generalization remain. Future research may explore hybrid architectures that combine the strengths of RNNs with other meta-learning paradigms, such as attention mechanisms or more explicit memory architectures, to enhance their flexibility, scalability, and transparency.
\subsection{Memory-Augmented Neural Networks (MANNs)}
\label{sec:5\_2\_memory-augmented\_neural\_networks\_(manns)}

Model-based meta-learning aims to design neural network architectures that are intrinsically capable of rapid adaptation and in-context learning, moving beyond simple parameter adaptation. A prominent approach within this paradigm involves Memory-Augmented Neural Networks (MANNs), which integrate external memory modules to store and retrieve task-relevant information in a differentiable manner. This architectural innovation allows meta-learners to leverage explicit memory mechanisms for rapid adaptation and few-shot learning.

The foundational concept of MANNs draws inspiration from the idea of equipping neural networks with a "working memory" similar to a computer, enabling them to read from and write to an external data store. Key architectures in this domain include Neural Turing Machines (NTMs) \cite{graves2014neural} and Differentiable Neural Computers (DNCs) \cite{graves2016hybrid}. These models extend traditional recurrent neural networks by coupling them with a large, addressable memory matrix. The network learns to control memory access through differentiable read and write heads, allowing gradient-based optimization of both the network's internal parameters and its memory interaction strategies. This differentiable nature is crucial, as it permits end-to-end training of the entire system, enabling the meta-learner to acquire sophisticated strategies for information storage and retrieval tailored to new tasks.

A seminal work in applying this concept to meta-learning is the Memory-Augmented Neural Network (MANN) proposed by \cite{santoro2016323}. This paper demonstrated how an external memory module, inspired by NTMs, could enable a neural network to perform one-shot learning. The MANN learns to store representations of individual examples or features encountered during a new task in its external memory. When presented with a query example from the same task, the network can then retrieve relevant information from memory to make an accurate prediction, effectively "remembering" the task-specific knowledge without requiring extensive gradient updates to its internal weights. This mechanism facilitates rapid adaptation by allowing the meta-learner to leverage an explicit, episodic memory of past experiences within the current task.

The integration of external memory in MANNs offers several distinct advantages for meta-learning. Firstly, it provides a mechanism for sophisticated in-context learning, where the network can process a sequence of examples from a new task and dynamically update its internal state and external memory to build a rich understanding of the task's specifics. This allows for knowledge transfer across diverse tasks, as the network learns general strategies for how to encode and retrieve useful information, rather than just learning task-specific parameters. Secondly, MANNs can mitigate catastrophic forgetting, a common challenge in traditional neural networks, by storing distinct task-relevant information in separate memory locations or by learning to selectively overwrite memory. This enables the network to quickly adapt to new tasks while retaining knowledge from previous ones.

Despite their powerful capabilities, MANNs face inherent challenges. The complexity of training memory-augmented networks, particularly in optimizing the intricate interactions between the controller network and the memory module, can be substantial. Furthermore, the scalability of memory access mechanisms and the overall memory size can become a limitation for very large or complex datasets, as managing and searching through vast memory stores can be computationally intensive. These challenges have led to continued research into more efficient memory architectures and alternative model-based meta-learning approaches that leverage different forms of in-context learning, such as those based on attention and temporal convolutions, or probabilistic function approximation, to achieve rapid adaptation with potentially simpler or more scalable designs.
\subsection{Transformer-Based and Other Explicit Architectural Designs}
\label{sec:5\_3\_transformer-based\_\_and\_\_other\_explicit\_architectural\_designs}

Building upon the model-based meta-learning paradigm, this subsection focuses on explicit architectural designs that intrinsically support rapid adaptation and in-context learning without relying on explicit gradient updates during task-specific adaptation. These approaches prioritize designing models whose very structure facilitates swift knowledge integration and adaptation. We explore Transformer-based architectures leveraging attention mechanisms, hypernetworks for dynamic parameter generation, and modular networks for context-dependent component selection.

The Transformer architecture, renowned for its powerful self-attention mechanism, has become a cornerstone for achieving in-context learning and emergent meta-learning capabilities. Unlike optimization-based methods that perform explicit gradient updates for task-specific adaptation, Transformers can process a support set (task examples) as part of their input sequence. This allows the model to implicitly learn a task-specific function or adapt its behavior purely through forward passes, effectively inferring the task from context \cite{d8d680aea59295c020b9d53d78dd8d954a876845, 282a380fb5ac26d99667224cef8c630f6882704f}. This phenomenon is particularly prominent in large language models (LLMs) and Vision Transformers (ViTs), where few-shot prompting—concatenating task instructions and input-output examples to the query—enables generalization to new inputs for the same task without any weight updates. The self-attention mechanism dynamically weighs the relevance of different input tokens, allowing the model to 'attend' to support examples to infer task structure and adapt predictions for the query, mimicking a meta-learning process. This implicit adaptation is a key characteristic that aligns Transformers with model-based meta-learning.

Beyond these emergent properties, Transformers are also explicitly integrated into meta-learning frameworks. For instance, \cite{kumar2024he9} proposes a Transformer-based aggregation function within a meta-learner specifically for real-world class incremental learning. This design concurrently processes the entire support and query sets, using attention to prioritize crucial samples and enhance the impact of relevant information during inference. Crucially, this method enables the meta-learner to complete class incremental tasks without retraining or explicit gradient updates to its core parameters during the incremental learning phase, demonstrating strong generalization even without specific training for this exact task. The strength of Transformer-based meta-learners lies in their ability to scale to vast datasets, learn highly expressive representations, and inherently handle sequential or contextual information. However, their computational cost, especially for very long contexts, and the heavy reliance on extensive pre-training data remain significant considerations for broader applicability, particularly in data-scarce meta-learning scenarios where pre-training on massive datasets is not feasible. Moreover, the 'black-box' nature of how Transformers perform in-context adaptation can hinder interpretability compared to more explicit parameter generation or module selection mechanisms.

Hypernetworks represent another explicit architectural design enabling rapid, gradient-free adaptation by dynamically generating parameters for a 'main' network. In this paradigm, a meta-learner (the hypernetwork) produces task-specific weights or biases for a base model based on a given task description or support set. This mechanism allows for adaptation without requiring explicit gradient descent on the base model's parameters during the inner loop, shifting the learning burden from parameter optimization to parameter generation.

A prominent example is HyperMAML \cite{przewiezlikowski2022d4y}, which generalizes Model-Agnostic Meta-Learning (MAML) by replacing its gradient-based inner-loop adaptation with a trainable hypernetwork. While MAML relies on a few gradient steps to adapt a universal initialization, HyperMAML employs a hypernetwork to directly output weight updates ($\Delta\theta$) for the base model. This hypernetwork takes as input encoded support set embeddings, the base model's predictions on the support set, and the ground-truth labels. The critical innovation is that the hypernetwork generates these updates in a single step, \textit{without requiring loss calculation or gradient backpropagation during the task-specific adaptation phase}. The task-specific parameters are simply computed as $\theta' = \theta + \Delta\theta$. This approach offers a more powerful and efficient adaptation mechanism, allowing for more significant and flexible weight modifications than MAML's limited gradient steps, while also being more computationally efficient by avoiding second-order optimization. Although the meta-training process still optimizes the hypernetwork and universal weights via gradient descent, the \textit{task-specific adaptation} itself is gradient-free for the base model. This design addresses MAML's limitations regarding insufficient updates and computational overhead, offering a biologically plausible alternative to gradient-based adaptation. Other works, such as those employing Feature-wise Linear Modulation (FiLM) layers, can also be seen as a form of hypernetwork where a small network generates affine transformation parameters for feature maps, allowing for conditional computation and adaptation based on context \cite{perez2018film}. The primary challenge with hypernetworks lies in their own meta-training complexity and ensuring the hypernetwork has sufficient capacity to generate high-quality parameters for a diverse range of tasks without overfitting.

Modular networks offer a distinct approach to explicit architectural adaptation by dynamically reconfiguring their structure or selecting components based on the task at hand. Instead of adapting parameters through gradients or generating them, these approaches adapt the \textit{architecture} itself. A meta-learner can be designed to learn a policy for selecting, composing, or routing a subset of pre-trained modules, or even generating new modules, tailored to a novel task. This allows for efficient knowledge transfer by leveraging a library of specialized functionalities. The core idea is that the meta-learner learns \textit{which} components are relevant for a given task and orchestrates their usage, with the adaptation to a new task primarily involving a forward pass through the selected or configured modules, without gradient updates to the modules themselves during this phase.

For instance, Neural Module Networks (NMNs) \cite{andreas2016neural, andreas2017learning} demonstrate this principle by dynamically assembling differentiable modules (e.g., for "find," "relate," "query") into a task-specific network structure based on a natural language query. While the initial training of individual modules and the meta-learner (which learns to parse queries into network layouts) involves gradients, the \textit{execution} for a new query (task) involves a forward pass through the dynamically constructed network. Similarly, architectures employing Mixture-of-Experts (MoE) can be viewed through a meta-learning lens, where a "gating network" acts as a meta-learner, learning to select or weight the outputs of different "expert" sub-networks based on the input or task context \cite{shazeer2017outrageously}. For new tasks, the gating network routes the input to the most appropriate experts, effectively adapting the computation path without modifying the expert weights. This paradigm offers advantages in interpretability, as the activated modules can reveal the model's reasoning, and efficiency, as only relevant components are engaged. However, challenges include learning effective composition or routing policies, ensuring seamless integration of diverse modules, and managing the combinatorial complexity of module selection for highly flexible systems.

These explicit architectural designs—Transformers, hypernetworks, and modular networks—represent distinct yet complementary strategies within model-based meta-learning to achieve rapid adaptation without explicit gradient updates during the inner loop. Transformers excel at implicit in-context learning by leveraging their attention mechanisms to dynamically interpret task examples within a sequence, demonstrating emergent meta-learning capabilities, particularly in large pre-trained models. Their strength lies in handling complex sequential dependencies and scaling to massive datasets, but their adaptation mechanism is often opaque, and their computational cost for very long contexts can be prohibitive. Hypernetworks, conversely, focus on generating task-specific parameters, offering a direct and potentially more expressive way to modify a base model's weights in a single step, circumventing the limitations of gradient-based adaptation. They provide finer-grained control over adaptation than implicit Transformer mechanisms but introduce their own meta-training complexity and capacity challenges. Modular networks aim for structural adaptation, reconfiguring or selecting functional components to suit new tasks, which can lead to highly efficient and interpretable solutions by activating only relevant parts of the model. However, learning robust composition or routing policies and managing the combinatorial explosion of module interactions remain significant hurdles.

A critical comparison reveals trade-offs across these paradigms. Transformers, especially large pre-trained ones, demand vast computational resources for initial training and rely on the implicit emergent properties of attention, making their adaptation mechanism less interpretable. Their generalization to truly out-of-distribution tasks can be limited by the diversity of their pre-training data. Hypernetworks offer a more explicit and potentially more powerful adaptation than MAML-like gradient steps, but their meta-training can be complex, and the hypernetwork's capacity must be carefully balanced to avoid underfitting or overfitting the parameter generation task. Modular networks offer the highest potential for interpretability and efficiency during adaptation, as only relevant components are activated. However, the meta-learning problem shifts to learning an effective \textit{composition policy}, which can be challenging to generalize across vastly different tasks. The data requirements also vary: large Transformers benefit immensely from massive pre-training, while hypernetworks and modular networks typically require meta-training on a distribution of tasks, which can still be substantial. Future research directions include combining the strengths of these approaches, such as using Transformers as powerful meta-learners to control hypernetworks or orchestrate modular compositions, to achieve more robust, efficient, and interpretable rapid adaptation.


\label{sec:advanced_challenges_and_properties:_robustness,_uncertainty,_and_continual_adaptation}

\section{Advanced Challenges and Properties: Robustness, Uncertainty, and Continual Adaptation}
\label{sec:advanced\_challenges\_\_and\_\_properties:\_robustness,\_uncertainty,\_\_and\_\_continual\_adaptation}

\subsection{Probabilistic Meta-Learning for Task Inference and Uncertainty Quantification}
\label{sec:6\_1\_probabilistic\_meta-learning\_for\_task\_inference\_\_and\_\_uncertainty\_quantification}

Probabilistic meta-learning represents a crucial paradigm for developing robust and trustworthy AI systems, particularly in scenarios demanding not only accurate predictions but also reliable uncertainty estimates. This approach focuses on modeling distributions over functions, enabling models to infer task-specific latent variables and condition predictions on these beliefs, thereby facilitating Bayes-adaptive decision-making in uncertain environments.

The foundational concept of modeling distributions over functions was significantly advanced by \textcite{garnelo2018} with the introduction of Conditional Neural Processes (CNPs). CNPs learn to map a set of context points to a distribution over functions, providing both a prediction and a measure of uncertainty at any target point. Building upon this, \textcite{kim2019} proposed Attentive Neural Processes (ANPs), which integrate attention mechanisms into the CNP framework. ANPs allow the model to selectively weigh the importance of different context points, significantly enhancing performance and flexibility in capturing intricate function structures compared to their non-attentive predecessors.

Extending these probabilistic principles to reinforcement learning (RL), where task uncertainty is paramount for efficient exploration, \textcite{zintgraf2019zat} introduced VariBAD (Variational Bayes-Adaptive Deep RL). VariBAD meta-learns an approximate Bayes-adaptive policy by jointly training a Variational Auto-Encoder (VAE) for posterior inference over latent MDP embeddings and a policy conditioned on this belief. This allows the agent to perform structured online exploration by explicitly reasoning about task uncertainty, a significant step towards tractable Bayes-optimal policies. However, VariBAD's reliance on on-policy experience limited its sample efficiency during meta-training.

To address this, \textcite{rakelly2019m09} developed PEARL (Probabilistic Embeddings for Actor-Critic RL), an off-policy meta-RL algorithm that dramatically improves meta-training sample efficiency. PEARL leverages probabilistic context variables and a permutation-invariant encoder to infer task-specific information, enabling structured exploration through posterior sampling. This decoupling of task inference from policy optimization, coupled with off-policy learning, allowed for substantial gains in data efficiency and asymptotic performance. The utility of probabilistic context variables was further demonstrated in Meta-Inverse Reinforcement Learning (IRL) by \textcite{yu2019o41}. Their PEMIRL method enables few-shot reward inference from unstructured, heterogeneous demonstrations by using probabilistic context variables and mutual information regularization, effectively learning robust reward functions without explicit task labels.

A critical challenge in real-world applications is learning from static, offline datasets. \textcite{dorfman2020mgv} tackled this with BOReL (Bayesian Offline Reinforcement Learning), an off-policy VariBAD variant designed for Offline Meta-RL. BOReL learns exploration strategies from static datasets, formalizing the concept of "MDP ambiguity" where offline data might not contain sufficient information to distinguish between different underlying tasks. This work highlights the inherent limitations of data identifiability in offline settings while providing practical solutions for learning meta-exploration from pre-collected data.

Recent advancements continue to refine probabilistic meta-learning for increasingly complex scenarios. \textcite{zintgraf2021hoc} extended the Bayes-adaptive framework to interactive settings, proposing a meta-learning approach for Deep Interactive Bayesian Reinforcement Learning that models beliefs over other agents' strategies. For non-stationary and dynamic environments, \textcite{bing2022om0} introduced a training strategy applicable to such settings, utilizing Gaussian Mixture Models for clustered task distributions to achieve zero-shot adaptation. Furthermore, \textcite{wang2024d09} proposed a Contrastive Learning-Based Bayes-Adaptive Meta-Reinforcement Learning (CBAMRL) algorithm, which employs contrastive learning for more compact and sufficient task representations, leading to robust zero-shot adaptation in applications like active pantograph control.

In summary, probabilistic meta-learning has evolved from foundational models of function distributions to sophisticated frameworks that enable Bayes-adaptive behavior, efficient exploration, and robust task inference in complex RL and IRL settings. While significant progress has been made in quantifying uncertainty and adapting to new tasks with limited data, challenges remain in scaling these methods to extremely high-dimensional observation spaces, ensuring robust inference in highly ambiguous or severely limited offline data regimes, and developing more theoretically grounded approaches for handling non-stationarity and concept drift in continuous learning.
\subsection{Meta-Learning for Continual and Lifelong Learning}
\label{sec:6\_2\_meta-learning\_for\_continual\_\_and\_\_lifelong\_learning}

Continual and lifelong learning aims to equip AI systems with the ability to sequentially acquire new knowledge from a stream of data without suffering from catastrophic forgetting, a pervasive challenge where models lose proficiency on previously learned tasks when adapting to new ones \cite{son2023lda}. Meta-learning, with its inherent "learning to learn" capability, offers a powerful paradigm for developing algorithmic solutions to this problem, enabling models to adapt efficiently and robustly over time while preserving past knowledge.

While early meta-learning research primarily focused on rapid adaptation to novel tasks or few-shot generalization at a single point in time, exemplified by methods like the Relation Network \cite{sung2017nc5} which learned a non-linear comparison function for few-shot classification, these approaches did not directly address the complexities of sequential knowledge accumulation and catastrophic forgetting in non-stationary environments. The field has since evolved to explicitly tackle the challenge of continual learning, integrating meta-learning principles with various strategies to mitigate forgetting.

A significant line of research adapts optimization-based meta-learning to the continual setting. These methods aim to learn an initialization or an update rule that is robust to sequential task shifts. For instance, \cite{rajasegaran2020llk} introduced iTAML (Incremental Task-Agnostic Meta-learning), a novel approach that seeks to maintain an equilibrium across all encountered tasks by learning a set of generalized parameters. iTAML proposes a new meta-update rule designed to explicitly avoid catastrophic forgetting. Unlike many continual learning methods that assume task boundaries, iTAML is task-agnostic, automatically identifying tasks and adapting rapidly with minimal updates, demonstrating substantial improvements in class-incremental settings on large-scale datasets like CIFAR100 and ImageNet. This highlights how meta-learning can be used to optimize the \textit{process} of incremental learning itself, rather than just task-specific parameters.

Another crucial strategy involves combining meta-learning with memory-based mechanisms, particularly experience replay, to manage task interference and preserve past knowledge. The subsection's focus explicitly mentions methods combining meta-learning with experience replay, which is addressed by works like \cite{holla20202od}. They propose Meta-Learning with Sparse Experience Replay for Lifelong Language Learning, extending online meta-learning (OML) and neuromodulatory meta-learning (ANML) algorithms with an episodic memory module. Crucially, replayed examples from memory are used as the \textit{query set} in the meta-learning outer-loop objective. This directly optimizes the model to prevent forgetting by maximizing the dot product between support and query set gradients, thereby minimizing interference and maximizing knowledge transfer. This approach operates under realistic lifelong learning constraints, such as single passes over data, no task identifiers, and sparse replay rates, demonstrating state-of-the-art performance in lifelong text classification and relation extraction. This integration of meta-learning with memory provides a powerful mechanism for balancing plasticity (learning new tasks) and stability (retaining old knowledge).

More recent advancements have explored principled statistical and Bayesian approaches to achieve robust forgetting immunity. \cite{lee2024snq} introduces Sequential Bayesian Meta-Continual Learning (SB-MCL), a novel framework that fundamentally rethinks forgetting immunity. This approach decouples deep representation learning from sequential knowledge updates. During meta-training, neural networks learn to transform complex raw data into a latent space suitable for robust statistical models. Crucially, during the actual continual learning phase, these deep neural networks remain fixed, preventing catastrophic forgetting. Instead, sequential knowledge updates are offloaded to simple statistical models that leverage \textit{exact Bayesian updates}. This methodology is theoretically grounded in the Fisher-Darmois-Koopman-Pitman theorem, justifying the use of exponential family distributions for lossless and efficient sequential Bayesian updates, thereby providing a principled guarantee against forgetting. SB-MCL meta-learns the \textit{ability to continually learn}, offering a domain-agnostic and model-agnostic solution that supports both supervised and unsupervised continual learning paradigms, and significantly enhances efficiency by eliminating the need for computationally expensive gradient descent during sequential learning.

The progression from optimization-based adaptation \cite{rajasegaran2020llk} and memory-augmented meta-learning \cite{holla20202od} to theoretically grounded Bayesian approaches \cite{lee2024snq} highlights the maturation of meta-learning for continual learning. While optimization- and replay-based methods offer flexibility and plasticity, they often face challenges in providing strong theoretical guarantees against forgetting. In contrast, SB-MCL's strength lies in its principled forgetting immunity, albeit by fixing the deep representation after meta-training, which might limit its adaptability to drastic domain shifts or novel task types that require fundamental changes in representation. The integration of meta-learning with modern architectures, such as transformer-based models for class incremental learning, as explored by \cite{kumar2024he9}, further demonstrates the field's dynamism, leveraging powerful pre-trained models to enhance generalization and mitigate forgetting in real-world scenarios.

Future research directions include scaling these principled Bayesian updates to extremely high-dimensional and complex scenarios, exploring hybrid frameworks that combine the plasticity of dynamic, experience-replay-based meta-learning strategies with the theoretical guarantees of Bayesian methods, and developing meta-learning approaches that can adapt both representations and learning rules continually. The overarching goal remains to create AI systems that can learn continuously throughout their operational lifespan, accumulating knowledge robustly and efficiently in dynamic real-world environments.
\subsection{Addressing Domain Generalization and Out-of-Distribution Robustness}
\label{sec:6\_3\_addressing\_domain\_generalization\_\_and\_\_out-of-distribution\_robustness}

The deployment of artificial intelligence in real-world settings critically hinges on its ability to generalize effectively to unseen target domains and maintain robust performance when faced with out-of-distribution (OOD) data. This challenge, often termed Domain Generalization (DG), requires models to perform reliably even when the test data deviates significantly from the training distribution. Meta-learning provides a powerful framework for enhancing model robustness against such novel environmental conditions or data distributions, ensuring reliable operation by learning to extract domain-invariant features or adapt rapidly to domain shifts. This section explores how meta-learning strategies are designed to achieve this crucial generalization, moving beyond traditional single-domain learning paradigms.

Foundational to understanding meta-learning's role in generalization is the Model-Agnostic Meta-Learning (MAML) framework \cite{finn2017vrt}. MAML learns an initialization that can be quickly adapted to new tasks with a few gradient steps. While initially applied to few-shot learning (generalization to novel \textit{classes}), its core principle of learning an adaptable representation inherently supports generalization to novel \textit{domains}. The idea is that by training on a diverse set of tasks (or domains), the model learns a meta-prior that is robust enough to generalize to an unseen target domain after minimal adaptation. This concept is further explored in theoretical works, such as \cite{finn2017vrt}, which formalizes how gradient-based meta-learning can approximate any learning algorithm, suggesting its broad capacity for generalization. Comprehensive overviews like \cite{huisman2020b7w} survey deep meta-learning techniques, categorizing them into metric-, model-, and optimization-based approaches, all of which contribute to different facets of generalization. Building on this, \cite{khoee2024ksk} provides the first dedicated survey on domain generalization through meta-learning, introducing a novel taxonomy and decision graph to structure the field and highlight how meta-learning specifically addresses OOD challenges.

To explicitly tackle domain generalization, meta-learning approaches often simulate domain shifts during meta-training. A prominent example is Meta-Learning for Domain Generalization (MLDG) \cite{li2018mldg}, which trains a model on a set of source domains such that it can generalize to unseen target domains. MLDG frames the problem as a bi-level optimization, where the inner loop trains on a subset of source domains and the outer loop optimizes for generalization to another held-out source domain. This forces the model to learn features that are robust across different domain shifts. Similarly, MetaReg \cite{balaji2018metareg} learns a regularization function that encourages domain-invariant representations, further enhancing generalization to OOD data. These methods leverage the meta-training process to explicitly learn how to cope with domain shifts, rather than just novel classes. Early efforts to enable models to generalize from few examples, which can be seen as a form of OOD scenario when new classes are encountered, include the Relation Network \cite{sung2017nc5}. While primarily a few-shot classification method, it meta-learned a deep, non-linear metric for comparing embeddings, providing a more flexible and robust comparison mechanism than fixed distance functions. This ability to learn a robust comparison function is beneficial when encountering novel data distributions. Extending this, \cite{sun2018iy7} introduced Meta-Transfer Learning (MTL) to adapt deep neural networks for few-shot tasks by learning scaling and shifting functions for DNN weights, demonstrating that deep architectures could be effectively adapted for generalization, a finding later supported by observations in \cite{huisman2020b7w} regarding larger backbones.

Despite the empirical success, understanding the theoretical underpinnings of meta-learning's generalization capabilities is crucial. \cite{chen2021j5t} provides a novel information-theoretic analysis, deriving data-dependent generalization bounds for MAML. This work offers a generic understanding of both conventional learning-to-learn and MAML, demonstrating tighter bounds compared to previous analyses and providing theoretical guarantees for OOD robustness in deep few-shot learning. However, meta-learning itself can suffer from generalization issues, such as underfitting or overfitting to the meta-training task distribution. \cite{wang2024bhk} addresses this fundamental theoretical challenge by rethinking the "learning to learn" paradigm to mitigate these issues in bi-level optimization. Their proposed TRLearner uses task relation matrices and relation-aware consistency regularization to calibrate meta-learning, offering theoretical guarantees for improved generalization across diverse tasks. This theoretical refinement is crucial for building more inherently robust meta-learners. Further theoretical insights into MAML's optimization process, particularly concerning the inner loop learning rate, are provided by \cite{bernacchia20211r0}. They theoretically derive that an optimal \textit{negative} inner loop learning rate during meta-training can enhance MAML's performance, a counter-intuitive finding that challenges conventional assumptions and offers new avenues for designing more robust meta-optimization strategies. This highlights the complexity of meta-training objectives and their impact on generalization.

Beyond theoretical refinements, practical strategies aim to enhance OOD robustness. One approach is adversarial meta-training. \cite{tian2023iyh} proposes an adversarial meta-training framework specifically for cross-domain few-shot learning. This framework uses max-min episodic iteration to dynamically generate pseudo-tasks that benefit learning cross-domain knowledge, while simultaneously training the meta-learning model to acquire robust meta-knowledge. This model-agnostic approach improves generalization for existing meta-learning methods when tasks are from unseen domains, directly addressing a critical OOD challenge. The rise of powerful pre-trained models has also influenced OOD robustness. \cite{li2023zn0} proposes SEML, which improves few-shot text classification by integrating self-supervised learning from unlabeled data via novel knowledge distillation and graph aggregation methods, thereby enriching representations and making models more robust to unseen text distributions. For adapting large pre-trained models, \cite{wang2024dai} introduces "Learning to Learn" (LoL), a meta-learning-informed episodic training strategy for prompt tuning in Vision-Language Models (VLMs). This method significantly mitigates overfitting to base classes and improves generalization to novel classes, addressing a critical OOD challenge for deploying powerful foundation models. It is also important to note that the complexity of meta-learning strategies is not always paramount. As demonstrated by \cite{baz2022n78} from the NeurIPS 2021 MetaDL challenge, fine-tuning strong pre-trained backbones without complex episodic meta-learning often dominated for few-shot image classification. This underscores the importance of robust base representations for OOD generalization and suggests that sometimes simpler, yet powerful, base models can be more effective than intricate meta-learning schemes, especially when the domain shift is not extreme or the pre-trained model is highly capable. Meta-learning's capacity for OOD robustness extends to complex real-world data challenges. For instance, \cite{nathaniel2023ycu} applies an ensemble of meta-trained deep models (MetaFlux) to estimate global carbon fluxes from sparse spatiotemporal observations. This application demonstrates meta-learning's ability to generalize robustly to data-sparse regions and extreme events, which are inherently OOD conditions in climate science, providing more reliable estimates in critical, under-sampled areas like the tropics.

In conclusion, meta-learning has evolved from establishing foundational principles for few-shot learning to explicitly addressing the complexities of domain generalization and OOD robustness. By learning adaptable initializations, domain-invariant features, or robust adaptation strategies through meta-training, these methods significantly enhance a model's ability to perform reliably in novel environments. While significant progress has been made in enabling rapid adaptation and ensuring robustness in dynamic settings, challenges remain in balancing computational cost with rapid adaptation, ensuring sufficient diversity of meta-training tasks for truly universal generalization, and potentially integrating causal inference to move beyond statistical correlations for more robust OOD performance. Further research is needed to develop meta-learning frameworks that can provably generalize across vastly different domains with minimal adaptation, ensuring reliable AI deployment in increasingly unpredictable real-world scenarios.


\label{sec:deep_meta-learning_in_sequential_decision_making_and_control}

\section{Deep Meta-Learning in Sequential Decision Making and Control}
\label{sec:deep\_meta-learning\_in\_sequential\_decision\_making\_\_and\_\_control}

\subsection{Learning to Reinforcement Learn and One-Shot Imitation}
\label{sec:7\_1\_learning\_to\_reinforcement\_learn\_\_and\_\_one-shot\_imitation}

The quest for intelligent agents capable of rapid adaptation and data-efficient skill acquisition in dynamic environments has led to the emergence of meta-learning as a powerful paradigm for sequential decision-making. This subsection delves into foundational works that pioneered the application of meta-learning to reinforcement learning (RL) and imitation learning, enabling agents to implicitly learn learning algorithms and acquire new skills from minimal demonstrations.

Early explorations into meta-learning demonstrated that recurrent neural networks (RNNs) could implicitly learn simple learning algorithms in supervised settings, effectively functioning as "meta-learners" by processing sequences of inputs and targets \cite{santoro2016323}. Building upon this insight, \textcite{wang20167px} introduced the seminal concept of "learning to reinforcement learn," extending RNN-based meta-learning to the more complex domain of sequential decision-making. Their work demonstrated that an LSTM, trained with a standard deep RL algorithm across a distribution of related tasks, could implicitly learn a distinct RL algorithm within its recurrent dynamics. By receiving previous actions and rewards as auxiliary inputs, the network's internal state effectively encoded a task-specific policy and exploration strategy, allowing it to adapt rapidly to new tasks from the training distribution without explicit weight updates during adaptation. While a proof-of-concept on constrained environments like bandit problems and simple Markov Decision Processes, this work fundamentally shifted the perspective from hand-designing RL algorithms to learning them directly from experience.

Concurrent advancements extended meta-learning to the realm of imitation learning, particularly for robotic skill acquisition. \textcite{finn20174c4} presented a groundbreaking application of Model-Agnostic Meta-Learning (MAML) to one-shot visual imitation learning. This approach enabled robots to acquire new visuomotor skills from a single visual demonstration by adapting a policy through gradient updates. The core innovation lay in training a policy initialization that, when fine-tuned with a single demonstration, could quickly converge to a proficient task-specific policy. To facilitate this, the authors introduced architectural modifications such as a two-head network and bias transformation, enhancing the stability and representational power of gradient-based meta-learning for end-to-end learning from raw pixel inputs. This work significantly addressed the data inefficiency of traditional imitation learning, paving the way for more generalist robots capable of rapid skill transfer. However, it did not explicitly address challenges like compounding errors or domain shift between demonstration and execution.

These foundational works laid critical groundwork for subsequent advancements in meta-RL and meta-imitation learning. The implicit RL algorithm learned by RNNs in \textcite{wang20167px} showcased the potential for flexible adaptation, but its black-box nature made analysis and generalization beyond the training distribution challenging. Similarly, \textcite{finn20174c4}'s MAML-based approach, while powerful for one-shot imitation, could still be sample-inefficient during the meta-training phase and faced limitations in complex, high-dimensional RL settings. Addressing these limitations, later works like \textcite{rakelly2019m09} introduced more sophisticated off-policy meta-RL algorithms, such as PEARL, which leveraged probabilistic context variables to explicitly model task uncertainty and enable structured exploration, significantly improving meta-training sample efficiency. This marked a progression from implicitly learned algorithms to more principled, probabilistic approaches for task inference and policy adaptation.

In conclusion, these early applications of meta-learning to sequential decision-making established a powerful paradigm for data-efficient policy acquisition and rapid skill transfer. By demonstrating that agents could either implicitly learn an RL algorithm through recurrent dynamics \cite{wang20167px} or quickly adapt a policy through gradient-based meta-learning from a single demonstration \cite{finn20174c4}, these works fundamentally challenged the traditional approach of learning each task from scratch. They highlighted the potential for agents to "learn to learn" in dynamic, interactive environments, setting the stage for future research into more robust, sample-efficient, and generalizable meta-learning algorithms for complex real-world challenges. Despite their success, challenges such as scalability to highly complex environments, theoretical guarantees for learned algorithms, and robust generalization to entirely novel tasks remain active areas of research.
\subsection{Efficient and Safe Meta-Reinforcement Learning}
\label{sec:7\_2\_efficient\_\_and\_\_safe\_meta-reinforcement\_learning}

Deploying adaptive reinforcement learning (RL) agents in real-world, safety-critical scenarios necessitates advanced meta-learning techniques that not only achieve rapid adaptation and sample efficiency but also provide robust guarantees for safe operation. This subsection reviews the progression of meta-RL towards off-policy learning, efficient adaptation from static datasets, and, crucially, the integration of provable safety mechanisms.

Early meta-reinforcement learning (meta-RL) approaches, while demonstrating the ability to learn adaptation mechanisms \cite{wang20167px}, often suffered from severe sample inefficiency due to their reliance on on-policy data and lacked robust mechanisms for reasoning about task uncertainty. A significant step towards addressing these limitations was the introduction of \textit{Probabilistic Embeddings for Actor-Critic RL} (PEARL) \cite{rakelly2019m09}. PEARL pioneered an off-policy meta-RL algorithm that leverages probabilistic context variables to encode task-specific information, enabling explicit reasoning about uncertainty. Its key innovations include a permutation-invariant encoder for online probabilistic filtering and a decoupled off-policy training strategy, which together yield substantial (20-100x) improvements in meta-training sample efficiency and facilitate structured exploration through posterior sampling. Building on the idea of learning task uncertainty, \textit{VariBAD} \cite{zintgraf2019zat} further advanced Bayes-adaptive deep RL by meta-learning an approximate Bayes-optimal policy. VariBAD jointly trains a variational auto-encoder for posterior inference over latent MDP embeddings and a policy conditioned on this belief, allowing for efficient, structured online exploration without requiring privileged task information. The versatility of probabilistic context variables was further demonstrated by \textit{PEMIRL} \cite{yu2019o41}, which extended this paradigm to Meta-Inverse Reinforcement Learning, enabling few-shot reward inference from unstructured, heterogeneous demonstrations by regularizing mutual information between context and trajectories.

While PEARL and VariBAD enhanced online sample efficiency, real-world applications often demand learning from static, pre-collected datasets. This led to the critical development of offline meta-reinforcement learning. \textit{BOReL} \cite{dorfman2020mgv} tackled this challenge by proposing an off-policy VariBAD variant specifically designed for offline learning. BOReL formalized the concept of "MDP ambiguity," highlighting the inherent limitations of data identifiability when learning exploration strategies from static, exploitative datasets. It demonstrated how to learn effective exploration from such data, marking a crucial step towards meta-RL deployment in data-scarce environments. This offline meta-learning capability has recently been leveraged in complex adaptive control systems. For instance, \textit{MAGICVFM} \cite{lupu20249p4} integrates Visual Foundation Models with offline meta-learning and composite adaptive control for real-time, stable ground vehicle control in challenging terrains. This approach showcases meta-learning's role in robust physical system autonomy, providing mathematical guarantees of exponential stability and robustness, which are vital for reliable operation. Other works have also explored efficient adaptation in dynamic settings, such as \textit{MetaABR} \cite{li20246fg} for adaptive bitrate selection in video streaming and methods for fast adaptive task offloading in edge computing \cite{wang2020tae}.

Beyond efficiency and offline capabilities, the paramount concern of safety in real-world deployment has driven the emergence of meta-safe RL frameworks. A significant advancement in this direction is the \textit{CMDP-within-online framework for Meta-Safe Reinforcement Learning} \cite{khattar2024sr6}. This groundbreaking work introduces the first provable guarantees for task-averaged regret and constraint violations in meta-RL. By encapsulating each task as a Constrained Markov Decision Process (CMDP) and employing an online meta-learner that updates policies and learning rates based on inexact upper bounds on optimality gaps and constraint violations, it provides a critical step towards deploying adaptive RL agents in safety-critical scenarios. This framework is designed to ensure adherence to constraints, which is paramount for reliable operation. Further illustrating the need for robust and safe adaptation in dynamic environments, \textit{CBAMRL} \cite{wang2024d09} proposes a contrastive learning-based Bayes-adaptive meta-RL algorithm for active pantograph control in high-speed railways. This method achieves zero-shot adaptation in non-stationary environments and provides well-structured task representations, implicitly contributing to operational safety by maintaining stable contact force. Similarly, \textit{GM2DQL} \cite{ma20243e9} addresses eco-routing with multi-objective meta-deep Q-learning, balancing fuel efficiency and travel time, where safety is an implicit concern through optimal route selection.

The trajectory of meta-reinforcement learning has thus evolved from foundational efforts in learning adaptive algorithms to sophisticated frameworks that prioritize not only sample efficiency and off-policy learning but also, critically, safety. The progression from probabilistic context variables in PEARL and VariBAD to offline learning with BOReL, and finally to provably safe meta-RL with frameworks like \cite{khattar2024sr6}, underscores a concerted effort to bridge the gap between theoretical adaptability and practical, reliable deployment. Future research will likely focus on scaling these provable safety guarantees to more complex, high-dimensional systems, integrating them with the adaptive capabilities of large foundation models, and ensuring robustness against unforeseen domain shifts, thereby paving the way for truly autonomous and trustworthy AI agents in safety-critical applications.
\subsection{Meta-Learning for Adaptive Control in Embodied AI}
\label{sec:7\_3\_meta-learning\_for\_adaptive\_control\_in\_embodied\_ai}

Achieving robust and stable control for embodied AI systems, particularly in robotics, presents significant challenges due to complex, often unmodeled dynamics, external disturbances, and varying environmental conditions. Traditional control methods often struggle with these uncertainties, requiring precise system models or extensive manual tuning. Meta-learning offers a powerful paradigm to enable rapid adaptation and enhance the reliability of autonomous systems in such intricate physical environments by learning to quickly adjust control policies or system models. The critical distinction in this domain lies in moving beyond mere performance optimization to providing mathematical stability guarantees essential for safe and predictable real-world deployment.

Early explorations into meta-learning for robotics primarily focused on improving data efficiency and rapid skill acquisition within the reinforcement learning (RL) paradigm. \cite{finn20174c4} pioneered the application of Model-Agnostic Meta-Learning (MAML) to visual imitation learning, demonstrating how a robot could adapt a policy from raw pixel inputs using a single visual demonstration. This work highlighted meta-learning's capacity for few-shot skill acquisition and policy adaptation. Similarly, \cite{yu2018nm7} extended this by proposing domain-adaptive meta-learning for one-shot imitation from human demonstrations, enabling robots to learn new behaviors despite significant domain shifts in perspective, environment, and embodiment. While these foundational efforts established meta-learning's utility for rapid task adaptation and data-efficient policy learning, they primarily operated within the RL framework, focusing on optimizing task performance rather than explicitly addressing real-time control stability or providing formal guarantees against unmodeled physical dynamics. The critical aspects of robust control, such as Lyapunov stability or disturbance rejection, remained largely outside their scope.

As the field matured, the focus expanded towards enabling robust adaptation in dynamic and non-stationary environments, bridging the gap towards adaptive control. \cite{bing2022om0} introduced a meta-reinforcement learning strategy applicable to non-stationary environments, demonstrating competitive asymptotic performance and superior sample efficiency in continuous robotic control benchmarks. Their approach, leveraging Gaussian mixture models for task representation, allows agents to learn basic distinct behaviors and adapt to changing conditions, crucial for operating in unpredictable physical settings. Complementing this, \cite{meng2024nqq} addressed the sim-to-real transfer problem for meta-policies in robotics using Progressive Neural Networks (MetaPNN). By meta-training a policy for multiple source tasks and transferring it via PNN, their method effectively bridges the reality gap with mismatched dynamics, allowing robots to adapt to new situations in the real world with improved learning efficiency and performance. These works underscore meta-learning's ability to handle environmental variability and dynamics discrepancies, laying important groundwork for more explicit adaptive control.

A significant advancement in directly integrating meta-learning with adaptive control theory for stability and robustness guarantees has emerged. \cite{oconnell2022twd} introduced Neural-Fly, a learning-based approach for agile flight in strong, dynamic winds, which is a prime example of adaptive control for embodied AI. Neural-Fly incorporates pretrained representations through deep learning, leveraging the observation that aerodynamics in different wind conditions share a common, low-dimensional representation. It employs a learning algorithm called Domain Adversarially Invariant Meta-Learning (DAIML) to learn this shared representation from minimal flight data (e.g., 12 minutes). Crucially, with this learned representation as a basis, Neural-Fly uses a \textit{composite adaptation law} to update a set of linear coefficients online. This design not only achieves precise flight control with substantially smaller tracking error than state-of-the-art nonlinear and adaptive controllers but also provides \textit{exponential stability} and \textit{robustness guarantees}. This methodology demonstrates how meta-learning can facilitate rapid online adaptation while adhering to rigorous control-theoretic principles, even extrapolating to unseen wind conditions and transferring across different drones.

Building on this trajectory, \cite{lupu20249p4} presented MAGICVFM, a framework that integrates meta-learning with visual foundation models (VFMs) and composite adaptive control for real-time terrain adaptation in ground vehicles. This addresses the complex challenge of controlling off-road vehicles amidst dynamic, unmodeled ground interactions like slippage. MAGICVFM leverages VFMs to extract rich, generalizable visual features from terrain, which are then fed into a Deep Neural Network (DNN) trained offline using meta-learning to model residual dynamics. A key innovation is that only the final layer of this DNN is adapted online by a composite adaptive controller, which provides \textit{mathematical guarantees of exponential stability and robustness} for real-time control. This approach exemplifies how meta-learning facilitates rapid adaptation to unmodeled dynamics by learning an adaptable model of the environment, while simultaneously offering critical stability assurances through its integration with established control theory. The synergy between advanced perception (VFMs), meta-learning for adaptable modeling, and robust adaptive control is crucial for pushing the boundaries of autonomous systems in complex physical environments.

In conclusion, meta-learning has evolved from foundational policy and reward adaptation techniques to sophisticated frameworks that enable robust and stable control for embodied AI. The integration of meta-learning with composite adaptive control, as demonstrated by Neural-Fly \cite{oconnell2022twd} for UAVs and MAGICVFM \cite{lupu20249p4} for ground vehicles, represents a significant advancement. These approaches provide both rapid adaptation to unmodeled dynamics and mathematical stability guarantees, bridging advanced perception, learning, and control. Future directions will likely focus on extending these guarantees to a wider range of complex physical interactions, exploring the integration of even more diverse multi-modal foundation models for enhanced environmental understanding, and developing tighter theoretical links between meta-learning's generalization capabilities and the formal stability properties of adaptive control systems. This continuous advancement is crucial for pushing the boundaries of autonomous systems in complex, real-world physical environments.


\label{sec:modern_frontiers:_foundation_models,_safety,_and_theoretical_refinements}

\section{Modern Frontiers: Foundation Models, Safety, and Theoretical Refinements}
\label{sec:modern\_frontiers:\_foundation\_models,\_safety,\_\_and\_\_theoretical\_refinements}

\subsection{Meta-Learning for Adapting Large Foundation Models}
\label{sec:8\_1\_meta-learning\_for\_adapting\_large\_foundation\_models}

The emergence of large pre-trained models, including Vision-Language Models (VLMs) and Large Language Models (LLMs), has fundamentally reshaped artificial intelligence. These models, often termed "foundation models," possess vast knowledge acquired from extensive pre-training. However, efficiently adapting these massive models to new, often data-scarce, downstream tasks remains a significant challenge. Meta-learning provides a powerful paradigm to address this by enabling models to "learn to learn," facilitating rapid and generalizable adaptation with minimal data and computational overhead.

Historically, meta-learning focused on learning optimal initial model parameters for faster adaptation. Model-Agnostic Meta-Learning (MAML) \cite{Finn\_MAML\_2017} and its first-order approximation, Reptile \cite{Nichol\_Reptile\_2018}, exemplify this by training a model's initialization such that a few gradient steps on a new task yield strong performance. While foundational, these gradient-based methods often incur high computational costs due to bi-level optimization and second-order derivatives, limiting their direct scalability to the immense parameter counts of modern foundation models. This computational bottleneck necessitated a shift in the meta-learning paradigm: instead of learning initial weights for an entire model, the focus moved to learning \textit{how to efficiently interact with} or \textit{tune} these largely frozen, pre-trained giants.

This shift has prominently manifested in parameter-efficient fine-tuning (PEFT) methods, where meta-learning guides the adaptation of a small subset of parameters or external modules. A leading approach is "Learning to Prompt" (L2P) \cite{Chen\_L2P\_2021}, where a meta-learner is trained to generate task-specific, learnable prompts that guide a frozen VLM for few-shot adaptation. This significantly reduces the number of parameters requiring fine-tuning, making adaptation highly efficient. For LLMs, meta-learning has been extensively surveyed as a key strategy for prompt-based learning, PEFT, and in-context learning, enabling adaptation with minimal data and computational cost \cite{Wang\_Meta-Learning\_2022, lee2021jou}. Recent advancements, such as "Learning to Learn Better Visual Prompts" (LoL) \cite{wang2024dai}, integrate meta-learning's N-way K-shot episodic training into prompt tuning for VLMs. This meta-training strategy explicitly addresses overfitting issues common in prompt tuning, significantly enhancing generalization from base classes to novel, unseen classes by optimizing the prompt learning process itself.

Beyond prompt-based methods, meta-learning is also being applied to other critical PEFT techniques. Low-Rank Adaptation (LoRA), which fine-tunes low-rank incremental update matrices on top of frozen pre-trained weights, has proven highly effective. However, LoRA's performance can depend heavily on the choice of matrix rank. To address this, AutoLoRA \cite{zhang2024a5a} introduces a meta-learning framework that automatically identifies the optimal rank for each LoRA layer. By learning selection variables associated with rank-1 matrices, AutoLoRA dynamically determines which components to retain, thereby improving both efficiency and performance compared to uniform or exhaustively searched rank assignments. Similarly, meta-learning can be applied to other PEFT modules like adapters, where the meta-learner could optimize adapter initializations or architectures for faster downstream convergence. This broader application of meta-learning to various PEFT strategies highlights its versatility in making foundation models adaptable. For instance, in transformer-based NLP models, meta-learning can facilitate class incremental learning by enabling models to generalize to newly introduced classes without full retraining, emphasizing its role in continuous adaptation for large models \cite{kumar2024he9}.

The choice between meta-learning prompts and meta-learning lightweight modules (like LoRA or adapters) involves trade-offs. Prompt-based methods primarily modify the input context, offering simplicity and often requiring minimal changes to the foundation model architecture. However, their capacity to encode complex task-specific information might be limited. In contrast, meta-learning PEFT modules directly modify the model's internal representations or computation, potentially offering greater flexibility and expressiveness for adaptation, albeit with slightly more architectural complexity. Both approaches leverage meta-learning to optimize \textit{how} the foundation model learns, rather than \textit{what} it learns, making them crucial for efficient deployment.

Furthermore, meta-learning plays a role in enabling foundation models to adapt to complex, real-time scenarios. For instance, the MAGICVFM framework \cite{lupu20249p4} demonstrates meta-learning's utility in adapting Visual Foundation Models (VFMs) for real-time ground interaction control in autonomous vehicles. By combining offline meta-learning with online composite adaptive control, it adaptively modifies only the last layer of a deep neural network, providing mathematically guaranteed stability and efficient adaptation to dynamic, unmodeled terrains. While robust generalization to out-of-distribution (OOD) data and domain shifts is a critical challenge for foundation models, meta-learning contributes to this by designing training strategies that promote this capability. For a detailed discussion on meta-learning strategies for domain generalization and OOD robustness, including methods like adversarial meta-training frameworks for cross-domain few-shot learning \cite{tian2023iyh}, refer to Subsection 6.3.

In conclusion, meta-learning is indispensable for unlocking the full potential of large foundation models. It has evolved from learning initial model weights to developing sophisticated strategies for parameter-efficient adaptation, such as prompt learning and selective module tuning. While significant progress has been made in improving generalization and efficiency, challenges remain in developing universally robust and theoretically grounded meta-adaptation strategies that can consistently perform across extreme domain shifts and ensure safety in critical applications. Future research will likely focus on deeper theoretical understandings of meta-learning's interaction with foundation models, exploring more adaptive and context-aware prompt generation, and developing methods that offer stronger guarantees for out-of-distribution generalization and safety in real-world deployment.
\subsection{Meta-Learning for Robustness, Safety, and Trustworthy AI Systems}
\label{sec:8\_2\_meta-learning\_for\_robustness,\_safety,\_\_and\_\_trustworthy\_ai\_systems}

Building truly trustworthy AI systems extends beyond mere performance, demanding guarantees of robustness, safety, transparency, fairness, and long-term reliability, particularly in safety-critical applications and dynamic real-world environments. Meta-learning, by enabling AI systems to "learn to learn" and rapidly adapt, offers a powerful paradigm to imbue models with these crucial attributes, fostering greater public trust and responsible AI development. This section explores how meta-learning contributes to holistically trustworthy AI by addressing these multifaceted challenges.

A foundational aspect of trustworthiness is robust generalization. An AI system cannot be trusted if its performance degrades unpredictably on novel tasks or out-of-distribution data. Meta-learning inherently aims to generalize across a distribution of tasks, but ensuring this generalization is theoretically sound is critical. \cite{chen2021j5t} provided a significant step by deriving novel information-theoretic generalization bounds for meta-learning algorithms, including MAML. Their data-dependent bounds offer a more rigorous understanding of meta-learning's reliability, moving beyond empirical observations to establish a scientific basis for trustworthy performance. This theoretical grounding is crucial for understanding \textit{why} meta-learned models can be expected to generalize. Furthermore, practical robustness against out-of-distribution data is essential; meta-learning techniques for domain generalization, as comprehensively surveyed by \cite{khoee2024ksk}, are vital for enabling models to perform reliably on unseen target domains (as discussed in Section 6.3). This ensures that adaptive AI systems maintain their integrity even when faced with novel environmental conditions or data shifts. Complementing this, \cite{wang2024bhk} addressed the challenge of underfitting and overfitting in meta-learning by introducing task relation matrices and consistency regularization, providing theoretical guarantees for improved generalization, thereby enhancing the inherent reliability of meta-learned models.

Beyond foundational robustness, explicit safety guarantees are paramount for deploying AI in sensitive domains. Meta-learning's adaptive nature allows for the integration of safety constraints that can generalize across tasks. In meta-reinforcement learning (meta-RL), where agents learn to adapt policies rapidly, ensuring safety during adaptation has been a significant challenge. \cite{khattar2024sr6} made a landmark contribution by introducing a "CMDP-within-online" framework for Meta-Safe Reinforcement Learning. This pioneering work provides the first provable guarantees for task-averaged regret and constraint violations, even when dealing with inexact estimations and non-convex Constrained Markov Decision Processes (CMDPs). This framework is critical for deploying meta-RL agents in safety-critical contexts, such as autonomous systems, where adherence to safety protocols is non-negotiable (further elaborated in Section 7.2). Similarly, for embodied AI, \cite{lupu20249p4} presented MAGICVFM, a meta-learning approach for ground interaction control with visual foundation models, offering mathematical guarantees of exponential stability and robustness for real-time terrain adaptation in off-road vehicles (discussed in Section 7.3). Such guarantees are essential for reliable and safe operation in complex physical environments.

Trustworthy AI also demands transparency, interpretability, and reliable confidence estimation, especially in human-machine interfaces where decisions can have high stakes. Traditional deep learning models often operate as "black boxes" and can exhibit overconfidence, which is detrimental to trust. \cite{tam2024a1h} addressed this by developing a deep metric meta-learning framework for EMG-based hand gesture recognition. By re-framing the problem as representation learning, they created a robust class proximity-based confidence estimator that provides interpretable decision-making and informs decision rejection. This is crucial for safety-critical applications like prosthetic control, where erroneous predictions are unacceptable and human operators need to understand the system's certainty. Expanding on this, \cite{aqeel2025zql} proposed Confident Meta-learning (CoMet) for unsupervised anomaly detection, enabling models to learn from uncurated datasets while remaining robust to noise and providing reliable confidence scores. The broader integration of Explainable AI (XAI) with meta-learning, as highlighted by \cite{daglarli20216fl} in the context of Cyber-Physical Systems (CPS), underscores the growing need for adaptive systems that can not only learn rapidly but also explain their reasoning, thereby enhancing transparency and user trust.

A truly holistic approach to trustworthy AI must also address fairness and bias mitigation. Meta-learning offers unique capabilities to learn fair decision-making processes that generalize across diverse demographic groups or contexts. For instance, in information retrieval, datasets can be biased, leading to systematic discrimination. To counter this, \cite{wang2024so2} proposed a Meta Curriculum-based Fair Ranking (MCFR) framework. This framework uses a meta-learner to optimize a weighting function that makes the ranking loss focus more on minority groups, thereby alleviating data bias through gradient-based "learning to learn." This demonstrates meta-learning's capacity to adapt the learning process itself to achieve fairer outcomes. Similarly, \cite{chen2024b4d} introduced a Locational Meta-Referee (Meta-Ref) for fast adaptation of locational fairness. This Meta-Ref dynamically adjusts learning rates for training samples from different locations to advocate for fair performance across spatial regions, mitigating biases that might arise from distinct geographical data distributions. These works illustrate how meta-learning can proactively learn to identify and correct biases, leading to more equitable and trustworthy AI systems.

Finally, long-term reliability is crucial for maintaining trustworthiness in dynamic operational environments. AI systems must be able to continually adapt without forgetting previously acquired knowledge. Meta-learning plays a pivotal role in developing algorithmic solutions to catastrophic forgetting, a major challenge in continual learning (as detailed in Section 6.2). \cite{holla20202od} demonstrated this by combining meta-learning with sparse experience replay for lifelong language learning. Their OML-ER and ANML-ER algorithms achieved state-of-the-art performance under realistic constraints (single pass, no task IDs, limited memory), effectively mitigating forgetting and ensuring the continuous reliability of AI systems as they adapt to new information over time.

In conclusion, meta-learning is evolving into a cornerstone for building holistically trustworthy AI systems. Its adaptive nature enables solutions that address foundational robustness through theoretical guarantees \cite{chen2021j5t} and domain generalization \cite{khoee2024ksk}. It provides critical safety guarantees in meta-RL \cite{khattar2024sr6} and stable adaptive control \cite{lupu20249p4} for safety-critical applications. Furthermore, meta-learning fosters transparency and reliable confidence estimation in human-machine interfaces \cite{tam2024a1h} and offers pathways for interpretability \cite{daglarli20216fl}. Crucially, it contributes to fairness by learning to mitigate biases across tasks and demographic groups \cite{wang2024so2, chen2024b4d}, and ensures long-term reliability through continual learning \cite{holla20202od}. While significant progress has been made, future work must continue to focus on scaling these guarantees to even more complex real-world scenarios, developing standardized metrics for trustworthiness, and further integrating interpretability and accountability into the meta-learning process to solidify public confidence in responsible AI development.
\subsection{Rethinking Meta-Learning: Theoretical Calibration and Generalization}
\label{sec:8\_3\_rethinking\_meta-learning:\_theoretical\_calibration\_\_and\_\_generalization}

The advancement of meta-learning from empirical successes to a robust scientific discipline necessitates a profound theoretical calibration and algorithmic refinement, particularly concerning its core generalization capabilities. This subsection delves into cutting-edge research that fundamentally rethinks the meta-learning model itself, aiming to provide theoretical guarantees for improved generalization across diverse and often unseen tasks, thereby strengthening the field's scientific foundation. The focus is on understanding \textit{why} meta-learners generalize, how to optimize their processes more effectively, and how to mitigate issues like underfitting and overfitting through principled approaches.

A critical challenge in meta-learning has been the lack of rigorous theoretical guarantees for its generalization performance, especially for deep neural networks. Addressing this, \cite{chen2021j5t} offers a significant contribution by presenting a novel information-theoretic analysis of meta-learning algorithms. Their work provides a generic understanding of both conventional "learning-to-learn" frameworks and modern Model-Agnostic Meta-Learning (MAML) algorithms. Crucially, they derive a data-dependent generalization bound for a stochastic variant of MAML, demonstrating it to be non-vacuous for deep few-shot learning. This is a substantial improvement over previous bounds, which often relied on less tight metrics like the square norm of gradients, providing a deeper, more actionable insight into the factors governing meta-learner generalization. This information-theoretic perspective moves beyond simply observing performance to explaining the underlying mechanisms of knowledge transfer and adaptation.

Beyond understanding generalization, there is a growing need for more principled and unified frameworks for meta-learning optimization. \cite{franceschi2018u1q} introduces such a framework based on bilevel programming, which elegantly unifies gradient-based hyperparameter optimization and meta-learning. This approach conceptualizes the meta-learning problem as a bi-level optimization where the outer variables represent hyperparameters or meta-learner parameters, and the inner problem describes the task-specific adaptation. By explicitly accounting for the optimization dynamics of the inner objective, this framework provides sufficient conditions under which approximate solutions converge to exact ones. This theoretical lens offers a powerful way to "rethink" the meta-learning process by providing a clear mathematical foundation for optimizing the learning-to-learn objective, moving beyond heuristic choices to a more calibrated and theoretically grounded approach.

Further theoretical calibration has been applied to specific, widely adopted meta-learning paradigms. For instance, \cite{bernacchia20211r0} provides a rigorous mathematical analysis of MAML, a cornerstone of optimization-based meta-learning. Their work uncovers the critical role of negative learning rates in MAML's inner loop for achieving optimal adaptation. This finding moves beyond empirical observations and heuristic choices of learning rates, offering a deeper, theoretically informed understanding of how MAML's bi-level optimization truly functions and how to calibrate it for superior performance. Building on the practical limitations of MAML's gradient-based inner loop, such as insufficient weight modification and computational overhead, \cite{przewiezlikowski2022d4y} proposed HyperMAML. This method \textit{reimagines} the adaptation mechanism by replacing the traditional gradient-based inner loop with a trainable Hypernetwork that directly generates the adapted weights. This offers a learned, non-gradient-based update rule, designed for more effective few-shot adaptation and improved generalization by explicitly addressing the shortcomings of standard gradient-based adaptation.

A significant stride in rethinking the meta-learning model and directly addressing core generalization issues, including underfitting and overfitting, is presented by \cite{wang2024bhk}. This work introduces a novel "Learning lens" perspective, explicitly conceptualizing the meta-learning function $F\_\theta$ as comprising initialization layers and a "meta-layer" implemented via gradient optimization. Crucially, it proposes TRLearner, a plug-and-play method that extracts task relation matrices and applies relation-aware consistency regularization to calibrate the meta-learning process. This innovative approach directly tackles the pervasive problems of underfitting and overfitting in existing meta-learning methods, providing theoretical guarantees for improved generalization by ensuring that models can mutually reinforce each other through task similarity. This work represents a comprehensive effort to inject theoretical rigor into the design of meta-learning architectures and objectives, thereby strengthening the field's scientific foundation.

In conclusion, the modern frontier of meta-learning is characterized by a concerted drive to refine its theoretical underpinnings and algorithmic mechanisms for superior generalization. From developing information-theoretic bounds to establishing principled bilevel optimization frameworks, and from calibrating existing algorithms like MAML to fundamentally rethinking adaptation mechanisms and incorporating task-relation awareness, the field is moving towards a more robust scientific foundation. Future directions will undoubtedly involve further integration of these theoretical guarantees with practical algorithms, exploring how these calibrated meta-learning models can seamlessly adapt to increasingly diverse and complex task distributions while mitigating fundamental issues like underfitting and overfitting with greater theoretical certainty.


\label{sec:conclusion_and_future_outlook}

\section{Conclusion and Future Outlook}
\label{sec:conclusion\_\_and\_\_future\_outlook}

\subsection{Summary of Key Advancements and Impact}
\label{sec:9\_1\_summary\_of\_key\_advancements\_\_and\_\_impact}

Deep Meta-Learning has fundamentally reshaped the landscape of artificial intelligence, moving beyond static, task-specific learning to empower models with the ability to "learn to learn" \cite{peng20209of}. This paradigm shift has been crucial in addressing the inherent limitations of traditional deep learning, particularly its reliance on extensive labeled data and its struggle with rapid adaptation and robust generalization to novel tasks and out-of-distribution scenarios \cite{peng20209of}. The field's evolution has been marked by significant breakthroughs across several methodological families, each contributing distinct mechanisms for achieving rapid adaptation and efficient knowledge transfer.

Early foundational work laid the intellectual groundwork for agents capable of implicitly discovering and refining their own learning rules. This was exemplified by recurrent neural networks (RNNs) learning internal reinforcement learning algorithms within their recurrent dynamics, effectively "learning to reinforcement learn" \cite{wang20167px}. Further biologically inspired advancements, such as Neuro-Modulated Networks (NMNs) \cite{vecoven2018hc1}, demonstrated how dynamic tuning of activation functions could yield faster and more stable adaptive behaviors. The ambition to enable online, single-lifetime meta-learning of objective functions was realized with algorithms like FRODO, which uses meta-gradient descent to discover its own RL objective \cite{xu2020txy, sutton2022jss}.

A cornerstone of rapid adaptation emerged with \textbf{optimization-based meta-learning}, most notably Model-Agnostic Meta-Learning (MAML) \cite{finn20174c4}. MAML introduced the powerful concept of learning an adaptable initialization that can be quickly fine-tuned to new tasks with just a few gradient steps, demonstrating its universality in approximating a wide range of learning algorithms \cite{finn2017vrt}. While initially computationally intensive due to second-order derivatives, subsequent advancements like first-order approximations \cite{nagabandi2018esl} significantly improved its scalability. Theoretical insights have further refined our understanding, revealing counter-intuitive findings such as the optimal inner-loop learning rate for meta-training potentially being negative \cite{bernacchia20211r0}. Beyond learning initializations, this paradigm has evolved to learn explicit optimizers (e.g., Meta-SGD) or leverage differentiable solvers, offering greater flexibility in adapting the learning process itself \cite{sutton2022jss}. The impact of these methods is evident in diverse applications, from eco-routing \cite{ma20243e9} to adaptive task offloading in edge computing \cite{wang2020tae}.

Concurrently, \textbf{metric-based meta-learning} revolutionized few-shot learning by focusing on learning effective embedding spaces and comparison mechanisms \cite{peng20209of}. The seminal Relation Network \cite{sung2017nc5} marked a significant advancement by meta-learning a deep, non-linear metric function to compute similarity scores, moving beyond fixed distance functions. This approach emphasizes robust representation learning, where examples from novel classes can be classified by comparing them to a small support set, often via learned prototypes. Modern enhancements integrate self-supervised learning from unlabeled data to enrich feature representations, as seen in SEML for few-shot text classification \cite{li2023zn0}. Furthermore, meta-learning has proven instrumental in adapting large foundation models, with methods like "Learning to Learn" (LoL) employing episodic training to mitigate overfitting in visual prompt tuning for Vision-Language Models, thereby improving generalization to novel classes \cite{wang2024dai}.

\textbf{Model-based meta-learning} provides an alternative pathway to rapid adaptation through architectural innovations that intrinsically process and adapt to new task information. Early efforts utilized recurrent neural networks (RNNs) as meta-optimizers, implicitly learning update rules by processing task data sequentially \cite{peng20209of}. A significant leap came with Memory-Augmented Neural Networks (MANNs), such as Neural Turing Machines (NTMs) and Differentiable Neural Computers (DNCs), which integrate external memory modules to store and retrieve task-specific knowledge, enabling more sophisticated forms of in-context learning \cite{peng20209of}. More recently, Transformer-based architectures have emerged as powerful model-based meta-learners, leveraging attention mechanisms to achieve rapid adaptation and in-context learning without explicit gradient updates, by processing task descriptions and examples directly within their forward pass.

A critical advancement has been the development of \textbf{probabilistic meta-learning}, which explicitly models distributions over functions and provides crucial uncertainty estimates, essential for robust decision-making in data-scarce or unreliable environments \cite{peng20209of}. VariBAD \cite{zintgraf2019zat} pioneered meta-learning approximate Bayes-adaptive policies by inferring latent MDP embeddings. This was further refined by PEARL \cite{rakelly2019m09}, an off-policy meta-RL algorithm that significantly improved meta-training sample efficiency through probabilistic context variables and permutation-invariant encoders. These probabilistic frameworks have extended to few-shot reward inference in Inverse Reinforcement Learning \cite{yu2019o41} and enabled robust learning from static datasets in Offline Meta-RL with methods like BOReL \cite{dorfman2020mgv}. Their utility is demonstrated in complex scenarios like interactive Bayesian RL in multi-agent settings \cite{zintgraf2021hoc} and active control systems \cite{wang2024d09}.

The profound influence of Deep Meta-Learning is evident in its transformative impact across diverse real-world applications and its contribution to building more robust, safe, and interpretable intelligent systems. In sequential decision-making, meta-learning enables rapid policy acquisition and one-shot imitation learning for robotics \cite{finn20174c4}, leading to advanced adaptive control for embodied AI, such as stable terrain adaptation in ground vehicles \cite{lupu20249p4}. The field has made significant strides in \textbf{efficient and safe meta-reinforcement learning}, with frameworks like Meta-Safe RL providing the first provable guarantees for task-averaged regret and constraint violations in safety-critical domains \cite{khattar2024sr6}. Beyond safety, meta-learning offers algorithmic solutions to \textbf{continual and lifelong learning}, mitigating catastrophic forgetting in models by combining meta-learning with sparse experience replay \cite{holla20202od}. It also enhances \textbf{domain generalization and out-of-distribution robustness}, crucial for reliable AI deployment in dynamic environments. Furthermore, meta-learning plays a pivotal role in efficiently adapting large \textbf{foundation models} to new tasks with minimal data, unlocking their vast knowledge for diverse downstream applications \cite{wang2024dai}. Theoretical advancements, such as deriving tighter generalization bounds for MAML \cite{chen2021j5t} and rethinking the meta-learning model itself to address underfitting/overfitting through task relation matrices \cite{wang2024bhk}, continue to strengthen the field's scientific foundation.

In conclusion, Deep Meta-Learning has evolved from a theoretical concept to a practical necessity, enabling AI systems to adapt rapidly, learn from minimal data, and generalize robustly across complex and dynamic environments. Its advancements across optimization-based, metric-based, model-based, and probabilistic paradigms have collectively pushed the boundaries of intelligent systems, integrating with foundation models, prioritizing safety and robustness, and exploring new applications. This underscores its pivotal role in the future of adaptable and autonomous AI.
\subsection{Remaining Challenges and Open Questions}
\label{sec:9\_2\_remaining\_challenges\_\_and\_\_open\_questions}

Despite the remarkable progress in Deep Meta-Learning, particularly in enabling few-shot learning and rapid adaptation, the field continues to grapple with several fundamental challenges and open questions. These unresolved issues are critical for guiding future research towards the development of more capable, efficient, theoretically grounded, and reliable meta-learning systems \cite{hospedales2020m37}.

One of the most pressing challenges is achieving \textbf{robust generalization to truly novel and diverse task distributions}, especially in extreme out-of-distribution (OOD) scenarios, and mitigating the phenomenon of "meta-memorization." While meta-learning aims to acquire an adaptive learning algorithm, its effectiveness can be severely limited by the similarity between meta-training and meta-test tasks \cite{peng20209of}. Early frameworks like MAML \cite{finn20174c4} provided a powerful basis, yet subsequent analysis revealed that meta-learners could overfit to the meta-training task distribution, learning to solve specific tasks rather than a general adaptation strategy \cite{tseng2020m83}. This "meta-memorization" implies that the model might rely on spurious correlations or task-identifying information rather than truly learning to adapt, necessitating careful task design or regularization \cite{yin2019cct}. Addressing this requires moving beyond simply learning robust feature representations, as discussed in Section 6.3, to developing meta-learning algorithms that are inherently robust to \textit{structural shifts} in task distributions. Recent theoretical advancements, such as the "Learning Lens" proposed by TRLearner \cite{wang2024bhk}, aim to tackle underfitting and overfitting by leveraging task relation matrices and consistency regularization, offering theoretical guarantees for improved generalization. Similarly, information-theoretic analyses are providing novel generalization bounds for meta-learning, offering deeper insights into the conditions under which meta-learners can truly generalize beyond their training distribution \cite{chen2021j5t}. The core open question here is whether representation-centric or algorithm-centric solutions are fundamentally more effective for OOD generalization, or how they can be optimally combined to prevent meta-memorization and ensure true adaptive capacity.

Another critical concern lies in \textbf{balancing computational efficiency with model expressiveness and scalability} for very diverse task distributions. Meta-training, particularly with bi-level optimization involving second-order gradients (as in MAML), can be computationally intensive and memory-demanding, limiting its application to large models or extensive task sets \cite{huisman2020b7w}. While first-order approximations (Section 3.2) have mitigated some of these costs, a fundamental trade-off persists between the accuracy of gradient estimation and the stability of meta-convergence, especially for non-convex meta-objectives. The challenge extends to scaling meta-training itself: real-world applications often demand adaptation across a vast number of heterogeneous tasks, which can be prohibitively expensive to curate and train on. Innovations in architectural design, such as Neuro-Modulated Networks (NMNs) that dynamically tune activation function parameters \cite{vecoven2018hc1}, offer avenues for more efficient parameter scaling and faster adaptation. Furthermore, the development of meta-adaptive optimizers like MADA \cite{ozkara2024nst} and offline meta-reinforcement learning methods such as BOReL \cite{dorfman2020mgv} are crucial for improving sample efficiency and reducing the computational burden associated with data collection and training. The open question remains: can we design meta-learning systems that are both highly expressive, capable of learning complex adaptive behaviors, and yet remain computationally tractable and scalable to the immense diversity of real-world tasks?

The development of \textbf{more theoretically grounded algorithms} is paramount to move beyond empirical successes and provide stronger guarantees for meta-learning systems. Many existing algorithms are largely empirically driven, lacking comprehensive theoretical understanding of their convergence properties, generalization bounds, and robustness. For instance, even for foundational algorithms like MAML, the optimal configuration of inner-loop learning rates during meta-training has been a subject of debate. Counter-intuitively, theoretical analysis has shown that an optimal \textit{negative} inner-loop learning rate can exist during meta-training for MAML, a finding that challenges conventional assumptions about gradient descent and highlights the depth of theoretical unknowns in meta-optimization \cite{bernacchia20211r0}. This points to a need for a more principled understanding of meta-optimization dynamics. Probabilistic meta-learning approaches, such as VariBAD \cite{zintgraf2019zat} and PEARL \cite{rakelly2019m09}, represent a significant step towards theoretical grounding by explicitly modeling task uncertainty and enabling Bayes-adaptive exploration. Moreover, for safety-critical applications, the emergence of Meta-Safe Reinforcement Learning (Meta-SRL) frameworks that provide provable guarantees for task-averaged regret and constraint violations marks a crucial advancement in building theoretically robust and reliable adaptive systems \cite{khattar2024sr6}. The overarching question is how to develop a unified theoretical framework that can explain and predict the behavior of diverse meta-learning paradigms, offering prescriptive guidance for algorithm design rather than relying solely on empirical validation.

The \textbf{optimal synergy and fundamental challenges in combining meta-learning with other AI paradigms} represent a rapidly expanding frontier. While meta-learning has been successfully integrated with various fields (as discussed in Sections 7 and 8), the deeper research questions at these intersections remain largely open. In Meta-Reinforcement Learning (Meta-RL), for example, beyond improving sample efficiency (Section 7.2), critical challenges persist in areas such as robust exploration in novel environments, effective transfer of meta-learned policies across vastly different task distributions, and the development of multi-agent meta-RL systems that can coordinate and adapt in complex interactive settings \cite{beck2023x24}. For Foundation Models (Section 8.1), the challenge is not merely adapting them to new tasks, but understanding how meta-learning can enable these models to truly \textit{learn new capabilities} or \textit{reason adaptively} in a few-shot manner, rather than just performing parameter-efficient fine-tuning or prompt engineering. How can the episodic training structure of meta-learning be reconciled with the decentralized, non-IID nature of Federated Learning to achieve privacy-preserving and globally adaptive models? Furthermore, how can meta-learning optimally leverage the rich representations learned through self-supervised learning to facilitate more sophisticated forms of adaptive knowledge transfer, moving beyond simply using them as better features? These integrations demand a deeper understanding of how different learning paradigms can complement each other at a fundamental algorithmic level, rather than just being combined for specific applications.

Finally, the \textbf{interpretability of complex meta-learned behaviors} remains a significant open question. Deep learning models are often considered black boxes, and meta-learning adds another layer of abstraction, making it even more challenging to understand \textit{how} a meta-learner arrives at its adaptive strategies. Dissecting the mechanisms of adaptation, especially in models that implicitly learn optimization algorithms (e.g., RNN-based meta-learners, Section 5.1) or dynamically modulate their own architectures (e.g., NMNs \cite{vecoven2018hc1}), is profoundly difficult. While general Explainable AI (XAI) approaches are evolving, their application to meta-learning is still nascent. Some progress has been made in specific contexts, such as deriving confidence estimates from learned data distributions in metric-based meta-learning for interpretable predictions \cite{tam2024a1h}. However, for highly complex meta-learned policies, particularly those operating in safety-critical domains, a comprehensive understanding of \textit{why} a meta-learner adapts in a certain way, and \textit{what knowledge} it has truly acquired, is crucial for building trustworthy AI systems. This necessitates research into developing "meta-interpretability" techniques that can shed light on the meta-learning process itself.

In conclusion, while Deep Meta-Learning has achieved remarkable strides, the field is still grappling with fundamental issues of generalization to truly novel and diverse tasks, balancing computational demands with model power, and establishing robust theoretical foundations. Future research must focus on developing meta-learning systems that are not only highly adaptive and sample-efficient but also theoretically sound, computationally scalable, interpretable, and synergistically integrated with other AI paradigms, enabling their responsible deployment in a wider array of safety-critical and real-world applications.
\subsection{Ethical Considerations and Societal Implications}
\label{sec:9\_3\_ethical\_considerations\_\_and\_\_societal\_implications}

The rapid advancements in Deep Meta-Learning, while promising unprecedented adaptability and efficiency in AI systems, concurrently raise critical ethical considerations and societal implications that demand proactive attention. As AI agents learn to learn and adapt with increasing autonomy, the potential for misuse, challenges in ensuring fairness and transparency, and the evolving dynamics of human-AI collaboration become paramount concerns.

The very power of highly adaptive meta-learned AI, exemplified by early deep meta-reinforcement learning frameworks \cite{wang20167px}, introduces inherent risks, particularly when deployed in safety-critical domains. The ability of these systems to rapidly acquire new skills and adapt to novel environments means that any unintended or malicious behavior could propagate quickly and unpredictably. Addressing this, recent research has begun to explicitly integrate safety into meta-learning frameworks. For instance, \cite{khattar2024sr6} introduces a "CMDP-within-online" framework for Meta-Safe Reinforcement Learning, providing the first provable guarantees for task-averaged regret and constraint violations. This work is crucial for mitigating risks by ensuring that meta-learned agents strictly adhere to safety constraints even while adapting to unseen tasks. Similarly, for embodied AI operating in physical environments, \cite{lupu20249p4} proposes MAGICVFM, a meta-learning adaptation for ground interaction control with visual foundation models, offering mathematical guarantees of exponential stability and robustness. Such guarantees are fundamental for ensuring the safe and reliable operation of autonomous systems, thereby fostering public trust and accountability.

Beyond direct safety, ensuring fairness and transparency in meta-learned systems presents a significant challenge. The complex, adaptive nature of these models can obscure their decision-making processes, making it difficult to identify and rectify biases or understand why a particular adaptation occurred. This lack of interpretability can lead to overconfidence in system outputs, especially in sensitive applications. To address this, \cite{tam2024a1h} leverages deep metric meta-learning for robust and interpretable EMG-based hand gesture recognition, explicitly tackling the issues of overconfidence and lack of interpretability through a robust class proximity-based confidence estimator. This innovation is vital for safety-critical human-machine interfaces, where transparent decision-making is essential for user trust and system reliability. Furthermore, the ability of meta-learned systems to generalize robustly to out-of-distribution data is a prerequisite for fairness and accountability. As highlighted by the survey on Domain Generalization through Meta-Learning \cite{khoee2024ksk}, effective generalization ensures that models perform reliably across diverse, potentially unseen scenarios, reducing the likelihood of discriminatory outcomes. Improvements in meta-learning calibration, such as those proposed by \cite{wang2024bhk} with TRLearner, which mitigates underfitting and overfitting through task-relation-aware consistency regularization, contribute to more reliable and thus fairer systems. The increasing integration of meta-learning with powerful foundation models, as seen in prompt tuning for Vision-Language Models \cite{wang2024dai}, further amplifies these concerns, as biases embedded in large pre-trained models could be rapidly propagated and adapted to new tasks.

The implications for human-AI collaboration are also profound. As meta-learned systems become more autonomous and capable of discovering their own learning strategies \cite{wang20167px, rakelly2019m09}, the nature of human oversight and interaction must evolve. Effective collaboration hinges on trust, which is built upon predictability, reliability, and transparency. The development of interpretable meta-learning models \cite{tam2024a1h} and systems with provable safety guarantees \cite{khattar2024sr6, lupu20249p4} are crucial steps towards building this trust. Ultimately, the goal is to develop responsible meta-learning frameworks that prioritize safety, accountability, and alignment with human values, ensuring that these powerful technologies contribute positively to society while mitigating potential risks and fostering public confidence. The ongoing research in robust generalization, interpretable adaptation, and provably safe meta-learning represents a critical trajectory towards achieving this delicate balance.


\newpage
\section*{References}
\addcontentsline{toc}{section}{References}

\begin{thebibliography}{285}

\bibitem{sung2017nc5}
Flood Sung, Yongxin Yang, Li Zhang, et al. (2017). \textit{Learning to Compare: Relation Network for Few-Shot Learning}. 2018 IEEE/CVF Conference on Computer Vision and Pattern Recognition.

\bibitem{hospedales2020m37}
Timothy M. Hospedales, Antreas Antoniou, P. Micaelli, et al. (2020). \textit{Meta-Learning in Neural Networks: A Survey}. IEEE Transactions on Pattern Analysis and Machine Intelligence.

\bibitem{santoro2016323}
Adam Santoro, Sergey Bartunov, M. Botvinick, et al. (2016). \textit{Meta-Learning with Memory-Augmented Neural Networks}. International Conference on Machine Learning.

\bibitem{sun2018iy7}
Qianru Sun, Yaoyao Liu, Tat-Seng Chua, et al. (2018). \textit{Meta-Transfer Learning for Few-Shot Learning}. Computer Vision and Pattern Recognition.

\bibitem{bertinetto2018ur2}
Luca Bertinetto, João F. Henriques, Philip H. S. Torr, et al. (2018). \textit{Meta-learning with differentiable closed-form solvers}. International Conference on Learning Representations.

\bibitem{rakelly2019m09}
Kate Rakelly, Aurick Zhou, Deirdre Quillen, et al. (2019). \textit{Efficient Off-Policy Meta-Reinforcement Learning via Probabilistic Context Variables}. International Conference on Machine Learning.

\bibitem{wang20167px}
Jane X. Wang, Z. Kurth-Nelson, Hubert Soyer, et al. (2016). \textit{Learning to reinforcement learn}. Annual Meeting of the Cognitive Science Society.

\bibitem{franceschi2018u1q}
Luca Franceschi, P. Frasconi, Saverio Salzo, et al. (2018). \textit{Bilevel Programming for Hyperparameter Optimization and Meta-Learning}. International Conference on Machine Learning.

\bibitem{pan2019pue}
Zheyi Pan, Yuxuan Liang, Weifeng Wang, et al. (2019). \textit{Urban Traffic Prediction from Spatio-Temporal Data Using Deep Meta Learning}. Knowledge Discovery and Data Mining.

\bibitem{lanctot2017m2v}
Marc Lanctot, V. Zambaldi, A. Gruslys, et al. (2017). \textit{A Unified Game-Theoretic Approach to Multiagent Reinforcement Learning}. Neural Information Processing Systems.

\bibitem{finn20174c4}
Chelsea Finn, Tianhe Yu, Tianhao Zhang, et al. (2017). \textit{One-Shot Visual Imitation Learning via Meta-Learning}. Conference on Robot Learning.

\bibitem{huisman2020b7w}
M. Huisman, Jan N. van Rijn, and A. Plaat (2020). \textit{A survey of deep meta-learning}. Artificial Intelligence Review.

\bibitem{oconnell2022twd}
M. O'Connell, Guanya Shi, Xichen Shi, et al. (2022). \textit{Neural-Fly enables rapid learning for agile flight in strong winds}. Science Robotics.

\bibitem{zhu2020rb5}
Hancheng Zhu, Leida Li, Jinjian Wu, et al. (2020). \textit{MetaIQA: Deep Meta-Learning for No-Reference Image Quality Assessment}. Computer Vision and Pattern Recognition.

\bibitem{herzen2021300}
J. Herzen, Francesco Lässig, Samuele Giuliano Piazzetta, et al. (2021). \textit{Darts: User-Friendly Modern Machine Learning for Time Series}. Journal of machine learning research.

\bibitem{wang2020tae}
Jin Wang, Jia Hu, G. Min, et al. (2020). \textit{Fast Adaptive Task Offloading in Edge Computing Based on Meta Reinforcement Learning}. IEEE Transactions on Parallel and Distributed Systems.

\bibitem{yu2018nm7}
Tianhe Yu, Chelsea Finn, Annie Xie, et al. (2018). \textit{One-Shot Imitation from Observing Humans via Domain-Adaptive Meta-Learning}. Robotics: Science and Systems.

\bibitem{zintgraf2019zat}
Luisa M. Zintgraf, Kyriacos Shiarlis, Maximilian Igl, et al. (2019). \textit{VariBAD: A Very Good Method for Bayes-Adaptive Deep RL via Meta-Learning}. International Conference on Learning Representations.

\bibitem{li2018soc}
Junnan Li, Yongkang Wong, Qi Zhao, et al. (2018). \textit{Learning to Learn From Noisy Labeled Data}. Computer Vision and Pattern Recognition.

\bibitem{guo2021zpk}
Zhichun Guo, Chuxu Zhang, W. Yu, et al. (2021). \textit{Few-Shot Graph Learning for Molecular Property Prediction}. The Web Conference.

\bibitem{li20219tk}
Shuang Li, Kaixiong Gong, Chi Harold Liu, et al. (2021). \textit{MetaSAug: Meta Semantic Augmentation for Long-Tailed Visual Recognition}. Computer Vision and Pattern Recognition.

\bibitem{tian2022znj}
Yingjie Tian, Xiaoxi Zhao, and Wei-Hsin Huang (2022). \textit{Meta-learning approaches for learning-to-learn in deep learning: A survey}. Neurocomputing.

\bibitem{oh2017x02}
Junhyuk Oh, Satinder Singh, Honglak Lee, et al. (2017). \textit{Zero-Shot Task Generalization with Multi-Task Deep Reinforcement Learning}. International Conference on Machine Learning.

\bibitem{papoudakis2019gyl}
Georgios Papoudakis, Filippos Christianos, Arrasy Rahman, et al. (2019). \textit{Dealing with Non-Stationarity in Multi-Agent Deep Reinforcement Learning}. arXiv.org.

\bibitem{yoon2019k84}
Jinsung Yoon, Sercan Ö. Arik, and Tomas Pfister (2019). \textit{Data Valuation using Reinforcement Learning}. International Conference on Machine Learning.

\bibitem{rajasegaran2020llk}
Jathushan Rajasegaran, Salman Hameed Khan, Munawar Hayat, et al. (2020). \textit{iTAML: An Incremental Task-Agnostic Meta-learning Approach}. Computer Vision and Pattern Recognition.

\bibitem{yin2019cct}
Mingzhang Yin, G. Tucker, Mingyuan Zhou, et al. (2019). \textit{Meta-Learning without Memorization}. International Conference on Learning Representations.

\bibitem{qiao2019p6r}
Limeng Qiao, Yemin Shi, Jia Li, et al. (2019). \textit{Transductive Episodic-Wise Adaptive Metric for Few-Shot Learning}. IEEE International Conference on Computer Vision.

\bibitem{gao2020h75}
Kuiliang Gao, Bing Liu, Xuchu Yu, et al. (2020). \textit{Deep Relation Network for Hyperspectral Image Few-Shot Classification}. Remote Sensing.

\bibitem{patacchiola2020kpq}
Massimiliano Patacchiola, Jack Turner, Elliot J. Crowley, et al. (2020). \textit{Bayesian Meta-Learning for the Few-Shot Setting via Deep Kernels}. Neural Information Processing Systems.

\bibitem{nagabandi2018esl}
Anusha Nagabandi, Chelsea Finn, and S. Levine (2018). \textit{Deep Online Learning via Meta-Learning: Continual Adaptation for Model-Based RL}. International Conference on Learning Representations.

\bibitem{finn2017vrt}
Chelsea Finn, and S. Levine (2017). \textit{Meta-Learning and Universality: Deep Representations and Gradient Descent can Approximate any Learning Algorithm}. International Conference on Learning Representations.

\bibitem{such2019xok}
F. Such, Aditya Rawal, J. Lehman, et al. (2019). \textit{Generative Teaching Networks: Accelerating Neural Architecture Search by Learning to Generate Synthetic Training Data}. International Conference on Machine Learning.

\bibitem{fei20211x6}
Nanyi Fei, Yizhao Gao, Zhiwu Lu, et al. (2021). \textit{Z-Score Normalization, Hubness, and Few-Shot Learning}. IEEE International Conference on Computer Vision.

\bibitem{wang2021ya6}
Shaofei Wang, Marko Mihajlovic, Qianli Ma, et al. (2021). \textit{MetaAvatar: Learning Animatable Clothed Human Models from Few Depth Images}. Neural Information Processing Systems.

\bibitem{jang2019a48}
Yunhun Jang, Hankook Lee, Sung Ju Hwang, et al. (2019). \textit{Learning What and Where to Transfer}. International Conference on Machine Learning.

\bibitem{zhu2022zp1}
Taiyu Zhu, Kezhi Li, P. Herrero, et al. (2022). \textit{Personalized Blood Glucose Prediction for Type 1 Diabetes Using Evidential Deep Learning and Meta-Learning}. IEEE Transactions on Biomedical Engineering.

\bibitem{gao20223fn}
Jun Gao, Dongze Wu, Feng Yin, et al. (2022). \textit{MetaLoc: Learning to Learn Wireless Localization}. IEEE Journal on Selected Areas in Communications.

\bibitem{bartler2021i8o}
Alexander Bartler, Andreas Bühler, Felix Wiewel, et al. (2021). \textit{MT3: Meta Test-Time Training for Self-Supervised Test-Time Adaption}. International Conference on Artificial Intelligence and Statistics.

\bibitem{memon2022j2y}
M. Memon, Pardeep Kumar, and R. Iqbal (2022). \textit{Meta Deep Learn Leaf Disease Identification Model for Cotton Crop}. De Computis.

\bibitem{yang2018p36}
Zhaoyang Yang, K. Merrick, Lianwen Jin, et al. (2018). \textit{Hierarchical Deep Reinforcement Learning for Continuous Action Control}. IEEE Transactions on Neural Networks and Learning Systems.

\bibitem{guo2020acf}
Huifeng Guo, Bo Chen, Ruiming Tang, et al. (2020). \textit{An Embedding Learning Framework for Numerical Features in CTR Prediction}. Knowledge Discovery and Data Mining.

\bibitem{dixit20218dd}
Sonal Dixit, N. Verma, and A. K. Ghosh (2021). \textit{Intelligent Fault Diagnosis of Rotary Machines: Conditional Auxiliary Classifier GAN Coupled With Meta Learning Using Limited Data}. IEEE Transactions on Instrumentation and Measurement.

\bibitem{zhou20200ls}
Allan Zhou, Tom Knowles, and Chelsea Finn (2020). \textit{Meta-Learning Symmetries by Reparameterization}. International Conference on Learning Representations.

\bibitem{rajasegaran2020glw}
Jathushan Rajasegaran, Salman Hameed Khan, Munawar Hayat, et al. (2020). \textit{Self-supervised Knowledge Distillation for Few-shot Learning}. British Machine Vision Conference.

\bibitem{wang2022va1}
Yulei Wang, Xi Chen, Fengchao Wang, et al. (2022). \textit{Meta-Learning based Hyperspectral Target Detection using Siamese Network}. IEEE Transactions on Geoscience and Remote Sensing.

\bibitem{dufumier2021ec1}
Benoit Dufumier, P. Gori, J. Victor, et al. (2021). \textit{Contrastive Learning with Continuous Proxy Meta-Data for 3D MRI Classification}. International Conference on Medical Image Computing and Computer-Assisted Intervention.

\bibitem{wistuba2021wha}
Martin Wistuba, and Josif Grabocka (2021). \textit{Few-Shot Bayesian Optimization with Deep Kernel Surrogates}. International Conference on Learning Representations.

\bibitem{ren2019nu0}
Yuxiang Ren, Bo Liu, Chao Huang, et al. (2019). \textit{Heterogeneous Deep Graph Infomax}. arXiv.org.

\bibitem{zhang2020p3y}
Shen Zhang, Fei Ye, Bingnan Wang, et al. (2020). \textit{Few-Shot Bearing Fault Diagnosis Based on Model-Agnostic Meta-Learning}. IEEE transactions on industry applications.

\bibitem{zhou20188lr}
Fengwei Zhou, Bin Wu, and Zhenguo Li (2018). \textit{Deep Meta-Learning: Learning to Learn in the Concept Space}. arXiv.org.

\bibitem{bing2022om0}
Zhenshan Bing, David Lerch, Kai Huang, et al. (2022). \textit{Meta-Reinforcement Learning in Non-Stationary and Dynamic Environments}. IEEE Transactions on Pattern Analysis and Machine Intelligence.

\bibitem{tian20200qx}
Shan Tian, Songsong Mo, Liwei Wang, et al. (2020). \textit{Deep Reinforcement Learning-Based Approach to Tackle Topic-Aware Influence Maximization}. Data Science and Engineering.

\bibitem{cai20215z1}
Rizhao Cai, Zhi Li, Renjie Wan, et al. (2021). \textit{Learning Meta Pattern for Face Anti-Spoofing}. IEEE Transactions on Information Forensics and Security.

\bibitem{wang2021i3l}
Yaqing Wang, Fenglong Ma, Haoyu Wang, et al. (2021). \textit{Multimodal Emergent Fake News Detection via Meta Neural Process Networks}. Knowledge Discovery and Data Mining.

\bibitem{nam2022z75}
Taewook Nam, Shao-Hua Sun, Karl Pertsch, et al. (2022). \textit{Skill-based Meta-Reinforcement Learning}. International Conference on Learning Representations.

\bibitem{zhu2022d9a}
Hancheng Zhu, Leida Li, Jinjian Wu, et al. (2022). \textit{Generalizable No-Reference Image Quality Assessment via Deep Meta-Learning}. IEEE transactions on circuits and systems for video technology (Print).

\bibitem{wang20204p9}
Zhen Wang, Guosheng Hu, and Q. Hu (2020). \textit{Training Noise-Robust Deep Neural Networks via Meta-Learning}. Computer Vision and Pattern Recognition.

\bibitem{xu2020txy}
Zhongwen Xu, H. V. Hasselt, Matteo Hessel, et al. (2020). \textit{Meta-Gradient Reinforcement Learning with an Objective Discovered Online}. Neural Information Processing Systems.

\bibitem{xue2022ram}
Siqiao Xue, C. Qu, X. Shi, et al. (2022). \textit{A Meta Reinforcement Learning Approach for Predictive Autoscaling in the Cloud}. Knowledge Discovery and Data Mining.

\bibitem{li20208tg}
Shunkai Li, Xin Wang, Yingdian Cao, et al. (2020). \textit{Self-Supervised Deep Visual Odometry With Online Adaptation}. Computer Vision and Pattern Recognition.

\bibitem{chai2022kv5}
Yitian Chai, Hongyu Zhang, Beijun Shen, et al. (2022). \textit{Cross-Domain Deep Code Search with Meta Learning}. International Conference on Software Engineering.

\bibitem{yi2021547}
Hongsuk Yi, and Khac-Hoai Nam Bui (2021). \textit{An Automated Hyperparameter Search-Based Deep Learning Model for Highway Traffic Prediction}. IEEE transactions on intelligent transportation systems (Print).

\bibitem{pang2018qqo}
Kunkun Pang, Mingzhi Dong, Yang Wu, et al. (2018). \textit{Meta-Learning Transferable Active Learning Policies by Deep Reinforcement Learning}. arXiv.org.

\bibitem{zhang2020s15}
Haimiao Zhang, Baodong Liu, Hengyong Yu, et al. (2020). \textit{MetaInv-Net: Meta Inversion Network for Sparse View CT Image Reconstruction}. IEEE Transactions on Medical Imaging.

\bibitem{zintgraf2021lv1}
Luisa M. Zintgraf, Sebastian Schulze, Cong Lu, et al. (2021). \textit{VariBAD: Variational Bayes-Adaptive Deep RL via Meta-Learning}. Journal of machine learning research.

\bibitem{ouyang2021c4t}
W. Ouyang, Xiuwu Zhang, Shukui Ren, et al. (2021). \textit{Learning Graph Meta Embeddings for Cold-Start Ads in Click-Through Rate Prediction}. Annual International ACM SIGIR Conference on Research and Development in Information Retrieval.

\bibitem{li2019gpj}
Juncheng Li, X. Wang, Siliang Tang, et al. (2019). \textit{Unsupervised Reinforcement Learning of Transferable Meta-Skills for Embodied Navigation}. Computer Vision and Pattern Recognition.

\bibitem{yu2019o41}
Lantao Yu, Tianhe Yu, Chelsea Finn, et al. (2019). \textit{Meta-Inverse Reinforcement Learning with Probabilistic Context Variables}. Neural Information Processing Systems.

\bibitem{chen2021j5t}
Qi Chen, Changjian Shui, and M. Marchand (2021). \textit{Generalization Bounds For Meta-Learning: An Information-Theoretic Analysis}. Neural Information Processing Systems.

\bibitem{zintgraf2021hoc}
Luisa M. Zintgraf, Sam Devlin, K. Ciosek, et al. (2021). \textit{Deep Interactive Bayesian Reinforcement Learning via Meta-Learning}. Adaptive Agents and Multi-Agent Systems.

\bibitem{xu20199h0}
Hao Xu, Haibin Chang, and Dongxiao Zhang (2019). \textit{DL-PDE: Deep-learning based data-driven discovery of partial differential equations from discrete and noisy data}. Communications in Computational Physics.

\bibitem{ding2021284}
Kaize Ding, Jianling Wang, James Caverlee, et al. (2021). \textit{Meta Propagation Networks for Graph Few-shot Semi-supervised Learning}. AAAI Conference on Artificial Intelligence.

\bibitem{reed2017sxd}
Scott E. Reed, Yutian Chen, T. Paine, et al. (2017). \textit{Few-shot Autoregressive Density Estimation: Towards Learning to Learn Distributions}. International Conference on Learning Representations.

\bibitem{shao2021loj}
Huikai Shao, Dexing Zhong, Xuefeng Du, et al. (2021). \textit{Few-Shot Learning for Palmprint Recognition via Meta-Siamese Network}. IEEE Transactions on Instrumentation and Measurement.

\bibitem{jomaa20190ul}
H. Jomaa, Josif Grabocka, and L. Schmidt-Thieme (2019). \textit{Dataset2Vec: learning dataset meta-features}. Data mining and knowledge discovery.

\bibitem{woo2022f3e}
Gerald Woo, Chenghao Liu, Doyen Sahoo, et al. (2022). \textit{Learning Deep Time-index Models for Time Series Forecasting}. International Conference on Machine Learning.

\bibitem{park2020m5z}
Seongho Park, S. Han, Ji-Won Baek, et al. (2020). \textit{Meta Variance Transfer: Learning to Augment from the Others}. International Conference on Machine Learning.

\bibitem{chen2019oep}
Guangyi Chen, Tianren Zhang, Jiwen Lu, et al. (2019). \textit{Deep Meta Metric Learning}. IEEE International Conference on Computer Vision.

\bibitem{marra20192vv}
G. Marra, Francesco Giannini, Michelangelo Diligenti, et al. (2019). \textit{Integrating Learning and Reasoning with Deep Logic Models}. ECML/PKDD.

\bibitem{casebeer20225fs}
Jonah Casebeer, Nicholas J. Bryan, and Paris Smaragdis (2022). \textit{Meta-AF: Meta-Learning for Adaptive Filters}. IEEE/ACM Transactions on Audio Speech and Language Processing.

\bibitem{nobakht2022p86}
Mehrnoosh Nobakht, R. Javidan, and A. Pourebrahimi (2022). \textit{DEMD-IoT: a deep ensemble model for IoT malware detection using CNNs and network traffic}. Evolutionary Systematics.

\bibitem{vasconcelos2021fn3}
C. Vasconcelos, H. Larochelle, Vincent Dumoulin, et al. (2021). \textit{Impact of Aliasing on Generalization in Deep Convolutional Networks}. IEEE International Conference on Computer Vision.

\bibitem{libin2020x8v}
Pieter J. K. Libin, Arno Moonens, T. Verstraeten, et al. (2020). \textit{Deep reinforcement learning for large-scale epidemic control}. ECML/PKDD.

\bibitem{algan2020u0v}
G. Algan, and I. Ulusoy (2020). \textit{Meta Soft Label Generation for Noisy Labels}. International Conference on Pattern Recognition.

\bibitem{lan20196o7}
Lin Lan, Zhenguo Li, X. Guan, et al. (2019). \textit{Meta Reinforcement Learning with Task Embedding and Shared Policy}. International Joint Conference on Artificial Intelligence.

\bibitem{huang20214b2}
Chao Huang, Zhangjie Cao, Yunbo Wang, et al. (2021). \textit{MetaSets: Meta-Learning on Point Sets for Generalizable Representations}. Computer Vision and Pattern Recognition.

\bibitem{chen2022z45}
Wenlin Chen, Austin Tripp, and José Miguel Hernández-Lobato (2022). \textit{Meta-learning Adaptive Deep Kernel Gaussian Processes for Molecular Property Prediction}. International Conference on Learning Representations.

\bibitem{abdollahzadeh2021zfy}
Milad Abdollahzadeh, Touba Malekzadeh, and Ngai-Man Cheung (2021). \textit{Revisit Multimodal Meta-Learning through the Lens of Multi-Task Learning}. Neural Information Processing Systems.

\bibitem{chen2019xg0}
Shangyu Chen, Wenya Wang, and Sinno Jialin Pan (2019). \textit{MetaQuant: Learning to Quantize by Learning to Penetrate Non-differentiable Quantization}. Neural Information Processing Systems.

\bibitem{xu2018rdh}
Tianbing Xu, Qiang Liu, Liang Zhao, et al. (2018). \textit{Learning to Explore with Meta-Policy Gradient}. International Conference on Machine Learning.

\bibitem{shen2022kdk}
Xiangyu Shen, Haifeng Zheng, Xinxin Feng, et al. (2022). \textit{ML-HGR-Net: A Meta-Learning Network for FMCW Radar Based Hand Gesture Recognition}. IEEE Sensors Journal.

\bibitem{zhang2021hh1}
Huayi Zhang, Lei Cao, Peter M. VanNostrand, et al. (2021). \textit{ELITE: Robust Deep Anomaly Detection with Meta Gradient}. Knowledge Discovery and Data Mining.

\bibitem{lin2022i6k}
Hong Lin, Rita Tse, Su-Kit Tang, et al. (2022). \textit{Few-shot learning approach with multi-scale feature fusion and attention for plant disease recognition}. Frontiers in Plant Science.

\bibitem{tseng2020m83}
Hung-Yu Tseng, Yi-Wen Chen, Yi-Hsuan Tsai, et al. (2020). \textit{Regularizing Meta-Learning via Gradient Dropout}. Asian Conference on Computer Vision.

\bibitem{peng20209of}
Huimin Peng (2020). \textit{A Comprehensive Overview and Survey of Recent Advances in Meta-Learning}. arXiv.org.

\bibitem{fong20183q5}
S. Fong, S. Deb, and Xin-She Yang (2018). \textit{How meta-heuristic algorithms contribute to deep learning in the hype of big data analytics}. Unpublished manuscript.

\bibitem{liu2022tgc}
Chang Liu, Xiang Yu, Yao-Hung Hubert Tsai, et al. (2022). \textit{Learning to Learn across Diverse Data Biases in Deep Face Recognition}. Computer Vision and Pattern Recognition.

\bibitem{bing2022xo7}
Zhenshan Bing, A. Koch, Xiangtong Yao, et al. (2022). \textit{Meta-Reinforcement Learning via Language Instructions}. IEEE International Conference on Robotics and Automation.

\bibitem{nakasi2020w5x}
Rose Nakasi, Ernest Mwebaze, A. Zawedde, et al. (2020). \textit{A new approach for microscopic diagnosis of malaria parasites in thick blood smears using pre-trained deep learning models}. SN Applied Sciences.

\bibitem{przewiezlikowski2022d4y}
Marcin Przewiezlikowski, P. Przybysz, J. Tabor, et al. (2022). \textit{HyperMAML: Few-Shot Adaptation of Deep Models with Hypernetworks}. Neurocomputing.

\bibitem{xu2018rjq}
Tianbing Xu, Qiang Liu, Liang Zhao, et al. (2018). \textit{Learning to Explore via Meta-Policy Gradient}. International Conference on Machine Learning.

\bibitem{ding2019a79}
Yue Ding, Xia Tian, Lirong Yin, et al. (2019). \textit{Multi-scale Relation Network for Few-Shot Learning Based on Meta-learning}. International Conference on Virtual Storytelling.

\bibitem{dorfman2020gku}
Ron Dorfman, and Aviv Tamar (2020). \textit{Offline Meta Learning of Exploration}. Unpublished manuscript.

\bibitem{wang20210y3}
Ze Wang, Zichen Miao, Xiantong Zhen, et al. (2021). \textit{Learning to Learn Dense Gaussian Processes for Few-Shot Learning}. Neural Information Processing Systems.

\bibitem{millea2021bfu}
Adrian Millea (2021). \textit{Deep Reinforcement Learning for Trading - A Critical Survey}. International Conference on Data Technologies and Applications.

\bibitem{lindsey202075a}
Jack W Lindsey, and Ashok Litwin-Kumar (2020). \textit{Learning to Learn with Feedback and Local Plasticity}. Neural Information Processing Systems.

\bibitem{gao2022y3s}
Jun Gao, Ceyao Zhang, Qinglei Kong, et al. (2022). \textit{MetaLoc: Learning to Learn Indoor RSS Fingerprinting Localization over Multiple Scenarios}. ICC 2022 - IEEE International Conference on Communications.

\bibitem{ren2022fc5}
Xinyue Ren, Weiwei Zhang, Ming-hui Wu, et al. (2022). \textit{Meta-YOLO: Meta-Learning for Few-Shot Traffic Sign Detection via Decoupling Dependencies}. Applied Sciences.

\bibitem{wen2022sql}
Xin-Cheng Wen, Cuiyun Gao, Jiaxin Ye, et al. (2022). \textit{Meta-Path Based Attentional Graph Learning Model for Vulnerability Detection}. IEEE Transactions on Software Engineering.

\bibitem{vecoven2018hc1}
Nicolas Vecoven, D. Ernst, Antoine Wehenkel, et al. (2018). \textit{Introducing neuromodulation in deep neural networks to learn adaptive behaviours}. PLoS ONE.

\bibitem{liang2021juf}
Xuejian Liang, Yehui Zhang, and Junping Zhang (2021). \textit{Attention Multisource Fusion-Based Deep Few-Shot Learning for Hyperspectral Image Classification}. IEEE Journal of Selected Topics in Applied Earth Observations and Remote Sensing.

\bibitem{holla2020r6z}
Nithin Holla, Pushkar Mishra, H. Yannakoudakis, et al. (2020). \textit{Learning to Learn to Disambiguate: Meta-Learning for Few-Shot Word Sense Disambiguation}. Findings.

\bibitem{fernando2018lt5}
Chrisantha Fernando, Jakub Sygnowski, Simon Osindero, et al. (2018). \textit{Meta-learning by the Baldwin effect}. Annual Conference on Genetic and Evolutionary Computation.

\bibitem{jankowski20138zb}
Norbert Jankowski, Wlodzislaw Duch, and Krzysztof Grabczewski (2013). \textit{Meta-Learning in Computational Intelligence}. Meta-Learning in Computational Intelligence.

\bibitem{banayeeanzade2021zke}
Mohammadamin Banayeeanzade, Rasoul Mirzaiezadeh, Hosein Hasani, et al. (2021). \textit{Generative vs. Discriminative: Rethinking The Meta-Continual Learning}. Neural Information Processing Systems.

\bibitem{campedelli2021jja}
G. Campedelli, Mihovil Bartulovic, and K. Carley (2021). \textit{Learning future terrorist targets through temporal meta-graphs}. Scientific Reports.

\bibitem{jiang20220tg}
Xiruo Jiang, Sheng Liu, Xili Dai, et al. (2022). \textit{Deep Metric Learning Based on Meta-Mining Strategy With Semiglobal Information}. IEEE Transactions on Neural Networks and Learning Systems.

\bibitem{zheng2021olf}
Wenbo Zheng, Lan Yan, Chao Gou, et al. (2021). \textit{Learning to learn by yourself: Unsupervised meta‐learning with self‐knowledge distillation for COVID‐19 diagnosis from pneumonia cases}. International Journal of Intelligent Systems.

\bibitem{alandoli2021pqm}
M. Al-Andoli, Shing Chiang Tan, W. Cheah, et al. (2021). \textit{A Review on Community Detection in Large Complex Networks from Conventional to Deep Learning Methods: A Call for the Use of Parallel Meta-Heuristic Algorithms}. IEEE Access.

\bibitem{li2021tkg}
Denghao Li, Pablo Ortega, Xia Wei, et al. (2021). \textit{Model-Agnostic Meta-Learning for EEG Motor Imagery Decoding in Brain-Computer-Interfacing}. International IEEE/EMBS Conference on Neural Engineering.

\bibitem{daglarli2020nzw}
Evren Daglarli (2020). \textit{Explainable Artificial Intelligence (xAI) Approaches and Deep Meta-Learning Models}. Advances and Applications in Deep Learning.

\bibitem{phaphuangwittayakul2022api}
Aniwat Phaphuangwittayakul, Fangli Ying, Yi Guo, et al. (2022). \textit{Few-shot image generation based on contrastive meta-learning generative adversarial network}. The Visual Computer.

\bibitem{nie2021pcz}
Jing Nie, Nianyi Wang, Jingbin Li, et al. (2021). \textit{Meta-learning prediction of physical and chemical properties of magnetized water and fertilizer based on LSTM}. Plant Methods.

\bibitem{zheng20200ig}
Wenzhao Zheng, Jiwen Lu, and Jie Zhou (2020). \textit{Deep Metric Learning via Adaptive Learnable Assessment}. Computer Vision and Pattern Recognition.

\bibitem{chen2022ccz}
Yixiong Chen, Chunhui Zhang, C. Ding, et al. (2022). \textit{Generating and Weighting Semantically Consistent Sample Pairs for Ultrasound Contrastive Learning}. IEEE Transactions on Medical Imaging.

\bibitem{qu2022mu6}
H. Qu, Sheng Liu, Jun Li, et al. (2022). \textit{Adaptation and Learning to Learn (ALL): An Integrated Approach for Small-Sample Parking Occupancy Prediction}. Mathematics.

\bibitem{alajaji2020b6c}
D. Alajaji, and H. Alhichri (2020). \textit{Few Shot Scene Classification in Remote Sensing using Meta-Agnostic Machine}. 2020 6th Conference on Data Science and Machine Learning Applications (CDMA).

\bibitem{jiang2021uo6}
Mengjuan Jiang, Fanzhang Li, and Li Liu (2021). \textit{Continual meta-learning algorithm}. Applied intelligence (Boston).

\bibitem{rkhami2021c1c}
Anouar Rkhami, Yassine Hadjadj-Aoul, and A. Outtagarts (2021). \textit{Learn to improve: A novel deep reinforcement learning approach for beyond 5G network slicing}. Consumer Communications and Networking Conference.

\bibitem{holla20202od}
Nithin Holla, Pushkar Mishra, H. Yannakoudakis, et al. (2020). \textit{Meta-Learning with Sparse Experience Replay for Lifelong Language Learning}. arXiv.org.

\bibitem{bhathiya2020avm}
H. S. Bhathiya, and Uthayasanker Thayasivam (2020). \textit{Meta Learning for Few-Shot Joint Intent Detection and Slot-Filling}. International Conference on Machine Learning Technologies.

\bibitem{bernacchia20211r0}
A. Bernacchia (2021). \textit{Meta-learning with negative learning rates}. International Conference on Learning Representations.

\bibitem{kumar2017p0v}
Saurabh Kumar, Pararth Shah, Dilek Z. Hakkani-Tür, et al. (2017). \textit{Federated Control with Hierarchical Multi-Agent Deep Reinforcement Learning}. arXiv.org.

\bibitem{daglarli20216fl}
Evren Daglarli (2021). \textit{Explainable Artificial Intelligence (xAI) Approaches and Deep Meta-Learning Models for Cyber-Physical Systems}. Advances in Systems Analysis, Software Engineering, and High Performance Computing.

\bibitem{baz2022n78}
Adrian El Baz, André Carvalho, Hong Chen, et al. (2022). \textit{Lessons learned from the NeurIPS 2021 MetaDL challenge: Backbone fine-tuning without episodic meta-learning dominates for few-shot learning image classification}. Neural Information Processing Systems.

\bibitem{dorfman2020mgv}
Ron Dorfman, and Aviv Tamar (2020). \textit{Offline Meta Reinforcement Learning}. arXiv.org.

\bibitem{vuorio2018gwb}
Risto Vuorio, D.-Y. Cho, Daejoong Kim, et al. (2018). \textit{Meta Continual Learning}. arXiv.org.

\bibitem{ma2021kfz}
Ying Ma, G. Zhong, Wen Liu, et al. (2021). \textit{ML-CGAN: Conditional Generative Adversarial Network with a Meta-learner Structure for High-Quality Image Generation with Few Training Data}. Cognitive Computation.

\bibitem{tian2016j46}
Zhonghuan Tian, and S. Fong (2016). \textit{Survey of Meta-Heuristic Algorithms for Deep Learning Training}. Unpublished manuscript.

\bibitem{pinedaarango2021254}
Sebastian Pineda-Arango, Felix Heinrich, Kiran Madhusudhanan, et al. (2021). \textit{Multimodal Meta-Learning for Time Series Regression}. AALTD@ECML/PKDD.

\bibitem{foliadis20223y2}
Anastasios Foliadis, M. H. C. García, R. Stirling-Gallacher, et al. (2022). \textit{Multi-Environment based Meta-Learning with CSI Fingerprints for Radio Based Positioning}. IEEE Wireless Communications and Networking Conference.

\bibitem{sutton2022jss}
R. Sutton (2022). \textit{A History of Meta-gradient: Gradient Methods for Meta-learning}. arXiv.org.

\bibitem{huo2022rp4}
Haomiao Huo, Jindan Xu, Gege Su, et al. (2022). \textit{Intelligent MIMO Detection Using Meta Learning}. IEEE Wireless Communications Letters.

\bibitem{hurtado2021h2q}
J. Hurtado, Hans Lobel, and Álvaro Soto (2021). \textit{Overcoming Catastrophic Forgetting Using Sparse Coding and Meta Learning}. IEEE Access.

\bibitem{lee2021jou}
Hung-yi Lee, Ngoc Thang Vu, and Shang-Wen Li (2021). \textit{Meta Learning and Its Applications to Natural Language Processing}. Annual Meeting of the Association for Computational Linguistics.

\bibitem{peixoto20180pd}
B. Peixoto, S. Avila, Zanoni Dias, et al. (2018). \textit{Breaking down violence: A deep-learning strategy to model and classify violence in videos}. ARES.

\bibitem{yuan20205j8}
Pengyu Yuan, Aryan Mobiny, J. Jahanipour, et al. (2020). \textit{Few Is Enough: Task-Augmented Active Meta-Learning for Brain Cell Classification}. International Conference on Medical Image Computing and Computer-Assisted Intervention.

\bibitem{zhang2021yox}
Zitian Zhang, Fuyou Li, Xiaoli Chu, et al. (2021). \textit{dmTP: A Deep Meta-Learning Based Framework for Mobile Traffic Prediction}. IEEE wireless communications.

\bibitem{luo202123f}
Yaoru Luo, Guole Liu, Yuanhao Guo, et al. (2021). \textit{Deep Neural Networks Learn Meta-Structures from Noisy Labels in Semantic Segmentation}. AAAI Conference on Artificial Intelligence.

\bibitem{chen2021yqh}
Mengting Chen, Xinggang Wang, Heng Luo, et al. (2021). \textit{Learning to focus: cascaded feature matching network for few-shot image recognition}. Science China Information Sciences.

\bibitem{behl2018rjm}
Harkirat Singh Behl, M. Najafi, and Philip H. S. Torr (2018). \textit{Meta-Learning Deep Visual Words for Fast Video Object Segmentation}. IEEE/RJS International Conference on Intelligent RObots and Systems.

\bibitem{xu2019brv}
Zhixiong Xu, Lei Cao, and Xi-liang Chen (2019). \textit{Learning to Learn: Hierarchical Meta-Critic Networks}. IEEE Access.

\bibitem{sultana202094g}
N. Sultana, Jeffrey Chan, and A. K. Qin (2020). \textit{Learning to Optimise General TSP Instances}. arXiv.org.

\bibitem{furfaro20197q6}
R. Furfaro, T. Campbell, R. Linares, et al. (2019). \textit{Space Debris Identification and Characterization via Deep Meta-Learning}. Unpublished manuscript.

\bibitem{gu2019tvc}
Keren Gu, S. Greydanus, Luke Metz, et al. (2019). \textit{Meta-Learning Biologically Plausible Semi-Supervised Update Rules}. bioRxiv.

\bibitem{zhang2021p9j}
Yaobin Zhang, Weihong Deng, Yaoyao Zhong, et al. (2021). \textit{Adaptive Label Noise Cleaning with Meta-Supervision for Deep Face Recognition}. IEEE International Conference on Computer Vision.

\bibitem{saadallah2021ihn}
A. Saadallah, and K. Morik (2021). \textit{Online Ensemble Aggregation using Deep Reinforcement Learning for Time Series Forecasting}. International Conference on Data Science and Advanced Analytics.

\bibitem{luedtke2020uub}
Alexander Luedtke, M. Carone, N. Simon, et al. (2020). \textit{Learning to learn from data: Using deep adversarial learning to construct optimal statistical procedures}. Science Advances.

\bibitem{nasrabadi201801h}
Morteza Zakeri Nasrabadi, S. Parsa, and A. Kalaee (2018). \textit{Format-aware learn&fuzz: deep test data generation for efficient fuzzing}. Neural computing & applications (Print).

\bibitem{puri20202sx}
R. Puri, A. Zakhor, and Raul Puri (2020). \textit{Few Shot Learning For Point Cloud Data Using Model Agnostic Meta Learning}. International Conference on Information Photonics.

\bibitem{beck2023x24}
Jacob Beck, Risto Vuorio, E. Liu, et al. (2023). \textit{A Tutorial on Meta-Reinforcement Learning}. Found. Trends Mach. Learn..

\bibitem{zhang2023t7k}
Baoquan Zhang, and Demin Yu (2023). \textit{MetaDiff: Meta-Learning with Conditional Diffusion for Few-Shot Learning}. AAAI Conference on Artificial Intelligence.

\bibitem{khoee2024ksk}
Arsham Gholamzadeh Khoee, Yinan Yu, and R. Feldt (2024). \textit{Domain Generalization through Meta-Learning: A Survey}. Artificial Intelligence Review.

\bibitem{hao2023zfk}
Xiaoyang Hao, Zhixi Feng, Shuyuan Yang, et al. (2023). \textit{Automatic Modulation Classification via Meta-Learning}. IEEE Internet of Things Journal.

\bibitem{nathaniel2023ycu}
Juan Nathaniel, Jiangong Liu, and P. Gentine (2023). \textit{MetaFlux: Meta-learning global carbon fluxes from sparse spatiotemporal observations}. Scientific Data.

\bibitem{singh2023zo5}
Kuljeet Singh, and Deepti Malhotra (2023). \textit{Meta-Health: Learning-to-Learn (Meta-learning) as a Next Generation of Deep Learning Exploring Healthcare Challenges and Solutions for Rare Disorders: A Systematic Analysis}. Archives of Computational Methods in Engineering.

\bibitem{wang2023srr}
Peiqi Wang, Jingde Li, Shubei Wang, et al. (2023). \textit{A new meta-transfer learning method with freezing operation for few-shot bearing fault diagnosis}. Measurement science and technology.

\bibitem{liang2023zzh}
Xiaoxia Liang, M. Zhang, Guojin Feng, et al. (2023). \textit{A Novel Deep Model with Meta-learning for Rolling Bearing Few-shot Fault Diagnosis}. Journal of Dynamics Monitoring and Diagnostics.

\bibitem{tian2023iyh}
Pinzhuo Tian, and Shaorong Xie (2023). \textit{An Adversarial Meta-Training Framework for Cross-Domain Few-Shot Learning}. IEEE transactions on multimedia.

\bibitem{bian2024041}
Wanyu Bian, Albert Jang, and Fang Liu (2024). \textit{Multi-task Magnetic Resonance Imaging Reconstruction using Meta-learning}. Magnetic Resonance Imaging.

\bibitem{schwarz2022jfu}
Jonathan Schwarz, and Y. Teh (2022). \textit{Meta-Learning Sparse Compression Networks}. Trans. Mach. Learn. Res..

\bibitem{son2023lda}
Jaehyeon Son, Soochan Lee, and Gunhee Kim (2023). \textit{When Meta-Learning Meets Online and Continual Learning: A Survey}. IEEE Transactions on Pattern Analysis and Machine Intelligence.

\bibitem{wang2024d09}
Hui Wang, Zhiwei Han, Xufan Wang, et al. (2024). \textit{Contrastive Learning-Based Bayes-Adaptive Meta-Reinforcement Learning for Active Pantograph Control in High-Speed Railways}. IEEE Transactions on Transportation Electrification.

\bibitem{rao20232e3}
Haocong Rao, Cyril Leung, and Chun Miao (2023). \textit{Hierarchical Skeleton Meta-Prototype Contrastive Learning with Hard Skeleton Mining for Unsupervised Person Re-Identification}. International Journal of Computer Vision.

\bibitem{zhang2023jz8}
Yiqiang Zhang, Jiaxing Che, Yijun Hu, et al. (2023). \textit{Real-Time Ocean Current Compensation for AUV Trajectory Tracking Control Using a Meta-Learning and Self-Adaptation Hybrid Approach}. Italian National Conference on Sensors.

\bibitem{xiu2023ga8}
Xi Xiu, Jialun Li, Yujie Long, et al. (2023). \textit{MRLCC: an adaptive cloud task scheduling method based on meta reinforcement learning}. Journal of Cloud Computing.

\bibitem{cai2023ro7}
Shaokang Cai, Dezhi Han, and Dun Li (2023). \textit{A Feedback Semi-Supervised Learning With Meta-Gradient for Intrusion Detection}. IEEE Systems Journal.

\bibitem{lee20230j8}
Suyoung Lee, Myungsik Cho, and Young-Jin Sung (2023). \textit{Parameterizing Non-Parametric Meta-Reinforcement Learning Tasks via Subtask Decomposition}. Neural Information Processing Systems.

\bibitem{guarino2023zsq}
Idio Guarino, Chao Wang, A. Finamore, et al. (2023). \textit{Many or Few Samples?: Comparing Transfer, Contrastive and Meta-Learning in Encrypted Traffic Classification}. Traffic Monitoring and Analysis.

\bibitem{aqeel2025zql}
Muhammad Aqeel, Shakiba Sharifi, Marco Cristani, et al. (2025). \textit{Towards Real Unsupervised Anomaly Detection Via Confident Meta-Learning}. arXiv.org.

\bibitem{lee2024snq}
Soochan Lee, Hyeonseong Jeon, Jaehyeon Son, et al. (2024). \textit{Learning to Continually Learn with the Bayesian Principle}. International Conference on Machine Learning.

\bibitem{li2023asx}
Zhongshen Li, Junru Jin, Wenjia He, et al. (2023). \textit{CoraL: interpretable contrastive meta-learning for the prediction of cancer-associated ncRNA-encoded small peptides}. Briefings Bioinform..

\bibitem{schmidgall20238t4}
Samuel Schmidgall, and Joe Hays (2023). \textit{Meta-SpikePropamine: learning to learn with synaptic plasticity in spiking neural networks}. Frontiers in Neuroscience.

\bibitem{wang2023x5w}
Weijie Wang, Hong Zhao, Yikun Yang, et al. (2023). \textit{Few-shot short utterance speaker verification using meta-learning}. PeerJ Computer Science.

\bibitem{wang2023ryf}
Caiyong Wang, Haiqing Li, Wen-bin Ma, et al. (2023). \textit{MetaScleraSeg: an effective meta-learning framework for generalized sclera segmentation}. Neural computing & applications (Print).

\bibitem{li2023fhe}
Wenmei Li, Qing Liu, Yu Zhang, et al. (2023). \textit{Few-Shot Hyperspectral Image Classification Using Meta Learning and Regularized Finetuning}. IEEE Transactions on Geoscience and Remote Sensing.

\bibitem{yang20238th}
Jin Yang, Liang Bao, Wenjing Liu, et al. (2023). \textit{On a Meta Learning-Based Scheduler for Deep Learning Clusters}. IEEE Transactions on Cloud Computing.

\bibitem{raja2023hco}
K. Raja, and S. Kannimuthu (2023). \textit{Deep learning-based feature selection and prediction system for autism spectrum disorder using a hybrid meta-heuristics approach}. Journal of Intelligent & Fuzzy Systems.

\bibitem{fahim2023jsu}
M. Fahim, Vishal Sharma, R. Hunter, et al. (2023). \textit{Healthy Aging: A Deep Meta-Class Sequence Model to Integrate Intelligence in Digital Twin}. IEEE Journal of Translational Engineering in Health and Medicine.

\bibitem{hu2022y0i}
Cong Hu, Haoji Xu, and Xiaojun Wu (2022). \textit{Substitute Meta-Learning for Black-Box Adversarial Attack}. IEEE Signal Processing Letters.

\bibitem{wang2023kho}
Zhiqiang Wang, Man Li, H. Ou, et al. (2023). \textit{A Few-Shot Malicious Encrypted Traffic Detection Approach Based on Model-Agnostic Meta-Learning}. Security and Communication Networks.

\bibitem{li2023zn0}
Hui Li, Guimin Huang, Yiqun Li, et al. (2023). \textit{SEML: Self-Supervised Information-Enhanced Meta-learning for Few-Shot Text Classification}. International Journal of Computational Intelligence Systems.

\bibitem{chi20235vq}
Sunyi Chi, B. Dong, Yiming Xu, et al. (2023). \textit{APAM: Adaptive Pre-Training and Adaptive Meta Learning in Language Model for Noisy Labels and Long-Tailed Learning}. International Conference on Machine Learning and Applications.

\bibitem{manoharan2021r46}
Praveen Manoharan, Malini Pooni Venkat, S. Nagarathinam, et al. (2021). \textit{Learn to chill: intelligent chiller scheduling using meta-learning and deep reinforcement learning}. International Conference on Systems for Energy-Efficient Built Environments.

\bibitem{visca20217nt}
Marco Visca, R. Powell, Yang Gao, et al. (2021). \textit{Deep Meta-Learning Energy-Aware Path Planner for Unmanned Ground Vehicles in Unknown Terrains}. IEEE Access.

\bibitem{so2021y48}
Chaehan So (2021). \textit{Exploring Meta Learning: Parameterizing the Learning-to-learn Process for Image Classification}. Digital Signal Processing and Signal Processing Education Workshop.

\bibitem{yang2022oxf}
Zhixiong Yang, Jingyuan Xia, T. Qiu, et al. (2022). \textit{A Meta-Learning based Gradient Descent Algorithm for MU-MIMO Beamforming}. International Conference on Wireless Communications and Signal Processing.

\bibitem{gupta2021fbg}
Akash Gupta, P. Jonnalagedda, B. Bhanu, et al. (2021). \textit{Ada-VSR: Adaptive Video Super-Resolution with Meta-Learning}. ACM Multimedia.

\bibitem{zheng2021dx4}
Wenbo Zheng, Lan Yan, Fei-Yue Wang, et al. (2021). \textit{Learning from the Web: Webly Supervised Meta-Learning for Masked Face Recognition}. 2021 IEEE/CVF Conference on Computer Vision and Pattern Recognition Workshops (CVPRW).

\bibitem{zhang2021sbz}
Peng Zhang, Chenbing Liu, Xingshuo Chang, et al. (2021). \textit{Metric-based Meta-Learning Model for Few-Shot PolSAR Image Terrain Classification}. Radar.

\bibitem{wang2024jzu}
Chien-Yao Wang, I-Hau Yeh, and Hongpeng Liao (2024). \textit{YOLOv9: Learning What You Want to Learn Using Programmable Gradient Information}. European Conference on Computer Vision.

\bibitem{zhu20249wq}
Fenghao Zhu, Xinquan Wang, Chongwen Huang, et al. (2024). \textit{Robust Beamforming for RIS-Aided Communications: Gradient-Based Manifold Meta Learning}. IEEE Transactions on Wireless Communications.

\bibitem{yaghoubi2024j56}
Elaheh Yaghoubi, Elnaz Yaghoubi, Ahmed Khamees, et al. (2024). \textit{A systematic review and meta-analysis of artificial neural network, machine learning, deep learning, and ensemble learning approaches in field of geotechnical engineering}. Neural computing & applications (Print).

\bibitem{ren20242qj}
Lei Ren, Tingyu Mo, and Xuejun Cheng (2024). \textit{Meta-learning Based Domain Generalization Framework for Fault Diagnosis With Gradient Aligning and Semantic Matching}. IEEE Transactions on Industrial Informatics.

\bibitem{ik20248rp}
Gültekin Işık, and Ishak Paçal (2024). \textit{Few-shot classification of ultrasound breast cancer images using meta-learning algorithms}. Neural computing & applications (Print).

\bibitem{weilenmann2024ve2}
C. Weilenmann, A. Ziogas, T. Zellweger, et al. (2024). \textit{Single neuromorphic memristor closely emulates multiple synaptic mechanisms for energy efficient neural networks}. Nature Communications.

\bibitem{xia2024qx2}
Jingyuan Xia, Zhixiong Yang, Shengxi Li, et al. (2024). \textit{Blind Super-Resolution via Meta-Learning and Markov Chain Monte Carlo Simulation}. IEEE Transactions on Pattern Analysis and Machine Intelligence.

\bibitem{zhao2024f4b}
Haoming Zhao, Liang Ou, Ziming Zhang, et al. (2024). \textit{The value of deep learning-based X-ray techniques in detecting and classifying K-L grades of knee osteoarthritis: a systematic review and meta-analysis}. European Radiology.

\bibitem{li20242rv}
Jiawen Li, Tao Zhou, and Haoyan Cui (2024). \textit{Brain-Inspired Deep Meta-Reinforcement Learning for Active Coordinated Fault-Tolerant Load Frequency Control of Multi-Area Grids}. IEEE Transactions on Automation Science and Engineering.

\bibitem{dong2024110}
Zheng Dong, Renhe Jiang, Haotian Gao, et al. (2024). \textit{Heterogeneity-Informed Meta-Parameter Learning for Spatiotemporal Time Series Forecasting}. Knowledge Discovery and Data Mining.

\bibitem{liao2024o1z}
Huaizhang Liao, Jingyuan Xia, Zhixiong Yang, et al. (2024). \textit{Meta-Learning Based Domain Prior With Application to Optical-ISAR Image Translation}. IEEE transactions on circuits and systems for video technology (Print).

\bibitem{naskar202446a}
Gourab Naskar, Sk Mohiuddin, Samir Malakar, et al. (2024). \textit{Deepfake detection using deep feature stacking and meta-learning}. Heliyon.

\bibitem{zhang2024a5a}
Ruiyi Zhang, Rushi Qiang, Sai Ashish Somayajula, et al. (2024). \textit{AutoLoRA: Automatically Tuning Matrix Ranks in Low-Rank Adaptation Based on Meta Learning}. North American Chapter of the Association for Computational Linguistics.

\bibitem{sharma2024zlw}
Nelson Sharma, Aswini Ghosh, R. Misra, et al. (2024). \textit{Deep Meta Q-Learning Based Multi-Task Offloading in Edge-Cloud Systems}. IEEE Transactions on Mobile Computing.

\bibitem{ouyang2024xj0}
Tiancheng Ouyang, Yingying Su, Chengchao Wang, et al. (2024). \textit{Combined Meta-Learning With CNN-LSTM Algorithms for State-of-Health Estimation of Lithium-Ion Battery}. IEEE transactions on power electronics.

\bibitem{wang2025zze}
Changdong Wang, Zhou Shu, Jingli Yang, et al. (2025). \textit{Learning to Imbalanced Open Set Generalize: A Meta-Learning Framework for Enhanced Mechanical Diagnosis}. IEEE Transactions on Cybernetics.

\bibitem{chia2024ltk}
Yu-Hsin Chia, Wei-Hao Liao, S. Vyas, et al. (2024). \textit{In Vivo Intelligent Fluorescence Endo‐Microscopy by Varifocal Meta‐Device and Deep Learning}. Advancement of science.

\bibitem{li20246fg}
Wenzhong Li, Xiang Li, Yeting Xu, et al. (2024). \textit{MetaABR: A Meta-Learning Approach on Adaptative Bitrate Selection for Video Streaming}. IEEE Transactions on Mobile Computing.

\bibitem{yang2024rh9}
Jing Yang, and Xiaomin Wang (2024). \textit{Meta-learning with deep flow kernel network for few shot cross-domain remaining useful life prediction}. Reliability Engineering & System Safety.

\bibitem{gao20242uv}
Yuan Gao, Wei Chen, Fajun Li, et al. (2024). \textit{Meta‐Attention Deep Learning for Smart Development of Metasurface Sensors}. Advancement of science.

\bibitem{huang2024hlo}
Hai-Hui Huang, Jun Shu, and Yong Liang (2024). \textit{MUMA: A Multi-Omics Meta-Learning Algorithm for Data Interpretation and Classification}. IEEE journal of biomedical and health informatics.

\bibitem{liu2024hko}
Shicheng Liu, and Minghui Zhu (2024). \textit{Meta Inverse Constrained Reinforcement Learning: Convergence Guarantee and Generalization Analysis}. International Conference on Learning Representations.

\bibitem{li2024x2t}
Jiawen Li, and Yuanyuan Cheng (2024). \textit{Deep Meta-Reinforcement Learning-Based Data-Driven Active Fault Tolerance Load Frequency Control for Islanded Microgrids Considering Internet of Things}. IEEE Internet of Things Journal.

\bibitem{khattar2024sr6}
Vanshaj Khattar, Yuhao Ding, Bilgehan Sel, et al. (2024). \textit{A CMDP-within-online framework for Meta-Safe Reinforcement Learning}. International Conference on Learning Representations.

\bibitem{yen2024cxp}
Tzu-Yun Yen, Chan-Shien Ho, Yueh-peng Chen, et al. (2024). \textit{Diagnostic Accuracy of Deep Learning for the Prediction of Osteoporosis Using Plain X-rays: A Systematic Review and Meta-Analysis}. Diagnostics.

\bibitem{you2024xuq}
Linlin You, Qiyang Chen, Haohao Qu, et al. (2024). \textit{FMGCN: Federated Meta Learning-Augmented Graph Convolutional Network for EV Charging Demand Forecasting}. IEEE Internet of Things Journal.

\bibitem{hao2024dyp}
Xiaoyang Hao, Zhixi Feng, Tongqing Peng, et al. (2024). \textit{Meta-Learning Guided Label Noise Distillation for Robust Signal Modulation Classification}. IEEE Internet of Things Journal.

\bibitem{lang20246m8}
Shujun Lang, Xu Liu, Mingliang Zhou, et al. (2024). \textit{A Full-Reference Image Quality Assessment Method via Deep Meta-Learning and Conformer}. IEEE transactions on broadcasting.

\bibitem{xu2024ywn}
Meng Xu, Yi Mei, Fangfang Zhang, et al. (2024). \textit{Niching Genetic Programming to Learn Actions for Deep Reinforcement Learning in Dynamic Flexible Scheduling}. IEEE Transactions on Evolutionary Computation.

\bibitem{ding2024jjo}
Yang Ding, Heng Zhang, and Ting Qiu (2024). \textit{Deep learning approach to predict autism spectrum disorder: a systematic review and meta-analysis}. BMC Psychiatry.

\bibitem{wang2024dai}
Fengxiang Wang, Wanrong Huang, Shaowu Yang, et al. (2024). \textit{Learning to Learn Better Visual Prompts}. AAAI Conference on Artificial Intelligence.

\bibitem{nussenbaum2024z82}
Kate Nussenbaum, and Catherine A. Hartley (2024). \textit{Understanding the development of reward learning through the lens of meta-learning}. Nature Reviews Psychology.

\bibitem{alsaleh2024vdv}
Aqilah M. Alsaleh, Eid Albalawi, A. Algosaibi, et al. (2024). \textit{Few-Shot Learning for Medical Image Segmentation Using 3D U-Net and Model-Agnostic Meta-Learning (MAML)}. Diagnostics.

\bibitem{ghassemi20241ek}
Sina Ghassemi, Tianyi Zhang, Ward van Breda, et al. (2024). \textit{Unsupervised Multimodal Learning for Dependency-Free Personality Recognition}. IEEE Transactions on Affective Computing.

\bibitem{li20254kd}
Jiawen Li, and Tao Zhou (2025). \textit{Efficient Replay Deep Meta-Reinforcement Learning for Active Fault-Tolerant Control of Solid Oxide Fuel Cell Systems Considering Multivariable Coordination}. IEEE Transactions on Transportation Electrification.

\bibitem{zhao202420b}
Jianjin Zhao, Qi Li, Yueping Hong, et al. (2024). \textit{MetaRockETC: Adaptive Encrypted Traffic Classification in Complex Network Environments via Time Series Analysis and Meta-Learning}. IEEE Transactions on Network and Service Management.

\bibitem{zhang2024xg0}
Lei Zhang, Yuhang Zhou, Yi Yang, et al. (2024). \textit{Meta Invariance Defense Towards Generalizable Robustness to Unknown Adversarial Attacks}. IEEE Transactions on Pattern Analysis and Machine Intelligence.

\bibitem{ruwurm2024806}
Marc Rußwurm, Sherrie Wang, B. Kellenberger, et al. (2024). \textit{Meta-learning to address diverse Earth observation problems across resolutions}. Communications Earth &amp; Environment.

\bibitem{jang2024pyi}
Jungik Jang, Jisung Pyo, Young-il Yoon, et al. (2024). \textit{Meta-Transformer: A Meta-Learning Framework for Scalable Automatic Modulation Classification}. IEEE Access.

\bibitem{su2024h1g}
Hao Su, Qingtao Yao, Ling Xiang, et al. (2024). \textit{Semi-Supervised Temporal Meta-Learning Framework for Wind Turbine Bearing Fault Diagnosis Under Limited Annotation Data}. IEEE Transactions on Instrumentation and Measurement.

\bibitem{li20246zp}
Yao Li, Dawei Yuan, Tao Zhang, et al. (2024). \textit{Meta-Learning for Multi-Family Android Malware Classification}. ACM Transactions on Software Engineering and Methodology.

\bibitem{lupu20249p4}
E. Lupu, Fengze Xie, James A. Preiss, et al. (2024). \textit{MAGICVFM-Meta-Learning Adaptation for Ground Interaction Control With Visual Foundation Models}. IEEE Transactions on robotics.

\bibitem{briscik2024cpd}
Mitja Briscik, Gabriele Tazza, M. Dillies, et al. (2024). \textit{Supervised multiple kernel learning approaches for multi-omics data integration}. BioData Mining.

\bibitem{sun2024kbv}
Jingmin Sun, Zecheng Zhang, and Hayden Schaeffer (2024). \textit{LeMON: Learning to Learn Multi-Operator Networks}. arXiv.org.

\bibitem{eghbali2024huh}
Y. Eghbali, Shiva Kazemi Taskou, Mohammad Robat Mili, et al. (2024). \textit{Providing URLLC Service in Multi-STAR-RIS Assisted and Full-Duplex Cellular Wireless Systems: A Meta-Learning Approach}. IEEE Communications Letters.

\bibitem{zhu2024ok7}
Yihuan Zhu, Yunan Liu, Chunpeng Wang, et al. (2024). \textit{Intermediate Domain-Based Meta Learning Framework for Adaptive Object Detection}. IEEE transactions on circuits and systems for video technology (Print).

\bibitem{long202400t}
Jianyu Long, Rongxin Zhang, Yibin Chen, et al. (2024). \textit{A Customized Meta-Learning Framework for Diagnosing New Faults From Unseen Working Conditions With Few Labeled Data}. IEEE/ASME transactions on mechatronics.

\bibitem{gu20252u3}
Yan Gu, Zhaoze Liu, Shuhong Dai, et al. (2025). \textit{Deep Reinforcement Learning for Job Scheduling and Resource Management in Cloud Computing: An Algorithm-Level Review}. arXiv.org.

\bibitem{zhang2024mf0}
Lin Zhang, Bo Zhang, Botian Shi, et al. (2024). \textit{Few-Shot Cross-Domain Object Detection With Instance-Level Prototype-Based Meta-Learning}. IEEE transactions on circuits and systems for video technology (Print).

\bibitem{liu2024jz5}
Sheng Liu, Linlin You, Rui Zhu, et al. (2024). \textit{AFM3D: An Asynchronous Federated Meta-Learning Framework for Driver Distraction Detection}. IEEE transactions on intelligent transportation systems (Print).

\bibitem{ozkara2024nst}
Kaan Ozkara, Can Karakus, Parameswaran Raman, et al. (2024). \textit{MADA: Meta-Adaptive Optimizers through hyper-gradient Descent}. International Conference on Machine Learning.

\bibitem{cui2024bov}
Qi Cui, Weixuan Tang, Zhili Zhou, et al. (2024). \textit{Meta Security Metric Learning for Secure Deep Image Hiding}. IEEE Transactions on Dependable and Secure Computing.

\bibitem{wang2024so2}
Yuan Wang, Zhiqiang Tao, and Yi Fang (2024). \textit{A Unified Meta-Learning Framework for Fair Ranking With Curriculum Learning}. IEEE Transactions on Knowledge and Data Engineering.

\bibitem{ma2024vk4}
Han Ma, Baoyu Fan, B. Ng, et al. (2024). \textit{VL-Meta: Vision-Language Models for Multimodal Meta-Learning}. Mathematics.

\bibitem{amorim20240xf}
Lucas B V de Amorim, George D. C. Cavalcanti, and Rafael M. O. Cruz (2024). \textit{Meta-Scaler: A Meta-Learning Framework for the Selection of Scaling Techniques}. IEEE Transactions on Neural Networks and Learning Systems.

\bibitem{shen2024hea}
Maohao Shen, J. J. Ryu, Soumya Ghosh, et al. (2024). \textit{Are Uncertainty Quantification Capabilities of Evidential Deep Learning a Mirage?}. Neural Information Processing Systems.

\bibitem{ferrini20249g0}
Francesco Ferrini, Antonio Longa, Andrea Passerini, et al. (2024). \textit{A Self-Explainable Heterogeneous GNN for Relational Deep Learning}. Trans. Mach. Learn. Res..

\bibitem{wang2024tpb}
Chenxing Wang (2024). \textit{Towards Effective Fusion and Forecasting of Multimodal Spatio-temporal Data for Smart Mobility}. International Conference on Information and Knowledge Management.

\bibitem{song2024epb}
Wen-qi Song, Di Wu, Weiming Shen, et al. (2024). \textit{Early fault detection for rolling bearings: A meta‐learning approach}. IET Collaborative Intelligent Manufacturing.

\bibitem{wang20245h1}
Zih-Wei Wang, and Vincent W. S. Wong (2024). \textit{Bayesian Meta-Learning for Adaptive Traffic Prediction in Wireless Networks}. IEEE Transactions on Mobile Computing.

\bibitem{tam2024a1h}
S. Tam, S. T. P. Raghu, Étienne Buteau, et al. (2024). \textit{Towards Robust and Interpretable EMG-based Hand Gesture Recognition using Deep Metric Meta Learning}. arXiv.org.

\bibitem{qiao2024l71}
Ruihua Qiao, Tao Jiang, and Weiyong Yu (2024). \textit{Meta-Learning-Based Fronthaul Compression for Cloud Radio Access Networks}. IEEE Transactions on Wireless Communications.

\bibitem{xing2024n9q}
Yun Xing, Qing Guo, Xiaofeng Cao, et al. (2024). \textit{MetaRepair: Learning to Repair Deep Neural Networks from Repairing Experiences}. ACM Multimedia.

\bibitem{cheng2024mky}
Yingsong Cheng, Xinya Wang, Yong Ma, et al. (2024). \textit{General Hyperspectral Image Super-Resolution via Meta-Transfer Learning}. IEEE Transactions on Neural Networks and Learning Systems.

\bibitem{xia20246dc}
Pengcheng Xia, Yixiang Huang, Chengliang Liu, et al. (2024). \textit{Learn to Supervise: Deep Reinforcement Learning-Based Prototype Refinement for Few-Shot Motor Fault Diagnosis}. IEEE Transactions on Neural Networks and Learning Systems.

\bibitem{yang20243gt}
Fan Yang, Nor Azman Ismail, Y. Y. Pang, et al. (2024). \textit{A Systematic Literature Review of Deep Learning Approaches for Sketch-Based Image Retrieval: Datasets, Metrics, and Future Directions}. IEEE Access.

\bibitem{zhang2024ycr}
Lianwei Zhang, Dongjiang Niu, Beiyi Zhang, et al. (2024). \textit{Property-Guided Few-Shot Learning for Molecular Property Prediction With Dual-View Encoder and Relation Graph Learning Network}. IEEE journal of biomedical and health informatics.

\bibitem{raymond202441h}
Christian Raymond (2024). \textit{Meta-Learning Loss Functions for Deep Neural Networks}. arXiv.org.

\bibitem{khalid2024pss}
Mudassar Khalid, C. Pluempitiwiriyawej, S. Wangsiripitak, et al. (2024). \textit{The Applications of Deep Learning in ECG Classification for Disease Diagnosis: A Systematic Review and Meta-Data Analysis}. Engineering Journal.

\bibitem{kumar2024he9}
Sandeep Kumar, Amit Sharma, Vikrant Shokeen, et al. (2024). \textit{Meta-learning for real-world class incremental learning: a transformer-based approach}. Scientific Reports.

\bibitem{kukanov20249bs}
Ivan Kukanov, Janne Laakkonen, Tomi Kinnunen, et al. (2024). \textit{Meta-Learning Approaches For Improving Detection of Unseen Speech Deepfakes}. Spoken Language Technology Workshop.

\bibitem{chen20245h8}
Cheng Chen, Hao Fang, Yuxiao Yang, et al. (2024). \textit{Model-agnostic meta-learning for EEG-based inter-subject emotion recognition}. Journal of Neural Engineering.

\bibitem{liu2024az5}
Yonghao Liu, Mengyu Li, Ximing Li, et al. (2024). \textit{Meta-GPS++: Enhancing Graph Meta-Learning with Contrastive Learning and Self-Training}. ACM Transactions on Knowledge Discovery from Data.

\bibitem{xu2024mf9}
Zexing Xu, Linjun Zhang, Sitan Yang, et al. (2024). \textit{F-FOMAML: GNN-Enhanced Meta-Learning for Peak Period Demand Forecasting with Proxy Data}. arXiv.org.

\bibitem{chen2024b4d}
Weiye Chen, Yiqun Xie, Xiaowei Jia, et al. (2024). \textit{Referee-Meta-Learning for Fast Adaptation of Locational Fairness}. AAAI Conference on Artificial Intelligence.

\bibitem{wen2024xmk}
Gang Wen, and Limin Li (2024). \textit{MMOSurv: meta-learning for few-shot survival analysis with multi-omics data}. Bioinformatics.

\bibitem{liao2024jm9}
Yang Liao, Jing Zhao, Jiyong Bian, et al. (2024). \textit{From mechanism to application: Decrypting light‐regulated denitrifying microbiome through geometric deep learning}. iMeta.

\bibitem{meng2024nqq}
Wei Meng, Hao Ju, Tongxu Ai, et al. (2024). \textit{Transferring Meta-Policy From Simulation to Reality via Progressive Neural Network}. IEEE Robotics and Automation Letters.

\bibitem{wu2024v0z}
Haotian Wu, Maojun Zhang, Yulin Shao, et al. (2024). \textit{MIMO Channel as a Neural Function: Implicit Neural Representations for Extreme CSI Compression in Massive MIMO Systems}. arXiv.org.

\bibitem{farrell2024mpy}
M. Farrell, Nicolas Le Guillarme, Liam Brierley, et al. (2024). \textit{The changing landscape of text mining: a review of approaches for ecology and evolution}. Proceedings B.

\bibitem{ma20243e9}
Xin Ma, Yuanchang Xie, and Chunxiao Chigan (2024). \textit{Graph Convolutional Network Based Multi-Objective Meta-Deep Q-Learning for Eco-Routing}. IEEE transactions on intelligent transportation systems (Print).

\bibitem{gven2024a3n}
Eray Güven, and G. Kurt (2024). \textit{Learning-to-Learn the Wave Angle Estimation}. IEEE Transactions on Communications.

\bibitem{wang2024bhk}
Jingyao Wang, Wenwen Qiang, Jiangmeng Li, et al. (2024). \textit{Rethinking Meta-Learning from a Learning Lens}. arXiv.org.

\bibitem{pu2024m1b}
Q. Pu, Youkun Chen, Mu Zhou, et al. (2024). \textit{Bayesian Meta-Learning: Toward Fast Adaptation in Neural Network Positioning Techniques}. IEEE Internet of Things Journal.

\end{thebibliography}

\end{document}