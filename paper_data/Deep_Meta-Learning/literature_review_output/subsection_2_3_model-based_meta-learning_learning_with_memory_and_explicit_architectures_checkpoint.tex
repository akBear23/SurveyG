\subsection{Model-Based Meta-Learning: Learning with Memory and Explicit Architectures}

Model-based meta-learning represents a distinct paradigm that focuses on designing specific network architectures intrinsically capable of rapid adaptation or processing new task information, often without explicit gradient updates in an inner loop. This approach contrasts with optimization-based methods by embedding the meta-learning capability directly into the model's structure, leveraging mechanisms like external memory modules or integrated attention and temporal convolutions for efficient in-context learning.

One of the pioneering works in this area is the Memory-Augmented Neural Network (MANN) \cite{santoro2016323}. MANN introduced the concept of augmenting a neural network with an external memory module, inspired by Neural Turing Machines, to store and retrieve task-specific information. This architecture learns to read from and write to its memory based on the current task context, enabling it to rapidly adapt to new tasks and perform one-shot learning by directly recalling relevant experiences rather than through parameter updates. While groundbreaking in demonstrating the power of external memory for meta-learning, the complexity of training and the scalability of these explicit memory access mechanisms can be a significant challenge.

Addressing some of the complexities associated with explicit external memory, \cite{Mishra2018} proposed A Simple Neural Attentive Meta-Learner (SNAIL). SNAIL leverages a combination of temporal convolutions and attention mechanisms to process sequences of experience, allowing for efficient in-context learning. By integrating these standard deep learning components, SNAIL's architecture can directly process a support set and produce predictions for query examples, effectively learning to adapt by attending to relevant past information and aggregating temporal context without requiring explicit gradient-based fine-tuning. This approach offers a more streamlined and often more practical alternative for achieving rapid adaptation compared to the more intricate control mechanisms of memory-augmented networks.

Another significant architectural innovation within model-based meta-learning is the Conditional Neural Process (CNP) \cite{Garnelo2018}, which introduces a probabilistic framework for meta-learning. CNPs are designed to learn a distribution over functions, enabling them to not only make predictions but also quantify uncertainty, a crucial aspect for robust decision-making. The architecture consists of an encoder that processes a set of context points (support set) to produce a latent representation, and a decoder that uses this representation to predict the distribution of target points (query set). This allows the model to directly infer properties of a function from a few examples, effectively "processing new task information" through its learned function approximation capabilities. While differing in its probabilistic objective, CNP exemplifies a model-based approach by using a specialized architecture to directly learn and represent functions from data, providing a distinct path to meta-learning that offers valuable uncertainty estimates not typically available in MANN or SNAIL.

In summary, model-based meta-learning has evolved through architectural innovations that empower models to adapt rapidly without relying on inner-loop gradient optimization. From the pioneering use of external memory in MANN \cite{santoro2016323} to the integration of temporal convolutions and attention in SNAIL \cite{Mishra2018} for efficient in-context learning, and the probabilistic function approximation of CNPs \cite{Garnelo2018}, this paradigm explores how intrinsic architectural designs can confer meta-learning capabilities. A common challenge across these approaches lies in balancing the architectural complexity with scalability to very complex tasks and large datasets, and ensuring that the learned architectures are sufficiently generalizable to truly novel tasks. Future directions may involve hybrid models that combine the strengths of these architectural innovations with elements from other meta-learning paradigms to achieve even more robust and versatile meta-learners.