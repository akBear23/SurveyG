\subsection{Domain-Specific Adaptation and Generalization}

Meta-learning offers a compelling paradigm for addressing the inherent challenges of data scarcity, spatiotemporal heterogeneity, and the critical need for rapid, cost-effective deployment in diverse, unseen operational environments. By learning to learn, meta-learning enables models to quickly adapt and generalize to new tasks or domains with minimal new data, showcasing its practical efficacy across various scientific and engineering fields.

A prominent application demonstrating meta-learning's utility in rapid adaptation is wireless localization. Traditional fingerprinting-based methods struggle with environment-specificity, demanding extensive data collection and retraining for each new physical setting. To overcome this, \cite{gao20223fn} and \cite{gao2022y3s} introduce MetaLoc, a pioneering framework that leverages Model-Agnostic Meta-Learning (MAML) to learn optimal "meta-parameters" – essentially a robust model initialization – from historical tasks. This allows a deep neural network to quickly adapt to new environments with minimal new data and computationally inexpensive updates, significantly enhancing scalability and cost-effectiveness. Building upon this, \cite{pu2024m1b} further refines neural network positioning by proposing a Bayesian meta-learning approach. This method enhances robustness by inferring the Bayesian posterior, effectively mitigating model uncertainty and preventing overfitting when adapting to new environments with very limited samples, thus improving the reliability of rapid adaptation in dynamic wireless settings.

Beyond static environments, meta-learning proves invaluable in tackling complex spatiotemporal heterogeneity. In climate science, accurately estimating global carbon fluxes (e.g., Gross Primary Production) is hampered by sparse and unbalanced in-situ observations, particularly in crucial regions like the tropics. \cite{nathaniel2023ycu} introduces MetaFlux, which employs an MAML-adapted meta-learning ensemble to upscale these sparse spatiotemporal observations. This approach provides robust estimates even in data-poor regions and demonstrates enhanced robustness in predicting extreme flux events, significantly outperforming non-meta-learning baselines. Generalizing this concept, \cite{dong2024110} proposes HimNet, a Heterogeneity-Informed Meta-Parameter Learning scheme for spatiotemporal time series forecasting. HimNet implicitly characterizes spatiotemporal heterogeneity through learnable embeddings and dynamically generates context-specific parameters from compact meta-parameter pools, addressing the limitations of prior methods that rely on auxiliary features or suffer from high computational costs. This represents a significant advancement in leveraging heterogeneity to inform model adaptation. Similarly, \cite{pan2019pue} addresses urban traffic prediction, another domain characterized by complex spatio-temporal correlations, using a deep meta-learning model (ST-MetaNet) that collectively predicts traffic by capturing diverse spatial and temporal patterns.

Meta-learning also provides critical solutions for few-shot learning scenarios where data is inherently scarce. For instance, few-shot short utterance speaker verification, crucial for applications like online payments, faces challenges due to the limited availability of voice samples. \cite{wang2023x5w} addresses this by employing a meta-learning approach, specifically Prototypical Networks enhanced with an ECAPA-TDNN feature extractor and an episodic training strategy that incorporates global classification. This enables the model to learn more discriminative speaker features and achieve identification with minimal voice samples, outperforming traditional methods. The utility extends to other specialized domains: \cite{wang2023srr} introduces Meta-Transfer Learning with Freezing Operation (MTLFO) for few-shot bearing fault diagnosis, which learns new knowledge rapidly from small samples while avoiding overfitting. In remote sensing, \cite{alajaji2020b6c} applies MAML for few-shot scene classification, demonstrating its ability to classify new, unseen classes from limited labeled samples. Furthermore, \cite{cheng2024mky} proposes a meta-transfer learning framework for general hyperspectral image super-resolution, tackling data scarcity and significant domain differences by accumulating diverse task experiences and gradually expanding the number of bands. Even in video processing, \cite{gupta2021fbg} presents Ada-VSR, an adaptive video super-resolution method that uses meta-transfer learning to quickly adapt to novel degradation conditions with only a few gradient updates, significantly reducing inference time.

In conclusion, the literature clearly demonstrates meta-learning's profound impact on domain-specific adaptation and generalization. By enabling rapid learning from limited data and effectively handling complex heterogeneity, meta-learning addresses critical real-world challenges across wireless communication, climate science, security, manufacturing, and remote sensing. However, ongoing research continues to explore ways to balance the computational overhead of meta-learning with scalability, develop more universally robust meta-objectives, and reduce the reliance on diverse meta-training data to fully unlock its potential for truly adaptable and cost-effective AI systems.