\subsection{Learning Similarity Measures: Matching, Prototypical, and Relation Networks}

A distinct family of meta-learning approaches addresses few-shot classification by focusing on learning robust similarity measures within a discriminative embedding space. These methods aim to classify novel instances by comparing them directly to a small set of labeled support examples, thereby enabling accurate generalization from minimal data.

Pioneering this direction, Matching Networks \cite{Vinyals2016} introduced an end-to-end differentiable framework that learns to map a small labeled support set and an unlabeled query to a classification. This is achieved through an attention-based comparison function, where the network dynamically weighs the contribution of each support example to classify the new instance, effectively performing a non-parametric classification in a learned feature space. While innovative, the direct comparison to every support example can be computationally intensive as the support set size grows.

Building upon this foundation, Prototypical Networks \cite{Snell2017} simplified the similarity learning paradigm by proposing that each class in a learned embedding space can be represented by a single "prototype." This prototype is typically computed as the mean of the embedded support examples for that class. Classification of a new query instance then involves assigning it to the class whose prototype is closest in the embedding space, often using Euclidean distance. This approach offers improved computational efficiency and enhanced interpretability compared to Matching Networks, as class representations are explicitly defined, and it demonstrated competitive performance in few-shot classification tasks.

Further generalizing the concept of similarity, Relation Networks \cite{Sung2018} moved beyond fixed distance metrics or attention mechanisms by learning a non-linear "relation function" to explicitly compute similarity scores. This network takes the concatenated embeddings of a query example and a support example (or a class prototype) as input and outputs a scalar score indicating their similarity. By learning this relation function, the model gains greater flexibility in defining what constitutes "similarity," allowing for more complex and adaptive comparisons between embedded instances. This approach often leads to more robust similarity measures, especially when simple distance metrics might be insufficient.

The core ideas of these metric-based approaches continue to be extended and applied to more complex few-shot scenarios. For instance, the principles of Prototypical Networks have been adapted to tackle challenging tasks like few-shot cross-domain object detection. The Instance-level Prototype learning Network (IPNet) \cite{zhang2024mf0} addresses data deficiency in target domains by fusing cropped instances from both source and target domains to learn representative prototypes for each class. These learned prototypes are then utilized to discriminate feature salience and facilitate domain alignment, demonstrating the adaptability of prototype-based methods beyond simple image classification to more intricate problems involving object localization and domain generalization.

In summary, Matching, Prototypical, and Relation Networks collectively represent a powerful family of meta-learning techniques that excel in few-shot classification by learning discriminative feature spaces and robust similarity measures. Their effectiveness stems from their intuitive comparison mechanisms, allowing accurate generalization from minimal examples. However, their primary limitation often lies in their task-specificity, as they are predominantly designed for classification tasks and might struggle to generalize to problems requiring complex structural changes or where a simple distance metric or learned relation function is insufficient for capturing task-specific nuances beyond feature comparison. Future research may explore hybrid approaches that combine the strengths of metric learning with other meta-learning paradigms to enhance their applicability to a broader range of tasks.