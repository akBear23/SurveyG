\subsection{Memory-Augmented Neural Networks for In-Context Learning}

Model-based meta-learning approaches offer a distinct paradigm for achieving rapid, in-context adaptation by designing network architectures that intrinsically facilitate fast information integration and knowledge transfer. Unlike gradient-based methods that learn to optimize parameters or metric-based approaches that learn similarity functions, this category focuses on equipping neural networks with explicit mechanisms for storing and retrieving task-specific information, enabling one-shot learning without explicit gradient updates.

A pioneering work in this domain is the Memory-Augmented Neural Network (MANN) proposed by \cite{santoro2016323}. Inspired by Neural Turing Machines (NTMs), MANN augments a standard neural network with an external memory module, allowing the network to learn to store and retrieve task-relevant information through differentiable read and write operations. This architectural innovation enables the model to quickly adapt to new tasks by leveraging previously encountered experiences stored in its memory, effectively performing one-shot learning by recalling specific data points or features rather than re-optimizing its weights. The core contribution of MANN lies in demonstrating how an explicit memory component can mitigate catastrophic forgetting and facilitate rapid knowledge acquisition, allowing the network to learn an internal "algorithm" for fast information integration. However, the complexity of training such memory-augmented networks and the scalability of their memory access mechanisms can pose significant challenges, particularly for very large datasets or highly diverse tasks.

Building upon the principles of in-context learning, the Simple Neural Attentive Meta-Learner (SNAIL) \cite{mishra2018simple} offers an alternative architectural approach that integrates temporal convolutions and attention mechanisms. SNAIL processes sequences of experience (support set and query examples) through a combination of causal convolutions, which efficiently aggregate information from past steps in the sequence, and a task-specific attention mechanism, which allows the network to selectively focus on the most relevant information for the current prediction. This design enables SNAIL to learn an internal meta-learning algorithm that can rapidly adapt to new tasks by attending to and integrating information from the context set, all within a single forward pass without requiring explicit gradient updates during adaptation. By leveraging standard deep learning components like convolutions and attention, SNAIL provides a more streamlined and potentially more scalable architecture for in-context learning compared to the explicit memory controllers of MANN, while still achieving powerful meta-learning capabilities across various tasks.

In summary, memory-augmented and attention-based neural networks represent a powerful class of meta-learning models that achieve rapid, in-context adaptation through architectural design rather than explicit parameter optimization. While MANN \cite{santoro2016323} introduced the foundational concept of external memory for one-shot learning, SNAIL \cite{mishra2018simple} refined this idea by integrating temporal convolutions and attention for efficient sequential processing. These models excel at learning an intrinsic 'algorithm' for fast information integration, demonstrating how architectural innovation can intrinsically equip neural networks with powerful meta-learning capabilities. However, challenges remain in scaling these architectures to extremely complex tasks and ensuring efficient memory management or attention mechanisms without incurring prohibitive computational costs or architectural complexity. Future research may explore hybrid approaches that combine the strengths of these model-based methods with more efficient memory structures or integrate them with other meta-learning paradigms.