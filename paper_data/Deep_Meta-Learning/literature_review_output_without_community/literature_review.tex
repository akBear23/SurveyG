\documentclass[12pt,a4paper]{article}
    \usepackage[utf8]{inputenc}
    \usepackage[T1]{fontenc}
    \usepackage{amsmath,amsfonts,amssymb}
    \usepackage{graphicx}
    \usepackage[margin=2.5cm]{geometry}
    \usepackage{setspace}
    \usepackage{natbib}
    \usepackage{url}
    \usepackage{hyperref}
    \usepackage{booktabs}
    \usepackage{longtable}
    \usepackage{array}
    \usepackage{multirow}
    \usepackage{wrapfig}
    \usepackage{float}
    \usepackage{colortbl}
    \usepackage{pdflscape}
    \usepackage{tabu}
    \usepackage{threeparttable}
    \usepackage{threeparttablex}
    \usepackage[normalem]{ulem}
    \usepackage{makecell}
    \usepackage{xcolor}

    % Set line spacing
    \doublespacing

    % Configure hyperref
    \hypersetup{
        colorlinks=true,
        linkcolor=blue,
        filecolor=magenta,      
        urlcolor=cyan,
        citecolor=red,
    }

    % Title and author information
    \title{A Comprehensive Literature Review with Self-Reflection}
    \author{Literature Review}
    \date{\today}

    \begin{document}

    \maketitle

    % Abstract (optional)
    \begin{abstract}
    This literature review provides a comprehensive analysis of recent research in the field. The review synthesizes findings from 285 research papers, identifying key themes, methodological approaches, and future research directions.
    \end{abstract}

    \newpage
    \tableofcontents
    \newpage

    \label{sec:introduction_to_deep_meta-learning}

\section{Introduction to Deep Meta-Learning}
\label{sec:introduction\_to\_deep\_meta-learning}

\subsection{The Challenge of Generalization and Data Efficiency in Deep Learning}
\label{sec:1\_1\_the\_challenge\_of\_generalization\_\_and\_\_data\_efficiency\_in\_deep\_learning}

Despite their remarkable successes in specific domains, standard deep learning models fundamentally struggle with two critical limitations: their insatiable demand for extensive labeled datasets and their inherent difficulty in generalizing effectively to out-of-distribution (OOD) scenarios. This often leads to 'brittleness' and a lack of flexibility, starkly contrasting with human-like learning capabilities that can adapt quickly and robustly from limited experience \cite{huisman2020b7w}. The inherent inefficiency in data utilization and the difficulty in transferring learned knowledge to novel contexts underscore the critical need for more advanced learning paradigms.

The problem of \textbf{data inefficiency} is pervasive across deep learning applications. Training state-of-the-art deep neural networks (DNNs) typically necessitates vast quantities of meticulously labeled data, a process that is both costly and time-consuming. This reliance on large datasets limits the applicability of deep learning in domains where data collection is inherently expensive, scarce, or privacy-sensitive. For instance, in deep reinforcement learning (RL), agents often require millions of interactions with an environment to learn a single task, a phenomenon termed the "sample complexity crisis" \cite{beck2023x24}. This stands in stark contrast to humans, who can often grasp new concepts or master new skills from just a few examples or limited trial-and-error. The computational resources required for training these data-hungry models further exacerbate this inefficiency, making rapid deployment and adaptation challenging \cite{huisman2020b7w}.

Beyond data volume, the \textbf{generalization capabilities} of standard deep learning models are often surprisingly fragile. While they excel at interpolating within their training distribution, their performance degrades significantly when faced with even slight shifts in the data distribution, a phenomenon known as out-of-distribution (OOD) generalization failure. This 'brittleness' manifests in several critical ways:
\begin{enumerate}
    \item \textbf{Sensitivity to Domain Shift:} Models trained on one specific data distribution often fail catastrophically when applied to a slightly different, yet semantically similar, target domain. For example, deep face recognition models, despite high accuracy on balanced datasets, exhibit significant performance drops when confronted with diverse data biases related to ethnicity, head pose, occlusion, or blur, highlighting their struggle to generalize across multiple variation factors \cite{liu2022tgc}. This indicates that models learn spurious correlations or "shortcut learning" rather than robust, invariant features.
    \item \textbf{Architectural Limitations and Aliasing:} Even with data augmentation, deep convolutional networks can suffer from structural limitations that prevent robust generalization, particularly under natural corruptions or OOD conditions. Research suggests that issues like aliasing, where high-frequency information is misrepresented as lower frequencies, can fundamentally hinder a network's ability to generalize, requiring architectural modifications like non-trainable low-pass filters to mitigate these inherent weaknesses \cite{vasconcelos2021fn3}.
    \item \textbf{Vulnerability to Adversarial Attacks:} A profound demonstration of brittleness is the susceptibility of DNNs to adversarial attacks. Minor, imperceptible perturbations to input data can cause models to misclassify with high confidence, revealing a lack of true understanding and robustness. Current defenses primarily focus on known attack types, leaving models highly vulnerable to \textit{unknown} adversarial attacks, underscoring a critical gap in generalizable robustness \cite{zhang2024xg0}.
    \item \textbf{Unreliable Uncertainty Quantification:} Even when models provide predictions, their associated uncertainty estimates can be unreliable, particularly in OOD scenarios. Methods like evidential deep learning, designed to quantify predictive uncertainty, have been shown to produce non-vanishing epistemic uncertainties even with infinite data, suggesting that their uncertainty quantification capabilities may be a "mirage" \cite{shen2024hea}. This lack of trustworthy uncertainty directly impacts the reliability and safety of AI systems in real-world deployments.
    \item \textbf{Catastrophic Forgetting:} In sequential learning tasks, standard deep learning models tend to catastrophically forget previously acquired knowledge when trained on new tasks. This inability to continually accumulate and retain knowledge makes lifelong learning a significant challenge, further limiting their flexibility and adaptability in dynamic environments.
\end{enumerate}

In essence, while traditional deep learning has achieved remarkable feats in pattern recognition within its training distribution, its fundamental limitations in data efficiency and robust OOD generalization remain significant hurdles. The 'brittleness' and lack of flexibility observed in current AI systems, contrasted with the rapid, generalizable learning capabilities of humans, highlight a critical need for more advanced learning paradigms. These challenges have motivated the emergence of "learning to learn" approaches, where models are explicitly designed to acquire the ability to adapt rapidly and efficiently to novel tasks and environments, thereby setting the stage for the field of Deep Meta-Learning.
\subsection{Introducing Deep Meta-Learning: Learning to Learn}
\label{sec:1\_2\_introducing\_deep\_meta-learning:\_learning\_to\_learn}

Deep Meta-Learning marks a pivotal shift in artificial intelligence, moving beyond systems that merely perform specific tasks to those capable of acquiring the fundamental learning process itself. This paradigm, often termed "learning to learn," empowers AI models to adapt swiftly to novel tasks with minimal data, directly addressing the pervasive generalization and data efficiency problems inherent in traditional deep learning \cite{hospedales2020m37, huisman2020b7w}. By enabling models to develop transferable knowledge about \textit{how} to learn, Deep Meta-Learning sets the stage for more autonomous and versatile AI.

The core of Deep Meta-Learning revolves around training models on a distribution of tasks to extract meta-knowledge that facilitates rapid adaptation to new, unseen tasks. This meta-knowledge can manifest in several key forms, broadly categorized as learning to initialize parameters effectively, learning to optimize efficiently, or learning to compare examples for rapid classification \cite{huisman2020b7w}. Each approach offers a distinct mechanism for acquiring an inductive bias that accelerates learning on new tasks.

One prominent category is \textbf{optimization-based meta-learning}, which focuses on learning an effective initialization or an efficient optimization procedure. A seminal contribution in this area is Model-Agnostic Meta-Learning (MAML) by \textcite{finn2017model}. MAML proposes to learn a set of initial parameters such that a few gradient steps on a new task will lead to rapid and substantial performance improvement. This approach is "model-agnostic" because it can be applied to any model trained with gradient descent, making it a powerful framework for learning adaptable initializations across diverse architectures. Complementing this, earlier work explored the concept of learning to optimize directly. For instance, \textcite{wang20167px} demonstrated that a recurrent neural network (RNN), specifically an LSTM, could implicitly learn a complete reinforcement learning algorithm within its recurrent dynamics. By processing past actions and rewards, the RNN effectively learned its own exploration-exploitation strategies and policy update rules, showcasing a powerful form of "learning how to optimize efficiently" to overcome the data demands and task specialization of conventional deep reinforcement learning. This historical trajectory of meta-gradient methods, where higher-level learning processes find good choices for meta-parameters, has been extensively reviewed by \textcite{sutton2022jss}, highlighting the long-standing interest in learning to optimize. Further advancements, such as Meta-Transfer Learning (MTL) \cite{sun2018iy7}, also fall under this umbrella by learning scaling and shifting functions for deep neural network weights, effectively teaching the model how to initialize parameters for rapid adaptation in few-shot scenarios.

Another fundamental paradigm is \textbf{metric-based meta-learning}, which centers on learning a transferable similarity metric or comparison function. Instead of adapting model parameters directly, these methods learn to embed data into a feature space where distances directly correspond to semantic similarity, allowing for classification or clustering of novel classes with few examples. Prototypical Networks, introduced by \textcite{snell2017prototypical}, exemplify this by learning a metric space where each class is represented by a prototype (the mean of its support examples), and classification is performed by finding the nearest prototype. Building on this, \textcite{sung2017nc5} introduced Relation Networks, which learn a deep, non-linear "relation function" to explicitly compare query examples with support examples. This allows the model to infer relationships and similarities between data points, enabling effective few-shot classification by learning a more flexible and powerful comparison mechanism than fixed distance metrics.

A third category, often referred to as \textbf{model-based} or \textbf{black-box meta-learning}, involves using a meta-learner (often a recurrent neural network) that implicitly learns an adaptation algorithm. These models are trained to process task-specific data sequentially, updating their internal state to adapt to new tasks without relying on explicit gradient-based meta-optimization. The aforementioned work by \textcite{wang20167px} on RNNs learning RL algorithms can also be seen as an early example of this, where the RNN acts as a black box that learns to adapt its policy based on sequential experience. This approach offers significant flexibility and can learn complex adaptation strategies, albeit often with reduced interpretability regarding the learned adaptation process.

In summary, Deep Meta-Learning provides a sophisticated framework to overcome the limitations of traditional deep learning, particularly concerning data efficiency and generalization. By enabling AI systems to learn the underlying learning process—whether through adaptable parameter initialization, efficient optimization strategies, or robust example comparison—it paves the way for more autonomous, versatile, and adaptable AI. These foundational paradigms, established by seminal works, have laid the groundwork for the field's subsequent expansion into increasingly complex domains and applications, which will be explored in detail in the following sections.
\subsection{Scope and Organization of the Review}
\label{sec:1\_3\_scope\_\_and\_\_organization\_of\_the\_review}

This literature review is meticulously structured to provide a comprehensive and pedagogically sound understanding of Deep Meta-Learning, tracing its evolution from foundational principles to advanced applications and future challenges. The aim is to guide the reader through a logical progression, ensuring a holistic grasp of the field's theoretical underpinnings, methodological diversity, and practical implications. This structured approach is crucial for navigating a rapidly evolving field, as highlighted by recent surveys that emphasize the need for clear taxonomies and problem definitions to foster further advancements \cite{hospedales2020m37, son2023lda}.

The review commences with \textbf{Section 1: Introduction to Deep Meta-Learning}, which establishes the foundational context. It begins by outlining the inherent limitations of traditional deep learning, particularly concerning data efficiency and generalization to novel tasks. This sets the stage for introducing Deep Meta-Learning as a powerful paradigm designed to overcome these challenges by enabling models to "learn to learn" \cite{hospedales2020m37}. This initial section defines the core principles and potential of the field, providing a necessary backdrop for the subsequent detailed discussions.

Following this, \textbf{Section 2: Foundational Paradigms and Early Breakthroughs} delves into the core conceptual frameworks that initially established Deep Meta-Learning. This section systematically introduces the three primary methodological categories: optimization-based, metric-based, and black-box approaches. It explores how these early paradigms offered distinct solutions to few-shot learning and rapid adaptation, laying the essential groundwork for subsequent advancements. For instance, optimization-based methods, often rooted in meta-gradient principles \cite{sutton2022jss}, learn adaptable initializations, while metric-based techniques focus on learning transferable comparison functions, and black-box methods implicitly learn adaptation algorithms.

Building upon these foundations, \textbf{Section 3: Advancements in Optimization-Based Meta-Learning} is dedicated to the significant evolution of this dominant paradigm. This section explores how researchers have addressed initial computational challenges, deepened theoretical understanding, and developed more sophisticated adaptation mechanisms. It covers scalable variants of seminal algorithms, theoretical insights into their dynamics, the role of hypernetworks for learned adaptation, and recent efforts to rethink meta-learning through calibration and task relations. This dedicated focus reflects the extensive and impactful advancements within this methodological stream, which has seen the most widespread development and application.

The review then transitions to specialized domains and critical capabilities. \textbf{Section 4: Meta-Learning for Continual and Biologically Inspired Adaptation} explores how meta-learning enables systems to adapt and evolve over extended periods. This includes robust continual learning, addressing the challenge of catastrophic forgetting through frameworks that facilitate persistent knowledge accumulation and adaptation to evolving environments \cite{son2023lda}. It also delves into biologically plausible mechanisms that offer alternative paradigms for robust, energy-efficient adaptation, drawing inspiration from natural learning processes. This section highlights meta-learning's push towards resilient and lifelong learning capabilities.

\textbf{Section 5: Meta-Reinforcement Learning: Learning to Act, Explore, and Adapt Continually} focuses on the application of meta-learning in reinforcement learning (RL). This critical area covers the evolution from sample-efficient off-policy and offline meta-RL algorithms to principled exploration strategies, culminating in the integration of safety guarantees for real-world deployment. The section demonstrates meta-learning's transformative role in creating autonomous, adaptable, and robust agents capable of navigating complex and dynamic environments responsibly.

Recognizing the imperative for reliable AI, \textbf{Section 6: Robustness, Generalization, and Trustworthiness} addresses the critical challenges of deploying meta-learning systems in unpredictable environments. This section emphasizes ensuring meta-learned models are robust to out-of-distribution data, capable of generalizing reliably to novel contexts (Domain Generalization), and provide interpretable and trustworthy decisions. It covers methods for calibrating meta-learning for robust generalization and discusses the emerging importance of interpretability for safety-critical human-machine interfaces.

The review then ventures into the cutting-edge frontiers with \textbf{Section 7: Meta-Learning in the Era of Foundation Models and Embodied AI}. This section highlights the powerful and synergistic integration of meta-learning with large pre-trained foundation models, such as Visual Foundation Models (VFMs) for robotic control and Vision-Language Models (VLMs) for prompt tuning \cite{lee2021jou}. It also discusses the unique benchmarking methodologies and practical challenges encountered when applying meta-learning to models of unprecedented scale and autonomy, emphasizing the need for robust evaluation protocols to guide future development.

Finally, \textbf{Section 8: Conclusion and Future Directions} synthesizes the key advancements and evolutionary trajectory of Deep Meta-Learning. It recapitulates the field's journey, identifies remaining theoretical gaps and significant practical challenges, and discusses promising future research avenues. This includes novel integrations, such as with causal inference, and crucial ethical considerations inherent in developing increasingly autonomous and intelligent learning systems, charting a course for the next generation of meta-learning research \cite{hospedales2020m37}.

By adopting this structured, progressive narrative, this review aims to provide readers with a comprehensive and insightful journey through the dynamic landscape of Deep Meta-Learning, from its foundational concepts to its potential future trajectory and societal impact. This organization ensures that each section builds logically upon the last, offering a clear and coherent understanding of the field's complexity and its profound implications for the future of artificial intelligence.


\label{sec:foundational_paradigms_and_early_breakthroughs}

\section{Foundational Paradigms and Early Breakthroughs}
\label{sec:foundational\_paradigms\_\_and\_\_early\_breakthroughs}

\subsection{Optimization-Based Meta-Learning: Learning Adaptable Initializations}
\label{sec:2\_1\_optimization-based\_meta-learning:\_learning\_adaptable\_initializations}

The quest for artificial intelligence systems capable of rapid adaptation to novel tasks with limited data has long been a cornerstone challenge in deep learning. This pursuit led to the emergence of "learning to learn" (meta-learning) paradigms, where the learning algorithm itself is optimized \cite{thrun1998learning, schmidhuber1987evolutionary}. Early efforts in this direction included methods for learning optimizers directly \cite{andrychowicz2016learning}, setting the stage for more sophisticated meta-gradient approaches that optimize higher-level parameters \cite{sutton2022jss}. Within this context, a particularly influential paradigm, especially for few-shot learning and generalization, involves optimizing a model's initial parameters such that it can swiftly adapt to new tasks through only a few gradient steps. This approach, often termed optimization-based meta-learning, represents a significant evolution in enabling models to acquire new skills efficiently.

A seminal contribution to this paradigm is the Model-Agnostic Meta-Learning (MAML) algorithm, introduced by \cite{finn2017model}. MAML revolutionized meta-learning by proposing to learn an optimal set of initial parameters ($\theta$) for a deep neural network. The core idea is that these initial parameters are optimized such that only a few gradient steps on a new, unseen task will lead to rapid and effective adaptation. This is achieved through a distinctive nested optimization process, fundamentally structured as a bi-level optimization problem \cite{franceschi2018u1q}. An \textit{inner loop} performs task-specific adaptation: for each sampled task $\mathcal{T}\_i$ from a distribution of tasks $p(\mathcal{T})$, the model's parameters are updated using a few gradient steps on a small support set $D\_i^{sup}$. This yields task-specific adapted parameters $\theta'\_i = \theta - \alpha \nabla\_{\theta} \mathcal{L}\_{\mathcal{T}\_i}(D\_i^{sup}; \theta)$, where $\alpha$ is the inner-loop learning rate. The crucial innovation lies in the \textit{outer loop}, which updates the meta-parameters (the initial parameters $\theta$) based on the performance of the \textit{adapted} model $\theta'\_i$ on a separate query set $D\_i^{query}$ for that same task. The meta-objective is to minimize the expected loss over the query sets across all tasks: $\min\_{\theta} \mathbb{E}\_{\mathcal{T}\_i \sim p(\mathcal{T})} [\mathcal{L}\_{\mathcal{T}\_i}(D\_i^{query}; \theta'\_i)]$. This bi-level optimization ensures that the learned initialization is maximally amenable to fast fine-tuning on new, but related, tasks.

MAML's model-agnostic nature is a key strength and a primary reason for its foundational impact. It makes no assumptions about the underlying model architecture (e.g., convolutional neural networks, recurrent neural networks) or the type of task (e.g., classification, regression, reinforcement learning), as long as the model is differentiable. This versatility allowed MAML to be widely applied across diverse domains, establishing a robust framework for learning to learn that significantly advanced the field of few-shot learning and generalization. The algorithm implicitly seeks an initialization point in the parameter space from which the model can quickly descend to a low-loss region for any new task with minimal gradient updates. This makes the initial parameters a highly sensitive and adaptable starting point, rather than a fixed solution.

Despite its groundbreaking conceptual contributions and broad applicability, the original MAML algorithm inherently introduced practical limitations. Its reliance on computing second-order derivatives (or computationally expensive approximations thereof) for the outer loop optimization posed a significant computational burden and memory consumption. This is because the meta-gradient requires differentiating through the inner-loop gradient updates, which involves computing Hessians or Jacobian-vector products. This characteristic, while central to its design for achieving an optimal initialization, became a bottleneck for scaling MAML to very deep and wide neural networks and larger, more complex real-world problems. The computational expense and memory footprint limited its practical deployment, particularly in resource-constrained environments or for models with billions of parameters.

In conclusion, the optimization-based meta-learning paradigm, spearheaded by MAML, fundamentally reshaped the field by demonstrating that learning an optimal initial parameterization is a powerful strategy for achieving rapid adaptation and few-shot learning in deep neural networks. Its distinctive inner and outer loop optimization, coupled with its model-agnostic design, established a robust framework for learning to learn. While its original formulation presented significant computational challenges due to the necessity of second-order derivatives, these limitations were instrumental in motivating subsequent research into more scalable and efficient variants, such as first-order approximations like FOMAML and Reptile, which will be detailed in Section 3.1. These advancements, building upon MAML's foundational principles, paved the way for the broader adoption and refinement of optimization-based meta-learning.
\subsection{Metric-Based Meta-Learning: Learning to Compare}
\label{sec:2\_2\_metric-based\_meta-learning:\_learning\_to\_compare}

In the pursuit of enabling machine learning models to generalize rapidly from limited data, a prominent paradigm within meta-learning focuses on acquiring a transferable similarity metric or comparison function. This approach, known as metric-based meta-learning, shifts the learning objective from directly adapting model parameters to understanding and applying similarity concepts across diverse tasks. By inferring relationships between data points, this paradigm offers an intuitive and powerful path towards data-efficient learning, particularly in few-shot and zero-shot scenarios.

The foundational idea behind metric-based meta-learning is to learn an embedding space where examples from the same class are close together, and examples from different classes are far apart. Early seminal works established this principle by focusing on learning robust embedding functions, while often relying on pre-defined distance metrics for comparison. For instance, \textbf{Matching Networks} \cite{vinyals2016matching} introduced an attention mechanism to compare a query example with every support example, effectively learning a weighted nearest-neighbor classifier in the learned embedding space. The similarity was typically measured using cosine similarity, and the model was trained end-to-end to optimize this comparison process. Following this, \textbf{Prototypical Networks} \cite{snell2017prototypical} simplified the comparison by proposing that each class could be represented by a "prototype" vector, which is the mean of its support set embeddings. Classification then involves assigning a query example to the class whose prototype is closest in the embedding space, often using Euclidean distance. Both Matching Networks and Prototypical Networks demonstrated remarkable effectiveness in few-shot learning by learning powerful embedding functions, yet they relied on fixed, pre-defined distance metrics (cosine or Euclidean) for the final comparison step.

A significant advancement in this domain came with the introduction of the \textbf{Relation Network (RN)} by \cite{sung2017nc5}, which explicitly addressed the limitation of fixed distance metrics by learning \textit{how to compare} examples. Rather than relying on a pre-defined or simple linear similarity function, the RN meta-learns a deep, non-linear comparison function, thereby enhancing the model's ability to infer complex relationships between data points and generalize to novel classes with minimal examples. The architecture of the Relation Network comprises two key components: an embedding module ($\text{f}\_{\phi}$) and a relation module ($\text{g}\_{\psi}$). The embedding module processes both query and support examples, transforming them into rich feature representations. These feature maps are then concatenated and fed into the relation module, which is typically a convolutional neural network designed to output a scalar 'relation score' indicating the degree of similarity or relatedness between the input pair. This end-to-end training, often utilizing a Mean Squared Error (MSE) loss, allows the network to learn a sophisticated comparison mechanism that is highly adaptable.

The core innovation of \cite{sung2017nc5} lies in its departure from prior metric learning methods by introducing a \textit{learnable, non-linear relation function}. This allows the RN to capture more complex and nuanced relationships between data points than fixed distance functions, leading to superior generalization capabilities in few-shot scenarios. This approach avoids the computational overhead of fine-tuning methods like MAML and the architectural complexity of RNN-based meta-learners, offering a simpler yet highly effective feed-forward solution. Furthermore, the Relation Network demonstrates remarkable versatility, extending its applicability beyond few-shot classification to zero-shot learning. This is achieved by adapting the embedding module to process semantic class descriptions (e.g., attribute vectors) alongside visual features, allowing the relation module to infer similarities between images and abstract class concepts. This unified framework underscores the power of a learned comparison function to bridge different data modalities and learning paradigms.

Despite the strengths of metric-based meta-learning, particularly its intuitive nature and data efficiency, several challenges persist. A fundamental issue in learned embedding spaces is the "hubness problem," where certain embedding vectors (hubs) become nearest neighbors to a disproportionately large number of other vectors, regardless of their true semantic similarity \cite{fei20211x6}. This can degrade the performance and reliability of distance-based classification. While Relation Networks mitigate some of these issues by learning a non-linear comparison, the scalability of comparison functions when dealing with a vast number of classes or complex, multi-modal relationships remains an active area of research. Simple concatenation of feature maps might not always suffice for intricate relational reasoning.

Future research in metric-based meta-learning is exploring more sophisticated relation modules and adaptive comparison functions. For instance, methods are emerging that learn adaptive deep distance metrics through cascaded feature matching, allowing for more flexible and robust similarity computations \cite{chen2021yqh}. To address the complexity of relationships, there is a growing trend towards incorporating graph-based relational reasoning. Researchers are investigating the use of Graph Neural Networks (GNNs) or relation graph learning networks to model intricate interactions within support sets or between support and query examples, enabling more powerful relational inference for few-shot learning \cite{liu2024az5, ferrini20249g0, zhang2024ycr}. Moreover, the integration of meta-mining strategies with deep metric learning offers avenues to enhance the learning process by intelligently selecting informative samples or tasks for meta-training, thereby improving the quality of learned embeddings and comparison functions \cite{jiang20220tg}. Understanding the robustness of these learned metrics to out-of-distribution tasks and their interpretability also remains a critical area of development, ensuring that 'learning to compare' evolves towards even more robust, generalizable, and trustworthy meta-learning systems.
\subsection{Black-Box Meta-Learning: Implicit Adaptation through Recurrent Networks}
\label{sec:2\_3\_black-box\_meta-learning:\_implicit\_adaptation\_through\_recurrent\_networks}

Early explorations into meta-learning leveraged the inherent capacity of recurrent neural networks (RNNs) and other black-box architectures to implicitly learn adaptation algorithms. These methods train a single, often recurrent, model to process task-specific data sequentially, allowing its internal state to evolve and adapt to the nuances of a new task without relying on explicit gradient-based meta-optimization steps during adaptation. This approach offers significant flexibility, enabling the discovery of complex, non-linear adaptation strategies, particularly in domains like reinforcement learning where sequential decision-making is paramount. However, this flexibility often comes at the cost of reduced interpretability regarding the precise mechanisms of the learned adaptation process.

A seminal contribution in this area is the work by \cite{wang20167px}, which introduced deep meta-reinforcement learning (deep meta-RL). This framework demonstrated that a recurrent neural network, specifically an LSTM, could implicitly learn a full-fledged reinforcement learning procedure. The LSTM is trained using a standard deep RL algorithm (e.g., Advantage Actor-Critic) across a distribution of related tasks. Crucially, during an episode, the network receives not only the current observation but also auxiliary inputs indicating the action taken on the previous step and the reward received from that action. Through this sequential processing, the recurrent dynamics of the LSTM itself learn to implement an internal RL algorithm, adapting its policy based on the history of interactions within an episode, even when the network's weights are frozen. This implicit adaptation allows the learned RL algorithm to develop its own exploration strategies and policy update rules, which are tailored to exploit the underlying structure of the training domain for rapid adaptation. Experiments in various bandit tasks and simple Markov Decision Problems showed that this meta-RL approach could outperform established bandit algorithms like UCB and Thompson sampling, validating its ability to learn effective exploration-exploitation strategies.

Beyond learning an entire RL algorithm, other black-box approaches have explored different forms of implicit adaptation. For instance, \cite{vecoven2018hc1} introduced Neuro-Modulated Networks (NMNs), where a "modulator" network, often recurrent, dynamically adjusts the weights or activations of a "main" network. This mechanism allows the system to learn adaptive behaviors by implicitly controlling the computational flow and processing within the main network based on task context. The modulator network, acting as a black box, learns to infer and apply appropriate modulations, enabling flexible responses to unforeseen problems without explicit meta-optimization of the main network's parameters. These neuromodulatory approaches represent another facet of implicit adaptation, where a learned, dynamic control signal facilitates task-specific adjustments.

While offering remarkable flexibility and the ability to discover novel adaptation strategies, black-box meta-learning methods inherently face challenges. The primary limitation is the reduced interpretability of the learned adaptation process; understanding \textit{how} the RNN's internal state translates into a specific adaptation strategy remains opaque. Furthermore, while these models can generalize well to new tasks drawn from the \textit{same distribution} as the training tasks, their ability to adapt to significantly out-of-distribution tasks can be limited, as the learned implicit algorithm is specialized to the statistics of the training environment. The computational intensity of training complex recurrent architectures for meta-learning, especially for very long sequences or high-dimensional state spaces, also poses scalability concerns. These limitations, particularly the lack of interpretability and the desire for more principled adaptation, have motivated the development of more explicit meta-learning paradigms, such as gradient-based meta-optimization, which offer clearer insights into the adaptation process and often provide stronger generalization guarantees. Nevertheless, black-box recurrent approaches remain a foundational and powerful pathway, particularly in domains like meta-reinforcement learning, where the sequential nature of tasks aligns naturally with recurrent processing.


\label{sec:advancements_in_optimization-based_meta-learning}

\section{Advancements in Optimization-Based Meta-Learning}
\label{sec:advancements\_in\_optimization-based\_meta-learning}

\subsection{MAML and its Scalable Variants}
\label{sec:3\_1\_maml\_\_and\_\_its\_scalable\_variants}

The advent of Model-Agnostic Meta-Learning (MAML) marked a significant milestone in gradient-based meta-learning, offering a principled approach for deep neural networks to rapidly adapt to new tasks with minimal data \cite{Finn2017}. MAML's core innovation lies in learning a set of initial parameters such that a few gradient steps on a new task yield substantial performance improvements. This is achieved by meta-optimizing these initial parameters, requiring the computation of second-order derivatives (or their approximations) to ensure the learned initialization is "sensitive" to rapid adaptation \cite{Finn2017}. While theoretically powerful and model-agnostic, this reliance on second-order derivatives introduced significant practical limitations, including high computational cost, substantial memory consumption, and increased implementation complexity, particularly when applied to very deep and wide neural networks. These challenges restricted MAML's applicability to larger models and datasets, hindering its broader adoption in real-world scenarios.

To address these practical bottlenecks, scalable alternatives emerged, primarily focusing on reducing the computational overhead. A prominent example is Reptile, which proposed a significantly simpler, first-order meta-learning algorithm \cite{Nichol2018}. Instead of complex second-order optimization, Reptile iteratively samples a task, performs several gradient steps to obtain task-specific parameters, and then moves the global meta-parameters (the initialization) \textit{towards} these task-specific parameters using only first-order gradients. This approach was demonstrated to be a first-order approximation of MAML's objective, effectively achieving similar meta-learning goals with vastly improved efficiency \cite{Nichol2018}. Reptile's simplicity and computational efficiency significantly lowered the barrier to entry for applying meta-learning to deep networks, making it more practical for large-scale deep learning problems and fostering wider adoption.

Beyond first-order approximations, further advancements sought to enhance both efficiency and generalization by decoupling meta-learning from high-dimensional parameter optimization. One such direction involved learning in a low-dimensional latent space, where meta-parameters are generated or adapted more efficiently than directly optimizing the full high-dimensional model parameters. This approach improves efficiency by reducing the dimensionality of the meta-learning problem and can enhance generalization by learning more abstract, transferable representations. For instance, in complex control tasks, methods like Neural-Fly leverage similar principles by learning shared representations and confining task-specific adaptations to a low-dimensional space \cite{oconnell2022twd}. This allows for rapid online adaptation in dynamic environments, such as agile flight in strong winds, by updating a small set of linear coefficients that mix basis elements derived from pretrained representations \cite{oconnell2022twd}. The ability to adapt quickly and precisely in such challenging conditions underscores the practical benefits of these scalable meta-learning approaches.

The development of these scalable variants has significantly broadened the applicability of gradient-based meta-learning. For example, in the domain of No-Reference Image Quality Assessment (NR-IQA), where annotated data is scarce, meta-learning has been successfully applied to learn meta-knowledge shared across various distortion types \cite{zhu2020rb5}. The MetaIQA framework utilizes deep meta-learning to acquire a quality prior model that can then be quickly fine-tuned on a target NR-IQA task, demonstrating strong generalization from synthetic to authentic distortions \cite{zhu2020rb5}. This exemplifies how efficient meta-learning, by learning transferable prior knowledge, can address data scarcity problems in specialized applications. The evolution from the computationally intensive original MAML to its more scalable and efficient variants, through first-order approximations and latent space learning, has thus been crucial in making gradient-based meta-learning a practical and powerful tool for a diverse range of real-world scenarios, from few-shot learning to complex control and specialized data-scarce domains.

Despite these advancements, challenges remain in ensuring robust generalization to truly novel, out-of-distribution tasks, and in developing meta-learning algorithms that can continually adapt without catastrophic forgetting. Future research directions will likely continue to explore more sophisticated ways to balance efficiency with robust generalization, particularly in dynamic and open-ended learning environments.
\subsection{Theoretical Insights and Differentiable Solvers}
\label{sec:3\_2\_theoretical\_insights\_\_and\_\_differentiable\_solvers}

The advancement of optimization-based meta-learning increasingly demands a rigorous theoretical foundation and the development of more efficient and interpretable base learners, moving beyond purely empirical successes. This subsection delves into two critical, often complementary, research thrusts: the analytical dissection of meta-optimization dynamics and the integration of differentiable closed-form or convex solvers within the meta-learning framework. These efforts collectively aim to demystify the "learning to learn" paradigm, fostering principled algorithm design and enhancing both the efficiency and interpretability of the adaptation process.

One significant direction focuses on enhancing the efficiency and interpretability of the inner-loop adaptation by leveraging differentiable solvers. \cite{bertinetto2018ur2} made a notable contribution by proposing meta-learning with differentiable closed-form solvers. Their work innovatively integrated classical machine learning algorithms, such as Ridge Regression and Logistic Regression (via Iteratively Reweighted Least Squares), as the base learners within a deep meta-learning framework. A key technical contribution was the efficient backpropagation through these solvers, even with high-dimensional deep features, by employing the Woodbury identity to transform computationally intensive matrix inversions into more tractable forms. This approach enables end-to-end meta-optimization of both the deep feature extractor and the base learner's hyperparameters. The primary advantage lies in grounding the adaptation process in well-understood analytical solutions, thereby enhancing interpretability and often computational efficiency compared to iterative gradient descent. However, a key limitation of this paradigm is its reliance on base learners with closed-form or efficiently solvable convex optimization problems. This restriction can limit the expressivity of the task-specific model, particularly for highly non-linear or complex tasks, making it challenging to extend these benefits to arbitrary deep neural network architectures.

Complementing these methodological advancements, a robust body of theoretical analysis has emerged to illuminate the intricate mechanics of gradient-based meta-optimization. A foundational inquiry, addressed by \cite{finn2017vrt}, explored the universality of Model-Agnostic Meta-Learning (MAML). Their work demonstrated that deep representations combined with standard gradient descent possess sufficient capacity to approximate any learning algorithm. This insight is crucial for understanding \textit{why} MAML is model-agnostic and capable of learning diverse adaptation strategies, establishing its broad expressive power and positioning it as a powerful, general-purpose meta-learner. This theoretical grounding highlights MAML's potential to learn effective adaptation rules rather than merely finding good feature extractors.

Further theoretical scrutiny has focused on the specific dynamics of MAML's inner loop. \cite{bernacchia20211r0} provided a groundbreaking theoretical analysis using advanced mathematical tools like Random Matrix Theory and the Neural Tangent Kernel (NTK) approximation for wide networks. Their research rigorously examined how MAML adapts its parameters, leading to the counter-intuitive discovery of optimal \textit{negative} learning rates during the meta-training phase for the inner loop. This finding challenges conventional assumptions about gradient descent, offering a deeper, principled understanding of how MAML's gradient-based adaptation functions to optimize the meta-objective. It suggests that, under certain conditions (e.g., overparameterized linear models or wide neural networks in the lazy training regime), the meta-learner might benefit from "unlearning" or moving away from the task-specific optimum in a specific direction during the inner loop to improve meta-generalization. However, the applicability of these specific negative learning rate findings to general deep, non-linear MAML variants with multiple inner-loop steps and finite width remains an area for further investigation, given the simplifying assumptions made in the theoretical derivations.

Beyond specific algorithm dynamics, broader theoretical frameworks have also been developed. \cite{franceschi2018u1q} introduced bilevel programming as a unifying framework for gradient-based hyperparameter optimization and meta-learning. This perspective formally models the meta-learning problem as a nested optimization, where the outer loop optimizes meta-parameters (e.g., initializations) based on the outcome of an inner-loop optimization (task-specific adaptation). This framework provides a general theoretical lens for analyzing the convergence and properties of many optimization-based meta-learning algorithms, including MAML, by explicitly accounting for the inner objective's optimization dynamics.

Moreover, understanding the generalization capabilities of meta-learning algorithms is paramount. \cite{chen2021j5t} contributed a novel information-theoretic analysis to derive data-dependent generalization bounds for meta-learning, including a stochastic variant of MAML. Their work offers a generic understanding of both conventional learning-to-learn and MAML, providing bounds that are empirically shown to be significantly tighter than previous gradient-norm-dependent bounds. This provides crucial insights into the factors that govern a meta-learner's ability to generalize to unseen tasks, moving beyond empirical observations to quantify the theoretical limits and conditions for robust performance.

In synthesis, the integration of differentiable solvers and rigorous theoretical analyses represents critical advancements in optimization-based meta-learning. Differentiable solvers offer a path to efficient, interpretable, and analytically grounded inner-loop adaptation, albeit often with limitations on model complexity and expressivity. Concurrently, theoretical works, from MAML's universality to the intricacies of its inner-loop dynamics and generalization bounds, provide essential descriptive understanding of \textit{why} and \textit{how} these complex systems learn. The unresolved challenge lies in developing meta-learning frameworks that seamlessly combine the analytical rigor and efficiency of differentiable solvers with the deep theoretical understanding of complex meta-optimization landscapes, especially for deep, non-linear models. Future research could explore hybrid models that leverage analytical solutions for specific components while retaining the flexibility of gradient-based adaptation for others, or develop more general theoretical tools capable of analyzing such composite meta-learning systems without overly restrictive assumptions. This convergence promises to yield meta-learners that are not only empirically effective but also theoretically sound, computationally efficient, and inherently more interpretable.
\subsection{Hypernetworks for Learned Adaptation}
\label{sec:3\_3\_hypernetworks\_for\_learned\_adaptation}

Achieving rapid and effective adaptation to novel tasks is a cornerstone of meta-learning. While optimization-based approaches like Model-Agnostic Meta-Learning (MAML) \cite{finn2017model} have demonstrated success by learning an adaptable initialization, they often contend with computational overhead, particularly from second-order derivatives, and inherent limitations in the expressiveness of gradient-based inner-loop updates. Hypernetworks offer a distinct methodological departure, providing a powerful mechanism for learned adaptation by directly generating the parameters of a base learner, thereby circumventing the explicit gradient updates characteristic of MAML's inner loop.

The concept of a hypernetwork, a neural network that generates the weights or parameters for another neural network (the "base learner"), was formally introduced by Ha et al. \cite{ha2016hypernetworks}. This paradigm fundamentally shifts the adaptation mechanism within meta-learning frameworks. Instead of relying on iterative gradient descent to fine-tune an initialization, the meta-learner, conceptualized as a hypernetwork, directly produces the task-specific parameters for the base learner through a single, learned feed-forward computation. This approach aligns with earlier explorations in meta-learning that investigated recurrent models trained to output the parameters of a learned model or directly generate predictions for new test inputs, as discussed by Finn et al. \cite{finn2017vrt}. These early works highlighted the potential for learned algorithms to approximate any learning strategy, with parameter generation being a key facet.

A significant recent advancement in this area is \textit{HyperMAML} \cite{przewiezlikowski2022d4y}, which explicitly applies hypernetworks for few-shot adaptation of deep models within an optimization-based meta-learning context. Przewi˛e´zlikowski et al. \cite{przewiezlikowski2022d4y} identified key limitations of traditional MAML, including its reliance on gradient-based inner-loop updates that can be computationally intensive (often requiring second-order derivatives for efficient meta-optimization) and potentially insufficient for making sufficiently large or expressive modifications to the base model's weights to capture complex task-specific nuances. HyperMAML directly addresses these issues by replacing MAML's gradient-based inner loop with a trainable hypernetwork.

The core technical innovation of HyperMAML lies in its meta-learner architecture: a hypernetwork that takes task-specific information (e.g., an embedding of the support set, base model predictions, and ground-truth labels) as input and directly outputs the weight updates ($\Delta\theta$) for the base learner. This mechanism transforms the iterative, gradient-based adaptation process into a single, learned forward pass through the hypernetwork. The benefits are multifaceted: it enables significantly more rapid adaptation by directly generating task-specific parameters, bypassing multiple gradient descent steps. Furthermore, hypernetworks can learn more complex, non-linear mappings from task information to base-learner parameters, potentially leading to more expressive and powerful adaptations than simple gradient steps from a shared initialization. Critically, this methodology inherently avoids the computational overhead associated with calculating second-order derivatives or backpropagating through multiple inner-loop steps, thereby reducing both computational cost and memory requirements during meta-training \cite{przewiezlikowski2022d4y}. Experimental validation has shown HyperMAML to consistently outperform classical MAML across several few-shot learning benchmarks, demonstrating its ability to achieve optimal task adaptation with a single update where MAML requires multiple steps.

Despite these advantages, hypernetworks for learned adaptation present unique challenges. A primary concern is \textbf{scalability}: directly generating all parameters for very large base models can lead to a hypernetwork with an astronomically large output dimension, making it computationally prohibitive and difficult to train. To mitigate this, researchers have explored various architectural innovations. These include generating low-rank updates \cite{rusu2018meta} or smaller "delta" parameters, employing weight decomposition techniques (e.g., tensor products), generating parameters for only a \textit{subset} of layers (e.g., the last layer), or producing conditional modulation parameters (e.g., scaling and shifting parameters for normalization layers, as in FiLM layers \cite{perez2018film}). Another challenge lies in the \textbf{design complexity} of the hypernetwork itself; determining its optimal depth, width, and input representation is non-trivial. The meta-training objective can be intricate, and ensuring the hypernetwork learns to generate \textit{meaningful} parameters that lead to robust base-learner performance across diverse tasks is a significant hurdle. Furthermore, while hypernetworks aim for expressive adaptation, they can be susceptible to meta-overfitting if the hypernetwork becomes too specialized to the meta-training tasks, limiting their generalization to truly novel scenarios. The \textbf{interpretability} of the learned adaptation strategy is also often reduced compared to explicit gradient steps.

In summary, hypernetworks represent a powerful and flexible approach to learned adaptation within meta-learning, offering a distinct alternative to gradient-based methods by directly generating task-specific parameters. This paradigm promises rapid, expressive adaptation while avoiding the computational burdens of second-order derivatives. While challenges related to scalability, architectural design, and generalization persist, ongoing research into efficient parameter generation and conditional modulation techniques continues to expand the toolkit for optimization-based meta-learning, pushing towards more efficient and adaptable AI systems.
\subsection{Rethinking Meta-Learning: Calibration and Task Relations}
\label{sec:3\_4\_rethinking\_meta-learning:\_calibration\_\_and\_\_task\_relations}

Achieving robust generalization across diverse tasks remains a central challenge in meta-learning, frequently hampered by issues of underfitting and overfitting during the meta-training process. Early efforts in few-shot learning, such as the approach by \cite{bertinetto2018ur2}, focused on efficient adaptation by integrating differentiable closed-form solvers like Ridge Regression as base learners. While this method offered computational efficiency and adaptability by jointly optimizing a deep feature extractor with the base learner's hyperparameters, its reliance on primarily linear models as solvers did not fully address the complex calibration issues inherent in deep, gradient-based meta-learning frameworks.

The advent of optimization-based meta-learning, exemplified by methods like Model-Agnostic Meta-Learning (MAML), introduced a powerful paradigm for learning adaptable initializations. However, these methods presented their own set of challenges. Theoretical investigations, such as those by \cite{bernacchia20211r0}, began to unravel the intricate dynamics of MAML's inner-loop optimization, revealing complexities like the potential for optimal negative learning rates. This theoretical understanding hinted at the delicate balance required for effective adaptation. Practically, MAML-like approaches often faced limitations, including insufficient weight modification during adaptation and significant computational overhead, as highlighted by \cite{przewiezlikowski2022d4y} with their HyperMAML framework. HyperMAML sought to mitigate these issues by replacing the gradient-based inner loop with a trainable Hypernetwork, demonstrating a shift towards learned, non-gradient-based adaptation mechanisms to improve efficiency and performance. Despite these advancements, a fundamental problem persisted: how to consistently calibrate the meta-learning process to prevent underfitting to complex task distributions or overfitting to specific training tasks, thereby ensuring robust generalization to unseen tasks.

A recent advancement that fundamentally rethinks this process is presented by \cite{wang2024bhk}, which proposes a novel "Learning Lens" to address the critical issues of underfitting and overfitting in meta-learning. This work introduces a new conceptualization of the meta-learning function $F\_\theta$, modeling it not as a monolithic entity but as a combination of initialization layers and a distinct 'meta-layer' implemented via gradient optimization. This explicit decomposition allows for a more granular understanding and control over the meta-learning process. To calibrate this refined model, \cite{wang2024bhk} introduces TRLearner, a plug-and-play method designed to enhance generalization and reduce excess risk. TRLearner achieves this by leveraging task relation matrices, which capture the similarities and differences between tasks within a given distribution. It then applies relation-aware consistency regularization, a mechanism that guides the meta-optimization process to ensure that models trained on similar tasks yield consistent predictions, while allowing for divergence on dissimilar tasks. This principled approach to understanding and exploiting task similarities leads to improved generalization, reduced excess risk, and a more robust performance across a diverse range of tasks.

The work by \cite{wang2024bhk} represents a significant step forward by moving beyond ad-hoc solutions to underfitting and overfitting. By providing a novel conceptualization of the meta-learner and introducing a calibration mechanism grounded in task relations, it offers a more robust and theoretically informed pathway to building meta-learning systems that can generalize effectively. However, challenges remain in the scalability of constructing and leveraging task relation matrices for extremely large and diverse task distributions, and in extending this calibration framework to other meta-learning paradigms beyond optimization-based methods. Future research could explore more efficient ways to infer task relations dynamically and integrate similar calibration principles into metric-based or model-based meta-learning approaches.


\label{sec:meta-learning_for_continual_and_biologically_inspired_adaptation}

\section{Meta-Learning for Continual and Biologically Inspired Adaptation}
\label{sec:meta-learning\_for\_continual\_\_and\_\_biologically\_inspired\_adaptation}

\subsection{Continual Learning and Lifelong Adaptation}
\label{sec:4\_1\_continual\_learning\_\_and\_\_lifelong\_adaptation}

Continual Learning (CL) and Lifelong Adaptation (LLA) represent a paramount challenge and a critical frontier in artificial intelligence, aiming to equip models with the capacity to learn sequentially from an endless stream of tasks or data over extended periods \cite{son2023lda, hospedales2020m37}. Unlike traditional deep learning, which often assumes a static dataset and a single training phase, CL demands that models continuously acquire new knowledge and skills without degrading performance on previously learned tasks. The central impediment to achieving this is "catastrophic forgetting," a phenomenon where deep neural networks abruptly lose knowledge of prior tasks upon learning new ones \cite{son2023lda}. This inherent instability stems from the fundamental plasticity-stability dilemma: the model must be plastic enough to adapt to novel information but stable enough to retain existing knowledge.

Meta-learning, often described as "learning to learn," offers a powerful paradigm to address this dilemma by focusing on acquiring generalizable learning strategies rather than task-specific knowledge \cite{peng20209of}. Within the meta-learning framework, the goal is to develop "meta-continual learning" (MCL) systems that can meta-learn the ability to learn continually, thereby facilitating persistent knowledge accumulation, effective knowledge transfer, and robust adaptation to constantly evolving environments \cite{son2023lda}. This moves beyond static task adaptation, characteristic of many initial meta-learning applications, towards dynamic, ongoing learning processes.

Early meta-learning efforts, while not explicitly designed for sequential learning, laid crucial groundwork by emphasizing rapid adaptation and knowledge transfer across tasks \cite{hospedales2020m37}. For instance, optimization-based meta-learners like Model-Agnostic Meta-Learning (MAML) \cite{finn2017vrt} learn an optimal initialization that allows for swift adaptation to a new task with only a few gradient steps. Similarly, metric-based approaches learn transferable similarity functions to generalize to novel classes from limited examples \cite{peng20209of}. These foundational methods excel at enabling rapid adaptation to \textit{individual} new tasks by leveraging shared structures across a distribution of tasks. However, they do not inherently provide explicit mechanisms to prevent catastrophic forgetting when tasks arrive \textit{sequentially} over time. The challenge for MCL lies in extending these rapid adaptation capabilities to a persistent, non-stationary stream of tasks, where the meta-learner itself must learn how to balance plasticity and stability.

To overcome catastrophic forgetting and enable true lifelong adaptation, meta-learning research has evolved to incorporate explicit mechanisms for knowledge retention and robust adaptation within sequential learning contexts. These approaches generally fall into several broad categories, which will be explored in detail in subsequent subsections:

\begin{enumerate}
    \item \textbf{Regularization-based MCL:} These methods focus on meta-learning strategies to protect important knowledge or modulate plasticity during sequential updates. This can involve learning how to selectively update parameters, constrain parameter changes, or regularize the learning process to prevent drastic shifts that would erase prior knowledge.
    \item \textbf{Replay-based MCL:} This category leverages meta-learning to optimize the use of past experiences. Instead of simply storing and replaying old data, meta-learning can be used to learn \textit{what} to replay, \textit{when} to replay, or \textit{how} to integrate replayed experiences most effectively into the learning process to mitigate forgetting.
    \item \textbf{Architecture-based and Decoupled MCL:} These approaches introduce novel structural designs or algorithmic decoupling to achieve inherent forgetting immunity. This might involve meta-learning specialized network architectures, dynamic network expansion, or separating representation learning from sequential knowledge integration to ensure that core representations remain stable while new information is assimilated.
    \item \textbf{Biologically Inspired Mechanisms:} Drawing inspiration from natural intelligence, some meta-learning methods explore biologically plausible mechanisms, such as local plasticity rules or neuromodulation, to achieve robust and energy-efficient lifelong adaptation, addressing the biological implausibility of traditional backpropagation in continuous learning settings.
\end{enumerate}

The integration of meta-learning with continual learning is crucial for developing AI systems capable of operating autonomously in dynamic, real-world environments. This includes scenarios where models must continually adapt to evolving data distributions, learn new skills in embodied AI, or maintain performance with large foundation models without constant re-training. The ability to persistently accumulate knowledge and adapt to novel circumstances without forgetting is a prerequisite for truly intelligent and versatile AI, setting the stage for more resilient and perpetually learning systems. While significant progress has been made, challenges remain in scaling these approaches to an unbounded stream of diverse tasks, formalizing the optimal balance between plasticity and stability, and developing theoretically sound and computationally efficient mechanisms to prevent catastrophic forgetting. The subsequent subsections will delve into the specific methodologies and advancements within these categories, highlighting how meta-learning is pushing the boundaries towards robust lifelong adaptation.
\subsection{Overcoming Catastrophic Forgetting with Robust Meta-Learning}
\label{sec:4\_2\_overcoming\_catastrophic\_forgetting\_with\_robust\_meta-learning}

Catastrophic forgetting, the tendency of deep neural networks to lose proficiency on previously acquired tasks when sequentially trained on new data, remains a formidable challenge in continual learning. This phenomenon critically hinders the development of truly lifelong learning systems that can accumulate knowledge over extended periods without degradation \cite{son2023lda}. Meta-learning offers a powerful paradigm to address this by enabling models to "learn to learn" more effectively, particularly when integrated with robust knowledge retention mechanisms. The field has evolved from focusing solely on rapid adaptation to novel tasks to explicitly designing meta-learning frameworks that ensure knowledge preservation across a continuous stream of experiences.

Early meta-learning research primarily focused on rapid adaptation to novel tasks with limited data, a crucial step for few-shot learning but not directly for mitigating forgetting in continual settings. For instance, the Relation Network \cite{sung2017nc5} pioneered a method for few-shot learning by meta-learning a transferable, non-linear comparison function to assess similarity between examples, enabling quick generalization to new categories. Similarly, \cite{bertinetto2018ur2} advanced few-shot learning by integrating differentiable closed-form solvers, such as ridge regression, as efficient base learners within a meta-learning framework, allowing for fast, data-dependent adaptation. While these approaches significantly improved the ability of models to quickly acquire new skills from scarce data, their core limitation in the context of continual learning was the lack of inherent strategies for robust, long-term knowledge accumulation across sequential tasks. They did not explicitly provide mechanisms to prevent the erosion of previously learned knowledge when faced with a continuous stream of new data, often operating under the assumption of distinct, isolated tasks rather than a truly lifelong learning scenario \cite{son2023lda}.

To tackle catastrophic forgetting directly, several meta-learning strategies have emerged, broadly categorized into gradient-based, replay-based, and Bayesian approaches.

\subsubsection*{Gradient-Based Meta-Continual Learning}
One line of work extends optimization-based meta-learning, such as MAML, to continual learning by learning an optimal initialization or an adaptive learning rule that minimizes forgetting. Online Meta-Learning (OML) and A Neuromodulatory Meta-Learning (ANML) algorithm are examples where the meta-learner is trained to produce parameters or modulate learning rates such that the model can quickly adapt to new tasks while retaining performance on old ones \cite{holla20202od, son2023lda}. These methods aim to find a "sweet spot" in the parameter space that facilitates rapid, non-interfering adaptation. The meta-learner essentially learns \textit{how} to update the base learner's parameters in a way that is robust to sequential task changes. However, these approaches often incur significant computational overhead due to the nested optimization structure, even with first-order approximations, and can be sensitive to the distribution of tasks encountered during meta-training. If the continual learning tasks drift significantly from this distribution, the meta-learned adaptation strategy might become ineffective, leading to suboptimal performance or even forgetting.

\subsubsection*{Replay-Based Meta-Continual Learning}
Another prominent strategy integrates meta-learning with experience replay, where a small subset of past data is stored and replayed alongside new data to prevent forgetting. Meta-learning can be used to optimize \textit{how} to select these samples or \textit{how} to integrate them into the learning process. A notable advancement in this area is presented by \cite{holla20202od}, who propose OML-ER and ANML-ER for lifelong language learning. These methods extend existing meta-learning algorithms by augmenting them with an episodic memory module for \textit{sparse experience replay}. Crucially, the replayed examples from memory are explicitly used as the \textit{query set} in the meta-learning outer-loop objective. This innovative approach directly optimizes the model to prevent forgetting by maximizing the dot product between support and query set gradients, thereby minimizing interference and maximizing knowledge transfer. \cite{holla20202od} demonstrated state-of-the-art performance on lifelong text classification and relation extraction benchmarks under realistic constraints, such as single passes over data, no task identifiers, and limited memory. While effective, replay-based methods face challenges related to memory capacity, the computational cost of replaying, and the critical problem of selecting the most representative samples to store, especially when memory is severely limited. The effectiveness of sparse replay, as shown by \cite{holla20202od}, helps mitigate these issues but does not eliminate them entirely.

\subsubsection*{Bayesian Meta-Continual Learning}
Addressing the critical problem of catastrophic forgetting with strong theoretical guarantees, recent advancements have introduced novel meta-learning frameworks that leverage fundamental Bayesian principles and statistical models. A significant step in this direction is the Sequential Bayesian Meta-Continual Learning (SB-MCL) framework proposed by \cite{lee2024snq}. This innovative approach tackles catastrophic forgetting by fundamentally decoupling deep representation learning from sequential knowledge integration.

In SB-MCL, deep neural networks are meta-trained to transform complex raw data into a latent space whose characteristics are amenable to exact Bayesian updates. Crucially, once meta-trained, the parameters of these neural networks are fixed during the actual continual learning phase, preventing any modification that could lead to forgetting. Instead, sequential knowledge updates are offloaded to simple, robust statistical models that operate on the fixed deep representations. These statistical models, often belonging to the exponential family, perform exact Bayesian updates, which are theoretically guaranteed to be lossless and forgetting-immune. This theoretical underpinning is derived from the Fisher-Darmois-Koopman-Pitman theorem, ensuring strong guarantees for long-term knowledge accumulation. By meta-learning the ability to generate representations suitable for these robust statistical models, \cite{lee2024snq} provides a principled and efficient solution to catastrophic forgetting, avoiding the computationally expensive gradient descent steps typically required for knowledge consolidation in traditional continual learning methods.

\subsubsection*{Critical Comparison and Synthesis}
Each of these meta-learning paradigms offers distinct advantages and disadvantages in combating catastrophic forgetting. Gradient-based methods provide flexibility in adapting model parameters but often struggle with computational cost and sensitivity to task distribution shifts. Replay-based meta-learning, exemplified by \cite{holla20202od}, offers strong empirical performance under realistic constraints by strategically leveraging past data, but its effectiveness is inherently tied to memory management and sample selection. In contrast, Bayesian meta-continual learning, such as SB-MCL \cite{lee2024snq}, provides compelling theoretical guarantees for lossless knowledge accumulation by decoupling representation learning from knowledge integration.

However, SB-MCL's strength—the fixed deep representation—also presents a potential limitation. If the distribution of new, unseen continual learning tasks deviates significantly from the distribution of tasks seen during the initial meta-training phase, the fixed representation might become suboptimal or even inadequate. This could lead to a failure mode where the model cannot generate suitable latent features for the statistical models, thus hindering effective learning of new tasks despite the forgetting-immune update mechanism. This trade-off highlights a fundamental challenge: balancing the stability of knowledge retention with the flexibility required to adapt to truly novel and evolving environments. While SB-MCL offers computational efficiency by avoiding gradient updates during continual learning, the initial meta-training phase for learning the robust representation can still be resource-intensive.

The development of such robust meta-learning frameworks, whether through adaptive optimization, intelligent replay, or Bayesian principles, marks a significant stride towards truly lifelong learning systems. By offering diverse strategies for lossless knowledge accumulation and addressing the complexities of deep representation alongside the stability of sequential knowledge integration, these approaches pave the way for AI agents that can continuously learn and adapt without compromising their past expertise. Future research will likely explore hybrid approaches that combine the strengths of these paradigms, scaling them to even more complex data modalities and integrating them with other advanced meta-learning techniques to enhance their applicability and robustness across a wider spectrum of real-world continual learning scenarios.
\subsection{Biologically Inspired Mechanisms for Adaptive Learning}
\label{sec:4\_3\_biologically\_inspired\_mechanisms\_for\_adaptive\_learning}

The quest for artificial intelligence systems that exhibit the efficiency, robustness, and flexibility characteristic of biological learning has led to a growing interest in biologically inspired meta-learning. This subsection explores approaches that draw directly from the brain's adaptive processes, aiming to overcome the fundamental biological implausibility of traditional deep learning paradigms, particularly backpropagation. Backpropagation, while highly effective, relies on non-local error signals, symmetric feedback weights, and precise derivative computations that are not observed in biological neural networks, leading to challenges in energy efficiency, online learning, and robustness in dynamic environments. By mimicking biological mechanisms such as neuromodulation, local synaptic plasticity, and sparse, event-driven computation, these methods seek to bridge the gap between artificial and natural intelligence, fostering more energy-efficient, robust, and flexible adaptive learning.

One prominent avenue of research focuses on \textbf{neuromodulatory control}, drawing inspiration from how neurochemicals like dopamine and serotonin dynamically alter neural processing in the brain. \cite{vecoven2018hc1} introduced Neuro-Modulated Networks (NMNs), where a dedicated "modulator" network learns to dynamically adjust the parameters of a "main" network's activation functions (e.g., slope and bias) based on the current task context. This mechanism allows the NMN to exhibit flexible, context-dependent responses, akin to how neuromodulators influence neural activity and learning in biological organisms. Crucially, the NMN architecture offers a scalable solution, as the number of modulated parameters scales linearly with the number of neurons, unlike some hypernetwork approaches where parameter generation can scale quadratically with connections. In meta-reinforcement learning tasks, NMNs demonstrated faster learning and more stable training compared to standard recurrent neural networks, highlighting their capacity for efficient adaptation to varying environmental parameters.

A second critical direction involves \textbf{meta-learning local synaptic plasticity rules}, directly addressing the biological implausibility of backpropagation's global credit assignment problem. Traditional backpropagation requires precise, non-local error signals and often symmetric feedback weights, which are biologically unrealistic. To circumvent this, \cite{lindsey202075a} proposed the "Feedback and Local Plasticity" (FLP) framework. In FLP, an outer meta-learning loop discovers local, Hebbian-like synaptic plasticity rules that update weights based solely on information available at the synapse (pre- and post-synaptic activity, and a local feedback signal). A key innovation is the use of \textit{decoupled} feedback weights, which are distinct from the feedforward weights, thus avoiding the "weight transport problem" inherent in backpropagation. This decoupling, combined with local learning rules, significantly reduces the need for global communication and memory, contributing to potential energy efficiency. FLP demonstrated performance comparable to gradient-based meta-learners on few-shot classification and regression, but critically, it \textit{significantly outperformed} them in continual learning scenarios, showcasing its robustness to catastrophic forgetting by enabling stable, online adaptation.

Further advancing the concept of meta-learned local plasticity, \cite{gu2019tvc} explored meta-learning a \textit{parametric synaptic update rule} that is spatially localized to individual neurons. Their approach trains a meta-network to generate weight updates as a product of pre- and post-synaptic neuronal outputs, where the combination of inputs (bottom-up, top-down, recurrent) to each neuron for generating the learning signal is itself a meta-learned function. This framework directly addresses the challenge of discovering effective biologically plausible rules that can train deep networks, moving beyond hand-designed Hebbian rules that often struggle with complex tasks. By learning the structure of these local rules, \cite{gu2019tvc} demonstrated their ability to drive task-relevant learning for semi-supervised tasks, offering another compelling alternative to backpropagation's global update mechanisms.

Beyond continuous-valued networks, the field is also exploring \textbf{meta-learning in Spiking Neural Networks (SNNs)}, which are inherently more biologically realistic and energy-efficient due to their sparse, event-driven communication. SNNs process information through discrete spikes, mimicking the brain's communication style, and can operate at significantly lower power consumption compared to conventional ANNs. However, training deep SNNs effectively remains a challenge. \cite{schmidgall20238t4} introduced Meta-SpikePropamine, a bi-level optimization framework that meta-learns three-factor synaptic plasticity rules for SNNs. These three-factor rules, inspired by neuroscience, incorporate pre-synaptic activity, post-synaptic activity, and a global neuromodulatory signal (the "third factor") to modulate synaptic changes. By training SNNs with these meta-learned plasticity rules, Meta-SpikePropamine demonstrated the ability to solve challenging online learning problems, effectively bridging the gap between neuroscience-derived learning models and the performance capabilities of deep learning. This approach leverages the inherent energy efficiency of SNNs while developing sophisticated, biologically plausible learning mechanisms.

Collectively, these biologically inspired meta-learning approaches represent a significant paradigm shift. Neuromodulatory networks offer dynamic, context-dependent adaptation of neural properties, providing a flexible control mechanism. Meta-learned local plasticity rules (e.g., FLP, Gu et al.) directly tackle the credit assignment problem by discovering alternatives to backpropagation that are local, potentially more energy-efficient by reducing global communication, and robust for online and continual learning. Furthermore, integrating meta-learning with SNNs (e.g., Meta-SpikePropamine) pushes towards systems that are not only adaptive but also intrinsically more energy-efficient and biologically plausible in their computational substrate.

Despite these advancements, significant challenges remain. A critical question is the extent to which these meta-learned local rules and neuromodulatory strategies can generalize to entirely different network architectures or task distributions beyond their meta-training environment. The theoretical guarantees for robustness and generalization, particularly for complex, meta-learned plasticity rules, are often less developed than for gradient-based methods. Furthermore, while energy efficiency is a stated goal, rigorous comparative analyses of the actual energy consumption of these biologically inspired models versus conventional deep learning on large-scale tasks are still emerging. Future research will need to focus on scaling these approaches to more complex real-world scenarios, integrating multiple biological mechanisms (e.g., combining neuromodulation with local plasticity in SNNs), and developing robust theoretical frameworks to understand their generalization capabilities and computational trade-offs.


\label{sec:meta-reinforcement_learning:_learning_to_act,_explore,_and_adapt_continually}

\section{Meta-Reinforcement Learning: Learning to Act, Explore, and Adapt Continually}
\label{sec:meta-reinforcement\_learning:\_learning\_to\_act,\_explore,\_\_and\_\_adapt\_continually}

\subsection{Efficient Off-Policy Meta-RL and Probabilistic Contexts}
\label{sec:5\_1\_efficient\_off-policy\_meta-rl\_\_and\_\_probabilistic\_contexts}

A critical bottleneck hindering the widespread application of meta-reinforcement learning (meta-RL) in real-world scenarios has been its inherent sample inefficiency \cite{beck2023x24, hospedales2020m37}. Traditional meta-RL algorithms often necessitate vast amounts of interaction data during both meta-training and subsequent task adaptation, largely due to their reliance on on-policy data collection. This demand for extensive online experience can negate the benefits of rapid adaptation, making deployment in data-scarce or expensive environments impractical. Addressing this, a significant research thrust has focused on integrating off-policy learning principles with meta-RL, particularly through the innovative use of probabilistic context variables to infer task identities and enable more efficient data utilization.

A seminal contribution in this direction is the Probabilistic Embeddings for Actor-Critic Reinforcement Learning (PEARL) framework, introduced by \cite{rakelly2019m09}. Prior meta-RL methods, such as those based on Model-Agnostic Meta-Learning (MAML) \cite{finn2017model}, often struggled to effectively leverage off-policy data due to the inherent mismatch between the policy used to collect data and the policy being optimized, especially when adapting parameters directly. PEARL circumvents these challenges by proposing an off-policy meta-RL algorithm that drastically improves sample efficiency during meta-training and enables rapid adaptation to new tasks with minimal online interaction.

The core technical innovation of PEARL lies in its use of \textbf{probabilistic context variables ($Z$)} to encapsulate task-specific information. This latent variable conditions the policy $\pi(a|s,z)$, allowing the agent to explicitly reason about task uncertainty, a crucial capability for effective exploration and adaptation in novel environments. To infer these probabilistic contexts, PEARL employs an amortized variational inference network, $q\_{\phi}(z|c)$, which estimates the posterior $p(z|c)$ from a history of collected experience $c$. This inference network is designed as a permutation-invariant function of prior experience, ensuring robust and efficient task identification regardless of the order of observed transitions.

Crucially, PEARL achieves its remarkable sample efficiency through a \textbf{decoupled off-policy training strategy}. Unlike prior methods that struggled to combine recurrent policies with off-policy learning, PEARL separates the training of the policy (actor-critic) from the training of the probabilistic encoder. The policy and critic are trained using standard off-policy data from a replay buffer, treating the inferred context $z$ as an additional state input. Simultaneously, the encoder $q(z|c)$ is trained with context batches, which can be sampled differently to optimize the learning of task inference. This decoupling allows PEARL to leverage the data efficiency of off-policy RL algorithms while still performing sophisticated task inference.

Furthermore, the probabilistic nature of the context variables facilitates \textbf{principled exploration} through a meta-learned variant of posterior sampling. At meta-test time, an agent samples a task hypothesis $z$ from its current posterior belief (initially a prior) and acts optimally according to this sampled $z$ for an entire episode. This strategy enables temporally-extended and structured exploration, allowing the agent to coherently test different task hypotheses and update its belief based on the collected experience. This is particularly beneficial in sparse reward environments where random exploration is often ineffective. Empirically, PEARL demonstrated a 20-100X improvement in meta-training sample efficiency and achieved superior asymptotic performance on various continuous control meta-learning benchmarks compared to existing methods \cite{rakelly2019m09}.

Despite its significant advancements, PEARL, like many meta-RL approaches, presents certain limitations. The reliance on an amortized variational inference network means that the learned posterior $q(z|c)$ may not perfectly capture the true task posterior $p(z|c)$, potentially leading to suboptimal task inference or exploration, especially for highly complex or ambiguous tasks. The performance can also be sensitive to the quality and diversity of the off-policy data in the replay buffer. Furthermore, while PEARL's approach to exploration via posterior sampling is principled, the computational overhead of maintaining and training the probabilistic encoder, along with the need for sufficient task diversity during meta-training, can still be substantial. Other works, such as \cite{xu2018rdh}, have also explored learning flexible exploration policies independent of the actor to improve sample efficiency in DDPG, highlighting a broader interest in meta-learned exploration strategies beyond task inference.

The introduction of PEARL marked a significant step towards making meta-RL more practical for real-world applications where data collection is expensive or time-consuming. By providing a robust framework for off-policy meta-learning and principled exploration via probabilistic contexts, it laid foundational groundwork. The success of PEARL in leveraging off-policy data naturally raised the question of whether adaptation could be learned from purely static, offline datasets, a challenge addressed by subsequent research into offline meta-RL. Moreover, its use of probabilistic contexts to handle uncertainty provided a crucial building block for developing agents with safety guarantees, paving the way for more robust and reliable meta-RL systems in safety-critical domains. The continued exploration of probabilistic contexts remains a vital direction for enhancing the adaptability and data efficiency of intelligent agents in complex, uncertain environments.
\subsection{Offline Meta-Reinforcement Learning}
\label{sec:5\_2\_offline\_meta-reinforcement\_learning}

The paradigm of offline meta-reinforcement learning (meta-RL) represents a crucial advancement for scenarios where online interaction is either infeasible, too costly, or unsafe. This approach enables agents to learn to adapt to new tasks solely from static, pre-collected datasets without any further online interaction, thereby addressing a critical challenge for real-world RL deployment by leveraging existing data to train meta-learners that can then rapidly adapt to novel tasks.

While off-policy meta-RL methods, such as Probabilistic Embeddings for Actor-Critic RL (PEARL) \cite{rakelly2019m09}, significantly improved meta-training sample efficiency by decoupling data collection from policy updates, they still necessitated some online interaction for agents to adapt to truly novel tasks. This reliance on online experience for task inference or adaptation updates remained a bottleneck for many real-world applications. The critical challenge of entirely eliminating online interaction during meta-learning was directly addressed by \cite{dorfman2020mgv} with Bayes-Optimal Offline Reinforcement Learning (BOReL).

BOReL extends the probabilistic context variable framework to a fully offline setting, enabling agents to meta-learn adaptation strategies exclusively from static, pre-collected datasets \cite{dorfman2020mgv}. The core innovation of BOReL lies in its ability to train a meta-policy that can infer task-specific information and adapt its behavior solely by processing offline data, without any further online environment interaction. Technically, BOReL builds upon off-policy variants of algorithms like VariBAD, where a permutation-invariant belief encoder (often a VAE) processes historical context trajectories from the offline dataset. This encoder learns to approximate the posterior distribution over latent task variables, effectively augmenting each state in the offline trajectories with a neural belief estimate, transforming the data into a Bayes-Adaptive MDP (BAMDP) format. The meta-policy is then trained to be Bayes-optimal with respect to this inferred belief, ensuring robust adaptation to new tasks based purely on the static data. This method effectively bridges the gap between data-rich offline learning, where vast amounts of interaction data can be curated, and data-scarce online adaptation, where new tasks must be tackled with minimal or no new experience.

A significant challenge identified and formalized by BOReL is \textbf{MDP ambiguity} \cite{dorfman2020mgv}. This occurs when the offline dataset does not contain sufficient information to distinguish between different underlying MDPs (i.e., different tasks), making it impossible for the meta-learner to form an accurate belief about the current task. This problem is particularly acute when the offline data primarily consists of exploitative trajectories from conventional agents, lacking the diverse exploration needed to identify task specifics. BOReL proposes principled data collection strategies (e.g., ensuring identifying state-action pairs are visited) and a "reward relabeling trick" to mitigate ambiguity when it stems solely from reward function differences. This highlights a crucial limitation: the effectiveness of offline meta-RL is highly dependent on the quality and diversity of the multi-task offline dataset, especially regarding its coverage of task-identifying information.

Beyond inferring task context, offline meta-learning has also been explored for extracting reusable components. \cite{nam2022z75} propose a skill-based meta-RL method that leverages prior experience from offline datasets to enable meta-learning on long-horizon, sparse-reward tasks. Their approach involves (1) extracting reusable skills and a skill prior from offline datasets, (2) meta-training a high-level policy to efficiently compose these learned skills into long-horizon behaviors, and (3) rapidly adapting the meta-trained policy to solve unseen target tasks. This demonstrates an alternative paradigm for utilizing offline data, not just for task inference, but for learning a transferable library of fundamental behaviors that can be quickly recombined for novel, complex tasks, significantly reducing the need for online interactions during adaptation.

More recently, the capabilities of offline meta-learning have been extended to highly complex, real-world adaptive control problems. \cite{lupu20249p4} introduced MAGICVFM (Meta-Learning Adaptation for Ground Interaction Control With Visual Foundation Models), which integrates offline meta-learning with visual foundation models (VFMs) and online composite adaptive control. In this framework, offline meta-learning is leveraged to train a robust adaptive controller that processes terrain images via VFMs, feeding features into a deep neural network for residual dynamics. This meta-learned system can then adapt its last layer in real-time for stable control of off-road vehicles, providing mathematical guarantees of exponential stability and robustness. This demonstrates how offline meta-learning can be a cornerstone for building highly adaptive and robust embodied AI systems, by pre-training general adaptation capabilities from diverse offline datasets and then enabling rapid, safe, and stable online refinement, even in safety-critical domains.

In conclusion, offline meta-RL has evolved from foundational concepts of learned RL algorithms to sophisticated frameworks capable of leveraging static data for rapid and robust adaptation. Methods like BOReL have been pivotal in demonstrating the feasibility of learning to adapt without any online interaction during meta-training, while skill-based approaches broaden the utility of offline data. Recent advancements like MAGICVFM showcase the practical utility of this paradigm in complex, safety-critical applications. However, critical challenges remain, including enhancing robustness to out-of-distribution offline data, addressing the inherent limitations posed by MDP ambiguity, and developing stronger theoretical guarantees for complex offline adaptation. Future directions will likely focus on further integrating with large foundation models and exploring methods to robustly handle imperfect or limited offline datasets to unlock even broader real-world deployment capabilities across diverse and dynamic environments.
\subsection{Meta-Learning for Safe Reinforcement Learning}
\label{sec:5\_3\_meta-learning\_for\_safe\_reinforcement\_learning}

The deployment of adaptive reinforcement learning (RL) systems in safety-critical real-world domains, such as autonomous driving, robotics, and healthcare, necessitates stringent adherence to safety protocols. Traditional meta-RL approaches, while adept at rapid adaptation and sample efficiency, have largely overlooked the crucial aspect of provable safety, posing a significant barrier to their responsible real-world integration. This subsection delves into the critical advancements in integrating explicit safety considerations directly into the meta-learning process, ensuring that agents can rapidly adapt to novel tasks while minimizing constraint violations and providing robust guarantees.

Early foundational work in meta-reinforcement learning (meta-RL) primarily focused on enabling agents to learn new tasks quickly and efficiently. For instance, the concept of deep meta-RL, where a recurrent neural network implicitly learns an RL procedure, demonstrated faster adaptation to new tasks within a distribution \cite{wang20167px, sutton2022jss}. Similarly, Model-Agnostic Meta-Learning (MAML) \cite{finn20174c4} provided a general framework for learning adaptable initializations, which was subsequently applied to RL to enable rapid policy adaptation. Further advancements, such as PEARL (Probabilistic Embeddings for Actor-Critic RL) \cite{rakelly2019m09}, significantly improved sample efficiency during meta-training by disentangling task inference from control using probabilistic context variables. While these methods marked substantial progress in meta-RL's adaptability and efficiency (as discussed in Section 5.1 for PEARL's mechanisms), their primary objective was reward maximization and learning speed, without explicit mechanisms or guarantees for safety constraint satisfaction. This oversight created a critical gap: an agent rapidly adapting to an unseen task might inadvertently violate safety constraints, rendering it unsuitable for safety-critical applications \cite{beck2023x24, hospedales2020m37}.

Addressing this crucial limitation, the field has begun to explore meta-learning frameworks that inherently incorporate safety. A significant methodological shift in this direction is the "CMDP-within-online" framework for Meta-Safe Reinforcement Learning introduced by \cite{khattar2024sr6}. This groundbreaking work extends algorithmic discovery to safety-critical domains by encapsulating a safe RL algorithm, specifically one designed for Constrained Markov Decision Processes (CMDPs), within an online meta-learner. In a CMDP, an agent aims to maximize cumulative reward while ensuring that the expected cumulative cost of certain "unsafe" actions or states remains below a predefined threshold.

The core innovation of \cite{khattar2024sr6} lies in its ability to provide the first provable guarantees for meta-safe RL. The framework updates meta-initialization policies and learning rates based on \textit{inexact} upper bounds of optimality gaps (how far the current policy is from the optimal reward) and, critically, \textit{constraint violations} (how much the current policy exceeds safety cost limits). This is a practical necessity because, in complex, non-convex CMDPs, exact estimations are often intractable. The inexact upper bounds are estimated using off-policy stationary distribution corrections, leveraging techniques like DualDICE \cite{nachum2019dualdice} to infer discounted state visitation distributions from collected trajectory data. These estimations are then used in an Online Gradient Descent (OGD) scheme to adapt the meta-initialization and learning rates. The theoretical guarantees include task-averaged regret bounds for both reward maximization and constraint violations, demonstrating that the meta-learner can adapt to unseen tasks while provably converging towards policies that satisfy safety constraints over a distribution of tasks. The analysis further leverages concepts from tame geometry and o-minimal structures to characterize the optimization landscape of CMDPs, providing crucial insights into bounding policy distances and estimation errors, which are essential for the validity of the regret bounds.

Despite its significant contributions, the CMDP-within-online framework has certain assumptions and limitations. It assumes meta-initialization policies have full support over the state-action space and that objective/constraint functions are definable in an o-minimal structure, which, while mild for many real-world functions, are theoretical requirements. Furthermore, the current theoretical guarantees are for \textit{task-averaged} regret, meaning safety is guaranteed on average across a distribution of tasks, rather than providing worst-case guarantees for every individual task instance. The framework is also exemplified with primal-based safe RL algorithms and softmax parametrization, suggesting potential avenues for extension to dual-based methods or other policy representations.

Beyond direct constraint satisfaction, the broader goal of safe meta-RL is also supported by research into robust generalization and trustworthiness. For instance, works exploring generalization bounds for meta-learning \cite{chen2021j5t} provide information-theoretic analyses that can help understand and improve the reliability of meta-learned policies. Similarly, advancements in calibrating meta-learning for robust generalization, such as the TRLearner framework \cite{wang2024bhk}, which leverages task relation matrices and consistency regularization, contribute to the overall trustworthiness of meta-RL systems. By mitigating issues like meta-overfitting and underfitting, these methods ensure that meta-learned policies perform reliably across diverse tasks, which is an implicit but critical component of safety in unpredictable environments.

In conclusion, the evolution of meta-learning for safe reinforcement learning represents a crucial step towards making adaptive RL systems viable for real-world, safety-critical applications. By moving beyond mere efficiency and adaptability to explicitly integrating provable safety guarantees, research like the CMDP-within-online framework is transforming meta-RL into a more responsible and trustworthy solution. Future directions will likely involve extending these provable guarantees to more complex, partially observable, and multi-agent safety-critical scenarios, developing more robust methods for handling highly dynamic and uncertain environments with real-time safety constraints, and exploring the integration of other safety mechanisms such as formal verification or shielding with meta-learning paradigms. This push towards provably safe and robust adaptation is essential for the widespread and ethical deployment of intelligent autonomous systems.
\subsection{Continual Adaptation in Model-Based Meta-Reinforcement Learning}
\label{sec:5\_4\_continual\_adaptation\_in\_model-based\_meta-reinforcement\_learning}

Agents operating in dynamic and evolving environments critically require the ability to continually adapt their behavior and internal understanding of the world. Meta-learning, particularly when integrated with model-based reinforcement learning (MBRL), offers a powerful paradigm for achieving persistent online adaptation, enabling agents to refine their environmental models and prevent catastrophic forgetting.

Early explorations into meta-reinforcement learning laid the groundwork for adaptive agents by demonstrating that neural networks could implicitly learn generalizable learning algorithms. For instance, \textcite{wang20167px} introduced a deep meta-RL framework where a recurrent neural network (RNN), trained across a distribution of tasks, implicitly learns an entire reinforcement learning procedure within its recurrent dynamics. This foundational work showed that an agent could learn an internal mechanism for rapid adaptation, including exploration and policy updates, based on observed rewards and actions, thereby setting the stage for more explicit model-based adaptation.

Building upon this, the Model-Based Online Learner (MOLe) proposed by \textcite{nagabandi2018esl} directly addresses continual online adaptation in MBRL by combining Model-Agnostic Meta-Learning (MAML) with an Expectation-Maximization (EM) algorithm and a Dirichlet Process-based mixture model. MOLe enables agents to learn a distribution of dynamics models and continuously refine their internal representations of the environment by either identifying the best-fitting model from the mixture or adapting existing ones to new dynamics \cite{nagabandi2018esl}. This approach is crucial for preventing catastrophic forgetting, as it maintains a diverse set of learned models, allowing for persistent improvement in performance over extended periods in non-stationary environments.

Further advancing model-based adaptation, \textcite{zintgraf2019zat} introduced VariBAD, a method for Bayes-adaptive deep RL that meta-learns a variational inference network. This network is trained to infer a latent context variable that effectively represents the current Markov Decision Process (MDP), allowing the agent to adapt its policy based on a continually refined internal model (belief) of the task \cite{zintgraf2019zat}. By performing efficient posterior inference over task parameters, VariBAD facilitates principled exploration and rapid adaptation to new tasks within a family, significantly enhancing the agent's ability to adapt its understanding of the environment in a principled, Bayesian manner.

More recently, the principles of continual adaptation in model-based settings have been extended to complex embodied AI systems. \textcite{lupu20249p4} presented MAGICVFM, a framework that integrates Visual Foundation Models (VFMs) with offline meta-learning and online composite adaptive control for robust ground interaction. This method processes terrain images via VFMs to extract features, which are then used to adapt a neural network's last layer in real-time to model residual dynamics, effectively continually refining an internal model of terrain interaction \cite{lupu20249p4}. MAGICVFM provides mathematical guarantees of exponential stability and robustness, demonstrating how agents can achieve safe and effective long-term operation in dynamic physical environments by continuously adapting their internal models based on rich sensory inputs.

In summary, the literature demonstrates a clear progression towards enabling robust and continual adaptation in model-based meta-reinforcement learning. From implicit learning of RL algorithms to explicit mixture models for dynamics and principled belief-based adaptation, these methods empower agents to persistently improve performance, adapt to changing dynamics, and tackle novel situations without succumbing to catastrophic forgetting. Future directions include developing more sophisticated and interpretable model representations, enhancing sample efficiency for real-world deployment, and integrating stronger theoretical guarantees for safety and robustness in highly dynamic and unpredictable environments.


\label{sec:robustness,_generalization,_and_trustworthiness}

\section{Robustness, Generalization, and Trustworthiness}
\label{sec:robustness,\_generalization,\_\_and\_\_trustworthiness}

\subsection{Domain Generalization through Meta-Learning}
\label{sec:6\_1\_domain\_generalization\_through\_meta-learning}

The problem of Domain Generalization (DG) poses a significant challenge for artificial intelligence systems: models must exhibit robust performance on entirely unseen target domains without any access to their data during training. This contrasts with Domain Adaptation, where some target domain data is available, and highlights the need for truly generalizable learning principles. Meta-learning offers a powerful paradigm to address this by enabling models to "learn to learn" generalizable principles rather than acquiring domain-specific knowledge, thereby facilitating the extraction of domain-invariant features or the development of robust adaptation strategies across diverse source domains \cite{khoee2024ksk}.

The core idea behind meta-learning for DG is to simulate domain shifts during the meta-training phase. By constructing meta-tasks where the "meta-training" data comes from a subset of available source domains and the "meta-testing" data comes from another, the meta-learner is forced to optimize for generalization across these simulated shifts. This process encourages the model to learn features or adaptation mechanisms that are robust to distributional changes, rather than overfitting to specific source domain characteristics.

A foundational approach in this area is Meta-Learning for Domain Generalization (MLDG) \cite{li2018learning}. MLDG adapts the Model-Agnostic Meta-Learning (MAML) framework to the DG problem. It trains a meta-learner by splitting the available source domains into a meta-training set and a meta-test set. In the inner loop, the model adapts to data from a domain within the meta-training set. In the outer loop, the meta-parameters are updated based on the model's performance on a domain from the meta-test set. This episodic training strategy explicitly optimizes for a model initialization that can quickly adapt to a new, unseen domain, effectively learning to generalize across domain shifts. This mechanism directly addresses the DG objective by promoting the learning of features and parameters that are inherently transferable and less sensitive to specific domain characteristics.

The field's understanding of meta-learning for DG has been significantly structured by recent comprehensive analyses. \cite{khoee2024ksk} provides the first dedicated survey on Domain Generalization through Meta-Learning, introducing a novel taxonomy and a decision graph to classify various meta-learning methodologies tailored for DG. This work systematically highlights how meta-learning approaches are specifically designed to handle out-of-distribution (OOD) data. Their taxonomy categorizes methods based on strategies for feature extraction (e.g., learning domain-invariant features, disentangling domain-specific factors) and classifier learning (e.g., learning robust classifiers, adapting classifiers). This structured view clarifies that meta-learning for DG often involves learning a feature extractor that is robust to domain shifts, or a meta-learner that can quickly adapt a base model to a new domain without seeing any data from it during meta-training.

Beyond MLDG's optimization-based approach, other meta-learning strategies have been leveraged for DG. Feature-based meta-learning methods aim to learn representations that are inherently invariant to domain shifts. This can be achieved through various techniques, such as adversarial training, where a domain discriminator attempts to distinguish between features from different domains, and the feature extractor is trained to fool this discriminator, thereby producing domain-agnostic features.

An example of combining adversarial training with meta-learning for cross-domain generalization is presented by \cite{tian2023iyh}. This work introduces an adversarial meta-training framework specifically for cross-domain few-shot learning, where the challenge is to generalize to new classes from unseen domains. While it addresses few-shot learning, its core mechanism directly tackles the generalization to \textit{unseen domains}. The framework utilizes a max-min episodic iteration: in the maximization phase, it dynamically generates "hard" pseudo-tasks that simulate challenging domain shifts, pushing the meta-learner to its generalization limits. In the minimization phase, the meta-learning model is trained to learn robust, cross-domain meta-knowledge that performs well on these difficult tasks. This adversarial approach forces the meta-learner to extract features and learn adaptation strategies that are resilient to significant distributional changes, thereby improving generalization across diverse and novel domains.

Despite its promise, meta-learning for DG faces several challenges. A critical limitation is the requirement for sufficient diversity among source domains during meta-training to ensure that the learned meta-knowledge truly generalizes to novel, unseen domains, rather than merely overfitting to the distribution of source domains. The performance can be sensitive to the magnitude of distributional shift between training and target domains. Furthermore, the computational cost associated with meta-training, particularly for optimization-based methods, can be substantial. Theoretical guarantees for OOD performance in meta-DG settings are also an active area of research, as understanding \textit{why} certain meta-learning strategies lead to better generalization is crucial for developing more robust systems.

In conclusion, meta-learning has emerged as a powerful paradigm for addressing Domain Generalization by enabling models to learn how to extract domain-invariant features or develop robust adaptation strategies. From foundational methods like MLDG that simulate domain shifts during meta-training to advanced adversarial meta-training frameworks, the field is continuously developing sophisticated techniques. By focusing on learning generalizable principles rather than domain-specific knowledge, meta-learning significantly enhances the ability of AI systems to generalize effectively to novel contexts. Future directions include developing more sophisticated theoretical guarantees for OOD performance, exploring meta-learning's role in mitigating bias across domains, and further integrating it with emerging foundation models to tackle even more complex and unpredictable real-world domain shifts.
\subsection{Calibrating Meta-Learning for Robust Generalization}
\label{sec:6\_2\_calibrating\_meta-learning\_for\_robust\_generalization}

Achieving robust generalization is a foundational imperative for meta-learning systems, enabling models to reliably adapt to a diverse array of novel tasks, including those unseen during meta-training. This pursuit necessitates calibrating the meta-learning process itself to mitigate critical issues such as meta-overfitting, where the meta-learner becomes overly specialized to the meta-training task distribution, and meta-underfitting, where it fails to capture sufficient transferable knowledge. Calibration, in this context, refers to the systematic adjustment and fine-tuning of the meta-learning algorithm's internal mechanisms to ensure consistent and reliable performance across varying task complexities and distributions, thereby enhancing the overall reliability and broad applicability of meta-learning in complex real-world scenarios \cite{peng20209of, wang2024bhk}. This differs from statistical calibration of confidence, which focuses on aligning predicted probabilities with true correctness, though both contribute to trustworthiness.

Early efforts to enhance meta-learning robustness, particularly when integrating with deep neural networks, often focused on improving the base learner's adaptation. For instance, Meta-Transfer Learning (MTL) \cite{sun2018iy7} was introduced to address the propensity of deep networks to overfit with limited samples in few-shot learning. MTL calibrates adaptation by learning task-specific scaling and shifting functions for deep neural network weights, coupled with a hard task (HT) meta-batch scheme. This approach aims to make deep networks adapt more robustly by preventing overfitting to scarce data. However, such methods primarily modify the base learner's behavior rather than fundamentally rethinking the meta-optimization process itself.

More recent advancements delve into the core mechanisms of meta-learning, categorizing calibration strategies into several key areas:

\subsubsection*{Optimization Dynamics Calibration}
A critical avenue for calibration involves refining the meta-optimization dynamics. The meta-gradient methods, which learn meta-parameters through gradient descent \cite{sutton2022jss}, highlight the importance of correctly setting these parameters. For example, theoretical work by \cite{bernacchia20211r0} provides crucial insights into calibrating the inner-loop learning rate in Model-Agnostic Meta-Learning (MAML). Their analysis, leveraging random matrix theory and the Neural Tangent Kernel framework, revealed the counter-intuitive finding that the optimal inner-loop learning rate during \textit{meta-training} can be \textit{negative}. This challenges conventional assumptions about gradient descent, suggesting that a carefully calibrated, potentially negative, learning rate can optimize the meta-initialization for better adaptation to new tasks. This finding is significant because it provides a theoretical basis for improving MAML's robustness by directly influencing the meta-learned model's adaptability, moving beyond empirical tuning to a principled understanding of meta-optimization. The trade-off, however, lies in the complexity of deriving and implementing such optimal rates, which often rely on simplifying assumptions (e.g., linear models, infinitely wide networks).

\subsubsection*{Regularization-Based Calibration}
Regularization techniques have emerged as a powerful means to prevent meta-overfitting and meta-underfitting by imposing constraints on the meta-learner's behavior or learned representations. \cite{yin2019cct} proposed an information-theoretic meta-regularization objective designed to prevent meta-learners from memorizing meta-training tasks. By prioritizing data-driven adaptation, this method ensures that the meta-learner genuinely learns \textit{how} to adapt rather than simply recalling solutions for specific meta-training tasks. This is particularly vital in scenarios where meta-training tasks might not be mutually exclusive, preventing the meta-learner from ignoring task-specific training data and thus fostering more robust generalization to truly novel tasks. While effective in preventing memorization, the challenge with information-theoretic approaches can be the computational cost of estimating mutual information and the difficulty in setting the regularization strength.

Building upon the conceptual rethinking of meta-learning's inherent challenges (as introduced in Section 3.4), another powerful calibration strategy involves leveraging task relationships. TRLearner \cite{wang2024bhk} is a plug-and-play method that calibrates meta-learning by extracting task relation matrices and applying relation-aware consistency regularization. This technique addresses both meta-underfitting and meta-overfitting by ensuring that the meta-learned model produces consistent performance on tasks identified as similar. The theoretical analysis in \cite{wang2024bhk} indicates that classifiers can mutually reinforce each other by leveraging features from similar tasks, leading to improved generalization and reduced excess risk. TRLearner's strength lies in its ability to adapt to task diversity and complexity by regulating task information, offering a more robust approach compared to methods that solely rely on data augmentation or model overparameterization. However, the effectiveness of TRLearner depends on the quality of the extracted task relation matrices, which can be computationally intensive to determine accurately.

\subsubsection*{Probabilistic and Uncertainty-Aware Calibration}
The inherent uncertainty in meta-learning, especially with limited data, necessitates probabilistic calibration. Bayesian meta-learning frameworks, as surveyed by \cite{peng20209of}, offer a principled approach to quantifying uncertainty in meta-learned parameters and predictions. By modeling posterior distributions over model parameters or task embeddings, these methods provide more robust and statistically calibrated confidence estimates. This form of calibration is crucial for reliable decision-making in real-world applications, allowing systems to understand when a meta-learned model is likely to perform well or poorly, thereby enhancing its trustworthiness. While Bayesian methods offer strong theoretical guarantees for uncertainty quantification, their computational complexity can be a significant limitation, especially for large deep learning models.

\subsubsection*{Theoretical Foundations for Generalization}
Theoretical underpinnings also contribute significantly to calibration efforts by providing a deeper understanding of generalization properties. \cite{chen2021j5t} derived novel information-theoretic generalization bounds for MAML, offering a generic understanding of its generalization properties. These data-dependent bounds provide a tighter analytical measure of how well meta-learned models can generalize, moving beyond empirical observations to offer theoretical guarantees for reliability. Such theoretical analyses are instrumental in guiding the design of more robust meta-learning algorithms by identifying the factors that most influence generalization, thereby informing more effective calibration strategies.

\subsubsection*{Calibration for Large Models}
The imperative for robust generalization extends to the adaptation of large pre-trained models. For instance, prompt tuning, a common method for adapting Vision-Language Models (VLMs), often suffers from severe overfitting to base classes and poor generalization to novel classes. To counter this, \cite{wang2024dai} proposed LoL (Learning to Learn Better Visual Prompts), a meta-learning-grounded approach. By integrating meta-learning's N-way K-shot episodic training strategy into prompt tuning, LoL effectively treats prompt adaptation as a meta-learning problem. This significantly improves generalization from base classes to novel, unseen classes, thereby enhancing the robustness and applicability of prompt-tuned VLMs in diverse contexts. This exemplifies how meta-learning principles can calibrate the adaptation process of powerful foundation models, ensuring their reliability beyond the training distribution, by explicitly structuring the learning process to mimic few-shot scenarios.

In conclusion, calibrating meta-learning for robust generalization is a multifaceted and continuously evolving research area that integrates diverse strategies. These range from refining base learner adaptation \cite{sun2018iy7} and optimizing meta-learning rates \cite{bernacchia20211r0}, to employing information-theoretic regularization \cite{yin2019cct} and task-relation-aware consistency \cite{wang2024bhk}. While optimization dynamics calibration directly tunes the meta-learner's update rule, regularization-based methods impose constraints on learned representations or adaptation behavior. Probabilistic methods, in contrast, provide statistical calibration by quantifying uncertainty, which complements the other approaches by enabling more trustworthy decision-making. The application of these calibration principles to adapt large foundation models \cite{wang2024dai} underscores their practical significance in contemporary AI. Future research will likely focus on developing more universally applicable and theoretically grounded calibration mechanisms that can scale across diverse meta-learning paradigms, ultimately solidifying meta-learning's role in building truly generalizable and trustworthy AI systems.
\subsection{Towards Interpretable and Trustworthy Meta-Learning}
\label{sec:6\_3\_towards\_interpretable\_\_and\_\_trustworthy\_meta-learning}

The increasing deployment of meta-learning systems in safety-critical human-machine interfaces necessitates a paradigm shift from mere predictive accuracy to ensuring interpretability and trustworthiness. In contexts where human operators must understand and confidently rely on AI's adaptive capabilities, transparent decision-making is paramount. This subsection explores key advancements in building meta-learning systems that offer explicit mechanisms for interpretability and provide robust guarantees for trustworthiness, particularly when adapting to novel tasks.

One direct avenue for enhancing interpretability in meta-learning involves providing clear confidence estimators for individual predictions. \cite{tam2024a1h} exemplifies this by re-framing EMG-based hand gesture recognition as a representation learning problem using deep metric meta-learning. Their system, based on a Siamese DCNN with triplet loss, achieves robust generalization for novel users and gestures. Critically, it provides interpretable confidence estimators derived from class proximity in the learned embedding space. These estimators enable the system to reject uncertain predictions, thereby enhancing trustworthiness by allowing human operators to understand the basis of AI decisions and intervene when confidence is low. However, the reliability of such proximity-based methods in high-dimensional embedding spaces can be compromised by phenomena like the "hubness problem" \cite{fei20211x6}. In this scenario, certain class prototypes become the nearest neighbor to many test instances regardless of their true class, leading to unreliable confidence estimates. \cite{fei20211x6} demonstrates that employing z-score feature normalization can mitigate these negative effects, improving the robustness and thus the trustworthiness of distance-based classifications within meta-learning frameworks. These works collectively suggest that while metric-based approaches offer an intuitive path to interpretability, their trustworthiness is contingent on robust embedding spaces and careful calibration to avoid inherent geometric pitfalls.

Beyond the challenges of proximity-based confidence, the broader field of uncertainty quantification (UQ) for interpretability faces fundamental scrutiny, which has significant implications for meta-learning. Meta-learning systems inherently deal with novel tasks and potentially out-of-distribution data, making reliable UQ crucial for assessing their generalization capabilities and informing human trust. \cite{shen2024hea} critically questions the reliability of modern UQ approaches like Evidential Deep Learning (EDL). Despite their perceived strong empirical performance, \cite{shen2024hea} reveals that EDL methods can be better interpreted as out-of-distribution detection algorithms rather than faithful uncertainty quantifiers, with epistemic uncertainties that may be unreliable or non-vanishing even with infinite data. This unreliability is particularly concerning for meta-learning, as the model's uncertainty estimates must be dependable not just for a single data distribution, but consistently across a range of novel tasks, a significantly harder challenge. This highlights a crucial challenge for interpretable meta-learning: the interpretability provided by confidence scores is only as reliable as the underlying uncertainty quantification mechanism. For meta-learning systems to be truly trustworthy, the methods used to estimate confidence must themselves be rigorously validated for their fidelity to actual uncertainty, especially in the context of unseen task distributions.

Given the fundamental challenges in reliably quantifying uncertainty, an alternative and complementary path to trustworthiness in high-stakes applications is to enforce safety through provable constraint satisfaction. This is particularly relevant for autonomous agents learning to adapt in dynamic environments. \cite{khattar2024sr6} introduces a novel "CMDP-within-online" framework for Meta-Safe Reinforcement Learning, crucial for adaptive agents operating in high-stakes environments. This framework provides the first provable guarantees for meta-safe RL in terms of task-averaged regret for both reward maximization and constraint violations. This ensures that meta-learned policies adhere strictly to safety constraints even when adapting to unseen tasks, a significant step towards trustworthy autonomous systems. This approach directly addresses the trustworthiness concern by offering mathematical assurances of safety, rather than relying solely on interpretable confidence estimates. However, the framework's theoretical guarantees rely on assumptions such as the "shrinkage simplex set" for meta-initialization policies and the o-minimal structure of objective functions, which, while reasonable in many contexts, define the scope of its applicability. Furthermore, the guarantees are for task-averaged regret, implying performance over a sequence of tasks rather than strict per-task guarantees in all scenarios, which may be a consideration for extremely high-risk single-task deployments. This work, while also discussed in the context of Meta-Reinforcement Learning in Section 5.3, is highlighted here specifically for its contribution to \textit{trustworthiness} through \textit{provable guarantees}, which is a distinct facet from simply enabling safe RL.

The collective efforts highlighted here signify a crucial evolution in meta-learning research: a deliberate shift towards building not just intelligent, but also \textit{responsible} AI. By prioritizing interpretable confidence (while acknowledging its inherent challenges and the limitations of underlying UQ methods) and providing provable safety guarantees, these works move meta-learning beyond purely performance-driven metrics. Future directions will likely involve further formalizing and standardizing interpretability metrics, especially in light of the critical findings regarding UQ reliability \cite{shen2024hea}, to ensure that reported confidence is truly indicative of uncertainty. Extending provable safety guarantees \cite{khattar2024sr6} to a wider range of meta-learning paradigms and complex real-world scenarios, potentially relaxing some of the current theoretical assumptions, will also be paramount. Finally, the development of comprehensive benchmarks for evaluating trustworthiness, encompassing not only accuracy and generalization but also the fidelity of uncertainty estimates and adherence to safety protocols, will be essential for the widespread adoption of meta-learning in complex, human-centric applications. This ensures that as meta-learning capabilities advance, they do so in a manner that fosters human understanding, reliance, and safety.


\label{sec:meta-learning_in_the_era_of_foundation_models_and_embodied_ai}

\section{Meta-Learning in the Era of Foundation Models and Embodied AI}
\label{sec:meta-learning\_in\_the\_era\_of\_foundation\_models\_\_and\_\_embodied\_ai}

\subsection{Adapting Visual Foundation Models for Robotic Control}
\label{sec:7\_1\_adapting\_visual\_foundation\_models\_for\_robotic\_control}

Achieving robust, real-time adaptive control for embodied AI, particularly in complex and unmodeled physical environments like off-road terrains, presents a formidable challenge. Visual Foundation Models (VFMs), combined with meta-learning, offer a promising avenue to endow robots with the necessary perception and adaptive capabilities to operate safely and autonomously.

Early explorations into meta-learning for reinforcement learning laid foundational groundwork, demonstrating that neural networks could implicitly learn adaptation strategies. For instance, \cite{wang20167px} introduced deep meta-reinforcement learning, showing that a recurrent neural network (RNN) trained across a distribution of tasks could learn an internal reinforcement learning algorithm within its recurrent dynamics. This allowed the RNN to adapt its policy based on past actions and rewards within an episode, thereby enhancing sample efficiency and flexibility for new tasks. Building on this, \cite{finn20174c4} extended meta-learning to visual robotic control, pioneering one-shot visual imitation learning via Model-Agnostic Meta-Learning (MAML). Their approach enabled a robot to acquire new skills from a single visual demonstration by meta-learning a policy that could be rapidly adapted through gradient updates, addressing the data inefficiency of learning from raw pixel inputs. While this significantly advanced policy adaptation, it still focused on learning specific policies rather than more fundamental transferable knowledge.

To address the deeper challenge of generalization and robust policy transfer, subsequent work moved towards meta-learning more fundamental components of the learning process, such as reward functions. \cite{yu2019o41} introduced Meta-Inverse Reinforcement Learning with Probabilistic Context Variables (PEMIRL), a framework that leverages a deep latent variable model to infer reward functions from unstructured, multi-task demonstrations. By learning a probabilistic embedding that captures task variations, PEMIRL enabled more robust and transferable reward inference, which is crucial for generalizing across diverse tasks where explicit reward engineering is intractable. However, these methods, while advancing adaptation and generalization, often lacked explicit mathematical guarantees for stability in real-time physical control systems.

The integration of powerful visual perception with robust, mathematically guaranteed adaptive control has culminated in advanced frameworks like MAGICVFM \cite{lupu20249p4}. This pioneering work addresses the critical need for stable operation of embodied AI, such as off-road vehicles, in complex, unmodeled physical environments. MAGICVFM synergistically combines the powerful visual perception capabilities of Visual Foundation Models (VFMs) with an offline meta-learning strategy and an online composite adaptive control scheme \cite{lupu20249p4}. Specifically, VFMs process high-dimensional terrain images to extract rich features, which are then fed into a deep neural network (DNN) trained via offline meta-learning to model residual dynamics. Crucially, only the last layer of this DNN is adapted online by a composite adaptive controller, providing real-time adjustments to maintain stable operation \cite{lupu20249p4}. This unique architecture not only enables robust adaptation to phenomena like slippage and varying terrain conditions but also provides rigorous mathematical guarantees for exponential stability and robustness, a significant advancement for safety-critical robotic deployment.

Meta-learning, in this context, provides a crucial layer of adaptability, allowing robots to generalize across diverse environmental conditions and unexpected scenarios. By learning a robust initialization or an adaptive mechanism that can quickly tune model parameters, meta-learning enhances the autonomy and safety of robots in real-world deployment. The integration of VFMs ensures that this adaptation is grounded in rich, generalizable visual understanding, moving beyond task-specific feature engineering.

Despite these advancements, several challenges remain. While MAGICVFM offers stability guarantees for specific control architectures, extending these guarantees to broader classes of VFMs and more complex, high-dimensional interaction dynamics is an active area of research. Furthermore, the computational efficiency of online adaptation for even larger foundation models and the interpretability of their adaptive behaviors in safety-critical situations warrant further investigation. Future work will likely focus on developing more comprehensive theoretical frameworks for stability and robustness in VFM-driven meta-adaptive control, alongside exploring methods for more efficient and transparent online adaptation.
\subsection{Learning to Learn Better Visual Prompts for VLMs}
\label{sec:7\_2\_learning\_to\_learn\_better\_visual\_prompts\_for\_vlms}

The advent of large Vision-Language Models (VLMs) has revolutionized multimodal AI, yet their efficient and robust adaptation to novel downstream tasks remains a significant challenge. A primary concern is the tendency of lightweight adaptation strategies, such as prompt tuning, to severely overfit to base classes, leading to suboptimal generalization performance on unseen data \cite{wang2024dai}. Meta-learning, a paradigm focused on "learning to learn" effective adaptation strategies, offers a powerful solution by enabling VLMs to acquire new visual concepts rapidly with minimal fine-tuning, thereby unlocking their full potential for few-shot learning in multimodal contexts \cite{hospedales2020m37}. This approach is particularly vital for foundation models, where full fine-tuning is computationally prohibitive, making parameter-efficient (tuning only prompts) and sample-efficient (few-shot adaptation) methods indispensable.

Building upon the foundational principles of optimization-based meta-learning (as discussed in Section 2.1 and 3.1), which involve learning adaptable initializations or optimization processes, recent work has applied these concepts to VLM prompt tuning. The core idea is to meta-optimize the prompt generation mechanism itself, rather than just the prompts for a single task. This aligns with the bilevel programming framework for hyperparameter optimization and meta-learning \cite{franceschi2018u1q}, where prompt parameters can be viewed as meta-parameters optimized at an outer level to guide an inner-loop adaptation process for specific tasks.

A prominent example of this integration is LoL (Learning to Learn) by \cite{wang2024dai}, which directly addresses the overfitting problem in VLM prompt tuning. LoL integrates the classic N-way K-shot episodic training strategy, a cornerstone of meta-learning for few-shot adaptation, into the visual prompt tuning process. In its two-stage approach, LoL first fine-tunes prompts on base classes using a standard Context Optimization (CoOp) method. Subsequently, a meta-learning stage further refines these prompts by sampling multiple classification tasks from the base classes in an N-way K-shot manner. During this meta-learning stage, the model performs inner-loop gradient updates on a support set to adapt to a specific task, and then outer-loop updates on a query set to meta-optimize the prompt generation strategy across tasks. This meta-optimization explicitly trains the system to generate prompts that generalize better from base classes to novel ones, effectively mitigating the severe overfitting observed in traditional prompt tuning methods. The success of LoL highlights how meta-gradients \cite{sutton2022jss} can be leveraged to optimize higher-level learning processes, in this case, the prompt generation itself.

Beyond LoL, other meta-learning approaches are emerging to enhance VLMs. \cite{ma2024vk4} introduced VL-Meta, a framework for multimodal meta-learning designed for Vision-Language Models, particularly for tasks like Visual Question Answering (VQA). VL-Meta employs light model structures such as vision-language and multimodal fusion mappers to efficiently utilize pre-trained models and map features into a shared space. Crucially, it constructs a meta-task pool from small amounts of data to improve generalization and proposes token-level training with a multi-task fusion loss. While LoL focuses on improving generalization for classification by mitigating overfitting in prompt tuning, VL-Meta emphasizes broader multimodal understanding and efficiency in low-resource VQA settings, demonstrating diverse applications of meta-learning in VLMs.

Despite these advancements, challenges remain. The generalization capabilities of meta-learned prompts, especially in cross-domain few-shot learning scenarios, are critical. Methods like the adversarial meta-training framework proposed by \cite{tian2023iyh} for general few-shot learning could inspire future work on making VLM prompts more robust to domain shifts by dynamically generating challenging pseudo-tasks during meta-training. Furthermore, the issue of meta-overfitting or underfitting, depending on task complexity, can still affect meta-learning systems \cite{wang2024bhk}. To address this, \cite{wang2024bhk} introduced TRLearner, which calibrates meta-learning by leveraging task relation matrices and relation-aware consistency regularization. Applying such principles to VLM prompt tuning could lead to more robust and generalizable prompts by ensuring that prompts adapted to similar tasks exhibit consistent performance, thereby reducing excess risk and improving generalization.

It is also important to critically assess the efficacy of complex meta-learning strategies. While episodic training is powerful, some findings, such as those from the NeurIPS 2021 MetaDL challenge \cite{baz2022n78}, suggest that sometimes simpler backbone fine-tuning, when coupled with excellent representations, can be highly competitive or even outperform intricate meta-learning setups for few-shot image classification. This underscores the need for careful consideration of the trade-offs between meta-learning complexity and the inherent quality of the VLM's pre-trained representations.

In conclusion, meta-learning provides a powerful paradigm for enabling VLMs to adapt efficiently and generalize effectively, particularly by optimizing the prompt tuning process. By integrating techniques like N-way K-shot episodic training and exploring advanced meta-optimization frameworks, researchers are making significant strides in mitigating overfitting and enhancing generalization from base to novel classes. Future research should continue to explore more sophisticated meta-learning architectures for prompt generation, potentially integrating insights from task relation-aware regularization \cite{wang2024bhk} and adversarial meta-training \cite{tian2023iyh} to further enhance the robustness and flexibility of VLMs in diverse, real-world multimodal scenarios.
\subsection{Benchmarking and Challenges for Meta-Learning with Large Models}
\label{sec:7\_3\_benchmarking\_\_and\_\_challenges\_for\_meta-learning\_with\_large\_models}

The integration of meta-learning with large foundation models and embodied AI systems presents a distinct set of benchmarking methodologies and formidable practical challenges. Evaluating these complex, high-stakes environments demands not only specialized datasets and rigorous metrics for generalization and adaptation but also robust protocols to assess safety, reliability, and interpretability, all while contending with immense computational demands.

While foundational meta-learning benchmarks like Meta-Dataset \cite{triantafillou2019meta} (for few-shot image classification) and Meta-World \cite{yu2019meta} (for robotic manipulation) have been instrumental in advancing the field, their scale, task diversity, and fidelity often fall short for evaluating the unique challenges of modern large foundation models and embodied AI. For instance, Meta-World, while comprehensive for manipulation, may not fully capture the complexities of real-world physics or the vast array of tasks a generalist robot might encounter. The sheer scale of modern foundation models, often with billions of parameters, introduces immense computational demands for meta-training and subsequent adaptation. Early meta-learning efforts began to address these efficiency concerns; for example, Meta-Transfer Learning (MTL) \cite{sun2018iy7} enabled efficient adaptation of deep neural networks by learning scaling and shifting functions, mitigating overfitting on sparse data. Similarly, meta-learning with differentiable closed-form solvers like Ridge Regression \cite{bertinetto2018ur2} offered an efficient way to handle high-dimensional features, laying groundwork for today's large models. These foundational concerns around efficient adaptation become paramount when adapting models of unprecedented size. Recent work, such as LoL (Learning to Learn Better Visual Prompts) \cite{wang2024dai}, directly addresses the challenge of efficiently adapting large Vision-Language Models (VLMs) by integrating N-way K-shot episodic training into prompt tuning, thereby improving generalization and tackling overfitting. Similarly, VL-Meta \cite{ma2024vk4} proposes light model structures and meta-task pools to efficiently adapt VLMs for multimodal meta-learning, saving computational power and data. For large NLP models, transformer-based meta-learners are being developed for real-world class incremental learning, demonstrating significant generalization even without specific training for this task, highlighting the need for benchmarks that assess efficient, continuous adaptation of these massive models \cite{kumar2024he9, lee2021jou}.

Beyond computational efficiency, rigorous benchmarking for meta-learning with large models necessitates specialized datasets that accurately reflect real-world variability and robust metrics for out-of-distribution (OOD) generalization. These specialized datasets often require high-fidelity physics simulations, complex multi-modal inputs, long-tailed distributions of real-world phenomena, or diverse interaction patterns not typically found in academic benchmarks. The ability of a model to perform reliably on unseen data or tasks is paramount, especially when deploying large models in diverse real-world scenarios. \cite{khoee2024ksk} provides a comprehensive survey on meta-learning for Domain Generalization (DG), offering a novel taxonomy that systematizes the field and highlights how meta-learning approaches are specifically designed to handle OOD data. This underscores the need for structured evaluation protocols that go beyond in-distribution performance. Furthermore, evaluation metrics must extend beyond mere accuracy to encompass trustworthiness and interpretability. For instance, \cite{tam2024a1h} proposes a deep metric meta-learning framework for EMG-based hand gesture recognition that provides robust confidence estimation, moving towards critical metrics for dependable human-machine interaction. The challenge also extends to evaluating the meta-learning process itself, particularly when the learning objective is not fixed. \cite{xu2020txy} introduces FRODO, an algorithm that meta-learns its own objective function online, demonstrated on tasks like Atari games. Benchmarking such adaptive learning systems requires metrics that can assess the effectiveness and generalization of the \textit{learned} objective function across diverse tasks, pushing the boundaries of traditional performance evaluation.

The challenges become particularly acute in embodied AI and safety-critical systems, where real-time adaptation, stability, and provable guarantees are non-negotiable evaluation criteria. Benchmarking in these domains requires specialized environments and metrics that capture dynamic interactions and potential risks. \cite{oconnell2022twd} exemplifies this with Neural-Fly, a meta-learning approach enabling rapid online adaptation for UAVs in strong, dynamic winds. Evaluated in a real-weather wind tunnel, Neural-Fly demonstrates precise flight control, robustness guarantees, and the ability to extrapolate to unseen wind conditions, showcasing a specialized benchmarking methodology for embodied AI. Extending this, \cite{khattar2024sr6} introduces a novel "CMDP-within-online" framework for Meta-Safe Reinforcement Learning. This work is critical for high-stakes environments as it provides the first provable guarantees for constraint satisfaction and task-averaged optimality in meta-safe RL, establishing essential metrics for safety evaluation. The ability to learn new skills from limited demonstrations is also crucial; \cite{finn20174c4} demonstrated one-shot visual imitation learning via MAML, enabling robots to acquire complex skills from single visual demonstrations, highlighting the need for benchmarks that assess rapid skill transfer in physical systems. Furthermore, addressing the "reality gap" between simulation and real-world deployment is a key benchmarking challenge. \cite{meng2024nqq} proposes MetaPNN, combining meta-RL and progressive neural networks for sim-to-real transfer, showing improved learning efficiency and performance on real robot tasks, which necessitates evaluation protocols that rigorously test this transferability. Finally, \cite{lupu20249p4} presents MAGICVFM, an innovative approach for autonomous ground vehicle control that integrates Visual Foundation Models (VFMs) with offline meta-learning and online composite adaptive control. This system offers mathematical guarantees for exponential stability and robustness during real-time terrain adaptation, directly addressing the need for reliable and safe operation of large models in dynamic embodied AI systems. These examples underscore the shift towards specialized, high-fidelity benchmarking that prioritizes real-world applicability and safety \cite{beck2023x24}.

In conclusion, benchmarking meta-learning with large models and embodied AI systems is evolving rapidly, driven by the imperative for robust, reliable, and safe AI. The literature highlights a clear shift towards specialized evaluation methodologies, including real-world simulation environments and the development of metrics that encompass not only performance but also generalization, OOD robustness, interpretability, and provable safety guarantees. Key open research questions remain in developing standardized, scalable benchmarks that accurately reflect real-world variability, creating diverse and representative datasets for foundation model adaptation, and establishing comprehensive evaluation protocols to guide the development of truly robust and trustworthy meta-learning systems for unprecedented scale and autonomy.


\label{sec:conclusion_and_future_directions}

\section{Conclusion and Future Directions}
\label{sec:conclusion\_\_and\_\_future\_directions}

\subsection{Summary of Key Advancements}
\label{sec:8\_1\_summary\_of\_key\_advancements}

Deep Meta-Learning has emerged as a pivotal paradigm in artificial intelligence, addressing the fundamental challenge of building systems that can "learn to learn" efficiently and robustly. This subsection recapitulates the major breakthroughs in the field, tracing its evolution from foundational algorithmic developments to sophisticated applications in complex, real-world domains, ultimately highlighting its collective achievements in fostering reliable and responsible AI.

Early advancements laid the groundwork for meta-learning's ability to adapt deep neural networks rapidly. A seminal contribution by \cite{wang20167px} introduced deep meta-reinforcement learning, demonstrating that a recurrent neural network (RNN), when trained across a distribution of tasks, could implicitly learn a reinforcement learning algorithm within its recurrent dynamics. This allowed the network to adapt its policy based on past actions and rewards within an episode, overcoming the data inefficiency of traditional deep RL. Concurrently, \cite{sun2018iy7} proposed Meta-Transfer Learning (MTL) for few-shot learning, which enabled deep neural networks to adapt effectively with limited samples by learning scaling and shifting functions of network weights for each task, thereby bridging the gap between shallow meta-learning architectures and the representational power of deep models.

Building upon these foundations, the field progressed towards more explicit algorithmic discovery and principled adaptation. The implicit learning of \cite{wang20167px} was refined by works such as \cite{zintgraf2019zat}, which meta-learned a variational inference network for principled, belief-based adaptation in deep RL, allowing for more robust exploration. This was further extended to the challenging offline setting by \cite{dorfman2020mgv}, which introduced BOReL for offline meta-reinforcement learning, addressing scenarios where interaction with the environment is limited. A more radical form of algorithmic discovery was presented by \cite{xu2020txy}, where the meta-learner was tasked with discovering the objective function itself online, moving beyond handcrafted objectives. More recently, \cite{wang2024bhk} offered a deeper theoretical understanding by "Rethinking Meta-Learning from a Learning Lens," proposing TRLearner to explicitly model and calibrate the meta-learning function Fθ using task-relation-aware consistency regularization, directly addressing underfitting and overfitting issues that limit generalization. This principled approach to meta-learning calibration significantly enhances its reliability across diverse tasks.

Another significant trajectory has been bridging biological plausibility with practical scalability in adaptive learning, particularly in continual learning and embodied AI. \cite{nagabandi2018esl} proposed MOLe, combining MAML with an EM-CRP mixture for continual online adaptation in model-based RL, showcasing practical solutions for dynamic environments. Biologically inspired approaches, such as \cite{vecoven2018hc1} with Neuro-Modulated Networks and \cite{lindsey202075a} with meta-learned local plasticity rules, explored alternative learning mechanisms to overcome the biological implausibility of backpropagation. Recent work has extended this to robust and interpretable adaptive systems: \cite{tam2024a1h} leveraged deep metric meta-learning for robust and interpretable EMG-based hand gesture recognition, introducing a class proximity-based confidence estimator crucial for safety-critical human-machine interfaces. Furthermore, \cite{lupu20249p4} introduced MAGICVFM, integrating Visual Foundation Models (VFMs) with offline meta-learning and online composite adaptive control for real-time terrain adaptation in off-road vehicles, providing mathematical guarantees of exponential stability and robustness in complex physical systems.

The increasing demand for deployable AI systems has driven an overarching trend towards robustness, safety, and trustworthy generalization. \cite{khoee2024ksk} provided a foundational survey on Domain Generalization (DG) through meta-learning, offering a novel taxonomy and decision graph that formalizes the problem of generalizing to unseen target domains, a critical challenge for real-world AI. Building on this, \cite{khattar2024sr6} introduced a "CMDP-within-online" framework for Meta-Safe Reinforcement Learning, providing the first provable guarantees for task-averaged regret and constraint violations, a landmark for safety-critical meta-RL. The emphasis on trustworthiness is also evident in \cite{tam2024a1h}'s robust confidence estimator, which enables interpretable decision-making. Moreover, the adaptation of large foundation models has become a key area; \cite{wang2024dai} integrated meta-learning's episodic training into prompt tuning for Vision-Language Models (VLMs), significantly improving generalization to novel classes and addressing a crucial robustness challenge in adapting powerful pre-trained models. The mathematical stability guarantees provided by \cite{lupu20249p4} for adaptive control further underscore the field's commitment to reliable operation in complex physical environments.

Collectively, these advancements demonstrate significant strides towards building AI systems that can learn to learn reliably and responsibly. From tackling fundamental challenges in data efficiency and generalization through sophisticated algorithmic designs to ensuring safety, interpretability, and stability in real-world applications, Deep Meta-Learning has solidified its role as a cornerstone of modern AI. While substantial progress has been made, future research must continue to address the scalability of meta-learning to even broader task distributions, enhance its theoretical understanding for stronger generalization guarantees, and further integrate ethical considerations into its design for truly trustworthy autonomous systems.
\subsection{Open Challenges and Theoretical Gaps}
\label{sec:8\_2\_open\_challenges\_\_and\_\_theoretical\_gaps}

Despite the remarkable progress in Deep Meta-Learning, as detailed throughout this review, the field continues to grapple with fundamental unresolved theoretical questions and substantial practical hurdles. These challenges critically impede its widespread adoption and reliable deployment, particularly in real-world, high-stakes applications. As comprehensively surveyed by \cite{hospedales2020m37}, these open problems primarily revolve around mitigating meta-overfitting, reducing computational costs, enhancing interpretability, and establishing stronger theoretical guarantees for robustness and safety. Addressing these gaps is paramount for unlocking the full potential of learning-to-learn paradigms and requires further fundamental research into their underlying principles and limitations.

One of the most pervasive and persistent challenges is \textbf{meta-overfitting}, where the meta-learner itself overfits to the distribution of training tasks, leading to poor generalization on novel, unseen tasks, especially those that are out-of-distribution (OOD). While early meta-learning approaches, such as recurrent neural networks implicitly learning reinforcement learning algorithms \cite{wang20167px}, hinted at limitations in OOD generalization, this issue remains central. As discussed in Section 6.2, recent advancements like TRLearner \cite{wang2024bhk} have made significant strides by leveraging task relation matrices and relation-aware consistency regularization to calibrate meta-learning, aiming to reduce excess risk and improve generalization. Similarly, for Vision-Language Models (VLMs), a meta-learning-grounded episodic training strategy has been shown to mitigate overfitting during prompt tuning, enhancing generalization to novel classes \cite{wang2024dai} (Section 7.2). Furthermore, the problem of "memorization" of meta-training tasks, where the meta-learner fails to adapt to new data because it can solve all meta-training tasks zero-shot, has been tackled by regularization objectives that prioritize data-driven adaptation \cite{yin2019cct}. However, despite these efforts, a robust and unified theoretical framework for quantifying and preventing meta-overfitting across arbitrarily diverse and continually evolving OOD task distributions remains elusive. Information-theoretic generalization bounds, such as those proposed by \cite{chen2021j5t}, offer valuable insights by providing tighter bounds than previous gradient-norm-dependent analyses. Yet, these theoretical analyses often rely on simplifying assumptions (e.g., specific architectural choices, data distributions) that may not fully capture the complexities of deep meta-learning in highly non-convex landscapes, limiting their direct applicability for guaranteeing OOD generalization in arbitrary real-world scenarios. The practical challenge of defining, measuring, and leveraging "task similarity" for effective regularization across heterogeneous task sets also continues to be a complex area of research.

Another significant hurdle is the \textbf{substantial computational cost} associated with meta-training, particularly for gradient-based meta-learning algorithms like MAML and its derivatives. As elaborated in Section 3.1, these methods often incur high computational and memory overhead due to the necessity of computing second-order derivatives or backpropagating through multiple inner-loop gradient steps. While scalable alternatives like Reptile and the use of differentiable closed-form solvers \cite{bertinetto2018ur2} (Section 3.2) have reduced this burden, and Hypernetworks \cite{przewiezlikowski2022d4y} (Section 3.3) offer alternative adaptation mechanisms, the problem persists. The integration of meta-learning with increasingly complex models, such as large Foundation Models \cite{wang2024dai, lupu20249p4} (Section 7.3), exacerbates these computational demands, making efficient meta-training a critical bottleneck. Furthermore, theoretical investigations into meta-optimization, such as the discovery of optimal negative inner-loop learning rates during meta-training \cite{bernacchia20211r0}, highlight the intricate dynamics of meta-learning. While such findings provide crucial theoretical insights, their full practical exploitation and integration into scalable algorithms beyond simplified linear models or specific architectures remain an open challenge. Developing meta-learning algorithms that can efficiently adapt \textit{and} meta-train with a minimal computational footprint, especially for continuous online adaptation in resource-constrained environments, is a key area for future research.

\textbf{Enhancing the interpretability} of complex meta-learned strategies is also a major open challenge, often overlooked in the pursuit of performance. Many deep meta-learning systems operate as black-box models, making it difficult to understand \textit{how} they acquire their "learning-to-learn" capabilities or \textit{why} they make specific adaptive decisions. This lack of transparency is a significant barrier to deployment in sensitive or safety-critical domains where explainability is paramount. While some progress has been made towards local interpretability, such as the class proximity-based confidence estimators introduced by \cite{tam2024a1h} (Section 6.3) for rejecting uncertain predictions, these provide only a partial view. Beyond such confidence scores, there is a notable scarcity of methods to explain \textit{why} a meta-learner chooses a particular adaptation strategy, \textit{what features} it prioritizes for rapid learning, or \textit{how} it generalizes across tasks. The broader challenge of developing \textit{global} interpretability for the meta-learning process itself, rather than just local interpretability for task-specific predictions, remains largely unaddressed. This gap underscores the need for dedicated research into explainable AI (XAI) tailored specifically for meta-learning architectures, to foster trust and facilitate debugging.

Finally, establishing \textbf{stronger theoretical guarantees for robustness and safety} in real-world, high-stakes applications is paramount. Traditional meta-learning often prioritizes performance metrics like accuracy or reward, without explicit consideration for safety constraints or guaranteed stability. While significant progress has been made, particularly in meta-reinforcement learning and embodied AI, these guarantees are often domain-specific. For instance, the "CMDP-within-online" framework for Meta-Safe Reinforcement Learning \cite{khattar2024sr6} (Section 5.3) provides the first provable guarantees for task-averaged regret for both reward and constraint violations, marking a crucial step towards safe adaptation. Similarly, MAGICVFM \cite{lupu20249p4} (Section 7.1), designed for adaptive control of off-road vehicles, offers mathematical guarantees of exponential stability and robustness. However, a unified, generalizable theoretical framework for quantifying and guaranteeing the robustness, safety, and fairness of meta-learned systems across the diverse landscape of meta-learning paradigms is still lacking. Formalizing concepts like "robustness to task distribution shifts," "adversarial meta-tasks," and "fairness" in the meta-learning context, along with developing comprehensive certification methods, remain critical open research questions.

In conclusion, while Deep Meta-Learning has demonstrated remarkable capabilities, its widespread adoption hinges on overcoming these fundamental challenges. These challenges are often interconnected; for example, high computational costs can limit the exploration of more complex, but potentially safer or more interpretable, meta-learning architectures. Conversely, a lack of interpretability can hinder the diagnosis and mitigation of meta-overfitting or safety violations. Future research must therefore focus on developing more theoretically grounded approaches to combat meta-overfitting, designing computationally efficient algorithms for scaling to large foundation models, creating inherently interpretable meta-learning strategies, and establishing comprehensive theoretical guarantees for robustness and safety to ensure reliable, responsible, and trustworthy deployment.
\subsection{Ethical Considerations and Future Research Directions}
\label{sec:8\_3\_ethical\_considerations\_\_and\_\_future\_research\_directions}

The rapid advancement of deep meta-learning, enabling AI systems to learn to learn and adapt with unprecedented speed and data efficiency, introduces a complex landscape of ethical considerations, particularly as these systems are deployed in safety-critical and human-centric domains. Beyond mere technical performance, the profound implications of autonomous and adaptive AI necessitate a proactive commitment to ethical development, encompassing transparency, accountability, safety, fairness, and control over emergent behaviors.

A primary ethical challenge lies in the inherent black-box nature of many meta-learning approaches. While early frameworks, such as deep meta-reinforcement learning with recurrent neural networks \cite{wang20167px}, demonstrated remarkable implicit adaptation, the resulting emergent strategies can be opaque and difficult to interpret. This lack of transparency poses significant hurdles for auditing, debugging, and ensuring responsible behavior in autonomous systems. More contemporary discussions, as highlighted by \cite{daglarli20216fl} in the context of Cyber-Physical Systems, underscore that despite progress in Explainable AI (xAI), integrating these methods with meta-learning for full interpretability remains a critical open problem. While solutions like \cite{tam2024a1h}'s class proximity-based confidence estimators offer valuable insights into prediction certainty, they often provide post-hoc explanations rather than revealing the underlying adaptive logic of the meta-learner itself, thus not fully addressing the ethical imperative for transparent decision-making in high-stakes applications.

The paramount importance of safety and reliability in adaptive AI systems cannot be overstated. Recent advancements, such as \cite{khattar2024sr6}'s "CMDP-within-online" framework for Meta-Safe Reinforcement Learning, offer provable guarantees for task-averaged regret and constraint violations, representing a crucial step towards deploying meta-learning agents in high-stakes environments. Similarly, \cite{lupu20249p4}'s MAGICVFM provides mathematical guarantees of exponential stability and robustness for embodied AI control. However, a critical ethical perspective requires acknowledging the limitations of such guarantees. These often rely on strong assumptions about the environment, known constraint functions, or bounded disturbances, which may not hold in complex, unmodeled real-world scenarios. Over-reliance on theoretical guarantees without rigorous validation in diverse, unpredictable conditions can lead to a false sense of security, raising ethical questions about the true scope of safety and the accountability when systems fail outside their assumed operational envelopes. Furthermore, the robustness and generalizability of meta-learned models, as addressed by \cite{wang2024bhk}'s TRLearner through task-relation-aware consistency regularization, are ethical prerequisites for reliable deployment, mitigating issues like meta-overfitting or underfitting that could lead to unpredictable or unsafe behavior.

Beyond safety and interpretability, the ethical dimension of fairness and algorithmic bias is increasingly critical for adaptive AI. Meta-learning systems, by their nature, learn to adapt from distributions of tasks or data. If these meta-training distributions contain biases, the meta-learner can inadvertently perpetuate or even amplify these biases when adapting to new, unseen tasks, leading to systematic discrimination against certain demographic groups or locations. For instance, in information retrieval, biased datasets can lead to lower ranking utility for minority groups \cite{wang2024so2}. Similarly, in spatial applications, algorithms can exhibit implicit preferences for certain locations \cite{chen2024b4d}. Emerging research, such as \cite{wang2024so2}'s Meta Curriculum-based Fair Ranking (MCFR) and \cite{chen2024b4d}'s Referee-Meta-Learning (Meta-Ref), begins to tackle these issues by dynamically adjusting learning rates or weighting losses to advocate for fairer performance. However, these are nascent efforts, and the ethical challenge of defining, measuring, and ensuring fairness in continually adapting, multi-task systems, especially when dealing with diverse and evolving notions of equity, remains largely open.

Looking ahead, several promising research directions are essential for fostering truly responsible and trustworthy AI. Firstly, integrating meta-learning with causal inference holds significant potential for building more robust and explainable models. Moving beyond mere correlation, this approach aims to enable models to not only adapt but also understand the underlying causal mechanisms driving their adaptations. Future research should investigate meta-learning frameworks that learn causal graphs or invariant causal features across tasks, offering greater transparency, accountability, and improved out-of-distribution generalization, which are crucial for auditing and debugging complex autonomous systems.

Secondly, exploring new biologically inspired architectures could lead to more efficient, robust, and inherently interpretable learning mechanisms. Instead of solely relying on gradient-based optimization, future architectures could draw deeper inspiration from biological learning processes, such as meta-learned neuromodulatory mechanisms or local plasticity rules. These could offer intrinsic pathways to explainability, dynamically balancing plasticity and stability for lifelong adaptation with reduced computational overhead and improved generalization capabilities, potentially addressing some of the black-box concerns from a foundational architectural perspective.

Finally, the development of more robust, generalizable, and continually ethical meta-learning frameworks is paramount for truly lifelong and responsible AI. As surveyed by \cite{son2023lda}, the convergence of meta-learning with online and continual learning paradigms presents significant challenges and opportunities. The problem of generalizing to unseen target domains, comprehensively surveyed by \cite{khoee2024ksk}, remains a central focus, as ethical deployment demands consistent reliability across evolving environments. Moreover, ensuring that meta-learners continually maintain ethical properties (safety, fairness, interpretability) over extended periods is critical. Advances like Sequential Bayesian Meta-Continual Learning (SB-MCL) \cite{lee2024snq}, which provides robust forgetting immunity by decoupling deep representation from sequential knowledge integration, are vital for maintaining consistent and predictable behavior in lifelong learning systems. However, while SB-MCL prevents catastrophic forgetting of \textit{knowledge}, further research is needed to guarantee the \textit{ethical consistency} of that knowledge and its application as the system adapts over time. The efficient adaptation of powerful foundation models, such as in visual prompt tuning \cite{wang2024dai}, must also be guided by ethical considerations like bias mitigation and responsible deployment. The foundational work in meta-safe reinforcement learning \cite{khattar2024sr6}, fairness-aware meta-learning \cite{wang2024so2, chen2024b4d}, and interpretable systems \cite{tam2024a1h} provides critical initial steps. Still, significant research is required to scale these ethical safeguards to increasingly complex, autonomous, and continually adapting meta-learning systems, ensuring they contribute positively to the advancement of intelligent and trustworthy AI.


\newpage
\section*{References}
\addcontentsline{toc}{section}{References}

\begin{thebibliography}{285}

\bibitem{sung2017nc5}
Flood Sung, Yongxin Yang, Li Zhang, et al. (2017). \textit{Learning to Compare: Relation Network for Few-Shot Learning}. 2018 IEEE/CVF Conference on Computer Vision and Pattern Recognition.

\bibitem{hospedales2020m37}
Timothy M. Hospedales, Antreas Antoniou, P. Micaelli, et al. (2020). \textit{Meta-Learning in Neural Networks: A Survey}. IEEE Transactions on Pattern Analysis and Machine Intelligence.

\bibitem{santoro2016323}
Adam Santoro, Sergey Bartunov, M. Botvinick, et al. (2016). \textit{Meta-Learning with Memory-Augmented Neural Networks}. International Conference on Machine Learning.

\bibitem{sun2018iy7}
Qianru Sun, Yaoyao Liu, Tat-Seng Chua, et al. (2018). \textit{Meta-Transfer Learning for Few-Shot Learning}. Computer Vision and Pattern Recognition.

\bibitem{bertinetto2018ur2}
Luca Bertinetto, João F. Henriques, Philip H. S. Torr, et al. (2018). \textit{Meta-learning with differentiable closed-form solvers}. International Conference on Learning Representations.

\bibitem{rakelly2019m09}
Kate Rakelly, Aurick Zhou, Deirdre Quillen, et al. (2019). \textit{Efficient Off-Policy Meta-Reinforcement Learning via Probabilistic Context Variables}. International Conference on Machine Learning.

\bibitem{wang20167px}
Jane X. Wang, Z. Kurth-Nelson, Hubert Soyer, et al. (2016). \textit{Learning to reinforcement learn}. Annual Meeting of the Cognitive Science Society.

\bibitem{franceschi2018u1q}
Luca Franceschi, P. Frasconi, Saverio Salzo, et al. (2018). \textit{Bilevel Programming for Hyperparameter Optimization and Meta-Learning}. International Conference on Machine Learning.

\bibitem{pan2019pue}
Zheyi Pan, Yuxuan Liang, Weifeng Wang, et al. (2019). \textit{Urban Traffic Prediction from Spatio-Temporal Data Using Deep Meta Learning}. Knowledge Discovery and Data Mining.

\bibitem{lanctot2017m2v}
Marc Lanctot, V. Zambaldi, A. Gruslys, et al. (2017). \textit{A Unified Game-Theoretic Approach to Multiagent Reinforcement Learning}. Neural Information Processing Systems.

\bibitem{finn20174c4}
Chelsea Finn, Tianhe Yu, Tianhao Zhang, et al. (2017). \textit{One-Shot Visual Imitation Learning via Meta-Learning}. Conference on Robot Learning.

\bibitem{huisman2020b7w}
M. Huisman, Jan N. van Rijn, and A. Plaat (2020). \textit{A survey of deep meta-learning}. Artificial Intelligence Review.

\bibitem{oconnell2022twd}
M. O'Connell, Guanya Shi, Xichen Shi, et al. (2022). \textit{Neural-Fly enables rapid learning for agile flight in strong winds}. Science Robotics.

\bibitem{zhu2020rb5}
Hancheng Zhu, Leida Li, Jinjian Wu, et al. (2020). \textit{MetaIQA: Deep Meta-Learning for No-Reference Image Quality Assessment}. Computer Vision and Pattern Recognition.

\bibitem{herzen2021300}
J. Herzen, Francesco Lässig, Samuele Giuliano Piazzetta, et al. (2021). \textit{Darts: User-Friendly Modern Machine Learning for Time Series}. Journal of machine learning research.

\bibitem{wang2020tae}
Jin Wang, Jia Hu, G. Min, et al. (2020). \textit{Fast Adaptive Task Offloading in Edge Computing Based on Meta Reinforcement Learning}. IEEE Transactions on Parallel and Distributed Systems.

\bibitem{yu2018nm7}
Tianhe Yu, Chelsea Finn, Annie Xie, et al. (2018). \textit{One-Shot Imitation from Observing Humans via Domain-Adaptive Meta-Learning}. Robotics: Science and Systems.

\bibitem{zintgraf2019zat}
Luisa M. Zintgraf, Kyriacos Shiarlis, Maximilian Igl, et al. (2019). \textit{VariBAD: A Very Good Method for Bayes-Adaptive Deep RL via Meta-Learning}. International Conference on Learning Representations.

\bibitem{li2018soc}
Junnan Li, Yongkang Wong, Qi Zhao, et al. (2018). \textit{Learning to Learn From Noisy Labeled Data}. Computer Vision and Pattern Recognition.

\bibitem{guo2021zpk}
Zhichun Guo, Chuxu Zhang, W. Yu, et al. (2021). \textit{Few-Shot Graph Learning for Molecular Property Prediction}. The Web Conference.

\bibitem{li20219tk}
Shuang Li, Kaixiong Gong, Chi Harold Liu, et al. (2021). \textit{MetaSAug: Meta Semantic Augmentation for Long-Tailed Visual Recognition}. Computer Vision and Pattern Recognition.

\bibitem{tian2022znj}
Yingjie Tian, Xiaoxi Zhao, and Wei-Hsin Huang (2022). \textit{Meta-learning approaches for learning-to-learn in deep learning: A survey}. Neurocomputing.

\bibitem{oh2017x02}
Junhyuk Oh, Satinder Singh, Honglak Lee, et al. (2017). \textit{Zero-Shot Task Generalization with Multi-Task Deep Reinforcement Learning}. International Conference on Machine Learning.

\bibitem{papoudakis2019gyl}
Georgios Papoudakis, Filippos Christianos, Arrasy Rahman, et al. (2019). \textit{Dealing with Non-Stationarity in Multi-Agent Deep Reinforcement Learning}. arXiv.org.

\bibitem{yoon2019k84}
Jinsung Yoon, Sercan Ö. Arik, and Tomas Pfister (2019). \textit{Data Valuation using Reinforcement Learning}. International Conference on Machine Learning.

\bibitem{rajasegaran2020llk}
Jathushan Rajasegaran, Salman Hameed Khan, Munawar Hayat, et al. (2020). \textit{iTAML: An Incremental Task-Agnostic Meta-learning Approach}. Computer Vision and Pattern Recognition.

\bibitem{yin2019cct}
Mingzhang Yin, G. Tucker, Mingyuan Zhou, et al. (2019). \textit{Meta-Learning without Memorization}. International Conference on Learning Representations.

\bibitem{qiao2019p6r}
Limeng Qiao, Yemin Shi, Jia Li, et al. (2019). \textit{Transductive Episodic-Wise Adaptive Metric for Few-Shot Learning}. IEEE International Conference on Computer Vision.

\bibitem{gao2020h75}
Kuiliang Gao, Bing Liu, Xuchu Yu, et al. (2020). \textit{Deep Relation Network for Hyperspectral Image Few-Shot Classification}. Remote Sensing.

\bibitem{patacchiola2020kpq}
Massimiliano Patacchiola, Jack Turner, Elliot J. Crowley, et al. (2020). \textit{Bayesian Meta-Learning for the Few-Shot Setting via Deep Kernels}. Neural Information Processing Systems.

\bibitem{nagabandi2018esl}
Anusha Nagabandi, Chelsea Finn, and S. Levine (2018). \textit{Deep Online Learning via Meta-Learning: Continual Adaptation for Model-Based RL}. International Conference on Learning Representations.

\bibitem{finn2017vrt}
Chelsea Finn, and S. Levine (2017). \textit{Meta-Learning and Universality: Deep Representations and Gradient Descent can Approximate any Learning Algorithm}. International Conference on Learning Representations.

\bibitem{such2019xok}
F. Such, Aditya Rawal, J. Lehman, et al. (2019). \textit{Generative Teaching Networks: Accelerating Neural Architecture Search by Learning to Generate Synthetic Training Data}. International Conference on Machine Learning.

\bibitem{fei20211x6}
Nanyi Fei, Yizhao Gao, Zhiwu Lu, et al. (2021). \textit{Z-Score Normalization, Hubness, and Few-Shot Learning}. IEEE International Conference on Computer Vision.

\bibitem{wang2021ya6}
Shaofei Wang, Marko Mihajlovic, Qianli Ma, et al. (2021). \textit{MetaAvatar: Learning Animatable Clothed Human Models from Few Depth Images}. Neural Information Processing Systems.

\bibitem{jang2019a48}
Yunhun Jang, Hankook Lee, Sung Ju Hwang, et al. (2019). \textit{Learning What and Where to Transfer}. International Conference on Machine Learning.

\bibitem{zhu2022zp1}
Taiyu Zhu, Kezhi Li, P. Herrero, et al. (2022). \textit{Personalized Blood Glucose Prediction for Type 1 Diabetes Using Evidential Deep Learning and Meta-Learning}. IEEE Transactions on Biomedical Engineering.

\bibitem{gao20223fn}
Jun Gao, Dongze Wu, Feng Yin, et al. (2022). \textit{MetaLoc: Learning to Learn Wireless Localization}. IEEE Journal on Selected Areas in Communications.

\bibitem{bartler2021i8o}
Alexander Bartler, Andreas Bühler, Felix Wiewel, et al. (2021). \textit{MT3: Meta Test-Time Training for Self-Supervised Test-Time Adaption}. International Conference on Artificial Intelligence and Statistics.

\bibitem{memon2022j2y}
M. Memon, Pardeep Kumar, and R. Iqbal (2022). \textit{Meta Deep Learn Leaf Disease Identification Model for Cotton Crop}. De Computis.

\bibitem{yang2018p36}
Zhaoyang Yang, K. Merrick, Lianwen Jin, et al. (2018). \textit{Hierarchical Deep Reinforcement Learning for Continuous Action Control}. IEEE Transactions on Neural Networks and Learning Systems.

\bibitem{guo2020acf}
Huifeng Guo, Bo Chen, Ruiming Tang, et al. (2020). \textit{An Embedding Learning Framework for Numerical Features in CTR Prediction}. Knowledge Discovery and Data Mining.

\bibitem{dixit20218dd}
Sonal Dixit, N. Verma, and A. K. Ghosh (2021). \textit{Intelligent Fault Diagnosis of Rotary Machines: Conditional Auxiliary Classifier GAN Coupled With Meta Learning Using Limited Data}. IEEE Transactions on Instrumentation and Measurement.

\bibitem{zhou20200ls}
Allan Zhou, Tom Knowles, and Chelsea Finn (2020). \textit{Meta-Learning Symmetries by Reparameterization}. International Conference on Learning Representations.

\bibitem{rajasegaran2020glw}
Jathushan Rajasegaran, Salman Hameed Khan, Munawar Hayat, et al. (2020). \textit{Self-supervised Knowledge Distillation for Few-shot Learning}. British Machine Vision Conference.

\bibitem{wang2022va1}
Yulei Wang, Xi Chen, Fengchao Wang, et al. (2022). \textit{Meta-Learning based Hyperspectral Target Detection using Siamese Network}. IEEE Transactions on Geoscience and Remote Sensing.

\bibitem{dufumier2021ec1}
Benoit Dufumier, P. Gori, J. Victor, et al. (2021). \textit{Contrastive Learning with Continuous Proxy Meta-Data for 3D MRI Classification}. International Conference on Medical Image Computing and Computer-Assisted Intervention.

\bibitem{wistuba2021wha}
Martin Wistuba, and Josif Grabocka (2021). \textit{Few-Shot Bayesian Optimization with Deep Kernel Surrogates}. International Conference on Learning Representations.

\bibitem{ren2019nu0}
Yuxiang Ren, Bo Liu, Chao Huang, et al. (2019). \textit{Heterogeneous Deep Graph Infomax}. arXiv.org.

\bibitem{zhang2020p3y}
Shen Zhang, Fei Ye, Bingnan Wang, et al. (2020). \textit{Few-Shot Bearing Fault Diagnosis Based on Model-Agnostic Meta-Learning}. IEEE transactions on industry applications.

\bibitem{zhou20188lr}
Fengwei Zhou, Bin Wu, and Zhenguo Li (2018). \textit{Deep Meta-Learning: Learning to Learn in the Concept Space}. arXiv.org.

\bibitem{bing2022om0}
Zhenshan Bing, David Lerch, Kai Huang, et al. (2022). \textit{Meta-Reinforcement Learning in Non-Stationary and Dynamic Environments}. IEEE Transactions on Pattern Analysis and Machine Intelligence.

\bibitem{tian20200qx}
Shan Tian, Songsong Mo, Liwei Wang, et al. (2020). \textit{Deep Reinforcement Learning-Based Approach to Tackle Topic-Aware Influence Maximization}. Data Science and Engineering.

\bibitem{cai20215z1}
Rizhao Cai, Zhi Li, Renjie Wan, et al. (2021). \textit{Learning Meta Pattern for Face Anti-Spoofing}. IEEE Transactions on Information Forensics and Security.

\bibitem{wang2021i3l}
Yaqing Wang, Fenglong Ma, Haoyu Wang, et al. (2021). \textit{Multimodal Emergent Fake News Detection via Meta Neural Process Networks}. Knowledge Discovery and Data Mining.

\bibitem{nam2022z75}
Taewook Nam, Shao-Hua Sun, Karl Pertsch, et al. (2022). \textit{Skill-based Meta-Reinforcement Learning}. International Conference on Learning Representations.

\bibitem{zhu2022d9a}
Hancheng Zhu, Leida Li, Jinjian Wu, et al. (2022). \textit{Generalizable No-Reference Image Quality Assessment via Deep Meta-Learning}. IEEE transactions on circuits and systems for video technology (Print).

\bibitem{wang20204p9}
Zhen Wang, Guosheng Hu, and Q. Hu (2020). \textit{Training Noise-Robust Deep Neural Networks via Meta-Learning}. Computer Vision and Pattern Recognition.

\bibitem{xu2020txy}
Zhongwen Xu, H. V. Hasselt, Matteo Hessel, et al. (2020). \textit{Meta-Gradient Reinforcement Learning with an Objective Discovered Online}. Neural Information Processing Systems.

\bibitem{xue2022ram}
Siqiao Xue, C. Qu, X. Shi, et al. (2022). \textit{A Meta Reinforcement Learning Approach for Predictive Autoscaling in the Cloud}. Knowledge Discovery and Data Mining.

\bibitem{li20208tg}
Shunkai Li, Xin Wang, Yingdian Cao, et al. (2020). \textit{Self-Supervised Deep Visual Odometry With Online Adaptation}. Computer Vision and Pattern Recognition.

\bibitem{chai2022kv5}
Yitian Chai, Hongyu Zhang, Beijun Shen, et al. (2022). \textit{Cross-Domain Deep Code Search with Meta Learning}. International Conference on Software Engineering.

\bibitem{yi2021547}
Hongsuk Yi, and Khac-Hoai Nam Bui (2021). \textit{An Automated Hyperparameter Search-Based Deep Learning Model for Highway Traffic Prediction}. IEEE transactions on intelligent transportation systems (Print).

\bibitem{pang2018qqo}
Kunkun Pang, Mingzhi Dong, Yang Wu, et al. (2018). \textit{Meta-Learning Transferable Active Learning Policies by Deep Reinforcement Learning}. arXiv.org.

\bibitem{zhang2020s15}
Haimiao Zhang, Baodong Liu, Hengyong Yu, et al. (2020). \textit{MetaInv-Net: Meta Inversion Network for Sparse View CT Image Reconstruction}. IEEE Transactions on Medical Imaging.

\bibitem{zintgraf2021lv1}
Luisa M. Zintgraf, Sebastian Schulze, Cong Lu, et al. (2021). \textit{VariBAD: Variational Bayes-Adaptive Deep RL via Meta-Learning}. Journal of machine learning research.

\bibitem{ouyang2021c4t}
W. Ouyang, Xiuwu Zhang, Shukui Ren, et al. (2021). \textit{Learning Graph Meta Embeddings for Cold-Start Ads in Click-Through Rate Prediction}. Annual International ACM SIGIR Conference on Research and Development in Information Retrieval.

\bibitem{li2019gpj}
Juncheng Li, X. Wang, Siliang Tang, et al. (2019). \textit{Unsupervised Reinforcement Learning of Transferable Meta-Skills for Embodied Navigation}. Computer Vision and Pattern Recognition.

\bibitem{yu2019o41}
Lantao Yu, Tianhe Yu, Chelsea Finn, et al. (2019). \textit{Meta-Inverse Reinforcement Learning with Probabilistic Context Variables}. Neural Information Processing Systems.

\bibitem{chen2021j5t}
Qi Chen, Changjian Shui, and M. Marchand (2021). \textit{Generalization Bounds For Meta-Learning: An Information-Theoretic Analysis}. Neural Information Processing Systems.

\bibitem{zintgraf2021hoc}
Luisa M. Zintgraf, Sam Devlin, K. Ciosek, et al. (2021). \textit{Deep Interactive Bayesian Reinforcement Learning via Meta-Learning}. Adaptive Agents and Multi-Agent Systems.

\bibitem{xu20199h0}
Hao Xu, Haibin Chang, and Dongxiao Zhang (2019). \textit{DL-PDE: Deep-learning based data-driven discovery of partial differential equations from discrete and noisy data}. Communications in Computational Physics.

\bibitem{ding2021284}
Kaize Ding, Jianling Wang, James Caverlee, et al. (2021). \textit{Meta Propagation Networks for Graph Few-shot Semi-supervised Learning}. AAAI Conference on Artificial Intelligence.

\bibitem{reed2017sxd}
Scott E. Reed, Yutian Chen, T. Paine, et al. (2017). \textit{Few-shot Autoregressive Density Estimation: Towards Learning to Learn Distributions}. International Conference on Learning Representations.

\bibitem{shao2021loj}
Huikai Shao, Dexing Zhong, Xuefeng Du, et al. (2021). \textit{Few-Shot Learning for Palmprint Recognition via Meta-Siamese Network}. IEEE Transactions on Instrumentation and Measurement.

\bibitem{jomaa20190ul}
H. Jomaa, Josif Grabocka, and L. Schmidt-Thieme (2019). \textit{Dataset2Vec: learning dataset meta-features}. Data mining and knowledge discovery.

\bibitem{woo2022f3e}
Gerald Woo, Chenghao Liu, Doyen Sahoo, et al. (2022). \textit{Learning Deep Time-index Models for Time Series Forecasting}. International Conference on Machine Learning.

\bibitem{park2020m5z}
Seongho Park, S. Han, Ji-Won Baek, et al. (2020). \textit{Meta Variance Transfer: Learning to Augment from the Others}. International Conference on Machine Learning.

\bibitem{chen2019oep}
Guangyi Chen, Tianren Zhang, Jiwen Lu, et al. (2019). \textit{Deep Meta Metric Learning}. IEEE International Conference on Computer Vision.

\bibitem{marra20192vv}
G. Marra, Francesco Giannini, Michelangelo Diligenti, et al. (2019). \textit{Integrating Learning and Reasoning with Deep Logic Models}. ECML/PKDD.

\bibitem{casebeer20225fs}
Jonah Casebeer, Nicholas J. Bryan, and Paris Smaragdis (2022). \textit{Meta-AF: Meta-Learning for Adaptive Filters}. IEEE/ACM Transactions on Audio Speech and Language Processing.

\bibitem{nobakht2022p86}
Mehrnoosh Nobakht, R. Javidan, and A. Pourebrahimi (2022). \textit{DEMD-IoT: a deep ensemble model for IoT malware detection using CNNs and network traffic}. Evolutionary Systematics.

\bibitem{vasconcelos2021fn3}
C. Vasconcelos, H. Larochelle, Vincent Dumoulin, et al. (2021). \textit{Impact of Aliasing on Generalization in Deep Convolutional Networks}. IEEE International Conference on Computer Vision.

\bibitem{libin2020x8v}
Pieter J. K. Libin, Arno Moonens, T. Verstraeten, et al. (2020). \textit{Deep reinforcement learning for large-scale epidemic control}. ECML/PKDD.

\bibitem{algan2020u0v}
G. Algan, and I. Ulusoy (2020). \textit{Meta Soft Label Generation for Noisy Labels}. International Conference on Pattern Recognition.

\bibitem{lan20196o7}
Lin Lan, Zhenguo Li, X. Guan, et al. (2019). \textit{Meta Reinforcement Learning with Task Embedding and Shared Policy}. International Joint Conference on Artificial Intelligence.

\bibitem{huang20214b2}
Chao Huang, Zhangjie Cao, Yunbo Wang, et al. (2021). \textit{MetaSets: Meta-Learning on Point Sets for Generalizable Representations}. Computer Vision and Pattern Recognition.

\bibitem{chen2022z45}
Wenlin Chen, Austin Tripp, and José Miguel Hernández-Lobato (2022). \textit{Meta-learning Adaptive Deep Kernel Gaussian Processes for Molecular Property Prediction}. International Conference on Learning Representations.

\bibitem{abdollahzadeh2021zfy}
Milad Abdollahzadeh, Touba Malekzadeh, and Ngai-Man Cheung (2021). \textit{Revisit Multimodal Meta-Learning through the Lens of Multi-Task Learning}. Neural Information Processing Systems.

\bibitem{chen2019xg0}
Shangyu Chen, Wenya Wang, and Sinno Jialin Pan (2019). \textit{MetaQuant: Learning to Quantize by Learning to Penetrate Non-differentiable Quantization}. Neural Information Processing Systems.

\bibitem{xu2018rdh}
Tianbing Xu, Qiang Liu, Liang Zhao, et al. (2018). \textit{Learning to Explore with Meta-Policy Gradient}. International Conference on Machine Learning.

\bibitem{shen2022kdk}
Xiangyu Shen, Haifeng Zheng, Xinxin Feng, et al. (2022). \textit{ML-HGR-Net: A Meta-Learning Network for FMCW Radar Based Hand Gesture Recognition}. IEEE Sensors Journal.

\bibitem{zhang2021hh1}
Huayi Zhang, Lei Cao, Peter M. VanNostrand, et al. (2021). \textit{ELITE: Robust Deep Anomaly Detection with Meta Gradient}. Knowledge Discovery and Data Mining.

\bibitem{lin2022i6k}
Hong Lin, Rita Tse, Su-Kit Tang, et al. (2022). \textit{Few-shot learning approach with multi-scale feature fusion and attention for plant disease recognition}. Frontiers in Plant Science.

\bibitem{tseng2020m83}
Hung-Yu Tseng, Yi-Wen Chen, Yi-Hsuan Tsai, et al. (2020). \textit{Regularizing Meta-Learning via Gradient Dropout}. Asian Conference on Computer Vision.

\bibitem{peng20209of}
Huimin Peng (2020). \textit{A Comprehensive Overview and Survey of Recent Advances in Meta-Learning}. arXiv.org.

\bibitem{fong20183q5}
S. Fong, S. Deb, and Xin-She Yang (2018). \textit{How meta-heuristic algorithms contribute to deep learning in the hype of big data analytics}. Unpublished manuscript.

\bibitem{liu2022tgc}
Chang Liu, Xiang Yu, Yao-Hung Hubert Tsai, et al. (2022). \textit{Learning to Learn across Diverse Data Biases in Deep Face Recognition}. Computer Vision and Pattern Recognition.

\bibitem{bing2022xo7}
Zhenshan Bing, A. Koch, Xiangtong Yao, et al. (2022). \textit{Meta-Reinforcement Learning via Language Instructions}. IEEE International Conference on Robotics and Automation.

\bibitem{nakasi2020w5x}
Rose Nakasi, Ernest Mwebaze, A. Zawedde, et al. (2020). \textit{A new approach for microscopic diagnosis of malaria parasites in thick blood smears using pre-trained deep learning models}. SN Applied Sciences.

\bibitem{przewiezlikowski2022d4y}
Marcin Przewiezlikowski, P. Przybysz, J. Tabor, et al. (2022). \textit{HyperMAML: Few-Shot Adaptation of Deep Models with Hypernetworks}. Neurocomputing.

\bibitem{xu2018rjq}
Tianbing Xu, Qiang Liu, Liang Zhao, et al. (2018). \textit{Learning to Explore via Meta-Policy Gradient}. International Conference on Machine Learning.

\bibitem{ding2019a79}
Yue Ding, Xia Tian, Lirong Yin, et al. (2019). \textit{Multi-scale Relation Network for Few-Shot Learning Based on Meta-learning}. International Conference on Virtual Storytelling.

\bibitem{dorfman2020gku}
Ron Dorfman, and Aviv Tamar (2020). \textit{Offline Meta Learning of Exploration}. Unpublished manuscript.

\bibitem{wang20210y3}
Ze Wang, Zichen Miao, Xiantong Zhen, et al. (2021). \textit{Learning to Learn Dense Gaussian Processes for Few-Shot Learning}. Neural Information Processing Systems.

\bibitem{millea2021bfu}
Adrian Millea (2021). \textit{Deep Reinforcement Learning for Trading - A Critical Survey}. International Conference on Data Technologies and Applications.

\bibitem{lindsey202075a}
Jack W Lindsey, and Ashok Litwin-Kumar (2020). \textit{Learning to Learn with Feedback and Local Plasticity}. Neural Information Processing Systems.

\bibitem{gao2022y3s}
Jun Gao, Ceyao Zhang, Qinglei Kong, et al. (2022). \textit{MetaLoc: Learning to Learn Indoor RSS Fingerprinting Localization over Multiple Scenarios}. ICC 2022 - IEEE International Conference on Communications.

\bibitem{ren2022fc5}
Xinyue Ren, Weiwei Zhang, Ming-hui Wu, et al. (2022). \textit{Meta-YOLO: Meta-Learning for Few-Shot Traffic Sign Detection via Decoupling Dependencies}. Applied Sciences.

\bibitem{wen2022sql}
Xin-Cheng Wen, Cuiyun Gao, Jiaxin Ye, et al. (2022). \textit{Meta-Path Based Attentional Graph Learning Model for Vulnerability Detection}. IEEE Transactions on Software Engineering.

\bibitem{vecoven2018hc1}
Nicolas Vecoven, D. Ernst, Antoine Wehenkel, et al. (2018). \textit{Introducing neuromodulation in deep neural networks to learn adaptive behaviours}. PLoS ONE.

\bibitem{liang2021juf}
Xuejian Liang, Yehui Zhang, and Junping Zhang (2021). \textit{Attention Multisource Fusion-Based Deep Few-Shot Learning for Hyperspectral Image Classification}. IEEE Journal of Selected Topics in Applied Earth Observations and Remote Sensing.

\bibitem{holla2020r6z}
Nithin Holla, Pushkar Mishra, H. Yannakoudakis, et al. (2020). \textit{Learning to Learn to Disambiguate: Meta-Learning for Few-Shot Word Sense Disambiguation}. Findings.

\bibitem{fernando2018lt5}
Chrisantha Fernando, Jakub Sygnowski, Simon Osindero, et al. (2018). \textit{Meta-learning by the Baldwin effect}. Annual Conference on Genetic and Evolutionary Computation.

\bibitem{jankowski20138zb}
Norbert Jankowski, Wlodzislaw Duch, and Krzysztof Grabczewski (2013). \textit{Meta-Learning in Computational Intelligence}. Meta-Learning in Computational Intelligence.

\bibitem{banayeeanzade2021zke}
Mohammadamin Banayeeanzade, Rasoul Mirzaiezadeh, Hosein Hasani, et al. (2021). \textit{Generative vs. Discriminative: Rethinking The Meta-Continual Learning}. Neural Information Processing Systems.

\bibitem{campedelli2021jja}
G. Campedelli, Mihovil Bartulovic, and K. Carley (2021). \textit{Learning future terrorist targets through temporal meta-graphs}. Scientific Reports.

\bibitem{jiang20220tg}
Xiruo Jiang, Sheng Liu, Xili Dai, et al. (2022). \textit{Deep Metric Learning Based on Meta-Mining Strategy With Semiglobal Information}. IEEE Transactions on Neural Networks and Learning Systems.

\bibitem{zheng2021olf}
Wenbo Zheng, Lan Yan, Chao Gou, et al. (2021). \textit{Learning to learn by yourself: Unsupervised meta‐learning with self‐knowledge distillation for COVID‐19 diagnosis from pneumonia cases}. International Journal of Intelligent Systems.

\bibitem{alandoli2021pqm}
M. Al-Andoli, Shing Chiang Tan, W. Cheah, et al. (2021). \textit{A Review on Community Detection in Large Complex Networks from Conventional to Deep Learning Methods: A Call for the Use of Parallel Meta-Heuristic Algorithms}. IEEE Access.

\bibitem{li2021tkg}
Denghao Li, Pablo Ortega, Xia Wei, et al. (2021). \textit{Model-Agnostic Meta-Learning for EEG Motor Imagery Decoding in Brain-Computer-Interfacing}. International IEEE/EMBS Conference on Neural Engineering.

\bibitem{daglarli2020nzw}
Evren Daglarli (2020). \textit{Explainable Artificial Intelligence (xAI) Approaches and Deep Meta-Learning Models}. Advances and Applications in Deep Learning.

\bibitem{phaphuangwittayakul2022api}
Aniwat Phaphuangwittayakul, Fangli Ying, Yi Guo, et al. (2022). \textit{Few-shot image generation based on contrastive meta-learning generative adversarial network}. The Visual Computer.

\bibitem{nie2021pcz}
Jing Nie, Nianyi Wang, Jingbin Li, et al. (2021). \textit{Meta-learning prediction of physical and chemical properties of magnetized water and fertilizer based on LSTM}. Plant Methods.

\bibitem{zheng20200ig}
Wenzhao Zheng, Jiwen Lu, and Jie Zhou (2020). \textit{Deep Metric Learning via Adaptive Learnable Assessment}. Computer Vision and Pattern Recognition.

\bibitem{chen2022ccz}
Yixiong Chen, Chunhui Zhang, C. Ding, et al. (2022). \textit{Generating and Weighting Semantically Consistent Sample Pairs for Ultrasound Contrastive Learning}. IEEE Transactions on Medical Imaging.

\bibitem{qu2022mu6}
H. Qu, Sheng Liu, Jun Li, et al. (2022). \textit{Adaptation and Learning to Learn (ALL): An Integrated Approach for Small-Sample Parking Occupancy Prediction}. Mathematics.

\bibitem{alajaji2020b6c}
D. Alajaji, and H. Alhichri (2020). \textit{Few Shot Scene Classification in Remote Sensing using Meta-Agnostic Machine}. 2020 6th Conference on Data Science and Machine Learning Applications (CDMA).

\bibitem{jiang2021uo6}
Mengjuan Jiang, Fanzhang Li, and Li Liu (2021). \textit{Continual meta-learning algorithm}. Applied intelligence (Boston).

\bibitem{rkhami2021c1c}
Anouar Rkhami, Yassine Hadjadj-Aoul, and A. Outtagarts (2021). \textit{Learn to improve: A novel deep reinforcement learning approach for beyond 5G network slicing}. Consumer Communications and Networking Conference.

\bibitem{holla20202od}
Nithin Holla, Pushkar Mishra, H. Yannakoudakis, et al. (2020). \textit{Meta-Learning with Sparse Experience Replay for Lifelong Language Learning}. arXiv.org.

\bibitem{bhathiya2020avm}
H. S. Bhathiya, and Uthayasanker Thayasivam (2020). \textit{Meta Learning for Few-Shot Joint Intent Detection and Slot-Filling}. International Conference on Machine Learning Technologies.

\bibitem{bernacchia20211r0}
A. Bernacchia (2021). \textit{Meta-learning with negative learning rates}. International Conference on Learning Representations.

\bibitem{kumar2017p0v}
Saurabh Kumar, Pararth Shah, Dilek Z. Hakkani-Tür, et al. (2017). \textit{Federated Control with Hierarchical Multi-Agent Deep Reinforcement Learning}. arXiv.org.

\bibitem{daglarli20216fl}
Evren Daglarli (2021). \textit{Explainable Artificial Intelligence (xAI) Approaches and Deep Meta-Learning Models for Cyber-Physical Systems}. Advances in Systems Analysis, Software Engineering, and High Performance Computing.

\bibitem{baz2022n78}
Adrian El Baz, André Carvalho, Hong Chen, et al. (2022). \textit{Lessons learned from the NeurIPS 2021 MetaDL challenge: Backbone fine-tuning without episodic meta-learning dominates for few-shot learning image classification}. Neural Information Processing Systems.

\bibitem{dorfman2020mgv}
Ron Dorfman, and Aviv Tamar (2020). \textit{Offline Meta Reinforcement Learning}. arXiv.org.

\bibitem{vuorio2018gwb}
Risto Vuorio, D.-Y. Cho, Daejoong Kim, et al. (2018). \textit{Meta Continual Learning}. arXiv.org.

\bibitem{ma2021kfz}
Ying Ma, G. Zhong, Wen Liu, et al. (2021). \textit{ML-CGAN: Conditional Generative Adversarial Network with a Meta-learner Structure for High-Quality Image Generation with Few Training Data}. Cognitive Computation.

\bibitem{tian2016j46}
Zhonghuan Tian, and S. Fong (2016). \textit{Survey of Meta-Heuristic Algorithms for Deep Learning Training}. Unpublished manuscript.

\bibitem{pinedaarango2021254}
Sebastian Pineda-Arango, Felix Heinrich, Kiran Madhusudhanan, et al. (2021). \textit{Multimodal Meta-Learning for Time Series Regression}. AALTD@ECML/PKDD.

\bibitem{foliadis20223y2}
Anastasios Foliadis, M. H. C. García, R. Stirling-Gallacher, et al. (2022). \textit{Multi-Environment based Meta-Learning with CSI Fingerprints for Radio Based Positioning}. IEEE Wireless Communications and Networking Conference.

\bibitem{sutton2022jss}
R. Sutton (2022). \textit{A History of Meta-gradient: Gradient Methods for Meta-learning}. arXiv.org.

\bibitem{huo2022rp4}
Haomiao Huo, Jindan Xu, Gege Su, et al. (2022). \textit{Intelligent MIMO Detection Using Meta Learning}. IEEE Wireless Communications Letters.

\bibitem{hurtado2021h2q}
J. Hurtado, Hans Lobel, and Álvaro Soto (2021). \textit{Overcoming Catastrophic Forgetting Using Sparse Coding and Meta Learning}. IEEE Access.

\bibitem{lee2021jou}
Hung-yi Lee, Ngoc Thang Vu, and Shang-Wen Li (2021). \textit{Meta Learning and Its Applications to Natural Language Processing}. Annual Meeting of the Association for Computational Linguistics.

\bibitem{peixoto20180pd}
B. Peixoto, S. Avila, Zanoni Dias, et al. (2018). \textit{Breaking down violence: A deep-learning strategy to model and classify violence in videos}. ARES.

\bibitem{yuan20205j8}
Pengyu Yuan, Aryan Mobiny, J. Jahanipour, et al. (2020). \textit{Few Is Enough: Task-Augmented Active Meta-Learning for Brain Cell Classification}. International Conference on Medical Image Computing and Computer-Assisted Intervention.

\bibitem{zhang2021yox}
Zitian Zhang, Fuyou Li, Xiaoli Chu, et al. (2021). \textit{dmTP: A Deep Meta-Learning Based Framework for Mobile Traffic Prediction}. IEEE wireless communications.

\bibitem{luo202123f}
Yaoru Luo, Guole Liu, Yuanhao Guo, et al. (2021). \textit{Deep Neural Networks Learn Meta-Structures from Noisy Labels in Semantic Segmentation}. AAAI Conference on Artificial Intelligence.

\bibitem{chen2021yqh}
Mengting Chen, Xinggang Wang, Heng Luo, et al. (2021). \textit{Learning to focus: cascaded feature matching network for few-shot image recognition}. Science China Information Sciences.

\bibitem{behl2018rjm}
Harkirat Singh Behl, M. Najafi, and Philip H. S. Torr (2018). \textit{Meta-Learning Deep Visual Words for Fast Video Object Segmentation}. IEEE/RJS International Conference on Intelligent RObots and Systems.

\bibitem{xu2019brv}
Zhixiong Xu, Lei Cao, and Xi-liang Chen (2019). \textit{Learning to Learn: Hierarchical Meta-Critic Networks}. IEEE Access.

\bibitem{sultana202094g}
N. Sultana, Jeffrey Chan, and A. K. Qin (2020). \textit{Learning to Optimise General TSP Instances}. arXiv.org.

\bibitem{furfaro20197q6}
R. Furfaro, T. Campbell, R. Linares, et al. (2019). \textit{Space Debris Identification and Characterization via Deep Meta-Learning}. Unpublished manuscript.

\bibitem{gu2019tvc}
Keren Gu, S. Greydanus, Luke Metz, et al. (2019). \textit{Meta-Learning Biologically Plausible Semi-Supervised Update Rules}. bioRxiv.

\bibitem{zhang2021p9j}
Yaobin Zhang, Weihong Deng, Yaoyao Zhong, et al. (2021). \textit{Adaptive Label Noise Cleaning with Meta-Supervision for Deep Face Recognition}. IEEE International Conference on Computer Vision.

\bibitem{saadallah2021ihn}
A. Saadallah, and K. Morik (2021). \textit{Online Ensemble Aggregation using Deep Reinforcement Learning for Time Series Forecasting}. International Conference on Data Science and Advanced Analytics.

\bibitem{luedtke2020uub}
Alexander Luedtke, M. Carone, N. Simon, et al. (2020). \textit{Learning to learn from data: Using deep adversarial learning to construct optimal statistical procedures}. Science Advances.

\bibitem{nasrabadi201801h}
Morteza Zakeri Nasrabadi, S. Parsa, and A. Kalaee (2018). \textit{Format-aware learn&fuzz: deep test data generation for efficient fuzzing}. Neural computing & applications (Print).

\bibitem{puri20202sx}
R. Puri, A. Zakhor, and Raul Puri (2020). \textit{Few Shot Learning For Point Cloud Data Using Model Agnostic Meta Learning}. International Conference on Information Photonics.

\bibitem{beck2023x24}
Jacob Beck, Risto Vuorio, E. Liu, et al. (2023). \textit{A Tutorial on Meta-Reinforcement Learning}. Found. Trends Mach. Learn..

\bibitem{zhang2023t7k}
Baoquan Zhang, and Demin Yu (2023). \textit{MetaDiff: Meta-Learning with Conditional Diffusion for Few-Shot Learning}. AAAI Conference on Artificial Intelligence.

\bibitem{khoee2024ksk}
Arsham Gholamzadeh Khoee, Yinan Yu, and R. Feldt (2024). \textit{Domain Generalization through Meta-Learning: A Survey}. Artificial Intelligence Review.

\bibitem{hao2023zfk}
Xiaoyang Hao, Zhixi Feng, Shuyuan Yang, et al. (2023). \textit{Automatic Modulation Classification via Meta-Learning}. IEEE Internet of Things Journal.

\bibitem{nathaniel2023ycu}
Juan Nathaniel, Jiangong Liu, and P. Gentine (2023). \textit{MetaFlux: Meta-learning global carbon fluxes from sparse spatiotemporal observations}. Scientific Data.

\bibitem{singh2023zo5}
Kuljeet Singh, and Deepti Malhotra (2023). \textit{Meta-Health: Learning-to-Learn (Meta-learning) as a Next Generation of Deep Learning Exploring Healthcare Challenges and Solutions for Rare Disorders: A Systematic Analysis}. Archives of Computational Methods in Engineering.

\bibitem{wang2023srr}
Peiqi Wang, Jingde Li, Shubei Wang, et al. (2023). \textit{A new meta-transfer learning method with freezing operation for few-shot bearing fault diagnosis}. Measurement science and technology.

\bibitem{liang2023zzh}
Xiaoxia Liang, M. Zhang, Guojin Feng, et al. (2023). \textit{A Novel Deep Model with Meta-learning for Rolling Bearing Few-shot Fault Diagnosis}. Journal of Dynamics Monitoring and Diagnostics.

\bibitem{tian2023iyh}
Pinzhuo Tian, and Shaorong Xie (2023). \textit{An Adversarial Meta-Training Framework for Cross-Domain Few-Shot Learning}. IEEE transactions on multimedia.

\bibitem{bian2024041}
Wanyu Bian, Albert Jang, and Fang Liu (2024). \textit{Multi-task Magnetic Resonance Imaging Reconstruction using Meta-learning}. Magnetic Resonance Imaging.

\bibitem{schwarz2022jfu}
Jonathan Schwarz, and Y. Teh (2022). \textit{Meta-Learning Sparse Compression Networks}. Trans. Mach. Learn. Res..

\bibitem{son2023lda}
Jaehyeon Son, Soochan Lee, and Gunhee Kim (2023). \textit{When Meta-Learning Meets Online and Continual Learning: A Survey}. IEEE Transactions on Pattern Analysis and Machine Intelligence.

\bibitem{wang2024d09}
Hui Wang, Zhiwei Han, Xufan Wang, et al. (2024). \textit{Contrastive Learning-Based Bayes-Adaptive Meta-Reinforcement Learning for Active Pantograph Control in High-Speed Railways}. IEEE Transactions on Transportation Electrification.

\bibitem{rao20232e3}
Haocong Rao, Cyril Leung, and Chun Miao (2023). \textit{Hierarchical Skeleton Meta-Prototype Contrastive Learning with Hard Skeleton Mining for Unsupervised Person Re-Identification}. International Journal of Computer Vision.

\bibitem{zhang2023jz8}
Yiqiang Zhang, Jiaxing Che, Yijun Hu, et al. (2023). \textit{Real-Time Ocean Current Compensation for AUV Trajectory Tracking Control Using a Meta-Learning and Self-Adaptation Hybrid Approach}. Italian National Conference on Sensors.

\bibitem{xiu2023ga8}
Xi Xiu, Jialun Li, Yujie Long, et al. (2023). \textit{MRLCC: an adaptive cloud task scheduling method based on meta reinforcement learning}. Journal of Cloud Computing.

\bibitem{cai2023ro7}
Shaokang Cai, Dezhi Han, and Dun Li (2023). \textit{A Feedback Semi-Supervised Learning With Meta-Gradient for Intrusion Detection}. IEEE Systems Journal.

\bibitem{lee20230j8}
Suyoung Lee, Myungsik Cho, and Young-Jin Sung (2023). \textit{Parameterizing Non-Parametric Meta-Reinforcement Learning Tasks via Subtask Decomposition}. Neural Information Processing Systems.

\bibitem{guarino2023zsq}
Idio Guarino, Chao Wang, A. Finamore, et al. (2023). \textit{Many or Few Samples?: Comparing Transfer, Contrastive and Meta-Learning in Encrypted Traffic Classification}. Traffic Monitoring and Analysis.

\bibitem{aqeel2025zql}
Muhammad Aqeel, Shakiba Sharifi, Marco Cristani, et al. (2025). \textit{Towards Real Unsupervised Anomaly Detection Via Confident Meta-Learning}. arXiv.org.

\bibitem{lee2024snq}
Soochan Lee, Hyeonseong Jeon, Jaehyeon Son, et al. (2024). \textit{Learning to Continually Learn with the Bayesian Principle}. International Conference on Machine Learning.

\bibitem{li2023asx}
Zhongshen Li, Junru Jin, Wenjia He, et al. (2023). \textit{CoraL: interpretable contrastive meta-learning for the prediction of cancer-associated ncRNA-encoded small peptides}. Briefings Bioinform..

\bibitem{schmidgall20238t4}
Samuel Schmidgall, and Joe Hays (2023). \textit{Meta-SpikePropamine: learning to learn with synaptic plasticity in spiking neural networks}. Frontiers in Neuroscience.

\bibitem{wang2023x5w}
Weijie Wang, Hong Zhao, Yikun Yang, et al. (2023). \textit{Few-shot short utterance speaker verification using meta-learning}. PeerJ Computer Science.

\bibitem{wang2023ryf}
Caiyong Wang, Haiqing Li, Wen-bin Ma, et al. (2023). \textit{MetaScleraSeg: an effective meta-learning framework for generalized sclera segmentation}. Neural computing & applications (Print).

\bibitem{li2023fhe}
Wenmei Li, Qing Liu, Yu Zhang, et al. (2023). \textit{Few-Shot Hyperspectral Image Classification Using Meta Learning and Regularized Finetuning}. IEEE Transactions on Geoscience and Remote Sensing.

\bibitem{yang20238th}
Jin Yang, Liang Bao, Wenjing Liu, et al. (2023). \textit{On a Meta Learning-Based Scheduler for Deep Learning Clusters}. IEEE Transactions on Cloud Computing.

\bibitem{raja2023hco}
K. Raja, and S. Kannimuthu (2023). \textit{Deep learning-based feature selection and prediction system for autism spectrum disorder using a hybrid meta-heuristics approach}. Journal of Intelligent & Fuzzy Systems.

\bibitem{fahim2023jsu}
M. Fahim, Vishal Sharma, R. Hunter, et al. (2023). \textit{Healthy Aging: A Deep Meta-Class Sequence Model to Integrate Intelligence in Digital Twin}. IEEE Journal of Translational Engineering in Health and Medicine.

\bibitem{hu2022y0i}
Cong Hu, Haoji Xu, and Xiaojun Wu (2022). \textit{Substitute Meta-Learning for Black-Box Adversarial Attack}. IEEE Signal Processing Letters.

\bibitem{wang2023kho}
Zhiqiang Wang, Man Li, H. Ou, et al. (2023). \textit{A Few-Shot Malicious Encrypted Traffic Detection Approach Based on Model-Agnostic Meta-Learning}. Security and Communication Networks.

\bibitem{li2023zn0}
Hui Li, Guimin Huang, Yiqun Li, et al. (2023). \textit{SEML: Self-Supervised Information-Enhanced Meta-learning for Few-Shot Text Classification}. International Journal of Computational Intelligence Systems.

\bibitem{chi20235vq}
Sunyi Chi, B. Dong, Yiming Xu, et al. (2023). \textit{APAM: Adaptive Pre-Training and Adaptive Meta Learning in Language Model for Noisy Labels and Long-Tailed Learning}. International Conference on Machine Learning and Applications.

\bibitem{manoharan2021r46}
Praveen Manoharan, Malini Pooni Venkat, S. Nagarathinam, et al. (2021). \textit{Learn to chill: intelligent chiller scheduling using meta-learning and deep reinforcement learning}. International Conference on Systems for Energy-Efficient Built Environments.

\bibitem{visca20217nt}
Marco Visca, R. Powell, Yang Gao, et al. (2021). \textit{Deep Meta-Learning Energy-Aware Path Planner for Unmanned Ground Vehicles in Unknown Terrains}. IEEE Access.

\bibitem{so2021y48}
Chaehan So (2021). \textit{Exploring Meta Learning: Parameterizing the Learning-to-learn Process for Image Classification}. Digital Signal Processing and Signal Processing Education Workshop.

\bibitem{yang2022oxf}
Zhixiong Yang, Jingyuan Xia, T. Qiu, et al. (2022). \textit{A Meta-Learning based Gradient Descent Algorithm for MU-MIMO Beamforming}. International Conference on Wireless Communications and Signal Processing.

\bibitem{gupta2021fbg}
Akash Gupta, P. Jonnalagedda, B. Bhanu, et al. (2021). \textit{Ada-VSR: Adaptive Video Super-Resolution with Meta-Learning}. ACM Multimedia.

\bibitem{zheng2021dx4}
Wenbo Zheng, Lan Yan, Fei-Yue Wang, et al. (2021). \textit{Learning from the Web: Webly Supervised Meta-Learning for Masked Face Recognition}. 2021 IEEE/CVF Conference on Computer Vision and Pattern Recognition Workshops (CVPRW).

\bibitem{zhang2021sbz}
Peng Zhang, Chenbing Liu, Xingshuo Chang, et al. (2021). \textit{Metric-based Meta-Learning Model for Few-Shot PolSAR Image Terrain Classification}. Radar.

\bibitem{wang2024jzu}
Chien-Yao Wang, I-Hau Yeh, and Hongpeng Liao (2024). \textit{YOLOv9: Learning What You Want to Learn Using Programmable Gradient Information}. European Conference on Computer Vision.

\bibitem{zhu20249wq}
Fenghao Zhu, Xinquan Wang, Chongwen Huang, et al. (2024). \textit{Robust Beamforming for RIS-Aided Communications: Gradient-Based Manifold Meta Learning}. IEEE Transactions on Wireless Communications.

\bibitem{yaghoubi2024j56}
Elaheh Yaghoubi, Elnaz Yaghoubi, Ahmed Khamees, et al. (2024). \textit{A systematic review and meta-analysis of artificial neural network, machine learning, deep learning, and ensemble learning approaches in field of geotechnical engineering}. Neural computing & applications (Print).

\bibitem{ren20242qj}
Lei Ren, Tingyu Mo, and Xuejun Cheng (2024). \textit{Meta-learning Based Domain Generalization Framework for Fault Diagnosis With Gradient Aligning and Semantic Matching}. IEEE Transactions on Industrial Informatics.

\bibitem{ik20248rp}
Gültekin Işık, and Ishak Paçal (2024). \textit{Few-shot classification of ultrasound breast cancer images using meta-learning algorithms}. Neural computing & applications (Print).

\bibitem{weilenmann2024ve2}
C. Weilenmann, A. Ziogas, T. Zellweger, et al. (2024). \textit{Single neuromorphic memristor closely emulates multiple synaptic mechanisms for energy efficient neural networks}. Nature Communications.

\bibitem{xia2024qx2}
Jingyuan Xia, Zhixiong Yang, Shengxi Li, et al. (2024). \textit{Blind Super-Resolution via Meta-Learning and Markov Chain Monte Carlo Simulation}. IEEE Transactions on Pattern Analysis and Machine Intelligence.

\bibitem{zhao2024f4b}
Haoming Zhao, Liang Ou, Ziming Zhang, et al. (2024). \textit{The value of deep learning-based X-ray techniques in detecting and classifying K-L grades of knee osteoarthritis: a systematic review and meta-analysis}. European Radiology.

\bibitem{li20242rv}
Jiawen Li, Tao Zhou, and Haoyan Cui (2024). \textit{Brain-Inspired Deep Meta-Reinforcement Learning for Active Coordinated Fault-Tolerant Load Frequency Control of Multi-Area Grids}. IEEE Transactions on Automation Science and Engineering.

\bibitem{dong2024110}
Zheng Dong, Renhe Jiang, Haotian Gao, et al. (2024). \textit{Heterogeneity-Informed Meta-Parameter Learning for Spatiotemporal Time Series Forecasting}. Knowledge Discovery and Data Mining.

\bibitem{liao2024o1z}
Huaizhang Liao, Jingyuan Xia, Zhixiong Yang, et al. (2024). \textit{Meta-Learning Based Domain Prior With Application to Optical-ISAR Image Translation}. IEEE transactions on circuits and systems for video technology (Print).

\bibitem{naskar202446a}
Gourab Naskar, Sk Mohiuddin, Samir Malakar, et al. (2024). \textit{Deepfake detection using deep feature stacking and meta-learning}. Heliyon.

\bibitem{zhang2024a5a}
Ruiyi Zhang, Rushi Qiang, Sai Ashish Somayajula, et al. (2024). \textit{AutoLoRA: Automatically Tuning Matrix Ranks in Low-Rank Adaptation Based on Meta Learning}. North American Chapter of the Association for Computational Linguistics.

\bibitem{sharma2024zlw}
Nelson Sharma, Aswini Ghosh, R. Misra, et al. (2024). \textit{Deep Meta Q-Learning Based Multi-Task Offloading in Edge-Cloud Systems}. IEEE Transactions on Mobile Computing.

\bibitem{ouyang2024xj0}
Tiancheng Ouyang, Yingying Su, Chengchao Wang, et al. (2024). \textit{Combined Meta-Learning With CNN-LSTM Algorithms for State-of-Health Estimation of Lithium-Ion Battery}. IEEE transactions on power electronics.

\bibitem{wang2025zze}
Changdong Wang, Zhou Shu, Jingli Yang, et al. (2025). \textit{Learning to Imbalanced Open Set Generalize: A Meta-Learning Framework for Enhanced Mechanical Diagnosis}. IEEE Transactions on Cybernetics.

\bibitem{chia2024ltk}
Yu-Hsin Chia, Wei-Hao Liao, S. Vyas, et al. (2024). \textit{In Vivo Intelligent Fluorescence Endo‐Microscopy by Varifocal Meta‐Device and Deep Learning}. Advancement of science.

\bibitem{li20246fg}
Wenzhong Li, Xiang Li, Yeting Xu, et al. (2024). \textit{MetaABR: A Meta-Learning Approach on Adaptative Bitrate Selection for Video Streaming}. IEEE Transactions on Mobile Computing.

\bibitem{yang2024rh9}
Jing Yang, and Xiaomin Wang (2024). \textit{Meta-learning with deep flow kernel network for few shot cross-domain remaining useful life prediction}. Reliability Engineering & System Safety.

\bibitem{gao20242uv}
Yuan Gao, Wei Chen, Fajun Li, et al. (2024). \textit{Meta‐Attention Deep Learning for Smart Development of Metasurface Sensors}. Advancement of science.

\bibitem{huang2024hlo}
Hai-Hui Huang, Jun Shu, and Yong Liang (2024). \textit{MUMA: A Multi-Omics Meta-Learning Algorithm for Data Interpretation and Classification}. IEEE journal of biomedical and health informatics.

\bibitem{liu2024hko}
Shicheng Liu, and Minghui Zhu (2024). \textit{Meta Inverse Constrained Reinforcement Learning: Convergence Guarantee and Generalization Analysis}. International Conference on Learning Representations.

\bibitem{li2024x2t}
Jiawen Li, and Yuanyuan Cheng (2024). \textit{Deep Meta-Reinforcement Learning-Based Data-Driven Active Fault Tolerance Load Frequency Control for Islanded Microgrids Considering Internet of Things}. IEEE Internet of Things Journal.

\bibitem{khattar2024sr6}
Vanshaj Khattar, Yuhao Ding, Bilgehan Sel, et al. (2024). \textit{A CMDP-within-online framework for Meta-Safe Reinforcement Learning}. International Conference on Learning Representations.

\bibitem{yen2024cxp}
Tzu-Yun Yen, Chan-Shien Ho, Yueh-peng Chen, et al. (2024). \textit{Diagnostic Accuracy of Deep Learning for the Prediction of Osteoporosis Using Plain X-rays: A Systematic Review and Meta-Analysis}. Diagnostics.

\bibitem{you2024xuq}
Linlin You, Qiyang Chen, Haohao Qu, et al. (2024). \textit{FMGCN: Federated Meta Learning-Augmented Graph Convolutional Network for EV Charging Demand Forecasting}. IEEE Internet of Things Journal.

\bibitem{hao2024dyp}
Xiaoyang Hao, Zhixi Feng, Tongqing Peng, et al. (2024). \textit{Meta-Learning Guided Label Noise Distillation for Robust Signal Modulation Classification}. IEEE Internet of Things Journal.

\bibitem{lang20246m8}
Shujun Lang, Xu Liu, Mingliang Zhou, et al. (2024). \textit{A Full-Reference Image Quality Assessment Method via Deep Meta-Learning and Conformer}. IEEE transactions on broadcasting.

\bibitem{xu2024ywn}
Meng Xu, Yi Mei, Fangfang Zhang, et al. (2024). \textit{Niching Genetic Programming to Learn Actions for Deep Reinforcement Learning in Dynamic Flexible Scheduling}. IEEE Transactions on Evolutionary Computation.

\bibitem{ding2024jjo}
Yang Ding, Heng Zhang, and Ting Qiu (2024). \textit{Deep learning approach to predict autism spectrum disorder: a systematic review and meta-analysis}. BMC Psychiatry.

\bibitem{wang2024dai}
Fengxiang Wang, Wanrong Huang, Shaowu Yang, et al. (2024). \textit{Learning to Learn Better Visual Prompts}. AAAI Conference on Artificial Intelligence.

\bibitem{nussenbaum2024z82}
Kate Nussenbaum, and Catherine A. Hartley (2024). \textit{Understanding the development of reward learning through the lens of meta-learning}. Nature Reviews Psychology.

\bibitem{alsaleh2024vdv}
Aqilah M. Alsaleh, Eid Albalawi, A. Algosaibi, et al. (2024). \textit{Few-Shot Learning for Medical Image Segmentation Using 3D U-Net and Model-Agnostic Meta-Learning (MAML)}. Diagnostics.

\bibitem{ghassemi20241ek}
Sina Ghassemi, Tianyi Zhang, Ward van Breda, et al. (2024). \textit{Unsupervised Multimodal Learning for Dependency-Free Personality Recognition}. IEEE Transactions on Affective Computing.

\bibitem{li20254kd}
Jiawen Li, and Tao Zhou (2025). \textit{Efficient Replay Deep Meta-Reinforcement Learning for Active Fault-Tolerant Control of Solid Oxide Fuel Cell Systems Considering Multivariable Coordination}. IEEE Transactions on Transportation Electrification.

\bibitem{zhao202420b}
Jianjin Zhao, Qi Li, Yueping Hong, et al. (2024). \textit{MetaRockETC: Adaptive Encrypted Traffic Classification in Complex Network Environments via Time Series Analysis and Meta-Learning}. IEEE Transactions on Network and Service Management.

\bibitem{zhang2024xg0}
Lei Zhang, Yuhang Zhou, Yi Yang, et al. (2024). \textit{Meta Invariance Defense Towards Generalizable Robustness to Unknown Adversarial Attacks}. IEEE Transactions on Pattern Analysis and Machine Intelligence.

\bibitem{ruwurm2024806}
Marc Rußwurm, Sherrie Wang, B. Kellenberger, et al. (2024). \textit{Meta-learning to address diverse Earth observation problems across resolutions}. Communications Earth &amp; Environment.

\bibitem{jang2024pyi}
Jungik Jang, Jisung Pyo, Young-il Yoon, et al. (2024). \textit{Meta-Transformer: A Meta-Learning Framework for Scalable Automatic Modulation Classification}. IEEE Access.

\bibitem{su2024h1g}
Hao Su, Qingtao Yao, Ling Xiang, et al. (2024). \textit{Semi-Supervised Temporal Meta-Learning Framework for Wind Turbine Bearing Fault Diagnosis Under Limited Annotation Data}. IEEE Transactions on Instrumentation and Measurement.

\bibitem{li20246zp}
Yao Li, Dawei Yuan, Tao Zhang, et al. (2024). \textit{Meta-Learning for Multi-Family Android Malware Classification}. ACM Transactions on Software Engineering and Methodology.

\bibitem{lupu20249p4}
E. Lupu, Fengze Xie, James A. Preiss, et al. (2024). \textit{MAGICVFM-Meta-Learning Adaptation for Ground Interaction Control With Visual Foundation Models}. IEEE Transactions on robotics.

\bibitem{briscik2024cpd}
Mitja Briscik, Gabriele Tazza, M. Dillies, et al. (2024). \textit{Supervised multiple kernel learning approaches for multi-omics data integration}. BioData Mining.

\bibitem{sun2024kbv}
Jingmin Sun, Zecheng Zhang, and Hayden Schaeffer (2024). \textit{LeMON: Learning to Learn Multi-Operator Networks}. arXiv.org.

\bibitem{eghbali2024huh}
Y. Eghbali, Shiva Kazemi Taskou, Mohammad Robat Mili, et al. (2024). \textit{Providing URLLC Service in Multi-STAR-RIS Assisted and Full-Duplex Cellular Wireless Systems: A Meta-Learning Approach}. IEEE Communications Letters.

\bibitem{zhu2024ok7}
Yihuan Zhu, Yunan Liu, Chunpeng Wang, et al. (2024). \textit{Intermediate Domain-Based Meta Learning Framework for Adaptive Object Detection}. IEEE transactions on circuits and systems for video technology (Print).

\bibitem{long202400t}
Jianyu Long, Rongxin Zhang, Yibin Chen, et al. (2024). \textit{A Customized Meta-Learning Framework for Diagnosing New Faults From Unseen Working Conditions With Few Labeled Data}. IEEE/ASME transactions on mechatronics.

\bibitem{gu20252u3}
Yan Gu, Zhaoze Liu, Shuhong Dai, et al. (2025). \textit{Deep Reinforcement Learning for Job Scheduling and Resource Management in Cloud Computing: An Algorithm-Level Review}. arXiv.org.

\bibitem{zhang2024mf0}
Lin Zhang, Bo Zhang, Botian Shi, et al. (2024). \textit{Few-Shot Cross-Domain Object Detection With Instance-Level Prototype-Based Meta-Learning}. IEEE transactions on circuits and systems for video technology (Print).

\bibitem{liu2024jz5}
Sheng Liu, Linlin You, Rui Zhu, et al. (2024). \textit{AFM3D: An Asynchronous Federated Meta-Learning Framework for Driver Distraction Detection}. IEEE transactions on intelligent transportation systems (Print).

\bibitem{ozkara2024nst}
Kaan Ozkara, Can Karakus, Parameswaran Raman, et al. (2024). \textit{MADA: Meta-Adaptive Optimizers through hyper-gradient Descent}. International Conference on Machine Learning.

\bibitem{cui2024bov}
Qi Cui, Weixuan Tang, Zhili Zhou, et al. (2024). \textit{Meta Security Metric Learning for Secure Deep Image Hiding}. IEEE Transactions on Dependable and Secure Computing.

\bibitem{wang2024so2}
Yuan Wang, Zhiqiang Tao, and Yi Fang (2024). \textit{A Unified Meta-Learning Framework for Fair Ranking With Curriculum Learning}. IEEE Transactions on Knowledge and Data Engineering.

\bibitem{ma2024vk4}
Han Ma, Baoyu Fan, B. Ng, et al. (2024). \textit{VL-Meta: Vision-Language Models for Multimodal Meta-Learning}. Mathematics.

\bibitem{amorim20240xf}
Lucas B V de Amorim, George D. C. Cavalcanti, and Rafael M. O. Cruz (2024). \textit{Meta-Scaler: A Meta-Learning Framework for the Selection of Scaling Techniques}. IEEE Transactions on Neural Networks and Learning Systems.

\bibitem{shen2024hea}
Maohao Shen, J. J. Ryu, Soumya Ghosh, et al. (2024). \textit{Are Uncertainty Quantification Capabilities of Evidential Deep Learning a Mirage?}. Neural Information Processing Systems.

\bibitem{ferrini20249g0}
Francesco Ferrini, Antonio Longa, Andrea Passerini, et al. (2024). \textit{A Self-Explainable Heterogeneous GNN for Relational Deep Learning}. Trans. Mach. Learn. Res..

\bibitem{wang2024tpb}
Chenxing Wang (2024). \textit{Towards Effective Fusion and Forecasting of Multimodal Spatio-temporal Data for Smart Mobility}. International Conference on Information and Knowledge Management.

\bibitem{song2024epb}
Wen-qi Song, Di Wu, Weiming Shen, et al. (2024). \textit{Early fault detection for rolling bearings: A meta‐learning approach}. IET Collaborative Intelligent Manufacturing.

\bibitem{wang20245h1}
Zih-Wei Wang, and Vincent W. S. Wong (2024). \textit{Bayesian Meta-Learning for Adaptive Traffic Prediction in Wireless Networks}. IEEE Transactions on Mobile Computing.

\bibitem{tam2024a1h}
S. Tam, S. T. P. Raghu, Étienne Buteau, et al. (2024). \textit{Towards Robust and Interpretable EMG-based Hand Gesture Recognition using Deep Metric Meta Learning}. arXiv.org.

\bibitem{qiao2024l71}
Ruihua Qiao, Tao Jiang, and Weiyong Yu (2024). \textit{Meta-Learning-Based Fronthaul Compression for Cloud Radio Access Networks}. IEEE Transactions on Wireless Communications.

\bibitem{xing2024n9q}
Yun Xing, Qing Guo, Xiaofeng Cao, et al. (2024). \textit{MetaRepair: Learning to Repair Deep Neural Networks from Repairing Experiences}. ACM Multimedia.

\bibitem{cheng2024mky}
Yingsong Cheng, Xinya Wang, Yong Ma, et al. (2024). \textit{General Hyperspectral Image Super-Resolution via Meta-Transfer Learning}. IEEE Transactions on Neural Networks and Learning Systems.

\bibitem{xia20246dc}
Pengcheng Xia, Yixiang Huang, Chengliang Liu, et al. (2024). \textit{Learn to Supervise: Deep Reinforcement Learning-Based Prototype Refinement for Few-Shot Motor Fault Diagnosis}. IEEE Transactions on Neural Networks and Learning Systems.

\bibitem{yang20243gt}
Fan Yang, Nor Azman Ismail, Y. Y. Pang, et al. (2024). \textit{A Systematic Literature Review of Deep Learning Approaches for Sketch-Based Image Retrieval: Datasets, Metrics, and Future Directions}. IEEE Access.

\bibitem{zhang2024ycr}
Lianwei Zhang, Dongjiang Niu, Beiyi Zhang, et al. (2024). \textit{Property-Guided Few-Shot Learning for Molecular Property Prediction With Dual-View Encoder and Relation Graph Learning Network}. IEEE journal of biomedical and health informatics.

\bibitem{raymond202441h}
Christian Raymond (2024). \textit{Meta-Learning Loss Functions for Deep Neural Networks}. arXiv.org.

\bibitem{khalid2024pss}
Mudassar Khalid, C. Pluempitiwiriyawej, S. Wangsiripitak, et al. (2024). \textit{The Applications of Deep Learning in ECG Classification for Disease Diagnosis: A Systematic Review and Meta-Data Analysis}. Engineering Journal.

\bibitem{kumar2024he9}
Sandeep Kumar, Amit Sharma, Vikrant Shokeen, et al. (2024). \textit{Meta-learning for real-world class incremental learning: a transformer-based approach}. Scientific Reports.

\bibitem{kukanov20249bs}
Ivan Kukanov, Janne Laakkonen, Tomi Kinnunen, et al. (2024). \textit{Meta-Learning Approaches For Improving Detection of Unseen Speech Deepfakes}. Spoken Language Technology Workshop.

\bibitem{chen20245h8}
Cheng Chen, Hao Fang, Yuxiao Yang, et al. (2024). \textit{Model-agnostic meta-learning for EEG-based inter-subject emotion recognition}. Journal of Neural Engineering.

\bibitem{liu2024az5}
Yonghao Liu, Mengyu Li, Ximing Li, et al. (2024). \textit{Meta-GPS++: Enhancing Graph Meta-Learning with Contrastive Learning and Self-Training}. ACM Transactions on Knowledge Discovery from Data.

\bibitem{xu2024mf9}
Zexing Xu, Linjun Zhang, Sitan Yang, et al. (2024). \textit{F-FOMAML: GNN-Enhanced Meta-Learning for Peak Period Demand Forecasting with Proxy Data}. arXiv.org.

\bibitem{chen2024b4d}
Weiye Chen, Yiqun Xie, Xiaowei Jia, et al. (2024). \textit{Referee-Meta-Learning for Fast Adaptation of Locational Fairness}. AAAI Conference on Artificial Intelligence.

\bibitem{wen2024xmk}
Gang Wen, and Limin Li (2024). \textit{MMOSurv: meta-learning for few-shot survival analysis with multi-omics data}. Bioinformatics.

\bibitem{liao2024jm9}
Yang Liao, Jing Zhao, Jiyong Bian, et al. (2024). \textit{From mechanism to application: Decrypting light‐regulated denitrifying microbiome through geometric deep learning}. iMeta.

\bibitem{meng2024nqq}
Wei Meng, Hao Ju, Tongxu Ai, et al. (2024). \textit{Transferring Meta-Policy From Simulation to Reality via Progressive Neural Network}. IEEE Robotics and Automation Letters.

\bibitem{wu2024v0z}
Haotian Wu, Maojun Zhang, Yulin Shao, et al. (2024). \textit{MIMO Channel as a Neural Function: Implicit Neural Representations for Extreme CSI Compression in Massive MIMO Systems}. arXiv.org.

\bibitem{farrell2024mpy}
M. Farrell, Nicolas Le Guillarme, Liam Brierley, et al. (2024). \textit{The changing landscape of text mining: a review of approaches for ecology and evolution}. Proceedings B.

\bibitem{ma20243e9}
Xin Ma, Yuanchang Xie, and Chunxiao Chigan (2024). \textit{Graph Convolutional Network Based Multi-Objective Meta-Deep Q-Learning for Eco-Routing}. IEEE transactions on intelligent transportation systems (Print).

\bibitem{gven2024a3n}
Eray Güven, and G. Kurt (2024). \textit{Learning-to-Learn the Wave Angle Estimation}. IEEE Transactions on Communications.

\bibitem{wang2024bhk}
Jingyao Wang, Wenwen Qiang, Jiangmeng Li, et al. (2024). \textit{Rethinking Meta-Learning from a Learning Lens}. arXiv.org.

\bibitem{pu2024m1b}
Q. Pu, Youkun Chen, Mu Zhou, et al. (2024). \textit{Bayesian Meta-Learning: Toward Fast Adaptation in Neural Network Positioning Techniques}. IEEE Internet of Things Journal.

\end{thebibliography}

\end{document}