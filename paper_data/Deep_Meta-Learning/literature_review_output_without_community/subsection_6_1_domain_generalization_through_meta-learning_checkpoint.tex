\subsection{Domain Generalization through Meta-Learning}
The problem of Domain Generalization (DG) poses a significant challenge for artificial intelligence systems: models must exhibit robust performance on entirely unseen target domains without any access to their data during training. This contrasts with Domain Adaptation, where some target domain data is available, and highlights the need for truly generalizable learning principles. Meta-learning offers a powerful paradigm to address this by enabling models to "learn to learn" generalizable principles rather than acquiring domain-specific knowledge, thereby facilitating the extraction of domain-invariant features or the development of robust adaptation strategies across diverse source domains \cite{khoee2024ksk}.

The core idea behind meta-learning for DG is to simulate domain shifts during the meta-training phase. By constructing meta-tasks where the "meta-training" data comes from a subset of available source domains and the "meta-testing" data comes from another, the meta-learner is forced to optimize for generalization across these simulated shifts. This process encourages the model to learn features or adaptation mechanisms that are robust to distributional changes, rather than overfitting to specific source domain characteristics.

A foundational approach in this area is Meta-Learning for Domain Generalization (MLDG) \cite{li2018learning}. MLDG adapts the Model-Agnostic Meta-Learning (MAML) framework to the DG problem. It trains a meta-learner by splitting the available source domains into a meta-training set and a meta-test set. In the inner loop, the model adapts to data from a domain within the meta-training set. In the outer loop, the meta-parameters are updated based on the model's performance on a domain from the meta-test set. This episodic training strategy explicitly optimizes for a model initialization that can quickly adapt to a new, unseen domain, effectively learning to generalize across domain shifts. This mechanism directly addresses the DG objective by promoting the learning of features and parameters that are inherently transferable and less sensitive to specific domain characteristics.

The field's understanding of meta-learning for DG has been significantly structured by recent comprehensive analyses. \cite{khoee2024ksk} provides the first dedicated survey on Domain Generalization through Meta-Learning, introducing a novel taxonomy and a decision graph to classify various meta-learning methodologies tailored for DG. This work systematically highlights how meta-learning approaches are specifically designed to handle out-of-distribution (OOD) data. Their taxonomy categorizes methods based on strategies for feature extraction (e.g., learning domain-invariant features, disentangling domain-specific factors) and classifier learning (e.g., learning robust classifiers, adapting classifiers). This structured view clarifies that meta-learning for DG often involves learning a feature extractor that is robust to domain shifts, or a meta-learner that can quickly adapt a base model to a new domain without seeing any data from it during meta-training.

Beyond MLDG's optimization-based approach, other meta-learning strategies have been leveraged for DG. Feature-based meta-learning methods aim to learn representations that are inherently invariant to domain shifts. This can be achieved through various techniques, such as adversarial training, where a domain discriminator attempts to distinguish between features from different domains, and the feature extractor is trained to fool this discriminator, thereby producing domain-agnostic features.

An example of combining adversarial training with meta-learning for cross-domain generalization is presented by \cite{tian2023iyh}. This work introduces an adversarial meta-training framework specifically for cross-domain few-shot learning, where the challenge is to generalize to new classes from unseen domains. While it addresses few-shot learning, its core mechanism directly tackles the generalization to *unseen domains*. The framework utilizes a max-min episodic iteration: in the maximization phase, it dynamically generates "hard" pseudo-tasks that simulate challenging domain shifts, pushing the meta-learner to its generalization limits. In the minimization phase, the meta-learning model is trained to learn robust, cross-domain meta-knowledge that performs well on these difficult tasks. This adversarial approach forces the meta-learner to extract features and learn adaptation strategies that are resilient to significant distributional changes, thereby improving generalization across diverse and novel domains.

Despite its promise, meta-learning for DG faces several challenges. A critical limitation is the requirement for sufficient diversity among source domains during meta-training to ensure that the learned meta-knowledge truly generalizes to novel, unseen domains, rather than merely overfitting to the distribution of source domains. The performance can be sensitive to the magnitude of distributional shift between training and target domains. Furthermore, the computational cost associated with meta-training, particularly for optimization-based methods, can be substantial. Theoretical guarantees for OOD performance in meta-DG settings are also an active area of research, as understanding *why* certain meta-learning strategies lead to better generalization is crucial for developing more robust systems.

In conclusion, meta-learning has emerged as a powerful paradigm for addressing Domain Generalization by enabling models to learn how to extract domain-invariant features or develop robust adaptation strategies. From foundational methods like MLDG that simulate domain shifts during meta-training to advanced adversarial meta-training frameworks, the field is continuously developing sophisticated techniques. By focusing on learning generalizable principles rather than domain-specific knowledge, meta-learning significantly enhances the ability of AI systems to generalize effectively to novel contexts. Future directions include developing more sophisticated theoretical guarantees for OOD performance, exploring meta-learning's role in mitigating bias across domains, and further integrating it with emerging foundation models to tackle even more complex and unpredictable real-world domain shifts.