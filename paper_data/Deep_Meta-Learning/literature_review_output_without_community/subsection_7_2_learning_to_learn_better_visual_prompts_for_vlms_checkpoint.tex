\subsection{Learning to Learn Better Visual Prompts for VLMs}
The advent of large Vision-Language Models (VLMs) has revolutionized multimodal AI, yet their efficient and robust adaptation to novel downstream tasks remains a significant challenge. A primary concern is the tendency of lightweight adaptation strategies, such as prompt tuning, to severely overfit to base classes, leading to suboptimal generalization performance on unseen data \cite{wang2024dai}. Meta-learning, a paradigm focused on "learning to learn" effective adaptation strategies, offers a powerful solution by enabling VLMs to acquire new visual concepts rapidly with minimal fine-tuning, thereby unlocking their full potential for few-shot learning in multimodal contexts \cite{hospedales2020m37}. This approach is particularly vital for foundation models, where full fine-tuning is computationally prohibitive, making parameter-efficient (tuning only prompts) and sample-efficient (few-shot adaptation) methods indispensable.

Building upon the foundational principles of optimization-based meta-learning (as discussed in Section 2.1 and 3.1), which involve learning adaptable initializations or optimization processes, recent work has applied these concepts to VLM prompt tuning. The core idea is to meta-optimize the prompt generation mechanism itself, rather than just the prompts for a single task. This aligns with the bilevel programming framework for hyperparameter optimization and meta-learning \cite{franceschi2018u1q}, where prompt parameters can be viewed as meta-parameters optimized at an outer level to guide an inner-loop adaptation process for specific tasks.

A prominent example of this integration is LoL (Learning to Learn) by \cite{wang2024dai}, which directly addresses the overfitting problem in VLM prompt tuning. LoL integrates the classic N-way K-shot episodic training strategy, a cornerstone of meta-learning for few-shot adaptation, into the visual prompt tuning process. In its two-stage approach, LoL first fine-tunes prompts on base classes using a standard Context Optimization (CoOp) method. Subsequently, a meta-learning stage further refines these prompts by sampling multiple classification tasks from the base classes in an N-way K-shot manner. During this meta-learning stage, the model performs inner-loop gradient updates on a support set to adapt to a specific task, and then outer-loop updates on a query set to meta-optimize the prompt generation strategy across tasks. This meta-optimization explicitly trains the system to generate prompts that generalize better from base classes to novel ones, effectively mitigating the severe overfitting observed in traditional prompt tuning methods. The success of LoL highlights how meta-gradients \cite{sutton2022jss} can be leveraged to optimize higher-level learning processes, in this case, the prompt generation itself.

Beyond LoL, other meta-learning approaches are emerging to enhance VLMs. \cite{ma2024vk4} introduced VL-Meta, a framework for multimodal meta-learning designed for Vision-Language Models, particularly for tasks like Visual Question Answering (VQA). VL-Meta employs light model structures such as vision-language and multimodal fusion mappers to efficiently utilize pre-trained models and map features into a shared space. Crucially, it constructs a meta-task pool from small amounts of data to improve generalization and proposes token-level training with a multi-task fusion loss. While LoL focuses on improving generalization for classification by mitigating overfitting in prompt tuning, VL-Meta emphasizes broader multimodal understanding and efficiency in low-resource VQA settings, demonstrating diverse applications of meta-learning in VLMs.

Despite these advancements, challenges remain. The generalization capabilities of meta-learned prompts, especially in cross-domain few-shot learning scenarios, are critical. Methods like the adversarial meta-training framework proposed by \cite{tian2023iyh} for general few-shot learning could inspire future work on making VLM prompts more robust to domain shifts by dynamically generating challenging pseudo-tasks during meta-training. Furthermore, the issue of meta-overfitting or underfitting, depending on task complexity, can still affect meta-learning systems \cite{wang2024bhk}. To address this, \cite{wang2024bhk} introduced TRLearner, which calibrates meta-learning by leveraging task relation matrices and relation-aware consistency regularization. Applying such principles to VLM prompt tuning could lead to more robust and generalizable prompts by ensuring that prompts adapted to similar tasks exhibit consistent performance, thereby reducing excess risk and improving generalization.

It is also important to critically assess the efficacy of complex meta-learning strategies. While episodic training is powerful, some findings, such as those from the NeurIPS 2021 MetaDL challenge \cite{baz2022n78}, suggest that sometimes simpler backbone fine-tuning, when coupled with excellent representations, can be highly competitive or even outperform intricate meta-learning setups for few-shot image classification. This underscores the need for careful consideration of the trade-offs between meta-learning complexity and the inherent quality of the VLM's pre-trained representations.

In conclusion, meta-learning provides a powerful paradigm for enabling VLMs to adapt efficiently and generalize effectively, particularly by optimizing the prompt tuning process. By integrating techniques like N-way K-shot episodic training and exploring advanced meta-optimization frameworks, researchers are making significant strides in mitigating overfitting and enhancing generalization from base to novel classes. Future research should continue to explore more sophisticated meta-learning architectures for prompt generation, potentially integrating insights from task relation-aware regularization \cite{wang2024bhk} and adversarial meta-training \cite{tian2023iyh} to further enhance the robustness and flexibility of VLMs in diverse, real-world multimodal scenarios.