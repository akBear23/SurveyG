\subsection{Rethinking Meta-Learning: Calibration and Task Relations}

Achieving robust generalization across diverse tasks remains a central challenge in meta-learning, frequently hampered by issues of underfitting and overfitting during the meta-training process. Early efforts in few-shot learning, such as the approach by \cite{bertinetto2018ur2}, focused on efficient adaptation by integrating differentiable closed-form solvers like Ridge Regression as base learners. While this method offered computational efficiency and adaptability by jointly optimizing a deep feature extractor with the base learner's hyperparameters, its reliance on primarily linear models as solvers did not fully address the complex calibration issues inherent in deep, gradient-based meta-learning frameworks.

The advent of optimization-based meta-learning, exemplified by methods like Model-Agnostic Meta-Learning (MAML), introduced a powerful paradigm for learning adaptable initializations. However, these methods presented their own set of challenges. Theoretical investigations, such as those by \cite{bernacchia20211r0}, began to unravel the intricate dynamics of MAML's inner-loop optimization, revealing complexities like the potential for optimal negative learning rates. This theoretical understanding hinted at the delicate balance required for effective adaptation. Practically, MAML-like approaches often faced limitations, including insufficient weight modification during adaptation and significant computational overhead, as highlighted by \cite{przewiezlikowski2022d4y} with their HyperMAML framework. HyperMAML sought to mitigate these issues by replacing the gradient-based inner loop with a trainable Hypernetwork, demonstrating a shift towards learned, non-gradient-based adaptation mechanisms to improve efficiency and performance. Despite these advancements, a fundamental problem persisted: how to consistently calibrate the meta-learning process to prevent underfitting to complex task distributions or overfitting to specific training tasks, thereby ensuring robust generalization to unseen tasks.

A recent advancement that fundamentally rethinks this process is presented by \cite{wang2024bhk}, which proposes a novel "Learning Lens" to address the critical issues of underfitting and overfitting in meta-learning. This work introduces a new conceptualization of the meta-learning function $F_\theta$, modeling it not as a monolithic entity but as a combination of initialization layers and a distinct 'meta-layer' implemented via gradient optimization. This explicit decomposition allows for a more granular understanding and control over the meta-learning process. To calibrate this refined model, \cite{wang2024bhk} introduces TRLearner, a plug-and-play method designed to enhance generalization and reduce excess risk. TRLearner achieves this by leveraging task relation matrices, which capture the similarities and differences between tasks within a given distribution. It then applies relation-aware consistency regularization, a mechanism that guides the meta-optimization process to ensure that models trained on similar tasks yield consistent predictions, while allowing for divergence on dissimilar tasks. This principled approach to understanding and exploiting task similarities leads to improved generalization, reduced excess risk, and a more robust performance across a diverse range of tasks.

The work by \cite{wang2024bhk} represents a significant step forward by moving beyond ad-hoc solutions to underfitting and overfitting. By providing a novel conceptualization of the meta-learner and introducing a calibration mechanism grounded in task relations, it offers a more robust and theoretically informed pathway to building meta-learning systems that can generalize effectively. However, challenges remain in the scalability of constructing and leveraging task relation matrices for extremely large and diverse task distributions, and in extending this calibration framework to other meta-learning paradigms beyond optimization-based methods. Future research could explore more efficient ways to infer task relations dynamically and integrate similar calibration principles into metric-based or model-based meta-learning approaches.