\subsection{Benchmarking and Challenges for Meta-Learning with Large Models}

The integration of meta-learning with large foundation models and embodied AI systems presents a distinct set of benchmarking methodologies and formidable practical challenges. Evaluating these complex, high-stakes environments demands not only specialized datasets and rigorous metrics for generalization and adaptation but also robust protocols to assess safety, reliability, and interpretability, all while contending with immense computational demands.

While foundational meta-learning benchmarks like Meta-Dataset \cite{triantafillou2019meta} (for few-shot image classification) and Meta-World \cite{yu2019meta} (for robotic manipulation) have been instrumental in advancing the field, their scale, task diversity, and fidelity often fall short for evaluating the unique challenges of modern large foundation models and embodied AI. For instance, Meta-World, while comprehensive for manipulation, may not fully capture the complexities of real-world physics or the vast array of tasks a generalist robot might encounter. The sheer scale of modern foundation models, often with billions of parameters, introduces immense computational demands for meta-training and subsequent adaptation. Early meta-learning efforts began to address these efficiency concerns; for example, Meta-Transfer Learning (MTL) \cite{sun2018iy7} enabled efficient adaptation of deep neural networks by learning scaling and shifting functions, mitigating overfitting on sparse data. Similarly, meta-learning with differentiable closed-form solvers like Ridge Regression \cite{bertinetto2018ur2} offered an efficient way to handle high-dimensional features, laying groundwork for today's large models. These foundational concerns around efficient adaptation become paramount when adapting models of unprecedented size. Recent work, such as LoL (Learning to Learn Better Visual Prompts) \cite{wang2024dai}, directly addresses the challenge of efficiently adapting large Vision-Language Models (VLMs) by integrating N-way K-shot episodic training into prompt tuning, thereby improving generalization and tackling overfitting. Similarly, VL-Meta \cite{ma2024vk4} proposes light model structures and meta-task pools to efficiently adapt VLMs for multimodal meta-learning, saving computational power and data. For large NLP models, transformer-based meta-learners are being developed for real-world class incremental learning, demonstrating significant generalization even without specific training for this task, highlighting the need for benchmarks that assess efficient, continuous adaptation of these massive models \cite{kumar2024he9, lee2021jou}.

Beyond computational efficiency, rigorous benchmarking for meta-learning with large models necessitates specialized datasets that accurately reflect real-world variability and robust metrics for out-of-distribution (OOD) generalization. These specialized datasets often require high-fidelity physics simulations, complex multi-modal inputs, long-tailed distributions of real-world phenomena, or diverse interaction patterns not typically found in academic benchmarks. The ability of a model to perform reliably on unseen data or tasks is paramount, especially when deploying large models in diverse real-world scenarios. \cite{khoee2024ksk} provides a comprehensive survey on meta-learning for Domain Generalization (DG), offering a novel taxonomy that systematizes the field and highlights how meta-learning approaches are specifically designed to handle OOD data. This underscores the need for structured evaluation protocols that go beyond in-distribution performance. Furthermore, evaluation metrics must extend beyond mere accuracy to encompass trustworthiness and interpretability. For instance, \cite{tam2024a1h} proposes a deep metric meta-learning framework for EMG-based hand gesture recognition that provides robust confidence estimation, moving towards critical metrics for dependable human-machine interaction. The challenge also extends to evaluating the meta-learning process itself, particularly when the learning objective is not fixed. \cite{xu2020txy} introduces FRODO, an algorithm that meta-learns its own objective function online, demonstrated on tasks like Atari games. Benchmarking such adaptive learning systems requires metrics that can assess the effectiveness and generalization of the *learned* objective function across diverse tasks, pushing the boundaries of traditional performance evaluation.

The challenges become particularly acute in embodied AI and safety-critical systems, where real-time adaptation, stability, and provable guarantees are non-negotiable evaluation criteria. Benchmarking in these domains requires specialized environments and metrics that capture dynamic interactions and potential risks. \cite{oconnell2022twd} exemplifies this with Neural-Fly, a meta-learning approach enabling rapid online adaptation for UAVs in strong, dynamic winds. Evaluated in a real-weather wind tunnel, Neural-Fly demonstrates precise flight control, robustness guarantees, and the ability to extrapolate to unseen wind conditions, showcasing a specialized benchmarking methodology for embodied AI. Extending this, \cite{khattar2024sr6} introduces a novel "CMDP-within-online" framework for Meta-Safe Reinforcement Learning. This work is critical for high-stakes environments as it provides the first provable guarantees for constraint satisfaction and task-averaged optimality in meta-safe RL, establishing essential metrics for safety evaluation. The ability to learn new skills from limited demonstrations is also crucial; \cite{finn20174c4} demonstrated one-shot visual imitation learning via MAML, enabling robots to acquire complex skills from single visual demonstrations, highlighting the need for benchmarks that assess rapid skill transfer in physical systems. Furthermore, addressing the "reality gap" between simulation and real-world deployment is a key benchmarking challenge. \cite{meng2024nqq} proposes MetaPNN, combining meta-RL and progressive neural networks for sim-to-real transfer, showing improved learning efficiency and performance on real robot tasks, which necessitates evaluation protocols that rigorously test this transferability. Finally, \cite{lupu20249p4} presents MAGICVFM, an innovative approach for autonomous ground vehicle control that integrates Visual Foundation Models (VFMs) with offline meta-learning and online composite adaptive control. This system offers mathematical guarantees for exponential stability and robustness during real-time terrain adaptation, directly addressing the need for reliable and safe operation of large models in dynamic embodied AI systems. These examples underscore the shift towards specialized, high-fidelity benchmarking that prioritizes real-world applicability and safety \cite{beck2023x24}.

In conclusion, benchmarking meta-learning with large models and embodied AI systems is evolving rapidly, driven by the imperative for robust, reliable, and safe AI. The literature highlights a clear shift towards specialized evaluation methodologies, including real-world simulation environments and the development of metrics that encompass not only performance but also generalization, OOD robustness, interpretability, and provable safety guarantees. Key open research questions remain in developing standardized, scalable benchmarks that accurately reflect real-world variability, creating diverse and representative datasets for foundation model adaptation, and establishing comprehensive evaluation protocols to guide the development of truly robust and trustworthy meta-learning systems for unprecedented scale and autonomy.