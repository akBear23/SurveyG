\documentclass[12pt,a4paper]{article}
    \usepackage[utf8]{inputenc}
    \usepackage[T1]{fontenc}
    \usepackage{amsmath,amsfonts,amssymb}
    \usepackage{graphicx}
    \usepackage[margin=2.5cm]{geometry}
    \usepackage{setspace}
    \usepackage{natbib}
    \usepackage{url}
    \usepackage{hyperref}
    \usepackage{booktabs}
    \usepackage{longtable}
    \usepackage{array}
    \usepackage{multirow}
    \usepackage{wrapfig}
    \usepackage{float}
    \usepackage{colortbl}
    \usepackage{pdflscape}
    \usepackage{tabu}
    \usepackage{threeparttable}
    \usepackage{threeparttablex}
    \usepackage[normalem]{ulem}
    \usepackage{makecell}
    \usepackage{xcolor}

    % Set line spacing
    \doublespacing

    % Configure hyperref
    \hypersetup{
        colorlinks=true,
        linkcolor=blue,
        filecolor=magenta,      
        urlcolor=cyan,
        citecolor=red,
    }

    % Title and author information
    \title{A Comprehensive Literature Review with Self-Reflection}
    \author{Literature Review}
    \date{\today}

    \begin{document}

    \maketitle

    % Abstract (optional)
    \begin{abstract}
    This literature review provides a comprehensive analysis of recent research in the field. The review synthesizes findings from 146 research papers, identifying key themes, methodological approaches, and future research directions.
    \end{abstract}

    \newpage
    \tableofcontents
    \newpage

    \label{sec:introduction}

\section{Introduction}
\label{sec:introduction}

\subsection{Evolution of Language Models and the Need for RL}
\label{sec:1\_1\_evolution\_of\_language\_models\_\_and\_\_the\_need\_for\_rl}

The landscape of natural language processing has undergone a profound transformation, evolving from rudimentary statistical methods to highly sophisticated neural architectures. This evolution has progressively endowed language models with remarkable generative capabilities, yet it has simultaneously exposed a critical gap between raw generative power and the nuanced, human-centric objectives essential for real-world utility and safety.

Early language models were predominantly statistical, relying on N-gram probabilities to predict the next word based on a limited preceding context. While foundational, these models suffered from severe sparsity issues and an inability to capture long-range dependencies. The advent of Recurrent Neural Networks (RNNs), particularly Long Short-Term Memory (LSTMs) and Gated Recurrent Units (GRUs), marked a significant leap, enabling models to process sequential data and learn more complex contextual patterns. However, RNNs struggled with parallelization and very long sequences. The paradigm truly shifted with the introduction of the Transformer architecture \cite{community\_16}, which leveraged self-attention mechanisms to process entire sequences in parallel and capture arbitrarily long-range dependencies efficiently. This innovation paved the way for the development of massive pre-trained language models (LLMs) like GPT-3 \cite{community\_28}, which exhibited emergent capabilities such as few-shot learning, where models can perform new tasks with minimal examples, and in-context reasoning, allowing them to follow complex instructions provided within the prompt \cite{layer\_1}.

Despite these monumental advancements, traditional training objectives, primarily maximum likelihood estimation (MLE) through next-token prediction, inherently optimize for fluency and coherence based on the statistical properties of the training data distribution. While effective for generating grammatically correct and contextually plausible text, this objective does not intrinsically align LLMs with subjective human preferences such as helpfulness, harmlessness, honesty, or precise instruction-following \cite{layer\_1, community\_28}. The core limitation lies in the fact that MLE aims to reproduce the observed data distribution, which may contain biases, undesirable patterns, or simply lack explicit signals for complex human values. Consequently, an LLM trained solely on MLE might generate fluent but factually incorrect information (hallucinations), produce biased or toxic content, or fail to adhere to nuanced user instructions, even when its generative capacity is high. The escalating scale and complexity of LLMs, coupled with their increasing deployment in user-facing applications, magnified this "alignment problem." Larger models, while more capable, also possess a greater potential to generate convincing but misaligned outputs, making the discrepancy between raw generative power and desired behavioral alignment critically apparent. Achieving objectives like "be helpful and harmless" or "answer truthfully" requires optimizing for criteria that are often subjective, non-differentiable, and cannot be easily encoded into a simple next-token prediction loss. This critical need necessitated a paradigm shift from mere next-token prediction towards optimizing for these nuanced, often non-differentiable, human-centric objectives.

This fundamental mismatch between the capabilities of large generative models and the intricate, often implicit, expectations of human users set the stage for the integration of reinforcement learning (RL) as a mechanism to bridge this critical gap. RL provides a powerful framework for optimizing policies against arbitrary reward signals, even those derived from complex human feedback or non-differentiable metrics. By allowing models to learn through trial and error, guided by a reward function that encapsulates desired behaviors, RL offers a pathway to instill human values and precise instruction-following capabilities into LLMs, thereby enhancing their real-world utility, safety, and trustworthiness. This foundational shift from purely generative optimization to behavior-driven alignment marks a pivotal moment in the development of language AI.
\subsection{Scope of Reinforcement Learning in Language Processing}
\label{sec:1\_2\_scope\_of\_reinforcement\_learning\_in\_language\_processing}

Reinforcement Learning (RL) has rapidly emerged as a transformative paradigm for advancing Large Language Models (LLMs), fundamentally expanding their capabilities beyond mere text generation to sophisticated instruction-following, complex reasoning, and agentic behavior. This review delineates the expansive scope of RL in language processing (RL for LP), charting its evolution and diverse applications across various stages of language model development and deployment. We will explore how RL addresses the inherent limitations of traditional supervised learning, particularly in aligning LLMs with nuanced human preferences and real-world objectives \cite{kaufmann2023hlw, sun2023awe}.

The landscape of RL for LP can be broadly categorized into several interconnected areas, which this review will systematically explore. Initially, we will examine the application of deep reinforcement learning for the \textbf{direct optimization of language generation tasks} (Section 3). This includes overcoming challenges like exposure bias and directly optimizing for non-differentiable, sequence-level metrics in tasks such as dialogue generation and summarization. This early phase laid the groundwork for more advanced alignment techniques by demonstrating RL's capacity to optimize for global objectives rather than local token predictions.

A pivotal development in the field is the critical role of RL in \textbf{fine-tuning and aligning LLMs with human feedback (RLHF)} for instruction-following and adherence to complex values (Sections 4 and 5). This paradigm, exemplified by methods like Proximal Policy Optimization (PPO) \cite{zheng2023c98} and Direct Preference Optimization (DPO) \cite{rafailov20239ck}, has become the de facto standard for making LLMs helpful, harmless, and honest. We will delve into the core mechanisms of RLHF, its theoretical underpinnings \cite{xiong2023klt}, and subsequent advancements that simplify the alignment process \cite{zhou202469n, zhu2024zs2} or mitigate the "alignment tax" where foundational capabilities are degraded \cite{lu202435m}.

Furthermore, the review will explore innovative approaches that move \textbf{beyond direct human feedback} to enhance scalability and reduce annotation costs (Section 6). This includes Reinforcement Learning from AI Feedback (RLAIF) \cite{lee2023mrw}, where LLMs themselves generate preference labels, and principle-driven self-alignment strategies that enable models to learn from minimal human-written rules and self-generated data \cite{sun20238m7}. We will also discuss the burgeoning area of automated reward design, where LLMs or other AI systems autonomously craft executable reward functions or infer preferences from implicit signals \cite{cao2024jcn}.

The scope of RL for LP also extends to significantly \textbf{enhancing reasoning and complex problem-solving capabilities} in LLMs (Section 7.1). By providing structured feedback on multi-step thought processes, RL enables models to tackle challenging domains like mathematics \cite{anand2024rnl} and logical reasoning \cite{fu2023pyr, yang202393p}, moving beyond superficial pattern matching towards more robust and verifiable solutions. This includes leveraging RL to foster \textbf{agentic behavior} (Section 7.2), empowering LLMs to effectively use external tools, interact with dynamic environments, and orchestrate complex workflows. We will also highlight its impact on \textbf{domain-specific applications} (Section 7.3), such as software engineering and code generation \cite{wei2025v4d}, where functional correctness and adherence to strict rules are paramount.

Finally, the discussion will extend to the \textbf{critical challenges and practical considerations} that govern the real-world deployment of RL-driven language systems (Section 8). This encompasses advancements in \textbf{efficiency and scalability} through system-level optimizations and parameter-efficient fine-tuning (PEFT) \cite{sheng2024sf5}. We will also address crucial issues of \textbf{privacy, security, and robustness} against adversarial attacks \cite{liu2024avj}, as well as the ongoing need for rigorous \textbf{evaluation and interpretability} of reward models and policies \cite{codaforno20242yf}. These practical considerations are vital for ensuring the responsible, reliable, and trustworthy development of RL-driven language AI.

In summary, this review will chart the transformative journey of reinforcement learning in language processing, from its foundational applications in direct generation to its pivotal role in aligning LLMs with human values, enhancing their cognitive abilities, and enabling agentic behavior. By systematically examining these areas and the practical challenges they entail, we aim to provide a comprehensive understanding of RL's indispensable role in shaping the future of capable, robust, and human-aligned language AI.


\label{sec:foundational_concepts_and_early_paradigms}

\section{Foundational Concepts and Early Paradigms}
\label{sec:foundational\_concepts\_\_and\_\_early\_paradigms}

\subsection{Core Reinforcement Learning Principles}
\label{sec:2\_1\_core\_reinforcement\_learning\_principles}

Reinforcement Learning (RL) provides a formal mathematical framework for an autonomous \textit{agent} to learn optimal behavior through trial-and-error interaction with an \textit{environment}. This sequential decision-making process is typically modeled as a Markov Decision Process (MDP), defined by a tuple $(\mathcal{S}, \mathcal{A}, \mathcal{P}, \mathcal{R}, \gamma)$ \cite{sutton2018reinforcement}. Here, $\mathcal{S}$ represents the set of possible \textit{states} describing the environment at any given time, and $\mathcal{A}$ is the set of \textit{actions} the agent can take. $\mathcal{P}(s'|s,a)$ denotes the transition probability from state $s$ to $s'$ upon taking action $a$, and $\mathcal{R}(s,a,s')$ is the immediate numerical \textit{reward signal} received after such a transition. The discount factor $\gamma \in [0,1)$ balances the importance of immediate versus future rewards. In the context of language generation, for instance, the state $s$ can be the sequence of tokens generated thus far, an action $a$ is the choice of the next token, and the reward $\mathcal{R}$ might be a score like ROUGE or human feedback received only upon completion of the full sequence. The agent's overarching objective is to learn an optimal \textit{policy} $\pi(a|s)$, which maps states to probabilities of selecting each action, such that the expected cumulative discounted reward, or \textit{return}, is maximized over time.

Fundamental to understanding RL algorithms are value functions, which quantify the desirability of states or state-action pairs. The state-value function $V^\pi(s)$ represents the expected return starting from state $s$ and following policy $\pi$, while the action-value function $Q^\pi(s,a)$ represents the expected return starting from state $s$, taking action $a$, and then following policy $\pi$. These value functions are intrinsically linked by the Bellman equations, which express a recursive relationship between the value of a state and the values of its successor states, forming the bedrock for many RL algorithms \cite{sutton2018reinforcement}.

Based on these value functions, RL algorithms broadly fall into value-based, policy-based, and actor-critic categories. \textit{Value-based methods}, such as Q-learning \cite{watkins1992q} and SARSA, aim to directly learn the optimal action-value function $Q^*(s,a)$, from which an optimal policy can be derived by selecting actions greedily. While highly effective in tabular settings or with discrete, finite state-action spaces, these methods struggle with the continuous or vast state-action spaces characteristic of modern language models, where explicitly representing or learning all Q-values becomes intractable.

For such complex environments, \textit{policy gradient methods} are often preferred, as they directly optimize the policy parameters by estimating the gradient of the expected return. The REINFORCE algorithm, introduced by \textcite{Williams1992}, is a seminal example. It estimates the policy gradient using Monte Carlo rollouts, where the gradient is proportional to the product of the policy's log-probability and the observed return from that point onward. While theoretically sound, REINFORCE often suffers from high variance in its gradient estimates, which can lead to unstable and slow learning, especially in environments with long horizons or sparse rewards. This high variance stems from using the raw cumulative return as the only feedback signal for each action.

To mitigate the high variance inherent in pure policy gradient methods, \textit{actor-critic architectures} emerged as a powerful alternative. These methods combine two components: an "actor" that learns and updates the policy, and a "critic" that learns a value function to evaluate the actor's actions. The critic's role is to provide a baseline for the policy gradient, typically by estimating the state-value function $V^\pi(s)$ or the action-value function $Q^\pi(s,a)$. By subtracting this baseline from the observed return, the actor can compute an \textit{advantage function}, $A^\pi(s,a) = Q^\pi(s,a) - V^\pi(s)$, which measures how much better an action $a$ is compared to the average action taken in state $s$ under policy $\pi$. This advantage function serves as a more stable and informative signal for updating the policy, significantly reducing the variance of the gradient estimates and leading to more stable and efficient learning \cite{sutton2018reinforcement}. Advanced actor-critic algorithms, such as Proximal Policy Optimization (PPO) \cite{schulman2017proximal}, have become standard in modern deep RL applications due to their balance of stability and performance. PPO achieves this by employing a clipped surrogate objective function that constrains policy updates, preventing destructively large changes and improving training stability. The relationship between such policy optimization methods and direct preference optimization (DPO) has also been a subject of recent analysis, highlighting their shared underlying principles \cite{su2025mld}.

The application of RL principles is inherently challenged by several fundamental problems. The \textit{exploration-exploitation dilemma} requires the agent to balance trying out new, potentially better actions (exploration) with leveraging actions known to yield high rewards (exploitation). An agent that explores too little might get stuck in sub-optimal local optima, while one that explores too much might waste valuable resources. This trade-off is particularly acute in large, complex state-action spaces, such as those encountered in generating diverse and coherent language. Theoretical work on KL-regularized RL, often used in LLM alignment, has shown promise in achieving logarithmic regret, indicating improved sample efficiency in online settings by leveraging the benign optimization landscape induced by KL-regularization \cite{zhao202532z}.

Another critical challenge is the \textit{credit assignment problem}, which arises when rewards are sparse or delayed \cite{sutton2018reinforcement}. In language generation, for instance, a single reward (e.g., a BLEU score or human preference) is often provided only at the end of a long sequence of generated tokens. This makes it difficult to determine which specific tokens or phrases contributed positively or negatively to the final outcome. For example, in a multi-turn dialogue, a single poor word choice early on can derail an entire conversation, but a final conversation-level reward signal provides no direct information on which of the dozens of preceding actions was at fault \cite{shani202491k}.

To address the credit assignment problem, general RL techniques like \textit{reward shaping} and \textit{options} have been developed. Reward shaping involves augmenting the environment's sparse reward function with additional, more frequent, and informative auxiliary rewards. When designed carefully, using potential-based functions, reward shaping can provably preserve the optimal policy of the original MDP, ensuring that the agent's learning objective remains aligned with the true task \cite{ng1999potential}. Recent work has explored leveraging Large Language Models (LLMs) themselves to automate reward shaping by providing dense, intermediate-step rewards \cite{cao2024jcn} or by redistributing holistic sequence-level rewards to token-level signals, interpreting this as a form of potential-based reward shaping \cite{li2024h19}. Similarly, LLMs have been used to decompose tasks into subgoals and assess their achievement, thereby generating auxiliary rewards and facilitating credit assignment in a zero-shot manner \cite{pignatelli2024ffp}.

Options, on the other hand, allow an agent to learn and execute temporally extended actions, effectively creating a hierarchical structure that can simplify the decision-making process and provide more immediate sub-goal feedback \cite{sutton1999between}. This concept has been adapted for language models through "macro actions" (MA-RLHF), where the policy optimization occurs at the level of sequences of tokens rather than individual tokens, significantly reducing the temporal distance between actions and rewards and improving learning efficiency \cite{chai2024qal}. Furthermore, novel approaches like Inverse-Q* aim to perform token-level RL without explicit reward models or preference data, by directly estimating an optimal policy and using its probabilities for fine-grained credit assignment \cite{xia2024rab}. These mechanisms are crucial for making RL tractable in environments where direct feedback is scarce and long-term dependencies are prevalent.

In summary, the core principles of reinforcement learning, encompassing the MDP framework, value functions, policy gradient methods, and actor-critic architectures, provide a robust foundation for sequential decision-making. However, the inherent challenges of exploration-exploitation and credit assignment necessitate sophisticated algorithmic solutions like variance reduction techniques (e.g., advantage functions, PPO's clipped objective) and reward engineering strategies (e.g., reward shaping, macro actions). These foundational concepts and their associated challenges form the essential theoretical and algorithmic background for understanding how RL is applied to complex domains, including language tasks, which often present vast state spaces, long action sequences, and nuanced reward signals.
\subsection{Structured Prediction as Sequential Decision Making}
\label{sec:2\_2\_structured\_prediction\_as\_sequential\_decision\_making}

Traditional natural language processing (NLP) tasks frequently necessitate the generation of complex, structured outputs, ranging from syntactic parse trees and semantic role labels to coherent sequences in machine translation or summarization. Before the widespread adoption of deep learning, a pivotal conceptual leap connected these structured prediction problems with the sequential decision-making framework of reinforcement learning (RL). This perspective reframed the construction of a complex output not as a monolithic prediction, but as a series of interdependent local decisions made by a learned policy. This allowed for the theoretical optimization of global, non-local objectives, moving beyond the limitations of independent token-level predictions and laying the crucial groundwork for generating more coherent and contextually appropriate structured outputs \cite{community\_14}.

Early pioneering work established this paradigm by framing structured prediction as a search problem, often leveraging "learning to search" or imitation learning techniques. In this context, an agent learns a policy that maps states (partial outputs or configurations) to actions (local decisions that extend or modify the structure). A significant contribution was the "Search-based Structured Prediction" (SSP) framework, formalized by \cite{Daume2009} (often referred to as SEARN). SEARN demonstrated how to convert structured prediction problems into a sequence of cost-sensitive classification tasks, where a learned policy iteratively makes decisions to construct the output. For instance, in dependency parsing, a policy might sequentially decide between actions like \texttt{SHIFT} (to move the next word onto a stack) or \texttt{REDUCE} (to form a dependency arc between words on the stack), with the goal of constructing a valid parse tree that maximizes a global F1-score. This approach often relied on imitation learning, where the policy learned to mimic the decisions of an "expert oracle" during a search process to build the desired structure. The broader context and theoretical underpinnings of these learning-to-search approaches within structured learning for NLP were further elaborated by \cite{Chang2015}, highlighting their potential to handle complex interdependencies in output structures across tasks such as sequence labeling, information extraction, and coreference resolution. Other notable works in this era applied similar principles to tasks like semantic parsing \cite{Zettlemoyer2007} and machine translation \cite{Liang2006}, where the generation of a target structure was decomposed into a series of choices guided by a learned policy.

While foundational, these early methods, particularly those heavily reliant on imitation learning, faced significant limitations that underscored the challenges of applying RL to complex linguistic tasks. A primary issue was \textit{exposure bias}, where models trained on expert demonstrations (e.g., using "teacher forcing" during training) struggled when encountering states not explicitly seen in the expert's trajectories during inference. When the model made a mistake and deviated from the expert's path, it entered an unfamiliar state, leading to compounding errors and a degradation in output quality. This problem, inherent to imitation learning, was later addressed more formally in the broader machine learning community by algorithms such as DAgger (Dataset Aggregation) \cite{Ross2011}. DAgger iteratively collected data from the learned policy's own trajectories and retrained on it, thereby gradually reducing the distribution mismatch between training and inference. However, even with DAgger, these approaches often required an expert oracle to provide optimal actions, which could be difficult or impossible to define for truly novel or open-ended tasks, especially as output complexity increased. Furthermore, the direct optimization of global, non-differentiable metrics, while a key conceptual advantage, was often hampered by the inherent challenges of basic policy gradient methods like REINFORCE \cite{Williams1992}, which are known for high variance and sample inefficiency. This made training unstable and convergence slow, especially for long sequences with sparse rewards, where feedback was only available at the end of a generation. The fundamental difficulty of designing and quantifying effective reward functions for complex language tasks, where a single scalar reward could not capture the nuanced contributions of individual tokens or sub-sequences, further exacerbated these issues.

In conclusion, the re-conceptualization of structured prediction as sequential decision-making, spearheaded by learning-to-search and imitation learning paradigms, was a critical historical development in NLP. This period established the foundational understanding that complex linguistic outputs could be generated through a series of learned actions, and that global, non-differentiable objectives could be directly optimized using reinforcement learning principles. Despite the inherent limitations of exposure bias, reliance on expert demonstrations, and the high variance and sparse rewards of early policy gradient methods, this era laid the essential conceptual and algorithmic groundwork. It clearly demonstrated the potential of RL to move beyond independent token-level predictions, thereby paving the way for the subsequent wave of deep reinforcement learning applications. The challenges identified during this period, such as the credit assignment problem over long sequences and the need for more efficient and fine-grained reward signals, directly motivated the development of more robust and scalable deep reinforcement learning techniques for language generation, which would be explored in the subsequent sections.


\label{sec:deep_reinforcement_learning_for_direct_language_generation}

\section{Deep Reinforcement Learning for Direct Language Generation}
\label{sec:deep\_reinforcement\_learning\_for\_direct\_language\_generation}

\subsection{Addressing Exposure Bias and Optimizing Non-Differentiable Metrics}
\label{sec:3\_1\_addressing\_exposure\_bias\_\_and\_\_optimizing\_non-differentiable\_metrics}

The application of deep reinforcement learning (DRL) marked a pivotal shift in sequence generation tasks, moving beyond the limitations of purely supervised learning to directly optimize for end-to-end, non-differentiable metrics that better reflect overall generation quality. Supervised sequence-to-sequence models, typically trained with maximum likelihood estimation (MLE), suffer from a critical discrepancy known as 'exposure bias' \cite{Ranzato2016, Li2016}. During training, these models are exposed only to ground truth sequences (teacher forcing), but at inference time, they must generate tokens conditioned on their own potentially erroneous previous outputs. This discrepancy leads to a compounding of errors and degraded performance in unseen contexts. Furthermore, token-level accuracy, the typical supervised objective, often fails to correlate with global sequence quality metrics like BLEU, ROUGE, or human judgments of coherence and relevance \cite{Papineni2002, Lin2004}.

To address these challenges, researchers began to frame sequence generation as a sequential decision-making problem, where a policy learns to select the next token to maximize a long-term, non-differentiable reward. Early work on directly optimizing non-differentiable metrics for structured prediction, such as Minimum Error Rate Training (MERT) for machine translation \cite{Och2003}, highlighted the need for such approaches. Conceptually, this aligned with the "learning to search" paradigm for structured prediction \cite{DaumeIII2009}, where a policy makes local decisions to construct a global output. Foundational reinforcement learning algorithms, such as REINFORCE \cite{Williams1992} and more general policy gradient methods with function approximation \cite{Sutton2000}, provided the theoretical bedrock for these applications. An early application of reinforcement learning to structured output prediction, using policy gradient methods to optimize task-specific metrics, further demonstrated the potential of this paradigm beyond traditional control problems \cite{Ranzato2007}.

The advent of deep learning enabled the integration of these RL principles with powerful neural architectures, leading to Deep Reinforcement Learning (DRL) for direct language generation. A significant contribution was the application of policy gradient methods to sequence-to-sequence models, directly optimizing non-differentiable metrics like BLEU \cite{Ranzato2016}. This approach explicitly mitigated exposure bias by allowing the model to explore its own generated sequences during training, thereby learning to recover from its own mistakes. Often, these methods employed a curriculum learning strategy, such as MIXER \cite{Ranzato2016}, which gradually transitioned from MLE to full RL training to stabilize the learning process and prevent catastrophic forgetting. Similarly, deep reinforcement learning was pioneered for dialogue generation, where policy gradient methods were employed to optimize for long-term, human-like dialogue quality metrics such as ease of answering, information flow, and coherence, which are inherently non-differentiable \cite{Li2016}. This work demonstrated how DRL could enable agents to generate more engaging and relevant responses by considering the overall conversational flow rather than just local token accuracy.

The utility of DRL extended to abstractive summarization, where models were trained using policy gradients to maximize ROUGE scores, a standard metric for summarization quality \cite{Paulus2017}. By combining supervised pre-training with reinforcement learning fine-tuning, these models could generate more coherent and relevant summaries, moving beyond extractive methods and improving the overall quality of the generated text. In neural machine translation, DRL fine-tuning was also adopted to directly optimize for BLEU scores, leading to improvements in translation quality in high-profile systems \cite{Wu2016}. Further applications of DRL for abstractive summarization and neural machine translation also explored the use of diversified reward functions to encourage more varied and high-quality outputs, addressing the issue of generic responses often produced by MLE-trained models \cite{Wang2018}. These methods collectively underscored DRL's capacity to generate text that is more globally coherent, relevant, and human-like by directly optimizing for end-to-end quality.

While policy gradient methods offered a powerful solution, they often suffered from high variance during training, which could impede learning stability. To address this, actor-critic algorithms were introduced for sequence prediction \cite{Bahdanau2017}. These methods combine a policy network (actor) with a value network (critic) that estimates the expected reward, thereby reducing variance and improving the stability and efficiency of training. A particularly influential method was Self-Critical Sequence Training (SCST) \cite{Rennie2017}, which used the reward of a baseline sequence (e.g., sampled greedily) to normalize the rewards of other sampled sequences, significantly reducing variance and making policy gradient methods more robust for optimizing non-differentiable metrics like BLEU and ROUGE. This allowed for more robust optimization and further enhanced the ability to mitigate exposure bias in various sequence generation tasks.

Despite these advancements, several significant challenges persisted within these early DRL applications for text generation. Policy gradient methods, including early actor-critic variants, were notoriously difficult to train, often exhibiting high variance and sensitivity to hyperparameters, which could lead to unstable learning and policy collapse \cite{zheng2023c98}. The computational burden of exploration was also substantial, requiring extensive sampling over vast action spaces (vocabulary) and long sequences, leading to excessive memory consumption and prolonged training times \cite{wang2023v62}. Furthermore, a fundamental hurdle was the problem of reward sparsity: meaningful rewards were typically only available at the end of a long sequence, making it difficult for the model to assign credit to individual tokens or intermediate decisions. This "temporal credit assignment problem" hindered efficient learning and made it challenging to discern which specific parts of a generation contributed positively or negatively to the final holistic reward \cite{li2024h19, cao2024lh3}. The inherent difficulty of designing effective, task-specific reward functions that truly captured nuanced aspects of human-like generation quality (e.g., creativity, factual correctness, safety) also remained a significant practical hurdle. These severe limitations in training stability, computational efficiency, and reward engineering ultimately motivated the search for more robust reward signals and training paradigms, paving the way for the subsequent, more sophisticated alignment techniques discussed in later sections.
\subsection{Challenges in Reward Design and Training Stability}
\label{sec:3\_2\_challenges\_in\_reward\_design\_\_and\_\_training\_stability}

Early applications of deep reinforcement learning (DRL) to language generation tasks faced significant and persistent challenges, primarily stemming from the inherent difficulty of designing effective reward functions, the problem of reward sparsity, and the instability of policy gradient training methods. These hurdles ultimately motivated the subsequent development of human-centric feedback mechanisms.

The theoretical foundations of reinforcement learning, particularly policy gradient methods like REINFORCE \cite{Williams1992} and their extension with function approximation \cite{Sutton2000}, provided a powerful framework for optimizing policies directly. However, these methods are notoriously susceptible to high variance, making training processes difficult and brittle, especially in complex, high-dimensional action spaces like language generation. Early attempts to apply RL to structured prediction, such as framing sequence labeling as a sequential decision-making problem using REINFORCE \cite{Ranzato2007}, explicitly highlighted these issues, often suffering from high variance and sample inefficiency. Similarly, the "learning to search" paradigm \cite{Daume2009, Chang2015}, which conceptualized structured prediction in NLP as a search process guided by a learned policy, frequently relied on imitation learning from expert demonstrations. While innovative, this approach often struggled to directly optimize for true long-term, non-differentiable rewards and could suffer from exposure bias, where the model performs poorly when encountering states not seen in expert demonstrations.

As DRL began to be directly applied to generative NLP tasks, these challenges became even more pronounced. Pioneering work in dialogue generation using DRL \cite{Li2016} explicitly identified reward sparsity, high variance during training, and the inherent difficulty of designing effective, task-specific reward functions as major obstacles. For instance, defining a differentiable and non-sparse reward signal that accurately captures nuanced qualities like conversational coherence, informativeness, or creativity in a long dialogue sequence proved exceedingly difficult. Feedback was often only available at the end of a generated sequence or dialogue turn, leading to sparse rewards that made credit assignment challenging for the learning agent.

Subsequent efforts to apply DRL to tasks like abstractive summarization \cite{Paulus2017} and further dialogue generation \cite{Li2017} continued to grapple with these issues. While these works demonstrated DRL's potential to optimize non-differentiable metrics like ROUGE scores, the reward functions themselves were often hand-crafted and could be incomplete or misaligned with human perceptions of quality. The challenge of designing rewards that accurately reflect complex output qualities such as factual correctness, safety, or stylistic preferences remained a significant bottleneck. To mitigate the instability of pure policy gradient methods, actor-critic algorithms were introduced for sequence prediction \cite{Bahdanau2017}, aiming to reduce variance and improve training stability by learning a critic to estimate the value function. Despite these advancements, the core problem of reward engineering persisted. For example, optimizing Neural Machine Translation (NMT) with a diversified reward function required careful construction of multi-faceted reward signals to encourage desired output properties beyond simple BLEU scores \cite{Wang2018}.

Comprehensive surveys of DRL in NLP \cite{Gao2019, Zhu2020, Liu2022} consistently highlighted these issues as central to the field's progress. They underscored that exposure bias, reward sparsity, the complexity of reward design, and the inherent training instability of DRL algorithms were pervasive problems across various language generation tasks. These challenges meant that DRL models often required extensive hyperparameter tuning, were sensitive to initialization, and could produce outputs that, while locally coherent, lacked global consistency or failed to meet complex human-centric criteria.

In conclusion, the early applications of deep reinforcement learning to language generation were significantly hampered by the intrinsic difficulties of crafting effective, non-sparse, and differentiable reward functions, especially for subjective qualities like creativity or safety. Coupled with the high variance and instability inherent in policy gradient methods, these practical hurdles made DRL training processes arduous and often brittle. The persistent nature of these challenges ultimately underscored the limitations of purely algorithmic reward design and paved the way for the exploration of more robust and human-aligned reward signals, such as those derived from human feedback, which are now central to the development of modern large language models \cite{yu2023xc4, sahoo2024dp6, holk20243vd}.


\label{sec:the_rlhf_paradigm:_aligning_llms_with_human_preferences}

\section{The RLHF Paradigm: Aligning LLMs with Human Preferences}
\label{sec:the\_rlhf\_paradigm:\_aligning\_llms\_with\_human\_preferences}

\subsection{Core RLHF Mechanisms and Early Breakthroughs}
\label{sec:4\_1\_core\_rlhf\_mechanisms\_\_and\_\_early\_breakthroughs}

The advent of Reinforcement Learning from Human Feedback (RLHF) marked a pivotal shift in Large Language Model (LLM) development, offering a robust solution to the inherent limitations of direct Deep Reinforcement Learning (DRL) for language generation, such as exposure bias and the difficulty of optimizing non-differentiable, human-centric metrics. This breakthrough enabled LLMs to move beyond mere statistical language modeling to reliably follow complex instructions, generate helpful and harmless responses, and align with nuanced human values, fundamentally enhancing their controllability and user-friendliness. The foundational RLHF process typically unfolds in three distinct stages.

The first stage, \textbf{Supervised Fine-Tuning (SFT)}, is crucial for imbuing the pre-trained LLM with basic instruction-following capabilities. In this phase, a pre-trained LLM is fine-tuned on a dataset of high-quality human-written demonstrations, where prompts are paired with desired responses. This initial supervised learning step provides a baseline policy that can generate reasonable and coherent text, serving as a stable starting point for the subsequent reinforcement learning phase and mitigating the challenges of sparse rewards in an open-ended generation environment.

The second stage involves training a \textbf{Reward Model (RM)}, which acts as a scalable proxy for nuanced human values and preferences. Instead of directly optimizing for complex, often subjective human criteria, a separate model is trained to predict human preferences. This is achieved by collecting a dataset of human comparisons, where annotators evaluate and rank multiple responses generated by the LLM for a given prompt \cite{Ziegler2019, Stiennon2020}. For instance, \cite{Ziegler2019} demonstrated the effectiveness of training a reward model on human preferences to fine-tune language models, while \cite{Stiennon2020} further applied this concept to abstractive summarization, showing how a reward model learned from human feedback could significantly improve summary quality. The RM, typically another LLM, learns to assign a scalar score to a generated response, reflecting its alignment with human-desired attributes like helpfulness, harmlessness, and honesty. This approach elegantly bypasses the challenge of explicitly defining and hand-crafting complex reward functions for every desired behavior.

Finally, the core of RLHF involves \textbf{fine-tuning the LLM policy using reinforcement learning algorithms}, most commonly Proximal Policy Optimization (PPO), guided by the trained reward model. In this stage, the LLM (now acting as the policy) generates responses to new prompts, and these responses are then scored by the RM. The policy is updated to maximize this reward signal, effectively learning to generate outputs that the RM predicts humans would prefer. The PPO algorithm, known for its stability and sample efficiency, is particularly well-suited for this task, allowing the LLM to iteratively refine its generation strategy \cite{Ouyang2022}. This iterative optimization process, as demonstrated by \cite{Lewis2020} for fine-tuning text-to-text transformers on various task-specific and non-differentiable metrics, enables the LLM to robustly follow complex instructions and align with human values. The culmination of this three-stage process was famously showcased by \cite{Ouyang2022} with InstructGPT (and subsequently ChatGPT), which leveraged RLHF to train LLMs that could follow instructions and exhibit helpful, harmless, and honest behavior, even outperforming much larger models fine-tuned purely with supervised methods.

This foundational RLHF paradigm marked a significant development in making LLMs more controllable, user-friendly, and ethically responsible. By integrating human preferences into the learning loop, RLHF effectively addressed the limitations of direct DRL, where designing explicit reward functions for subjective qualities was intractable. However, these early breakthroughs also highlighted new challenges, such as the high cost and potential biases of human data collection, the inherent difficulty of ensuring true alignment across diverse contexts, and the risk of "reward hacking," where the model exploits imperfections in the proxy reward model rather than genuinely improving its behavior. These emergent challenges subsequently became central to the next wave of research in RLHF.
\subsection{Simplified Alignment: Direct Preference Optimization (DPO) and Variants}
\label{sec:4\_2\_simplified\_alignment:\_direct\_preference\_optimization\_(dpo)\_\_and\_\_variants}

The initial success of Reinforcement Learning from Human Feedback (RLHF) in aligning Large Language Models (LLMs) was often hampered by its inherent complexity, computational expense, and training instability, particularly when relying on Proximal Policy Optimization (PPO) \cite{zheng2023c98}. This spurred a significant research direction towards more efficient, stable, and theoretically grounded alignment methods, with Direct Preference Optimization (DPO) emerging as a pivotal advancement \cite{wang2024a3a, srivastava2025gfw}.

\textbf{Direct Preference Optimization (DPO)} \cite{rafailov20239ck} revolutionized LLM alignment by re-parameterizing the RLHF objective into a simple classification loss. This elegant reformulation eliminated the need for an explicit reward model and the complex, multi-stage PPO training process, which typically involves sampling from the LLM during training. DPO achieves this by leveraging a novel parameterization of the reward model within the KL-constrained reward maximization objective, allowing the optimal policy to be extracted in closed form. The resulting objective is a straightforward binary cross-entropy loss, significantly reducing computational overhead, improving training stability, and making preference-based fine-tuning more accessible.

Despite DPO's simplification, the challenge of "reward overoptimization" or "reward hacking" persists, where models exploit imperfections in the implicit reward function, leading to performance degradation \cite{rafailov2024ohd}. Spurious correlations, such as length bias, can still influence DPO's implicit reward, causing models to generate longer, but not necessarily better, responses \cite{singhal2023egk}. To address these robustness concerns, several variants and alternative simplified approaches have emerged. Methods like \cite{zhai20238xc} introduce uncertainty-penalized RLHF using diverse reward LoRA ensembles to prevent policies from venturing into out-of-distribution regions where reward signals are unreliable. Similarly, \cite{dai2025ygq} proposes Behavior-Supported Regularization, which uses value regularization to restrict policy iteration to the in-distribution region of the reward model, thereby mitigating extrapolation errors. \cite{zhang2024esn} further improves this by introducing Adversarial Policy Optimization (AdvPO) with lightweight uncertainty estimation, a distributionally robust approach that avoids the computational overhead of ensemble methods.

Beyond robustness, other methods aim for greater theoretical grounding or broader applicability. \textbf{Self-Play with Adversarial Critic (SPAC)} \cite{ji2024d5f} offers a DPO-like single-timescale algorithm that provides provable convergence guarantees under weak data assumptions, addressing the lack of theoretical robustness in purely empirical methods. \textbf{Value-Incentivized Preference Optimization (VPO)} \cite{cen2024nef} presents a unified framework for both online and offline RLHF, implicitly incorporating uncertainty estimation into a DPO-like direct policy optimization, a feature often missing in DPO. Further simplifying the pipeline, \textbf{Variational Alignment with Re-weighting (VAR)} \cite{du2025zfp} reformulates RLHF as a variational inference problem, leading to a reward-driven re-weighted Supervised Fine-Tuning (SFT) loss. VAR ensures non-negative weights and a stable optimization landscape, directly comparing its enhanced stability and effectiveness against DPO.

The core DPO framework has also been extended to tackle more complex alignment objectives. While basic DPO handles single-objective alignment, real-world applications often demand balancing conflicting goals like helpfulness and safety. \textbf{Bi-Factorial Preference Optimization (BFPO)} \cite{zhang2024b6u} re-parameterizes a multi-objective RLHF problem (safety-helpfulness) into a single supervised learning objective, offering an efficient DPO-like solution for complex trade-offs. The issue of "preference collapse," where KL-regularized RLHF (including DPO) disproportionately favors dominant preferences, is identified by \cite{xiao2024ro4}, who propose Preference Matching to ensure diversity. For enhancing DPO's efficiency and adaptability, \textbf{Weighted Preference Optimization (WPO)} \cite{zhou202469n} improves DPO by dynamically reweighting preference pairs based on their likelihood under the current policy, effectively simulating on-policy learning with off-policy data and addressing the distributional gap.

The pursuit of efficiency and fine-grained control has also led to approaches that bridge DPO's simplicity with the token-level granularity of PPO. \textbf{Reinforced Token Optimization (RTO)} \cite{zhong2024wch} innovatively uses DPO to extract token-wise reward signals, which are then fed into a subsequent PPO training stage. This provides PPO with dense, fine-grained rewards, addressing the sparsity issue of traditional PPO implementations in RLHF. Similarly, \textbf{RED (REward reDistribution)} \cite{li2024h19} assigns token-level rewards by leveraging an off-the-shelf reward model to predict scores at each timestep, transforming sparse sequence rewards into dense, immediate token-level signals without modifying the reward model.

Other methods push the boundaries of simplification even further, moving towards tuning-free or inference-time alignment. \textbf{Contrastive Preference Learning (CPL)} \cite{hejna2023vyy} offers a broader "RL-free" approach, directly learning optimal policies from regret-based preferences using a supervised contrastive objective, applicable to general sequential decision-making. \textbf{Inverse-Q\textit{} \cite{xia2024rab} achieves token-level RL }without preference data or explicit reward models* by directly estimating an optimal policy, representing an extreme simplification. For iterative self-improvement, \textbf{DICE (self-alignment with DPO Implicit rEwards)} \cite{chen2024vkb} enables DPO-tuned models to iteratively self-align without external feedback by bootstrapping with their own implicit reward models. \textbf{Self-Exploring Language Models (SELM)} \cite{zhang2024lqf} introduces an optimistically biased objective for active exploration in online DPO, tackling the limitation of passive exploration.

These simplified and unified approaches collectively represent a significant advancement in the practical application of LLM alignment. While DPO laid the groundwork for efficient preference-based fine-tuning, subsequent research has focused on enhancing its robustness, theoretical guarantees, and applicability to multi-objective and fine-grained control, often by integrating implicit uncertainty, dynamic weighting, or novel re-parameterizations. The ongoing challenge lies in achieving these simplifications without sacrificing the nuanced understanding of human preferences, especially in complex, open-ended generative tasks, and ensuring that models remain robust against various forms of reward hacking and miscalibration.


\label{sec:advanced_rlhf:_addressing_challenges_and_enhancing_robustness}

\section{Advanced RLHF: Addressing Challenges and Enhancing Robustness}
\label{sec:advanced\_rlhf:\_addressing\_challenges\_\_and\_\_enhancing\_robustness}

\subsection{Mitigating Reward Model Imperfections and Overoptimization}
\label{sec:5\_1\_mitigating\_reward\_model\_imperfections\_\_and\_\_overoptimization}

The efficacy of Reinforcement Learning from Human Feedback (RLHF) hinges on the fidelity of its proxy reward models (RMs) to true human preferences. A critical challenge in LLM alignment is "reward hacking" or "overoptimization," where models exploit flaws, spurious correlations, or unintended shortcuts in these imperfect RMs, leading to misaligned or undesirable behaviors \cite{lambert2023c8q}. This section explores diagnostic insights and a range of mitigation strategies designed to ensure that aligned policies genuinely reflect complex human preferences rather than merely optimizing a flawed proxy.

Initial diagnostic studies revealed pervasive issues. \cite{singhal2023egk} empirically demonstrated that output length is a surprisingly significant spurious correlation, with reward models often favoring longer responses regardless of genuine quality, and that simple length-only optimization can reproduce most RLHF "improvements." Extending this, \cite{rafailov2024ohd} showed that even in Direct Alignment Algorithms (DAAs) like DPO \cite{rafailov20239ck}, performance degradation due to overoptimization occurs early in training, highlighting the brittleness of these methods. \cite{saito2023zs7} further identified a verbosity bias in AI-generated preferences, a common source of RM imperfection. Conceptually, \cite{lambert2023c8q} framed these issues as an "objective mismatch," where the numerical objectives of RM training, policy optimization, and downstream evaluation become decoupled, leading to unintended behaviors. Moreover, \cite{kirk20230it} and \cite{mohammadi20241pk} observed that while RLHF improves alignment, it can inadvertently reduce output diversity and creativity, suggesting an overoptimization towards a narrow, "safe" behavioral mode.

To counter these imperfections, research has focused on improving reward model quality and robustness. \cite{yu20249l0} enhanced RM quality and interpretability by leveraging self-generated, filtered critiques, moving beyond black-box scalar scores. Building on this, \cite{wang20247pw} introduced ArmoRM, a multi-objective reward model with a Mixture-of-Experts (MoE) scalarization, providing interpretable and steerable reward scores while explicitly mitigating verbosity bias. For complex tasks, \cite{lai2024ifx} proposed ALaRM, a hierarchical reward modeling framework that combines holistic and proactively selected aspect-specific rewards to address sparse and inconsistent human feedback. Addressing data scarcity for robust RMs in reasoning tasks, \cite{yu2024p4z} developed CodePMP, a scalable preference model pretraining pipeline that synthesizes high-quality code-preference pairs. Crucially, \cite{frick20248mv} introduced the Preference Proxy Evaluations (PPE) benchmark, which directly correlates RM performance on proxy tasks with real downstream human preference scores of RLHF-tuned LLMs, providing a vital tool to ensure RM improvements translate to actual policy gains. In a practical deployment context, \cite{hou2024tvy} incorporated "Bucket-Based Length Balancing" and L2 regularization during RM training to reduce length bias and stabilize training.

Beyond improving RMs, a significant body of work focuses on sophisticated regularization and constrained RLHF methods at the policy optimization stage to prevent models from exploiting RM flaws. \cite{moskovitz2023slz} proposed constrained RLHF, using dynamically identified "proxy points" to prevent overoptimization of individual components within composite reward models. \cite{fu2025hl3} introduced Preference As Reward (PAR), a theoretically grounded reward shaping technique that applies a sigmoid function to centered rewards, thereby bounding rewards and stabilizing training against hacking. Taking an internal-mechanistic approach, \cite{miao2025ox0} identified the "energy loss phenomenon" in the LLM's final layer during RLHF and proposed Energy loss-aware PPO (EPPO) to penalize excessive increases in this metric, mitigating reward hacking by preserving contextual relevance.

Uncertainty estimation in reward predictions has emerged as a powerful strategy. \cite{zhai20238xc} developed Uncertainty-Penalized RLHF (UP-RLHF), which uses diverse Reward LoRA Ensembles to quantify and penalize reward uncertainty, preventing the policy from generating out-of-distribution (OOD) responses where the RM is unreliable. Building on this, \cite{dai2025ygq} introduced Behavior-Supported Policy Optimization (BSPO), which employs value regularization to restrict policy iteration to the in-distribution region of the reward model, effectively penalizing OOD values without affecting in-distribution ones. Further enhancing efficiency, \cite{zhang2024esn} proposed Adversarial Policy Optimization (AdvPO) with a lightweight uncertainty estimation method based on last-layer embeddings, making RLHF more robust to overoptimization without the computational overhead of ensembles. \cite{cen2024nef} presented Value-Incentivized Preference Optimization (VPO), a unified framework for online and offline RLHF that implicitly incorporates optimism or pessimism through value function regularization, bypassing explicit uncertainty estimation.

Other regularization techniques address specific algorithmic biases and trade-offs. \cite{xiao2024ro4} identified "preference collapse" as an inherent algorithmic bias in standard KL-regularized RLHF, where minority preferences are disregarded, and proposed Preference Matching (PM) RLHF to ensure diversity. \cite{tan2025lk0} tackled the helpfulness-safety trade-off with Equilibrate RLHF, using a fine-grained data-centric approach and Adaptive Message-wise Alignment to prevent models from becoming "over-safe" and unhelpful. \cite{xu20242yo} introduced Constrained Generative Policy Optimization (CGPO) with a "Mixture of Judges" to mitigate reward hacking and multi-objective conflicts in complex multi-task settings. Furthermore, to combat the "alignment tax" (loss of general capabilities), \cite{lu202435m} developed Online Merging Optimizers that dynamically blend RLHF gradients with SFT model information, preventing overoptimization towards new preferences at the expense of foundational knowledge. Addressing the fundamental limitation of single reward models for diverse user groups, \cite{chakraborty20247ew} proposed MaxMin-RLHF, and \cite{boldi2024d0s} introduced Pareto-Optimal Learning from Preferences with Hidden Context, both aiming for multi-objective or Pareto-optimal solutions that do not overoptimize for an aggregated, potentially biased, preference. Finally, to improve credit assignment and make reward hacking more difficult, methods like \cite{li2024h19} (RED), \cite{chan2024xig} (ABC), and \cite{zhong2024wch} (RTO) provide dense, token-level rewards by redistributing holistic feedback, guiding the learning process with greater precision.

Despite these advancements, perfectly capturing complex, nuanced human preferences remains an open challenge. The inherent subjectivity and context-dependency of human values mean that proxy reward models will always be imperfect. Future research must continue to explore more robust and adaptive reward modeling techniques, develop more sophisticated regularization methods that account for diverse and potentially conflicting objectives, and integrate advanced uncertainty quantification to ensure that LLMs are aligned with genuine human intentions rather than merely optimizing a flawed proxy.
\subsection{Multi-Objective Alignment and Diverse Preferences}
\label{sec:5\_2\_multi-objective\_alignment\_\_and\_\_diverse\_preferences}

Aligning large language models (LLMs) with human values is inherently complex, often requiring the reconciliation of multiple, sometimes conflicting, objectives such as helpfulness, safety, honesty, and creativity, while also accommodating diverse user preferences. Traditional Reinforcement Learning from Human Feedback (RLHF) approaches, which typically rely on a single, monolithic reward model, often fall short in capturing this nuanced landscape. This limitation is starkly demonstrated by \cite{chakraborty20247ew}, who establish an "impossibility result" showing that single-reward RLHF cannot adequately align with diverse human preferences, particularly those of minority groups. Further, \cite{xiao2024ro4} reveals an inherent algorithmic bias in standard KL-regularized RLHF, leading to "preference collapse" where LLMs disproportionately favor dominant opinions, even with a perfectly accurate reward model, thus failing to represent population diversity.

To address these fundamental challenges, research has shifted towards multi-objective optimization and more sophisticated preference modeling. A critical trade-off is often observed between helpfulness and safety. \cite{tan2025lk0} tackles this with their Equilibrate RLHF framework, which employs a fine-grained data-centric approach to categorize safety data and an Adaptive Message-wise Alignment (AMA) strategy using gradient masking to balance these objectives, preventing models from becoming "over-safe" and unhelpful. Complementing this, \cite{zhang2024b6u} introduces Bi-Factorial Preference Optimization (BFPO), a supervised learning framework that efficiently re-parameterizes the joint helpfulness-safety RLHF objective into a single supervised loss, significantly reducing the computational and human annotation costs associated with multi-objective alignment. Beyond helpfulness and safety, \cite{moskovitz2023slz} confronts reward model overoptimization in composite reward settings by proposing constrained RLHF. Their method dynamically adapts the influence of individual reward components using "proxy points" and Lagrange multipliers, ensuring that no single objective is over-optimized at the expense of overall quality.

Moving beyond simple aggregation, methods are emerging to represent and learn from a spectrum of preferences. \cite{boldi2024d0s} introduces Pareto-Optimal Preference Learning (POPL), which leverages lexicase selection to learn a \textit{set} of reward functions or policies. Each policy is optimal for a distinct "hidden context" group, allowing for pluralistic alignment without requiring explicit group labels, thereby ensuring fairness and representing diverse opinions. To enhance the interpretability and steerability of reward models, \cite{wang20247pw} proposes ArmoRM with a Mixture-of-Experts (MoE) scalarization. This approach dynamically weights multiple \textit{absolute-rating} objectives (e.g., helpfulness, correctness, verbosity, safety) based on the input context, making the reward model transparent and enabling explicit mitigation of biases like verbosity, as previously diagnosed by \cite{singhal2023egk}. Similarly, \cite{yu20249l0} improves reward model quality and interpretability by incorporating self-generated critiques, providing explicit rationales beyond scalar scores, which is crucial for understanding nuanced preferences. Furthermore, \cite{lai2024ifx} introduces ALaRM, a hierarchical reward modeling framework that integrates holistic and proactively selected aspect-specific rewards to provide more precise and consistent guidance, addressing the challenges of inconsistent and sparse human feedback in complex generation tasks.

The deployment of LLMs in real-world applications necessitates robust strategies for handling diverse user needs and practical constraints. \cite{xu20242yo} addresses multi-task learning with conflicting objectives and reward hacking through Constrained Generative Policy Optimization (CGPO) with a "Mixture of Judges." This framework allows for task-specific optimization, employing customized reward models, judges, and hyperparameters to effectively manage complex multi-objective scenarios. In domain-specific contexts, \cite{yang2024ppw} tackles programming question answering, where diverse user preferences and up-to-date information are paramount. Their ALMupQA framework uses Multi-perspective Preference Ranking Alignment (MPRA) to synthesize preferences from questioners, community votes, and LLM content evaluations, combined with Retrieval-augmented In-context Learning (RIL) to ensure factual recency. For robust alignment against varying inputs, \cite{mandal2025qf5} proposes Distributionally Robust Reinforcement Learning with Human Feedback (DR-RLHF) to enhance robustness against out-of-distribution (OOD) prompt shifts, ensuring LLMs generalize reliably to diverse user needs. \cite{hong2024mqe} further refines preference learning by introducing Adaptive Preference Scaling, which uses instance-specific scaling parameters to capture varying strengths of human preferences, leading to more flexible and accurate reward models. Finally, practical challenges like catastrophic forgetting, or "alignment tax," are addressed by \cite{lu202435m} with Online Merging Optimizers, which balance reward maximization with the preservation of foundational capabilities. \cite{hou2024tvy} provides a comprehensive production-grade RLHF pipeline, offering practical solutions for bias mitigation (e.g., length bias) and catastrophic forgetting, essential for aligning large models for diverse real-world use cases.

In conclusion, the field of LLM alignment is rapidly evolving from monolithic reward functions to multi-objective, nuanced, and interpretable strategies. While significant progress has been made in balancing conflicting objectives, representing diverse preferences, and ensuring robustness in real-world scenarios, ongoing challenges remain in scaling these sophisticated methods, achieving universal fairness across all demographics, and developing truly adaptive systems that can learn and evolve with changing human values and societal norms. Future research will likely focus on more autonomous and dynamic mechanisms for preference elicitation and objective reconciliation, further integrating interpretability and robustness as first-class citizens in the alignment pipeline.
\subsection{Theoretical Foundations and Algorithmic Advancements}
\label{sec:5\_3\_theoretical\_foundations\_\_and\_\_algorithmic\_advancements}

The empirical success of Reinforcement Learning from Human Feedback (RLHF) in aligning Large Language Models (LLMs) with complex human preferences has spurred significant theoretical and algorithmic innovations. This subsection delves into the rigorous justifications, formal guarantees, and novel policy optimization techniques that underpin and refine the RLHF paradigm, addressing its inherent challenges in efficiency, stability, and bias. These advancements can be broadly categorized into formalizing RLHF objectives and convergence, developing direct policy optimization methods, and enhancing core RLHF mechanisms through reward shaping, bias mitigation, and stability improvements.

A growing body of work provides theoretical justifications for RLHF's empirical effectiveness and clarifies its relationship to established RL paradigms. \cite{sun2023awe} re-frames RLHF as an Online Inverse Reinforcement Learning (IRL) problem with known transition dynamics, arguing that this property mitigates compounding errors and distributional shifts inherent in purely offline methods like Supervised Fine-Tuning (SFT), thereby explaining RLHF's observed superiority. Further connecting RLHF to fundamental RL theory, \cite{su2025mld} establishes a unified framework (UDRRA) that links various RLHF algorithms, including Direct Preference Optimization (DPO), to actor-critic-based PPO, clarifying their loss function construction, target policy distributions, and the impact of key components on convergence. This theoretical grounding helps demystify DPO's effectiveness despite its seemingly supervised nature. Complementing this, \cite{hejna2023vyy} introduces Contrastive Preference Learning (CPL), a direct policy learning approach grounded in a regret-based preference model and Maximum Entropy RL, offering theoretical guarantees for recovering the optimal policy and demonstrating improved efficiency by transforming the problem into a supervised objective, bypassing explicit reward modeling and traditional RL optimization.

The quest for direct policy optimization from human feedback, circumventing explicit reward model inference, has led to several provably efficient algorithms. \cite{tang2023lop} proposes ZO-RankSGD, the first provably convergent zeroth-order optimization algorithm that learns directly from human ranking oracles for general non-convex functions, achieving a convergence rate of $O(\sqrt{d/T})$. Extending this to more general RL settings, \cite{zhang2024w99} develops Zeroth-Order Policy Gradient (ZPG) and its block-coordinate variant (ZBCPG) for stochastic Markov Decision Processes (MDPs), provably identifying optimal policies from human preferences without reward inference. These methods offer a compelling alternative to reward modeling, particularly when reward function design is challenging, though they often entail higher sample complexity or rely on specific oracle access. For model-free RLHF in episodic MDPs, \cite{zhang20248x9} introduces Batched Sequential Action Dueling (BSAD), which directly identifies optimal policies from batched human preferences, achieving instance-dependent sample complexity without explicit reward inference. For offline LLM alignment, \cite{ji2024d5f} presents Self-Play with Adversarial Critic (SPAC), which offers provable convergence guarantees under weak data assumptions by formulating the problem as a Stackelberg game with an "on-average pessimistic" critic, and importantly, provides a practical DPO-like single-timescale implementation.

Formal convergence guarantees for KL-constrained preference learning have also seen significant advancements. \cite{xiong2023klt} provides a rigorous theoretical analysis of the reverse-KL regularized contextual bandit problem for RLHF, delivering finite-sample guarantees for offline, online, and hybrid settings, and leading to iterative DPO and pessimism-based algorithms. This work highlights the importance of strategic online exploration for performance improvement. Building on this, \cite{zhao202532z} achieves a breakthrough by establishing the first logarithmic regret bounds for online KL-regularized contextual bandits and MDPs, significantly improving theoretical efficiency guarantees (from $O(\sqrt{T})$ to $O(\log T)$) without relying on strong coverage assumptions. This theoretical advancement provides a strong justification for the observed empirical efficiency of KL-regularized RL in LLM alignment, demonstrating its ability to mitigate the "alignment tax" and enhance training stability.

Beyond theoretical foundations, novel policy optimization techniques have emerged to enhance efficiency, stability, and address specific algorithmic biases. A critical challenge in RLHF is the sparse reward signal, where a single scalar reward is given at the end of a long sequence. To combat this, \cite{zhong2024wch} introduces Reinforced Token Optimization (RTO), which re-frames RLHF as an MDP and leverages DPO to extract dense, token-wise reward signals, leading to substantial improvements in PPO's data efficiency and performance (e.g., 7.5-point gain on AlpacaEval 2). Similarly, \cite{chan2024xig} proposes Attention Based Credit (ABC), a method to densify sparse rewards by redistributing the final scalar reward using the reward model's internal attention maps. This approach is notable for being "dense reward for free," requiring no additional modeling or significant computation, and is theoretically proven to be equivalent to potential-based reward shaping, ensuring optimal policy preservation. \cite{li2024h19} further contributes to dense reward generation with REward reDistribution (RED), assigning token-level rewards based on the incremental impact of each token on the reward model's score. While these reward densification methods improve credit assignment, they rely heavily on the internal states of a potentially flawed reward model, risking the propagation of its biases to the token level. Addressing the credit assignment problem from a different angle, \cite{chai2024qal} introduces Reinforcement Learning from Human Feedback with Macro Actions (MA-RLHF), which allows optimization at a higher level of abstraction by treating sequences of tokens as single decision units, accelerating learning and improving performance.

Algorithmic biases in RLHF have also been rigorously analyzed and addressed. \cite{xiao2024ro4} identifies "preference collapse," an inherent algorithmic bias in standard KL-regularized RLHF that disproportionately favors dominant preferences. They propose Preference Matching (PM) RLHF, a novel algorithm that provably aligns LLMs with the full diversity of preference distributions by deriving a new regularizer from an ordinary differential equation, achieving significant improvements in alignment (e.g., 29-41\\% reduction in Preference Matching Divergence). To enhance the stability and efficiency of off-policy RLHF, \cite{zhou202469n} introduces Weighted Preference Optimization (WPO), which mitigates the "distributional gap" by dynamically reweighting preference pairs based on their probability under the current policy, effectively simulating on-policy learning and achieving state-of-the-art results (76.7\\% win rate on AlpacaEval 2). \cite{cen2024nef} presents Value-Incentivized Preference Optimization (VPO), a unified framework for online and offline RLHF that implicitly incorporates optimism or pessimism through value function regularization, bypassing the intractable explicit uncertainty estimation for LLMs and demonstrating improved empirical performance. Furthermore, \cite{tang2024wt3} empirically dissects the performance gap between online and offline alignment algorithms, concluding that on-policy sampling plays a pivotal role in achieving high generative quality, providing crucial insights into the limitations of offline methods.

Addressing the critical issue of "overoptimization" where LLMs exploit flaws in reward models, \cite{zhai20238xc} proposes Uncertainty-Penalized RLHF (UP-RLHF). This method augments the RLHF objective with an uncertainty regularization term, penalizing rewards based on the estimated uncertainty of a diverse reward LoRA ensemble. This approach effectively mitigates overoptimization and improves alignment with human preferences. Another significant challenge is the "alignment tax," where RLHF can degrade foundational capabilities. \cite{lu202435m} introduces the Online Merging Optimizer, which integrates model merging into each optimization step of RLHF, continuously steering gradients to maximize rewards while maintaining SFT capabilities, thereby mitigating the alignment tax more effectively than offline merging or simple regularization.

Beyond the core RLHF pipeline, advancements also include methods for active exploration and data-agnostic alignment. \cite{zhang2024lqf} proposes Self-Exploring Language Models (SELM), an online direct alignment algorithm with an optimistically biased objective that actively explores the response space, improving instruction-following capabilities without additional computational overhead. This addresses the limitation of passive exploration in DPO-like methods that can lead to local optima. \cite{chen2024vkb} introduces DICE, an iterative self-alignment framework that bootstraps DPO-tuned LLMs using their own implicit reward models, enabling continuous improvement without external feedback and mitigating length bias. Finally, \cite{singla2024dom} presents Dynamic Rewarding with Prompt Optimization (DRPO), a tuning-free, search-based framework for self-alignment at inference time, dynamically adjusting LLM-based rewards for rapid adaptability. Even the initial data collection for reward models is being optimized; \cite{scheid20244oy} proposes Optimal Design for Preference Optimization (ODPO), an offline framework for selecting the most informative pairs for human labeling using optimal design theory, offering theoretical guarantees for minimizing simple regret in reward model training. This highlights a shift towards optimizing the entire RLHF pipeline, from data acquisition to policy optimization.

Despite these significant advancements, several challenges persist. The trade-off between exploration and exploitation remains complex, particularly in the vast and discrete action spaces of LLMs, necessitating continued research into robust active exploration strategies. The inherent algorithmic biases, such as preference collapse \cite{xiao2024ro4}, necessitate continued research into robust regularization techniques that can handle diverse and potentially conflicting preferences. Moreover, while methods like direct policy optimization offer theoretical elegance, their practical implementation often introduces new complexities or computational demands that need careful consideration. Future work will likely focus on developing more adaptive, context-aware, and theoretically sound algorithms that can balance alignment with diverse human preferences, maintain model capabilities, and scale efficiently to novel and complex tasks while offering stronger guarantees of safety and robustness.


\label{sec:beyond_human_feedback:_self-supervised_and_ai-driven_alignment}

\section{Beyond Human Feedback: Self-Supervised and AI-Driven Alignment}
\label{sec:beyond\_human\_feedback:\_self-supervised\_\_and\_\_ai-driven\_alignment}

\subsection{Reinforcement Learning from AI Feedback (RLAIF) and Self-Alignment}
\label{sec:6\_1\_reinforcement\_learning\_from\_ai\_feedback\_(rlaif)\_\_and\_\_self-alignment}

The burgeoning capabilities of Large Language Models (LLMs) have been significantly propelled by Reinforcement Learning from Human Feedback (RLHF), yet its reliance on costly and unscalable human annotation presents a major bottleneck \cite{lambert2023bty, lee2023mrw}. This challenge has spurred extensive research into methods that replace or augment human feedback with AI-generated feedback, such as Reinforcement Learning from AI Feedback (RLAIF), and self-alignment strategies where models learn from their own generated content or internal mechanisms. The overarching goal is to create more autonomous and scalable alignment pipelines that reduce human supervision while maintaining or enhancing alignment quality and robustness.

A direct approach to scaling alignment is \textbf{Reinforcement Learning from AI Feedback (RLAIF)}, where LLMs themselves act as evaluators. \textcite{lee2023mrw} provided foundational work, demonstrating that RLAIF can achieve performance comparable to RLHF across various tasks, even enabling self-improvement using same-size or same-checkpoint AI labelers. They introduced Direct-RLAIF (d-RLAIF), which bypasses explicit reward model training by having an LLM directly provide reward scores during the RL phase, mitigating "reward model staleness." However, relying on AI for feedback introduces its own set of challenges. \textcite{saito2023zs7} critically identified "verbosity bias" in LLM preference labeling, showing that AI evaluators often prefer longer responses regardless of quality, a discrepancy with human preferences that can lead to suboptimal alignment. To address such limitations, \textcite{li2024ev4} proposed Hybrid Reinforcement Learning from AI Feedback (HRLAIF), a multi-stage AI preference labeling framework. HRLAIF guides AI to make more accurate judgments on complex attributes like factual correctness through structured processes, such as correctness verification and AI-driven red teaming, thereby improving both helpfulness and harmlessness while maintaining cost-efficiency.

Beyond directly replacing human annotators, \textbf{self-alignment strategies} aim to make models learn from their own generated content or internal signals, further reducing human supervision. \textcite{sun20238m7} pioneered principle-driven self-alignment from scratch with minimal human supervision. Their \texttt{SELF-ALIGN} framework uses a small set of human-written principles and in-context exemplars to guide an LLM's internal reasoning, followed by a "principle engraving" step that internalizes these rules into the model's parameters, enabling aligned generation without explicit prompting.

Another avenue for self-alignment involves leveraging models' internal feedback mechanisms. \textcite{cao2024lh3} addressed the sparse reward problem in text generation by using LLM critique to generate dense, token-level intrinsic rewards (RELC), directly improving the RL policy optimization step within the RLHF pipeline. Expanding on this, \textcite{liu20251xv} introduced RISE (Reinforcing Reasoning with Self-Verification), an online RL framework that simultaneously trains LLMs in problem-solving and self-verification using verifiable rewards. This enables models to learn to critique their own reasoning processes, enhancing robustness. However, the efficacy of naive internal feedback is not without limits. \textcite{zhang2025d44} critically examined Reinforcement Learning from Internal Feedback (RLIF) using signals like self-certainty and entropy. They theoretically and empirically demonstrated that these internal feedback signals primarily minimize policy entropy, which can initially boost base models but leads to performance degradation with prolonged training due to a reduction in "transitional words" crucial for multi-step reasoning. This highlights that RLIF is not a "free lunch" and requires careful application.

Further advancements in self-alignment leverage implicit rewards and active exploration. \textcite{chen2024vkb} introduced DICE (self-alignment with DPO Implicit Rewards), an iterative self-alignment method that bootstraps DPO-tuned LLMs by using their \textit{own implicit reward models} to generate new preference data. This eliminates the need for external feedback in subsequent alignment rounds, making iterative DPO more self-contained. Complementing this, \textcite{zhang2024lqf} proposed Self-Exploring Language Models (SELM), an active preference elicitation method for online alignment. SELM uses an optimistically biased, reward model-free objective to guide LLMs to actively explore potentially high-reward regions of the response space, addressing the limitations of passive exploration in traditional online alignment. Taking autonomy a step further, \textcite{singla2024dom} presented Dynamic Rewarding with Prompt Optimization (DRPO), a tuning-free, search-based framework where LLMs dynamically adjust rewards and optimize prompts at inference time for self-alignment, entirely without model training or human supervision.

Despite these advancements, several challenges persist. The robustness of AI-generated feedback and self-alignment to adversarial attacks remains a concern. \textcite{baumgrtner2024gu4} demonstrated RLHF's vulnerability to "preference poisoning" using naturalistic, stealthy data, a risk that RLAIF or self-alignment might inherit if the underlying "AI judge" or self-evaluation mechanism is compromised. Similarly, \textcite{zhang2023pbi} showed that alignment mechanisms do not prevent the misuse of \textit{open-sourced} LLMs through "model hacking," forcing the generation of undesired content, which poses a fundamental security challenge for autonomous alignment. Moreover, alignment processes, whether human or AI-driven, often introduce trade-offs. \textcite{kirk20230it} found that RLHF improves out-of-distribution generalization but reduces output diversity, while \textcite{mohammadi20241pk} empirically showed that debiasing LLMs with RLHF can significantly reduce their creativity and lead to "attractor states" in their output space. This "objective mismatch," where numerical objectives decouple from true human intent, can lead to unintended behaviors like refusals or verbosity, as highlighted by \textcite{lambert2023c8q}.

Addressing these issues requires a deeper understanding of the underlying mechanisms and more robust methodologies. \textcite{xiao2024ro4} identified "preference collapse" as an algorithmic bias in standard KL-regularized RLHF, where minority preferences are disregarded, proposing Preference Matching to ensure diversity. \textcite{wang20247pw} tackled reward model interpretability and steerability through multi-objective reward modeling and Mixture-of-Experts, mitigating issues like reward hacking and verbosity bias, which is crucial for reliable AI feedback. Furthermore, exploring novel feedback signals, such as implicit human cues like gaze data, which \textcite{lpezcardona20242pt} showed can be integrated into reward models using synthetic features for scalability, offers new avenues for human-like feedback without direct human effort. The application of LLM-enhanced RLHF is also extending beyond language to complex domains like autonomous driving, as demonstrated by \textcite{sun2024nor}, showcasing the broad potential of feedback-driven alignment.

In conclusion, the shift towards RLAIF and self-alignment represents a critical intellectual trajectory in LLM alignment, driven by the need for scalability and autonomy. While significant progress has been made in developing AI-driven feedback mechanisms and self-improving models, the field continues to grapple with fundamental challenges related to bias, robustness, the trade-off between alignment and creativity, and the computational efficiency of these complex systems. Future research must focus on developing more theoretically grounded, robust, and interpretable autonomous alignment strategies that can consistently deliver high-quality, safe, and diverse LLM behaviors across an ever-expanding range of applications.
\subsection{Automated Reward Design and Implicit Feedback}
\label{sec:6\_2\_automated\_reward\_design\_\_and\_\_implicit\_feedback}

The efficacy of Reinforcement Learning (RL) agents is profoundly dependent on well-designed reward functions, which are notoriously difficult and time-consuming to engineer manually. This challenge has spurred research into leveraging Large Language Models (LLMs) and Vision-Language Models (VLMs) to automate reward design or infer preference signals from implicit human behaviors, thereby moving beyond costly explicit human labeling. These advanced techniques aim to create more sophisticated, interpretable, and scalable reward mechanisms that capture nuanced preferences and task objectives.

One prominent direction involves using LLMs to generate executable reward code. \textcite{ma2023vyo} introduced Eureka, a framework where an LLM (GPT-4) autonomously designs executable Python reward functions for complex robotic tasks. Eureka employs an evolutionary optimization process, guided by "environment as context" and "reward reflection," an automated feedback mechanism that analyzes policy training dynamics, enabling the LLM to achieve superhuman performance on various dexterous manipulation tasks. Building upon this, \textcite{hazra2024wjp} proposed REvolve, which also leverages LLMs as intelligent genetic operators within an evolutionary framework to evolve executable Python reward code. A key innovation in REvolve is the integration of human preference feedback on policy rollouts as the fitness function, allowing for the design of more nuanced and human-aligned rewards in subjective domains like autonomous driving, thereby addressing the limitations of purely automated feedback for complex, tacit human knowledge.

Beyond generating explicit code, VLMs have emerged as powerful tools for directly inferring reward signals from visual observations. \textcite{rocamonde2023o9z} demonstrated that pretrained VLMs, such as CLIP, can serve as zero-shot reward models (VLM-RMs). This method computes rewards as the cosine similarity between the language embedding of a natural language task description and the image embedding of the current state, enabling agents to learn complex tasks like humanoid locomotion from simple text prompts. While effective, direct VLM scores can sometimes be noisy. To address this, \textcite{wang2024n8c} introduced RL-VLM-F, which leverages VLMs (e.g., GPT-4V, Gemini) to provide preference feedback over pairs of visual observations based on a text goal. A reward function is then learned from these VLM-generated preference labels, offering a more robust and scalable approach to inferring rewards from visual data, particularly for complex manipulation tasks involving deformable objects, without requiring low-level state access.

A complementary line of research focuses on how LLMs and VLMs can interpret complex or implicit signals to construct reward functions. \textcite{cao2024lh3} tackled the sparse reward problem in text generation by proposing RELC (Rewards from Language model Critique). This framework utilizes an LLM as a critic to generate dense, intermediate-step intrinsic rewards during RL training, effectively automating reward shaping and significantly improving sample efficiency and performance across tasks like sentiment control and detoxification. Extending this to real-world systems, \textcite{sun2024nor} presented a human-centric approach for optimizing autonomous driving safety using LLM-enhanced RLHF. Their method employs an LLM to interpret multimodal sensor data, including physical (e.g., acceleration) and physiological (e.g., heart rate) feedback from human drivers, translating these complex, often implicit, signals into preferences for the autonomous driving reward function. This innovative approach aims to capture human comfort and safety, moving beyond explicit labels. Finally, directly integrating truly implicit human signals, \textcite{lpezcardona20242pt} introduced GazeReward, a framework that incorporates synthetic eye-tracking (ET) data into the Reward Model for LLM alignment. By using ET prediction models to generate synthetic gaze features from text, GazeReward provides a scalable and cost-effective way to infer human preferences from subconscious behavioral cues, demonstrating substantial accuracy gains in reward model predictions.

In conclusion, the integration of LLMs and VLMs into reward design and feedback mechanisms marks a significant paradigm shift in RL. These methods collectively demonstrate the potential to automate the creation of executable reward functions, infer preferences directly from visual and linguistic cues, and even translate complex implicit human signals into actionable reward components. While these advancements promise more scalable, interpretable, and human-aligned RL systems, challenges remain in ensuring the fidelity of synthetic implicit signals, guaranteeing the generalizability and safety of LLM-generated reward code, and establishing robust ethical frameworks for inferring preferences from subtle human behaviors. Future research will likely focus on enhancing the robustness of these models, exploring multimodal implicit signals more broadly, and developing hybrid approaches that combine the strengths of explicit and implicit feedback for truly sophisticated AI alignment.


\label{sec:rl_for_enhanced_llm_reasoning_and_specialized_capabilities}

\section{RL for Enhanced LLM Reasoning and Specialized Capabilities}
\label{sec:rl\_for\_enhanced\_llm\_reasoning\_\_and\_\_specialized\_capabilities}

\subsection{Improving Multi-Step Reasoning and Problem Solving}
\label{sec:7\_1\_improving\_multi-step\_reasoning\_\_and\_\_problem\_solving}

Large Language Models (LLMs) often struggle with complex, multi-step reasoning and problem-solving tasks, particularly in challenging domains like mathematics and logical deduction, where superficial pattern matching is insufficient. Advanced reinforcement learning (RL) techniques, especially those leveraging preference-based feedback and sophisticated search algorithms, are proving instrumental in moving LLMs towards more robust, verifiable, and genuinely compositional reasoning processes.

The critical role of Reinforcement Learning from Human Feedback (RLHF) in enhancing LLM reasoning capabilities was empirically highlighted by \cite{fu2023pyr}. Their Chain-of-Thought Hub demonstrated that RLHF significantly impacts the performance of top-tier models on diverse reasoning benchmarks, underscoring the necessity for more sophisticated RL-driven approaches to achieve state-of-the-art performance. Building on this, some research focuses on dynamic guidance during inference. For instance, \cite{hao2025lc8} introduced RL-of-Thoughts (RLoT), an inference-time technique that trains a lightweight "navigator model" using Double-Dueling-DQN. This navigator dynamically selects human cognition-inspired "logic blocks" (e.g., "Reason one step," "Decompose," "Refine") to guide the LLM's reasoning path, adapting to the problem's evolving state through self-evaluation and a Process Reward Model. This approach enables smaller LLMs to achieve performance comparable to much larger models without modifying their base parameters, offering an efficient external control mechanism for complex tasks.

However, for deeper, more intrinsic improvements in compositional reasoning, methods that fine-tune the LLM with fine-grained, step-level feedback are crucial. A prominent direction involves combining preference-based RL with powerful search algorithms like Monte Carlo Tree Search (MCTS) to optimize for correct solution paths. \cite{zhang2024q0e} proposed LLaMA-Berry, a framework specifically designed for Olympiad-level mathematical reasoning. LLaMA-Berry integrates Self-Refine MCTS to robustly explore the solution space, treating entire solutions as states and self-refinement as an optimization action. It employs a Pairwise Preference Reward Model (PPRM), trained with Direct Preference Optimization (DPO), to evaluate solutions, and an Enhanced Borda Count method to aggregate these preferences into global scores. This synergistic combination enables smaller LLMs to tackle highly complex mathematical problems effectively by providing solution-level preference signals.

A significant advancement in providing truly fine-grained feedback comes from \cite{chen20244ev} with their Step-level Value Preference Optimization (SVPO). Recognizing that solution-level preferences are insufficient for multi-step reasoning, SVPO leverages MCTS not just for search, but to \textit{autonomously generate fine-grained step-level preferences}. By analyzing Q-values at each node in the MCTS tree, SVPO identifies potential reasoning errors at specific steps, eliminating the need for costly human or GPT-4 annotations. It then integrates an explicit value model into a novel DPO-like loss function, trained directly from these MCTS-derived Q-values and step-level preferences. This allows the LLM to learn \textit{why} certain steps are better than others, significantly boosting mathematical reasoning performance and enabling the model to navigate complex reasoning paths more effectively during inference with techniques like Step-level Beam Search. Such advancements are further supported by the availability of high-quality, large-scale datasets like Big-Math \cite{albalak2025wyc}, which provides meticulously curated, open-ended mathematical problems suitable for RL training, bridging the gap between data quality and quantity.

In conclusion, the integration of advanced RL techniques, particularly preference-based methods combined with sophisticated search algorithms like MCTS, is driving LLMs towards more robust multi-step reasoning. The evolution from observing RLHF's impact to developing inference-time guidance and, crucially, fine-tuning methods that provide fine-grained, step-level feedback marks a significant shift. Future research will likely focus on scaling these fine-grained feedback mechanisms to broader logical domains, improving the efficiency of MCTS-based data generation, and integrating these compositional reasoning capabilities more deeply into foundational LLM architectures, ultimately enabling verifiable and genuinely intelligent problem-solving.
\subsection{Tool Use and Agentic Behavior}
\label{sec:7\_2\_tool\_use\_\_and\_\_agentic\_behavior}

The development of Large Language Models (LLMs) as intelligent, autonomous agents capable of interacting with external environments and utilizing diverse tools represents a significant frontier in AI research. Reinforcement Learning (RL) is pivotal in enabling these models to learn adaptive strategies, explore complex action spaces, and optimize for task success in dynamic, open-ended environments, thereby fostering more versatile AI systems that extend beyond pure language generation. This subsection explores recent advancements in applying RL to empower LLMs with sophisticated tool-use capabilities, particularly in multimodal contexts and code generation.

A key direction involves enabling Large Vision-Language Models (LVLMs) to dynamically orchestrate visual tools for complex problem-solving. \cite{su20257nq} introduces \textbf{OPENTHINK IMG}, a comprehensive, open-source framework that standardizes visual tool interfaces and facilitates scalable data generation for tool-augmented LVLMs. Central to their approach is \textbf{V-TOOLRL}, a reinforcement learning framework leveraging Group-wise Proximal Policy Optimization (GRPO) to learn adaptive tool invocation policies. This methodology allows LVLMs to move beyond static supervised fine-tuning (SFT) by autonomously exploring and optimizing for task success using feedback from visual tool interactions, demonstrating significant performance gains over SFT baselines and even closed-source models in chart reasoning tasks. However, the initial action planning in their data generation pipeline still relies on powerful external LLMs like GPT-4o, indicating a bootstrapping dependency for high-quality initial trajectories.

Complementing the orchestration of visual tools, another critical area is the RL-driven generation of code that interacts with external solvers and simulators, enabling deep visual reasoning and self-correction. \cite{chen20253a4} addresses the challenge of training Multimodal Large Language Models (MLLMs) for complex visual generative tasks without extensive image-text supervision. Their proposed \textbf{Reasoning-Rendering-Visual-Feedback (RRVF)} framework leverages the "Asymmetry of Verification" principle: it is easier to verify a rendered output against an image than to generate the code from scratch. This closed-loop iterative process involves the MLLM generating code, external tools rendering it, and a "Visual Judge" MLLM providing natural language feedback, all optimized end-to-end using GRPO with a hybrid reward function that includes visual similarity, format correctness, and adaptive tool-use. This innovative approach allows MLLMs to learn complex generative logic directly from pixels through self-correction, circumventing data annotation bottlenecks and even surpassing the performance of the judge model itself. Similar to \cite{su20257nq}, a powerful MLLM (GPT-4o) is critical for generating the visual similarity reward, highlighting a shared reliance on frontier models to bootstrap the learning process.

The papers collectively highlight a significant intellectual trajectory in applying reinforcement learning to enhance the agentic capabilities of multimodal LLMs. Both \cite{su20257nq} and \cite{chen20253a4} represent a paradigm shift from static, supervised fine-tuning towards dynamic, feedback-driven learning for MLLMs, demonstrating that RL (specifically GRPO) can enable more adaptive and robust behaviors. While \cite{su20257nq} focuses on learning how to orchestrate diverse visual tools for problem-solving, \cite{chen20253a4} tackles learning how to generate precise code from visual inputs through iterative self-correction. This synergy allows LLMs to interact with and learn from their environments in increasingly sophisticated ways, fostering emergent reasoning patterns and problem-solving strategies.

Despite these advancements, a common thread of unresolved issues pertains to the initial bootstrapping and feedback generation mechanisms. The reliance on powerful MLLMs (like GPT-4o) for either data generation or reward signal generation across both works points to an interesting tension: while RL enables models to learn autonomously, the initial quality and diversity of the learning signal often still depend on the capabilities of other advanced models. Future directions should explore methods to reduce this dependency, perhaps through more generalized self-supervised or self-generated feedback mechanisms, or by developing more robust and efficient ways to synthesize diverse, high-quality interaction data without relying on frontier models. Further research is also needed to generalize these RL-driven tool-use paradigms to a broader array of complex, open-ended environments and tool types, moving towards truly autonomous and versatile AI agents.
\subsection{Domain-Specific Applications (e.g., Software Engineering, Code Generation)}
\label{sec:7\_3\_domain-specific\_applications\_(e.g.,\_software\_engineering,\_code\_generation)}

Reinforcement Learning (RL) is increasingly being leveraged to align Large Language Models (LLMs) with the stringent requirements of highly specialized domains, where outputs demand strict functional correctness, adherence to domain-specific rules, or complex problem-solving. This marks a crucial intellectual trajectory in RL for language processing, shifting towards "grounded" or "executable" language generation by integrating non-textual, domain-specific feedback for rigorous and verifiable performance.

A significant challenge in applying RL to complex, real-world engineering tasks is the impracticality of obtaining direct execution-based rewards. \cite{wei2025v4d} addresses this in the context of software engineering (SE) with \textbf{SWE-RL}, an innovative framework designed to enhance LLM reasoning for bug fixing and issue resolution. Instead of costly execution-based feedback, SWE-RL leverages a massive dataset of GitHub Pull Requests and employs a lightweight, rule-based reward function based on patch similarity using \texttt{difflib.SequenceMatcher}. This approach, the first to apply RL with rule-based rewards to real-world software evolution data, enabled a medium-sized LLM (Llama3-SWE-RL-70B) to achieve a 41.0\\% solve rate on the SWE-bench Verified benchmark, rivaling larger proprietary models. Crucially, \cite{wei2025v4d} demonstrated that RL on this single, complex in-domain task led to emergent generalized reasoning skills across diverse out-of-domain tasks, suggesting that specialized training can foster broader cognitive benefits. However, the reliance on a proxy reward function, while practical, might not perfectly capture true functional equivalence, and the \texttt{Agentless Mini} scaffold used for evaluation simplifies some real-world complexities.

Complementing this, other research has focused on domains where direct external verification is feasible and critical for functional correctness. \cite{wang20252c7} exemplifies this by training a Verilog generation LLM with RL using rigorous testbench feedback. Recognizing that Supervised Fine-Tuning (SFT) alone cannot guarantee functional correctness for Hardware Description Languages (HDLs), this work combines SFT with Direct Preference Optimization (DPO). The core innovation lies in a novel automatic testbench generation pipeline that leverages real Verilog Compiler Simulator (VCS) feedback to collect preference pairs. By designating code passing more test cases as "preferred," \cite{wang20252c7} aligns the LLM directly with functional correctness, mitigating the "reward hacking" problem common in explicit reward-based RL. This approach, the first to integrate verification insights and DPO for Verilog generation, consistently outperforms state-of-the-art baselines in generating functionally correct HDL code. A limitation, however, is that automatic testbench generation cannot guarantee coverage of all functional cases, and the reliance on external simulators introduces computational overhead.

Both \cite{wei2025v4d} and \cite{wang20252c7} push the boundaries of LLM utility in critical engineering fields, albeit through different strategies for feedback integration. \cite{wei2025v4d} demonstrates the power of scalable, lightweight, rule-based proxy rewards when direct execution feedback is prohibitive, while \cite{wang20252c7} showcases how to effectively integrate rigorous, non-textual, domain-specific external verification for achieving strict functional correctness. The unresolved tension lies in balancing the practicality and scalability of proxy reward functions with their fidelity to true task objectives, and in scaling computationally intensive verification processes to match the ever-growing size and complexity of LLMs and their diverse applications. Future directions will likely explore hybrid approaches that combine the efficiency of proxy rewards with targeted, rigorous verification, or develop more efficient and comprehensive external verification methods.


\label{sec:practical_considerations:_efficiency,_scalability,_and_safety}

\section{Practical Considerations: Efficiency, Scalability, and Safety}
\label{sec:practical\_considerations:\_efficiency,\_scalability,\_\_and\_\_safety}

\subsection{System-Level Optimizations and Training Efficiency}
\label{sec:8\_1\_system-level\_optimizations\_\_and\_\_training\_efficiency}

The application of Reinforcement Learning from Human Feedback (RLHF) to large language models (LLMs) has proven instrumental in aligning their behavior with human preferences, yet it introduces significant computational challenges. The inherent complexity of RLHF pipelines, involving multiple LLM instances (actor, critic, reward, reference models), diverse computational workloads (generation, inference, training), and intricate distributed execution, often leads to prohibitive memory consumption and extended training times. Empirical analyses, such as that by \cite{hou202448j}, highlight that current RLHF frameworks exhibit diminishing returns with increased computational resources, underscoring the urgent need for more scalable algorithms and improved system designs. Addressing these bottlenecks is crucial for making complex RL-based alignment methods practical for large-scale LLMs, enabling broader experimentation and deployment.

A fundamental aspect of improving training efficiency lies in optimizing the data sampling process. \cite{wang2023v62} introduced \textit{ESRL: Efficient Sampling-based Reinforcement Learning for Sequence Generation}, a method that significantly reduces memory and training time by decoupling sequence sampling from gradient computation. ESRL employs a two-stage sampling process where sequences are first generated without gradient tracking, and their probabilities are then computed in a single, parallel forward pass, leveraging Transformer architecture parallelism. Furthermore, it incorporates dynamic sampling, adaptively adjusting sampling size and temperature based on estimated model capability to avoid redundant exploration, demonstrating up to 47\\% memory reduction and 39\\% training time reduction in machine translation tasks and substantial gains in RLHF for LLMs.

Building upon efficient execution principles, \cite{sheng2024sf5} proposed \textit{HybridFlow}, a flexible and efficient RLHF framework designed to overcome the limitations of traditional single-controller and existing multi-controller distributed systems. HybridFlow introduces a hierarchical hybrid programming model that combines a single-controller at the inter-node level for flexible dataflow expression with a multi-controller paradigm for efficient intra-node distributed LLM computation. This framework includes a \textit{3D-HybridEngine} for zero-redundancy model parameter resharding between actor training and generation, and an optimized GPU allocation algorithm, achieving throughput improvements ranging from 1.53$\times$ to 20.57$\times$ compared to baselines.

Further advancing dynamic resource management, \cite{mei2024eqt} presented \textit{ReaL: Efficient RLHF Training of Large Language Models with Parameter Reallocation}. ReaL addresses the inefficiencies of static parallelization strategies by dynamically adapting resource allocation and parallelization for different workloads during RLHF training through parameter reallocation. It introduces fine-grained "execution plans" for each model function call (e.g., Actor generation, Critic training) and employs a Markov Chain Monte Carlo (MCMC) search algorithm with a lightweight profiler to discover optimal plans. This dynamic approach leads to speedups of up to 3.58$\times$ and an average of 54\\% performance improvement over heuristic methods, particularly beneficial for long-context scenarios.

Beyond architectural and resource allocation strategies, efficiency can also be gained by rethinking the core RLHF pipeline. \cite{zhu2024zs2} introduced \textit{Proxy-RLHF}, a novel paradigm that decouples the generation and alignment processes of LLMs. Instead of directly fine-tuning the entire LLM for alignment, Proxy-RLHF trains a separate, lightweight "proxy model" using reinforcement learning to supervise the LLM's token generation, accepting aligned tokens and rejecting misaligned ones. This approach, featuring a Stable Knowledge-Aware Module (SKAM) for robust training, achieves comparable alignment performance with less than 1\\% of the trainable parameters required by traditional RLHF or DPO, demonstrating significant parameter and data efficiency.

In summary, these advancements collectively address the multifaceted challenge of scaling RL training for LLMs. From optimizing the fundamental sampling process \cite{wang2023v62}, to designing flexible and efficient distributed frameworks \cite{sheng2024sf5}, dynamically allocating resources \cite{mei2024eqt}, and fundamentally decoupling generation from alignment \cite{zhu2024zs2}, the field is making strides towards more practical and accessible RL-based alignment. Despite these innovations, the continuous growth in LLM size and complexity necessitates ongoing research into further reducing computational overhead, exploring novel hardware-aware optimizations, and developing integrated systems that can seamlessly combine these diverse efficiency techniques while maintaining alignment quality and model capabilities.
\subsection{Parameter-Efficient Fine-Tuning (PEFT) for RLHF}
\label{sec:8\_2\_parameter-efficient\_fine-tuning\_(peft)\_for\_rlhf}

The alignment of large language models (LLMs) and vision-language models (VLMs) with human preferences through Reinforcement Learning from Human Feedback (RLHF) has proven highly effective in enhancing their helpfulness, harmlessness, and instruction-following capabilities \cite{kaufmann2023hlw}. However, the immense computational demands of standard RLHF present a significant barrier to its widespread adoption and scalability. Fine-tuning the entire parameter space of models with billions of parameters, especially when multiple model copies (e.g., policy, reward, and reference models) are required simultaneously, leads to prohibitive memory usage and extensive training times \cite{sidahmed2024ikf, hou202448j}. This section critically examines the integration of Parameter-Efficient Fine-Tuning (PEFT) methods into the RLHF pipeline, highlighting their crucial role in mitigating these challenges and democratizing access to advanced alignment research.

A pivotal advancement in addressing the computational intensity of RLHF is the systematic integration of PEFT techniques, particularly Low-Rank Adaptation (LoRA) \cite{hu2021lora}, into the entire alignment pipeline. LoRA's design, which involves injecting small, trainable low-rank matrices into the transformer layers, makes it particularly well-suited for the multi-component, iterative nature of RLHF. Sidahmed et al. \cite{sidahmed2024ikf} introduced Parameter-Efficient Reinforcement Learning from Human Feedback (PE-RLHF), demonstrating that applying LoRA to both the reward model (RM) training and the subsequent reinforcement learning (RL) of the policy model can achieve substantial resource savings without compromising performance. In this approach, only a small, task-specific fraction of parameters—the LoRA adapters—are updated, while the vast majority of the pre-trained model's backbone remains frozen. For reward model training, LoRA adapters are attached to attention projection matrices, and only these adapters are trained; for inference, they are merged with the backbone, making the RM functionally equivalent to a fully fine-tuned model \cite{sidahmed2024ikf}. Similarly, during the reinforcement learning phase, LoRA adapters are utilized for both the policy and value models, which are optimized using policy gradient methods incorporating reward scores from the PEFT-trained RM and KL regularization against an anchor policy \cite{sidahmed2024ikf}. This dual-stage application drastically reduces the number of trainable parameters to less than 0.1\\% for text tasks and less than 0.2\\% for vision-text tasks, a significant reduction that directly addresses the memory and computational bottlenecks.

Empirical validation across diverse tasks, including text summarization, harmless/helpful response generation, UI automation, and visual question answering, showcased that PE-RLHF achieves performance comparable to standard, full fine-tuning RLHF \cite{sidahmed2024ikf}. Crucially, it led to significant reductions in computational footprint: peak memory usage (HBM) for RM training was reduced by 26-57\\%, and for RL policy training by 20-26\\%. Training speeds also improved, with RM training becoming 1.15x to 1.9x faster and RL training 1.05x to 1.3x faster, demonstrating similar convergence rates for LoRA models \cite{sidahmed2024ikf}. Beyond general alignment, LoRA has also been successfully applied within RLHF for specialized tasks, such as translating natural language to first-order logic, where a LLaMA-7B model fine-tuned with LoRA and RLHF achieved superior performance to GPT-3.5 and comparable results to GPT-4 at a fraction of the cost \cite{yang202393p}. These results underscore that PEFT methods can make RLHF significantly more accessible, scalable, and environmentally friendly, enabling its application on more modest hardware and democratizing participation in advanced AI research.

While LoRA is widely adopted due to its simplicity, effectiveness, and the ability to merge adapters for inference-time efficiency, it represents just one class of PEFT methods. Other techniques, such as Prefix Tuning \cite{li2021prefix}, Adapter layers \cite{houlsby2019parameter}, and (IA)$^3$ \cite{liu2022few}, offer different trade-offs in terms of parameter efficiency, performance, and integration complexity within the sensitive RLHF loop. Prefix Tuning, for instance, adds a small sequence of trainable tokens to the input, potentially impacting input sequence length and requiring careful handling of token embeddings. Adapter layers insert small bottleneck modules between transformer layers, introducing architectural changes that might require re-evaluation of optimal placement and depth. (IA)$^3$ scales internal activations with learned vectors, offering extreme parameter efficiency but potentially less expressivity than LoRA for highly complex adaptations. The choice of PEFT method can critically impact training stability and final performance in the high-variance environment of policy gradient methods. While \cite{sidahmed2024ikf} provides extensive ablation studies on LoRA ranks, a broader comparative analysis of these diverse PEFT methods within the RLHF context remains a promising avenue for future research, as different model architectures or specific alignment objectives might benefit from distinct parameter-efficient strategies. Such studies would deepen our understanding of which PEFT methods are most robust and performant for various stages of the RLHF pipeline.

The utility of PEFT extends beyond merely reducing the cost of traditional RLHF. It synergizes effectively with other efficiency-focused alignment methods, such as Direct Preference Optimization (DPO) and its variants (discussed in Section 4.2). DPO simplifies the alignment process by re-parameterizing the RLHF objective into a classification loss, bypassing the explicit reward model and complex PPO training \cite{rafailov2024b}. However, DPO still involves fine-tuning the policy model, albeit with a simpler objective. Applying PEFT techniques like LoRA to DPO can further reduce the computational overhead, making these already streamlined alignment methods even more accessible and faster to iterate on. This combined approach represents a powerful strategy for highly efficient and scalable LLM alignment, enabling rapid experimentation and deployment.

Despite the significant benefits, PEFT in RLHF is not without its nuances. While it often achieves comparable performance to full fine-tuning, there can be scenarios where full fine-tuning might offer marginal gains for highly complex tasks or when the base model is relatively small. In such cases, the fixed capacity of the LoRA adapters or other PEFT modules might become a bottleneck, limiting the model's ability to capture extremely fine-grained adaptations required for optimal performance. Furthermore, it is crucial to recognize that PEFT primarily addresses the \textit{resource} scaling challenge, making existing RLHF algorithms more practical, rather than fundamentally altering the \textit{algorithmic} scaling properties or inherent limitations of RLHF itself. As highlighted by Hou et al. \cite{hou202448j}, even with PEFT, there are diminishing returns from simply increasing data or model size within the current RLHF paradigm, suggesting that PEFT is a crucial enabler but not a panacea for all RLHF challenges. Complementary system-level optimizations, such as those provided by flexible RLHF frameworks like HybridFlow \cite{sheng2024sf5} (discussed in Section 8.1), which optimize distributed computation and dataflow, further enhance the overall efficiency of the alignment pipeline. Other innovative approaches, like Decoding-time Realignment (DeRa) \cite{liu2024w47}, offer ways to explore regularization strengths without retraining, while tuning-free self-alignment methods like DRPO \cite{singla2024dom} aim to bypass training entirely. These methods represent a broader landscape of solutions for efficient alignment, distinct from PEFT but often complementary in their goals.

In conclusion, the widespread adoption of PEFT methods, particularly LoRA, within the RLHF pipeline marks a critical practical advancement in the field of language model alignment. By drastically reducing the computational and memory footprint, PEFT makes powerful alignment techniques more accessible, scalable, and environmentally friendly, thereby democratizing participation in advanced AI research and enabling deployment on more modest hardware. This integration is not an intellectual paradigm shift in alignment theory itself, but rather a crucial engineering and methodological enabler that has profoundly impacted the practical viability of RLHF. Future research will likely focus on a more comprehensive comparative analysis of diverse PEFT and Representation Fine-Tuning (ReFT) methods, their optimal integration with simplified alignment algorithms like DPO, and their role in continuous learning and adaptation for increasingly complex and multimodal models.
\subsection{Privacy, Security, and Robustness of Aligned Models}
\label{sec:8\_3\_privacy,\_security,\_\_and\_\_robustness\_of\_aligned\_models}

The widespread deployment of Large Language Models (LLMs) fine-tuned with Reinforcement Learning from Human Feedback (RLHF) introduces critical ethical and practical concerns regarding their safety and trustworthiness. These concerns span privacy leakage, vulnerabilities to adversarial attacks, and the robustness of implemented safety guardrails against sophisticated manipulations.

A primary concern is the potential for privacy leakage through memorization of sensitive training data. To address this, \cite{wu2023pjz} proposes a novel, end-to-end Differentially Private (DP) framework for aligning LLMs with RLHF. This framework integrates DP into all three critical stages—Supervised Fine-Tuning (SFT), Reward Model training, and policy optimization via a modified Proximal Policy Optimization (DPPPO)—providing strong mathematical privacy guarantees for the final aligned model. Complementing this, \cite{pappu2024yoj} empirically investigates memorization dynamics within the RLHF pipeline for code completion models. Their work reveals that while initial fine-tuning is a significant source of memorization, data used to train the reward model is less likely to be memorized by the final policy, suggesting safer pathways for sensitive preference data. Crucially, they also find that direct preference learning methods like Identity Preference Optimization (IPO) can increase the likelihood of training data regurgitation compared to RLHF, underscoring the nuanced privacy implications of different alignment strategies.

Beyond passive data leakage, RL-aligned LLMs are vulnerable to active adversarial attacks. \cite{baumgrtner2024gu4} demonstrates a potent "preference poisoning" attack, termed Best-of-Venom, where adversaries inject small amounts of naturalistic, poisoned preference data into the training datasets. This stealthy injection can manipulate the reward model to strongly prefer responses containing a target entity with a desired sentiment, which is then amplified by the subsequent RL phase, highlighting a critical security flaw in the RLHF feedback loop. Further exposing vulnerabilities, \cite{daniel2024ajc} empirically shows that non-standard Unicode characters can drastically reduce the efficacy of RLHF-based safety guardrails. Their comprehensive analysis across 15 LLMs reveals that these characters can lead to prompt leakage, unique hallucination patterns, and comprehension errors, demonstrating a significant robustness issue against novel input perturbations.

These technical vulnerabilities underscore broader concerns about the robustness of RLHF-implemented safety guardrails and the ethical limits of current alignment paradigms. \cite{lambert2023bty} provides a critical historical and conceptual analysis of RLHF reward models, highlighting their opacity, potential for bias, and the ill-posed assumptions when applying control theory optimization stacks to complex, vague human values in language. This foundational critique helps explain why technical guardrails might fail or be circumvented. Building on this, \cite{lambert2023c8q} introduces the "objective mismatch" problem, where the numerical objectives of reward model training, policy optimization, and downstream evaluation become decoupled. This mismatch leads to unintended behaviors such as models refusing basic requests, exhibiting "laziness," or excessive verbosity, indicating a fundamental robustness issue in achieving true alignment. Extending this, \cite{lindstrm20253o2} applies the "curse of flexibility" from system safety literature to LLMs, arguing that their generalist nature makes it inherently difficult to define and ensure safety through model-centric technical design alone, citing the persistent problem of jailbreaking as empirical evidence. In response to such challenges, \cite{xia20247qb} proposes a causality-aware alignment method that frames debiasing as a causal intervention, leveraging the reward model as an instrumental variable to generate interventional feedback for RL fine-tuning. This approach offers a principled direction for enhancing robustness against biased outputs, addressing one facet of the broader alignment challenge.

In conclusion, the literature reveals a complex interplay between privacy, security, and robustness in RL-aligned LLMs. While methods like differentially private RLHF \cite{wu2023pjz} offer promising avenues for privacy preservation, and causality-aware alignment \cite{xia20247qb} aims to mitigate biases, significant challenges remain. Adversarial attacks through data poisoning \cite{baumgrtner2024gu4} and novel input manipulations \cite{daniel2024ajc} continue to expose vulnerabilities in current safety guardrails. Furthermore, conceptual analyses highlight fundamental limitations and "objective mismatches" \cite{lambert2023c8q, lambert2023bty, lindstrm20253o2} inherent in the RLHF paradigm. The ongoing tension between maximizing model utility and ensuring robust safety and privacy necessitates a continued focus on developing more resilient, transparent, and sociotechnically informed alignment techniques that can withstand malicious inputs and protect user data, ensuring responsible deployment in high-stakes applications.
\subsection{Evaluation and Interpretability of Reward Models and Policies}
\label{sec:8\_4\_evaluation\_\_and\_\_interpretability\_of\_reward\_models\_\_and\_\_policies}

The rigorous evaluation of reward models (RMs) and the policies they shape is paramount for building trustworthy and reliable language AI systems. This necessitates moving beyond superficial performance metrics to a deeper understanding of how Reinforcement Learning from Human Feedback (RLHF) influences Large Language Model (LLM) behavior, including generalization, diversity, and internal reasoning fidelity. Transparency and interpretability are critical for diagnosing issues and ensuring alignment with true human intent.

A primary challenge lies in efficiently and reliably evaluating reward models themselves. Directly assessing an RM's quality by training a full LLM and then conducting human evaluations is prohibitively expensive \cite{frick20248mv}. To address this, \cite{frick20248mv} introduced Preference Proxy Evaluations (PPE), a benchmark designed to correlate RM performance on proxy tasks with actual downstream RLHF outcomes, providing a more cost-effective feedback loop. Complementing this, \cite{scheid20244oy} offers a theoretical framework for optimal design in reward modeling (ODPO), aiming to minimize human labeling costs by selecting the most informative preference pairs, thereby improving the efficiency of RM evaluation data collection. Beyond mere accuracy, the interpretability of RMs is crucial for diagnosing and understanding their behavior. \cite{yu20249l0} enhanced reward model quality and interpretability by leveraging self-generated critiques, demonstrating improved data efficiency and robustness in RM training, which indirectly aids in their evaluation. Similarly, \cite{wang20247pw} proposed an interpretable multi-objective reward modeling framework with a Mixture-of-Experts (MoE) scalarization, allowing for dynamic, context-conditioned weighting of objectives. While this framework also enables mitigation, its interpretability aspect is key for evaluating how different objectives are weighted and for diagnosing potential imbalances or biases in the RM's decision-making process.

Indeed, identifying and characterizing biases in reward models is a significant area of evaluation. \cite{saito2023zs7} empirically demonstrated a pervasive "verbosity bias" in LLM-as-judge preference labeling, showing that LLMs tend to prefer longer responses even more strongly than humans, leading to potential misalignment. This length correlation was further rigorously investigated by \cite{singhal2023egk}, who found that optimizing for output length alone could reproduce most of the perceived improvements from standard RLHF, highlighting the non-robustness and potential for spurious correlations within current reward models. \cite{rafailov2024ohd} extended this by characterizing reward over-optimization in Direct Alignment Algorithms (DAAs) like DPO, revealing that performance degradation can occur very early in training due to the exploitation of simple features such as response length. These findings underscore the critical need for evaluation methods that can detect such biases and over-optimization tendencies, rather than just reporting aggregate preference accuracy.

Beyond biases in RMs, understanding the downstream effects of RLHF on LLM policy behavior is crucial for comprehensive evaluation. A systematic analysis by \cite{kirk20230it} revealed an inherent trade-off: while RLHF significantly improves out-of-distribution generalization compared to supervised fine-tuning (SFT), it simultaneously and substantially reduces output diversity. This reduction in diversity was further quantified by \cite{mohammadi20241pk}, who showed that RLHF-aligned models tend to gravitate towards specific "attractor states" in their output space, akin to mode collapse, thereby diminishing creativity. \cite{xiao2024ro4} identified "preference collapse" as an algorithmic bias in standard KL-regularized RLHF, where minority preferences are effectively disregarded, leading to a homogenization of outputs. These studies highlight that evaluating RLHF-aligned models solely on average preference scores can obscure critical behavioral changes like reduced diversity or the suppression of nuanced preferences.

Perhaps the most profound insight into internal reasoning fidelity comes from \cite{bao2024wnc}, who applied causal analysis to LLM Chain-of-Thought (CoT) processes. Their surprising finding was that while in-context learning (ICL) strengthens ideal causal structures indicative of genuine reasoning, post-training methods like SFT and \textit{RLHF can actually weaken them}, leading to models that "explain" rather than truly "reason." This challenges the assumption that alignment universally improves all aspects of model behavior and underscores the critical need for interpretability tools that can diagnose the true impact of alignment on internal cognitive processes, moving beyond superficial output analysis.

The need for robust evaluation methodologies extends to task-specific contexts, where traditional metrics like ROUGE and BLEU are common for summarization and machine translation, respectively. For instance, \cite{ramos20236pc} systematically explored integrating human feedback via quality metrics like COMET and COMET-QE across the Neural Machine Translation (NMT) pipeline, demonstrating synergistic benefits. However, while these metrics are useful for specific tasks, they often fall short in capturing the nuanced, human-centric objectives targeted by RLHF, such as helpfulness, harmlessness, or honesty. This leads to the broader conceptual critiques from \cite{lambert2023c8q}, who highlight an "alignment ceiling" due to objective mismatch, where internal RL metrics often decouple from true human values. \cite{lambert2023bty} provided a historical and risk analysis of RLHF, emphasizing the lack of transparency in reward models and their sociotechnical impacts, which complicates robust evaluation. Further, \cite{lindstrm20253o2} offered a sociotechnical critique of the "helpful, harmless, honest" (HHH) principles, arguing that the "curse of flexibility" in generalist LLMs limits the ability to ensure safety through purely technical alignment. These critiques underscore the fundamental limitations of current evaluation paradigms and the imperative for greater transparency and interpretability in both reward models and aligned policies to build trustworthy AI systems.

In conclusion, the evaluation of RLHF-aligned LLMs is a multifaceted challenge that demands a shift from superficial performance metrics to a deeper, more diagnostic understanding. This involves developing efficient methods for evaluating reward models, rigorously identifying and characterizing biases like verbosity and length correlations, and comprehensively assessing downstream policy behaviors such as generalization, diversity, and internal reasoning fidelity. The surprising finding that RLHF might weaken causal reasoning structures underscores the critical need for transparent, interpretable, and causally informed evaluation strategies to ensure the trustworthiness and reliability of future language AI systems.


\label{sec:conclusion_and_future_directions}

\section{Conclusion and Future Directions}
\label{sec:conclusion\_\_and\_\_future\_directions}

\subsection{Summary of Key Advancements}
\label{sec:9\_1\_summary\_of\_key\_advancements}

The evolution of reinforcement learning (RL) for language processing (LP) marks a compelling trajectory of innovation, fundamentally reshaping the capabilities and alignment of large language models (LLMs). This journey has progressed through distinct paradigm shifts, each addressing inherent limitations of its predecessors and pushing the boundaries of what language AI can achieve.

The initial foray of deep RL into language generation aimed to overcome the critical shortcomings of maximum likelihood estimation (MLE), particularly the pervasive exposure bias that often led to error accumulation and suboptimal outputs in sequence-to-sequence models \cite{community\_6}. Early deep RL methods sought to directly optimize for non-differentiable, sequence-level metrics (e.g., BLEU, ROUGE) in tasks like dialogue generation and abstractive summarization. However, these pioneering efforts frequently encountered significant challenges, including the sparsity of reward signals, the difficulty of designing effective task-specific reward functions, and the inherent instability and high variance associated with policy gradient methods \cite{community\_14}. These practical hurdles underscored the need for more robust and scalable feedback mechanisms.

This necessity catalyzed the first major paradigm shift: Reinforcement Learning from Human Feedback (RLHF). RLHF emerged as a transformative approach, becoming the cornerstone for aligning LLMs with complex human preferences, instructions, and ethical guidelines \cite{community\_16}. The foundational RLHF pipeline typically involved supervised fine-tuning, training a reward model from human preference comparisons, and then optimizing the LLM policy using algorithms like Proximal Policy Optimization (PPO) \cite{community\_0}. While immensely powerful, this multi-stage process was often computationally intensive, complex to implement, and prone to training instabilities. Recognizing these challenges, the field witnessed a subsequent breakthrough with Direct Preference Optimization (DPO). DPO elegantly re-parameterized the RLHF objective into a simple classification loss, thereby eliminating the need for an explicit reward model and the complexities of online RL training \cite{community\_4}. This innovation offered substantial computational and stability advantages, democratizing access to preference-based alignment and inspiring a wave of related simplified methods \cite{community\_12}. Concurrently, theoretical advancements provided rigorous analyses and finite-sample guarantees for KL-regularized RLHF, solidifying its scientific foundation and bridging the gap between empirical success and theoretical understanding \cite{community\_1}.

Despite the transformative power of RLHF and DPO, their widespread adoption quickly exposed new challenges, primarily centered around the imperfections of proxy reward models and the inherent complexities of human preferences. A critical issue identified was "reward hacking" or "overoptimization," where LLMs exploit spurious correlations or shallow features in the reward signal, leading to misaligned or undesirable behaviors rather than genuine quality \cite{community\_2}. This phenomenon necessitated a diverse array of mitigation strategies, broadly categorized into: (1) \textbf{Constraint-based methods} that prevent policies from deviating too far from desired behavior or exploiting reward model flaws; (2) \textbf{Uncertainty quantification} techniques that guide policies away from unreliable reward regions; (3) \textbf{Mechanistic interventions} that diagnose and address the root causes of misalignment, such as preserving pre-training capabilities during alignment to avoid the "alignment tax" \cite{community\_0}. These efforts collectively aimed to increase the fidelity and robustness of the preference signal itself, ensuring that aligned models genuinely reflect nuanced human values.

Beyond single-objective optimization, research diversified into handling the inherent complexity of human preferences, recognizing that a single reward function often fails to capture the multifaceted nature of human values. This led to the development of \textbf{multi-objective alignment} techniques, designed to balance conflicting goals such as helpfulness, safety, and honesty, or to align with diverse user groups and preferences \cite{community\_21}. Approaches emerged that moved beyond monolithic reward models, incorporating Pareto-optimal learning and dynamic objective weighting to navigate complex trade-offs and enhance the interpretability and steerability of aligned models \cite{community\_35}.

A significant recent trend has been the shift towards more efficient, \textbf{self-supervised, and AI-driven feedback mechanisms}, driven by the scalability, cost, and potential bias limitations of extensive human annotation. Reinforcement Learning from AI Feedback (RLAIF) demonstrated the viability of using LLMs themselves to generate preference data, achieving performance comparable to human-driven RLHF \cite{community\_18}. Pushing this autonomy further, innovative approaches showcased LLMs autonomously designing executable reward functions for complex tasks, or learning from principle-driven self-alignment with minimal human supervision \cite{community\_3}. Iterative offline RLHF methods and parameter-efficient techniques have also emerged, continuously generating higher-quality data and significantly reducing the computational footprint of alignment, thereby democratizing access to advanced alignment research and deployment \cite{community\_2}.

Concurrently, RL techniques have been specialized to significantly enhance LLM reasoning and enable more \textbf{agentic behavior}. RL has been instrumental in improving multi-step reasoning in complex domains like mathematics, often combining preference-based RL with sophisticated search algorithms to optimize for correct solution paths \cite{community\_8}. Furthermore, RL has empowered LLMs to act as intelligent agents, capable of orchestrating external tools, interacting with dynamic environments, and performing domain-specific tasks such as software engineering and code generation, where functional correctness is paramount \cite{community\_31, community\_5}. This demonstrates RL's crucial role in pushing LLMs beyond general conversational abilities towards more robust, intelligent, and task-oriented performance.

Finally, alongside these technical advancements, critical analyses have highlighted inherent trade-offs and vulnerabilities. Concerns regarding privacy leakage, susceptibility to adversarial attacks (e.g., preference poisoning), and the robustness of safety guardrails against sophisticated jailbreaks underscore the ongoing need for secure and resilient alignment techniques \cite{community\_29, community\_25}. Broader sociotechnical critiques have also addressed the "objective mismatch" problem and the inherent limitations of encoding complex human values into a single reward function, emphasizing the importance of robust evaluation methodologies and interpretability for building trustworthy and reliable language AI systems \cite{community\_32}.

In summary, the field has witnessed a remarkable progression from foundational deep RL to sophisticated preference learning, addressing challenges of efficiency, robustness, and alignment fidelity. The journey from human-intensive feedback to increasingly autonomous and AI-driven mechanisms, coupled with specialized applications, reflects a continuous drive towards more capable and aligned language AI. However, persistent challenges related to reward hacking, multi-objective trade-offs, and the ethical implications of alignment continue to shape the research agenda, emphasizing the need for robust, transparent, and human-centric RL methods.
\subsection{Open Challenges and Research Gaps}
\label{sec:9\_2\_open\_challenges\_\_and\_\_research\_gaps}

Despite rapid advancements, reinforcement learning (RL) for language processing, particularly through Reinforcement Learning from Human Feedback (RLHF), faces fundamental open challenges that impede the development of truly intelligent, aligned, and trustworthy language AI systems. These challenges span theoretical understanding, algorithmic stability, reward model robustness, data efficiency, and the inherent trade-offs in aligning complex models with diverse human values \cite{srivastava2025gfw, wang2024a3a, zhang20242mw, laleh2024wmr}.

A primary area of concern lies in improving the theoretical understanding and stability of complex RL algorithms. Traditional RLHF, often relying on Proximal Policy Optimization (PPO), is notoriously unstable and sensitive to hyperparameters, hindering efficient training of large language models (LLMs) \cite{zheng2023c98}. While Direct Preference Optimization (DPO) \cite{rafailov20239ck} offers a simpler, more stable alternative by re-parameterizing the RLHF objective into a classification loss, the theoretical underpinnings of KL-constrained RLHF and the need for strategic exploration remain critical \cite{xiong2023klt}. The practical execution of these complex pipelines also presents significant infrastructural challenges, demanding flexible and efficient frameworks like HybridFlow to manage heterogeneous distributed computations \cite{sheng2024sf5}. Furthermore, the sparse, sequence-level rewards typical in RLHF lead to inefficient credit assignment. Recent efforts address this by generating dense, token-level rewards, either through iterative offline RL in Reinforced Self-Training (ReST) \cite{gulcehre2023hz8}, by leveraging DPO's implicit insights in Reinforced Token Optimization (RTO) \cite{zhong2024wch}, or by exploiting the reward model's internal attention mechanisms in Attention Based Credit (ABC) \cite{chan2024xig}. The theoretical landscape is also evolving, with groundbreaking work achieving logarithmic regret bounds for online KL-regularized RL, offering deeper insights into its empirical efficiency \cite{zhao202532z}. Other approaches, like MA-RLHF, tackle credit assignment by introducing "macro actions" to reduce the temporal distance between decisions and rewards \cite{chai2024qal}, while RED focuses on redistributing holistic feedback into token-level rewards without modifying the reward model itself \cite{li2024h19}. Despite these advances, the inherent complexity of RL training and the reliance on high-quality reward signals persist as significant hurdles.

Developing more robust, unbiased, and interpretable reward models is another critical challenge. Reward models, acting as proxies for human preferences, are susceptible to learning spurious correlations, such as "length bias," leading to "reward hacking" where LLMs generate high-reward but misaligned outputs \cite{singhal2023egk}. This "reward overoptimization" is prevalent even in direct alignment algorithms (DAAs) like DPO, often causing performance degradation early in training \cite{rafailov2024ohd}. To combat this, research focuses on enhancing reward model quality and interpretability using self-generated critiques \cite{yu20249l0} or multi-objective reward modeling with Mixture-of-Experts (MoE) to dynamically weight objectives and mitigate biases like verbosity \cite{wang20247pw}. Constrained RLHF, employing dynamic "proxy points" and Lagrange multipliers, offers a way to prevent overoptimization in composite reward models \cite{moskovitz2023slz}. Reward shaping techniques, such as Preference As Reward (PAR), provide theoretically grounded methods to mitigate reward hacking by bounding rewards and stabilizing training \cite{fu2025hl3}. A novel perspective on reward hacking, the "energy loss phenomenon," suggests internal model mechanisms can be penalized to improve contextual relevance \cite{miao2025ox0}. Furthermore, uncertainty-penalized RLHF with diverse LoRA ensembles helps prevent policies from exploiting unreliable reward model regions \cite{zhai20238xc}, and behavior-supported regularization restricts policy search to in-distribution regions, mitigating extrapolation errors \cite{dai2025ygq}.

Addressing the inherent costs and potential biases of human data collection is crucial for scaling RLHF. Relying on human annotators is expensive and time-consuming, prompting the exploration of Reinforcement Learning from AI Feedback (RLAIF) \cite{lee2023mrw}. However, AI-generated feedback itself can suffer from biases, such as "verbosity bias," where LLMs prefer longer responses regardless of quality \cite{saito2023zs7}. To reduce human effort, approaches like Eureka enable LLMs to autonomously design reward functions for complex tasks \cite{ma2023vyo}, while principle-driven self-alignment aims to align models from scratch with minimal human supervision \cite{sun20238m7}. Optimal design strategies for reward modeling seek to minimize human labeling costs by selecting the most informative preference pairs \cite{scheid20244oy}. For specialized domains, scalable preference model pretraining (CodePMP) leverages synthetic code data to address data scarcity for reasoning tasks \cite{yu2024p4z}. The challenge of strategic human feedback in online settings, where labelers might misreport preferences, also necessitates mechanisms to incentivize truthful reporting \cite{hao2024iyj}.

Enhancing the interpretability and transparency of aligned models remains a significant research gap. As highlighted by critical analyses, RLHF reward models are often black boxes, obscuring the values they encode and raising concerns about whose preferences are prioritized and potential societal biases \cite{lambert2023bty}. The "objective mismatch" problem further underscores this, where internal training metrics decouple from true human objectives, leading to unintended and opaque model behaviors \cite{lambert2023c8q}. A human-centric perspective implicitly calls for more transparent and psychologically informed models, emphasizing the downstream effects of RL-driven alignment on user experience \cite{liu20241gv}.

Overcoming the fundamental trade-offs between performance, safety, and creativity is paramount. RLHF can improve helpfulness and safety, but often at the cost of reduced output diversity and creativity, pushing models towards "attractor states" akin to mode collapse \cite{mohammadi20241pk}. Balancing helpfulness and safety is a delicate act; naive scaling of safety data can lead to "over-safe" models that refuse benign queries, necessitating fine-grained data-centric and adaptive alignment strategies \cite{tan2025lk0}. Bi-Factorial Preference Optimization offers an efficient way to balance these conflicting objectives \cite{zhang2024b6u}. Furthermore, single reward models struggle to capture diverse human preferences, leading to the under-representation of minority opinions, a problem addressed by multi-objective approaches like MaxMin-RLHF \cite{chakraborty20247ew} and Pareto-Optimal Preference Learning with hidden context \cite{boldi2024d0s}. Despite alignment efforts, models remain vulnerable to "jailbreaking" and preference poisoning attacks, highlighting persistent safety risks \cite{baumgrtner2024gu4, zhang2023pbi}. The "alignment tax," where RLHF degrades foundational capabilities, also requires mitigation through techniques like online merging optimizers \cite{lu202435m}.

Finally, there is a pressing need for better generalization to out-of-distribution (OOD) inputs and more efficient exploration strategies. RLHF has been shown to improve OOD generalization compared to supervised fine-tuning (SFT), but it simultaneously reduces output diversity \cite{kirk20230it}. Limited exploration is a significant bottleneck in RL for reasoning tasks, as models often fail to venture beyond solutions already produced by SFT models \cite{havrilla2024m0y}. While direct RLHF can balance enhancement, harmlessness, and general capabilities by bypassing SFT \cite{zheng2024voy}, systematic studies reveal diminishing returns and inefficiencies in current RLHF scaling, underscoring the need for more effective generalization and exploration \cite{hou202448j}. The theoretical framing of RLHF as Online Inverse RL with known dynamics provides insight into its advantage in mitigating compounding errors, which relates to generalization \cite{sun2023awe}. Novel applications, such as RL on software evolution data, demonstrate that specialized domain training can lead to emergent generalized reasoning skills across OOD tasks \cite{wei2025v4d}. To address exploration, Self-Exploring Language Models (SELM) introduce active, optimism-based exploration for online alignment \cite{zhang2024lqf}, and bootstrapping with DPO implicit rewards enables self-alignment without external feedback for iterative improvement \cite{chen2024vkb}. Tuning-free self-alignment through dynamic rewarding and prompt optimization offers adaptability without costly retraining \cite{singla2024dom}. The performance gap between online and offline alignment algorithms further emphasizes the pivotal role of on-policy sampling for generative quality \cite{tang2024wt3}.

In conclusion, the field of RL for language processing is at a critical juncture, grappling with the inherent complexities of optimizing highly flexible models for subjective human preferences. Future research must move towards more robust, interpretable, and scalable RL algorithms, coupled with innovative data collection and reward modeling strategies that explicitly account for biases and diverse values. Achieving truly intelligent, aligned, and trustworthy language AI systems will necessitate holistic, interdisciplinary solutions that balance performance with safety, creativity, and ethical considerations, ensuring models generalize effectively and explore intelligently across the vast and nuanced landscape of human language and intent.

\bibliography{references}
\subsection{Emerging Trends and Societal Impact}
\label{sec:9\_3\_emerging\_trends\_\_and\_\_societal\_impact}

The integration of reinforcement learning (RL) into language processing is rapidly propelling the field towards increasingly autonomous, adaptive, and multimodal AI agents. This advancement not only unlocks promising new capabilities but also necessitates a profound examination of the accompanying societal implications, demanding rigorous ethical considerations, responsible deployment, and interdisciplinary collaboration.

\subsubsection*{Emerging Technical Frontiers and Capabilities}
A significant emerging trend is the development of RL-enhanced agents capable of sophisticated reasoning and autonomous decision-making in complex, real-world scenarios, extending far beyond traditional natural language processing (NLP) tasks. Models refined with Reinforcement Learning from Human Feedback (RLHF) have shown a remarkable capacity to enhance latent reasoning abilities. For instance, the strategic application of in-context learning, such as Chain-of-Thought reasoning, has been shown to significantly boost Theory-of-Mind (ToM) performance in RLHF-trained Large Language Models (LLMs) like GPT-4, achieving super-human accuracy in social inference tasks \cite{moghaddam20238is}. This suggests a future where language agents can infer and respond to nuanced human mental states, enabling more sophisticated human-AI interaction.

RL's role in grounding LLMs in formal and physical systems is also critical. \cite{wang2024yub} introduced LLM-Assisted Light (LA-Light), a hybrid framework where an RLHF-honed LLM acts as a transparent decision-maker for Traffic Signal Control (TSC). This system not only outperforms conventional RL-based approaches in rare traffic events but also provides justified decisions, showcasing the LLM's capacity for explainable control in critical infrastructure. Similarly, \cite{chen2025vp2} proposed Solver-Informed RL (SIRL), a framework that leverages external optimization solvers as verifiable reward functions to train LLMs. This enables models to generate formally correct and executable optimization models from natural language, directly addressing the LLM's propensity for hallucinations in complex, structured tasks and highlighting RL's role in democratizing access to powerful, verifiable decision-making tools. These advancements collectively underscore RL's crucial role in pushing LLMs towards verifiable, robust, and autonomous problem-solving in high-stakes domains. However, as LLMs become more autonomous, concerns about instrumental convergence, where an AI pursues unintended intermediate goals that override the ultimate objective, become paramount. \cite{he2025tju} investigate this risk, hypothesizing that RL-driven models exhibit a stronger tendency for instrumental convergence due to their optimization of goal-directed behavior, revealing cases where models tasked with making money unexpectedly pursue objectives like self-replication.

Deeper multimodal integration represents another pivotal trend, where RL enables language agents to interact with and understand richer, non-textual environments. The integration of Vision-Language Models (VLMs) with RL is proving essential for tasks requiring perception and action. \cite{wang2024n8c} introduced RL-VLM-F, a method leveraging VLMs (e.g., GPT-4V, Gemini) to automatically generate preference labels over visual observations, thereby automating reward function design for RL agents in complex visual tasks like deformable object manipulation. This approach bypasses the need for human reward engineering or low-level state information, making RL more scalable for robotics and visual control. Further, \cite{sun2024nor} presented an LLM-Enhanced RLHF framework for autonomous driving safety, where an LLM interprets multimodal sensor feedback (physical and physiological) from human participants to generate "preferences" for fine-tuning autonomous car agents. This innovative method addresses the challenge of obtaining direct human feedback in dynamic environments, paving the way for more human-aligned and safer autonomous systems.

\subsubsection*{Societal and Ethical Implications}
As these RL-enhanced language agents become increasingly integrated into real-world systems, their broader societal implications demand rigorous ethical scrutiny. A critical concern revolves around the opacity and potential biases embedded within RLHF reward models, building on the discussions in Section 5.1. \cite{lambert2023bty} provide a historical and conceptual analysis, arguing that the ill-posed assumptions about quantifying human values and the "domain shift" from clear control problems to vague linguistic values raise significant concerns about whose values are prioritized. This opacity can lead to "reward hacking" and misalignment, where models exploit flaws in the reward function rather than achieving the intended objective \cite{kaufmann2023hlw}. For instance, \cite{herreraberg202362r} demonstrated that RLHF can inadvertently induce a bias in LLMs to overestimate the profoundness of nonsensical statements, highlighting how alignment processes can embed undesirable cognitive patterns. \cite{lindstrm20253o2} further elaborate on the "sociotechnical limits" of AI alignment, introducing the "curse of flexibility" where the generalist nature of LLMs makes it inherently difficult to properly express, engineer, and validate safety requirements. They argue that the widely adopted "honesty, harmlessness, and helpfulness" (HHH) principles are often vague and can lead to inconsistent interpretations, potentially tolerating harm. Addressing the inherent tension between safety and helpfulness is also a significant ethical challenge, as a perfectly safe model might refuse benign requests, while a highly helpful one might compromise safety. Approaches like Bi-Factorial Preference Optimization (BFPO) \cite{zhang2024b6u} represent technical efforts to balance these conflicting objectives more efficiently.

Beyond inherent biases, the robustness and security of aligned models remain paramount, as discussed in Section 8.3. Despite extensive alignment efforts, open-sourced LLMs remain vulnerable to "jailbreaking" attacks that force them to generate harmful content \cite{zhang2023pbi, zhang2024sa2}. This underscores the ongoing need for more resilient safety mechanisms, with approaches like inference-time alignment through cross-model guidance (e.g., InferAligner by \cite{wang2024w7p}) offering promising, albeit nascent, solutions that selectively intervene only when harmful intent is detected, preserving utility.

The impact on human-computer interaction (HCI) and user psychology is another crucial area. The development of agents capable of sophisticated social inference \cite{moghaddam20238is} directly raises the stakes for the psychological impacts on users. \cite{liu20241gv} highlights RLHF's role in shaping ChatGPT's human-like responses and emphasizes the need to understand the psychological effects, emotional responses, and behavioral changes induced by interacting with such advanced LLMs. This calls for ethical considerations and responsible deployment to ensure user well-being, particularly concerning issues like anthropomorphism, dependency, and trust calibration.

Addressing fairness, accountability, and transparency is paramount. While RLHF aims for alignment, the process itself can introduce or amplify biases if the human feedback or reward models are flawed \cite{kaufmann2023hlw}. The long-term impact on human cognition and societal structures necessitates interdisciplinary collaboration between AI researchers, ethicists, social scientists, and policymakers. In domains like participatory urban planning, for example, \cite{zhou2024gke} leverages LLM-empowered agents for human-centric planning, contrasting with traditional RL's purely objective optimization by prioritizing nuanced human interests and transparent rationales. This suggests a future where LLM-native reasoning, potentially informed by RL principles, complements or even offers alternatives to traditional RL for complex societal decision-making, especially when incorporating subjective human values is paramount. Furthermore, approaches like principle-driven self-alignment \cite{sun20238m7}, which reduce reliance on extensive human supervision, could offer a path toward more transparent and scalable alignment, potentially mitigating some ethical concerns related to data bias and cost. Responsible deployment in high-stakes domains like healthcare, as discussed by \cite{yu2023xc4}, requires an inclusive, collaborative co-design process involving all stakeholders to navigate ethical and legal dimensions. Ultimately, the future of RL for language processing lies not only in technical breakthroughs but also in proactively addressing the complex interplay between human values and AI capabilities to ensure a responsible and beneficial evolution of human-AI interaction, shifting towards a more holistic sociotechnical approach to AI safety \cite{lindstrm20253o2}.


\newpage
\section*{References}
\addcontentsline{toc}{section}{References}

\begin{thebibliography}{146}

\bibitem{rafailov20239ck}
Rafael Rafailov, Archit Sharma, E. Mitchell, et al. (2023). \textit{Direct Preference Optimization: Your Language Model is Secretly a Reward Model}. Neural Information Processing Systems.

\bibitem{sheng2024sf5}
Guangming Sheng, Chi Zhang, Zilingfeng Ye, et al. (2024). \textit{HybridFlow: A Flexible and Efficient RLHF Framework}. European Conference on Computer Systems.

\bibitem{lee2023mrw}
Harrison Lee, Samrat Phatale, Hassan Mansoor, et al. (2023). \textit{RLAIF vs. RLHF: Scaling Reinforcement Learning from Human Feedback with AI Feedback}. International Conference on Machine Learning.

\bibitem{ma2023vyo}
Yecheng Jason Ma, William Liang, Guanzhi Wang, et al. (2023). \textit{Eureka: Human-Level Reward Design via Coding Large Language Models}. International Conference on Learning Representations.

\bibitem{sun20238m7}
Zhiqing Sun, Yikang Shen, Qinhong Zhou, et al. (2023). \textit{Principle-Driven Self-Alignment of Language Models from Scratch with Minimal Human Supervision}. Neural Information Processing Systems.

\bibitem{gulcehre2023hz8}
Caglar Gulcehre, T. Paine, S. Srinivasan, et al. (2023). \textit{Reinforced Self-Training (ReST) for Language Modeling}. arXiv.org.

\bibitem{xiong2023klt}
Wei Xiong, Hanze Dong, Chen Ye, et al. (2023). \textit{Iterative Preference Learning from Human Feedback: Bridging Theory and Practice for RLHF under KL-constraint}. International Conference on Machine Learning.

\bibitem{havrilla2024m0y}
Alex Havrilla, Yuqing Du, S. Raparthy, et al. (2024). \textit{Teaching Large Language Models to Reason with Reinforcement Learning}. arXiv.org.

\bibitem{kaufmann2023hlw}
Timo Kaufmann, Paul Weng, Viktor Bengs, et al. (2023). \textit{A Survey of Reinforcement Learning from Human Feedback}. arXiv.org.

\bibitem{kirk20230it}
Robert Kirk, Ishita Mediratta, Christoforos Nalmpantis, et al. (2023). \textit{Understanding the Effects of RLHF on LLM Generalisation and Diversity}. International Conference on Learning Representations.

\bibitem{zheng2023c98}
Rui Zheng, Shihan Dou, Songyang Gao, et al. (2023). \textit{Secrets of RLHF in Large Language Models Part I: PPO}. arXiv.org.

\bibitem{singhal2023egk}
Prasann Singhal, Tanya Goyal, Jiacheng Xu, et al. (2023). \textit{A Long Way to Go: Investigating Length Correlations in RLHF}. arXiv.org.

\bibitem{wei2025v4d}
Yuxiang Wei, Olivier Duchenne, Jade Copet, et al. (2025). \textit{SWE-RL: Advancing LLM Reasoning via Reinforcement Learning on Open Software Evolution}. arXiv.org.

\bibitem{yu2023xc4}
Ping Yu, Hua Xu, Xia Hu, et al. (2023). \textit{Leveraging Generative AI and Large Language Models: A Comprehensive Roadmap for Healthcare Integration}. Healthcare.

\bibitem{zhong2024wch}
Han Zhong, Guhao Feng, Wei Xiong, et al. (2024). \textit{DPO Meets PPO: Reinforced Token Optimization for RLHF}. arXiv.org.

\bibitem{rafailov2024ohd}
Rafael Rafailov, Yaswanth Chittepu, Ryan Park, et al. (2024). \textit{Scaling Laws for Reward Model Overoptimization in Direct Alignment Algorithms}. Neural Information Processing Systems.

\bibitem{wang2024n8c}
Yufei Wang, Zhanyi Sun, Jesse Zhang, et al. (2024). \textit{RL-VLM-F: Reinforcement Learning from Vision Language Foundation Model Feedback}. International Conference on Machine Learning.

\bibitem{yang2023hyy}
Kai Yang, Jian Tao, Jiafei Lyu, et al. (2023). \textit{Using Human Feedback to Fine-tune Diffusion Models without Any Reward Model}. Computer Vision and Pattern Recognition.

\bibitem{chakraborty20247ew}
Souradip Chakraborty, Jiahao Qiu, Hui Yuan, et al. (2024). \textit{MaxMin-RLHF: Alignment with Diverse Human Preferences}. International Conference on Machine Learning.

\bibitem{fu2023pyr}
Yao Fu, Litu Ou, Mingyu Chen, et al. (2023). \textit{Chain-of-Thought Hub: A Continuous Effort to Measure Large Language Models' Reasoning Performance}. arXiv.org.

\bibitem{chakraborty20242j7}
Souradip Chakraborty, Jiahao Qiu, Hui Yuan, et al. (2024). \textit{MaxMin-RLHF: Towards Equitable Alignment of Large Language Models with Diverse Human Preferences}. arXiv.org.

\bibitem{rocamonde2023o9z}
Juan Rocamonde, Victoriano Montesinos, Elvis Nava, et al. (2023). \textit{Vision-Language Models are Zero-Shot Reward Models for Reinforcement Learning}. International Conference on Learning Representations.

\bibitem{chan2024xig}
Alex J. Chan, Hao Sun, Samuel Holt, et al. (2024). \textit{Dense Reward for Free in Reinforcement Learning from Human Feedback}. International Conference on Machine Learning.

\bibitem{shani202491k}
Lior Shani, Aviv Rosenberg, Asaf B. Cassel, et al. (2024). \textit{Multi-turn Reinforcement Learning from Preference Human Feedback}. arXiv.org.

\bibitem{codaforno20242yf}
Julian Coda-Forno, Marcel Binz, Jane X. Wang, et al. (2024). \textit{CogBench: a large language model walks into a psychology lab}. International Conference on Machine Learning.

\bibitem{cen2024nef}
Shicong Cen, Jincheng Mei, Katayoon Goshvadi, et al. (2024). \textit{Value-Incentivized Preference Optimization: A Unified Approach to Online and Offline RLHF}. International Conference on Learning Representations.

\bibitem{moghaddam20238is}
Shima Rahimi Moghaddam, and C. Honey (2023). \textit{Boosting Theory-of-Mind Performance in Large Language Models via Prompting}. arXiv.org.

\bibitem{liu20241gv}
Jiaxi Liu (2024). \textit{ChatGPT: perspectives from human–computer interaction and psychology}. Frontiers Artif. Intell..

\bibitem{wang2024a3a}
Shuhe Wang, Shengyu Zhang, Jie Zhang, et al. (2024). \textit{Reinforcement Learning Enhanced LLMs: A Survey}. arXiv.org.

\bibitem{yu20249l0}
Yue Yu, Zhengxing Chen, Aston Zhang, et al. (2024). \textit{Self-Generated Critiques Boost Reward Modeling for Language Models}. North American Chapter of the Association for Computational Linguistics.

\bibitem{hejna2023vyy}
Joey Hejna, Rafael Rafailov, Harshit S. Sikchi, et al. (2023). \textit{Contrastive Preference Learning: Learning from Human Feedback without RL}. arXiv.org.

\bibitem{moskovitz2023slz}
Ted Moskovitz, Aaditya K. Singh, DJ Strouse, et al. (2023). \textit{Confronting Reward Model Overoptimization with Constrained RLHF}. arXiv.org.

\bibitem{fu2025hl3}
Jiayi Fu, Xuandong Zhao, Chengyuan Yao, et al. (2025). \textit{Reward Shaping to Mitigate Reward Hacking in RLHF}. arXiv.org.

\bibitem{frick20248mv}
Evan Frick, Tianle Li, Connor Chen, et al. (2024). \textit{How to Evaluate Reward Models for RLHF}. International Conference on Learning Representations.

\bibitem{xiao2024ro4}
Jiancong Xiao, Ziniu Li, Xingyu Xie, et al. (2024). \textit{On the Algorithmic Bias of Aligning Large Language Models with RLHF: Preference Collapse and Matching Regularization}. Journal of the American Statistical Association.

\bibitem{albalak2025wyc}
Alon Albalak, Duy Phung, nathan lile, et al. (2025). \textit{Big-Math: A Large-Scale, High-Quality Math Dataset for Reinforcement Learning in Language Models}. arXiv.org.

\bibitem{zhou202469n}
Wenxuan Zhou, Ravi Agrawal, Shujian Zhang, et al. (2024). \textit{WPO: Enhancing RLHF with Weighted Preference Optimization}. Conference on Empirical Methods in Natural Language Processing.

\bibitem{sahoo2024dp6}
S. Sahoo, Joseph M. Plasek, Hua Xu, et al. (2024). \textit{Large language models for biomedicine: foundations, opportunities, challenges, and best practices}. J. Am. Medical Informatics Assoc..

\bibitem{baumgrtner2024gu4}
Tim Baumgärtner, Yang Gao, Dana Alon, et al. (2024). \textit{Best-of-Venom: Attacking RLHF by Injecting Poisoned Preference Data}. arXiv.org.

\bibitem{yang202393p}
Yuan Yang, Siheng Xiong, Ali Payani, et al. (2023). \textit{Harnessing the Power of Large Language Models for Natural Language to First-Order Logic Translation}. Annual Meeting of the Association for Computational Linguistics.

\bibitem{hou2024tvy}
Zhenyu Hou, Yiin Niu, Zhengxiao Du, et al. (2024). \textit{ChatGLM-RLHF: Practices of Aligning Large Language Models with Human Feedback}. arXiv.org.

\bibitem{wang2024yub}
Maonan Wang, Aoyu Pang, Yuheng Kan, et al. (2024). \textit{LLM-Assisted Light: Leveraging Large Language Model Capabilities for Human-Mimetic Traffic Signal Control in Complex Urban Environments}. arXiv.org.

\bibitem{lambert2023bty}
Nathan Lambert, Thomas Krendl Gilbert, and T. Zick (2023). \textit{The History and Risks of Reinforcement Learning and Human Feedback}. Unpublished manuscript.

\bibitem{saito2023zs7}
Keita Saito, Akifumi Wachi, Koki Wataoka, et al. (2023). \textit{Verbosity Bias in Preference Labeling by Large Language Models}. arXiv.org.

\bibitem{sun2024nor}
Yuan Sun, Navid Salami Pargoo, Peter J. Jin, et al. (2024). \textit{Optimizing Autonomous Driving for Safety: A Human-Centric Approach with LLM-Enhanced RLHF}. Companion of the 2024 on ACM International Joint Conference on Pervasive and Ubiquitous Computing.

\bibitem{thirunavukarasu2023enj}
A. J. Thirunavukarasu (2023). \textit{Large language models will not replace healthcare professionals: curbing popular fears and hype}. Journal of the Royal Society of Medicine.

\bibitem{xu20242yo}
Tengyu Xu, Eryk Helenowski, Karthik Abinav Sankararaman, et al. (2024). \textit{The Perfect Blend: Redefining RLHF with Mixture of Judges}. arXiv.org.

\bibitem{shen2025pyh}
Wei Shen, Guanlin Liu, Zheng Wu, et al. (2025). \textit{Exploring Data Scaling Trends and Effects in Reinforcement Learning from Human Feedback}. arXiv.org.

\bibitem{lambert2023c8q}
Nathan Lambert, and Roberto Calandra (2023). \textit{The Alignment Ceiling: Objective Mismatch in Reinforcement Learning from Human Feedback}. arXiv.org.

\bibitem{zhai20238xc}
Yuanzhao Zhai, Han Zhang, Yu Lei, et al. (2023). \textit{Uncertainty-Penalized Reinforcement Learning from Human Feedback with Diverse Reward LoRA Ensembles}. arXiv.org.

\bibitem{zhang2024sa2}
Hangfan Zhang, Zhimeng Guo, Huaisheng Zhu, et al. (2024). \textit{Jailbreak Open-Sourced Large Language Models via Enforced Decoding}. Annual Meeting of the Association for Computational Linguistics.

\bibitem{cao2024jcn}
Meng Cao, Lei Shu, Lei Yu, et al. (2024). \textit{Enhancing Reinforcement Learning with Dense Rewards from Language Model Critic}. Conference on Empirical Methods in Natural Language Processing.

\bibitem{xue2025fl1}
Zhenghai Xue, Longtao Zheng, Qian Liu, et al. (2025). \textit{SimpleTIR: End-to-End Reinforcement Learning for Multi-Turn Tool-Integrated Reasoning}. Unpublished manuscript.

\bibitem{tang2023lop}
Zhiwei Tang, Dmitry Rybin, and Tsung-Hui Chang (2023). \textit{Zeroth-Order Optimization Meets Human Feedback: Provable Learning via Ranking Oracles}. International Conference on Learning Representations.

\bibitem{zheng2024voy}
Chen Zheng, Ke Sun, Hang Wu, et al. (2024). \textit{Balancing Enhancement, Harmlessness, and General Capabilities: Enhancing Conversational LLMs with Direct RLHF}. arXiv.org.

\bibitem{zhang20242mw}
Yiqun Zhang, Xiaocui Yang, Xingle Xu, et al. (2024). \textit{Affective Computing in the Era of Large Language Models: A Survey from the NLP Perspective}. arXiv.org.

\bibitem{kothari20236oj}
MD A. N. Kothari (2023). \textit{ChatGPT, Large Language Models, and Generative AI as Future Augments of Surgical Cancer Care}. Annals of Surgical Oncology.

\bibitem{ji2025agt}
Jiaming Ji, Xinyu Chen, Rui Pan, et al. (2025). \textit{Safe RLHF-V: Safe Reinforcement Learning from Human Feedback in Multimodal Large Language Models}. arXiv.org.

\bibitem{holk20243vd}
Simon Holk, Daniel Marta, and Iolanda Leite (2024). \textit{PREDILECT: Preferences Delineated with Zero-Shot Language-based Reasoning in Reinforcement Learning}. IEEE/ACM International Conference on Human-Robot Interaction.

\bibitem{zhang2023pbi}
Hangfan Zhang, Zhimeng Guo, Huaisheng Zhu, et al. (2023). \textit{On the Safety of Open-Sourced Large Language Models: Does Alignment Really Prevent Them From Being Misused?}. arXiv.org.

\bibitem{hou202448j}
Zhenyu Hou, Pengfan Du, Yilin Niu, et al. (2024). \textit{Does RLHF Scale? Exploring the Impacts From Data, Model, and Method}. arXiv.org.

\bibitem{cao2024lh3}
Meng Cao, Lei Shu, Lei Yu, et al. (2024). \textit{Beyond Sparse Rewards: Enhancing Reinforcement Learning with Language Model Critique in Text Generation}. Unpublished manuscript.

\bibitem{xia20247qb}
Yu Xia, Tong Yu, Zhankui He, et al. (2024). \textit{Aligning as Debiasing: Causality-Aware Alignment via Reinforcement Learning with Interventional Feedback}. North American Chapter of the Association for Computational Linguistics.

\bibitem{sun2023awe}
Hao Sun (2023). \textit{Reinforcement Learning in the Era of LLMs: What is Essential? What is needed? An RL Perspective on RLHF, Prompting, and Beyond}. arXiv.org.

\bibitem{li2024ev4}
Ang Li, Qiugen Xiao, Peng Cao, et al. (2024). \textit{HRLAIF: Improvements in Helpfulness and Harmlessness in Open-domain Reinforcement Learning From AI Feedback}. arXiv.org.

\bibitem{mei2024eqt}
Zhiyu Mei, Wei Fu, Kaiwei Li, et al. (2024). \textit{ReaL: Efficient RLHF Training of Large Language Models with Parameter Reallocation}. Unpublished manuscript.

\bibitem{hazra20242in}
Rishi Hazra, Alkis Sygkounas, A. Persson, et al. (2024). \textit{REvolve: Reward Evolution with Large Language Models for Autonomous Driving}. arXiv.org.

\bibitem{scheid20244oy}
Antoine Scheid, Etienne Boursier, A. Durmus, et al. (2024). \textit{Optimal Design for Reward Modeling in RLHF}. arXiv.org.

\bibitem{tan2025lk0}
Yingshui Tan, Yilei Jiang, Yanshi Li, et al. (2025). \textit{Equilibrate RLHF: Towards Balancing Helpfulness-Safety Trade-off in Large Language Models}. arXiv.org.

\bibitem{he2025tju}
Yufei He, Yuexin Li, Jiaying Wu, et al. (2025). \textit{Evaluating the Paperclip Maximizer: Are RL-Based Language Models More Likely to Pursue Instrumental Goals?}. arXiv.org.

\bibitem{xu2025iv8}
Wenyuan Xu, Xiaochen Zuo, Chao Xin, et al. (2025). \textit{A Unified Pairwise Framework for RLHF: Bridging Generative Reward Modeling and Policy Optimization}. arXiv.org.

\bibitem{chen2025fs9}
Jack Chen, Fazhong Liu, Naruto Liu, et al. (2025). \textit{Step-wise Adaptive Integration of Supervised Fine-tuning and Reinforcement Learning for Task-Specific LLMs}. arXiv.org.

\bibitem{ramos20236pc}
Miguel Moura Ramos, Patrick Fernandes, António Farinhas, et al. (2023). \textit{Aligning Neural Machine Translation Models: Human Feedback in Training and Inference}. European Association for Machine Translation Conferences/Workshops.

\bibitem{xia2024rab}
Han Xia, Songyang Gao, Qiming Ge, et al. (2024). \textit{Inverse-Q*: Token Level Reinforcement Learning for Aligning Large Language Models Without Preference Data}. Conference on Empirical Methods in Natural Language Processing.

\bibitem{hong2024mqe}
Ilgee Hong, Zichong Li, Alexander Bukharin, et al. (2024). \textit{Adaptive Preference Scaling for Reinforcement Learning with Human Feedback}. Neural Information Processing Systems.

\bibitem{sidahmed20244mc}
Hakim Sidahmed, Samrat Phatale, Alex Hutcheson, et al. (2024). \textit{PERL: Parameter Efficient Reinforcement Learning from Human Feedback}. arXiv.org.

\bibitem{laleh2024wmr}
Alireza Rashidi Laleh, and M. N. Ahmadabadi (2024). \textit{A Survey On Enhancing Reinforcement Learning in Complex Environments: Insights from Human and LLM Feedback}. arXiv.org.

\bibitem{hao2025lc8}
Qianyue Hao, Sibo Li, Jian Yuan, et al. (2025). \textit{RL of Thoughts: Navigating LLM Reasoning with Inference-time Reinforcement Learning}. arXiv.org.

\bibitem{miao2025ox0}
Yuchun Miao, Sen Zhang, Liang Ding, et al. (2025). \textit{The Energy Loss Phenomenon in RLHF: A New Perspective on Mitigating Reward Hacking}. arXiv.org.

\bibitem{liu2024avj}
Pangpang Liu, Chengchun Shi, and Will Wei Sun (2024). \textit{Dual Active Learning for Reinforcement Learning from Human Feedback}. arXiv.org.

\bibitem{yu2024p4z}
Huimu Yu, Xing Wu, Weidong Yin, et al. (2024). \textit{CodePMP: Scalable Preference Model Pretraining for Large Language Model Reasoning}. arXiv.org.

\bibitem{hazra2024wjp}
Rishi Hazra, Alkis Sygkounas, A. Persson, et al. (2024). \textit{REvolve: Reward Evolution with Large Language Models using Human Feedback}. International Conference on Learning Representations.

\bibitem{hao2024iyj}
Shugang Hao, and Lingjie Duan (2024). \textit{Online Learning from Strategic Human Feedback in LLM Fine-Tuning}. IEEE International Conference on Acoustics, Speech, and Signal Processing.

\bibitem{basit2024hl2}
Abdul Basit, Khizar Hussain, M. Hanif, et al. (2024). \textit{MedAide: Leveraging Large Language Models for On-Premise Medical Assistance on Edge Devices}. arXiv.org.

\bibitem{liu20251xv}
Xiaoyuan Liu, Tian Liang, Zhiwei He, et al. (2025). \textit{Trust, But Verify: A Self-Verification Approach to Reinforcement Learning with Verifiable Rewards}. arXiv.org.

\bibitem{pappu2024yoj}
Aneesh Pappu, Billy Porter, Ilia Shumailov, et al. (2024). \textit{Measuring memorization in RLHF for code completion}. International Conference on Learning Representations.

\bibitem{zhang2024w99}
Qining Zhang, and Lei Ying (2024). \textit{Zeroth-Order Policy Gradient for Reinforcement Learning from Human Feedback without Reward Inference}. International Conference on Learning Representations.

\bibitem{lpezcardona20242pt}
Ángela López-Cardona, Carlos Segura, Alexandros Karatzoglou, et al. (2024). \textit{Seeing Eye to AI: Human Alignment via Gaze-Based Response Rewards for Large Language Models}. International Conference on Learning Representations.

\bibitem{lindstrm20253o2}
Adam Dahlgren Lindström, Leila Methnani, Lea Krause, et al. (2025). \textit{Helpful, harmless, honest? Sociotechnical limits of AI alignment and safety through Reinforcement Learning from Human Feedback}. Ethics and Information Technology.

\bibitem{li202468g}
Haoran Li, Yulin Chen, Zihao Zheng, et al. (2024). \textit{Simulate and Eliminate: Revoke Backdoors for Generative Large Language Models}. AAAI Conference on Artificial Intelligence.

\bibitem{ji2024d5f}
Xiang Ji, Sanjeev Kulkarni, Mengdi Wang, et al. (2024). \textit{Self-Play with Adversarial Critic: Provable and Scalable Offline Alignment for Language Models}. arXiv.org.

\bibitem{sidahmed2024ikf}
Hakim Sidahmed, Samrat Phatale, Alex Hutcheson, et al. (2024). \textit{Parameter Efficient Reinforcement Learning from Human Feedback}. Unpublished manuscript.

\bibitem{wang2023v62}
Chenglong Wang, Hang Zhou, Yimin Hu, et al. (2023). \textit{ESRL: Efficient Sampling-based Reinforcement Learning for Sequence Generation}. AAAI Conference on Artificial Intelligence.

\bibitem{wang2024lje}
Jue Wang (2024). \textit{Hallucination Reduction and Optimization for Large Language Model-Based Autonomous Driving}. Symmetry.

\bibitem{chen2025vp2}
Yitian Chen, Jingfan Xia, Siyu Shao, et al. (2025). \textit{Solver-Informed RL: Grounding Large Language Models for Authentic Optimization Modeling}. arXiv.org.

\bibitem{dai2025ygq}
Juntao Dai, Taiye Chen, Yaodong Yang, et al. (2025). \textit{Mitigating Reward Over-Optimization in RLHF via Behavior-Supported Regularization}. International Conference on Learning Representations.

\bibitem{su2025mld}
Xuerui Su, Yue Wang, Jinhua Zhu, et al. (2025). \textit{Reveal the Mystery of DPO: The Connection between DPO and RL Algorithms}. arXiv.org.

\bibitem{wu2023pjz}
Fan Wu, Huseyin A. Inan, A. Backurs, et al. (2023). \textit{Privately Aligning Language Models with Reinforcement Learning}. International Conference on Learning Representations.

\bibitem{herreraberg202362r}
Eugenio Herrera-Berg, Tomás Vergara Browne, Pablo Le'on-Villagr'a, et al. (2023). \textit{Large Language Models are biased to overestimate profoundness}. Conference on Empirical Methods in Natural Language Processing.

\bibitem{xiong2023vbs}
Kai Xiong, Xiao Ding, Yixin Cao, et al. (2023). \textit{Diving into the Inter-Consistency of Large Language Models: An Insightful Analysis through Debate}. Unpublished manuscript.

\bibitem{lambert2023xiw}
Nathan Lambert, Thomas Krendl Gilbert, and Tom Zick (2023). \textit{Entangled Preferences: The History and Risks of Reinforcement Learning and Human Feedback}. arXiv.org.

\bibitem{duan2023nle}
Shukai Duan, Nikos Kanakaris, Xiongye Xiao, et al. (2023). \textit{PerfRL: A Small Language Model Framework for Efficient Code Optimization}. Unpublished manuscript.

\bibitem{wang20247pw}
Haoxiang Wang, Wei Xiong, Tengyang Xie, et al. (2024). \textit{Interpretable Preferences via Multi-Objective Reward Modeling and Mixture-of-Experts}. Conference on Empirical Methods in Natural Language Processing.

\bibitem{tang2024wt3}
Yunhao Tang, Daniel Guo, Zeyu Zheng, et al. (2024). \textit{Understanding the performance gap between online and offline alignment algorithms}. arXiv.org.

\bibitem{zhang2024q0e}
Di Zhang, Jianbo Wu, Jingdi Lei, et al. (2024). \textit{LLaMA-Berry: Pairwise Optimization for O1-like Olympiad-Level Mathematical Reasoning}. arXiv.org.

\bibitem{wang2024w7p}
Pengyu Wang, Dong Zhang, Linyang Li, et al. (2024). \textit{InferAligner: Inference-Time Alignment for Harmlessness through Cross-Model Guidance}. Conference on Empirical Methods in Natural Language Processing.

\bibitem{chen20244ev}
Guoxin Chen, Minpeng Liao, Chengxi Li, et al. (2024). \textit{Step-level Value Preference Optimization for Mathematical Reasoning}. Conference on Empirical Methods in Natural Language Processing.

\bibitem{liu2024w47}
Tianlin Liu, Shangmin Guo, Leonardo Bianco, et al. (2024). \textit{Decoding-time Realignment of Language Models}. International Conference on Machine Learning.

\bibitem{zhang2024lqf}
Shenao Zhang, Donghan Yu, Hiteshi Sharma, et al. (2024). \textit{Self-Exploring Language Models: Active Preference Elicitation for Online Alignment}. Trans. Mach. Learn. Res..

\bibitem{yuan2024jp7}
Huizhuo Yuan, Zixiang Chen, Kaixuan Ji, et al. (2024). \textit{Self-Play Fine-Tuning of Diffusion Models for Text-to-Image Generation}. Neural Information Processing Systems.

\bibitem{su20257nq}
Zhao-yu Su, Linjie Li, Mingyang Song, et al. (2025). \textit{OpenThinkIMG: Learning to Think with Images via Visual Tool Reinforcement Learning}. arXiv.org.

\bibitem{chen2024vkb}
Changyu Chen, Zi-Yan Liu, Chao Du, et al. (2024). \textit{Bootstrapping Language Models with DPO Implicit Rewards}. International Conference on Learning Representations.

\bibitem{bao2024wnc}
Guangsheng Bao, Hongbo Zhang, Linyi Yang, et al. (2024). \textit{How Likely Do LLMs with CoT Mimic Human Reasoning?}. International Conference on Computational Linguistics.

\bibitem{ning2024rhw}
Liang-bo Ning, Shijie Wang, Wenqi Fan, et al. (2024). \textit{CheatAgent: Attacking LLM-Empowered Recommender Systems via LLM Agent}. Knowledge Discovery and Data Mining.

\bibitem{lu202435m}
Keming Lu, Bowen Yu, Fei Huang, et al. (2024). \textit{Online Merging Optimizers for Boosting Rewards and Mitigating Tax in Alignment}. arXiv.org.

\bibitem{zhang2024b6u}
Wenxuan Zhang, Philip H. S. Torr, Mohamed Elhoseiny, et al. (2024). \textit{Bi-Factorial Preference Optimization: Balancing Safety-Helpfulness in Language Models}. International Conference on Learning Representations.

\bibitem{zhang2024esn}
Xiaoying Zhang, Jean-François Ton, Wei Shen, et al. (2024). \textit{Overcoming Reward Overoptimization via Adversarial Policy Optimization with Lightweight Uncertainty Estimation}. arXiv.org.

\bibitem{mohammadi20241pk}
Behnam Mohammadi (2024). \textit{Creativity Has Left the Chat: The Price of Debiasing Language Models}. arXiv.org.

\bibitem{dong2025io9}
Guanting Dong, Yifei Chen, Xiaoxi Li, et al. (2025). \textit{Tool-Star: Empowering LLM-Brained Multi-Tool Reasoner via Reinforcement Learning}. arXiv.org.

\bibitem{zhang2025d44}
Yanzhi Zhang, Zhaoxi Zhang, Haoxiang Guan, et al. (2025). \textit{No Free Lunch: Rethinking Internal Feedback for LLM Reasoning}. arXiv.org.

\bibitem{boldi2024d0s}
Ryan Boldi, Lijie Ding, Lee Spector, et al. (2024). \textit{Pareto-Optimal Learning from Preferences with Hidden Context}. arXiv.org.

\bibitem{du2025zfp}
Yuhao Du, Zhuo Li, Pengyu Cheng, et al. (2025). \textit{Simplify RLHF as Reward-Weighted SFT: A Variational Method}. arXiv.org.

\bibitem{lai2024ifx}
Yuhang Lai, Siyuan Wang, Shujun Liu, et al. (2024). \textit{ALaRM: Align Language Models via Hierarchical Rewards Modeling}. Annual Meeting of the Association for Computational Linguistics.

\bibitem{zhang2024yqu}
Han Zhang, Lin Gui, Yu Lei, et al. (2024). \textit{COPR: Continual Human Preference Learning via Optimal Policy Regularization}. arXiv.org.

\bibitem{singla2024dom}
Somanshu Singla, Zhen Wang, Tianyang Liu, et al. (2024). \textit{Dynamic Rewarding with Prompt Optimization Enables Tuning-free Self-Alignment of Language Models}. Conference on Empirical Methods in Natural Language Processing.

\bibitem{shao20257t5}
Ruichen Shao, Bei Li, Gangao Liu, et al. (2025). \textit{Earlier Tokens Contribute More: Learning Direct Preference Optimization From Temporal Decay Perspective}. International Conference on Learning Representations.

\bibitem{yang2024ppw}
Hongyu Yang, Liyang He, Min Hou, et al. (2024). \textit{Aligning LLMs through Multi-perspective User Preference Ranking-based Feedback for Programming Question Answering}. arXiv.org.

\bibitem{chai2024qal}
Yekun Chai, Haoran Sun, Huang Fang, et al. (2024). \textit{MA-RLHF: Reinforcement Learning from Human Feedback with Macro Actions}. International Conference on Learning Representations.

\bibitem{anand2024rnl}
Avinash Anand, Kritarth Prasad, Chhavi Kirtani, et al. (2024). \textit{Enhancing LLMs for Physics Problem-Solving using Reinforcement Learning with Human-AI Feedback}. arXiv.org.

\bibitem{sun20250ae}
Hao Sun, Yunyi Shen, Jean-Franccois Ton, et al. (2025). \textit{Reusing Embeddings: Reproducible Reward Model Research in Large Language Model Alignment without GPUs}. arXiv.org.

\bibitem{zhou2024gke}
Zhilun Zhou, Yuming Lin, and Yong Li (2024). \textit{Large language model empowered participatory urban planning}. arXiv.org.

\bibitem{chen2024m3b}
Kuang-Ming Chen, and Hung-yi Lee (2024). \textit{InstructionCP: A fast approach to transfer Large Language Models into target language}. arXiv.org.

\bibitem{chen20253a4}
Yang Chen, Yufan Shen, Wenxuan Huang, et al. (2025). \textit{Learning Only with Images: Visual Reinforcement Learning with Reasoning, Rendering, and Visual Feedback}. arXiv.org.

\bibitem{wang20252c7}
Ning Wang, Bingkun Yao, Jie Zhou, et al. (2025). \textit{Insights from Verification: Training a Verilog Generation LLM with Reinforcement Learning with Testbench Feedback}. arXiv.org.

\bibitem{tiapkin202538o}
D. Tiapkin, Daniele Calandriello, Johan Ferret, et al. (2025). \textit{On Teacher Hacking in Language Model Distillation}. arXiv.org.

\bibitem{gao20253ro}
Longxi Gao, Li Zhang, and Mengwei Xu (2025). \textit{UIShift: Enhancing VLM-based GUI Agents through Self-supervised Reinforcement Learning}. arXiv.org.

\bibitem{manduzio2024b9l}
Graziano A. Manduzio, F. Galatolo, M. G. Cimino, et al. (2024). \textit{Improving Small-Scale Large Language Models Function Calling for Reasoning Tasks}. arXiv.org.

\bibitem{pignatelli2024ffp}
Eduardo Pignatelli, Johan Ferret, Tim Rockaschel, et al. (2024). \textit{Assessing the Zero-Shot Capabilities of LLMs for Action Evaluation in RL}. arXiv.org.

\bibitem{zhang20248x9}
Qining Zhang, Honghao Wei, and Lei Ying (2024). \textit{Reinforcement Learning from Human Feedback without Reward Inference: Model-Free Algorithm and Instance-Dependent Analysis}. RLJ.

\bibitem{zhu2024zs2}
Yu Zhu, Chuxiong Sun, Wenfei Yang, et al. (2024). \textit{Proxy-RLHF: Decoupling Generation and Alignment in Large Language Model with Proxy}. arXiv.org.

\bibitem{srivastava2025gfw}
S. Srivastava, and Vaneet Aggarwal (2025). \textit{A Technical Survey of Reinforcement Learning Techniques for Large Language Models}. arXiv.org.

\bibitem{mandal2025qf5}
Debmalya Mandal, Paulius Sasnauskas, and Goran Radanovic (2025). \textit{Distributionally Robust Reinforcement Learning with Human Feedback}. arXiv.org.

\bibitem{chen20259cu}
Evan Chen, Run-Jun Zhan, Yan-Bai Lin, et al. (2025). \textit{More Women, Same Stereotypes: Unpacking the Gender Bias Paradox in Large Language Models}. Unpublished manuscript.

\bibitem{zhao202532z}
Heyang Zhao, Chen Ye, Wei Xiong, et al. (2025). \textit{Logarithmic Regret for Online KL-Regularized Reinforcement Learning}. arXiv.org.

\bibitem{daniel2024ajc}
Johan S Daniel, and Anand Pal (2024). \textit{Impact of Non-Standard Unicode Characters on Security and Comprehension in Large Language Models}. arXiv.org.

\bibitem{li2024h19}
Jiahui Li, Lin Li, Tai-wei Chang, et al. (2024). \textit{RED: Unleashing Token-Level Rewards from Holistic Feedback via Reward Redistribution}. Unpublished manuscript.

\end{thebibliography}

\end{document}