\subsection{Evaluation and Interpretability of Reward Models and Policies}

The rigorous evaluation of reward models (RMs) and the policies they shape is paramount for building trustworthy and reliable language AI systems. This necessitates moving beyond superficial performance metrics to a deeper understanding of how Reinforcement Learning from Human Feedback (RLHF) influences Large Language Model (LLM) behavior, including generalization, diversity, and internal reasoning fidelity. Transparency and interpretability are critical for diagnosing issues and ensuring alignment with true human intent.

A primary challenge lies in efficiently and reliably evaluating reward models themselves. Directly assessing an RM's quality by training a full LLM and then conducting human evaluations is prohibitively expensive \cite{frick20248mv}. To address this, \cite{frick20248mv} introduced Preference Proxy Evaluations (PPE), a benchmark designed to correlate RM performance on proxy tasks with actual downstream RLHF outcomes, providing a more cost-effective feedback loop. Complementing this, \cite{scheid20244oy} offers a theoretical framework for optimal design in reward modeling (ODPO), aiming to minimize human labeling costs by selecting the most informative preference pairs, thereby improving the efficiency of RM evaluation data collection. Beyond mere accuracy, the interpretability of RMs is crucial for diagnosing and understanding their behavior. \cite{yu20249l0} enhanced reward model quality and interpretability by leveraging self-generated critiques, demonstrating improved data efficiency and robustness in RM training, which indirectly aids in their evaluation. Similarly, \cite{wang20247pw} proposed an interpretable multi-objective reward modeling framework with a Mixture-of-Experts (MoE) scalarization, allowing for dynamic, context-conditioned weighting of objectives. While this framework also enables mitigation, its interpretability aspect is key for evaluating how different objectives are weighted and for diagnosing potential imbalances or biases in the RM's decision-making process.

Indeed, identifying and characterizing biases in reward models is a significant area of evaluation. \cite{saito2023zs7} empirically demonstrated a pervasive "verbosity bias" in LLM-as-judge preference labeling, showing that LLMs tend to prefer longer responses even more strongly than humans, leading to potential misalignment. This length correlation was further rigorously investigated by \cite{singhal2023egk}, who found that optimizing for output length alone could reproduce most of the perceived improvements from standard RLHF, highlighting the non-robustness and potential for spurious correlations within current reward models. \cite{rafailov2024ohd} extended this by characterizing reward over-optimization in Direct Alignment Algorithms (DAAs) like DPO, revealing that performance degradation can occur very early in training due to the exploitation of simple features such as response length. These findings underscore the critical need for evaluation methods that can detect such biases and over-optimization tendencies, rather than just reporting aggregate preference accuracy.

Beyond biases in RMs, understanding the downstream effects of RLHF on LLM policy behavior is crucial for comprehensive evaluation. A systematic analysis by \cite{kirk20230it} revealed an inherent trade-off: while RLHF significantly improves out-of-distribution generalization compared to supervised fine-tuning (SFT), it simultaneously and substantially reduces output diversity. This reduction in diversity was further quantified by \cite{mohammadi20241pk}, who showed that RLHF-aligned models tend to gravitate towards specific "attractor states" in their output space, akin to mode collapse, thereby diminishing creativity. \cite{xiao2024ro4} identified "preference collapse" as an algorithmic bias in standard KL-regularized RLHF, where minority preferences are effectively disregarded, leading to a homogenization of outputs. These studies highlight that evaluating RLHF-aligned models solely on average preference scores can obscure critical behavioral changes like reduced diversity or the suppression of nuanced preferences.

Perhaps the most profound insight into internal reasoning fidelity comes from \cite{bao2024wnc}, who applied causal analysis to LLM Chain-of-Thought (CoT) processes. Their surprising finding was that while in-context learning (ICL) strengthens ideal causal structures indicative of genuine reasoning, post-training methods like SFT and *RLHF can actually weaken them*, leading to models that "explain" rather than truly "reason." This challenges the assumption that alignment universally improves all aspects of model behavior and underscores the critical need for interpretability tools that can diagnose the true impact of alignment on internal cognitive processes, moving beyond superficial output analysis.

The need for robust evaluation methodologies extends to task-specific contexts, where traditional metrics like ROUGE and BLEU are common for summarization and machine translation, respectively. For instance, \cite{ramos20236pc} systematically explored integrating human feedback via quality metrics like COMET and COMET-QE across the Neural Machine Translation (NMT) pipeline, demonstrating synergistic benefits. However, while these metrics are useful for specific tasks, they often fall short in capturing the nuanced, human-centric objectives targeted by RLHF, such as helpfulness, harmlessness, or honesty. This leads to the broader conceptual critiques from \cite{lambert2023c8q}, who highlight an "alignment ceiling" due to objective mismatch, where internal RL metrics often decouple from true human values. \cite{lambert2023bty} provided a historical and risk analysis of RLHF, emphasizing the lack of transparency in reward models and their sociotechnical impacts, which complicates robust evaluation. Further, \cite{lindstrm20253o2} offered a sociotechnical critique of the "helpful, harmless, honest" (HHH) principles, arguing that the "curse of flexibility" in generalist LLMs limits the ability to ensure safety through purely technical alignment. These critiques underscore the fundamental limitations of current evaluation paradigms and the imperative for greater transparency and interpretability in both reward models and aligned policies to build trustworthy AI systems.

In conclusion, the evaluation of RLHF-aligned LLMs is a multifaceted challenge that demands a shift from superficial performance metrics to a deeper, more diagnostic understanding. This involves developing efficient methods for evaluating reward models, rigorously identifying and characterizing biases like verbosity and length correlations, and comprehensively assessing downstream policy behaviors such as generalization, diversity, and internal reasoning fidelity. The surprising finding that RLHF might weaken causal reasoning structures underscores the critical need for transparent, interpretable, and causally informed evaluation strategies to ensure the trustworthiness and reliability of future language AI systems.