\subsection{Multi-Objective Alignment and Diverse Preferences}
Aligning large language models (LLMs) with human values is inherently complex, often requiring the reconciliation of multiple, sometimes conflicting, objectives such as helpfulness, safety, honesty, and creativity, while also accommodating diverse user preferences. Traditional Reinforcement Learning from Human Feedback (RLHF) approaches, which typically rely on a single, monolithic reward model, often fall short in capturing this nuanced landscape. This limitation is starkly demonstrated by \cite{chakraborty20247ew}, who establish an "impossibility result" showing that single-reward RLHF cannot adequately align with diverse human preferences, particularly those of minority groups. Further, \cite{xiao2024ro4} reveals an inherent algorithmic bias in standard KL-regularized RLHF, leading to "preference collapse" where LLMs disproportionately favor dominant opinions, even with a perfectly accurate reward model, thus failing to represent population diversity.

To address these fundamental challenges, research has shifted towards multi-objective optimization and more sophisticated preference modeling. A critical trade-off is often observed between helpfulness and safety. \cite{tan2025lk0} tackles this with their Equilibrate RLHF framework, which employs a fine-grained data-centric approach to categorize safety data and an Adaptive Message-wise Alignment (AMA) strategy using gradient masking to balance these objectives, preventing models from becoming "over-safe" and unhelpful. Complementing this, \cite{zhang2024b6u} introduces Bi-Factorial Preference Optimization (BFPO), a supervised learning framework that efficiently re-parameterizes the joint helpfulness-safety RLHF objective into a single supervised loss, significantly reducing the computational and human annotation costs associated with multi-objective alignment. Beyond helpfulness and safety, \cite{moskovitz2023slz} confronts reward model overoptimization in composite reward settings by proposing constrained RLHF. Their method dynamically adapts the influence of individual reward components using "proxy points" and Lagrange multipliers, ensuring that no single objective is over-optimized at the expense of overall quality.

Moving beyond simple aggregation, methods are emerging to represent and learn from a spectrum of preferences. \cite{boldi2024d0s} introduces Pareto-Optimal Preference Learning (POPL), which leverages lexicase selection to learn a *set* of reward functions or policies. Each policy is optimal for a distinct "hidden context" group, allowing for pluralistic alignment without requiring explicit group labels, thereby ensuring fairness and representing diverse opinions. To enhance the interpretability and steerability of reward models, \cite{wang20247pw} proposes ArmoRM with a Mixture-of-Experts (MoE) scalarization. This approach dynamically weights multiple *absolute-rating* objectives (e.g., helpfulness, correctness, verbosity, safety) based on the input context, making the reward model transparent and enabling explicit mitigation of biases like verbosity, as previously diagnosed by \cite{singhal2023egk}. Similarly, \cite{yu20249l0} improves reward model quality and interpretability by incorporating self-generated critiques, providing explicit rationales beyond scalar scores, which is crucial for understanding nuanced preferences. Furthermore, \cite{lai2024ifx} introduces ALaRM, a hierarchical reward modeling framework that integrates holistic and proactively selected aspect-specific rewards to provide more precise and consistent guidance, addressing the challenges of inconsistent and sparse human feedback in complex generation tasks.

The deployment of LLMs in real-world applications necessitates robust strategies for handling diverse user needs and practical constraints. \cite{xu20242yo} addresses multi-task learning with conflicting objectives and reward hacking through Constrained Generative Policy Optimization (CGPO) with a "Mixture of Judges." This framework allows for task-specific optimization, employing customized reward models, judges, and hyperparameters to effectively manage complex multi-objective scenarios. In domain-specific contexts, \cite{yang2024ppw} tackles programming question answering, where diverse user preferences and up-to-date information are paramount. Their ALMupQA framework uses Multi-perspective Preference Ranking Alignment (MPRA) to synthesize preferences from questioners, community votes, and LLM content evaluations, combined with Retrieval-augmented In-context Learning (RIL) to ensure factual recency. For robust alignment against varying inputs, \cite{mandal2025qf5} proposes Distributionally Robust Reinforcement Learning with Human Feedback (DR-RLHF) to enhance robustness against out-of-distribution (OOD) prompt shifts, ensuring LLMs generalize reliably to diverse user needs. \cite{hong2024mqe} further refines preference learning by introducing Adaptive Preference Scaling, which uses instance-specific scaling parameters to capture varying strengths of human preferences, leading to more flexible and accurate reward models. Finally, practical challenges like catastrophic forgetting, or "alignment tax," are addressed by \cite{lu202435m} with Online Merging Optimizers, which balance reward maximization with the preservation of foundational capabilities. \cite{hou2024tvy} provides a comprehensive production-grade RLHF pipeline, offering practical solutions for bias mitigation (e.g., length bias) and catastrophic forgetting, essential for aligning large models for diverse real-world use cases.

In conclusion, the field of LLM alignment is rapidly evolving from monolithic reward functions to multi-objective, nuanced, and interpretable strategies. While significant progress has been made in balancing conflicting objectives, representing diverse preferences, and ensuring robustness in real-world scenarios, ongoing challenges remain in scaling these sophisticated methods, achieving universal fairness across all demographics, and developing truly adaptive systems that can learn and evolve with changing human values and societal norms. Future research will likely focus on more autonomous and dynamic mechanisms for preference elicitation and objective reconciliation, further integrating interpretability and robustness as first-class citizens in the alignment pipeline.