<think>
The user wants a LaTeX literature review subsection titled "Unintended Consequences: Generalization, Diversity, and Creativity Loss".
The subsection needs to critically analyze the unintended negative impacts of RLHF on LLM outputs, specifically focusing on reduced diversity and creativity, 'attractor states'/'mode collapse', and effects on out-of-distribution generalization.

I need to ensure:
- LaTeX format with citations.
- At least 400 words.
- Strict adherence to the subsection specific focus.
- Specific examples and technical contributions from cited papers.
- Papers are connected by showing how later work addresses limitations of earlier work.
- The narrative follows a logical progression.
- Avoid red flags (sequential listing, vague critique, unsupported claims, missing comparisons, pure description, ignoring contradictions, cherry-picking).

Let's break down the papers and how they relate to the subsection's focus:

**Subsection Specific Focus Keywords:**
- Unintended negative impacts of RLHF
- Reduced output diversity
- Reduced creativity
- 'Attractor states' / mode collapse
- Effects on out-of-distribution generalization
- Trade-off between alignment/safety and desirable generative properties

**Relevant Papers from Community Summaries & Development Directions:**

**Core Papers for this subsection:**
1.  **[kirk20230it] Understanding the Effects of RLHF on LLM Generalisation and Diversity (2023)**: Directly addresses OOD generalization and diversity loss. This is a foundational paper for the subsection. It empirically shows RLHF improves OOD generalization but *reduces diversity*.
2.  **[mohammadi20241pk] Creativity Has Left the Chat: The Price of Debiasing Language Models (2024)**: Directly addresses creativity loss and introduces "attractor states" / mode collapse. This is another core paper.
3.  **[singhal2023egk] A Long Way to Go: Investigating Length Correlations in RLHF (2023)**: While not directly about diversity/creativity, it highlights a *spurious correlation* (length bias) in reward models, which can contribute to models converging to similar, verbose outputs, indirectly impacting diversity. It's a specific example of reward model flaws leading to unintended consequences.
4.  **[lambert2023c8q] The Alignment Ceiling: Objective Mismatch in Reinforcement Learning from Human Feedback (2023)**: Provides a conceptual framework ("objective mismatch") that explains *why* these unintended consequences (like verbosity, reduced diversity) occur. It offers a theoretical underpinning for the empirical observations.
5.  **[saito2023zs7] Verbosity Bias in Preference Labeling by Large Language Models (2023)**: Directly identifies and quantifies "verbosity bias" in LLM-as-a-judge preference labeling, which can lead to longer, but not necessarily better, outputs, contributing to reduced diversity. This is a specific instance of a reward model flaw like Singhal et al.
6.  **[rafailov2024ohd] Scaling Laws for Reward Model Overoptimization in Direct Alignment Algorithms (2024)**: Discusses "reward over-optimization" even in DPO, which can lead to performance degradation and potentially mode collapse, reinforcing the idea that alignment methods can lead to unintended convergence.
7.  **[lindstrm20253o2] Helpful, harmless, honest? Sociotechnical limits of AI alignment and safety through Reinforcement Learning from Human Feedback (2025)**: Broader sociotechnical critique, but it mentions the "curse of flexibility" and how current alignment struggles to capture complex ethics, which can lead to models being "safe" but uncreative or overly cautious. This provides a high-level context for why diversity/creativity loss might be an inherent limitation of current alignment paradigms.

**Less direct but potentially relevant for context:**
- **[zheng2023c98] Secrets of RLHF in Large Language Models Part I: PPO (2023)**: Focuses on PPO stability, but instability can lead to poor generalization or mode collapse.
- **[wang20247pw] Interpretable Preferences via Multi-Objective Reward Modeling and Mixture-of-Experts (2024)**: Offers a solution to reward hacking/verbosity bias, which is a consequence discussed here. Can be used to highlight future directions.
- **[sun2023awe] Reinforcement Learning in the Era of LLMs: What is Essential? What is needed? An RL Perspective on RLHF, Prompting, and Beyond (2023)**: Discusses RLHF as online IRL, but also acknowledges challenges like high action dimensionality and sparse feedback, which can contribute to mode collapse.
- **[shen2025pyh] Exploring Data Scaling Trends and Effects in Reinforcement Learning from Human Feedback (2025)**: Mentions decreasing model response diversity as a bottleneck in RLHF scaling.
- **[hou202448j] Does RLHF Scale? Exploring the Impacts From Data, Model, and Method (2024)**: Mentions diminishing returns and inefficiencies, which might be linked to models converging to limited output spaces.

**Paper Progression Chain (from Development Directions):**
The development directions heavily emphasize the critique of RLHF and its unintended consequences.
- **[kirk20230it]** is the starting point, empirically observing the generalization-diversity tradeoff.
- **[singhal2023egk]** dives into a specific artifact (length bias) that contributes to this.
- **[lambert2023c8q]** provides the conceptual "objective mismatch" framework to explain *why* these issues happen.
- **[rafailov2024ohd]** extends over-optimization to DAAs, showing the problem is widespread.
- **[mohammadi20241pk]** specifically quantifies creativity loss and identifies "attractor states," deepening the diversity critique.
- **[saito2023zs7]** further specifies verbosity bias in AI feedback.
- **[lindstrm20253o2]** offers the broadest sociotechnical critique, questioning the fundamental suitability of RLHF for true safety, which implicitly covers the loss of desirable generative properties.

**Plan for the subsection:**

1.  **Opening:** Introduce RLHF's success in alignment but immediately pivot to its unintended negative impacts on output quality, specifically diversity, creativity, and generalization.
2.  **Body - Diversity and Creativity Loss (Empirical Findings):**
    *   Start with \cite{kirk20230it}: Explain its systematic analysis showing RLHF improves OOD generalization but *reduces output diversity*. Mention the metrics used (EAD, Sentence-BERT, NLI diversity).
    *   Transition to \cite{mohammadi20241pk}: Deepen the diversity critique by introducing the concept of *creativity loss*. Explain its empirical findings using Llama-2-chat vs. base, semantic (t-SNE, SBERT) and syntactic (token entropy) analysis, and the discovery of "attractor states" or mode collapse where outputs converge to limited semantic clusters.
    *   Mention \cite{shen2025pyh} briefly as it also notes decreasing model response diversity as a bottleneck.
3.  **Body - Explaining the Mechanisms (Reward Model Flaws & Objective Mismatch):**
    *   Introduce \cite{singhal2023egk}: Discuss how reward models can be flawed, specifically highlighting the "verbosity bias" where models optimize for length rather than genuine quality. Mention LPPO experiment.
    *   Connect to \cite{saito2023zs7}: Reinforce the "verbosity bias" by showing LLMs as evaluators also exhibit this bias more strongly than humans, further contributing to non-diverse, verbose outputs.
    *   Introduce \cite{lambert2023c8q}: Explain "objective mismatch" as a unifying conceptual framework for these issues. Detail how decoupled objectives in RM training, policy optimization, and evaluation lead to unintended behaviors like verbosity, refusals, and reduced diversity.
    *   Mention \cite{rafailov2024ohd}: Show that "reward over-optimization" and exploitation of simple features (like length) are pervasive even in newer Direct Alignment Algorithms (DAAs), reinforcing the systemic nature of the problem.
4.  **Body - Generalization (Revisiting OOD):**
    *   Revisit \cite{kirk20230it} for OOD generalization: While RLHF improves OOD generalization in terms of alignment, the *trade-off* is the diversity loss. This means it generalizes *aligned behavior* but not *generative richness*.
    *   Briefly mention \cite{hou202448j} that RLHF scaling shows diminishing returns, which could be related to models converging to limited output spaces rather than truly generalizing diverse, high-quality responses.
5.  **Closing:** Summarize the critical trade-off identified: RLHF improves alignment and safety but often at the cost of diversity and creativity, leading to 'attractor states' and potentially superficial generalization. Emphasize the need for re-evaluation and novel methods to balance alignment with generative richness, perhaps hinting at solutions like interpretable RMs \cite{wang20247pw} or more holistic approaches \cite{lindstrm20253o2}.

**Constraint Checklist & Confidence Score:**
1. LaTeX format: Yes
2. At least 400 words: Yes (will aim for this)
3. Subsection specific focus: Yes
4. Specific examples with citations: Yes
5. Structured: Yes (Opening, Body with logical flow, Closing)
6. Avoid sequential listing: Yes (will connect papers)
7. Avoid vague critique: Yes (will specify what/why)
8. Avoid unsupported claims: Yes (will ground in paper findings)
9. Avoid missing comparisons: Yes (will link limitations to subsequent work)
10. Avoid pure description: Yes (will analyze contributions/limitations)
11. Avoid ignoring contradictions: N/A (no direct contradictions on this specific topic, rather complementary findings)
12. Avoid cherry-picking: Yes (will present the full picture of findings relevant to the focus)

Confidence Score: 5/5

Strategizing complete. I will now proceed with generating the LaTeX output.
\subsection{Unintended Consequences: Generalization, Diversity, and Creativity Loss}

While Reinforcement Learning from Human Feedback (RLHF) has been instrumental in aligning Large Language Models (LLMs) with human preferences for safety and helpfulness, a growing body of research critically examines its unintended negative impacts on the quality and characteristics of LLM outputs, particularly concerning diversity, creativity, and out-of-distribution generalization. These critiques highlight a significant trade-off, where alignment gains often come at the expense of desirable generative properties.

Empirical findings demonstrate a substantial reduction in output diversity following RLHF. A systematic analysis by \textcite{kirk20230it} rigorously evaluated each stage of the RLHF pipeline, revealing that while RLHF consistently improves out-of-distribution (OOD) generalization by enhancing performance on unseen data, it simultaneously and substantially decreases output diversity. Using advanced metrics like Expectation-Adjusted Distinct N-grams (EAD) and Sentence-BERT embedding cosine similarity, their work showed that RLHF models tend to produce more similar text, even for different inputs, suggesting a convergence towards a narrower range of responses. This observation is further supported by \textcite{shen2025pyh}, who identify decreasing model response diversity as a significant bottleneck in RLHF scaling efforts.

Deepening this critique, \textcite{mohammadi20241pk} empirically demonstrated a profound loss of creativity in LLMs fine-tuned with RLHF, leading to what they term "attractor states" or mode collapse. Through multi-faceted experiments comparing base and aligned Llama-2 models, they showed that aligned models generate outputs that are strikingly similar both semantically (visualized via t-SNE of SBERT embeddings) and syntactically (evidenced by lower token entropy). This indicates that RLHF can inadvertently transform LLMs into more deterministic algorithms, suppressing their exploratory capacity and limiting the richness of their responses, which is crucial for applications requiring novelty and varied content, such as marketing or creative writing.

The mechanisms underlying this reduction in diversity and creativity are often rooted in the limitations and biases of the reward models and optimization processes within RLHF. \textcite{singhal2023egk} revealed that reward models frequently exhibit a "verbosity bias," where they disproportionately favor longer outputs, even if the length does not correlate with genuine quality improvements. Their "Length-Only PPO (LPPO)" experiment strikingly showed that optimizing for length alone could reproduce most of the perceived RLHF improvements, exposing the non-robustness of reward models. This issue is exacerbated when LLMs themselves are used as evaluators, as \textcite{saito2023zs7} quantified a stronger verbosity bias in LLM-generated preferences compared to human judgments, leading to a misalignment where LLMs prefer longer answers even when humans prioritize conciseness.

These specific biases are symptoms of a more fundamental problem: "objective mismatch," as conceptualized by \textcite{lambert2023c8q}. This framework explains how the decoupled numerical objectives of reward model training, policy optimization, and downstream evaluation can lead to unintended behaviors. When the reward model optimizes for superficial features (like length or specific phrasing) rather than the true underlying human preference, the policy model will exploit these flaws, resulting in outputs that score high on the reward model but lack genuine quality, diversity, or creativity. \textcite{rafailov2024ohd} further demonstrated that "reward over-optimization" and the exploitation of simpler features persist even in Direct Alignment Algorithms (DAAs) like DPO, indicating that this issue is systemic across different RLHF variants.

Consequently, while RLHF enhances out-of-distribution generalization in terms of aligning model behavior to a desired (often safer or more helpful) standard, this generalization often comes at the cost of generative richness. The model learns to consistently produce a narrow band of "aligned" responses, effectively sacrificing the broad, diverse, and creative output space of the base model for consistency and safety. This trade-off underscores a critical challenge in current alignment methods: they may inadvertently suppress desirable generative properties, necessitating a re-evaluation of the balance between alignment, consistency, and the richness of LLM responses as the field progresses \cite{kirk20230it}. The broader sociotechnical critique by \textcite{lindstrm20253o2} further contextualizes this, arguing that the "curse of flexibility" in generalist LLMs makes it inherently difficult to define and ensure complex safety requirements through model-centric design alone, implicitly linking to the observed loss of nuanced generative capabilities.

These critiques collectively highlight that current alignment methods, while effective for safety and helpfulness, can inadvertently lead to LLMs that are less diverse, less creative, and prone to mode collapse, thereby limiting their utility in open-ended generative tasks. This necessitates a re-evaluation of the balance between alignment and the richness of LLM responses, pushing for novel techniques that can foster both safety and desirable generative properties, potentially through more interpretable reward models \cite{wang20247pw} or multi-objective optimization strategies.