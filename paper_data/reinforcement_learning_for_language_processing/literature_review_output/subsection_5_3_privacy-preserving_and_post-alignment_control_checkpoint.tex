\subsection*{Privacy-Preserving and Post-Alignment Control}
The responsible deployment of Reinforcement Learning (RL)-enhanced language models necessitates robust solutions for safeguarding sensitive user data during training and enabling dynamic control over model behavior post-alignment. This subsection explores advanced methodologies that address these critical considerations, focusing on privacy-preserving techniques and innovative approaches for inference-time behavioral adjustment.

Ensuring privacy throughout the multi-stage RL from Human Feedback (RLHF) process is paramount, as sensitive human feedback data can lead to privacy breaches. \cite{wu2023pjz} introduces a pioneering, end-to-end Differentially Private (DP) framework designed to align Large Language Models (LLMs) with mathematical privacy guarantees. Their core methodology integrates DP into three critical stages: DP Supervised Fine-Tuning (SFT) of the base LLM, DP training of the reward model using human preference data, and DP policy optimization via an adapted Proximal Policy Optimization (DPPPO). A key technical contribution is the identification and resolution of the necessity to train the reward model itself with DP, proving that a non-private reward model would invalidate overall privacy guarantees. The framework leverages DPSGD and modifies PPO by setting $T_{PPO} = 1$ for simplified privacy accounting, alongside integrating LoRA for enhanced stability and efficiency in DP training, establishing a foundational approach for building privacy-preserving instruction-following LLMs. While offering strong theoretical guarantees, the framework makes certain assumptions for privacy accounting, such as distinct user contributions across datasets, and the $T_{PPO} = 1$ choice, which simplifies analysis but might impact policy exploration.

Complementing privacy guarantees, the ability to dynamically adjust an LLM's alignment strength or other behavioral parameters at inference time, without requiring costly retraining, significantly enhances their adaptability and responsible deployment. Addressing the computational inflexibility of traditional alignment, \cite{liu2024w47} proposes Decoding-time Realignment (DeRa). This innovative technique is grounded in a theoretical proof demonstrating that aligned models with varying KL regularization strengths are geometric mixtures of a reference model and a single aligned model. DeRa leverages this insight to allow dynamic adjustment of alignment strength at inference time by linearly combining the logits of the reference and aligned models. This approach significantly reduces the computational burden of hyperparameter sweeps common in RLHF, DPO, and IPO, providing real-time control over the reward-regularization trade-off. While DeRa offers unprecedented flexibility and efficiency, its implementation incurs increased decoding time and memory usage due to requiring logits from two models, and it operates as an approximation rather than a perfect replication of fully retrained models.

Together, these advancements reflect a growing maturity in the field, aiming to make aligned LLMs more secure, adaptable, and compliant with privacy regulations. While \cite{wu2023pjz} focuses on securing the training data pipeline, \cite{liu2024w47} addresses the efficiency and flexibility of the deployed model's behavior. Future research will likely explore the synergistic integration of these two critical areas, investigating how dynamically controllable models can be trained with robust privacy guarantees, optimizing the complex trade-offs between privacy, utility, and inference-time efficiency for broader and more dynamic language processing applications.