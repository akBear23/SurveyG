\subsection{Unresolved Tensions and Theoretical Gaps}

The remarkable empirical successes of reinforcement learning (RL) in aligning large language models (LLMs) with human preferences have, paradoxically, outpaced a robust theoretical understanding of the underlying mechanisms and their limitations. This intellectual imbalance leaves the field reliant on a collection of ad-hoc fixes for fundamental problems it cannot yet formally define or guarantee against. Persistent challenges in reward design, the exploration-exploitation dilemma in vast language spaces, the generalization capabilities of aligned models, and a profound theoretical understanding of emergent behaviors constitute the core intellectual crisis facing RL for language processing \cite{wang2024a3a, srivastava2025gfw}. Moving beyond the current paradigm necessitates a deeper, more principled approach to these unresolved tensions.

One of the most critical and persistent challenges lies in the **complexities of reward design and the theoretical limits of preference learning**. As discussed in Section 5.2, reward models (RMs) are inherently imperfect proxies for true human intent, leading to phenomena like "reward hacking" and "overoptimization" \cite{singhal2023egk, rafailov2024ohd}. This is not merely a practical issue of data quality, but a fundamental theoretical problem of "objective mismatch" \cite{lambert2023bty}, where the optimization objective (maximizing a proxy reward) diverges from the true, often unquantifiable, human value. \cite{lambert2023bty} critically argues that applying optimization stacks designed for clear control problems to the ambiguous domain of language introduces "blind spots" and ill-posed assumptions about the quantifiability of human preferences. Further, \cite{xiao2024ro4} reveals an inherent algorithmic bias in standard KL-regularized RLHF, demonstrating "preference collapse" where minority preferences are disregarded even with an oracle reward model. This suggests that the very regularization mechanism, intended for stability, can fundamentally distort the preference landscape. While various mitigation strategies have been proposed (e.g., uncertainty quantification, mechanistic analysis, as seen in Section 5.2), these often serve as patches rather than addressing the root theoretical fragility. More principled approaches are emerging, such as formalizing RLHF as a reverse-KL regularized contextual bandit problem with finite-sample guarantees \cite{xiong2023klt}, or developing provable and scalable offline alignment methods like Self-Play with Adversarial Critic (SPAC) that converge to near-optimal policies under weak data coverage \cite{ji2024d5f}. The exploration of direct policy optimization from ranking oracles \cite{tang2023lop} also represents a theoretical shift towards bypassing the fallibility of explicit reward models altogether, offering a path to more robust alignment.

The **generalization capabilities of aligned models and their robustness to diverse inputs and objectives** also present profound theoretical and practical challenges. As highlighted in Sections 7.3 and 7.4, RLHF often improves out-of-distribution (OOD) generalization but frequently at the cost of reduced output diversity and creativity, leading to "attractor states" \cite{kirk20230it, mohammadi20241pk}. This "alignment tax" \cite{lu202435m} signifies a fundamental trade-off that current methods struggle to resolve without sacrificing core capabilities. Beyond diversity, aligned models exhibit vulnerabilities to adversarial attacks \cite{baumgrtner2024gu4, daniel2024ajc}, raising questions about the theoretical guarantees of safety and robustness. The "curse of flexibility" \cite{lambert2023bty} further suggests that truly aligning generalist LLMs with complex human values across all contexts might be fundamentally limited by the nature of the alignment process itself. Efforts like distributionally robust RLHF (DRO) \cite{mandal2025qf5} and inference-time alignment for harmlessness \cite{wang2024w7p} attempt to bolster robustness, while principle-driven self-alignment \cite{sun20238m7} explores alternative, less human-feedback-intensive paradigms. However, a comprehensive theoretical framework that reconciles helpfulness, harmlessness, and diversity, and guarantees robustness against a wide spectrum of adversarial and novel inputs, remains elusive.

Furthermore, the field grapples with the need for **more principled methods for multi-objective optimization**. LLMs are expected to balance multiple, often conflicting, objectives (e.g., helpfulness, safety, conciseness), and cater to diverse user preferences. Traditional single-scalar reward models, as discussed in Section 5.3, are inherently ill-equipped to capture this complexity, potentially marginalizing minority opinions \cite{chakraborty20247ew}. The "preference collapse" identified by \cite{xiao2024ro4} is a stark manifestation of this failure. To address this, approaches like MaxMin-RLHF \cite{chakraborty20247ew} and Pareto-Optimal Learning from Preferences with Hidden Context (POPL) \cite{boldi2024d0s} offer theoretical frameworks for learning policies that are optimal across diverse, unobserved preference groups. POPL, in particular, reframes the problem as multi-objective optimization, leveraging lexicase selection to generate a set of Pareto-optimal policies without requiring explicit group labels, thereby providing a more nuanced approach to pluralistic alignment. Despite these advancements, a unified, scalable, and theoretically robust approach for handling arbitrary, potentially conflicting objectives with strong guarantees remains an active and critical area of research.

A deeper **theoretical understanding of emergent behaviors** and the **formalization of RL's interplay with the internal reasoning processes of LLMs** is critically needed. While RLHF empirically improves LLM performance, its causal impact on internal reasoning remains largely opaque. Intriguingly, \cite{bao2024wnc} used a causal analysis framework to demonstrate that post-training methods like supervised fine-tuning (SFT) and RLHF can *weaken* the ideal causal structures indicative of genuine reasoning, suggesting a potential trade-off between optimizing for external metrics and fostering faithful internal reasoning. This challenges the assumption that alignment necessarily leads to more robust underlying cognitive processes. Similarly, \cite{zhang2025d44} critically analyzed Reinforcement Learning from Internal Feedback (RLIF), showing that common internal signals primarily minimize policy entropy, which can degrade performance. Conversely, other works leverage RL to foster desirable emergent behaviors, such as self-verification capabilities \cite{liu20251xv}, solver-informed RL for authentic optimization modeling \cite{chen2025vp2}, and RL of Thoughts (RLoT) for dynamically constructing reasoning paths at inference time \cite{hao2025lc8}. These efforts highlight the potential of RL to shape internal reasoning, but the fundamental mechanisms and their theoretical implications are still poorly understood.

Finally, the **exploration-exploitation dilemma in vast language spaces** continues to pose a significant theoretical and practical hurdle. The sheer size and discrete nature of the token space make efficient exploration difficult, especially with sparse or delayed rewards. This challenge is exacerbated by the tendency of passive exploration in online alignment methods to cluster responses around local optima, leaving vast, potentially high-reward regions unexplored \cite{zhang2024lqf}. Recent theoretical advancements, such as achieving the first logarithmic regret bounds for online KL-regularized RL \cite{zhao202532z}, provide a strong foundation for efficient exploration-exploitation. Practical solutions like Self-Exploring Language Models (SELM) \cite{zhang2024lqf} actively bias the model towards potentially high-reward regions, and the use of "macro actions" \cite{chai2024qal} aims to make exploration more tractable by reducing the temporal decision points. However, translating these theoretical insights and nascent practical strategies into universally robust and scalable exploration mechanisms for the open-ended, high-dimensional environments of LLMs remains an ongoing area of development.

In conclusion, the current landscape of RL for language processing is characterized by a critical need for a more robust theoretical foundation. Addressing the inherent limitations of preference learning, ensuring holistic generalization and robustness, developing principled methods for multi-objective and pluralistic alignment, and deeply understanding the causal impact of RL on LLM internal reasoning are paramount. Future research must move beyond purely empirical gains towards developing rigorous theoretical guarantees and deeper insights into the complex interplay between RL and the emergent capabilities of LLMs, paving the way for truly reliable, safe, and intelligent language AI.