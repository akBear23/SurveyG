\subsection{Verifiable and Structured Reward-Guided Reasoning}

Enhancing Large Language Model (LLM) reasoning and generation capabilities often necessitates objective, high-fidelity feedback that transcends the subjectivity and cost associated with human preferences. This subsection explores the emerging paradigm of Reinforcement Learning with Verifiable Rewards (RLVR), where objective, often programmatic, verifiers provide precise reward signals to drive specific, measurable improvements in tasks demanding formal correctness or robust tool integration.

A key challenge in multi-turn tool-integrated reasoning (TIR) for LLMs is the training instability and performance degradation caused by distributional drift from external tool feedback \cite{xue2025fl1}. To address this, \cite{xue2025fl1} introduces \texttt{SimpleTIR}, an end-to-end RL framework that stabilizes multi-turn TIR by employing a plug-and-play trajectory filtering algorithm. \texttt{SimpleTIR} identifies and discards entire trajectories containing "void turns"—LLM responses that fail to yield a complete code block or a final answer—effectively blocking harmful, high-magnitude gradients and enabling robust Zero RL training for emergent reasoning patterns. This approach demonstrates how external tool feedback can serve as a negative verifiable signal to prune problematic learning trajectories, leading to state-of-the-art performance in mathematical reasoning benchmarks.

Building upon the utility of verifiable rewards, \cite{liu20251xv} tackles the issue of "superficial self-reflection" in LLMs, where models struggle to robustly verify their own outputs despite being trained with outcome verifiers. Their proposed framework, RISE (Reinforcing Reasoning with Self-Verification), is an online RL approach that simultaneously trains LLMs in both problem-solving and self-verification. RISE leverages rule-based outcome verifiers not only to guide the generation of correct solutions but also to align the model's self-verification ability on-the-fly by constructing verification tasks from the model's own on-policy generated solutions, with the original solution reward serving as ground truth. This integrated approach significantly enhances both reasoning accuracy and the robustness of self-verification, demonstrating a more direct and comprehensive use of verifiable rewards to cultivate an explicit self-assessment capability.

Extending the application of RLVR to highly structured domains, \cite{chen2025vp2} introduces Solver-Informed RL (SIRL) to ground LLMs for authentic optimization modeling. LLMs traditionally struggle to generate formally correct, usable, and solver-compatible optimization models from natural language descriptions, often succumbing to hallucinations. SIRL addresses this by integrating external optimization solvers (e.g., Gurobi, CPLEX) as objective verifiers, which provide rich, precise feedback signals—including syntax correctness, feasibility status, and solution quality—as direct rewards for the RL process. This pioneering work establishes classical optimization solvers as powerful, domain-specific oracles for both high-quality data synthesis (via Instance-Enhanced Self-Consistency) and robust reward signal generation, achieving state-of-the-art performance in generating correct and reliable optimization models.

Collectively, these works underscore the transformative potential of verifiable and structured reward-guided reasoning. By leveraging objective, programmatic verifiers, these methods circumvent the inherent subjectivity and high cost of human feedback, enabling LLMs to achieve unprecedented levels of formal correctness and reliability in tasks ranging from multi-turn tool use to self-verification and complex optimization modeling. A common limitation, however, is the inherent reliance on the availability and reliability of such external verifiers, which may not exist for all tasks, particularly those requiring nuanced or subjective judgment. Future research could explore hybrid approaches that combine verifiable rewards with more generalized feedback mechanisms or investigate methods for generating robust verifiers for less formal domains.