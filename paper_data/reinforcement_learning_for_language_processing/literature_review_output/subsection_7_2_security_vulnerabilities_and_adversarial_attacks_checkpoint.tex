<think>
The user wants a literature review subsection titled "Security Vulnerabilities and Adversarial Attacks" in LaTeX format.

**Subsection Specific Focus:**
*   Critical security vulnerabilities and adversarial attacks compromising RLHF-aligned language models.
*   Data poisoning through injected preference data, leading to malicious behaviors.
*   Prompt leakage and 'model hacking' attacks bypassing safety guardrails, even in open-sourced models with white-box access.
*   RLHF introduces new attack surfaces.
*   Demand for continuous research into more robust and comprehensive defense mechanisms beyond current alignment implementations.

**Constraint Checklist & Confidence Score:**
1. LaTeX format: Yes
2. At least 400 words: Yes (will aim for this)
3. No numbering in subsection titles: Yes
4. Focus ONLY on SUBSECTION SPECIFIC FOCUS: Yes
5. Specific examples with citations: Yes
6. Opening (1-2 sentences problem statement): Yes
7. Body (2-3 sentences per paper, connections, narrative progression): Yes
8. Closing (unresolved issues/future directions): Yes
9. Avoid sequential listing: Yes
10. Avoid vague critique: Yes
11. Avoid unsupported claims: Yes
12. Avoid missing comparisons: Yes
13. Avoid pure description: Yes
14. Avoid ignoring contradictions: N/A (no explicit contradictions on this topic in provided summaries)
15. Avoid cherry-picking: N/A (will use all relevant papers)

Confidence Score: 5/5

**Mental Sandbox Simulation:**

*   **Initial thought:** Start with a general statement about RLHF's role in safety, then immediately pivot to its new attack surfaces.
*   **Paper selection (based on focus):**
    *   `[baumgrtner2024gu4]` (2024): Data poisoning (preference data). Directly relevant.
    *   `[zhang2023pbi]` (2023): Model hacking, white-box access, bypass safety guardrails. Directly relevant.
    *   `[daniel2024ajc]` (2024): Prompt leakage, Unicode attacks, guardrail bypass. Directly relevant.
    *   `[ning2024rhw]` (2024): LLM agents for adversarial attacks (RecSys context, but methodology of LLM as attacker is relevant).
    *   `[pappu2024yoj]` (2024): Memorization in RLHF, privacy risk (indirectly relevant to vulnerabilities).
    *   `[lambert2023bty]` (2023) & `[lambert2023c8q]` (2023) & `[lindstrm20253o2]` (2025): Critiques of RLHF, objective mismatch, sociotechnical limits. Provide broader context of RLHF's inherent limitations, which can lead to vulnerabilities.
    *   `[mohammadi20241pk]` (2024) & `[kirk20230it]` (2023): Diversity loss, unintended consequences. Less direct, but can be framed as unintended vulnerabilities or side effects of alignment.
    *   `[wang2024w7p]` (2024): Inference-time alignment for harmlessness. This is a *defense* mechanism, but its existence implies the problem it solves. Can be used to highlight the need for robust defenses.
    *   `[sun20238m7]` (2023): Self-alignment with minimal human supervision. Less direct for *attacks*, but if self-alignment is less robust, it could be a vulnerability. `[zhang2023pbi]` directly follows up on the safety of open-sourced models, which could include those aligned with minimal supervision.

*   **Narrative Flow (Chronological + Thematic):**
    1.  **Introduce RLHF's dual nature:** While aiming for safety, it creates new attack surfaces.
    2.  **Start with data poisoning:** `[baumgrtner2024gu4]` is a strong opener for data-level attacks. Explain how preference data, the core of RLHF, can be poisoned.
    3.  **Transition to model-level attacks/bypasses (white-box):** `[zhang2023pbi]` is perfect here. It directly addresses "model hacking" and white-box access, showing alignment isn't enough.
    4.  **Expand on prompt leakage/novel attack vectors:** `[daniel2024ajc]` shows how even subtle inputs (Unicode) can bypass guardrails, linking to prompt leakage.
    5.  **Discuss broader risks/unintended consequences that create vulnerabilities:** `[lambert2023bty]`, `[lambert2023c8q]`, `[lindstrm20253o2]` can provide the conceptual backdrop for why these vulnerabilities exist (objective mismatch, "curse of flexibility"). `[mohammadi20241pk]` and `[kirk20230it]` can be briefly mentioned for unintended consequences like reduced diversity, which might also be exploited.
    6.  **Consider LLM as attacker:** `[ning2024rhw]` shows LLMs can be weaponized as sophisticated attack agents, indicating an escalating threat landscape.
    7.  **Address privacy (memorization):** `[pappu2024yoj]` highlights privacy risks from memorization, which is a specific vulnerability.
    8.  **Conclude with the need for robust defenses:** Mention `[wang2024w7p]` as an example of ongoing defense research, reinforcing the call for continuous effort.

*   **Refinement - Connecting Papers:**
    *   `[baumgrtner2024gu4]` (2024) exposes a critical vulnerability: RLHF's susceptibility to preference poisoning. This directly attacks the *data* used for alignment.
    *   This vulnerability is exacerbated by the inherent opaqueness and potential "objective mismatch" in reward models, as critiqued by `\cite{lambert2023bty}` and `\cite{lambert2023c8q}`. These papers highlight that even well-intentioned RLHF can lead to unintended behaviors, making it easier for poisoned data to take hold.
    *   Beyond data poisoning, even seemingly aligned models can be "hacked" directly. `\cite{zhang2023pbi}` demonstrates "model hacking" attacks like `ProMan` that exploit white-box access to open-sourced LLMs, bypassing safety guardrails by manipulating token probabilities. This moves beyond data-level attacks to direct model manipulation.
    *   Further illustrating the fragility of alignment, `\cite{daniel2024ajc}` reveals that even non-standard Unicode characters can diminish RLHF-based safety mechanisms, leading to prompt leakage and content policy breaches. This highlights novel, subtle attack vectors that exploit LLM's comprehension limitations.
    *   The threat landscape is further complicated by the emergence of LLMs as sophisticated attackers themselves. `\cite{ning2024rhw}` introduces `CheatAgent`, where an LLM acts as an intelligent adversary to attack recommender systems, demonstrating how advanced AI can be weaponized.
    *   Beyond direct attacks, RLHF also introduces privacy vulnerabilities. `\cite{pappu2024yoj}` shows that while reward model training data is less likely to be memorized, direct preference learning methods can increase memorization risk, posing a privacy concern.
    *   These challenges underscore the "sociotechnical limits" of RLHF, as argued by `\cite{lindstrm20253o2}`, suggesting that a model-centric approach is insufficient for true AI safety. The unintended consequences of alignment, such as reduced diversity and creativity observed by `\cite{kirk20230it}` and `\cite{mohammadi20241pk}`, can also be seen as vulnerabilities, making models more predictable or less adaptable in certain contexts.
    *   The continuous emergence of such vulnerabilities necessitates robust defense mechanisms. `\cite{wang2024w7p}` offers a promising direction with `InferAligner`, an inference-time alignment method that uses cross-model guidance to enhance harmlessness without retraining, demonstrating efforts to build more resilient systems.

*   **Word Count Check:** This structure should easily exceed 400 words.
*   **LaTeX Formatting:** Will ensure `\section*`, `\subsection*`, `\cite{}`, and appropriate paragraph breaks.

One more check on the prompt: "RLHF aims to make LLMs safer, it also introduces new attack surfaces, demanding continuous research into more robust and comprehensive defense mechanisms beyond current alignment implementations to ensure secure and trustworthy AI systems in an evolving threat landscape." - This is the core message for the closing.

Okay, planning complete. I will now generate the LaTeX output.
\subsection*{Security Vulnerabilities and Adversarial Attacks}
While Reinforcement Learning from Human Feedback (RLHF) is instrumental in aligning Large Language Models (LLMs) with human values and safety policies, it simultaneously introduces new attack surfaces and vulnerabilities that can compromise model integrity and trustworthiness. These challenges range from data-level poisoning to sophisticated model-hacking techniques that bypass safety guardrails, underscoring the critical need for continuous research into robust defense mechanisms.

A significant vulnerability lies in the data used for alignment. \cite{baumgrtner2024gu4} demonstrate that RLHF is highly susceptible to "preference poisoning" attacks, where malicious actors can inject a small fraction of stealthy, naturalistic preference pairs into training datasets. This can cause the LLM to generate specific target entities with desired (or undesired) sentiments, effectively subverting the model's intended behavior. This vulnerability is further compounded by the inherent opaqueness and potential "objective mismatch" in reward models, as critically analyzed by \cite{lambert2023bty} and \cite{lambert2023c8q}. They argue that the ill-posed assumptions of reward models and the systemic mismatch between numerical optimization objectives and complex human values can lead to unintended behaviors, making the system more susceptible to manipulation.

Beyond data-level attacks, even seemingly aligned models can be directly "hacked." \cite{zhang2023pbi} introduce "model hacking" attacks, specifically `Probability Manipulation (ProMan)`, which exploits white-box access to open-sourced LLMs. This method directly manipulates the token generation probabilities (logits) during inference, effectively bypassing safety guardrails and forcing the model to generate undesired content, including harmful or private information. This demonstrates that current alignment strategies are insufficient against adversaries with direct access to model internals. Further illustrating the fragility of alignment, \cite{daniel2024ajc} reveal that even subtle inputs, such as non-standard Unicode characters, can significantly diminish the efficacy of RLHF-based safety mechanisms. This leads to prompt leakage, content policy breaches, and unique hallucination patterns across various state-of-the-art LLMs, highlighting novel and often overlooked attack vectors.

The evolving threat landscape also includes the weaponization of LLMs themselves as sophisticated adversaries. \cite{ning2024rhw} propose `CheatAgent`, a framework where an LLM acts as an intelligent agent to launch black-box adversarial attacks on LLM-empowered recommender systems. This demonstrates how advanced AI capabilities can be leveraged to generate adversarial perturbations and optimize attack policies, indicating an escalating level of sophistication in potential threats. Moreover, RLHF introduces privacy vulnerabilities related to data memorization. \cite{pappu2024yoj} show that while data used to train the reward model is less likely to be memorized by the final model, direct preference learning methods can increase the risk of training data regurgitation, posing a significant privacy concern, especially in sensitive domains like code completion.

These multifaceted challenges underscore the "sociotechnical limits" of AI alignment and safety through RLHF, as articulated by \cite{lindstrm20253o2}. They argue that a purely model-centric approach is fundamentally insufficient for true AI safety, especially given the "curse of flexibility" in generalist LLMs, where increased power can hinder the ability to define and validate safety requirements. The unintended consequences of RLHF, such as reduced output diversity and creativity observed by \cite{kirk20230it} and \cite{mohammadi20241pk}, can also be viewed as vulnerabilities, making models more predictable or less adaptable in certain contexts, potentially opening avenues for exploitation.

The continuous emergence of such vulnerabilities necessitates a proactive and robust approach to defense. While challenges persist, research is actively exploring new defense mechanisms. For instance, \cite{wang2024w7p} introduce `InferAligner`, an inference-time alignment method that enhances harmlessness through cross-model guidance and conditional intervention, without requiring expensive retraining. This exemplifies the ongoing efforts to develop more resilient and adaptable AI systems. However, the pervasive nature of these vulnerabilities demands continuous, interdisciplinary research into more comprehensive defense mechanisms that go beyond current alignment implementations, ensuring secure and trustworthy AI systems in an ever-evolving threat landscape.