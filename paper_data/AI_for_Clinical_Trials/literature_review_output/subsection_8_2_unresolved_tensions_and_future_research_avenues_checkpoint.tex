\subsection*{Unresolved Tensions and Future Research Avenues}

The integration of artificial intelligence (AI) into clinical trials, while promising transformative advancements, is fraught with persistent challenges and inherent tensions that impede its widespread and equitable adoption. Foremost among these is the fundamental disconnect between the rapid pace of AI innovation and the slow, deliberate processes required for rigorous clinical validation and regulatory adaptation. This section delves into these unresolved tensions, bridging the gap between theoretical potential and real-world implementation, and navigating critical issues such as data privacy versus data utility, to outline crucial future research directions.

A foundational tension lies in ensuring the intrinsic integrity and fairness of AI models, particularly as they are deployed in high-stakes clinical applications. \cite{kelly2019gw7} addresses this by proposing a novel decomposition of discrimination metrics into bias, variance, and noise components, offering a diagnostic framework to understand the root causes of unfairness. This work innovatively shifts the paradigm from merely mitigating discrimination to diagnosing its sources, often pointing to inadequate data collection as the culprit and advocating for data-centric interventions. However, a limitation noted by \cite{kelly2019gw7} is its assumption that observed performance differences are inherently discriminatory, without delving into causal inference or historical biases embedded in labels, thus highlighting a critical future research avenue for developing more robust and causally-aware AI models that ensure equitable outcomes.

Even with intrinsically fair and accurate models, the journey from theoretical potential to real-world implementation introduces significant complexities. The effective integration of AI into clinical workflows, particularly concerning human-AI collaboration, remains a major hurdle. \cite{rosenthal2025j23} empirically quantifies cognitive biases in human-AI interaction through a rigorous randomized controlled experiment with human experts. Their findings reveal significant biases like automation neglect and correlation neglect, demonstrating that AI assistance does not always improve human diagnostic quality and that optimal collaboration often involves delegating cases entirely to either humans or AI, but rarely to AI-assisted humans. This underscores the need for AI systems designed to be "bias-aware" in their interaction and for targeted training protocols for clinicians, ensuring that the human element does not inadvertently undermine AI's benefits. While AI-assisted analysis can reduce variability and improve prognostic value, as shown by \cite{gieraerts2020j5j} in COVID-19 lung involvement, the insights from \cite{rosenthal2025j23} suggest that the *manner* of deployment is paramount.

The rapid innovation in AI, exemplified by its potential in drug discovery and trial optimization \cite{blasiak2020fkz, chorev20230xi, ho2020xwh, patel2024jpj}, consistently outpaces the mechanisms for clinical validation and regulatory acceptance. \cite{macheka2024o73}'s systematic review of AI applications in cancer pathways reveals a critical gap: the majority of AI oncological research remains experimental, lacking prospective clinical validation and failing to translate measured AI efficacy into beneficial clinical outcomes. This review points to a lack of research standardization and health system interoperability as key barriers, directly addressing the tension between innovation and validation. Further, \cite{chen2024zvv} identifies specific trial design factors associated with the completion of AI clinical trials, highlighting that trials conducted in Europe and those with larger sample sizes are more likely to succeed. This emphasizes the practical challenges in designing and executing AI trials that meet regulatory and scientific rigor, necessitating a focus on addressing common reasons for study failure.

Another pervasive tension involves balancing data privacy with the imperative for data utility. While not explicitly detailed by a paper in this specific set, the broader field of AI for clinical trials consistently grapples with the need for large, diverse datasets to train robust models, often clashing with stringent data protection regulations and patient privacy concerns. This necessitates future research into privacy-preserving AI techniques, such as federated learning, and secure data sharing frameworks that can unlock the full potential of AI without compromising patient trust. Furthermore, the ethical and practical considerations extend beyond data privacy to encompass issues of accountability, transparency, and equitable access. \cite{sidiq2023692} highlights challenges in implementing AI for physiotherapy clinical trials in India, including data security, ethical considerations, and the need for specialized training, reinforcing the global nature of these barriers.

In conclusion, the path forward for AI in clinical trials demands a concerted effort to address these multifaceted tensions. Future research must prioritize developing more robust, generalizable, and causally-aware AI models that can perform reliably across diverse populations and clinical settings. Enhancing causal inference capabilities within AI models, particularly in understanding the true impact of interventions and the sources of bias, is paramount. Crucially, fostering interdisciplinary collaboration among AI researchers, clinicians, ethicists, regulators, and social scientists is essential to navigate the complex ethical, social, and practical barriers to widespread adoption and to ensure equitable access to AI-driven advancements in healthcare. This integrated approach will be key to realizing the transformative potential of AI in clinical trials responsibly and effectively.