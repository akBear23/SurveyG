\subsection*{Large Language Models (LLMs) for Documentation and Synthesis}

Large Language Models (LLMs) are rapidly transforming the landscape of clinical trials by automating and enhancing complex, text-heavy tasks, promising significant efficiency gains, improved consistency, and reduced administrative burden. This section delves into their burgeoning applications in generating trial protocols, drafting informed consent forms, extracting structured information from unstructured clinical notes, and synthesizing vast amounts of scientific literature for evidence generation, while critically addressing their inherent challenges.

One of the most impactful applications of LLMs is in the generation and optimization of clinical trial protocols. Traditionally, authoring detailed protocols is a time-consuming and error-prone process. Recent research demonstrates LLMs' capability to streamline this. For instance, \cite{maleki2024hwz} explored the use of GPT-4 for clinical trial protocol authoring. Their methodology involved detailed analysis and preparation of drug and study-level metadata, followed by prompt engineering to generate specific protocol sections. The study reported significant improvements in efficiency, accuracy, and customization, highlighting the potential for LLMs to reduce the manual effort involved. Similarly, \cite{liddicoat2025pdu} proposed a policy framework for developing application-specific language models (ASLMs) for clinical trial design, envisioning enhanced trial efficiency, inclusivity, and safety through automated protocol development. These studies move beyond theoretical potential, offering concrete examples of LLMs assisting in the foundational documentation of trials.

LLMs are also proving instrumental in improving patient communication, particularly concerning informed consent. Patient comprehension of complex medical jargon in informed consent forms (ICFs) remains a critical challenge. \cite{waters2025scl} investigated the potential of GPT-4 to generate patient-friendly summaries from cancer clinical trial ICFs. They evaluated two AI-driven approaches—direct and sequential summarization—finding that sequential summarization yielded higher accuracy and completeness, and significantly improved readability. The study also demonstrated LLMs' ability to create multiple-choice question-answer pairs (MCQAs) to gauge patient understanding, with high concordance to human-annotated responses. While promising, this work also underscored concerns regarding AI hallucinations, accuracy, and ethical considerations, emphasizing the need for refinement and regulatory oversight.

Beyond document generation, LLMs are powerful tools for information extraction and evidence synthesis. The extraction of structured information from unstructured clinical notes, a critical component of real-world evidence (RWE) generation, can be significantly expedited by LLMs \cite{fleurence2024vvo}. This capability allows for more efficient analysis of large collections of RWD, enhancing the speed and quality of RWE. For evidence synthesis, a task exemplified by the meticulous systematic reviews required for clinical practice guidelines like the ASCO guideline for adjuvant endocrine therapy in breast cancer \cite{burstein2019qgx}, LLMs offer substantial assistance. They can automate initial literature screening, summarize key findings, and extract relevant data points, thereby expediting the creation of evidence-based recommendations \cite{fleurence2024vvo}.

However, the application of LLMs in evidence appraisal is not without its complexities. \cite{woelfle2024q61} benchmarked human-AI collaboration for common evidence appraisal tools (PRISMA, AMSTAR, PRECIS-2) using various LLMs (Claude-3-Opus, GPT-4, GPT-3.5, Mixtral-8x22B). Their findings revealed that individual LLMs alone performed worse than human raters in assessing scientific reporting and methodological rigor. While human-AI collaboration improved accuracies (e.g., 89-96\% for PRISMA), it also highlighted the limitations of LLMs for complex tasks like PRECIS-2, where high deferral rates indicated persistent challenges. This suggests that while LLMs can reduce workload for certain aspects of evidence appraisal, they are not yet capable of fully autonomous, high-stakes critical evaluation. Furthermore, LLMs have been explored for summarizing safety-related tables in Clinical Study Reports (CSRs), where prompt engineering with GPT models showed potential but also highlighted the need for improved ingestion of tables, context, and fine-tuning to ensure factual accuracy and lean writing \cite{landman2024w8r}.

Despite these advancements, the deployment of LLMs in high-stakes clinical contexts necessitates a critical and cautious approach due to several inherent challenges. The potential for 'hallucination,' where models generate plausible but factually incorrect information, is a significant concern, as highlighted by \cite{waters2025scl} and further underscored by the need for robust Natural Language Inference (NLI) models to address factual inconsistency and vulnerability to adversarial inputs in biomedical contexts \cite{jullien2024flu}. Such inaccuracies could have severe implications in clinical documentation and patient safety. Moreover, inherent biases present in the training data can be perpetuated or amplified by LLMs, potentially leading to inequitable or inaccurate recommendations, a risk acknowledged by \cite{fleurence2024vvo}.

Therefore, the paramount need for stringent human oversight and rigorous validation processes cannot be overstated. Every piece of documentation or synthesis generated by an LLM must undergo thorough review by clinical experts to ensure accuracy, safety, and ethical compliance \cite{fleurence2024vvo, landman2024w8r}. Mitigation strategies for hallucination, such as retrieval-augmented generation (RAG) which grounds LLM outputs in verified external knowledge, and fine-tuning on domain-specific, curated clinical corpora, are crucial. The development of robust validation frameworks, transparent reporting mechanisms, and continued research into human-AI collaboration models will be essential for building trust and ensuring the responsible integration of LLMs into clinical trial operations, ultimately augmenting human expertise rather than replacing it.