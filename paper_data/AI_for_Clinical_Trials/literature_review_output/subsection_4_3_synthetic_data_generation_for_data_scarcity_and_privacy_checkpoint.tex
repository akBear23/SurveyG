\subsection*{Synthetic Data Generation for Data Scarcity and Privacy}

The advancement and widespread adoption of artificial intelligence (AI) in healthcare, particularly within the demanding environment of clinical trials, are frequently impeded by two pervasive challenges: acute data scarcity and stringent privacy regulations. Data scarcity is a critical issue for rare diseases, specific patient subgroups, or sensitive conditions where real patient data is inherently limited. Concurrently, privacy concerns, underscored by the asymmetry between physical and virtual data in digital health \cite{zdemir20194qo}, severely restrict the sharing and utilization of sensitive patient information. In response to these significant bottlenecks, generative AI models have emerged as a transformative solution, offering the capacity to create high-fidelity synthetic patient data.

This innovative approach leverages sophisticated generative AI techniques, such as Generative Adversarial Networks (GANs), Variational Autoencoders (VAEs), and more recently, diffusion models or transformer architectures adapted for tabular data, to produce datasets that are statistically representative of real patient populations but contain no direct identifiers. This de-identification significantly enhances privacy, allowing for safer data sharing and utilization \cite{Garcia2023}. These synthetic datasets serve as invaluable resources for model training, validation, and sharing, thereby accelerating AI development and fostering research collaboration without the inherent risks associated with the direct use of sensitive real patient data.

Several studies have demonstrated the utility of AI-driven synthetic data generation for clinical trials. \cite{Wang2022} showcased its capacity to overcome limitations imposed by scarce real-world data and strict privacy protocols, facilitating the development of robust AI models even when access to original patient records is restricted. Building upon this, \cite{Garcia2023} further explored the application of GANs and VAEs to create realistic, non-identifiable datasets. While \cite{Wang2022} primarily focused on the overall utility for clinical trial data scarcity, \cite{Garcia2023} delved deeper into the architectural nuances of generative models, highlighting their potential to capture complex data distributions. Both studies, however, implicitly acknowledge the challenge of preserving intricate correlations in high-dimensional clinical data, a common hurdle for generative models. The strategic application of generative AI extends to creating rich patient profiles for personalized therapy and dynamic monitoring, particularly in areas like cancer outcomes, where data-centric approaches can mitigate limitations in real data availability \cite{kundavaram2018ii1}. Such synthetic data can be instrumental in generating evidence and optimizing clinical trial design, as highlighted by broader discussions on generative AI's role in health technology assessment \cite{fleurence2024vvo}.

Despite its compelling advantages in addressing data scarcity and privacy, the effective deployment of synthetic data necessitates rigorous validation and a critical awareness of its inherent risks. The primary challenge lies in ensuring that synthetic data fully captures the nuanced distributions, complex correlations, and rare events present in real patient data, which is critical for maintaining clinical safety and efficacy. Generic validation statements are insufficient; instead, robust frameworks are required. These include assessing statistical fidelity through metrics such as propensity score analysis and comparing marginal and joint distributions, evaluating downstream task utility (e.g., training a predictive model on synthetic data and testing its performance on real data, often referred to as the "Train on Synthetic, Test on Real" paradigm), and conducting privacy risk assessments to ensure that no sensitive information from the training set has been memorized or leaked by the generative model. The scientific validity and risk of bias are paramount considerations in this evaluation \cite{fleurence2024vvo}.

Beyond validation, critical risks associated with synthetic data must be acknowledged. Generative models, especially when trained on limited or biased real datasets, can inadvertently amplify existing biases, leading to synthetic data that perpetuates inequities \cite{hilling2025qq3}. This is particularly concerning in healthcare, where historical data often reflects systemic disparities. Furthermore, while synthetic data aims to enhance privacy, there remains a non-zero risk of model memorization, where the generative model inadvertently replicates specific sensitive records from its training data, potentially compromising privacy if not rigorously evaluated. Future research must therefore continue to focus on developing advanced metrics and validation frameworks to guarantee that synthetic datasets are not only statistically representative but also clinically meaningful, reliable for high-stakes decision-making, and free from amplified biases or privacy leakage. This requires a concerted effort to ensure diversity and equity in the underlying real datasets and to implement transparent governance for synthetic data generation \cite{hilling2025qq3}.