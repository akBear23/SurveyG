\subsection{AI-Driven Patient Recruitment and Matching}
Patient recruitment and eligibility screening remain a critical bottleneck in clinical research, frequently causing significant delays, escalating costs, and contributing to trial failures \cite{askin2023wrv, cascini2022t0a}. Manual screening is a knowledge-intensive and time-consuming task for healthcare providers, often impeded by the sheer volume and complexity of patient data \cite{wang2024s40}. Artificial intelligence (AI), particularly through Natural Language Processing (NLP) and various machine learning techniques, offers transformative solutions to overcome these challenges by streamlining the identification and matching of eligible candidates to complex trial protocols \cite{ismail20233wp}. Indeed, patient recruitment is one of the most common and impactful applications of AI in clinical trials, recognized for its potential to accelerate trial initiation and enhance efficiency \cite{askin2023wrv, cascini2022t0a}.

Early efforts in this domain highlighted AI's potential to revolutionize patient matching. \cite{Liu2017} provided a foundational review, outlining how AI, leveraging NLP to interpret unstructured clinical notes and machine learning models to analyze structured data within Electronic Health Records (EHRs), could predict patient eligibility for clinical trials. This work underscored the critical need for robust systems capable of handling data heterogeneity and privacy concerns inherent in real-world clinical data. Building upon this conceptual understanding, \cite{Wang2018} proposed a concrete AI framework designed to automate patient recruitment, emphasizing the integration of diverse data sources and the use of predictive modeling and rule-based systems to optimize the screening process. Their approach aimed to enhance efficiency and accelerate trial initiation by systematically matching patient profiles against intricate eligibility criteria, thereby reducing manual screening failures.

Further advancements have seen the integration of more sophisticated AI methodologies, particularly deep learning, to improve the accuracy and efficiency of patient matching. \cite{Li2022} introduced a deep learning approach, specifically utilizing BERT-based models, for AI-powered patient recruitment. This method demonstrated superior capabilities in interpreting the nuanced clinical data found in EHRs, enabling more precise identification of suitable candidates and significantly improving the speed and accuracy of the matching process compared to earlier machine learning techniques. The ability of deep learning to discern complex patterns within vast, often noisy, datasets is crucial for navigating the intricate inclusion and exclusion criteria of modern clinical trials.

A significant challenge in leveraging EHRs for patient matching lies in the inherent complexities of unstructured clinical text, which often contains negation, temporality, abbreviations, and context-dependent language that can be difficult for algorithms to interpret accurately. Addressing these specific NLP hurdles, \cite{wang2024s40} presented an AI-based Clinical Trial Matching System (CTMS) specifically designed for Chinese patients with hepatocellular carcinoma. This system innovatively employed Iterated Dilated Convolutional Neural Networks (IDCNN) for Named Entity Recognition (NER) to extract medical entities and Text Convolutional Neural Networks (TextCNN) for entity-relationship linking, effectively handling the "cross-ambiguity and combinatorial ambiguity" unique to Chinese clinical records. Their retrospective study demonstrated high accuracy (92.9–98.0\%) and specificity (99.0–99.1\%), alongside a remarkable 98.7\% reduction in screening time compared to manual review. This showcases the power of tailored deep learning solutions to overcome linguistic and semantic complexities in diverse healthcare contexts, marking a critical evolution from general NLP applications to specialized models capable of extracting highly nuanced information essential for precise eligibility screening.

Beyond initial recruitment, AI also plays a crucial role in improving patient retention throughout the study, a factor critical for overall study success and the integrity of trial outcomes \cite{ismail20233wp}. AI models can analyze patient demographics, historical adherence data, and real-time engagement metrics to predict individuals at high risk of dropout, allowing for proactive interventions. For instance, AI-driven chatbots, leveraging advances in NLP, can enhance patient-clinician interaction by providing round-the-clock assistance, personalized information on trial processes, medication regimens, and potential side effects \cite{voola20229e1}. This consistent availability of information and support can significantly decrease the cognitive burden on patients, augment their comprehension of the trial process, and improve compliance with trial guidelines, thereby fostering better patient satisfaction and retention \cite{voola20229e1}. Furthermore, AI's capacity to harness biomarkers for accurately matching patients to clinical trials, as noted by \cite{Ho2020}, ensures that patients are directed towards trials where they are most likely to benefit, which inherently improves their engagement and likelihood of retention by aligning their therapeutic needs with study objectives.

The effective deployment and scalability of AI-driven recruitment and matching systems critically depend on secure and interoperable data infrastructure. These advanced AI models require access to vast amounts of sensitive patient data, often distributed across multiple institutions. Addressing the underlying challenges of data privacy, security, and secure exchange, \cite{Rana2022} proposed a decentralized access control model utilizing blockchain technology for healthcare data, including clinical trial information. Such an infrastructure is vital for enabling AI systems to securely access and process sensitive patient data across multiple institutions without compromising privacy, a prerequisite for the widespread adoption and scalability of AI-driven recruitment platforms. This facilitates the aggregation of diverse EHR data, which is essential for training robust predictive models and rule-based systems that can interpret nuanced clinical data effectively while adhering to stringent privacy regulations.

In conclusion, AI-driven patient recruitment and matching systems have undergone significant advancements, evolving from conceptual frameworks to sophisticated deep learning applications that leverage EHRs to identify eligible candidates, automate matching against complex protocols, and enhance patient retention. These innovations demonstrably improve efficiency, reduce screening failures, and accelerate trial initiation by overcoming a major bottleneck in clinical research. However, ongoing challenges persist, including ensuring the generalizability and robustness of models across diverse healthcare systems and patient populations, addressing ethical considerations related to potential biases in AI decision-making, and establishing robust, privacy-preserving data infrastructures to support these advanced systems. Future directions within this domain will continue to focus on developing more robust and adaptable NLP models for varied linguistic contexts, enhancing methods for training models on distributed data without compromising patient privacy, and improving the transparency and interpretability of algorithmic recommendations for clinical stakeholders.