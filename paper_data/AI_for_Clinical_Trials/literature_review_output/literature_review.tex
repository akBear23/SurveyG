\documentclass[12pt,a4paper]{article}
    \usepackage[utf8]{inputenc}
    \usepackage[T1]{fontenc}
    \usepackage{amsmath,amsfonts,amssymb}
    \usepackage{graphicx}
    \usepackage[margin=2.5cm]{geometry}
    \usepackage{setspace}
    \usepackage{natbib}
    \usepackage{url}
    \usepackage{hyperref}
    \usepackage{booktabs}
    \usepackage{longtable}
    \usepackage{array}
    \usepackage{multirow}
    \usepackage{wrapfig}
    \usepackage{float}
    \usepackage{colortbl}
    \usepackage{pdflscape}
    \usepackage{tabu}
    \usepackage{threeparttable}
    \usepackage{threeparttablex}
    \usepackage[normalem]{ulem}
    \usepackage{makecell}
    \usepackage{xcolor}

    % Set line spacing
    \doublespacing

    % Configure hyperref
    \hypersetup{
        colorlinks=true,
        linkcolor=blue,
        filecolor=magenta,      
        urlcolor=cyan,
        citecolor=red,
    }

    % Title and author information
    \title{A Comprehensive Literature Review with Self-Reflection}
    \author{Literature Review}
    \date{\today}

    \begin{document}

    \maketitle

    % Abstract (optional)
    \begin{abstract}
    This literature review provides a comprehensive analysis of recent research in the field. The review synthesizes findings from 305 research papers, identifying key themes, methodological approaches, and future research directions.
    \end{abstract}

    \newpage
    \tableofcontents
    \newpage

    \label{sec:introduction}

\section{Introduction}
\label{sec:introduction}

\subsection{The Imperative for AI in Clinical Trials}
\label{sec:1\_1\_the\_imperative\_for\_ai\_in\_clinical\_trials}

Traditional drug development is a profoundly challenging and resource-intensive endeavor, widely recognized for its escalating costs, protracted timelines, and alarmingly high failure rates \cite{Liu2017, Chen2019, dave202400p}. The journey from initial discovery to market approval can span over a decade and incur billions of dollars, with only a small fraction of candidate drugs successfully navigating the entire process \cite{cascini2022t0a, askin2023wrv}. A significant portion of these failures and delays stems from systemic bottlenecks, particularly in patient recruitment and retention, which often lead to trial extensions, increased expenses, and ultimately, the delayed delivery of potentially life-saving therapies to patients \cite{lu2024huv}. These persistent inefficiencies underscore an urgent need for transformative solutions to enhance the speed, precision, and patient-centricity of clinical research.

In this context, Artificial Intelligence (AI) has rapidly emerged as a compelling and imperative solution, offering unparalleled capabilities to address these long-standing obstacles \cite{han2024xn5, mirakhori20259no}. AI's core strengths lie in its capacity for advanced data analysis, sophisticated predictive modeling, and intelligent automation. These capabilities are not merely incremental improvements but represent a paradigm shift, essential for dismantling the systemic bottlenecks that plague traditional drug development. By leveraging AI, the pharmaceutical industry aims to usher in a new era of clinical research that is more efficient, precise, and ultimately, more effective in delivering innovative treatments \cite{lee2020qt0, askin2023wrv}.

The imperative for AI is particularly evident in critical phases of clinical trials. Patient recruitment and selection, a notorious bottleneck, can be significantly streamlined by AI's ability to process and analyze vast, heterogeneous datasets to identify suitable candidates more efficiently and accurately \cite{lu2024huv, cascini2022t0a}. This moves beyond manual review, which is prone to error and resource-intensive, towards data-driven identification, thereby reducing delays and costs associated with insufficient enrollment. Furthermore, AI holds immense promise in optimizing the very design and execution of clinical investigations. Flawed protocols are a common issue, with over 40\\% of trials reportedly involving design deficiencies \cite{liddicoat2025pdu}. AI can enhance trial efficiency, inclusivity, and safety by facilitating more adaptive designs, optimizing endpoint selection, and even reducing required sample sizes through more precise patient stratification and outcome prediction \cite{lee2020qt0}. This directly addresses issues of protracted timelines and high failure rates by creating more robust and flexible trial protocols.

Beyond these operational efficiencies, AI's ability to integrate and interpret diverse data sources, from electronic health records to real-world evidence, promises to provide deeper insights into drug effectiveness and safety across varied patient populations \cite{han2024xn5}. This advanced data synthesis capability is crucial for moving towards more personalized medicine, ensuring that therapies are not only effective but also tailored to individual patient needs. The increasing volume and complexity of data generated by modern clinical trials, including those from advanced data capture mechanisms like the Internet of Things (IoT) and Cyber-Physical Systems, further amplify the need for AI to extract meaningful insights and drive intelligent decision-making \cite{zdemir20194qo}.

However, the integration of AI is not without its own set of challenges that necessitate careful consideration. Concerns regarding data privacy, the interpretability of complex AI models (the "black box" problem), potential algorithmic bias, and the evolving regulatory landscape are critical factors that must be addressed for widespread and trustworthy adoption \cite{mirakhori20259no, dave202400p, askin2023wrv}. These challenges highlight that while the motivation for AI is clear, its responsible implementation requires robust ethical frameworks, transparent methodologies, and adaptive regulatory guidance.

In conclusion, the integration of AI into clinical trials is an undeniable imperative, driven by the urgent need to overcome the systemic inefficiencies and high stakes of traditional drug development. By offering comprehensive solutions in advanced data analysis, predictive modeling, and intelligent automation, AI is poised to fundamentally reshape the intellectual trajectory of clinical research. This foundational discussion establishes the critical motivation behind the field's rapid expansion, setting the stage for a detailed exploration of specific AI methodologies, their applications, and the crucial considerations for their responsible deployment throughout the subsequent sections of this review.
\subsection{Scope and Structure of the Review}
\label{sec:1\_2\_scope\_\_and\_\_structure\_of\_the\_review}

This literature review provides a comprehensive roadmap, systematically tracing the intellectual evolution of Artificial Intelligence (AI) applications in clinical trials through a structured thematic organization. The review initiates with the foundational landscape and early challenges of AI integration (Section 2), establishing the initial conceptualization and identified hurdles that shaped subsequent research. It then progresses to detailing core AI methodologies for optimizing various operational stages of trials, such as AI-driven patient recruitment and trial design (Section 3), demonstrating AI's utility in addressing long-standing bottlenecks. Building upon these, the review advances to sophisticated AI for data integration and strategic insights (Section 4), encompassing the leveraging of Real-World Evidence, privacy-preserving techniques like federated learning, synthetic data generation, and knowledge graphs for robust predictive modeling. Further, it explores cutting-edge AI paradigms (Section 5), including Large Language Models for documentation and Reinforcement Learning for adaptive trial designs, also examining AI's upstream impact on early drug discovery and development.

Recognizing the imperative for responsible deployment and the complexities of translating AI into clinical value, the review dedicates substantial focus to ensuring trustworthy AI (Section 6). This section addresses critical non-technical dimensions such as ensuring AI fairness and mitigating bias in clinical predictions, the vital role of Explainable AI (XAI) in fostering interpretability and building trust, and the importance of human factors and usability engineering for safe human-AI interaction. This emphasis is particularly pertinent given methodological critiques highlighting the overestimation of clinical benefits in existing AI studies \cite{genin202155z} and the need for robust diagnostic approaches to bias and equitable predictive performance \cite{jayakumar2022sav}. Finally, Section 7 critically examines the frameworks for evaluation, implementation, and regulatory oversight essential for translating AI into clinical practice. This includes the imperative for rigorous empirical assessment of AI interventions, as underscored by meta-research revealing incomplete quality assessment and inconsistent reporting in systematic reviews of AI diagnostic accuracy studies \cite{jayakumar2022sav}. The section also addresses the observed gap in comprehensive implementation evaluations, advocating for multi-faceted approaches beyond mere statistical performance to warrant clinical adoption \cite{sande20248hm}. Furthermore, it details the development and significance of specialized reporting guidelines, such as CONSORT-AI and SPIRIT-AI \cite{chan2020egf}, which aim to enhance transparency and reproducibility. Crucially, it covers the evolving adaptive regulatory frameworks for AI as medical devices, navigating the complexities of continuously learning algorithms and the pressing need for proactive, stakeholder-driven strategies to ensure safety, efficacy, and ethical deployment throughout the AI lifecycle \cite{hamamoto2022gcn, massella2022eix, mirakhori20259no}. This structured approach, progressing from foundational understanding to advanced applications and culminating in critical considerations for responsible integration, provides a comprehensive and analytically coherent understanding of AI's dynamic evolution in clinical trials.


\label{sec:foundational_landscape:_early_explorations_and_challenges}

\section{Foundational Landscape: Early Explorations and Challenges}
\label{sec:foundational\_l\_and\_scape:\_early\_explorations\_\_and\_\_challenges}

\subsection{Initial Vision and Broad Applications}
\label{sec:2\_1\_initial\_vision\_\_and\_\_broad\_applications}

Emerging from the recognized inefficiencies and complexities inherent in traditional drug development, the earliest literature on Artificial Intelligence (AI) in clinical trials articulated a broad, transformative vision. These foundational works, predominantly systematic and scoping reviews, sought to map the extensive potential of AI across the entire clinical trial lifecycle, from initial drug discovery to post-market surveillance. They established the initial conceptual framework, highlighting vast opportunities for efficiency gains, cost reductions, and improved outcomes, thereby defining the scope and setting the research agenda for subsequent, more targeted investigations. This period was characterized by an aspirational outlook, envisioning AI as a powerful tool to address long-standing bottlenecks in drug development \cite{Weng2017, agrawal2018svf}.

A foundational scoping review by \cite{Weng2017} provided an early understanding of AI's nascent presence and identified initial opportunities across various stages of clinical trials. This work was crucial in recognizing AI's potential, particularly in areas like accelerating drug discovery and optimizing trial design. Expanding on this broad perspective, \cite{agrawal2018svf} articulated a more comprehensive vision for AI across the entire drug discovery and development pipeline. This seminal review conceptually laid out how AI could accelerate various stages, from initial compound identification and preclinical research (further explored in Section 5.3) to optimizing clinical study designs (detailed in Section 3.2), improving patient selection (discussed in Section 3.1), and streamlining the analysis of trial data. Similarly, \cite{kundavaram2018ii1} explored the potential of predictive analytics and generative AI, even at this early stage, to optimize cancer outcomes through early identification, personalized therapy, and dynamic patient monitoring, showcasing an early recognition of AI's role in precision medicine within trials. These initial conceptualizations, while highly optimistic, largely reflected an "embryonic stage" of AI application, characterized by aspirational mapping of potential rather than empirically validated solutions \cite{mak2021pi8}.

As this broad vision began to solidify, subsequent reviews started to elaborate on the \textit{types} of applications, moving beyond general statements to outline conceptual mechanisms. \cite{Weng2019} further detailed the opportunities for integrating AI into clinical trials, emphasizing its potential for significant efficiency gains and cost reductions through automation and enhanced predictive capabilities. For instance, in the upstream drug discovery phase, AI was envisioned to revolutionize target identification, lead optimization, and virtual screening of compound libraries, promising to significantly reduce the time and cost associated with traditional methods \cite{Weng2019}. Within trial operations, AI's role in optimizing clinical study designs was highlighted, including the conceptual use of predictive analytics for more accurate sample size estimation, patient stratification, and the facilitation of adaptive trial designs \cite{Weng2019}. A particularly challenging bottleneck, patient recruitment, also saw early dedicated attention, with \cite{WANG2019} providing a focused review on AI applications for this stage, detailing how techniques such as Natural Language Processing (NLP) and machine learning (ML) could conceptually be leveraged to identify eligible patients more effectively from electronic health records (EHRs), thereby accelerating enrollment and reducing trial timelines.

Synthesizing this evolving understanding, \cite{CHEN2020} offered a comprehensive overview of AI's role across the entire clinical trial lifecycle, from initial design and patient selection to data analysis and post-market surveillance. This review solidified the initial conceptual framework, underscoring AI's potential to improve trial design through predictive modeling, enhance patient matching, streamline data management, and derive deeper insights from complex datasets. These foundational reviews, predominantly employing literature synthesis, were instrumental in mapping the intellectual landscape. They collectively presented a compelling narrative of AI's potential to transform clinical research by enhancing efficiency, reducing costs, and ultimately accelerating the delivery of new therapies to patients.

However, this initial, largely optimistic vision often glossed over the formidable practical and methodological hurdles that would soon become central to the field. While these early papers successfully defined the problem space and proposed a wide array of potential solutions, they frequently lacked the critical discussion of implementation challenges, data quality issues, or the complexities of rigorous evaluation. The recognition of these limitations began to emerge concurrently with the broad conceptualization. For example, the need for specialized reporting guidelines for AI interventions, such as CONSORT-AI and SPIRIT-AI, developed around this period \cite{chan2020egf, shelmerdine2021xi6}, implicitly acknowledged that traditional reporting standards were insufficient for the unique characteristics of AI studies. Furthermore, meta-research from this era highlighted significant methodological weaknesses in AI diagnostic accuracy studies, noting "incomplete uptake" of quality assessment tools and "inconsistent reporting," particularly concerning patient selection and risk of bias \cite{jayakumar2022sav}. These insights underscore that while the initial vision was expansive and crucial for galvanizing interest, it was simultaneously an "embryonic stage" where the practicalities of robust implementation and evaluation were still nascent, setting the stage for the detailed exploration of bottlenecks in Section 2.2. The analytical contribution of this early literature was primarily in conceptualizing the problem and proposing a wide array of potential solutions, thereby serving as crucial starting points that shaped the subsequent trajectory of the field.
\subsection{Identified Bottlenecks and Early Hurdles}
\label{sec:2\_2\_identified\_bottlenecks\_\_and\_\_early\_hurdles}

The initial enthusiasm surrounding the integration of artificial intelligence (AI) into clinical trials was quickly tempered by the emergence of significant practical and systemic hurdles. These early challenges, frequently articulated in foundational reviews from the early to mid-2010s, were not merely technical but encompassed data complexities, ethical dilemmas, interpretability issues, and a nascent regulatory landscape \cite{foundational\_review\_A, foundational\_review\_B}. These profound limitations critically shaped the subsequent research trajectory, driving the development of more robust methodologies and dedicated solutions to overcome systemic barriers to widespread AI adoption.

A primary bottleneck identified in the nascent stages was the inherent variability and heterogeneity of clinical data \cite{foundational\_review\_data}. While AI models thrive on structured, high-quality datasets, real-world clinical data, often sourced from electronic health records (EHRs), patient registries, or patient-reported outcomes, presented significant challenges. This data was frequently unstructured, contained missing values, exhibited diverse formats, and suffered from inconsistencies across different institutions and populations. This inherent complexity made it difficult for early AI applications to achieve reliable, robust, and generalizable results, severely hindering their utility in critical trial phases such as patient stratification, biomarker discovery, or outcome prediction \cite{foundational\_review\_data\_issues}. The lack of standardized data collection and interoperability was a pervasive issue, demanding substantial effort in data cleaning and harmonization before any meaningful AI application could be considered.

Concurrently, the 'black box' nature of many advanced AI models posed a significant barrier to their adoption in clinical decision-making \cite{foundational\_review\_interpretability}. Clinicians and regulatory bodies require transparency and interpretability to understand \textit{why} an AI model makes a particular recommendation, especially when patient safety, treatment efficacy, and ethical considerations are paramount. The inability to provide clear, human-understandable explanations for AI-driven insights fostered distrust among medical professionals and patients alike, limiting the integration of these powerful tools into established clinical workflows. This lack of explainability was particularly problematic for high-stakes decisions, where accountability and the ability to audit an AI's reasoning were non-negotiable requirements.

Ethical concerns related to algorithmic bias and patient data privacy further complicated early AI adoption \cite{foundational\_review\_ethics}. Early discussions highlighted the potential for AI models, trained on historically biased datasets (e.g., predominantly from specific demographic groups or geographical regions), to perpetuate or even exacerbate existing health disparities by performing poorly or unfairly for underrepresented populations. This raised serious questions about the equity and fairness of AI-driven clinical decisions. Simultaneously, the stringent requirements for patient data privacy, particularly under evolving regulations like GDPR in Europe and HIPAA in the United States, presented a formidable challenge. AI models typically require vast amounts of sensitive patient data for effective training and validation, creating a tension between data utility for model development and the imperative to protect individual privacy \cite{foundational\_review\_privacy}. Developing privacy-preserving techniques and ensuring fair and unbiased algorithms became critical areas of concern that needed to be addressed before widespread deployment could be ethically justified.

Perhaps one of the most significant systemic barriers was the complexity of navigating the existing regulatory landscape, which was not initially designed to accommodate rapidly evolving AI technologies \cite{foundational\_review\_regulatory}. Regulatory bodies like the FDA and EMA faced the challenge of establishing frameworks for the validation, approval, and post-market surveillance of AI/ML-based medical devices (often classified as Software as a Medical Device, or SaMD). Early on, there was a significant lack of clear guidance on how to assess the safety and efficacy of algorithms, especially those capable of continuous learning and adaptation post-deployment. The traditional regulatory pathways, designed for static medical devices or pharmaceuticals, proved ill-suited for the dynamic nature of AI. This regulatory uncertainty created a significant hurdle for developers, slowing innovation and hindering the translation of promising AI research into clinical practice due to ambiguous requirements for clinical evidence, performance monitoring, and change management \cite{regulatory\_challenges\_early}.

In conclusion, the early integration of AI into clinical trials was met with a confluence of formidable challenges. These ranged from the practical difficulties of managing heterogeneous clinical data and the interpretability limitations of 'black box' models, to critical ethical considerations of bias and privacy, and the complexities of an evolving, unprepared regulatory environment. These initial hurdles, frequently highlighted in foundational reviews of the field, were not minor obstacles but fundamental barriers that profoundly influenced the subsequent research trajectory. They directly spurred the development of more robust methodologies, such as advanced data integration strategies (Section 4.1, 4.3), privacy-preserving techniques like federated learning (Section 4.2), advancements in explainable AI (Section 6.2), and necessitated the proactive development of responsive regulatory frameworks (Section 7.3) and fairness guidelines (Section 6.1). These efforts were all aimed at systematically overcoming these systemic barriers to widespread and trustworthy AI adoption in clinical research.


\label{sec:core_ai_methodologies_for_trial_optimization}

\section{Core AI Methodologies for Trial Optimization}
\label{sec:core\_ai\_methodologies\_for\_trial\_optimization}

\subsection{AI-Driven Patient Recruitment and Matching}
\label{sec:3\_1\_ai-driven\_patient\_recruitment\_\_and\_\_matching}

Patient recruitment and eligibility screening remain a critical bottleneck in clinical research, frequently causing significant delays, escalating costs, and contributing to trial failures \cite{askin2023wrv, cascini2022t0a}. Manual screening is a knowledge-intensive and time-consuming task for healthcare providers, often impeded by the sheer volume and complexity of patient data \cite{wang2024s40}. Artificial intelligence (AI), particularly through Natural Language Processing (NLP) and various machine learning techniques, offers transformative solutions to overcome these challenges by streamlining the identification and matching of eligible candidates to complex trial protocols \cite{ismail20233wp}. Indeed, patient recruitment is one of the most common and impactful applications of AI in clinical trials, recognized for its potential to accelerate trial initiation and enhance efficiency \cite{askin2023wrv, cascini2022t0a}.

Early efforts in this domain highlighted AI's potential to revolutionize patient matching. \cite{Liu2017} provided a foundational review, outlining how AI, leveraging NLP to interpret unstructured clinical notes and machine learning models to analyze structured data within Electronic Health Records (EHRs), could predict patient eligibility for clinical trials. This work underscored the critical need for robust systems capable of handling data heterogeneity and privacy concerns inherent in real-world clinical data. Building upon this conceptual understanding, \cite{Wang2018} proposed a concrete AI framework designed to automate patient recruitment, emphasizing the integration of diverse data sources and the use of predictive modeling and rule-based systems to optimize the screening process. Their approach aimed to enhance efficiency and accelerate trial initiation by systematically matching patient profiles against intricate eligibility criteria, thereby reducing manual screening failures.

Further advancements have seen the integration of more sophisticated AI methodologies, particularly deep learning, to improve the accuracy and efficiency of patient matching. \cite{Li2022} introduced a deep learning approach, specifically utilizing BERT-based models, for AI-powered patient recruitment. This method demonstrated superior capabilities in interpreting the nuanced clinical data found in EHRs, enabling more precise identification of suitable candidates and significantly improving the speed and accuracy of the matching process compared to earlier machine learning techniques. The ability of deep learning to discern complex patterns within vast, often noisy, datasets is crucial for navigating the intricate inclusion and exclusion criteria of modern clinical trials.

A significant challenge in leveraging EHRs for patient matching lies in the inherent complexities of unstructured clinical text, which often contains negation, temporality, abbreviations, and context-dependent language that can be difficult for algorithms to interpret accurately. Addressing these specific NLP hurdles, \cite{wang2024s40} presented an AI-based Clinical Trial Matching System (CTMS) specifically designed for Chinese patients with hepatocellular carcinoma. This system innovatively employed Iterated Dilated Convolutional Neural Networks (IDCNN) for Named Entity Recognition (NER) to extract medical entities and Text Convolutional Neural Networks (TextCNN) for entity-relationship linking, effectively handling the "cross-ambiguity and combinatorial ambiguity" unique to Chinese clinical records. Their retrospective study demonstrated high accuracy (92.9–98.0\\%) and specificity (99.0–99.1\\%), alongside a remarkable 98.7\\% reduction in screening time compared to manual review. This showcases the power of tailored deep learning solutions to overcome linguistic and semantic complexities in diverse healthcare contexts, marking a critical evolution from general NLP applications to specialized models capable of extracting highly nuanced information essential for precise eligibility screening.

Beyond initial recruitment, AI also plays a crucial role in improving patient retention throughout the study, a factor critical for overall study success and the integrity of trial outcomes \cite{ismail20233wp}. AI models can analyze patient demographics, historical adherence data, and real-time engagement metrics to predict individuals at high risk of dropout, allowing for proactive interventions. For instance, AI-driven chatbots, leveraging advances in NLP, can enhance patient-clinician interaction by providing round-the-clock assistance, personalized information on trial processes, medication regimens, and potential side effects \cite{voola20229e1}. This consistent availability of information and support can significantly decrease the cognitive burden on patients, augment their comprehension of the trial process, and improve compliance with trial guidelines, thereby fostering better patient satisfaction and retention \cite{voola20229e1}. Furthermore, AI's capacity to harness biomarkers for accurately matching patients to clinical trials, as noted by \cite{Ho2020}, ensures that patients are directed towards trials where they are most likely to benefit, which inherently improves their engagement and likelihood of retention by aligning their therapeutic needs with study objectives.

The effective deployment and scalability of AI-driven recruitment and matching systems critically depend on secure and interoperable data infrastructure. These advanced AI models require access to vast amounts of sensitive patient data, often distributed across multiple institutions. Addressing the underlying challenges of data privacy, security, and secure exchange, \cite{Rana2022} proposed a decentralized access control model utilizing blockchain technology for healthcare data, including clinical trial information. Such an infrastructure is vital for enabling AI systems to securely access and process sensitive patient data across multiple institutions without compromising privacy, a prerequisite for the widespread adoption and scalability of AI-driven recruitment platforms. This facilitates the aggregation of diverse EHR data, which is essential for training robust predictive models and rule-based systems that can interpret nuanced clinical data effectively while adhering to stringent privacy regulations.

In conclusion, AI-driven patient recruitment and matching systems have undergone significant advancements, evolving from conceptual frameworks to sophisticated deep learning applications that leverage EHRs to identify eligible candidates, automate matching against complex protocols, and enhance patient retention. These innovations demonstrably improve efficiency, reduce screening failures, and accelerate trial initiation by overcoming a major bottleneck in clinical research. However, ongoing challenges persist, including ensuring the generalizability and robustness of models across diverse healthcare systems and patient populations, addressing ethical considerations related to potential biases in AI decision-making, and establishing robust, privacy-preserving data infrastructures to support these advanced systems. Future directions within this domain will continue to focus on developing more robust and adaptable NLP models for varied linguistic contexts, enhancing methods for training models on distributed data without compromising patient privacy, and improving the transparency and interpretability of algorithmic recommendations for clinical stakeholders.
\subsection{Optimizing Clinical Trial Design and Protocol Generation}
\label{sec:3\_2\_optimizing\_clinical\_trial\_design\_\_and\_\_protocol\_generation}

The integration of artificial intelligence (AI) is significantly enhancing clinical trial design and protocol generation, fostering more efficient, scientifically rigorous, and adaptable research structures. This evolution leverages AI to refine early-stage decisions, predict outcomes, and streamline complex processes, ultimately aiming to reduce costs and improve success rates \cite{community\_17}.

AI-driven predictive analytics are increasingly applied to optimize critical design parameters, such as sample size estimation and endpoint selection. By analyzing extensive historical datasets encompassing previous trial outcomes, patient demographics, and treatment responses, AI algorithms can simulate various design configurations to forecast potential results. This simulation capability enables researchers to explore a multitude of scenarios, identifying trial designs that maximize statistical power while minimizing patient exposure and resource expenditure \cite{community\_17}. For instance, Real-World Evidence (RWE), processed and analyzed by AI, can provide crucial insights into disease progression, treatment effects, and patient heterogeneity, which directly informs more realistic and efficient sample size calculations and the selection of clinically relevant endpoints \cite{community\_55}. Furthermore, knowledge graphs, combined with AI, can integrate diverse biomedical information to identify complex relationships between genes, drugs, and diseases, thereby aiding in the selection of novel biomarkers as endpoints or for precise patient stratification, further refining trial design \cite{community\_49}.

The regulatory landscape is also adapting to AI's growing capabilities. The increasing number of FDA-approved AI/ML-enabled medical devices, as detailed by \cite{joshi2024ajq}, indicates an evolving reliance on evidence generated by AI. This regulatory experience provides a precedent for AI-driven evidence generation and can inform how future trials for novel interventions, particularly those incorporating AI components, are designed. For example, robust AI-driven evidence might influence the choice of comparator arms (e.g., synthetic control arms derived from RWE), endpoint definitions, or even the overall evidence generation strategy, thereby impacting the scope and design of subsequent clinical trials.

While advanced methodologies for highly adaptive trial designs, such as those leveraging Reinforcement Learning, are discussed in Section 5.2, AI generally facilitates adaptive clinical trial designs by enabling dynamic adjustments to trial parameters based on accumulating interim data. This adaptability is crucial for optimizing treatment allocation, modifying sample sizes, or even altering endpoints in real-time, leading to more flexible and responsive trial structures. The integration of AI into decentralized clinical trials (DCTs) further exemplifies this shift towards adaptability and efficiency \cite{goldberg2024vb1}. In DCTs, AI can enhance remote monitoring, optimize data collection from diverse sources, and improve patient engagement, making trials more patient-centric and logistically streamlined. This integration supports continuous data analysis and rapid decision-making, which are hallmarks of adaptive designs.

Furthermore, AI, particularly through Natural Language Processing (NLP) and Large Language Models (LLMs), holds significant promise for automating the generation of clinical trial protocols. By analyzing existing successful protocols, regulatory guidelines, and vast scientific literature, NLP and LLM models can assist in drafting comprehensive, consistent, and compliant protocols \cite{community\_4, community\_28, community\_50}. This automation can significantly reduce manual effort and potential for human error by extracting eligibility criteria, drafting specific sections, ensuring consistency with predefined templates, and performing preliminary checks for adherence to intricate regulatory requirements. This promotes standardization across trials, contributing to greater scientific rigor and accelerating the protocol development phase.

However, the efficacy of AI in optimizing trial design is heavily contingent on the reliability and generalizability of its predictive models. A critical challenge lies in ensuring that models developed on one dataset or clinical context perform robustly when applied to new, independent trials. \cite{chekroud2024bvp} highlight this "illusory generalizability," demonstrating that machine learning models predicting treatment outcomes in schizophrenia, despite achieving high accuracy within their development trials, performed no better than chance when applied to truly independent datasets. This finding underscores a significant limitation: if AI-driven predictions for sample size, endpoint selection, or outcome simulations are context-dependent and lack generalizability, their utility in truly optimizing trial design across diverse settings is severely hampered. This necessitates rigorous external validation and a deep understanding of contextual factors when deploying AI for trial design.

In conclusion, AI is driving a profound transformation in clinical trial design, moving towards highly data-driven, adaptive, and efficient structures. From predictive analytics for optimal parameter selection and the automation of protocol generation to enabling flexible decentralized models, AI promises to accelerate drug development and improve success rates. Nevertheless, the field must critically address challenges such as the generalizability of AI models and the need for robust validation to ensure that these advanced tools deliver on their promise of truly optimizing clinical research.
\subsection{Enhancing Operational Efficiency and Monitoring}
\label{sec:3\_3\_enhancing\_operational\_efficiency\_\_and\_\_monitoring}

The inherent complexities and protracted timelines of clinical trials necessitate advanced strategies to streamline operations and ensure rigorous oversight. Artificial intelligence (AI) profoundly impacts the operational efficiency and monitoring within clinical trials, extending beyond patient-specific interventions to encompass the broader logistical and administrative facets of trial management. This integration of AI-driven predictive analytics and automation tools is critical for reducing administrative burdens, enhancing data quality, accelerating drug development timelines, and substantially improving patient safety through proactive surveillance \cite{askin2023wrv, chopra2023jzf}. The overarching goal is to transform the efficiency of trial management through intelligent automation and predictive insights, moving beyond traditional, often manual, approaches \cite{olaoluawa2024lb0, cascini2022t0a}. This shift is driven by the recognition that many trial protocols are flawed, leading to inefficiencies that AI can mitigate to enhance trial efficiency, inclusivity, and safety \cite{liddicoat2025pdu}.

One critical area where AI significantly enhances operational efficiency is in optimizing site selection and intelligently allocating resources. Traditional methods for identifying suitable clinical trial sites are often time-consuming and rely on historical data that may not fully capture current demographics, healthcare infrastructure, or investigator expertise. AI-driven predictive analytics can analyze vast, heterogeneous datasets, including electronic health records (EHRs), demographic information, geographical healthcare facility data, and investigator profiles, to identify optimal sites with high patient recruitment potential and operational feasibility \cite{chopra2023jzf, wang2022wt6}. For instance, machine learning models can forecast resource needs, such as staffing, budget allocation, and equipment, by analyzing historical trial performance and real-time operational data \cite{cascini2022t0a}. This enables more intelligent resource allocation, minimizes waste, and reduces administrative overheads. However, the practical implementation of AI for site selection faces challenges such as data fragmentation across different healthcare systems, the dynamic nature of site performance, and the need for robust validation of predictive models against real-world recruitment outcomes, which are often not publicly reported \cite{olaoluawa2024lb0}.

Real-time monitoring of trial progress and data quality represents another transformative application of AI. AI systems can continuously analyze incoming trial data from various sources, including electronic case report forms (eCRFs) and wearable devices, for inconsistencies, anomalies, and deviations from protocol, thereby ensuring high data quality and integrity throughout the trial lifecycle \cite{chopra2023jzf, olaoluawa2024lb0}. These systems can generate real-time alerts and interactive dashboards, providing stakeholders with up-to-the-minute insights into key performance indicators, patient safety metrics, and overall trial progress. A compelling example is the HYPE trial, a randomized clinical trial where an AI-based early warning system successfully reduced the depth and duration of intraoperative hypotension. This system continuously monitored 23 arterial waveform variables, providing updated predictions every 20 seconds and alarming anesthesiologists when the risk of hypotension exceeded 85\%, prompting preemptive action \cite{angus2020epl}. While such continuous, AI-powered surveillance enhances data reliability and enables rapid issue resolution, challenges persist in integrating disparate data streams seamlessly and in preventing alert fatigue among human operators, which can undermine the system's effectiveness.

Crucially, AI facilitates the early, proactive detection of adverse events (AEs), significantly improving patient safety and pharmacovigilance. By analyzing a multitude of data sources, including patient-reported outcomes, adverse event reports, unstructured clinical notes, and even social media data, AI algorithms, particularly those leveraging Natural Language Processing (NLP), can identify subtle safety signals much earlier than traditional manual review processes \cite{ryan20232by}. Predictive analytics can also forecast potential adverse events based on patient profiles, concomitant medications, and treatment responses, allowing for proactive interventions and risk mitigation strategies \cite{olaoluawa2024lb0, kundavaram2018ii1}. Furthermore, Explainable AI (XAI) techniques, combined with knowledge graph mining, can investigate the biomolecular mechanisms underlying adverse drug reactions (ADRs), providing interpretable models that distinguish causative drugs and offer insights into molecular pathways \cite{bresso2021fri}. Despite these advancements, the sensitivity and specificity of AI models for rare or novel AEs remain a challenge, often leading to high false positive rates that require extensive human review. The "black box" nature of some predictive models also hinders trust and interpretability for clinicians, posing a barrier to widespread adoption in safety-critical contexts \cite{olaoluawa2024lb0}.

In summary, AI's integration into clinical trial operations marks a paradigm shift towards more efficient, data-driven, and patient-centric trial management. From optimizing site selection and resource allocation to enabling real-time monitoring and proactive adverse event detection, AI-driven tools significantly reduce administrative burdens, enhance data quality, and accelerate the overall drug development timeline. However, the full realization of these benefits is contingent on overcoming persistent challenges related to data quality, interoperability across diverse operational systems, and the validation of AI models in real-world, dynamic clinical environments. Achieving these sophisticated operational efficiencies, therefore, fundamentally relies on robust data integration and advanced analytical capabilities, which are explored in the subsequent section.


\label{sec:advanced_ai_for_data_integration_and_strategic_insights}

\section{Advanced AI for Data Integration and Strategic Insights}
\label{sec:advanced\_ai\_for\_data\_integration\_\_and\_\_strategic\_insights}

\subsection{Leveraging Real-World Evidence (RWE) with AI}
\label{sec:4\_1\_leveraging\_real-world\_evidence\_(rwe)\_with\_ai}

The integration of Artificial Intelligence (AI) with Real-World Evidence (RWE) is fundamentally transforming clinical trial methodologies, offering unprecedented opportunities to accelerate drug development and gain deeper insights into therapeutic effectiveness and safety. RWE, derived from diverse sources such as electronic health records (EHRs), medical claims data, patient registries, and wearable devices, provides a rich, longitudinal view of patient health and treatment outcomes in routine clinical practice. AI's capacity to process, analyze, and interpret these vast and often unstructured datasets is crucial for harnessing the full potential of RWE in clinical research.

One primary application of AI in conjunction with RWE is the enhancement of patient selection and recruitment for clinical trials. Traditional recruitment methods are often time-consuming and costly, contributing significantly to trial delays. Early work by \cite{Liu2017} demonstrated the potential of deep learning models to identify eligible patients from EHR data, thereby streamlining the recruitment process. Similarly, \cite{Wang2018} explored AI-powered patient recruitment strategies, leveraging natural language processing (NLP) and rule-based systems to automate the screening of patient records and match them against complex inclusion/exclusion criteria. Building on these foundational efforts, more recent advancements, such as the AI enrichment strategy proposed by \cite{yang2024xk7}, focus on refining patient selection for specific conditions like sepsis. This model utilizes machine learning algorithms, coupled with conformal prediction for uncertainty estimation and SHAP for interpretability, to identify homogeneous patient subgroups from retrospective RWD (e.g., from Beth Israel Deaconess Medical Center and eICU database) who are most likely to benefit from a trial's intervention, thereby reducing heterogeneity and improving trial efficiency.

Beyond patient selection, AI-driven RWE is increasingly being utilized to augment traditional Randomized Controlled Trials (RCTs) by generating synthetic control arms or providing external comparators. This approach can reduce the need for large placebo groups, making trials more ethical and efficient, particularly for rare diseases or conditions with high unmet medical needs. \cite{Saria2020} highlighted the paradigm shift towards leveraging RWD and causal inference techniques to construct robust external control arms, thereby augmenting the evidence base derived from traditional trials. This allows for more flexible trial designs and potentially faster regulatory approvals. Further advancing this concept, \cite{Kim2023} showcased the cutting-edge application of generative AI, such as Generative Adversarial Networks (GANs) and Variational Autoencoders (VAEs), to create synthetic control arms. These AI models learn the underlying data distribution of real-world patient cohorts to generate synthetic patient data that closely mimics a control group, offering a powerful tool to reduce reliance on traditional placebo groups.

The integration of AI with RWE also facilitates comprehensive insights into drug effectiveness and safety in diverse patient populations and real-world settings. AI algorithms can extract and analyze complex patterns from RWE to identify previously unobserved adverse events or drug interactions, enhancing pharmacovigilance. For instance, \cite{Chen2021} demonstrated the use of deep learning and NLP for AI-driven adverse event detection in clinical trials, leveraging unstructured safety reports and EHR data to provide early warnings. However, leveraging RWE effectively comes with significant challenges. The distributed nature and privacy concerns associated with RWD necessitate advanced solutions for data integration and analysis. \cite{Li2022} addressed this by proposing federated learning for privacy-preserving clinical trial data analysis, enabling collaborative model training across multiple institutions without sharing raw patient data, which is crucial for maximizing the utility of diverse RWE sources.

Despite the immense potential, a critical challenge in leveraging AI with RWE is ensuring the generalizability and robustness of the developed models. As highlighted by \cite{chekroud2024bvp}, clinical prediction models, even when achieving high accuracy within their development datasets (often derived from RWD or aggregated trial data), frequently perform no better than chance when applied to truly independent, out-of-sample clinical trials. This "illusory generalizability" underscores the context-dependency of AI models and the need for rigorous external validation across diverse real-world settings to prevent biased or misleading conclusions. Therefore, while AI-driven RWE promises to accelerate drug development and improve trial design, ongoing research must focus on developing more robust, generalizable, and interpretable AI models, alongside establishing clear regulatory frameworks for the acceptance of AI-generated evidence and synthetic controls. Addressing issues of data quality, bias, and privacy will be paramount for the widespread and trustworthy adoption of RWE with AI in clinical research.
\subsection{Privacy-Preserving Data Analysis: Federated Learning}
\label{sec:4\_2\_privacy-preserving\_data\_analysis:\_federated\_learning}

The advancement of artificial intelligence (AI) in clinical research promises transformative improvements in drug discovery, trial design, and patient care. However, realizing this potential is critically hampered by the pervasive challenges of data privacy and security, particularly in multi-site clinical trials where sensitive patient data is distributed across numerous institutions. Traditional approaches to AI model training often necessitate centralizing large datasets, which creates significant regulatory hurdles (e.g., HIPAA, GDPR), exacerbates data silos, and poses substantial risks to patient confidentiality. This tension between the need for vast, diverse datasets to train robust AI models and the imperative to protect patient privacy has become a central bottleneck in collaborative medical research.

The general promise of AI in healthcare, as highlighted by works like \cite{ho2020xwh} in optimizing cancer therapy, drug discovery, and patient matching, underscores the immense value of leveraging extensive clinical data. Similarly, the efficiency demonstrated by AI platforms in accelerating drug development and optimizing combination therapy design, such as the IDentif.AI system for SARS-CoV-2 \cite{blasiak2020fkz}, illustrates the power of data-driven insights. To fully capitalize on these benefits across distributed healthcare ecosystems, innovative solutions are required to enable data utilization without compromising privacy.

In response to these critical challenges, Federated Learning (FL) has emerged as a pivotal methodological innovation. FL is a decentralized AI training paradigm that facilitates collaborative model development across numerous institutions without ever requiring the direct sharing of sensitive raw patient data. In an FL setup, each participating institution trains a local AI model on its own proprietary dataset. Instead of transmitting raw data, only aggregated model updates—such as weights or gradients—are securely sent to a central server. This server then aggregates these updates to create a global model, which is subsequently distributed back to the local institutions for further refinement. This iterative process allows for the aggregation of insights from distributed datasets, effectively overcoming persistent data silos and navigating complex regulatory barriers by keeping sensitive information localized.

This paradigm rigorously upholds patient confidentiality, a paramount ethical and legal concern in medical research, while simultaneously fostering essential collaborative research endeavors across the healthcare ecosystem. The development of more robust and diverse AI models, which can benefit from the rich, multimodal data available across different sites \cite{acosta2022sxu}, is significantly empowered by FL. It enables a broader patient cohort to contribute to model training, leading to models with enhanced generalizability and reduced bias, without the need for direct data exchange.

However, despite its conceptual elegance and immense promise, the practical implementation of FL in clinical trials faces several complex challenges. These include managing model heterogeneity across diverse participating sites, where variations in patient populations, clinical practices, and data collection methods can lead to discrepancies in local model performance. Ensuring robust global model performance without centralized access to raw data for quality control or debugging remains a significant technical hurdle. Furthermore, FL introduces considerable communication overhead, as frequent exchanges of model updates are necessary, which can be particularly challenging in environments with limited bandwidth or computational resources. Beyond technical considerations, the widespread adoption of FL in clinical settings necessitates careful consideration of governance frameworks, incentive structures for participating institutions, and the standardization of data formats and model architectures across diverse sites. The need for rigorous validation and understanding of AI models, as emphasized by \cite{thirunavukarasu2023wg0} regarding the clinical aptitude of AI assistants, extends equally to models trained via FL. Such models require extensive prospective validation, ethical oversight, and a clear understanding of their limitations and potential biases to gain trust and widespread adoption in highly regulated medical environments.

In conclusion, Federated Learning stands as a transformative methodological innovation, empowering the development of more robust and diverse AI models for clinical trials while rigorously upholding patient confidentiality and fostering essential collaborative research. While significant progress has been made, continued research is essential to address the practical, operational, and ethical complexities associated with its widespread adoption, paving the way for a new era of secure and collaborative AI-driven medical discovery.
\subsection{Synthetic Data Generation for Data Scarcity and Privacy}
\label{sec:4\_3\_synthetic\_data\_generation\_for\_data\_scarcity\_\_and\_\_privacy}

The advancement and widespread adoption of artificial intelligence (AI) in healthcare, particularly within the demanding environment of clinical trials, are frequently impeded by two pervasive challenges: acute data scarcity and stringent privacy regulations. Data scarcity is a critical issue for rare diseases, specific patient subgroups, or sensitive conditions where real patient data is inherently limited. Concurrently, privacy concerns, underscored by the asymmetry between physical and virtual data in digital health \cite{zdemir20194qo}, severely restrict the sharing and utilization of sensitive patient information. In response to these significant bottlenecks, generative AI models have emerged as a transformative solution, offering the capacity to create high-fidelity synthetic patient data.

This innovative approach leverages sophisticated generative AI techniques, such as Generative Adversarial Networks (GANs), Variational Autoencoders (VAEs), and more recently, diffusion models or transformer architectures adapted for tabular data, to produce datasets that are statistically representative of real patient populations but contain no direct identifiers. This de-identification significantly enhances privacy, allowing for safer data sharing and utilization \cite{Garcia2023}. These synthetic datasets serve as invaluable resources for model training, validation, and sharing, thereby accelerating AI development and fostering research collaboration without the inherent risks associated with the direct use of sensitive real patient data.

Several studies have demonstrated the utility of AI-driven synthetic data generation for clinical trials. \cite{Wang2022} showcased its capacity to overcome limitations imposed by scarce real-world data and strict privacy protocols, facilitating the development of robust AI models even when access to original patient records is restricted. Building upon this, \cite{Garcia2023} further explored the application of GANs and VAEs to create realistic, non-identifiable datasets. While \cite{Wang2022} primarily focused on the overall utility for clinical trial data scarcity, \cite{Garcia2023} delved deeper into the architectural nuances of generative models, highlighting their potential to capture complex data distributions. Both studies, however, implicitly acknowledge the challenge of preserving intricate correlations in high-dimensional clinical data, a common hurdle for generative models. The strategic application of generative AI extends to creating rich patient profiles for personalized therapy and dynamic monitoring, particularly in areas like cancer outcomes, where data-centric approaches can mitigate limitations in real data availability \cite{kundavaram2018ii1}. Such synthetic data can be instrumental in generating evidence and optimizing clinical trial design, as highlighted by broader discussions on generative AI's role in health technology assessment \cite{fleurence2024vvo}.

Despite its compelling advantages in addressing data scarcity and privacy, the effective deployment of synthetic data necessitates rigorous validation and a critical awareness of its inherent risks. The primary challenge lies in ensuring that synthetic data fully captures the nuanced distributions, complex correlations, and rare events present in real patient data, which is critical for maintaining clinical safety and efficacy. Generic validation statements are insufficient; instead, robust frameworks are required. These include assessing statistical fidelity through metrics such as propensity score analysis and comparing marginal and joint distributions, evaluating downstream task utility (e.g., training a predictive model on synthetic data and testing its performance on real data, often referred to as the "Train on Synthetic, Test on Real" paradigm), and conducting privacy risk assessments to ensure that no sensitive information from the training set has been memorized or leaked by the generative model. The scientific validity and risk of bias are paramount considerations in this evaluation \cite{fleurence2024vvo}.

Beyond validation, critical risks associated with synthetic data must be acknowledged. Generative models, especially when trained on limited or biased real datasets, can inadvertently amplify existing biases, leading to synthetic data that perpetuates inequities \cite{hilling2025qq3}. This is particularly concerning in healthcare, where historical data often reflects systemic disparities. Furthermore, while synthetic data aims to enhance privacy, there remains a non-zero risk of model memorization, where the generative model inadvertently replicates specific sensitive records from its training data, potentially compromising privacy if not rigorously evaluated. Future research must therefore continue to focus on developing advanced metrics and validation frameworks to guarantee that synthetic datasets are not only statistically representative but also clinically meaningful, reliable for high-stakes decision-making, and free from amplified biases or privacy leakage. This requires a concerted effort to ensure diversity and equity in the underlying real datasets and to implement transparent governance for synthetic data generation \cite{hilling2025qq3}.
\subsection{Knowledge Graphs for Predictive Modeling}
\label{sec:4\_4\_knowledge\_graphs\_for\_predictive\_modeling}

Predictive modeling in clinical trials is inherently complex, grappling with challenges such as accurately forecasting drug outcomes, stratifying diverse patient populations, and accounting for the vast heterogeneity within human biology. This complexity is compounded by the need to integrate disparate data types—ranging from clinical trial results and patient demographics to intricate biological pathways and chemical properties of drugs. Traditional 'black-box' AI models, while powerful, often lack the transparency and interpretability crucial for clinical decision-making, hindering their generalizability and adoption. Knowledge Graphs (KGs), particularly when combined with advanced AI techniques like geometric deep learning (GNNs), offer a structured, interpretable framework for reasoning over these intricate biomedical relationships, moving towards more transparent and generalizable AI solutions.

The application of KGs in biomedicine has evolved significantly, initially focusing on tasks like drug-target interaction prediction or drug repurposing by representing entities and their relationships as nodes and edges. For instance, early work explored using KGs to identify potential drug-drug interactions (DDIs) by modeling relationships between drugs and other entities like targets and genes. \cite{lin2020ghb} introduced Knowledge Graph Neural Network (KGNN), an end-to-end framework designed to capture high-order structures and semantic relations within KGs for DDI prediction. KGNN learns from the neighborhoods of each entity, integrating local receptive field information with the entity's representation to model long-distance correlations, demonstrating superior performance over classic and state-of-the-art models in this specific task. This highlights the utility of GNNs in leveraging rich neighborhood information within KGs for specific predictive challenges.

A pivotal advancement that extends KG capabilities to broader clinical trial outcomes is presented by PlaNet, a geometric deep learning framework introduced by \cite{brbic2024au3}. PlaNet is designed to predict drug outcomes, including efficacy and adverse events, by leveraging a massive clinical knowledge graph. Its core innovation lies in constructing a heterogeneous KG that integrates clinical trial data (represented as drug, condition, and population triplets) with extensive background biological and chemical knowledge from nine diverse databases. This comprehensive integration allows the model to simultaneously reason over population variability, disease biology, and drug chemistry—a critical enhancement over prior models that often lacked the ability to account for patient-specific factors or generalize across diverse contexts.

The methodology of PlaNet involves an unsupervised self-supervised learning phase to generate general-purpose, low-dimensional embeddings for all entities within the KG, effectively capturing its complex topology and heterogeneity \cite{brbic2024au3}. These pretrained embeddings are then fine-tuned for specific pharmacological tasks, such as predicting survival as an efficacy endpoint or the occurrence of serious adverse events. This approach directly addresses the need for robust drug outcome prediction by providing a context-rich understanding of the factors influencing treatment response. An enhanced version, PlaNetLM, further integrates language models like PubMedBERT, allowing for multi-modal reasoning that fuses structured knowledge with textual information, leading to improved predictive performance.

PlaNet's explicit modeling of population characteristics, derived from clinical trial eligibility criteria, is particularly crucial for patient stratification and understanding population heterogeneity. By estimating the effect of changing populations on trial outcomes, PlaNet offers valuable guidance for designing clinical trials and identifying specific patient subgroups that might benefit most from a particular treatment \cite{brbic2024au3}. This moves beyond simple predictions to provide deeper, context-rich insights into complex biomedical phenomena, fostering a shift towards precision medicine.

While PlaNet demonstrates strong performance, achieving an AUROC of 0.70 for efficacy prediction (with PlaNetLM boosting this by an additional 5\\% and outperforming a PubMedBERT baseline by 15\\%) \cite{brbic2024au3}, it is important to contextualize these metrics. While promising, the absolute value of 0.70 AUROC, without direct comparison to a wide array of established baselines or competing methods on identical tasks, requires careful interpretation regarding its clinical utility. Nevertheless, its robust generalization capabilities—predicting outcomes for novel drugs and drug combinations not seen during training by leveraging KG similarities—represent a significant step towards more adaptable AI solutions.

The claim of enhanced interpretability with KGs, while conceptually appealing due to their structured nature, warrants a more nuanced discussion, especially when combined with deep learning models. KGs \textit{facilitate} interpretability by providing a traceable path of relationships, allowing researchers to understand \textit{what} entities and relations are involved in a prediction. However, the interpretability of complex GNNs themselves remains a significant research area, as highlighted by \cite{wu2024jyd} in their broader review of AI in drug discovery, noting the "black box" nature of many deep learning approaches. While PlaNet's explicit knowledge representation helps, fully explaining \textit{why} a GNN makes a particular prediction is still challenging. In contrast, studies like \cite{bresso2021fri} explicitly focus on Explainable AI (XAI) for investigating Adverse Drug Reaction (ADR) mechanisms using KG mining, often employing simpler, inherently interpretable models like Decision Trees and Classification Rules. These models, while potentially less powerful in complex prediction tasks than GNNs, offer human-readable explanations that can directly inform the molecular mechanisms behind ADRs, showcasing a different trade-off between predictive power and direct interpretability.

Despite the advancements, the performance of KG-driven models is inherently tied to the quality, completeness, and scale of the underlying KGs and the availability of labeled training data \cite{brbic2024au3}. Future research will need to focus on enriching these KGs with even more granular real-world data, developing more sophisticated multi-modal reasoning techniques, and addressing challenges related to data standardization and interoperability across diverse clinical and biological datasets. Furthermore, developing robust XAI methods specifically tailored for GNNs on biomedical KGs is critical to fully realize their potential for transparent and trustworthy clinical application. The integration of KGs with geometric deep learning represents a transformative trajectory in AI for clinical trials, promising to deliver more interpretable, generalizable, and clinically actionable insights for drug development and personalized medicine, provided these challenges are systematically addressed.


\label{sec:advanced_ai_paradigms_for_dynamic_trial_management_and_upstream_impact}

\section{Advanced AI Paradigms for Dynamic Trial Management and Upstream Impact}
\label{sec:advanced\_ai\_paradigms\_for\_dynamic\_trial\_management\_\_and\_\_upstream\_impact}

\subsection{Large Language Models (LLMs) for Documentation and Synthesis}
\label{sec:5\_1\_large\_language\_models\_(llms)\_for\_documentation\_\_and\_\_synthesis}

Large Language Models (LLMs) are rapidly transforming the landscape of clinical trials by automating and enhancing complex, text-heavy tasks, promising significant efficiency gains, improved consistency, and reduced administrative burden. This section delves into their burgeoning applications in generating trial protocols, drafting informed consent forms, extracting structured information from unstructured clinical notes, and synthesizing vast amounts of scientific literature for evidence generation, while critically addressing their inherent challenges.

One of the most impactful applications of LLMs is in the generation and optimization of clinical trial protocols. Traditionally, authoring detailed protocols is a time-consuming and error-prone process. Recent research demonstrates LLMs' capability to streamline this. For instance, \cite{maleki2024hwz} explored the use of GPT-4 for clinical trial protocol authoring. Their methodology involved detailed analysis and preparation of drug and study-level metadata, followed by prompt engineering to generate specific protocol sections. The study reported significant improvements in efficiency, accuracy, and customization, highlighting the potential for LLMs to reduce the manual effort involved. Similarly, \cite{liddicoat2025pdu} proposed a policy framework for developing application-specific language models (ASLMs) for clinical trial design, envisioning enhanced trial efficiency, inclusivity, and safety through automated protocol development. These studies move beyond theoretical potential, offering concrete examples of LLMs assisting in the foundational documentation of trials.

LLMs are also proving instrumental in improving patient communication, particularly concerning informed consent. Patient comprehension of complex medical jargon in informed consent forms (ICFs) remains a critical challenge. \cite{waters2025scl} investigated the potential of GPT-4 to generate patient-friendly summaries from cancer clinical trial ICFs. They evaluated two AI-driven approaches—direct and sequential summarization—finding that sequential summarization yielded higher accuracy and completeness, and significantly improved readability. The study also demonstrated LLMs' ability to create multiple-choice question-answer pairs (MCQAs) to gauge patient understanding, with high concordance to human-annotated responses. While promising, this work also underscored concerns regarding AI hallucinations, accuracy, and ethical considerations, emphasizing the need for refinement and regulatory oversight.

Beyond document generation, LLMs are powerful tools for information extraction and evidence synthesis. The extraction of structured information from unstructured clinical notes, a critical component of real-world evidence (RWE) generation, can be significantly expedited by LLMs \cite{fleurence2024vvo}. This capability allows for more efficient analysis of large collections of RWD, enhancing the speed and quality of RWE. For evidence synthesis, a task exemplified by the meticulous systematic reviews required for clinical practice guidelines like the ASCO guideline for adjuvant endocrine therapy in breast cancer \cite{burstein2019qgx}, LLMs offer substantial assistance. They can automate initial literature screening, summarize key findings, and extract relevant data points, thereby expediting the creation of evidence-based recommendations \cite{fleurence2024vvo}.

However, the application of LLMs in evidence appraisal is not without its complexities. \cite{woelfle2024q61} benchmarked human-AI collaboration for common evidence appraisal tools (PRISMA, AMSTAR, PRECIS-2) using various LLMs (Claude-3-Opus, GPT-4, GPT-3.5, Mixtral-8x22B). Their findings revealed that individual LLMs alone performed worse than human raters in assessing scientific reporting and methodological rigor. While human-AI collaboration improved accuracies (e.g., 89-96\\% for PRISMA), it also highlighted the limitations of LLMs for complex tasks like PRECIS-2, where high deferral rates indicated persistent challenges. This suggests that while LLMs can reduce workload for certain aspects of evidence appraisal, they are not yet capable of fully autonomous, high-stakes critical evaluation. Furthermore, LLMs have been explored for summarizing safety-related tables in Clinical Study Reports (CSRs), where prompt engineering with GPT models showed potential but also highlighted the need for improved ingestion of tables, context, and fine-tuning to ensure factual accuracy and lean writing \cite{landman2024w8r}.

Despite these advancements, the deployment of LLMs in high-stakes clinical contexts necessitates a critical and cautious approach due to several inherent challenges. The potential for 'hallucination,' where models generate plausible but factually incorrect information, is a significant concern, as highlighted by \cite{waters2025scl} and further underscored by the need for robust Natural Language Inference (NLI) models to address factual inconsistency and vulnerability to adversarial inputs in biomedical contexts \cite{jullien2024flu}. Such inaccuracies could have severe implications in clinical documentation and patient safety. Moreover, inherent biases present in the training data can be perpetuated or amplified by LLMs, potentially leading to inequitable or inaccurate recommendations, a risk acknowledged by \cite{fleurence2024vvo}.

Therefore, the paramount need for stringent human oversight and rigorous validation processes cannot be overstated. Every piece of documentation or synthesis generated by an LLM must undergo thorough review by clinical experts to ensure accuracy, safety, and ethical compliance \cite{fleurence2024vvo, landman2024w8r}. Mitigation strategies for hallucination, such as retrieval-augmented generation (RAG) which grounds LLM outputs in verified external knowledge, and fine-tuning on domain-specific, curated clinical corpora, are crucial. The development of robust validation frameworks, transparent reporting mechanisms, and continued research into human-AI collaboration models will be essential for building trust and ensuring the responsible integration of LLMs into clinical trial operations, ultimately augmenting human expertise rather than replacing it.
\subsection{Reinforcement Learning for Adaptive Trial Designs}
\label{sec:5\_2\_reinforcement\_learning\_for\_adaptive\_trial\_designs}

The development of highly adaptive clinical trial designs represents a significant paradigm shift, moving away from static protocols towards dynamic, data-driven optimization. Reinforcement Learning (RL) has emerged as a particularly potent artificial intelligence (AI) methodology for this purpose, enabling real-time adjustments to trial parameters based on accumulating interim data \cite{zhang2022reinforcement, chen2022reinforcement}. This innovative application of RL holds profound potential to optimize trial efficiency, reduce patient exposure to ineffective treatments, and accelerate the identification of effective therapies, thereby leading to more ethical and successful trials.

At its core, RL for adaptive trial design frames the clinical trial process as a sequential decision-making problem, where an "agent" (the trial design algorithm) learns optimal policies by interacting with the "environment" (the evolving trial data and patient responses) \cite{zhang2022reinforcement}. This allows for dynamic adjustments to critical trial parameters such as sample size, treatment allocation, and stopping rules. For instance, \cite{zhang2022reinforcement} (and similarly \cite{chen2022reinforcement}) proposes an RL framework that dynamically adjusts treatment allocation ratios to favor more promising therapies as efficacy and safety data accumulate. This approach minimizes the number of patients exposed to less effective or harmful treatments, directly addressing ethical concerns while simultaneously improving the statistical power and efficiency of the trial. The RL agent learns through a reward function that balances objectives like maximizing the number of patients receiving the optimal treatment and minimizing trial duration.

Building upon foundational RL applications, more sophisticated techniques like multi-agent reinforcement learning (MARL) are being explored to handle the inherent complexities of clinical trials, where multiple interacting objectives or decision points exist \cite{li2023multi}. \cite{li2023multi} demonstrates how MARL can optimize adaptive clinical trial designs by allowing different agents to manage distinct aspects of the trial, such as one agent optimizing treatment allocation and another managing sample size re-estimation, leading to more robust and comprehensive adaptive strategies. This distributed decision-making capability of MARL is particularly beneficial for trials with multiple treatment arms or complex patient subgroups, where a single agent might struggle to manage all interdependencies.

A critical prerequisite for the successful deployment of these highly dynamic RL-driven designs is the availability of robust simulation environments for extensive validation \cite{kaddour2021ai}. Given the computational complexity and the high stakes involved in clinical trials, RL policies cannot be directly deployed without rigorous testing. AI-driven simulations, as highlighted by \cite{kaddour2021ai}, are instrumental in accelerating drug discovery and early-stage trial design by modeling complex biological systems and patient responses. These simulations provide the necessary sandbox for training and evaluating RL agents under various hypothetical scenarios, ensuring that the adaptive policies are safe, effective, and statistically sound before real-world implementation. For example, the IDentif.AI platform, described by \cite{blasiak2020fkz}, showcases how AI and digital drug development can rapidly optimize combination therapy designs against pathogens like SARS-CoV-2. While primarily focused on optimizing the \textit{treatment itself} rather than trial parameters, this work underscores the power of AI-driven optimization and simulation in a clinical context, which can be directly integrated into RL frameworks for adaptive trial design to inform optimal treatment arm configurations.

Despite the immense potential, the adoption of RL for adaptive trial designs faces significant challenges. The inherent computational complexity of training and validating RL agents, especially for multi-agent systems, demands substantial computational resources and sophisticated algorithmic development. Furthermore, the "black-box" nature of some deep RL models can hinder interpretability, posing a hurdle for regulatory acceptance and clinician trust. The critical need for robust simulation environments cannot be overstated; the fidelity of these simulations directly impacts the reliability of the learned RL policies. Future research must focus on developing more interpretable RL models, enhancing the efficiency of simulation-based validation, and establishing clear regulatory pathways for AI-driven adaptive trial designs to fully realize their transformative potential in delivering more ethical, efficient, and successful clinical trials.
\subsection{AI in Early Drug Discovery and Pre-clinical Development}
\label{sec:5\_3\_ai\_in\_early\_drug\_discovery\_\_and\_\_pre-clinical\_development}

The traditional drug discovery pipeline is notoriously time-consuming, expensive, and fraught with high failure rates, necessitating innovative approaches to accelerate the identification and optimization of promising therapeutic candidates. Artificial Intelligence (AI), particularly through advanced machine learning (ML) and deep learning (DL) algorithms, has emerged as a transformative force in the upstream stages of drug development, significantly impacting target identification, lead optimization, virtual screening, and the prediction of drug-target binding affinity. These applications directly contribute to a more efficient clinical trial pipeline by providing better-characterized compounds.

Early reviews, such as that by \cite{selvaraj2021n52}, highlighted the foundational role of AI and ML methods in computer-aided drug design, emphasizing their integration into processes like high-throughput virtual screening and the identification of novel lead compounds. This work underscored the potential for AI to dramatically improve the success rate of hit identification by leveraging available data resources. Building upon this, the advent of sophisticated deep learning models has further revolutionized structural biology, a critical component of target identification. For instance, \cite{nussinov2022vua} discussed the profound impact of AlphaFold in protein structure prediction, which provides highly accurate 3D models crucial for structure-based drug design and selecting optimal drug targets. However, \cite{nussinov2022vua} also critically noted that AlphaFold, while powerful, generates single ranked structures rather than conformational ensembles, thus not fully capturing dynamic biological mechanisms like allostery or the behavior of intrinsically disordered proteins, which are vital for understanding drug-target interactions.

More broadly, \cite{dave202400p} provided an updated perspective on how AI, encompassing ML and DL, is revolutionizing the pharmaceutical sector by simplifying and accelerating drug discovery processes. This includes AI's utility in identifying therapeutic targets, predicting the 3D structure of target proteins, forecasting drug-protein interactions, and enabling \textit{de novo} drug design. The authors emphasized AI's capacity to manage and analyze the vast volumes of data inherent in drug development, thereby making the process more manageable and less time-consuming. However, \cite{dave202400p} also pointed out ethical considerations regarding patient data privacy, the risk of bias, and the need for specialized skills and financial investment as limitations.

A comprehensive review by \cite{wu2024jyd} further detailed the specific technical contributions of various AI algorithms across drug screening and design. This work elucidated how ML algorithms like k-Nearest Neighbors (kNN), Random Forest (RF), Support Vector Machines (SVM), and Artificial Neural Networks (ANNs) are employed for tasks such as predicting small compound stability, neurotoxicity, and drug repositioning. Furthermore, \cite{wu2024jyd} highlighted the application of deep learning architectures, including Convolutional Neural Networks (CNNs) for peptide-protein interaction prediction, Generative Adversarial Networks (GANs) for generating novel molecular structures, and Recurrent Neural Networks (RNNs) for improving drug interaction extraction. The authors demonstrated how these methods significantly enhance the efficiency of identifying potential drug candidates and optimizing their properties. Critically, \cite{wu2024jyd} also addressed the limitations of these AI approaches, noting that traditional ML often struggles with heterogeneous information, while DL models demand high-quality, large datasets and suffer from "black box" interpretability issues, particularly challenging in the complex biological and chemical domains.

In conclusion, AI has undeniably transformed early drug discovery and pre-clinical development by offering sophisticated tools for target identification, virtual screening, lead optimization, and predicting critical molecular interactions. The field has progressed from predictive models to advanced generative AI capable of designing novel compounds. However, several challenges persist, including the need for higher quality and more extensive datasets, improving the interpretability of complex DL models, and developing AI systems that can accurately capture the dynamic and ensemble nature of biological molecules, as highlighted by the limitations of current protein structure prediction tools. Addressing these unresolved issues will be crucial for fully realizing AI's potential to deliver more promising and well-characterized drug candidates for clinical evaluation.


\label{sec:ensuring_trustworthy_ai:_fairness,_explainability,_and_human_factors}

\section{Ensuring Trustworthy AI: Fairness, Explainability, and Human Factors}
\label{sec:ensuring\_trustworthy\_ai:\_fairness,\_explainability,\_\_and\_\_human\_factors}

\subsection{Addressing AI Fairness and Bias in Clinical Predictions}
\label{sec:6\_1\_addressing\_ai\_fairness\_\_and\_\_bias\_in\_clinical\_predictions}

Ensuring fairness and mitigating bias in AI models used for clinical predictions and decision-making within trials represents a critical ethical and technical challenge, fundamental to the integrity of clinical research and the equitable delivery of healthcare \cite{pasricha2022cld}. AI systems must perform reliably and justly across diverse patient populations, necessitating a deep understanding of the multifaceted sources of discrimination and a shift from mere symptom mitigation to diagnostic and data-centric interventions. Biases can originate from various stages, including historical societal inequities reflected in data (historical bias), unrepresentative training datasets (representation bias), flawed data collection or labeling processes (measurement bias), and inappropriate evaluation metrics (evaluation bias) \cite{hilling2025qq3}.

Addressing these biases requires robust technical frameworks. Early work by \cite{kelly2019gw7} introduced a pivotal diagnostic framework that decomposes cost-based discrimination metrics (e.g., differences in false positive rates, false negative rates, or mean squared error across protected groups) into bias, variance, and noise components. This innovative approach allows researchers to pinpoint whether unfairness stems from model misspecification (bias), insufficient or unrepresentative data (variance), or irreducible inherent variability in the data itself (noise). By shifting the focus from post-hoc mitigation to root cause analysis, \cite{kelly2019gw7} proposed that the "cost of fairness" need not be a sacrifice of accuracy, but rather an investment in data quality and collection. The paper further provided practical tools, such as "discrimination learning curves" to quantify the value of additional data, and clustering techniques to identify subpopulations requiring more predictive variables, thereby guiding data-centric interventions. While this work significantly advanced the technical understanding of algorithmic fairness, it primarily assumed observed differences were discriminatory without delving into causal inference or explicitly correcting for historical biases embedded in labels \cite{kelly2019gw7}.

Building upon such diagnostic insights, a broader taxonomy of technical interventions has emerged to address bias throughout the AI lifecycle. These include: \textit{pre-processing} techniques that modify the training data before model development (e.g., re-weighting samples, re-sampling to balance protected groups, or debiasing features) to tackle representation and historical biases; \textit{in-processing} methods that incorporate fairness constraints directly into the model's objective function during training; and \textit{post-processing} techniques that adjust model outputs or decision thresholds after prediction to achieve desired fairness criteria. The emphasis on data-centric AI, where improvements to data quality and diversity are prioritized, is crucial. For instance, \cite{kundavaram2018ii1} demonstrated a data-centric approach using predictive analytics and generative AI to optimize cervical and breast cancer outcomes, specifically by detecting patterns in underprivileged communities to reduce health inequities. This highlights how targeted data collection and analysis can directly lead to more equitable predictive performance.

Beyond algorithmic and data-centric interventions, the broader methodological and systemic aspects of clinical trials are critical for ensuring fairness. Reporting guidelines, such as CONSORT-AI \cite{chan2020egf} and SPIRIT-AI \cite{rivera2020sg1}, play a crucial role by mandating transparent documentation of population characteristics, data sources, and model development, which are essential for identifying and scrutinizing potential biases in study design and outcome reporting. This transparency is vital for conducting the kind of detailed variance analysis proposed by \cite{kelly2019gw7}. The challenge of generalizability, as highlighted by \cite{chekroud2024bvp} in the context of schizophrenia treatment models, further underscores that models performing well in one dataset may fail in truly independent clinical contexts, often manifesting as significant fairness concerns across diverse patient subgroups. This reinforces the imperative for diverse training data and rigorous external validation. Furthermore, the ethical design of AI Randomized Controlled Trials (RCTs) must explicitly consider fairness, as discussed by \cite{grote2021iet}, ensuring that trial protocols do not inadvertently perpetuate or exacerbate existing health disparities. Proactive measures such as fairness audits, transparent AI model development processes, and early registration of clinical AI models are advocated to drive responsible AI adoption and ensure equitable outcomes \cite{hilling2025qq3}.

In conclusion, addressing AI fairness and bias in clinical predictions demands a multi-pronged approach that integrates diagnostic algorithmic techniques with a deep understanding of data provenance and rigorous methodological oversight. Moving forward, research must bridge the gap between developing intrinsically fair and accurate AI models and ensuring their safe, effective, and equitable integration into clinical practice. This involves not only refining data-centric interventions to reduce algorithmic bias and variance but also fostering inclusive global collaborations and developing proactive ethical and regulatory frameworks to guarantee trustworthy and equitable outcomes across all patient populations in clinical research \cite{hilling2025qq3, pasricha2022cld}. The ethical imperative demands a holistic approach that considers the entire AI lifecycle, from data acquisition and model development to deployment and post-market surveillance.
\subsection{Explainable AI (XAI) for Interpretability and Trust}
\label{sec:6\_2\_explainable\_ai\_(xai)\_for\_interpretability\_\_and\_\_trust}

The integration of complex Artificial Intelligence (AI) models into clinical trials, while promising, inherently introduces the "black box" problem, where model decisions are opaque and challenging for human understanding. Explainable AI (XAI) directly addresses this critical issue by providing methodologies to interpret AI predictions, thereby enhancing transparency, fostering trust among clinicians, patients, and regulatory bodies, and facilitating regulatory approval \cite{roy20223mf}. This subsection reviews the development and application of XAI techniques, emphasizing their crucial role in enabling clinicians to understand and validate AI-driven decisions, ensuring ethical considerations are met, and bridging the gap between advanced AI capabilities and their practical, responsible integration into clinical practice.

XAI techniques can broadly be categorized into several approaches relevant to clinical trials. \textbf{Feature attribution methods} (e.g., SHAP, LIME) identify the contribution of individual input features to a model's prediction, providing local explanations for specific instances. \textbf{Model-specific explanation methods} are tailored to certain architectures, such as Grad-CAM for convolutional neural networks, which highlights relevant regions in image data. \textbf{Surrogate models} involve training a simpler, interpretable model to approximate the behavior of a complex black-box model. Finally, \textbf{example-based explanations} provide insights by identifying similar training data points that influenced a prediction. These methods are vital for critical tasks like patient selection, safety monitoring, and outcome prediction, where understanding the 'why' behind an AI's recommendation is paramount.

The necessity for XAI intensifies as AI models in biomedicine become more sophisticated, integrating diverse data types. For instance, the development of \textit{multimodal biomedical AI} often involves complex deep learning architectures that combine imaging, genomic, and clinical text data \cite{acosta2022sxu}. While these models offer enhanced predictive power, their inherent complexity makes their decision-making processes particularly opaque. XAI techniques, such as multimodal feature attribution, can elucidate how different data modalities contribute to a given prediction. For example, by applying SHAP values to a multimodal model, researchers can quantify the relative importance of genetic markers versus imaging features in predicting disease progression, providing biologically plausible and clinically relevant insights for validation.

A prime example of AI's application in high-stakes clinical decision-making is the development of AI-derived biomarkers. \textcite{armstrong2023dwd} successfully developed and validated an AI-derived digital pathology-based biomarker to predict the benefit of long-term androgen deprivation therapy in men with localized high-risk prostate cancer. For such a biomarker to achieve widespread clinical adoption and regulatory approval, clinicians and patients must understand \textit{why} the AI makes a particular recommendation. Here, XAI techniques like Grad-CAM could highlight specific pathological regions or cellular patterns within digital pathology images that drive the biomarker's prediction, transforming a black-box output into actionable, interpretable insights. In a practical clinical trial setting, \textcite{angus2020epl} demonstrated an AI-based early warning system for hypotension during surgery. This system provided a risk score along with a "read-out of key variables" used by the algorithm, and anesthesiologists received training on interpreting these features and suggested actions. This exemplifies a direct application of XAI, where the AI's internal logic, even if simplified, is communicated to the user to foster understanding and guide intervention.

The critical importance of XAI is further underscored by challenges related to model generalizability and reliability. \textcite{chekroud2024bvp} highlight the concerning issue of "illusory generalizability" in clinical prediction models, where high accuracy on development datasets fails to translate to independent clinical trials. This lack of robustness severely undermines confidence and poses a significant barrier to practical application and regulatory acceptance. XAI plays a crucial diagnostic role by providing insights into the features or patterns a model relies upon. By revealing if a model is leveraging spurious correlations or context-specific features that do not generalize, XAI can guide the development of more robust, generalizable, and ultimately trustworthy AI systems, directly addressing the limitations identified by Chekroud et al.

Beyond interpretability, XAI is fundamental for addressing ethical considerations and regulatory compliance. The imperative for diversity and equity in healthcare AI, as highlighted by \textcite{hilling2025qq3}, necessitates transparent AI development and fairness audits. XAI methods can reveal biases embedded in models, for instance, by showing if predictions for certain demographic groups rely on different or less robust features, enabling targeted interventions to ensure equitable outcomes. Furthermore, regulatory bodies and reporting guidelines increasingly mandate transparency. The CONSORT-AI guidelines, for example, call for "clear descriptions of the AI intervention, skills required, study setting, inputs and outputs of the AI intervention, analysis of errors, and the human and AI interactions" \cite{parums2021k6f}. Similarly, meta-research studies reveal "poor standards of reporting" in AI diagnostic accuracy studies, underscoring the need for AI-specific quality assessment tools \cite{jayakumar2022sav}. XAI directly supports these requirements by making the AI's decision-making process auditable and understandable, which is crucial for demonstrating safety and efficacy, especially for FDA-approved AI/ML devices often cleared via the 510(k) pathway that relies on substantial equivalence rather than new clinical trials \cite{joshi2024ajq}.

Despite its advancements, XAI faces limitations. Explanations can sometimes be unstable (small input changes lead to large explanation changes), unfaithful to the true model logic, or overly simplistic, potentially misleading clinicians \cite{roy20223mf}. The challenge lies in developing XAI methods that are not only technically sound but also clinically meaningful, actionable, and scalable across diverse AI architectures and data types. Future research must focus on robust validation of XAI explanations in clinical contexts, ensuring they accurately reflect model behavior and genuinely enhance human understanding and decision-making, rather than merely providing a post-hoc rationalization. This continuous innovation in XAI techniques is essential for fostering trust among all stakeholders and accelerating the responsible integration of AI into clinical practice.
\subsection{Human Factors and Usability in AI-Driven Systems}
\label{sec:6\_3\_human\_factors\_\_and\_\_usability\_in\_ai-driven\_systems}

The successful integration of artificial intelligence (AI)-driven decision support systems into clinical trials and practice hinges critically on robust human factors engineering and usability. Without careful consideration of how humans interact with AI, these systems risk 'use error' and potential patient harm, regardless of their underlying algorithmic accuracy. The challenge lies in ensuring that AI outputs are not only interpretable and actionable but also seamlessly integrated into existing clinical workflows, necessitating iterative, science-based approaches to design and evaluation.

Early efforts to standardize the reporting of AI interventions in clinical trials recognized the paramount importance of human-AI interaction. The SPIRIT-AI extension provides guidelines for clinical trial protocols, recommending that investigators clearly describe the AI intervention, including necessary instructions and skills for use, its integration setting, data handling, and crucially, the nature of human-AI interaction and planned analysis of error cases \cite{rivera2020sg1}. Complementing this, the CONSORT-AI extension offers similar reporting guidelines for clinical trial reports, ensuring that these vital human factors considerations are transparently documented in published results \cite{chan2020egf}. These guidelines underscore a foundational shift towards mandating explicit consideration of the human element in AI clinical research.

Building upon these reporting frameworks, regulatory bodies have begun to translate broad requirements into practical expectations for manufacturers and researchers. The DECIDE-AI reporting guideline, for instance, focuses on the early-stage clinical evaluation of AI-based decision support systems, emphasizing the assessment of actual clinical performance, safety, and the human factors surrounding its use \cite{vasey2022oig}. This guideline advocates for the usability engineering process as an iterative, science-based methodology. This approach systematically applies knowledge from diverse fields to design products that are safe and effective for users, actively identifying, assessing, and mitigating potential patient and user safety risks throughout the device lifecycle \cite{vasey2022oig}. Specifically within the Great Britain medical device market, this guidance provides clarified regulatory interpretation for applying established usability engineering principles, thereby translating legal requirements into concrete steps for designing AI systems that minimize 'use error' and maximize clinical utility \cite{vasey2022oig}.

Despite the increasing emphasis on human factors in guidelines and regulations, empirical evidence reveals significant challenges in effective human-AI collaboration. Research by \textcite{rosenthal2025j23} empirically quantified cognitive biases in human-AI interaction among professional radiologists. Their large-scale randomized controlled experiment demonstrated that, even when AI performance was comparable to or surpassed human experts, AI assistance did not, on average, improve human diagnostic quality. This counterintuitive finding was attributed to human cognitive biases such as automation neglect (under-weighting AI predictions) and correlation neglect (treating human and AI information as statistically independent) \cite{rosenthal2025j23}. The study's critical insight was that optimal collaboration often involved delegating cases entirely to either humans or AI, but rarely to AI-assisted humans, due to these identified biases \cite{rosenthal2025j23}. This highlights that simply providing AI predictions is insufficient; the \textit{design} of the interaction and the \textit{context} of delegation are paramount to prevent 'use error' and ensure patient safety.

In conclusion, while reporting guidelines like SPIRIT-AI and CONSORT-AI, and regulatory frameworks such as DECIDE-AI, lay the groundwork for incorporating human factors into AI clinical trials, the empirical realities of human-AI interaction present complex challenges. The findings from studies like \textcite{rosenthal2025j23} underscore the critical need for continuous, iterative usability engineering throughout the development and deployment of AI-driven systems. Future research must bridge the gap between algorithmic accuracy and effective human integration by designing AI systems that are not only robust but also "bias-aware" in their interaction design, complemented by targeted training protocols for clinicians. This comprehensive approach is essential to maximize AI's clinical utility while minimizing risks within the intricate landscape of healthcare.


\label{sec:evaluation,_implementation,_and_regulatory_landscape}

\section{Evaluation, Implementation, and Regulatory Landscape}
\label{sec:evaluation,\_implementation,\_\_and\_\_regulatory\_l\_and\_scape}

\subsection{Empirical Assessment of AI Trial Quality and Impact}
\label{sec:7\_1\_empirical\_assessment\_of\_ai\_trial\_quality\_\_and\_\_impact}

The rapid proliferation of artificial intelligence (AI) interventions in healthcare necessitates a rigorous empirical assessment of their methodological quality, clinical impact, and reporting completeness within clinical trials. This subsection reviews systematic and meta-research studies that scrutinize the current landscape, identifying pervasive weaknesses, biases, and the critical gap between promising observational performance and demonstrated clinical benefit. It underscores the urgent need for comprehensive evaluation extending beyond purely technical metrics, emphasizing implementation outcomes and the development of AI-specific quality assessment tools to generate robust and reliable evidence.

Early empirical analyses of registered AI clinical trials reveal a rapidly expanding but methodologically nascent field. Cross-sectional studies by \cite{dong2020g8g} and \cite{liu2021lc8} characterized the landscape of AI trials in cancer diagnosis and emergency/intensive care units, respectively. \cite{dong2020g8g} found that most AI trials in cancer diagnosis were observational (72.1\\%) and lacked published results, with many interventional trials exhibiting methodological weaknesses such as a lack of masking. Similarly, \cite{liu2021lc8} observed a significant increase in AI trial registrations in ED and ICU settings, but critically noted that only 6.85\\% of completed trials had publicly available results, severely impeding knowledge dissemination. These findings were reinforced by \cite{wang2022yim}, whose broader cross-sectional analysis of 1725 AI-related trials across healthcare highlighted persistent design drawbacks and poor-quality result reporting. Further, \cite{sande20217w9}'s systematic review of AI in the ICU revealed that the vast majority of models remained in testing or prototyping, with high risks of bias in retrospective studies and a complete absence of studies reporting on AI models integrated into routine clinical practice.

Moving beyond the landscape of registered trials, systematic reviews of randomized controlled trials (RCTs) have provided crucial insights into the actual clinical impact and methodological rigor of AI interventions. \cite{zhou2021vqt} conducted a comprehensive systematic review of 65 RCTs evaluating AI prediction tools, revealing a significant disparity: while 61.5\\% of trials reported a positive clinical benefit, a substantial 38.5\\% showed no benefit over standard care. More critically, only 26.2\\% of these RCTs had an overall low risk of bias, with frequent issues in blinding and reporting quality (72.3\\% did not reference CONSORT). This study empirically demonstrated the pervasive methodological weaknesses and the significant gap between AI's promising \textit{in silico} performance and its demonstrated clinical benefit in rigorous settings. \cite{lam2022z48} corroborated these findings in another systematic review of 39 AI RCTs, noting limited and heterogeneous evidence, small sample sizes, and single-center designs that restrict generalizability. Similarly, \cite{siontis2021l0w} highlighted significant variation in the development and validation pathways of AI tools prior to their evaluation in RCTs, alongside heterogeneity in trial design and reporting. These empirical findings underscore the urgent need for more robust trial designs and transparent reporting to ensure the generation of high-quality evidence.

The identified methodological weaknesses and reporting deficiencies in primary AI trials have naturally led to questions about the adequacy of quality assessment tools used in evidence synthesis. \cite{jayakumar2022sav} conducted a meta-research study examining quality assessment standards in systematic reviews of AI diagnostic accuracy studies. Their analysis of 50 systematic reviews (encompassing 1110 primary studies) empirically demonstrated inconsistent and incomplete application of quality assessment tools like QUADAS-2. They found that a high or unclear risk of bias was prevalent in primary AI studies, particularly in patient selection (57.5\\%), underscoring the limitations of generic tools in capturing AI-specific biases. This study highlighted the critical need for an "AI-specific extension for quality assessment tools" to facilitate safe clinical translation. Responding to this need, \cite{kwong20242pu} applied a novel, AI-specific quality assessment tool, APPRAISE-AI, in a systematic review of NMIBC prediction studies. Their application revealed granular methodological pitfalls across dataset generation, model evaluation, and reproducibility, demonstrating that the reported superiority of AI models in lower-quality studies might be inflated. This work validates the necessity of specialized tools for a more nuanced and accurate appraisal of AI research quality.

Crucially, even when AI models demonstrate technical proficiency and clinical effectiveness, their translation into routine clinical practice remains a challenge, pointing to a critical gap in evaluation beyond purely technical and clinical metrics. \cite{sande20248hm} empirically analyzed 64 RCTs of AI-based Clinical Decision Support Systems (AICDSS), revealing a widespread neglect of \textit{implementation outcomes}. Their study found that 38\\% of RCTs reported no implementation outcomes, and critical factors such as adoption, appropriateness, implementation costs, sustainability, and penetration were reported in less than 10\\% of trials. This highlights that existing reporting guidelines, such as CONSORT-AI and SPIRIT-AI (\cite{ibrahim2021rcn}, \cite{chan2020egf}, \cite{rivera2020sg1}), while improving technical reporting, "fail to offer adequate measures for evaluating the success of implementing an AI" \cite{sande20248hm}. This empirical evidence strongly advocates for a multi-faceted evaluation approach that systematically integrates implementation science into AI clinical trials, including the use of hybrid designs and established implementation frameworks. This perspective is further supported by \cite{marwaha2022gj3}, who, building on the empirical findings of the performance-to-impact gap, called for an "implementation science of AI" to systematically identify optimal interventions and leverage real-world evidence for comprehensive evaluation.

In conclusion, the empirical assessment of AI trial quality and impact reveals a field grappling with significant methodological weaknesses, reporting deficiencies, and a persistent gap between technical promise and demonstrated clinical benefit. The literature consistently highlights pervasive biases, the inadequacy of generic quality assessment tools for AI-specific characteristics, and a critical oversight in evaluating implementation outcomes essential for real-world adoption. While prescriptive guidelines like DECIDE-AI (\cite{vasey2022yhn}) and frameworks like RADAR (\cite{boverhof2024izx}) are emerging to address these issues, the empirical evidence underscores that their effectiveness hinges on widespread adoption and a fundamental shift towards more holistic, AI-specific, and implementation-aware evaluation paradigms. Future research must prioritize rigorous study designs, transparent reporting of all relevant outcomes (including implementation factors), and the continuous development and application of specialized tools to ensure the generation of robust and reliable evidence, thereby advancing the responsible and effective integration of AI into clinical practice.
\subsection{Reporting Guidelines for AI Interventions (CONSORT-AI, SPIRIT-AI)}
\label{sec:7\_2\_reporting\_guidelines\_for\_ai\_interventions\_(consort-ai,\_spirit-ai)}

The rapid proliferation of artificial intelligence (AI) interventions in healthcare necessitates robust and transparent reporting standards to ensure the rigor, reproducibility, and critical appraisal of clinical trials. Without such guidelines, the unique complexities of AI models, their development, evaluation, and interaction with human users can lead to opaque research, hindering trust and safe clinical translation. To address this, specialized reporting guidelines such as CONSORT-AI and SPIRIT-AI have been developed to standardize the documentation of AI-driven medical research \cite{ibrahim2021rcn}.

The Consolidated Standards of Reporting Trials-Artificial Intelligence (CONSORT-AI) extension and its companion guideline for trial protocols, Standard Protocol Items: Recommendations for Interventional Trials-Artificial Intelligence (SPIRIT-AI), represent a significant step towards enhancing transparency in AI clinical trials \cite{chan2020egf, rivera2020sg1}. Developed through a rigorous multi-stakeholder consensus process involving literature reviews, expert consultations, Delphi surveys, and consensus meetings, these guidelines aim to provide a minimum set of reporting items essential for AI interventions \cite{chan2020egf, rivera2020sg1}. CONSORT-AI, for instance, adds 14 new items to the core CONSORT 2010 statement, recommending detailed descriptions of the AI intervention, including instructions for use, required skills, the clinical setting, handling of inputs and outputs, the nature of human-AI interaction, and an analysis of error cases \cite{chan2020egf}. Similarly, SPIRIT-AI extends the SPIRIT 2013 statement with 15 new items, ensuring that the design and methodology of planned AI trials are comprehensively documented from the outset \cite{rivera2020sg1}. These guidelines are crucial for assisting editors, peer reviewers, and the broader scientific community in understanding, interpreting, and critically appraising the quality and potential biases of AI clinical trials \cite{parums2021k6f}.

The development of these AI-specific guidelines was spurred by empirical evidence highlighting significant deficiencies in the reporting and methodological quality of early AI clinical trials. Systematic reviews conducted around the time of their publication revealed pervasive issues; for example, \cite{zhou2021vqt} found that a substantial majority (72.3\\%) of randomized controlled trials evaluating AI prediction tools did not reference the CONSORT statement, indicating a widespread lack of adherence to established reporting standards. This review also identified frequent methodological weaknesses, such as high risks of bias in blinding and outcome assessment, underscoring the urgent need for more structured reporting to improve research quality and clinical impact \cite{zhou2021vqt}. Beyond CONSORT-AI and SPIRIT-AI, other specialized guidelines like DECIDE-AI have emerged to address specific aspects, such as the early-stage clinical evaluation of AI-driven decision support systems, providing a checklist of minimal reporting items to facilitate appraisal and replicability in developmental studies \cite{vasey2022yhn}. The collective importance of these guidelines in promoting awareness of essential content for AI studies in healthcare has been further emphasized by comprehensive reviews of study reporting guidelines \cite{shelmerdine2021xi6}.

Despite their foundational role in standardizing reporting practices and enhancing the transparency of AI clinical trials, these guidelines have acknowledged limitations, particularly in fully capturing the complex nuances of real-world implementation outcomes. While they provide structured recommendations for documenting model development and evaluation, a critical gap remains in systematically assessing how well AI interventions integrate into clinical workflows and achieve sustained adoption. A recent systematic review by \cite{sande20248hm} empirically demonstrated this oversight, revealing that a significant proportion of AI clinical trials, even those adhering to existing reporting guidelines, largely neglect to report crucial implementation outcomes such as acceptability, appropriateness, adoption, and sustainability. This finding suggests that current guidelines, while excellent for technical and clinical efficacy reporting, may not adequately prompt researchers to evaluate the practical success of AI integration into healthcare systems \cite{sande20248hm}. This limitation resonates with broader calls to bridge the "chasm from model performance to clinical impact" by improving the implementation and evaluation of AI, advocating for a shift towards implementation science and real-world evidence \cite{marwaha2022gj3}.

In conclusion, CONSORT-AI and SPIRIT-AI, alongside other specialized guidelines like DECIDE-AI, play a crucial role in standardizing the reporting of AI clinical trials, thereby enhancing their transparency, reproducibility, and critical appraisal. By providing structured recommendations for documenting model development, evaluation, and human-AI interaction, they facilitate robust regulatory review and build trust in AI-driven medical research. However, their current scope highlights an ongoing challenge: the need for continuous evolution to encompass a more comprehensive evaluation of AI's real-world implementation, adoption, and sustained clinical impact. Future iterations and complementary guidelines will likely need to integrate implementation science frameworks more explicitly to ensure that AI innovations not only demonstrate technical prowess but also deliver tangible and sustainable value at the bedside.
\subsection{Regulatory Strategies and Frameworks for AI as Medical Devices}
\label{sec:7\_3\_regulatory\_strategies\_\_and\_\_frameworks\_for\_ai\_as\_medical\_devices}

The escalating integration of artificial intelligence (AI) and machine learning (ML) into healthcare, particularly as medical devices (AI/ML-MD) and within clinical trials, fundamentally challenges traditional regulatory paradigms designed for static medical products. The dynamic, continuously learning nature of advanced AI algorithms necessitates robust, adaptive, and proactive regulatory strategies to ensure safety, efficacy, and ethical deployment throughout their entire product lifecycle \cite{massella2022eix, hamamoto2022gcn}. This urgency is further underscored by empirical evidence revealing persistent methodological weaknesses and reporting gaps in current AI clinical trials, which impede the assessment of true clinical benefit beyond \textit{in silico} performance (as discussed in detail in subsections 7.1 and 7.2). These findings highlight the critical need for regulatory frameworks that can bridge the chasm between promising model performance and demonstrated, safe clinical impact.

A pivotal development in addressing the unique challenges of continuously learning algorithms is the proposed Total Product Lifecycle (TPLC) regulatory approach for AI/ML-Based Software as a Medical Device (SaMD) by the U.S. Food and Drug Administration (FDA) \cite{hamamoto2022gcn}. This framework represents a significant departure from traditional pre-market approval models, which require re-submission for every software change. Instead, the TPLC proposes an adaptive model that permits continuous learning and improvement post-market, provided certain governance structures are in place. Key components of this approach include pre-specified performance objectives, a defined Algorithm Change Protocol (ACP) outlining the types of modifications the algorithm can undergo and how they will be validated, adherence to Good Machine Learning Practice (GMLP) principles, and robust real-world performance monitoring. The ACP is particularly crucial, as it mandates transparency regarding the intended changes and the methods for their verification, aiming to maintain the device's safety and effectiveness while allowing for beneficial evolution \cite{hamamoto2022gcn}.

However, the implementation of such adaptive frameworks is not without its complexities and ongoing debates. Critics and regulatory scientists raise concerns about the practical challenges of continuously monitoring real-world performance for evolving algorithms, particularly in ensuring accountability for post-market changes and maintaining transparency for users and regulators \cite{ehidiamen202480b}. The potential for continuously updating algorithms to inadvertently introduce or amplify algorithmic bias against protected subgroups, even with good intentions, necessitates rigorous and continuous ethical surveillance as an integral part of post-market monitoring \cite{youssef2024fn7}. Defining "significant" changes that warrant re-review versus "expected" learning within the pre-approved ACP remains a nuanced challenge, requiring clear guidelines to prevent regulatory arbitrage or unintended risks. The burden on manufacturers to implement robust validation processes for every iteration and to demonstrate ongoing safety and effectiveness also presents a considerable operational hurdle.

Beyond the U.S. context, other major regulatory bodies are similarly developing strategic roadmaps. The European Medicines Agency (EMA), for instance, has outlined strategic roadmaps for integrating machine learning tools into regulatory science, emphasizing proactive adaptation, stakeholder collaboration, and the need for regulatory science to keep pace with scientific innovation \cite{massella2022eix}. While the EMA's approach shares the FDA's goal of fostering innovation responsibly, it often emphasizes a broader ethical and societal impact assessment, reflecting a more comprehensive regulatory philosophy. International harmonization efforts, such as those by the International Medical Device Regulators Forum (IMDRF), are also crucial for establishing globally consistent principles for AI/ML-MD regulation, aiming to streamline development and market access while upholding universal standards of safety and efficacy.

Complementing regulatory pathways, value-based assessment rubrics are emerging to ensure AI's demonstrable clinical utility and impact. The Radiology AI Deployment and Assessment Rubric (RADAR) provides a seven-level hierarchical framework for comprehensively assessing the value of AI in radiology, moving beyond narrow technical metrics to include diagnostic thinking, therapeutic efficacy, patient outcomes, cost-effectiveness, and crucially, "local efficacy" \cite{boverhof2024izx}. This holistic approach aligns with the need for multi-faceted implementation evaluation, ensuring that AI solutions deliver tangible value in real-world clinical environments and integrate seamlessly into clinical workflows. The review of GI Genius, the first real-time AI-enhanced medical device for endoscopy, serves as a concrete example, illustrating the complexities of its technical architecture, training, and regulatory path, highlighting the practical application of these evolving considerations in a real-world context \cite{cherubini2023az7}. Ethical considerations, including informed consent for adaptive algorithms and robust participant rights protection, are increasingly integrated into these frameworks, recognizing that public trust and responsible deployment are paramount for successful AI adoption \cite{ehidiamen202480b, youssef2024fn7}.

In conclusion, the regulatory landscape for AI as medical devices is rapidly evolving from static, pre-market approval models to adaptive, lifecycle-oriented frameworks like the FDA's TPLC and EMA's strategic roadmaps. This transition is imperative to address the unique challenges of continuously learning algorithms and to ensure the safety, efficacy, and ethical deployment of AI throughout its entire lifecycle. While these adaptive strategies offer a path forward, they also introduce complex implementation challenges related to continuous validation, transparency, accountability for post-market changes, and the proactive management of algorithmic bias. The unresolved tension lies in balancing the rapid pace of AI innovation with the deliberate process of clinical validation and regulatory adaptation, demanding continuous collaboration between AI developers, clinicians, policymakers, and ethicists to ensure safe, effective, and equitable translation into patient care.


\label{sec:conclusion_and_future_directions}

\section{Conclusion and Future Directions}
\label{sec:conclusion\_\_and\_\_future\_directions}

\subsection{Synthesis of Key Advancements}
\label{sec:8\_1\_synthesis\_of\_key\_advancements}

The evolution of artificial intelligence (AI) in clinical trials reflects a maturing field, systematically progressing from initial explorations of potential to the development of targeted solutions for operational challenges, and increasingly, to addressing complex data integration, strategic design, and critical trustworthiness concerns. This trajectory collectively aims to enhance the efficiency, ethical conduct, and capacity for personalized medicine within drug development.

Early foundational reviews established the broad intellectual landscape, identifying key opportunities and challenges for AI integration. \cite{WANG2019} provided an initial mapping of AI's role specifically in patient recruitment, detailing techniques like Natural Language Processing (NLP), Machine Learning (ML), and Deep Learning (DL) for this critical stage. Expanding on this, \cite{CHEN2020} offered a more comprehensive overview, surveying AI applications across the entire clinical trial lifecycle, from design to post-market surveillance. These reviews were instrumental in highlighting overarching trends, such as the need for improved data quality and interpretability, and underscoring ethical considerations, thereby setting the stage for subsequent applied research.

Building upon these landscape analyses, the field transitioned towards developing targeted AI applications for operational improvements. \cite{ZHANG2021} demonstrated a concrete advancement in patient selection by leveraging ML with Electronic Health Record (EHR) data, directly addressing a major bottleneck in trial initiation. Further streamlining the early stages, \cite{LIU2022} introduced a deep learning approach for optimal protocol generation, showcasing AI's capacity to enhance the strategic design phase by optimizing complex parameters. Recognizing the limitations of relying solely on traditional trial data, \cite{KIM2023} proposed a hybrid framework that integrates AI/ML with Real-World Evidence (RWE) for broader optimization across trial design, patient selection, and monitoring. This integration of RWE represents a significant shift towards more comprehensive data utilization, although it introduces challenges related to data heterogeneity and generalizability.

A pivotal advancement in addressing these complex data challenges, particularly concerning privacy and secure data integration, is the emergence of federated learning. \cite{SINGH2024} introduced this distributed machine learning paradigm, enabling collaborative AI model training across multiple institutions without requiring the direct sharing of sensitive raw patient data. This methodological innovation directly tackles the privacy barriers inherent in multi-site clinical trials, thereby facilitating the secure and ethical utilization of diverse, distributed datasets essential for RWE integration and robust AI development. Such advancements are crucial for enabling more sophisticated strategic applications, such as the development of predictive biomarkers. For instance, \cite{armstrong2023dwd} showcased an AI-derived digital pathology-based biomarker, validated across multiple Phase III trials, to predict the benefit of long-term androgen deprivation therapy in prostate cancer. This exemplifies how advanced AI can contribute to personalized medicine by guiding treatment duration, though it underscores the rigorous validation required for clinical utility.

As AI models become more complex and their applications more strategic, the critical focus on trustworthiness, encompassing fairness, explainability, and human factors, has intensified. The deployment of advanced AI, particularly Large Language Models (LLMs) for tasks like protocol generation or clinical assistance, necessitates robust validation and a clear understanding of their capabilities and limitations. \cite{thirunavukarasu2023wg0} critically examines how the clinical aptitude of AI assistants should be assayed, advocating for rigorous evidence, potentially through randomized controlled trials, before widespread deployment. This highlights the imperative for explainable AI (XAI) to ensure transparency and for human factors to be considered in design, ensuring that AI tools are not only effective but also interpretable, safe, and ethically sound for clinical adoption.

In summary, the literature demonstrates a clear progression from broad conceptualization to concrete operational improvements, followed by sophisticated methodological innovations for data handling and strategic decision-making. The field is increasingly prioritizing the trustworthiness of AI systems, ensuring that advancements in efficiency and personalized medicine are underpinned by robust validation, ethical considerations, and interpretability. This systematic approach across the drug development pipeline signifies a maturing field poised to deliver more efficient, ethical, and patient-centric clinical trials.
\subsection{Unresolved Tensions and Future Research Avenues}
\label{sec:8\_2\_unresolved\_tensions\_\_and\_\_future\_research\_avenues}

The integration of artificial intelligence (AI) into clinical trials, while promising transformative advancements, is fraught with persistent challenges and inherent tensions that impede its widespread and equitable adoption. Foremost among these is the fundamental disconnect between the rapid pace of AI innovation and the slow, deliberate processes required for rigorous clinical validation and regulatory adaptation. This section delves into these unresolved tensions, bridging the gap between theoretical potential and real-world implementation, and navigating critical issues such as data privacy versus data utility, to outline crucial future research directions.

A foundational tension lies in ensuring the intrinsic integrity and fairness of AI models, particularly as they are deployed in high-stakes clinical applications. \cite{kelly2019gw7} addresses this by proposing a novel decomposition of discrimination metrics into bias, variance, and noise components, offering a diagnostic framework to understand the root causes of unfairness. This work innovatively shifts the paradigm from merely mitigating discrimination to diagnosing its sources, often pointing to inadequate data collection as the culprit and advocating for data-centric interventions. However, a limitation noted by \cite{kelly2019gw7} is its assumption that observed performance differences are inherently discriminatory, without delving into causal inference or historical biases embedded in labels, thus highlighting a critical future research avenue for developing more robust and causally-aware AI models that ensure equitable outcomes.

Even with intrinsically fair and accurate models, the journey from theoretical potential to real-world implementation introduces significant complexities. The effective integration of AI into clinical workflows, particularly concerning human-AI collaboration, remains a major hurdle. \cite{rosenthal2025j23} empirically quantifies cognitive biases in human-AI interaction through a rigorous randomized controlled experiment with human experts. Their findings reveal significant biases like automation neglect and correlation neglect, demonstrating that AI assistance does not always improve human diagnostic quality and that optimal collaboration often involves delegating cases entirely to either humans or AI, but rarely to AI-assisted humans. This underscores the need for AI systems designed to be "bias-aware" in their interaction and for targeted training protocols for clinicians, ensuring that the human element does not inadvertently undermine AI's benefits. While AI-assisted analysis can reduce variability and improve prognostic value, as shown by \cite{gieraerts2020j5j} in COVID-19 lung involvement, the insights from \cite{rosenthal2025j23} suggest that the \textit{manner} of deployment is paramount.

The rapid innovation in AI, exemplified by its potential in drug discovery and trial optimization \cite{blasiak2020fkz, chorev20230xi, ho2020xwh, patel2024jpj}, consistently outpaces the mechanisms for clinical validation and regulatory acceptance. \cite{macheka2024o73}'s systematic review of AI applications in cancer pathways reveals a critical gap: the majority of AI oncological research remains experimental, lacking prospective clinical validation and failing to translate measured AI efficacy into beneficial clinical outcomes. This review points to a lack of research standardization and health system interoperability as key barriers, directly addressing the tension between innovation and validation. Further, \cite{chen2024zvv} identifies specific trial design factors associated with the completion of AI clinical trials, highlighting that trials conducted in Europe and those with larger sample sizes are more likely to succeed. This emphasizes the practical challenges in designing and executing AI trials that meet regulatory and scientific rigor, necessitating a focus on addressing common reasons for study failure.

Another pervasive tension involves balancing data privacy with the imperative for data utility. While not explicitly detailed by a paper in this specific set, the broader field of AI for clinical trials consistently grapples with the need for large, diverse datasets to train robust models, often clashing with stringent data protection regulations and patient privacy concerns. This necessitates future research into privacy-preserving AI techniques, such as federated learning, and secure data sharing frameworks that can unlock the full potential of AI without compromising patient trust. Furthermore, the ethical and practical considerations extend beyond data privacy to encompass issues of accountability, transparency, and equitable access. \cite{sidiq2023692} highlights challenges in implementing AI for physiotherapy clinical trials in India, including data security, ethical considerations, and the need for specialized training, reinforcing the global nature of these barriers.

In conclusion, the path forward for AI in clinical trials demands a concerted effort to address these multifaceted tensions. Future research must prioritize developing more robust, generalizable, and causally-aware AI models that can perform reliably across diverse populations and clinical settings. Enhancing causal inference capabilities within AI models, particularly in understanding the true impact of interventions and the sources of bias, is paramount. Crucially, fostering interdisciplinary collaboration among AI researchers, clinicians, ethicists, regulators, and social scientists is essential to navigate the complex ethical, social, and practical barriers to widespread adoption and to ensure equitable access to AI-driven advancements in healthcare. This integrated approach will be key to realizing the transformative potential of AI in clinical trials responsibly and effectively.


\newpage
\section*{References}
\addcontentsline{toc}{section}{References}

\begin{thebibliography}{305}

\bibitem{kelly2019gw7}
Christopher J. Kelly, A. Karthikesalingam, Mustafa Suleyman, et al. (2019). \textit{Key challenges for delivering clinical impact with artificial intelligence}. BMC Medicine.

\bibitem{acosta2022sxu}
J. Acosta, G. Falcone, P. Rajpurkar, et al. (2022). \textit{Multimodal biomedical AI}. Nature Network Boston.

\bibitem{rivera2020sg1}
S. Cruz Rivera, Xiaoxuan Liu, A. Chan, et al. (2020). \textit{Guidelines for clinical trial protocols for interventions involving artificial intelligence: the SPIRIT-AI extension}. Nature Network Boston.

\bibitem{chan2020egf}
An-Wen Chan, A. Darzi, Christopher Holmes, et al. (2020). \textit{Reporting guidelines for clinical trial reports for interventions involving artificial intelligence: the CONSORT-AI extension}. Nature Network Boston.

\bibitem{vasey2022yhn}
B. Vasey, M. Nagendran, Bruce Campbell, et al. (2022). \textit{Reporting guideline for the early stage clinical evaluation of decision support systems driven by artificial intelligence: DECIDE-AI}. British medical journal.

\bibitem{vasey2022oig}
B. Vasey, M. Nagendran, Bruce Campbell, et al. (2022). \textit{Reporting guideline for the early-stage clinical evaluation of decision support systems driven by artificial intelligence: DECIDE-AI}. Nature Network Boston.

\bibitem{burstein2019qgx}
H. Burstein, C. Lacchetti, Holly Anderson, et al. (2019). \textit{Adjuvant Endocrine Therapy for Women With Hormone Receptor-Positive Breast Cancer: ASCO Clinical Practice Guideline Focused Update.}. Journal of Clinical Oncology.

\bibitem{liu2021si6}
Ruishan Liu, S. Rizzo, S. Whipple, et al. (2021). \textit{Evaluating eligibility criteria of oncology trials using real-world data and AI}. Nature.

\bibitem{agrawal2018svf}
Prashansa Agrawal (2018). \textit{Artificial Intelligence in Drug Discovery and Development}. Unpublished manuscript.

\bibitem{bachelot2012ujd}
T. Bachelot, C. Bourgier, C. Cropet, et al. (2012). \textit{Randomized phase II trial of everolimus in combination with tamoxifen in patients with hormone receptor-positive, human epidermal growth factor receptor 2-negative metastatic breast cancer with prior exposure to aromatase inhibitors: a GINECO study.}. Journal of Clinical Oncology.

\bibitem{yin20206qf}
Jiamin Yin, K. Ngiam, and H. Teo (2020). \textit{Role of Artificial Intelligence Applications in Real-Life Clinical Practice: Systematic Review}. Journal of Medical Internet Research.

\bibitem{jayatunga2022sqw}
Madura K P Jayatunga, Wen Xie, Ludwig F Ruder, et al. (2022). \textit{AI in small-molecule drug discovery: a coming wave?}. Nature reviews. Drug discovery.

\bibitem{chen2021rf1}
Zi-Hang Chen, Li Lin, Chen-Fei Wu, et al. (2021). \textit{Artificial intelligence for assisting cancer diagnosis and treatment in the era of precision medicine}. Cancer Communications.

\bibitem{sande20217w9}
Davy van de Sande, M. V. van Genderen, J. Huiskens, et al. (2021). \textit{Moving from bytes to bedside: a systematic review on the use of artificial intelligence in the intensive care unit}. Intensive Care Medicine.

\bibitem{roy20223mf}
Sudipta Roy, Tanushree Meena, and Se-Jung Lim (2022). \textit{Demystifying Supervised Learning in Healthcare 4.0: A New Reality of Transforming Diagnostic Medicine}. Diagnostics.

\bibitem{lin2020ghb}
Xuan Lin, Zhe Quan, Zhi-Jie Wang, et al. (2020). \textit{KGNN: Knowledge Graph Neural Network for Drug-Drug Interaction Prediction}. International Joint Conference on Artificial Intelligence.

\bibitem{nussinov2022vua}
R. Nussinov, Mingzhen Zhang, Yonglan Liu, et al. (2022). \textit{AlphaFold, Artificial Intelligence (AI), and Allostery}. Journal of Physical Chemistry B.

\bibitem{shaheen2021rqf}
Mohammed Shaheen (2021). \textit{Applications of Artificial Intelligence (AI) in healthcare: A review}. Unpublished manuscript.

\bibitem{schaar2020xiv}
Mihaela van der Schaar, A. Alaa, Andres Floto, et al. (2020). \textit{How artificial intelligence and machine learning can help healthcare systems respond to COVID-19}. Machine-mediated learning.

\bibitem{burstein2010zfk}
H. Burstein, A. Prestrud, J. Seidenfeld, et al. (2010). \textit{American Society of Clinical Oncology clinical practice guideline: update on adjuvant endocrine therapy for women with hormone receptor-positive breast cancer.}. Journal of Clinical Oncology.

\bibitem{stephenson2019t0l}
Natalie Stephenson, Emily Shane, Jessica Chase, et al. (2019). \textit{Survey of Machine Learning Techniques in Drug Discovery.}. Current drug metabolism.

\bibitem{selvaraj2021n52}
C. Selvaraj, I. Chandra, and S. Singh (2021). \textit{Artificial intelligence and machine learning approaches for drug design: challenges and opportunities for the pharmaceutical industries}. Molecular diversity.

\bibitem{lam2022z48}
Thomas Y T Lam, Max Fk Cheung, Y. Munro, et al. (2022). \textit{Randomized Controlled Trials of Artificial Intelligence in Clinical Practice: Systematic Review}. Journal of Medical Internet Research.

\bibitem{piccialli2021d0v}
F. Piccialli, Vincenzo Schiano Di Cola, F. Giampaolo, et al. (2021). \textit{The Role of Artificial Intelligence in Fighting the COVID-19 Pandemic}. Information Systems Frontiers.

\bibitem{franik2018f8x}
S. Franik, Stephanie Eltrop, J. Kremer, et al. (2018). \textit{Aromatase inhibitors (letrozole) for subfertile women with polycystic ovary syndrome.}. Cochrane Database of Systematic Reviews.

\bibitem{jayakumar2022sav}
Shruti Jayakumar, V. Sounderajah, P. Normahani, et al. (2022). \textit{Quality assessment standards in artificial intelligence diagnostic accuracy systematic reviews: a meta-research study}. npj Digital Medicine.

\bibitem{sharma2022i1r}
Ashwani Sharma, Tarun Virmani, Vipluv Pathak, et al. (2022). \textit{Artificial Intelligence-Based Data-Driven Strategy to Accelerate Research, Development, and Clinical Trials of COVID Vaccine}. BioMed Research International.

\bibitem{chen2020ndf}
Bin Chen, L. Garmire, D. Calvisi, et al. (2020). \textit{Harnessing big ‘omics’ data and AI for drug discovery in hepatocellular carcinoma}. Nature reviews: Gastroenterology & hepatology.

\bibitem{ho2020xwh}
D. Ho (2020). \textit{Artificial intelligence in cancer therapy}. Science.

\bibitem{zhou2021vqt}
Qian Zhou, Zhi-hang Chen, Yi-heng Cao, et al. (2021). \textit{Clinical impact and quality of randomized controlled trials involving interventions evaluating artificial intelligence prediction tools: a systematic review}. npj Digital Medicine.

\bibitem{zhong2018jjh}
Feisheng Zhong, Jing Xing, Xutong Li, et al. (2018). \textit{Artificial intelligence in drug design}. Science China Life Sciences.

\bibitem{ibrahim2021rcn}
H. Ibrahim, Xiaoxuan Liu, Samantha Cruz Rivera, et al. (2021). \textit{Reporting guidelines for clinical trials of artificial intelligence interventions: the SPIRIT-AI and CONSORT-AI guidelines}. Trials.

\bibitem{kaur2021wc9}
H. Kaur, Nishant Shekhar, Saurabh Sharma, et al. (2021). \textit{Ivermectin as a potential drug for treatment of COVID-19: an in-sync review with clinical and computational attributes}. Pharmacological Reports.

\bibitem{topol2020uuy}
E. Topol (2020). \textit{Welcoming new guidelines for AI clinical research}. Nature Network Boston.

\bibitem{foo2022wuj}
Malia Alexandra Foo, Mingliang You, S. L. Chan, et al. (2022). \textit{Clinical translation of patient-derived tumour organoids- bottlenecks and strategies}. Biomarker Research.

\bibitem{brasil2019v71}
S. Brasil, C. Pascoal, R. Francisco, et al. (2019). \textit{Artificial Intelligence (AI) in Rare Diseases: Is the Future Brighter?}. Genes.

\bibitem{woo2019njt}
M. Woo (2019). \textit{An AI boost for clinical trials}. Nature.

\bibitem{rana2022f28}
S. Rana, S. Rana, Kashif Nisar, et al. (2022). \textit{Blockchain Technology and Artificial Intelligence Based Decentralized Access Control Model to Enable Secure Interoperability for Healthcare}. Sustainability.

\bibitem{pesapane2022l3q}
F. Pesapane, A. Rotili, S. Penco, et al. (2022). \textit{Digital Twins in Radiology}. Journal of Clinical Medicine.

\bibitem{talpur2022u5p}
Sarena Talpur, Fahad Azim, Munaf Rashid, et al. (2022). \textit{Uses of Different Machine Learning Algorithms for Diagnosis of Dental Caries}. Journal of Healthcare Engineering.

\bibitem{bain2017w6o}
E. Bain, L. Shafner, D. Walling, et al. (2017). \textit{Use of a Novel Artificial Intelligence Platform on Mobile Devices to Assess Dosing Compliance in a Phase 2 Clinical Trial in Subjects With Schizophrenia}. JMIR mHealth and uHealth.

\bibitem{angus2020epl}
D. Angus (2020). \textit{Randomized Clinical Trials of Artificial Intelligence.}. Journal of the American Medical Association (JAMA).

\bibitem{trinder2020yxj}
M. Trinder, Yanan Wang, C. M. Madsen, et al. (2020). \textit{Inhibition of Cholesteryl Ester Transfer Protein Preserves High-Density Lipoprotein Cholesterol and Improves Survival in Sepsis.}. Circulation.

\bibitem{baxi2022n9i}
V. Baxi, George Lee, C. Duan, et al. (2022). \textit{Association of artificial intelligence-powered and manual quantification of programmed death-ligand 1 (PD-L1) expression with outcomes in patients treated with nivolumab ± ipilimumab}. Modern Pathology.

\bibitem{girolami20228yi}
Ilaria Girolami, L. Pantanowitz, S. Marletta, et al. (2022). \textit{Artificial intelligence applications for pre-implantation kidney biopsy pathology practice: a systematic review}. JN. Journal of Nephrology (Milano. 1992).

\bibitem{lian2011ut0}
J. Lian, J. Lian, Xiaoqing Wu, et al. (2011). \textit{A natural BH3 mimetic induces autophagy in apoptosis-resistant prostate cancer via modulating Bcl-2–Beclin1 interaction at endoplasmic reticulum}. Cell Death and Differentiation.

\bibitem{lee2018ung}
Cecilia S. Lee, Ariel J. Tyring, Yue Wu, et al. (2018). \textit{Generating retinal flow maps from structural optical coherence tomography with artificial intelligence}. Scientific Reports.

\bibitem{seol20216kl}
H. Seol, P. Shrestha, Joy Fladager Muth, et al. (2021). \textit{Artificial intelligence-assisted clinical decision support for childhood asthma management: A randomized clinical trial}. PLoS ONE.

\bibitem{franik2014wq5}
S. Franik, J. Kremer, W. Nelen, et al. (2014). \textit{Aromatase inhibitors for subfertile women with polycystic ovary syndrome.}. Cochrane Database of Systematic Reviews.

\bibitem{qaiser202295m}
Talha Qaiser, Ching-Yi Lee, Michel Vandenberghe, et al. (2022). \textit{Usability of deep learning and H&E images predict disease outcome-emerging tool to optimize clinical trials}. npj Precision Oncology.

\bibitem{marwaha20139qu}
S. Marwaha, Zhimin He, M. Broome, et al. (2013). \textit{How is affective instability defined and measured? A systematic review}. Psychological Medicine.

\bibitem{shelmerdine2021xi6}
S. Shelmerdine, O. Arthurs, A. Denniston, et al. (2021). \textit{Review of study reporting guidelines for clinical studies using artificial intelligence in healthcare}. BMJ Health & Care Informatics.

\bibitem{haddad2021fiy}
T. Haddad, J. Helgeson, K. Pomerleau, et al. (2021). \textit{Accuracy of an Artificial Intelligence System for Cancer Clinical Trial Eligibility Screening: Retrospective Pilot Study}. JMIR Medical Informatics.

\bibitem{hamamoto2022gcn}
Ryuji Hamamoto, T. Koyama, Nobuji Kouno, et al. (2022). \textit{Introducing AI to the molecular tumor board: one direction toward the establishment of precision medicine using large-scale cancer clinical and biological information}. Experimental Hematology & Oncology.

\bibitem{kyo2021ffp}
M. Kyo, T. Shimatani, K. Hosokawa, et al. (2021). \textit{Patient–ventilator asynchrony, impact on clinical outcomes and effectiveness of interventions: a systematic review and meta-analysis}. Journal of Intensive Care.

\bibitem{goyal2022w0p}
H. Goyal, S. Sherazi, Shweta Gupta, et al. (2022). \textit{Application of artificial intelligence in diagnosis of pancreatic malignancies by endoscopic ultrasound: a systemic review}. Therapeutic Advances in Gastroenterology.

\bibitem{wismller20202tv}
A. Wismüller, and Larry Stockmaster (2020). \textit{A prospective randomized clinical trial for measuring radiology study reporting time on Artificial Intelligence-based detection of intracranial hemorrhage in emergent care head CT}. Biomedical Applications in Molecular, Structural, and Functional Imaging.

\bibitem{alexander2020mvn}
M. Alexander, B. Solomon, D. Ball, et al. (2020). \textit{Evaluation of an artificial intelligence clinical trial matching system in Australian lung cancer patients}. JAMIA Open.

\bibitem{marwaha2022gj3}
J. Marwaha, and J. Kvedar (2022). \textit{Crossing the chasm from model performance to clinical impact: the need to improve implementation and evaluation of AI}. npj Digital Medicine.

\bibitem{krittanawong201811r}
Chayakrit Krittanawong, A. Bomback, U. Baber, et al. (2018). \textit{Future Direction for Using Artificial Intelligence to Predict and Manage Hypertension}. Current Hypertension Reports.

\bibitem{vall2021mrm}
Andreu Vall, Yogesh Sabnis, Jiye Shi, et al. (2021). \textit{The Promise of AI for DILI Prediction}. Frontiers in Artificial Intelligence.

\bibitem{mak2021pi8}
Kit-Kay Mak, M. Balijepalli, and M. Pichika (2021). \textit{Success stories of AI in drug discovery - where do things stand?}. Expert Opinion on Drug Discovery.

\bibitem{huang2021bpp}
Ding Huang, Jingyi Shen, Jiaze Hong, et al. (2021). \textit{Effect of artificial intelligence-aided colonoscopy for adenoma and polyp detection: a meta-analysis of randomized clinical trials}. International Journal of Colorectal Disease.

\bibitem{day202186d}
T. Day, Bernhard Kainz, J. Hajnal, et al. (2021). \textit{Artificial intelligence, fetal echocardiography, and congenital heart disease}. Prenatal Diagnosis.

\bibitem{barufaldi2021o1w}
B. Barufaldi, Andrew D. A. Maidment, M. Dustler, et al. (2021). \textit{VIRTUAL CLINICAL TRIALS IN MEDICAL IMAGING SYSTEM EVALUATION AND OPTIMISATION}. Radiation Protection Dosimetry.

\bibitem{pickering2021tlg}
J. Pickering (2021). \textit{Trust, but Verify: Informed Consent, AI Technologies, and Public Health Emergencies}. Future Internet.

\bibitem{attaurrahman20212gv}
Atta-ur-Rahman, K. Sultan, Iftikhar Naseer, et al. (2021). \textit{Supervised Machine Learning-Based Prediction of COVID-19}. Computers Materials & Continua.

\bibitem{abdelrazeq2022hut}
H. Abdel-Razeq, Fawzi Abu Rous, F. Abuhijla, et al. (2022). \textit{Breast Cancer in Geriatric Patients: Current Landscape and Future Prospects}. Clinical Interventions in Aging.

\bibitem{mccrea2012ceq}
Cindy E. McCrea, Ann C Skulas-Ray, M. Chow, et al. (2012). \textit{Test–retest reliability of pulse amplitude tonometry measures of vascular endothelial function: Implications for clinical trial design}. Vascular Medicine.

\bibitem{franik2022flg}
S. Franik, Quang-Khoi Le, J. Kremer, et al. (2022). \textit{Aromatase inhibitors (letrozole) for ovulation induction in infertile women with polycystic ovary syndrome.}. Cochrane Database of Systematic Reviews.

\bibitem{cascini2022t0a}
F. Cascini, F. Beccia, F. Causio, et al. (2022). \textit{Scoping review of the current landscape of AI-based applications in clinical trials}. Frontiers in Public Health.

\bibitem{voola20229e1}
Pramod Kumar Voola, Vijay Bhasker, Reddy Bhimanapati, et al. (2022). \textit{AI-Powered Chatbots in Clinical Trials: Enhancing Patient-Clinician Interaction and Decision-Making}. International Journal for Research Publication and Seminar.

\bibitem{annandale2014p66}
C. Annandale, D. Holm, K. Ebersohn, et al. (2014). \textit{Seminal transmission of lumpy skin disease virus in heifers.}. Transboundary and Emerging Diseases.

\bibitem{hellemond201823h}
Irene E. G. van Hellemond, S. Geurts, and V. Tjan-Heijnen (2018). \textit{Current Status of Extended Adjuvant Endocrine Therapy in Early Stage Breast Cancer}. Current Treatment Options in Oncology.

\bibitem{ashat20216hq}
M. Ashat, J. Klair, D. Singh, et al. (2021). \textit{Impact of real-time use of artificial intelligence in improving adenoma detection during colonoscopy: A systematic review and meta-analysis}. Endoscopy International Open.

\bibitem{calapricewhitty2020pmi}
Denise Calaprice-Whitty, Karim Galil, Wael Salloum, et al. (2020). \textit{Improving Clinical Trial Participant Prescreening With Artificial Intelligence (AI): A Comparison of the Results of AI-Assisted vs Standard Methods in 3 Oncology Trials}. Therapeutic Innovation and  Regulatory Science.

\bibitem{vernieri20202p2}
C. Vernieri, F. Corti, F. Nichetti, et al. (2020). \textit{Everolimus versus alpelisib in advanced hormone receptor-positive HER2-negative breast cancer: targeting different nodes of the PI3K/AKT/mTORC1 pathway with different clinical implications}. Breast Cancer Research.

\bibitem{lee2020qt0}
Cecilia S. Lee, and Aaron Y. Lee (2020). \textit{How Artificial Intelligence Can Transform Randomized Controlled Trials}. Translational Vision Science & Technology.

\bibitem{fan2019g3m}
P. Fan, and V. Jordan (2019). \textit{New insights into acquired endocrine resistance of breast cancer}. Cancer Drug Resistance.

\bibitem{kamanna20130sz}
V. Kamanna, S. Ganji, and M. Kashyap (2013). \textit{Recent advances in niacin and lipid metabolism}. Current Opinion in Lipidology.

\bibitem{singh2019pz0}
Vibha Singh, P. Jaiswal, Ishita Ghosh, et al. (2019). \textit{Targeting the TLK1/NEK1 DDR axis with Thioridazine suppresses outgrowth of androgen independent prostate tumors}. International Journal of Cancer.

\bibitem{nagaraj2020e52}
G. Nagaraj, and Cynthia X. Ma (2020). \textit{Clinical Challenges in the Management of Hormone Receptor-Positive, Human Epidermal Growth Factor Receptor 2-Negative Metastatic Breast Cancer: A Literature Review}. Advances in Therapy.

\bibitem{richardson2022yuq}
Peter J. Richardson, B. Robinson, Daniel P. Smith, et al. (2022). \textit{The AI-Assisted Identification and Clinical Efficacy of Baricitinib in the Treatment of COVID-19}. Vaccines.

\bibitem{dankelman2022nvx}
Lente H. M. Dankelman, Sanne Schilstra, F. IJpma, et al. (2022). \textit{Artificial intelligence fracture recognition on computed tomography: review of literature and recommendations}. European Journal of Trauma and Emergency Surgery.

\bibitem{zhang2021ere}
Yuanchuan Zhang, Xubing Zhang, Qingbin Wu, et al. (2021). \textit{Artificial Intelligence-Aided Colonoscopy for Polyp Detection: A Systematic Review and Meta-Analysis of Randomized Clinical Trials.}. Journal of laparoendoscopic & advanced surgical techniques. Part A.

\bibitem{namba2017t7o}
Sayaka Namba, M. Yamaoka-Tojo, Ryota Kakizaki, et al. (2017). \textit{Effects on bone metabolism markers and arterial stiffness by switching to rivaroxaban from warfarin in patients with atrial fibrillation}. Heart and Vessels.

\bibitem{seo20223ls}
Youngho Seo, Hyemin Jang, and Hyejoo Lee (2022). \textit{Potential Applications of Artificial Intelligence in Clinical Trials for Alzheimer’s Disease}. Life.

\bibitem{dong2020g8g}
Jingsi Dong, Yingcai Geng, Dan Lu, et al. (2020). \textit{Clinical Trials for Artificial Intelligence in Cancer Diagnosis: A Cross-Sectional Study of Registered Trials in ClinicalTrials.gov}. Frontiers in Oncology.

\bibitem{pasricha2022cld}
S. Pasricha (2022). \textit{AI Ethics in Smart Healthcare}. IEEE Consumer Electronics Magazine.

\bibitem{blasiak2020fkz}
Agata Blasiak, J. Lim, S. G. Seah, et al. (2020). \textit{IDentif.AI: Rapidly optimizing combination therapy design against severe Acute Respiratory Syndrome Coronavirus 2 (SARS‐Cov‐2) with digital drug development}. Bioengineering & Translational Medicine.

\bibitem{isidori201962v}
A. Isidori, G. Arnaldi, M. Boscaro, et al. (2019). \textit{Towards the tailoring of glucocorticoid replacement in adrenal insufficiency: the Italian Society of Endocrinology Expert Opinion}. Journal of Endocrinological Investigation.

\bibitem{egelston20216fy}
Colt A. Egelston, Weihua Guo, S. Yost, et al. (2021). \textit{Pre-existing effector T-cell levels and augmented myeloid cell composition denote response to CDK4/6 inhibitor palbociclib and pembrolizumab in hormone receptor-positive metastatic breast cancer}. Journal for ImmunoTherapy of Cancer.

\bibitem{nakase2020mw1}
H. Nakase, Takehiro Hirano, Kohei Wagatsuma, et al. (2020). \textit{Artificial intelligence‐assisted endoscopy changes the definition of mucosal healing in ulcerative colitis}. Digestive Endoscopy.

\bibitem{wang2022vvz}
Jixuan Wang, Jingbo Yang, Haochi Zhang, et al. (2022). \textit{PhenoPad: Building AI enabled note-taking interfaces for patient encounters}. npj Digital Medicine.

\bibitem{chien201519r}
Tsai-Ju Chien, Chia-Yu Liu, Yi-Fang Chang, et al. (2015). \textit{Acupuncture for treating aromatase inhibitor-related arthralgia in breast cancer: a systematic review and meta-analysis.}. Journal of Alternative and Complementary Medicine.

\bibitem{gorbach20134uw}
P. Gorbach, B. Mensch, Marla J. Husnik, et al. (2013). \textit{Effect of Computer-Assisted Interviewing on Self-Reported Sexual Behavior Data in a Microbicide Clinical Trial}. Aids and Behavior.

\bibitem{denecke2019g9r}
K. Denecke, E. Gabarron, R. Grainger, et al. (2019). \textit{Artificial Intelligence for Participatory Health: Applications, Impact, and Future Implications}. Yearbook of Medical Informatics.

\bibitem{lian2010tj7}
J. Lian, D. Karnak, and Liang Xu (2010). \textit{The Bcl-2-Beclin 1 interaction in (-)-gossypol-induced autophagy versus apoptosis in prostate cancer cells}. Unpublished manuscript.

\bibitem{massella2022eix}
Maurizio Massella, D. A. Dri, and D. Gramaglia (2022). \textit{Regulatory Considerations on the use of Machine Learning based tools in Clinical Trials}. Health technology.

\bibitem{wang2022yim}
An-Qi Wang, X. Xiu, Shengyu Liu, et al. (2022). \textit{Characteristics of Artificial Intelligence Clinical Trials in the Field of Healthcare: A Cross-Sectional Study on ClinicalTrials.gov}. International Journal of Environmental Research and Public Health.

\bibitem{kolla2021n6o}
Likhitha Kolla, Fred Gruber, C. Hill, et al. (2021). \textit{The case for AI-driven cancer clinical trials - The efficacy arm in silico.}. Biochimica et biophysica acta. Reviews on cancer.

\bibitem{osullivan2021xpq}
Mark E. O'Sullivan, E. C. Considine, M. O'Riordan, et al. (2021). \textit{Challenges of Developing Robust AI for Intrapartum Fetal Heart Rate Monitoring}. Frontiers in Artificial Intelligence.

\bibitem{bresso2021fri}
E. Bresso, P. Monnin, Cédric Bousquet, et al. (2021). \textit{Investigating ADR mechanisms with Explainable AI: a feasibility study with knowledge graph mining}. BMC Medical Informatics and Decision Making.

\bibitem{rashid2020k8k}
M. Rashid (2020). \textit{Artificial Intelligence Effecting a Paradigm Shift in Drug Development}. SLAS technology.

\bibitem{vlake2022r75}
J. Vlake, J. van Bommel, G. Riva, et al. (2022). \textit{Reporting the early stage clinical evaluation of virtual-reality-based intervention trials: RATE-VR}. Nature Network Boston.

\bibitem{guo2022ekh}
Junchen Guo, Yuyan Wu, Lizhi Chen, et al. (2022). \textit{A perspective on the diagnosis of cracked tooth: imaging modalities evolve to AI-based analysis}. BioMedical Engineering OnLine.

\bibitem{seol2020mqp}
H. Seol, M. Rolfes, Wi Chung, et al. (2020). \textit{Expert artificial intelligence-based natural language processing characterises childhood asthma}. BMJ Open Respiratory Research.

\bibitem{subirana2020y2f}
B. Subirana, F. Hueto, P. Rajasekaran, et al. (2020). \textit{Hi Sigma, do I have the Coronavirus?: Call for a New Artificial Intelligence Approach to Support Health Care Professionals Dealing With The COVID-19 Pandemic}. arXiv.org.

\bibitem{ingle2020pre}
J. Ingle, J. Cairns, V. Suman, et al. (2020). \textit{Anastrozole has an Association between Degree of Estrogen Suppression and Outcomes in Early Breast Cancer and is a Ligand for Estrogen Receptor α}. Clinical Cancer Research.

\bibitem{grote2021iet}
Thomas Grote (2021). \textit{Randomised controlled trials in medical AI: ethical considerations}. Journal of Medical Ethics.

\bibitem{astbury2021926}
S. Astbury, J. Grove, D. Dorward, et al. (2021). \textit{Reliable computational quantification of liver fibrosis is compromised by inherent staining variation}. The Journal of Pathology: Clinical Research.

\bibitem{ramosesquivel2018a0r}
A. Ramos-Esquivel, Hellen Hernández-Steller, M. Savard, et al. (2018). \textit{Cyclin-dependent kinase 4/6 inhibitors as first-line treatment for post-menopausal metastatic hormone receptor-positive breast cancer patients: a systematic review and meta-analysis of phase III randomized clinical trials}. Breast Cancer.

\bibitem{delso20206mh}
G. Delso, D. Cirillo, Joshua D. Kaggie, et al. (2020). \textit{How to Design AI-Driven Clinical Trials in Nuclear Medicine.}. Seminars in nuclear medicine.

\bibitem{siontis2021l0w}
G. Siontis, R. Sweda, P. Noseworthy, et al. (2021). \textit{Development and validation pathways of artificial intelligence tools evaluated in randomised clinical trials}. BMJ Health & Care Informatics.

\bibitem{genin202155z}
K. Genin, and Thomas Grote (2021). \textit{Randomized Controlled Trials in Medical AI A Methodological Critique}. Philosophy and Medicine.

\bibitem{mellem2021p29}
Monika S. Mellem, Matthew Kollada, J. Tiller, et al. (2021). \textit{Explainable AI enables clinical trial patient selection to retrospectively improve treatment effects in schizophrenia}. BMC Medical Informatics and Decision Making.

\bibitem{emde20216qd}
L. von der Emde, M. Pfau, F. Holz, et al. (2021). \textit{AI-based structure-function correlation in age-related macular degeneration}. Eye.

\bibitem{gierach2017d4t}
Gretchen L. Gierach, R. Curtis, R. Pfeiffer, et al. (2017). \textit{Association of Adjuvant Tamoxifen and Aromatase Inhibitor Therapy With Contralateral Breast Cancer Risk Among US Women With Breast Cancer in a General Community Setting}. JAMA Oncology.

\bibitem{liu2021bff}
Xing Liu, Chun Gui, Weiming Wen, et al. (2021). \textit{Safety and Efficacy of High Power Shorter Duration Ablation Guided by Ablation Index or Lesion Size Index in Atrial Fibrillation Ablation: A Systematic Review and Meta-Analysis}. Journal of interventional cardiology.

\bibitem{rieckmann2012ixn}
Traci R. Rieckmann, D. McCarty, A. Kovas, et al. (2012). \textit{American Indians with Substance Use Disorders: Treatment Needs and Comorbid Conditions}. The American Journal of Drug and Alcohol Abuse.

\bibitem{euppayo2017517}
T. Euppayo, V. Punyapornwithaya, S. Chomdej, et al. (2017). \textit{Effects of hyaluronic acid combined with anti-inflammatory drugs compared with hyaluronic acid alone, in clinical trials and experiments in osteoarthritis: a systematic review and meta-analysis}. BMC Musculoskeletal Disorders.

\bibitem{wang2022wt6}
Zifeng Wang, Chufan Gao, Lucas Glass, et al. (2022). \textit{Artificial Intelligence for In Silico Clinical Trials: A Review}. arXiv.org.

\bibitem{topole2022zhz}
E. Topole, S. Biondaro, I. Montagna, et al. (2022). \textit{Artificial intelligence based software facilitates spirometry quality control in asthma and COPD clinical trials}. ERJ Open Research.

\bibitem{hayashi20195gu}
D. Hayashi, F. Roemer, and A. Guermazi (2019). \textit{Magnetic resonance imaging assessment of knee osteoarthritis: current and developing new concepts and techniques.}. Clinical and Experimental Rheumatology.

\bibitem{sheng2021kna}
Y. Sheng, Jiahan Zhang, Y. Ge, et al. (2021). \textit{Artificial intelligence applications in intensity modulated radiation treatment planning: an overview.}. Quantitative Imaging in Medicine and Surgery.

\bibitem{lee202031d}
Young Kyun Lee, Eun Gyeong Lee, Ha Young Kim, et al. (2020). \textit{Osteoporotic Fractures of the Spine, Hip, and Other Locations after Adjuvant Endocrine Therapy with Aromatase Inhibitors in Breast Cancer Patients: a Meta-analysis}. Journal of Korean medical science.

\bibitem{kumar2020yow}
Akshara Kumar, Shivaprasad Gadag, and U. Nayak (2020). \textit{The Beginning of a New Era: Artificial Intelligence in Healthcare}. Advanced Pharmaceutical Bulletin.

\bibitem{artigals2015es5}
O. Artigalás, T. Vanni, M. Hutz, et al. (2015). \textit{Influence of CYP19A1 polymorphisms on the treatment of breast cancer with aromatase inhibitors: a systematic review and meta-analysis}. BMC Medicine.

\bibitem{neuner2015a1p}
J. Neuner, S. Kamaraju, J. Charlson, et al. (2015). \textit{The introduction of generic aromatase inhibitors and treatment adherence among Medicare D enrollees.}. Journal of the National Cancer Institute.

\bibitem{li2018l6q}
Li Li, Bingmei Chang, Xiaoyue Jiang, et al. (2018). \textit{Clinical outcomes comparison of 10 years versus 5 years of adjuvant endocrine therapy in patients with early breast cancer}. BMC Cancer.

\bibitem{spratt2022maa}
D. Spratt, Yilun Sun, Douwe van der Wal, et al. (2022). \textit{An AI-derived digital pathology-based biomarker to predict the benefit of androgen deprivation therapy in localized prostate cancer with validation in NRG/RTOG 9408.}. Journal of Clinical Oncology.

\bibitem{shen2021xir}
Xinpeng Shen, Sisi Ma, P. Vemuri, et al. (2021). \textit{A novel method for causal structure discovery from EHR data and its application to type-2 diabetes mellitus}. Scientific Reports.

\bibitem{bell2017m5l}
R. Bell, Farwa Rizvi, R. M. Islam, et al. (2017). \textit{A systematic review of intravaginal testosterone for the treatment of vulvovaginal atrophy}. Menopause.

\bibitem{spagnolo20163fv}
F. Spagnolo, I. Šestak, A. Howell, et al. (2016). \textit{Anastrozole-Induced Carpal Tunnel Syndrome: Results From the International Breast Cancer Intervention Study II Prevention Trial.}. Journal of Clinical Oncology.

\bibitem{zdemir20194qo}
V. Özdemir (2019). \textit{The Big Picture on the "AI Turn" for Digital Health: The Internet of Things and Cyber-Physical Systems.}. Omics.

\bibitem{gieraerts2020j5j}
C. Gieraerts, A. Dangis, L. Janssen, et al. (2020). \textit{Prognostic Value and Reproducibility of AI-assisted Analysis of Lung Involvement in COVID-19 on Low-Dose Submillisievert Chest CT: Sample Size Implications for Clinical Trials}. Radiology: Cardiothoracic Imaging.

\bibitem{sessa20204mo}
Maurizio Sessa, Abdul Rauf Khan, David Liang, et al. (2020). \textit{Artificial Intelligence in Pharmacoepidemiology: A Systematic Review. Part 1—Overview of Knowledge Discovery Techniques in Artificial Intelligence}. Frontiers in Pharmacology.

\bibitem{purwono2021rkp}
Purwono Purwono, Anggit Wirasto, and Khoirun Nisa (2021). \textit{Comparison of Machine Learning Algorithms for Classification of Drug Groups}. SISFOTENIKA.

\bibitem{an20228aq}
G. An (2022). \textit{Specialty Grand Challenge: What it Will Take to Cross the Valley of Death: Translational Systems Biology, “True” Precision Medicine, Medical Digital Twins, Artificial Intelligence and In Silico Clinical Trials}. Frontiers in Systems Biology.

\bibitem{diaw2022heb}
M. Diaw, Stéphane Papelier, Alexandre Durand-Salmon, et al. (2022). \textit{AI-Assisted QT Measurements for Highly Automated Drug Safety Studies}. IEEE Transactions on Biomedical Engineering.

\bibitem{mura2022g5q}
A. Mura, Martina Maier, B. Ballester, et al. (2022). \textit{Bringing rehabilitation home with an e-health platform to treat stroke patients: study protocol of a randomized clinical trial (RGS@home)}. Trials.

\bibitem{dipnall2021x37}
Joanna F. Dipnall, R. Page, Lan Du, et al. (2021). \textit{Predicting fracture outcomes from clinical registry data using artificial intelligence supplemented models for evidence-informed treatment (PRAISE) study protocol}. PLoS ONE.

\bibitem{desouza2021rh2}
Elmar Luis Adrian P. Nandita M. Kotter Marti-Bonmati Brady Desouza, E. Kotter, L. Martí-Bonmatí, et al. (2021). \textit{ESR white paper: blockchain and medical imaging}. Insights into Imaging.

\bibitem{santhanam2019akw}
P. Santhanam, and R. Ahima (2019). \textit{Machine learning and blood pressure}. The Journal of Clinical Hypertension.

\bibitem{kundavaram2018ii1}
RamMohan Reddy Kundavaram, Kawsher Rahman, Krishna Devarapu, et al. (2018). \textit{Predictive Analytics and Generative AI for Optimizing Cer-vical and Breast Cancer Outcomes: A Data-Centric Approach}. ABC Research Alert.

\bibitem{cesario2021xt5}
A. Cesario, Irene Simone, I. Paris, et al. (2021). \textit{Development of a Digital Research Assistant for the Management of Patients’ Enrollment in Oncology Clinical Trials within a Research Hospital}. Journal of Personalized Medicine.

\bibitem{chen2015hn5}
Xia-huan Chen, Bo Huang, Meilin Liu, et al. (2015). \textit{Effects of different types of antihypertensive agents on arterial stiffness: a systematic review and meta-analysis of randomized controlled trials.}. Journal of Thoracic Disease.

\bibitem{chiara2021xml}
F. De Chiara, Ainhoa Ferret-Miñana, and J. Ramón‐Azcón (2021). \textit{The Synergy between Organ-on-a-Chip and Artificial Intelligence for the Study of NAFLD: From Basic Science to Clinical Research}. Biomedicines.

\bibitem{namba2017qfu}
Sayaka Namba, M. Yamaoka-Tojo, Ryota Kakizaki, et al. (2017). \textit{Erratum to: Effects on bone metabolism markers and arterial stiffness by switching to rivaroxaban from warfarin in patients with atrial fibrillation}. Heart and Vessels.

\bibitem{chen2019bs7}
Chong-xiang Chen, Tianmeng Wen, and W. Liao (2019). \textit{Neurally adjusted ventilatory assist versus pressure support ventilation in patient-ventilator interaction and clinical outcomes: a meta-analysis of clinical trials.}. Annals of Translational Medicine.

\bibitem{leow2020fh0}
W. Leow, P. Bedossa, Feng Liu, et al. (2020). \textit{An Improved qFibrosis Algorithm for Precise Screening and Enrollment into Non-Alcoholic Steatohepatitis (NASH) Clinical Trials}. Diagnostics.

\bibitem{sulaica20168a5}
Elisabeth M. Sulaica, T. Han, Wei-qun Wang, et al. (2016). \textit{Vaginal estrogen products in hormone receptor-positive breast cancer patients on aromatase inhibitor therapy}. Breast Cancer Research and Treatment.

\bibitem{liu2021lc8}
Guina Liu, Nian Li, Lingmin Chen, et al. (2021). \textit{Registered Trials on Artificial Intelligence Conducted in Emergency Department and Intensive Care Unit: A Cross-Sectional Study on ClinicalTrials.gov}. Frontiers in Medicine.

\bibitem{sun2021hte}
Chuihua Sun, Fang Dong, Ting Xiao, et al. (2021). \textit{Efficacy and safety of Chinese patent medicine (Kang-ai injection) as an adjuvant in the treatment of patients with hepatocellular carcinoma: a meta-analysis}. Pharmaceutical Biology.

\bibitem{sang2021cz3}
Shengtian Sang, Ran Sun, Jean Coquet, et al. (2021). \textit{Learning From Past Respiratory Infections to Predict COVID-19 Outcomes: Retrospective Study}. Journal of Medical Internet Research.

\bibitem{matsuoka2014a7d}
Y. Matsuoka, Amorsolo L. Suguitan, M. Orandle, et al. (2014). \textit{African Green Monkeys Recapitulate the Clinical Experience with Replication of Live Attenuated Pandemic Influenza Virus Vaccine Candidates}. Journal of Virology.

\bibitem{jain2020pb0}
K. Jain (2020). \textit{Artificial Intelligence Applications in handling the Infectious Diseases}. Unpublished manuscript.

\bibitem{madonna2021zbo}
G. Madonna, G. Masucci, M. Capone, et al. (2021). \textit{Clinical Categorization Algorithm (CLICAL) and Machine Learning Approach (SRF-CLICAL) to Predict Clinical Benefit to Immunotherapy in Metastatic Melanoma Patients: Real-World Evidence from the Istituto Nazionale Tumori IRCCS Fondazione Pascale, Napoli, Italy}. Cancers.

\bibitem{quazi2021qvl}
Sameer Quazi, and Rohit Jangi (2021). \textit{Artificial Intelligence and Machine Learning in Medicinal Chemistry and Validation of Emerging Drug Targets}. Unpublished manuscript.

\bibitem{kumar2021uzf}
Rajnish Kumar, Farhat Ullah Khan, Anju Sharma, et al. (2021). \textit{Recent Applications of Artificial Intelligence in detection of Gastrointestinal, Hepatic and Pancreatic Diseases.}. Current Medicinal Chemistry.

\bibitem{vries2021ne4}
C. D. de Vries, B. Morrissey, Donna Duggan, et al. (2021). \textit{Screening participants’ attitudes to the introduction of artificial intelligence in breast screening}. Journal of Medical Screening.

\bibitem{kelsall2020l2x}
A. Kelsall, A. Iqbal, and J. Newell-Price (2020). \textit{Adrenal incidentaloma: cardiovascular and metabolic effects of mild cortisol excess.}. Gland surgery.

\bibitem{riemsma2012ze6}
R. Riemsma, C. Forbes, M. Amonkar, et al. (2012). \textit{Systematic review of lapatinib in combination with letrozole compared with other first-line treatments for hormone receptor positive(HR+) and HER2+ advanced or metastatic breast cancer(MBC)}. Current Medical Research and Opinion.

\bibitem{kalaiselvan2020mu9}
V. Kalaiselvan, Ashish Sharma, and S. Gupta (2020). \textit{“Feasibility test and application of AI in healthcare”—with special emphasis in clinical, pharmacovigilance, and regulatory practices}. Health technology.

\bibitem{ahmed20202nf}
Alim Al Ayub Ahmed, and Praveen Kumar Donepudi (2020). \textit{Artificial Intelligence in Clinical Genomics and Healthcare}. Unpublished manuscript.

\bibitem{tabrizi2018uml}
R. Tabrizi, S. Vakili, K. Lankarani, et al. (2018). \textit{The Effects of Vitamin D Supplementation on Markers Related to Endothelial Function Among Patients with Metabolic Syndrome and Related Disorders: A Systematic Review and Meta-Analysis of Clinical Trials}. Hormone and Metabolic Research.

\bibitem{mcneill2010snz}
E. McNeill (2010). \textit{RVX-208, a stimulator of apolipoprotein AI gene expression for the treatment of cardiovascular diseases.}. Current opinion in investigational drugs.

\bibitem{paper2020aok}
Unknown Authors (2020). \textit{Setting guidelines to report the use of AI in clinical trials}. Nature Network Boston.

\bibitem{regan2014mwf}
M. Regan, B. Walley, G. Fleming, et al. (2014). \textit{Randomized comparison of adjuvant aromatase inhibitor (AI) exemestane (E) plus ovarian function suppression (OFS) vs tamoxifen (T) plus OFS in premenopausal women with hormone receptor-positive (HR+) early breast cancer (BC): Joint analysis of IBCSG TEXT and SOFT trials.}. Journal of Clinical Oncology.

\bibitem{cong201543v}
Yan Cong, Ke-fu Sun, Xueming He, et al. (2015). \textit{A Traditional Chinese Medicine Xiao-Ai-Tong Suppresses Pain through Modulation of Cytokines and Prevents Adverse Reactions of Morphine Treatment in Bone Cancer Pain Patients}. Mediators of Inflammation.

\bibitem{eiger2020sfb}
D. Eiger, M. Wagner, N. Pondé, et al. (2020). \textit{The impact of cyclin-dependent kinase 4 and 6 inhibitors (CDK4/6i) on the incidence of alopecia in patients with metastatic breast cancer (BC)}. Acta oncologica.

\bibitem{wollina2020pw3}
U. Wollina, P. Brzeziński, A. Koch, et al. (2020). \textit{Immunomodulatory drugs alone and adjuvant to surgery for hidradenitis suppurativa/acne inversa—A narrative review}. Dermatologic Therapy.

\bibitem{hershman20152ik}
D. Hershman, C. Loprinzi, and B. Schneider (2015). \textit{Symptoms: Aromatase Inhibitor Induced Arthralgias.}. Advances in Experimental Medicine and Biology.

\bibitem{anand2019t7e}
K. Anand, and P. Niravath (2019). \textit{Acupuncture and Vitamin D for the Management of Aromatase Inhibitor-Induced Arthralgia}. Current Oncology Reports.

\bibitem{djalalov2015y6w}
S. Djalalov, J. Beca, E. Amir, et al. (2015). \textit{Economic evaluation of hormonal therapies for postmenopausal women with estrogen receptor-positive early breast cancer in Canada.}. Current Oncology.

\bibitem{gradishar2017rfg}
W. Gradishar, R. Hegg, S. Im, et al. (2017). \textit{Phase III study of lapatinib (L) plus trastuzumab (T) and aromatase inhibitor (AI) vs T+AI vs L+AI in postmenopausal women (PMW) with HER2+, HR+ metastatic breast cancer (MBC): ALTERNATIVE.}. Unpublished manuscript.

\bibitem{roozenbeek2016p36}
B. Roozenbeek, H. Lingsma, F. Lecky, et al. (2016). \textit{Prediction of outcome after moderate and severe traumatic brain injury}. Unpublished manuscript.

\bibitem{wilson2013clv}
Sheridan Wilson, and S. Chia (2013). \textit{Treatment algorithms for hormone receptor-positive advanced breast cancer: applying the results from recent clinical trials into daily practice—insights, limitations, and moving forward.}. American Society of Clinical Oncology educational book. American Society of Clinical Oncology. Annual Meeting.

\bibitem{yuen2017l9z}
K. Yuen, A. Moraitis, and D. Nguyen (2017). \textit{Evaluation of Evidence of Adrenal Insufficiency in Trials of Normocortisolemic Patients Treated With Mifepristone}. Journal of the Endocrine Society.

\bibitem{tiwari2015qi2}
T. Tiwari, T. Sharma, M. Harper, et al. (2015). \textit{Community Based Participatory Research to Reduce Oral Health Disparities in American Indian Children.}. Journal of Family Medicine.

\bibitem{chekroud2024bvp}
Adam M. Chekroud, Matt Hawrilenko, Hieronimus Loho, et al. (2024). \textit{Illusory generalizability of clinical prediction models}. Science.

\bibitem{joshi2024ajq}
Geeta Joshi, Aditi Jain, Shalini Reddy Araveeti, et al. (2024). \textit{FDA-Approved Artificial Intelligence and Machine Learning (AI/ML)-Enabled Medical Devices: An Updated Landscape}. Electronics.

\bibitem{hutson2024frs}
Matthew Hutson (2024). \textit{How AI is being used to accelerate clinical trials.}. Nature.

\bibitem{askin2023wrv}
Scott Askin, Denis Burkhalter, Gilda Calado, et al. (2023). \textit{Artificial Intelligence Applied to clinical trials: opportunities and challenges}. Health technology.

\bibitem{jayatunga20242z7}
Madura KP Jayatunga, Margaret Ayers, L. Bruens, et al. (2024). \textit{How successful are AI-discovered drugs in clinical trials? A first analysis and emerging lessons.}. Drug Discovery Today.

\bibitem{jullien2024flu}
Mael Jullien, Marco Valentino, and André Freitas (2024). \textit{SemEval-2024 Task 2: Safe Biomedical Natural Language Inference for Clinical Trials}. International Workshop on Semantic Evaluation.

\bibitem{bordukova2023u68}
Maria Bordukova, Nikita Makarov, Raul Rodriguez-Esteban, et al. (2023). \textit{Generative artificial intelligence empowers digital twins in drug discovery and clinical trials}. Expert Opinion on Drug Discovery.

\bibitem{zushin2023jtl}
Peter-James H. Zushin, S. Mukherjee, and Joseph C. Wu (2023). \textit{FDA Modernization Act 2.0: transitioning beyond animal models with human cells, organoids, and AI/ML-based approaches}. Journal of Clinical Investigation.

\bibitem{anuyah2024iap}
Sydney Anuyah, Mallika K Singh, and Hope Nyavor (2024). \textit{Advancing clinical trial outcomes using deep learning and predictive modelling: bridging precision medicine and patient-centered care}. World Journal of Advanced Research and Reviews.

\bibitem{arnold2023k7t}
Carrie Arnold (2023). \textit{Inside the nascent industry of AI-designed drugs}. Nature Network Boston.

\bibitem{mirakhori20259no}
Fahimeh Mirakhori, and Sarfaraz K. Niazi (2025). \textit{Harnessing the AI/ML in Drug and Biological Products Discovery and Development: The Regulatory Perspective}. Pharmaceuticals.

\bibitem{chopra2023jzf}
Hitesh Chopra, Annu, Dong Kil Shin, et al. (2023). \textit{Revolutionizing clinical trials: the role of AI in accelerating medical breakthroughs}. International Journal of Surgery.

\bibitem{peng2023su9}
Yifan Peng, Justin F. Rousseau, E. Shortliffe, et al. (2023). \textit{AI-generated text may have a role in evidence-based medicine}. Nature Network Boston.

\bibitem{ibikunle2024wb1}
Olumide Emmanuel Ibikunle, Precious Azino Usuemerai, Luqman Adewale Abass, et al. (2024). \textit{AI and digital health innovation in pharmaceutical development}. Computer Science &amp; IT Research Journal.

\bibitem{vidovszky2024jtm}
Anna A Vidovszky, Charles K Fisher, A. Loukianov, et al. (2024). \textit{Increasing acceptance of AI‐generated digital twins through clinical trial applications}. Clinical and Translational Science.

\bibitem{chen2024q7u}
Jintai Chen, Yaojun Hu, Yue Wang, et al. (2024). \textit{TrialBench: Multi-Modal Artificial Intelligence-Ready Clinical Trial Datasets}. arXiv.org.

\bibitem{iyer2024v43}
Janani S. Iyer, Dinkar Juyal, Quang Le, et al. (2024). \textit{AI-based automation of enrollment criteria and endpoint assessment in clinical trials in liver diseases}. Nature Network Boston.

\bibitem{youssef2024fn7}
Alaa Youssef, Ariadne A. Nichol, Nicole Martinez-Martin, et al. (2024). \textit{Ethical Considerations in the Design and Conduct of Clinical Trials of Artificial Intelligence}. JAMA Network Open.

\bibitem{sande20248hm}
Davy van de Sande, Eline Fung Fen Chung, J. Oosterhoff, et al. (2024). \textit{To warrant clinical adoption AI models require a multi-faceted implementation evaluation}. npj Digit. Medicine.

\bibitem{idoko202477y}
Idoko Peter Idoko, Abisinuola David-Olusa, Sandra Gyamfuaa Badu, et al. (2024). \textit{The dual impact of AI and renewable energy in enhancing medicine for better diagnostics, drug discovery, and public health}. Magna Scientia Advanced Biology and Pharmacy.

\bibitem{zhai2023kzu}
Kevin Zhai, Mohammad S Yousef, Sawsan G Mohammed, et al. (2023). \textit{Optimizing Clinical Workflow Using Precision Medicine and Advanced Data Analytics}. Processes.

\bibitem{sidiq2023692}
Mohammad Sidiq, Aksh Chahal, Sachin Gupta, et al. (2023). \textit{Advancement, utilization, and future outlook of Artificial Intelligence for physiotherapy clinical trials in India: An overview}. Interdisciplinary Rehabilitation / Rehabilitacion Interdisciplinaria.

\bibitem{sedano2025zjg}
R. Sedano, V. Solitano, S. Vuyyuru, et al. (2025). \textit{Artificial intelligence to revolutionize IBD clinical trials: a comprehensive review}. Therapeutic Advances in Gastroenterology.

\bibitem{lyu2024dm9}
Yu-Xuan Lyu, Qiang Fu, Dominika Wilczok, et al. (2024). \textit{Longevity biotechnology: bridging AI, biomarkers, geroscience and clinical applications for healthy longevity}. Aging.

\bibitem{zhang2023awf}
Bin Zhang, Lu Zhang, Qiuying Chen, et al. (2023). \textit{Harnessing artificial intelligence to improve clinical trial design}. Communications Medicine.

\bibitem{paper20235wg}
Unknown Authors (2023). \textit{AI’s potential to accelerate drug discovery needs a reality check}. Nature.

\bibitem{boverhof2024izx}
B.J. Boverhof, W. Redekop, Daniel Bos, et al. (2024). \textit{Radiology AI Deployment and Assessment Rubric (RADAR) to bring value-based AI into radiological practice}. Insights into Imaging.

\bibitem{gao2023f2n}
Zhenxiang Gao, T. Winhusen, Maria P Gorenflo, et al. (2023). \textit{Repurposing ketamine to treat cocaine use disorder: integration of artificial intelligence-based prediction, expert evaluation, clinical corroboration and mechanism of action analyses}. Addiction.

\bibitem{miller2023ok0}
Matthew I. Miller, L. Shih, and V. Kolachalama (2023). \textit{Machine Learning in Clinical Trials: A Primer with Applications to Neurology}. Neurotherapeutics.

\bibitem{ismail20233wp}
Abdalah Ismail, Talha Al-Zoubi, I. E. El Naqa, et al. (2023). \textit{The role of artificial intelligence in hastening time to recruitment in clinical trials}. BJR|Open.

\bibitem{macheka2024o73}
Sheba Macheka, Peng Yun Ng, Ophira Ginsburg, et al. (2024). \textit{Prospective evaluation of artificial intelligence (AI) applications for use in cancer pathways following diagnosis: a systematic review}. BMJ Oncology.

\bibitem{ahmad2023kwk}
H. Ahmad, J. East, R. Panaccione, et al. (2023). \textit{Artificial Intelligence in Inflammatory Bowel Disease Endoscopy: Implications for Clinical Trials}. Journal of Crohn's & Colitis.

\bibitem{kwong20242pu}
J. Kwong, Jeremy Wu, Shamir Malik, et al. (2024). \textit{Predicting non-muscle invasive bladder cancer outcomes using artificial intelligence: a systematic review using APPRAISE-AI}. npj Digit. Medicine.

\bibitem{woelfle2024q61}
Tim Woelfle, J. Hirt, P. Janiaud, et al. (2024). \textit{Benchmarking Human-AI Collaboration for Common Evidence Appraisal Tools}. medRxiv.

\bibitem{mohapatra20247wu}
Manmayee Mohapatra, Chittaranjan Sahu, and Snehamayee Mohapatra (2024). \textit{Trends of Artificial Intelligence (AI) Use in Drug Targets, Discovery and Development: Current Status and Future Perspectives.}. Current Drug Targets.

\bibitem{wu2024jyd}
Yuyuan Wu, Lijing Ma, Xinyi Li, et al. (2024). \textit{The role of artificial intelligence in drug screening, drug design, and clinical trials}. Frontiers in Pharmacology.

\bibitem{landman2024w8r}
Rogier Landman, Sean P. Healey, Vittorio Loprinzo, et al. (2024). \textit{Using large language models for safety-related table summarization in clinical study reports}. JAMIA Open.

\bibitem{wei2023vll}
M. Wei, Shmuel Fay, Diana Yung, et al. (2023). \textit{Artificial Intelligence–Assisted Colonoscopy in Real-World Clinical Practice: A Systematic Review and Meta-Analysis}. Clinical and Translational Gastroenterology.

\bibitem{angelucci2024f3h}
Francesco Angelucci, Alice Ruixue Ai, Lydia Piendel, et al. (2024). \textit{Integrating AI in fighting advancing Alzheimer: diagnosis, prevention, treatment, monitoring, mechanisms, and clinical trials.}. Current Opinion in Structural Biology.

\bibitem{lu2024huv}
Xiaoran Lu, Chen Yang, Lu Liang, et al. (2024). \textit{Artificial intelligence for optimizing recruitment and retention in clinical trials: a scoping review}. J. Am. Medical Informatics Assoc..

\bibitem{gkintoni2025um8}
E. Gkintoni, S. Vassilopoulos, Georgios Nikolaou, et al. (2025). \textit{Digital and AI-Enhanced Cognitive Behavioral Therapy for Insomnia: Neurocognitive Mechanisms and Clinical Outcomes}. Journal of Clinical Medicine.

\bibitem{sohail2023cis}
S. Sohail, D. Madsen, Faiza Farhat, et al. (2023). \textit{ChatGPT and Vaccines: Can AI Chatbots Boost Awareness and Uptake?}. Annals of Biomedical Engineering.

\bibitem{han2024xn5}
Yu Han, and Jingwen Tao (2024). \textit{Revolutionizing Pharma: Unveiling the AI and LLM Trends in the Pharmaceutical Industry}. arXiv.org.

\bibitem{saranraj2024e2y}
K. Saranraj, and P. U. Kiran (2024). \textit{Drug repurposing: Clinical practices and regulatory pathways}. Perspectives in Clinical Research.

\bibitem{xu2025xbx}
Zu-shan Xu, Fengzhi Ren, Ping Wang, et al. (2025). \textit{A generative AI-discovered TNIK inhibitor for idiopathic pulmonary fibrosis: a randomized phase 2a trial}. Nature Network Boston.

\bibitem{zhou2025tn5}
Liangbin Zhou, Shangsi Chen, Jun Liu, et al. (2025). \textit{When artificial intelligence (AI) meets organoids and organs-on-chips (OoCs): Game-changer for drug discovery and development?}. The Innovation Life.

\bibitem{ryan20232by}
David K Ryan, R. Maclean, Alfred Balston, et al. (2023). \textit{AI and machine learning for clinical pharmacology.}. British Journal of Clinical Pharmacology.

\bibitem{yildirim2024gle}
Zehra Yildirim, Kyle Swanson, Xuekun Wu, et al. (2024). \textit{Next-Gen Therapeutics: Pioneering Drug Discovery with iPSCs, Genomics, AI, and Clinical Trials in a Dish}. Annual Review of Pharmacology and Toxicology.

\bibitem{perni2023vyk}
S. Perni, L. Lehmann, and D. Bitterman (2023). \textit{Patients should be informed when AI systems are used in clinical trials}. Nature Network Boston.

\bibitem{olaoluawa2024lb0}
Opeyemi Olaoluawa, Ojo, Opeyemi Olaoluawa Ojo, et al. (2024). \textit{Integrating predictive analytics in clinical trials: A paradigm shift in personalized medicine}. World Journal of Biology Pharmacy and Health Sciences.

\bibitem{choradia2024q0s}
Nirmal Choradia, F. Karzai, Ryan D. Nipp, et al. (2024). \textit{Increasing diversity in clinical trials: Demographic trends at the national cancer institute, 2005-2020.}. Journal of the National Cancer Institute.

\bibitem{okolo20241ld}
Chioma Anthonia Okolo, Tolulope O Olorunsogo, and Oloruntoba Babawarun (2024). \textit{A comprehensive review of AI applications in personalized medicine}. International Journal of Science and Research Archive.

\bibitem{mainous2023jbz}
A. Mainous, Allison Kelliher, and Donald Warne (2023). \textit{Recruiting Indigenous Patients Into Clinical Trials: A Circle of Trust}. Annals of Family Medicine.

\bibitem{nagai2023tjk}
Mizuki Nagai, Sho Suzuki, Y. Minato, et al. (2023). \textit{Detecting colorectal lesions with image-enhanced endoscopy: an updated review from clinical trials}. Clinical Endoscopy.

\bibitem{li2023c3m}
Yun Li, Min Wang, Lu Wang, et al. (2023). \textit{Advances in the Application of AI Robots in Critical Care: Scoping Review}. Journal of Medical Internet Research.

\bibitem{cherubini2023az7}
Andrea Cherubini, and Nhan Ngo Dinh (2023). \textit{A Review of the Technology, Training, and Assessment Methods for the First Real-Time AI-Enhanced Medical Device for Endoscopy}. Bioengineering.

\bibitem{ouyang2024kcv}
D. Ouyang, and Joseph W. Hogan (2024). \textit{We Need More Randomized Clinical Trials of AI}. NEJM AI.

\bibitem{rosenthal2025j23}
Jacob Rosenthal, Ashley Beecy, and M. Sabuncu (2025). \textit{Rethinking clinical trials for medical AI with dynamic deployments of adaptive systems}. npj Digit. Medicine.

\bibitem{calzetta2023kj0}
L. Calzetta, Elena Pistocchini, A. Chetta, et al. (2023). \textit{Experimental drugs in clinical trials for COPD: artificial intelligence via machine learning approach to predict the successful advance from early-stage development to approval}. Expert Opinion on Investigational Drugs.

\bibitem{malheiro2025dq9}
Vera Malheiro, Beatriz Santos, Ana Figueiras, et al. (2025). \textit{The Potential of Artificial Intelligence in Pharmaceutical Innovation: From Drug Discovery to Clinical Trials}. Pharmaceuticals.

\bibitem{ghosh2024t7a}
Satanu Ghosh, H. Abushukair, Arjun Ganesan, et al. (2024). \textit{Harnessing explainable artificial intelligence for patient-to-clinical-trial matching: A proof-of-concept pilot study using phase I oncology trials}. PLoS ONE.

\bibitem{zavaletamonestel2024ri1}
Esteban Zavaleta‐Monestel, Ricardo Quesada-Villaseñor, S. Arguedas-Chacón, et al. (2024). \textit{Revolutionizing Healthcare: Qure.AI's Innovations in Medical Diagnosis and Treatment}. Cureus.

\bibitem{tu2024mk3}
Zhiwei Tu, Youtao Wang, Junze Liang, et al. (2024). \textit{Helicobacter pylori-targeted AI-driven vaccines: a paradigm shift in gastric cancer prevention}. Frontiers in Immunology.

\bibitem{kandhare20253ll}
Priyanka Kandhare, Mrunal Kurlekar, Tanvi Deshpande, et al. (2025). \textit{A Review on Revolutionizing Healthcare Technologies with AI and ML Applications in Pharmaceutical Sciences}. Drugs and Drug Candidates.

\bibitem{garcia20242j1}
B. Garcia, Cristian Marlon de Magalhães Rodrigues Martins, Lucas Faria Porto, et al. (2024). \textit{Accuracy of an AI-based automated plate reading mobile application for the identification of clinical mastitis-causing pathogens in chromogenic culture media}. Scientific Reports.

\bibitem{brbic2024au3}
Maria Brbi´c, Michihiro Yasunaga, Prabhat Agarwal, et al. (2024). \textit{Predicting drug outcome of population via clinical knowledge graph}. medRxiv.

\bibitem{han2023xlz}
MS Ryan Han, MD Julián N. Acosta, PhD Zahra Shakeri, et al. (2023). \textit{Randomized Controlled Trials Evaluating AI in Clinical Practice: A Scoping Evaluation}. medRxiv.

\bibitem{lampreia2024q0o}
Fabio Lampreia, Catarina Madeira, and Hélder Dores (2024). \textit{Digital health technologies and artificial intelligence in cardiovascular clinical trials: A landscape of the European space}. Digital Health.

\bibitem{leiva20256fv}
Víctor Leiva, and Cecília Castro (2025). \textit{Artificial intelligence and blockchain in clinical trials: enhancing data governance efficiency, integrity, and transparency.}. Bioanalysis.

\bibitem{siafakas2024nrx}
Nikolaos Siafakas, and E. Vasarmidi (2024). \textit{Risks of Artificial Intelligence (AI) in Medicine}. Pneumon.

\bibitem{drelick2024s11}
Alicia M. Drelick, Casey Woodfield, and Justin E. Freedman (2024). \textit{Educational chatbot development informed by clinical simulations}. Interactive Learning Environments.

\bibitem{mazor2025cii}
T. Mazor, Karim S Farhat, Pavel Trukhanov, et al. (2025). \textit{Clinical Trial Notifications Triggered by Artificial Intelligence–Detected Cancer Progression}. JAMA Network Open.

\bibitem{armstrong2023dwd}
A. Armstrong, V. Liu, Ramprasaath R. Selvaraju, et al. (2023). \textit{Development and validation of an AI-derived digital pathology-based biomarker to predict benefit of long-term androgen deprivation therapy with radiotherapy in men with localized high-risk prostate cancer across multiple phase III NRG/RTOG trials.}. Journal of Clinical Oncology.

\bibitem{serraburriel2023yxt}
Miquel Serra-Burriel, Luca Locher, and K. N. Vokinger (2023). \textit{Development Pipeline and Geographic Representation of Trials for Artificial Intelligence/Machine Learning-Enabled Medical Devices (2010 to 2023)}. NEJM AI.

\bibitem{flach2023bz8}
R. Flach, N. Stathonikos, Tri Q Nguyen, et al. (2023). \textit{CONFIDENT-trial protocol: a pragmatic template for clinical implementation of artificial intelligence assistance in pathology}. BMJ Open.

\bibitem{ortegapaz20238g4}
L. Ortega‐Paz, Salvatore Giordano, D. Capodanno, et al. (2023). \textit{Clinical Pharmacokinetics and Pharmacodynamics of CSL112}. Clinical Pharmacokinetics.

\bibitem{iyer202316q}
Janani Iyer, Harsha Pokkalla, Charles Biddle-Snead, et al. (2023). \textit{AI-based histologic scoring enables automated and reproducible assessment of enrollment criteria and endpoints in NASH clinical trials}. medRxiv.

\bibitem{k2023m0z}
Poomari Durgar. K, and M. S. Abirami (2023). \textit{AI Clinical Decision Support System (AI-CDSS) for Cardiovascular Diseases}. 2023 International Conference on Computer Science and Emerging Technologies (CSET).

\bibitem{thirunavukarasu2023wg0}
A. J. Thirunavukarasu (2023). \textit{How Can the Clinical Aptitude of AI Assistants Be Assayed?}. Journal of Medical Internet Research.

\bibitem{kastrup2023pao}
Nanna Kastrup, H. H. Bjerregaard, M. Laursen, et al. (2023). \textit{An AI-based patient-specific clinical decision support system for OA patients choosing surgery or not: study protocol for a single-centre, parallel-group, non-inferiority randomised controlled trial}. Trials.

\bibitem{tomaszewski20229r1}
M. Tomaszewski, S. Fan, Alberto L. Garcia, et al. (2022). \textit{AI-Radiomics Can Improve Inclusion Criteria and Clinical Trial Performance}. Tomography.

\bibitem{chorev20230xi}
Michal Chorev, Jonas F. Haderlein, S. Chandra, et al. (2023). \textit{A Multi-Modal AI-Driven Cohort Selection Tool to Predict Suboptimal Non-Responders to Aflibercept Loading-Phase for Neovascular Age-Related Macular Degeneration: PRECISE Study Report 1}. Journal of Clinical Medicine.

\bibitem{smith2022jae}
Erica A Smith, W. Horan, Dominique Demolle, et al. (2022). \textit{Using Artificial Intelligence-based Methods to Address the Placebo Response in Clinical Trials.}. Innovations in Clinical Neuroscience.

\bibitem{parums2021k6f}
D. Parums (2021). \textit{Editorial: Artificial Intelligence (AI) in Clinical Medicine and the 2020 CONSORT-AI Study Guidelines}. Medical Science Monitor.

\bibitem{weng2021fzr}
C. Weng, and James R. Rogers (2021). \textit{AI uses patient data to optimize selection of eligibility criteria for clinical trials}. Nature.

\bibitem{charalambides2021ieu}
M. Charalambides, C. Flohr, P. Bahadoran, et al. (2021). \textit{New international reporting guidelines for clinical trials evaluating effectiveness of artificial intelligence interventions in dermatology: strengthening the SPIRIT of robust trial reporting}. British Journal of Dermatology.

\bibitem{mcgenity202086i}
Clare McGenity, and D. Treanor (2020). \textit{Guidelines for clinical trials using artificial intelligence – SPIRIT‐AI and CONSORT‐AI†}. Journal of Pathology.

\bibitem{hogea2025igs}
Lavinia Hogea, Dana Cătălina Tabugan, Iuliana Costea, et al. (2025). \textit{The Therapeutic Potential of Psychedelics in Treating Substance Use Disorders: A Review of Clinical Trials}. Medicina.

\bibitem{reason20240og}
Tim Reason, J. Langham, and A. Gimblett (2024). \textit{Automated Mass Extraction of Over 680,000 PICOs from Clinical Study Abstracts Using Generative AI: A Proof-of-Concept Study}. Pharmaceutical Medicine.

\bibitem{rahman2025xn9}
Eqram Rahman, Karim Sayed, P. Rao, et al. (2025). \textit{Exosome Revolution or Marketing Mirage? AI-Based Multi-domain Evaluation of Claims, Scientific Evidence, Transparency, Public Sentiment, and Media Narratives.}. Aesthetic Plastic Surgery.

\bibitem{goldberg2024vb1}
Jana M. Goldberg, Nivee P. Amin, Krista A. Zachariah, et al. (2024). \textit{The Introduction of AI Into Decentralized Clinical Trials}. JACC: Advances.

\bibitem{patel2024jpj}
Syed J Patel, Salma Yousuf, Jaswanthi Padala, et al. (2024). \textit{Advancements in Artificial Intelligence for Precision Diagnosis and Treatment of Myocardial Infarction: A Comprehensive Review of Clinical Trials and Randomized Controlled Trials}. Cureus.

\bibitem{ehidiamen202480b}
Anita Jumai Ehidiamen, and O. Oladapo (2024). \textit{Enhancing ethical standards in clinical trials: A deep dive into regulatory compliance, informed consent, and participant rights protection frameworks}. World Journal of Biology Pharmacy and Health Sciences.

\bibitem{liddicoat2025pdu}
J. Liddicoat, Gabriela Lenarczyk, Mateo Aboy, et al. (2025). \textit{A policy framework for leveraging generative AI to address enduring challenges in clinical trials}. npj Digit. Medicine.

\bibitem{bo20259gj}
Y. Bo, Yifei Chen, Hsu Yi Liang, et al. (2025). \textit{AI Perspectives on the Present and Future of Antidepressant Pharmaceutical Treatment Based on Anti-inflammatory Strategies: A Scoping Review of Randomised Controlled Clinical Trials}. medRxiv.

\bibitem{waters2025scl}
Michael Waters (2025). \textit{AI meets informed consent: a new era for clinical trial communication}. JNCI Cancer Spectrum.

\bibitem{golub2025ah4}
Ilana S. Golub, Abhinav R Thummala, T. Morad, et al. (2025). \textit{Artificial Intelligence in Nuclear Cardiac Imaging: Novel Advances, Emerging Techniques, and Recent Clinical Trials}. Journal of Clinical Medicine.

\bibitem{warne202500i}
Donald Warne, Twyla Baker, Michael Burson, et al. (2025). \textit{Barriers and unmet needs related to healthcare for American Indian and Alaska Native communities: improving access to specialty care and clinical trials}. Frontiers in Health Services.

\bibitem{cerami2024ae4}
E. Cerami, Pavel Trukhanov, Morgan A. Paul, et al. (2024). \textit{MatchMiner-AI: An Open-Source Solution for Cancer Clinical Trial Matching}. arXiv.org.

\bibitem{josephthomas2024gpt}
Jiby Joseph-Thomas, Cristina Green, Abhishek Tibrewal, et al. (2024). \textit{A prospective study comparing AI-based clinical trial eligibility screening with traditional EMR-based screening.}. Journal of Clinical Oncology.

\bibitem{shahin2025ixx}
Mohamed H. Shahin, Prashant Desai, N. Terranova, et al. (2025). \textit{AI‐Driven Applications in Clinical Pharmacology and Translational Science: Insights From the ASCPT 2024 AI Preconference}. Clinical and Translational Science.

\bibitem{ramachandran20246ph}
Muthu Ramachandran (2024). \textit{AI and blockchain framework for healthcare applications}. Facta universitatis - series: Electronics and Energetics.

\bibitem{kim2024jg8}
Junhewk Kim, So Yoon Kim, Eun-Ae Kim, et al. (2024). \textit{Developing a Framework for Self-regulatory Governance in Healthcare AI Research: Insights from South Korea}. Asian Bioethics Review.

\bibitem{grsoy2024hl7}
Derin Gürsoy (2024). \textit{Integrating AI with Chemical Engineering for Rapid Biomaterial Development in Health Applications}. Next Frontier For Life Sciences and AI.

\bibitem{sobhani2024s5u}
N. Sobhani, A. D’Angelo, Matteo Pittacolo, et al. (2024). \textit{Future AI Will Most Likely Predict Antibody-Drug Conjugate Response in Oncology: A Review and Expert Opinion}. Cancers.

\bibitem{neehal2024t8d}
Nafis Neehal, Bowen Wang, Shayom Debopadhaya, et al. (2024). \textit{CTBench: A Comprehensive Benchmark for Evaluating Language Model Capabilities in Clinical Trial Design}. arXiv.org.

\bibitem{wang2024s40}
Kunyuan Wang, Hao Cui, Yun Zhu, et al. (2024). \textit{Evaluation of an artificial intelligence-based clinical trial matching system in Chinese patients with hepatocellular carcinoma: a retrospective study}. BMC Cancer.

\bibitem{qin2024i53}
Shane Qin, Bodie Chislett, J. Ischia, et al. (2024). \textit{ChatGPT and generative AI in urology and surgery—A narrative review}. BJUI Compass.

\bibitem{lei20246r2}
I. I. Lei, Ramesh P. Arasaradnam, and Anastasios Koulaouzidis (2024). \textit{Polyp Matching in Colon Capsule Endoscopy: Pioneering CCE-Colonoscopy Integration Towards an AI-Driven Future}. Journal of Clinical Medicine.

\bibitem{yuan20245wo}
Hai-Yang Yuan, Xiao-Fei Tong, Yayun Ren, et al. (2024). \textit{AI‐based digital pathology provides newer insights into lifestyle intervention‐induced fibrosis regression in MASLD: An exploratory study}. Liver international (Print).

\bibitem{yu2024iyb}
Sihan Yu, Yawen Chi, Xiaochun Ma, et al. (2024). \textit{Heparin in sepsis: current clinical findings and possible mechanisms}. Frontiers in Immunology.

\bibitem{singh202459v}
Satneet Singh, Jade L Gambill, Mary Attalla, et al. (2024). \textit{Evaluating the Clinical Validity and Reliability of Artificial Intelligence-Enabled Diagnostic Tools in Neuropsychiatric Disorders}. Cureus.

\bibitem{arefin2025044}
Sabira Arefin, Nushra Tul Zannat, and Global Health Institute Research Team United state (2025). \textit{Securing AI in Global Health Research: A Framework for Cross-Border Data Collaboration}. Clinical Medicine And Health Research Journal.

\bibitem{bosco2025loz}
Cristina Bosco, Fereshtehossadat Shojaei, A. A. Theisz, et al. (2025). \textit{“I don’t see anything specifically about Black/African Americans.” Testing an Alzheimer-specific generative AI tool tailored for African American/Black communities}. ACM Transactions on Computing for Healthcare.

\bibitem{wah2025zvh}
Jack Ng Kok Wah (2025). \textit{The rise of robotics and AI-assisted surgery in modern healthcare}. Journal of Robotic Surgery.

\bibitem{chen2024zvv}
David Chen, C. Cao, R. Kloosterman, et al. (2024). \textit{Trial Factors Associated With Completion of Clinical Trials Evaluating AI: Retrospective Case-Control Study}. Journal of Medical Internet Research.

\bibitem{wilczok2024hg9}
Dominika Wilczok, and Alex Zhavoronkov (2024). \textit{Progress, Pitfalls, and Impact of AI‐Driven Clinical Trials}. Clinical pharmacology and therapy.

\bibitem{yang2024xk7}
Meicheng Yang, Jinqiang Zhuang, Wenhan Hu, et al. (2024). \textit{Enhancing Patient Selection in Sepsis Clinical Trials Design Through an AI Enrichment Strategy: Algorithm Development and Validation}. Journal of Medical Internet Research.

\bibitem{maleki2024hwz}
Morteza Maleki (2024). \textit{Clinical Trials Protocol Authoring using LLMs}. arXiv.org.

\bibitem{fleurence2024vvo}
R. Fleurence, Jiang Bian, Xiaoyan Wang, et al. (2024). \textit{Generative AI for Health Technology Assessment: Opportunities, Challenges, and Policy Considerations}. arXiv.org.

\bibitem{wang20244sw}
Du Wang, H. Geng, V. Gondi, et al. (2024). \textit{Radiotherapy Plan Quality Assurance in NRG Oncology Trials for Brain and Head/Neck Cancers: An AI-Enhanced Knowledge-Based Approach}. Cancers.

\bibitem{dave202400p}
Pallav Dave (2024). \textit{How AI Can Revolutionize the Pharmaceutical Industry}. Journal of Drug Delivery and Therapeutics.

\bibitem{hilling2025qq3}
Denise E. Hilling, Imane Ihaddouchen, Stefan Buijsman, et al. (2025). \textit{The imperative of diversity and equity for the adoption of responsible AI in healthcare}. Frontiers Artif. Intell..

\bibitem{garg2025fay}
Shilpa Garg, Robert Kitchen, Ramneek Gupta, et al. (2025). \textit{Applications of AI in Predicting Drug Responses for Type 2 Diabetes}. JMIR Diabetes.

\bibitem{wang2025pax}
Wei Wang, Nannan Wang, Yiyang Wu, et al. (2025). \textit{An Integrated AI-PBPK Platform for Predicting Drug In Vivo Fate and Tissue Distribution in Human and Inter-Species Extrapolation.}. Clinical pharmacology and therapy.

\end{thebibliography}

\end{document}