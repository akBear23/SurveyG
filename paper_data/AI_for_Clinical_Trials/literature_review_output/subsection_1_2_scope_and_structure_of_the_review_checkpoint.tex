\subsection*{Scope and Structure of the Review}
This literature review provides a comprehensive roadmap, systematically tracing the intellectual evolution of Artificial Intelligence (AI) applications in clinical trials through a structured thematic organization. The review initiates with the foundational landscape and early challenges of AI integration (Section 2), establishing the initial conceptualization and identified hurdles that shaped subsequent research. It then progresses to detailing core AI methodologies for optimizing various operational stages of trials, such as AI-driven patient recruitment and trial design (Section 3), demonstrating AI's utility in addressing long-standing bottlenecks. Building upon these, the review advances to sophisticated AI for data integration and strategic insights (Section 4), encompassing the leveraging of Real-World Evidence, privacy-preserving techniques like federated learning, synthetic data generation, and knowledge graphs for robust predictive modeling. Further, it explores cutting-edge AI paradigms (Section 5), including Large Language Models for documentation and Reinforcement Learning for adaptive trial designs, also examining AI's upstream impact on early drug discovery and development.

Recognizing the imperative for responsible deployment and the complexities of translating AI into clinical value, the review dedicates substantial focus to ensuring trustworthy AI (Section 6). This section addresses critical non-technical dimensions such as ensuring AI fairness and mitigating bias in clinical predictions, the vital role of Explainable AI (XAI) in fostering interpretability and building trust, and the importance of human factors and usability engineering for safe human-AI interaction. This emphasis is particularly pertinent given methodological critiques highlighting the overestimation of clinical benefits in existing AI studies \cite{genin202155z} and the need for robust diagnostic approaches to bias and equitable predictive performance \cite{jayakumar2022sav}. Finally, Section 7 critically examines the frameworks for evaluation, implementation, and regulatory oversight essential for translating AI into clinical practice. This includes the imperative for rigorous empirical assessment of AI interventions, as underscored by meta-research revealing incomplete quality assessment and inconsistent reporting in systematic reviews of AI diagnostic accuracy studies \cite{jayakumar2022sav}. The section also addresses the observed gap in comprehensive implementation evaluations, advocating for multi-faceted approaches beyond mere statistical performance to warrant clinical adoption \cite{sande20248hm}. Furthermore, it details the development and significance of specialized reporting guidelines, such as CONSORT-AI and SPIRIT-AI \cite{chan2020egf}, which aim to enhance transparency and reproducibility. Crucially, it covers the evolving adaptive regulatory frameworks for AI as medical devices, navigating the complexities of continuously learning algorithms and the pressing need for proactive, stakeholder-driven strategies to ensure safety, efficacy, and ethical deployment throughout the AI lifecycle \cite{hamamoto2022gcn, massella2022eix, mirakhori20259no}. This structured approach, progressing from foundational understanding to advanced applications and culminating in critical considerations for responsible integration, provides a comprehensive and analytically coherent understanding of AI's dynamic evolution in clinical trials.