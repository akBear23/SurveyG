\subsection*{Knowledge Graphs for Predictive Modeling}

Predictive modeling in clinical trials is inherently complex, grappling with challenges such as accurately forecasting drug outcomes, stratifying diverse patient populations, and accounting for the vast heterogeneity within human biology. This complexity is compounded by the need to integrate disparate data types—ranging from clinical trial results and patient demographics to intricate biological pathways and chemical properties of drugs. Traditional 'black-box' AI models, while powerful, often lack the transparency and interpretability crucial for clinical decision-making, hindering their generalizability and adoption. Knowledge Graphs (KGs), particularly when combined with advanced AI techniques like geometric deep learning (GNNs), offer a structured, interpretable framework for reasoning over these intricate biomedical relationships, moving towards more transparent and generalizable AI solutions.

The application of KGs in biomedicine has evolved significantly, initially focusing on tasks like drug-target interaction prediction or drug repurposing by representing entities and their relationships as nodes and edges. For instance, early work explored using KGs to identify potential drug-drug interactions (DDIs) by modeling relationships between drugs and other entities like targets and genes. \cite{lin2020ghb} introduced Knowledge Graph Neural Network (KGNN), an end-to-end framework designed to capture high-order structures and semantic relations within KGs for DDI prediction. KGNN learns from the neighborhoods of each entity, integrating local receptive field information with the entity's representation to model long-distance correlations, demonstrating superior performance over classic and state-of-the-art models in this specific task. This highlights the utility of GNNs in leveraging rich neighborhood information within KGs for specific predictive challenges.

A pivotal advancement that extends KG capabilities to broader clinical trial outcomes is presented by PlaNet, a geometric deep learning framework introduced by \cite{brbic2024au3}. PlaNet is designed to predict drug outcomes, including efficacy and adverse events, by leveraging a massive clinical knowledge graph. Its core innovation lies in constructing a heterogeneous KG that integrates clinical trial data (represented as drug, condition, and population triplets) with extensive background biological and chemical knowledge from nine diverse databases. This comprehensive integration allows the model to simultaneously reason over population variability, disease biology, and drug chemistry—a critical enhancement over prior models that often lacked the ability to account for patient-specific factors or generalize across diverse contexts.

The methodology of PlaNet involves an unsupervised self-supervised learning phase to generate general-purpose, low-dimensional embeddings for all entities within the KG, effectively capturing its complex topology and heterogeneity \cite{brbic2024au3}. These pretrained embeddings are then fine-tuned for specific pharmacological tasks, such as predicting survival as an efficacy endpoint or the occurrence of serious adverse events. This approach directly addresses the need for robust drug outcome prediction by providing a context-rich understanding of the factors influencing treatment response. An enhanced version, PlaNetLM, further integrates language models like PubMedBERT, allowing for multi-modal reasoning that fuses structured knowledge with textual information, leading to improved predictive performance.

PlaNet's explicit modeling of population characteristics, derived from clinical trial eligibility criteria, is particularly crucial for patient stratification and understanding population heterogeneity. By estimating the effect of changing populations on trial outcomes, PlaNet offers valuable guidance for designing clinical trials and identifying specific patient subgroups that might benefit most from a particular treatment \cite{brbic2024au3}. This moves beyond simple predictions to provide deeper, context-rich insights into complex biomedical phenomena, fostering a shift towards precision medicine.

While PlaNet demonstrates strong performance, achieving an AUROC of 0.70 for efficacy prediction (with PlaNetLM boosting this by an additional 5\% and outperforming a PubMedBERT baseline by 15\%) \cite{brbic2024au3}, it is important to contextualize these metrics. While promising, the absolute value of 0.70 AUROC, without direct comparison to a wide array of established baselines or competing methods on identical tasks, requires careful interpretation regarding its clinical utility. Nevertheless, its robust generalization capabilities—predicting outcomes for novel drugs and drug combinations not seen during training by leveraging KG similarities—represent a significant step towards more adaptable AI solutions.

The claim of enhanced interpretability with KGs, while conceptually appealing due to their structured nature, warrants a more nuanced discussion, especially when combined with deep learning models. KGs *facilitate* interpretability by providing a traceable path of relationships, allowing researchers to understand *what* entities and relations are involved in a prediction. However, the interpretability of complex GNNs themselves remains a significant research area, as highlighted by \cite{wu2024jyd} in their broader review of AI in drug discovery, noting the "black box" nature of many deep learning approaches. While PlaNet's explicit knowledge representation helps, fully explaining *why* a GNN makes a particular prediction is still challenging. In contrast, studies like \cite{bresso2021fri} explicitly focus on Explainable AI (XAI) for investigating Adverse Drug Reaction (ADR) mechanisms using KG mining, often employing simpler, inherently interpretable models like Decision Trees and Classification Rules. These models, while potentially less powerful in complex prediction tasks than GNNs, offer human-readable explanations that can directly inform the molecular mechanisms behind ADRs, showcasing a different trade-off between predictive power and direct interpretability.

Despite the advancements, the performance of KG-driven models is inherently tied to the quality, completeness, and scale of the underlying KGs and the availability of labeled training data \cite{brbic2024au3}. Future research will need to focus on enriching these KGs with even more granular real-world data, developing more sophisticated multi-modal reasoning techniques, and addressing challenges related to data standardization and interoperability across diverse clinical and biological datasets. Furthermore, developing robust XAI methods specifically tailored for GNNs on biomedical KGs is critical to fully realize their potential for transparent and trustworthy clinical application. The integration of KGs with geometric deep learning represents a transformative trajectory in AI for clinical trials, promising to deliver more interpretable, generalizable, and clinically actionable insights for drug development and personalized medicine, provided these challenges are systematically addressed.