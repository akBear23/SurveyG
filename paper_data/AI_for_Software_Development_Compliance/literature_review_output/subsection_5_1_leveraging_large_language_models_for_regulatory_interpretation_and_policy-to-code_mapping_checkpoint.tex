\subsection*{Leveraging Large Language Models for Regulatory Interpretation and Policy-to-Code Mapping}

The increasing complexity of regulatory landscapes presents a significant challenge for software development, often leading to a substantial gap between legal mandates expressed in natural language and their technical implementation as code-level policies. Bridging this divide necessitates sophisticated tools capable of interpreting nuanced regulatory texts and translating them into actionable software requirements and verifiable code. This subsection explores the transformative role of Large Language Models (LLMs) in addressing this challenge, moving beyond traditional keyword matching to enable semantic reasoning for automated policy-to-code compliance mapping.

Early efforts to automate regulatory compliance leveraged traditional Natural Language Processing (NLP) techniques to extract rules and requirements. For instance, \cite{Chen2020} demonstrated how NLP could be utilized to analyze natural language regulatory documents, identifying and extracting compliance rules that could then be translated into software requirements. This approach was foundational in bridging the initial gap between legal text and technical specifications, laying the groundwork for more advanced semantic understanding. However, traditional NLP often struggled with the inherent ambiguity, context-dependency, and vastness of legal texts, limiting its ability to achieve deep contextual understanding and robust interpretation.

The advent of Large Language Models (LLMs) has marked a significant leap forward in this domain, offering unparalleled capabilities for understanding and generating human language. \cite{Wang2022} showcased the transformative potential of LLMs by demonstrating their ability to interpret complex, natural language regulatory texts and subsequently translate them into actionable software requirements. This work highlighted how LLMs could move beyond superficial keyword matching to grasp the deeper contextual meaning of regulations, thereby enabling a more accurate and comprehensive integration of legal mandates into software engineering processes. Building on this enhanced interpretative power, \cite{Kim2023} further advanced the application of LLMs specifically for automated policy-to-code compliance mapping. Their research illustrated how LLMs could directly map high-level compliance policies to concrete code implementations, effectively generating code-level policies or checks based on the nuanced understanding derived from regulatory input. This capability streamlines the integration of legal mandates into technical specifications, reducing manual effort and potential for human error.

Beyond mere interpretation and mapping, LLMs are also being leveraged for automated compliance assessment and proactive design. \cite{Chen2023} explored the utility of LLMs for automated software compliance assessment, emphasizing their capacity to understand nuanced regulatory language and provide explanations for compliance decisions. This not only enhances the accuracy of assessments but also contributes to the crucial aspect of auditability in regulated environments. Extending this proactive dimension, \cite{Li2024} proposed the use of generative AI, often powered by LLMs, to proactively design software components that are inherently compliant from the outset. This paradigm shift aims to embed compliance "left" in the software development lifecycle, moving from reactive detection of non-compliance to preventative design, thereby minimizing rework and ensuring compliance by design. Similarly, \cite{Gupta2023} explored AI-assisted generation of compliance-aware code, where AI actively guides developers or auto-generates code snippets that inherently adhere to compliance rules, further solidifying the shift towards proactive compliance.

Despite these significant advancements, several challenges remain. The inherent "black box" nature of many LLMs can hinder explainability and auditability, which are paramount in highly regulated industries. While some research like \cite{Chen2023} touches upon explanations, ensuring these are legally sufficient and contextually accurate remains an active area of research. Furthermore, the potential for LLM "hallucinations" or misinterpretations of complex legal jargon necessitates robust validation mechanisms and human-in-the-loop oversight. The computational cost associated with training and deploying large-scale LLMs, as well as the need for continuous adaptation to evolving regulations, also pose practical challenges. Future research must focus on developing more transparent and auditable LLM-based systems, integrating robust validation frameworks, and exploring efficient methods for continuous learning and adaptation to ensure the trustworthiness and scalability of automated policy-to-code mapping solutions.