\subsection{Blockchain for Immutable Compliance Evidence and Traceability}

The increasing reliance on Artificial Intelligence (AI) for automating compliance processes, spanning from proactive monitoring to complex policy interpretation \cite{Wang2019, Chen2020, Kim2023}, necessitates robust mechanisms to ensure the trustworthiness and auditability of these systems. While AI significantly enhances the efficiency and scope of compliance checking, moving beyond foundational automated methods \cite{Siegmund2012}, it also introduces challenges related to the "black box" nature of AI decision-making and the potential for data manipulation. This creates a critical demand for verifiable records and an unalterable chain of custody for compliance artifacts, particularly in highly regulated environments where demonstrable adherence to standards is paramount. Blockchain technology emerges as a powerful solution to address these concerns by providing an immutable and transparent ledger for all compliance-related activities and evidence generated by AI systems, thereby contributing to the broader goal of Trustworthy AI (TAI) \cite{zhang20232xp}.

Blockchain's distributed, cryptographic, and immutable properties are uniquely suited to enhance the integrity and auditability of AI-driven compliance. It offers a secure platform to record every step of the compliance process, from the ingestion of regulatory rules and the configuration of AI models to the generation of compliance evidence and the final decision outcomes. \cite{Li2021} proposed a seminal blockchain-based framework specifically targeting the immutability and traceability of compliance evidence generated by AI systems in software development. This framework leverages blockchain to create an unalterable record of all relevant data, including AI model inputs, outputs, intermediate decisions, and the compliance artifacts produced, ensuring these records cannot be tampered with post-generation.

Extending this, \cite{zhang20232xp} comprehensively surveys how blockchain can make AI trustworthy across its software development lifecycle (SDLC), classifying its contributions into planning, data collection, model development, and system deployment/use. For instance, in the **data collection** stage, blockchain can ensure data transparency, privacy, and accountability by immutably recording data provenance, consent mechanisms, and transformations applied to training datasets. This is crucial for verifying that AI models are trained on compliant, ethically sourced data. During **model development**, blockchain can log AI model versions, configuration parameters, training logs, and validation results, thereby enhancing model transparency, robustness, and fairness \cite{zhang20232xp}. This addresses the critical need for robust documentation and risk assessment in AI lifecycle models, as highlighted by \cite{haakman2020xky}, which identified documentation and model risk assessment as frequently overlooked yet essential stages in regulated environments like fintech. By recording these artifacts on a blockchain, organizations can create an unalterable audit trail for model governance.

The integration of blockchain establishes a verifiable and transparent audit trail, which is paramount for regulatory scrutiny and building confidence in automated compliance systems. By timestamping and cryptographically linking each piece of evidence to the blockchain, it becomes possible to trace the provenance of every compliance assertion back to its origin, ensuring an unalterable chain of custody. Concrete examples of compliance evidence that can be recorded on-chain include:
\begin{itemize}
    \item \textbf{Hashed Code Commits and Software Artifacts}: Cryptographic hashes of source code, configuration files, and deployment scripts, ensuring their integrity.
    \item \textbf{AI Model Metadata}: Version identifiers, training data hashes, hyperparameter settings, and performance metrics of AI models used for compliance checks.
    \item \textbf{Compliance Reports and Decisions}: AI-generated compliance reports, static analysis results, security vulnerability scans, and records of human overrides or approvals.
    \item \textbf{Regulatory Interpretations}: The specific regulatory rules or policies (e.g., GDPR articles) that an AI system was configured to enforce, potentially linked to their natural language interpretations.
\end{itemize}
This not only strengthens accountability for AI-driven compliance decisions but also provides irrefutable proof of adherence to regulations. For instance, if an AI system flags a potential non-compliance or certifies compliance, the underlying data, the AI's reasoning (if logged), and the final determination can all be immutably recorded, offering unparalleled transparency to auditors and stakeholders.

To achieve greater technical depth, the application of **smart contracts** is crucial. Smart contracts can be used to encode compliance rules and automate their execution on the blockchain. For example, a smart contract could automatically verify if a hashed code commit meets predefined security standards or if a dataset's provenance adheres to privacy regulations, recording the outcome immutably. This moves beyond mere data storage to active, automated compliance enforcement. Furthermore, the choice of **blockchain architecture** is critical for enterprise compliance. While public blockchains offer maximum decentralization, permissioned or consortium blockchains (e.g., Hyperledger Fabric) are often preferred in regulated environments due to their controlled access, enhanced privacy features, and higher transaction throughput. These architectures allow organizations to maintain data confidentiality while still leveraging blockchain's immutability for auditability.

While blockchain primarily addresses the integrity and traceability of compliance evidence, it complements other approaches aimed at enhancing trust in AI. For example, Explainable AI (XAI) techniques, as explored in the preceding subsection, focus on providing transparent and understandable reasons for AI's compliance decisions, thereby tackling the "black box" problem from an interpretability perspective. Together, blockchain and XAI offer a comprehensive strategy: blockchain ensures *what* happened is immutably recorded and traceable, while XAI clarifies *why* it happened. This combined approach directly supports the principles of transparency and accountability identified as critical yet often lacking in Responsible AI frameworks \cite{barletta202346k}.

Despite its significant advantages, the practical implementation of blockchain for AI-driven compliance evidence faces several challenges. These include the computational overhead associated with maintaining a distributed ledger, scalability issues when dealing with the vast amounts of data generated by AI systems, and the need for standardized protocols for recording diverse compliance artifacts on-chain. Furthermore, legal and regulatory acceptance of blockchain-based evidence is still evolving, requiring clear guidelines and frameworks. Future research must focus on optimizing blockchain scalability, developing interoperable standards for diverse compliance artifacts, and exploring hybrid architectures that balance on-chain immutability with off-chain data storage for sensitive information, thereby fully realizing its potential in securing AI-driven compliance.