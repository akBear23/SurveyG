\subsection{AI for Security and Vulnerability Compliance}
Leveraging Artificial Intelligence (AI) has become indispensable in enhancing software security and ensuring compliance with increasingly stringent security policies and regulatory frameworks. This subsection delves into AI-driven approaches for proactively identifying, predicting, and mitigating security risks throughout the software development lifecycle, explicitly linking these capabilities to the verification of compliance against established standards and policies.

The foundation for effective AI-driven security analysis relies on extensive and diverse datasets. Large-scale repositories like CodeNet \cite{puri2021d34}, comprising millions of code samples across numerous programming languages with rich metadata, serve as crucial training grounds for AI models. These datasets enable the development of AI systems capable of understanding code semantics, identifying patterns, and ultimately detecting security flaws that violate coding standards or security policies. Building upon such foundations, AI has significantly advanced defect prediction. While traditional methods relied on defect density, modern approaches increasingly utilize machine learning for more granular and accurate predictions, even in large-scale software systems with hundreds of millions of lines of code \cite{pradhan2020jc1}. Generative AI algorithms, for instance, are now being explored for automated test-case generation and bug identification, scrutinizing codebases and execution traces to detect coding mistakes, anomalous patterns, and potential security vulnerabilities \cite{bajaj2023psw}, thereby contributing to compliance with quality and security requirements.

More specifically, AI-driven techniques for vulnerability detection span various stages of the software development process, directly supporting compliance efforts. In Static Application Security Testing (SAST), deep learning models are employed to analyze source code without execution, identifying common weaknesses (CWEs) and insecure coding patterns that contravene organizational security policies or industry best practices like the OWASP Top 10. Advanced approaches include the use of Graph Neural Networks (GNNs) to analyze Code Property Graphs (CPGs). GNNs are particularly effective as they can model complex data and control flows represented in CPGs, enabling the detection of sink-source vulnerabilities that span multiple functions and are often missed by simpler pattern-matching approaches, thus enhancing the ability to verify adherence to secure coding guidelines. Transformer models, often fine-tuned on vast code corpora, are also emerging as powerful tools for identifying insecure code snippets and suggesting secure alternatives, directly aiding developers in writing compliant code. While Large Language Models (LLMs) are increasingly applied to vulnerability detection, their practical utility for industry use is still under scrutiny. A qualitative study by Kholoosi et al. \cite{kholoosi2024mh2} found that while security practitioners perceive LLMs like ChatGPT as beneficial for tasks like vulnerability detection and penetration testing, the actual outputs are often generic and may not be appropriate for real-world industrial application, highlighting a gap between perception and the precision required for verifiable compliance.

Beyond static analysis, AI is crucial for real-time threat detection and integration into modern development workflows, directly supporting continuous compliance. Saleh et al. \cite{saleh2024mrl} present a concrete AI-driven solution for real-time anomaly detection in Continuous Integration/Continuous Deployment (CI/CD) pipelines within cloud environments. Their hybrid Convolutional Neural Network-Long Short-Term Memory (CNN-LSTM) model achieves high accuracy (e.g., 98.69\% on the CSE-CIC-IDS2018 dataset) in identifying various cyberattacks such as DDoS and bot attacks. This capability is vital for proactive risk mitigation and ensuring the integrity of the software supply chain, thereby maintaining continuous compliance with operational security policies and preventing the deployment of compromised artifacts. This aligns with the broader DevSecOps paradigm, where AI-driven security techniques are integrated seamlessly into the workflow to automate security tasks, reduce manual effort, and maintain delivery speed while enhancing security, trust, and efficiency \cite{fu20246t0}. Such integration is fundamental for embedding compliance checks throughout the entire software development lifecycle, moving towards a vision of an AI-driven SDLC where quality and security are continuously assured \cite{terragni20245xq}.

It is important to acknowledge that while AI offers powerful detection capabilities, the increasing use of AI assistants in software development can inadvertently introduce new vulnerabilities. This necessitates even more robust AI-driven detection tools to identify flaws regardless of their origin, ensuring that the benefits of AI-assisted coding are maximized while associated security risks are minimized and compliance is maintained.

In conclusion, AI offers transformative potential for enhancing software security and compliance by providing advanced tools for vulnerability detection, real-time anomaly identification, and seamless integration into DevSecOps pipelines. These AI-driven approaches enable automated checking against specific security policies, standards like CWEs and OWASP, and regulatory frameworks. The literature highlights a dual imperative: developing more sophisticated AI techniques to detect vulnerabilities, including those potentially introduced by AI itself, and ensuring these detection systems provide verifiable evidence for compliance auditing. Future research must focus on bridging the gap between AI's analytical capabilities and the need for verifiable compliance, demanding robust, AI-driven evaluation frameworks and continuous refinement of AI-assisted secure development practices that are explicitly mapped to regulatory requirements.