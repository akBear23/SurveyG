\subsection{AI for Data Privacy and Regulatory Adherence}

Ensuring adherence to complex data privacy regulations and legal mandates, such as the General Data Protection Regulation (GDPR) \cite{amugongo2023vwb} and the California Consumer Privacy Act (CCPA), presents a significant challenge in software development. These regulations impose stringent requirements on data handling, necessitating principles like data minimization, purpose limitation, and robust security measures. Traditionally, compliance has relied heavily on manual effort and specialized legal expertise, leading to high costs, inconsistencies, and a reactive approach to violations. Artificial intelligence offers transformative potential to automate the *detection* of data privacy violations in software artifacts and to facilitate the *interpretation* of regulatory texts to define actionable compliance checks, thereby embedding privacy considerations more effectively into software engineering practices.

Early efforts in AI-driven compliance focused on establishing foundational frameworks and leveraging traditional machine learning (ML) and Natural Language Processing (NLP) for initial detection. \cite{Siddique2019} and \cite{Siddiqui2020} proposed high-level, AI-driven frameworks for continuous compliance checking across the entire Software Development Life Cycle (SDLC), utilizing ML and NLP for early non-compliance detection. Building on this, \cite{Al-Hajj2020} demonstrated a concrete application, employing supervised machine learning techniques such as Support Vector Machines and Naive Bayes to classify software requirements as compliant or non-compliant. This work validated the feasibility of ML for basic compliance assessments by identifying explicit rule violations in textual requirements. Similarly, \cite{liu2022g3w} surveyed the application of AI, particularly NLP and classification, in Requirements Engineering (RE) to improve requirement quality, a methodology directly applicable to identifying privacy-related requirements and detecting their absence or ambiguity. While these foundational works laid the conceptual groundwork for automated detection, they often focused on less complex artifacts or lacked deep technical validation against real-world, intricate privacy regulations.

More advanced AI techniques, particularly deep learning, have been applied to directly detect data privacy violations within source code. For instance, \cite{Li2023} proposed deep learning models, specifically Convolutional Neural Networks (CNNs) and Long Short-Term Memory (LSTMs), for detecting data privacy violations directly in source code. Their models were trained to identify code patterns indicative of non-compliance with regulations like GDPR and CCPA, addressing specific articles related to data handling and user consent. This approach moves beyond high-level document analysis to pinpoint actual implementation flaws, offering a more granular and precise detection capability. However, the effectiveness of such models heavily relies on the quality and representativeness of the training data, and they can struggle with novel attack vectors or highly obfuscated code.

A critical aspect of regulatory adherence is the accurate interpretation of complex legal texts and their translation into technical specifications for detection. The GDPR, as a prime example \cite{amugongo2023vwb}, mandates principles like data minimization and privacy by design, which require careful interpretation to translate into verifiable code-level checks. While the proactive generation of compliant code is discussed elsewhere, AI can assist in *interpreting* these regulations to *derive* the rules and patterns necessary for detection. \cite{Wang2022} demonstrated how Large Language Models (LLMs) can interpret complex, natural language regulatory texts and translate them into actionable software requirements. In the context of detection, this capability is invaluable for automatically extracting compliance rules, identifying relevant data types, and mapping legal obligations to specific code constructs or architectural patterns that an AI detection system can then monitor. This helps bridge the semantic gap between legal and technical domains, enabling the creation of more comprehensive and up-to-date detection rules.

As AI-driven systems become integral to privacy compliance, ensuring the privacy and trustworthiness of these AI tools themselves is paramount. The development of AI models often requires vast amounts of data, which can raise privacy concerns if sensitive code or user data is used for training. \cite{lin20242vi} proposed a decentralized governance framework for open-source AI-based Software Engineering (SE) tools centered on Federated Learning (FL). This approach enables multiple entities to collaboratively train and maintain AI code models, including those for privacy compliance detection, without directly sharing their proprietary local data, thereby preserving privacy during the development of the AI tool itself. Furthermore, the deployed AI models used for compliance checks must also be robust against privacy attacks. \cite{zhu2022oq6} introduced `SafeCompress`, a bi-objective optimized model compression framework that simultaneously optimizes model performance and safety, specifically addressing privacy leakage (e.g., membership inference attacks) in deep learning models. This ensures that the AI models embedded within software, which might handle sensitive data, are themselves privacy-preserving, aligning with the broader goal of trustworthy AI development \cite{fischer2020gef, brundage2020dn4}. The practical integration of ethical considerations, including data privacy, into AI development workflows remains a challenge, as highlighted by gap analyses showing discrepancies between ethical guidelines and industrial practice \cite{vakkuri2022wjr}.

Despite these advancements, challenges persist in AI-driven data privacy detection. Current methods often struggle with the dynamic nature of regulations, the ambiguity inherent in legal language, and the scalability required for large, complex software systems. False positives and negatives remain a concern, requiring significant human oversight. Moreover, while security is foundational for privacy (e.g., GDPR Article 32 mandates "security of processing"), general security compliance tools, such as those for anomaly detection in CI/CD pipelines \cite{saleh2024mrl}, must be explicitly linked to specific privacy principles to be considered direct privacy compliance measures. The "black box" nature of some advanced AI models also complicates auditing and justification of detected violations, posing a barrier to widespread adoption in highly regulated environments.