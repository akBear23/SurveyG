\subsection{Defining AI for Software Development Compliance}
The burgeoning field of Artificial Intelligence (AI) for software development compliance represents a critical and multifaceted domain, which this literature review approaches through a dual lens. At its core, "AI for Software Development Compliance" encapsulates two interconnected yet distinct dimensions: first, the strategic application of AI technologies to enhance and automate compliance-related tasks within the software development lifecycle (SDLC); and second, the imperative to ensure that AI systems themselves are developed and deployed in a compliant, ethical, and trustworthy manner. This dual perspective is foundational to understanding the landscape of research, challenges, and opportunities in this rapidly evolving area.

The first dimension, leveraging AI \textit{for} compliance, focuses on the deployment of advanced AI technologies—including machine learning (ML), natural language processing (NLP), and generative AI (GenAI)—to automate, streamline, and improve the efficiency and accuracy of compliance activities throughout the SDLC. This encompasses a broad spectrum of tasks, such as automated security vulnerability detection in codebases \cite{kang2023d5s}, ensuring adherence to complex regulatory mandates like data privacy laws \cite{chou2023ai}, and enhancing rigorous quality assurance (QA) processes. For instance, AI can analyze vast amounts of code, requirements documents, and design specifications to identify deviations from established security policies, predict potential defects, or even generate test cases that validate regulatory adherence \cite{sridhara20238b9}. The promise of this application lies in transforming traditionally manual, labor-intensive, and often error-prone compliance checks into more proactive, continuous, and scalable processes. By embedding AI-driven checks directly into development workflows, the aim is to foster "compliance by design" and "compliance by default," ensuring that software systems inherently meet specified standards from their inception, thereby reducing the need for costly retrospective remediation \cite{sridhara20238b9, chou2023ai}. This dimension is extensively explored in Sections 4 and 5 of this review, which detail core and advanced AI applications for compliance detection and proactive integration.

The second, equally critical dimension, addresses the need for compliance \textit{of} AI systems themselves. As AI becomes increasingly integral to software development, the systems we build must inherently embody principles of trustworthiness, ethics, and sustainability. This dimension moves beyond merely applying AI to compliance tasks and instead scrutinizes the internal integrity, societal impact, and governance of the AI systems being developed. Key concerns here include ensuring fairness and non-discrimination, where AI models must not perpetuate or amplify societal biases; promoting transparency and explainability, allowing stakeholders to understand how AI decisions are made; safeguarding privacy and ensuring robust data governance for sensitive information processed by AI; and establishing clear accountability for AI system outcomes \cite{brundage2020dn4, yeung20205sw}. Furthermore, the significant energy consumption and carbon footprint associated with training and deploying large AI models have emerged as a critical environmental sustainability compliance concern \cite{wu2021t2c, martinezfernandez2023ipo}. Addressing these issues requires a paradigm shift from abstract ethical principles to concrete, verifiable claims and actionable engineering practices \cite{vakkuri2022wjr}. This dimension necessitates the development of sociotechnical mechanisms, including institutional, software, and hardware solutions, to support external scrutiny, auditing, and accountability throughout the AI system development lifecycle \cite{brundage2020dn4}. The challenges and frameworks for achieving compliance \textit{of} AI systems are the primary focus of Section 3 and are further elaborated in Section 6, which discusses critical considerations for trustworthy AI-driven compliance systems.

In essence, "AI for Software Development Compliance" navigates a dual imperative: harnessing AI's transformative power to enhance the compliance posture of software products, while simultaneously ensuring that the AI systems themselves are constructed and operated in a manner that adheres to stringent ethical, regulatory, and societal standards. This review will explore the literature through this lens, examining both the technical advancements in AI-driven compliance automation and the critical frameworks for building trustworthy and responsible AI systems.