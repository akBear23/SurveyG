\subsection{Environmental Sustainability of AI Systems}

The escalating energy consumption and carbon footprint of Artificial Intelligence (AI) systems pose a significant environmental challenge, necessitating a dedicated focus on sustainability within AI compliance frameworks. Addressing this concern requires comprehensive methodologies for quantifying environmental impact, coupled with architectural and design strategies for building 'green AI' systems.

Early research highlighted the critical need for a holistic understanding of AI's environmental implications. \cite{wu2021t2c} introduced a foundational framework for characterizing AI's carbon footprint, moving beyond isolated model training costs to encompass the entire Machine Learning (ML) development cycle—from data processing and experimentation to training and inference—and the full life cycle of AI system hardware. This work empirically demonstrated that embodied carbon from hardware manufacturing can be a dominating factor, especially when operational emissions are mitigated by carbon-free energy, emphasizing the importance of hardware-software co-design for substantial reductions.

Building upon the imperative for comprehensive quantification, subsequent work focused on developing more granular and actionable measurement tools. \cite{dodge2022uqb} presented a practical framework for measuring Software Carbon Intensity (SCI) for AI workloads in cloud instances, uniquely leveraging real-time, location-based, and time-specific marginal emissions data. This approach allows for more informed decision-making, identifying that choosing optimal geographic regions and even the time of day for computation can significantly reduce operational carbon emissions, thereby empowering ML practitioners to make carbon-aware choices.

While quantifying the problem is crucial, the field has progressed towards embedding sustainability directly into the AI system design and development process. \cite{martnezfernndez2023ipo} proposed an architecture-centric approach, GAISSA, to integrate "greenability" as a first-class concern. This initiative introduces an AI-specific quality model for greenability, predictive models to guide sustainability-aware AI model training, and a catalogue of architecture and design patterns specifically tailored for building green AI-based systems. This moves beyond post-hoc measurement to proactive design, addressing the lack of concrete, actionable methods for implementing energy efficiency throughout the AI engineering lifecycle.

To further operationalize and assess these green practices, metrics are being developed to quantify the adoption of sustainable AI. \cite{sikand2023n63} introduced the "Green AI Quotient" (GAQ) as a novel metric designed to assess the "greenness" of any AI-based system and its development process. This metric aims to bridge the gap between sustainable AI research and its industry-scale adoption, encouraging the integration of energy-efficient AI research and practices into real-world projects.

In summary, the literature demonstrates a clear progression from identifying and holistically quantifying the significant environmental impact of AI systems \cite{wu2021t2c}, to developing granular, real-time measurement tools for cloud environments \cite{dodge2022uqb}, and finally to proposing systematic, architecture-centric design methodologies and metrics for building inherently sustainable AI systems \cite{martnezfernndez2023ipo, sikand2023n63}. Despite these advancements, challenges remain in standardizing metrics, ensuring widespread adoption of green AI practices across diverse industries, and continuously integrating these considerations into the rapidly evolving AI software development lifecycle to meet broader Environmental, Social, and Governance (ESG) compliance requirements. Future research must focus on developing automated tools and robust frameworks that seamlessly embed greenability from conception through deployment and operation, fostering a truly sustainable AI ecosystem.