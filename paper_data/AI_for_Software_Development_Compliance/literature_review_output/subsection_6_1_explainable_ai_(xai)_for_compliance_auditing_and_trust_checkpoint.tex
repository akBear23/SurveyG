\subsection{Explainable AI (XAI) for Compliance Auditing and Trust}

The increasing integration of Artificial Intelligence (AI) into critical software development processes, particularly in highly regulated domains, necessitates a robust solution to the inherent "black box" problem of many AI systems. For AI-driven compliance assessments to achieve widespread adoption and legal sufficiency, they must be not only accurate but also transparent, interpretable, and trustworthy. This imperative drives the application of Explainable AI (XAI) techniques, which aim to provide human-understandable reasons for AI-driven decisions, crucial for regulatory auditing, debugging AI models, and building trust among stakeholders. This subsection explores the role of XAI in achieving these goals, drawing on recent literature that operationalizes trustworthy AI and addresses the practical challenges of integrating ethical principles into the software development lifecycle.

The concept of explainability is a cornerstone of broader frameworks for Trustworthy AI (TAI) and Responsible AI (RAI), moving beyond abstract ethical principles to concrete implementation. \cite{baldassarre2024v2c} introduces POLARIS, a holistic framework that explicitly includes Explainability as one of its four foundational pillars (alongside Privacy, Security, and Fairness). This framework is designed to guide AI practitioners throughout the entire Software Development Life Cycle (SDLC), addressing the critical gap where most existing ethical AI guidelines offer high-level principles but lack actionable strategies or tools for technical implementation. Similarly, \cite{lu2021m0b} conducted an empirical study with AI scientists and engineers, identifying "Transparency & Explainability" as one of the most frequently discussed AI ethics principles, underscoring its practical importance. Their work proposes operationalized patterns to integrate responsible AI considerations, including explainability, across the AI system lifecycle, from requirements engineering to deployment. \cite{sanderson2022zra} further supports this, highlighting transparency and explainability as key ethical aspects in their empirical investigation of implementing responsible AI.

The need for XAI is particularly acute in compliance auditing, where decisions must be legally defensible and auditable. \cite{brundage2020dn4} emphasizes the importance of moving beyond abstract ethical principles to "verifiable claims" about AI systems. They propose a "toolbox" of mechanisms, including software-based solutions like audit trails and interpretability methods, specifically focused on supporting risk assessment and auditing. For AI systems making compliance decisions, interpretability is not merely about understanding the model; it's about providing evidence that the decision aligns with regulatory mandates, organizational policies, and ethical standards. This directly addresses the challenge posed by legal ambiguities and liability concerns for AI errors, as highlighted by \cite{mezrich202243j}, where the lack of clear explanations can hinder accountability and legal recourse.

XAI techniques contribute to building trust by making AI decisions transparent, thereby mitigating concerns about algorithmic bias and unfair outcomes. \cite{truby2020xrk} notes that automated decision-making algorithms have displayed evidence of bias, lack ethical governance, and limit transparency, leading to unfair outcomes. By providing insights into how an AI model arrived at a particular compliance assessment, XAI can help identify and rectify such biases, ensuring fairness and non-discrimination. \cite{cysneiros2020bew} reinforces this by proposing Non-Functional Requirements (NFRs) for socially responsible software, where Transparency is identified as a crucial enabler for Trust and a mitigant for Legal Disputes. In the context of compliance, this means explanations can demonstrate that an AI system adheres to principles like fairness and accountability, which are often implicit or explicit requirements in regulatory frameworks such as GDPR \cite{amugongo2023vwb}.

Despite the clear necessity, the practical implementation of XAI for compliance faces significant challenges. A rapid review of Responsible AI frameworks by \cite{barletta202346k} revealed a critical gap: only a small percentage of frameworks offer practical tools to support RAI implementation, and there is a stark imbalance in SDLC coverage, with most focusing on early requirements elicitation while neglecting later stages like testing and deployment. This indicates a broader deficiency in actionable guidance for integrating explainability throughout the entire software development lifecycle. Furthermore, while XAI methods like LIME (Local Interpretable Model-agnostic Explanations) and SHAP (SHapley Additive exPlanations) can provide local explanations for individual predictions, the question of "explanation sufficiency" for legal and regulatory purposes remains complex. What constitutes an adequate explanation for an auditor or a legal team may differ significantly from what a developer needs for debugging, requiring context-aware and audience-specific explanations. The trade-off between model fidelity and interpretability, the scalability of XAI methods to complex, large-scale AI systems, and the potential for explanations themselves to be misleading or incomplete are ongoing research areas. \cite{vakkuri2022wjr} further highlights a notable gap between high-level AI ethics guidelines and actual industrial practice, suggesting that operationalizing XAI effectively requires overcoming significant organizational and technical hurdles.

In conclusion, XAI is not merely a desirable feature but a critical enabler for the trustworthy and legally sufficient adoption of AI in compliance-critical software development. While frameworks like POLARIS \cite{baldassarre2024v2c} and empirical studies \cite{lu2021m0b, sanderson2022zra} underscore the foundational role of explainability in responsible AI, significant work remains to translate these principles into scalable, actionable, and legally robust XAI techniques. Future research must focus on developing context-aware XAI methods that cater to the diverse needs of stakeholders (developers, auditors, legal teams), providing explanations that are not only technically sound but also legally sufficient and easily auditable. This includes integrating XAI seamlessly into every phase of the SDLC, from design to deployment, to ensure continuous compliance assurance and foster genuine trust in AI-driven compliance systems.