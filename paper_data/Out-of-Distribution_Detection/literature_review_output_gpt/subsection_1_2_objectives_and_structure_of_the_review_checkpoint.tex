\subsection{Objectives and Structure of the Review}

In recent years, Out-of-Distribution (OOD) detection has emerged as a critical area of research due to the increasing deployment of machine learning models in safety-critical applications. This literature review aims to provide a comprehensive examination of the diverse methodologies developed for OOD detection, elucidating their objectives, strengths, and limitations. By systematically analyzing the evolution of these approaches, we seek to clarify the key questions surrounding OOD detection methodologies and their applications, ultimately facilitating a deeper understanding of their implications for future research and practical deployment.

The review is structured to progress from foundational concepts to advanced methodologies and applications, reflecting the trajectory of research in this domain. Initially, we explore the foundational techniques that have shaped OOD detection. For instance, the work by Zhang et al. (2021) introduces Mixture Outlier Exposure (MixOE), which enhances OOD detection in fine-grained environments by generating virtual outliers through mixing in-distribution (ID) and auxiliary outlier data \cite{zhang20212tb}. This method highlights the importance of leveraging diverse data sources to improve model robustness. However, it also exposes the limitations of relying on auxiliary data, as noted by Bitterwolf et al. (2022), who argue that many OOD detection methods converge to similar core principles, indicating a need for more distinct approaches \cite{bitterwolf2022rw0}.

As the field advanced, researchers began to address the challenges posed by specific contexts and data types. Liu et al. (2022) proposed Residual Pattern Learning (RPL) for pixel-wise OOD detection in semantic segmentation, emphasizing the need for models to adapt to varying contexts without compromising in-distribution accuracy \cite{liu2022fdj}. This work illustrates a shift towards context-specific solutions, which are further explored in subsequent studies, such as those by Gao et al. (2023) and Miao et al. (2023), who tackle OOD detection in long-tailed recognition scenarios and domain adaptation settings, respectively \cite{gao2023epm, miao2023brn}. These studies collectively highlight the growing recognition of the need for specialized methodologies that can effectively handle the complexities inherent in real-world data distributions.

The review also delves into representation learning and post-hoc refinement techniques, which have gained traction as researchers seek to enhance OOD detection without relying on auxiliary data. Zhou et al. (2022) critically evaluate reconstruction autoencoder-based methods, proposing a novel approach that rethinks the conditions under which reconstruction error serves as a valid uncertainty measure \cite{zhou202250i}. This line of inquiry is echoed in the work of Peng et al. (2024), who introduce ConjNorm, a tractable density estimation method that offers a unified framework for OOD detection without the need for auxiliary samples \cite{peng20243ji}. Such advancements emphasize the potential of training-free methods that leverage existing model architectures to improve OOD detection capabilities.

In the final sections, we explore the implications of these advancements for future research directions. For instance, the emergence of Vision-Language Models (VLMs) has opened new avenues for OOD detection, as highlighted by the survey conducted by Miyai et al. (2024), which discusses the evolving landscape of OOD detection in the context of VLMs and their applications \cite{miyai20247ro}. This shift underscores the importance of integrating multimodal approaches to enhance OOD detection performance, as evidenced by the work of Dong et al. (2024), which introduces a benchmark for multimodal OOD detection \cite{dong2024a8k}.

In conclusion, while significant progress has been made in OOD detection methodologies, unresolved challenges remain, particularly in the areas of model generalization, the effective use of auxiliary data, and the integration of multimodal information. Future research should focus on developing robust frameworks that can adapt to diverse data distributions and leverage the strengths of various modalities to enhance OOD detection performance in real-world applications.
```