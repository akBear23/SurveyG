\subsection*{Background: The Importance of OOD Detection}

Out-of-Distribution (OOD) detection has emerged as a critical area of research in machine learning and artificial intelligence, particularly due to its implications in safety-critical applications such as autonomous driving and healthcare. In these domains, misclassifications can lead to severe consequences, making the ability to identify OOD samples essential for ensuring the reliability of deployed models. As machine learning systems are increasingly applied in real-world scenarios, the challenge of distinguishing between in-distribution (ID) and OOD samples becomes paramount, necessitating robust detection methods.

Historically, OOD detection methods have evolved significantly, transitioning from early statistical approaches to contemporary deep learning techniques. Initial efforts often relied on simple heuristics such as confidence thresholds or distance metrics in feature space. For instance, Hendrycks and Gimpel \cite{Hendrycks_Gimpel_2017} introduced basic baselines using softmax confidence and Mahalanobis distance on features for OOD detection, laying the groundwork for subsequent developments. However, these methods struggled with the high-dimensional nature of real-world data, often leading to overconfident predictions for OOD samples.

Recent advancements have focused on enhancing feature representations to improve OOD discrimination. Zhou et al. \cite{zhou202250i} critically evaluated reconstruction-based autoencoder methods, proposing a novel framework that emphasizes a maximally compressed latent space for effective OOD detection. This work addresses the limitations of traditional autoencoders, which often fail to distinguish OOD samples due to their ability to reconstruct various inputs, thus highlighting the need for more nuanced feature extraction techniques.

Further expanding the scope of OOD detection, Yang et al. \cite{yang2022it3} introduced the Full-Spectrum OOD detection problem, which distinguishes between semantic and covariate shifts. This framework emphasizes the importance of recognizing that OOD samples can arise not only from new classes but also from changes in data distribution, a critical consideration for real-world applications where environmental conditions may vary.

In parallel, the exploration of OOD detection in novel data modalities has gained traction. Liu et al. \cite{liu202227x} pioneered unsupervised OOD detection for graph-structured data, addressing a previously under-explored area. Their approach leverages graph contrastive learning to capture the inherent structure of graph data, showcasing the adaptability of OOD detection methodologies across different data types.

Moreover, the introduction of auxiliary information has been a significant focus in enhancing OOD detection. Du et al. \cite{du20248xe} demonstrated that unlabeled data can be effectively utilized to improve OOD detection performance, providing theoretical guarantees for their gradient-based filtering mechanism. This work underscores the potential of leveraging additional data sources to bolster model robustness in the face of OOD inputs.

Despite these advancements, several challenges remain unresolved. For instance, while methods like PALM \cite{lu20249d4} improve upon traditional centroid-based approaches by modeling ID classes with multiple prototypes, the reliance on labeled data continues to limit their applicability in open-world scenarios where OOD examples are not predefined. Additionally, the challenge of multimodal OOD detection, as addressed by Dong et al. \cite{dong2024a8k}, highlights the necessity for dedicated algorithms that can exploit inter-modal interactions, a critical aspect for applications involving diverse sensor inputs.

In conclusion, while significant progress has been made in OOD detection methodologies, the field continues to face challenges related to data diversity, the integration of auxiliary information, and the need for robust frameworks capable of handling various data modalities. Future research should focus on developing more adaptable and comprehensive OOD detection systems that can operate effectively in dynamic real-world environments.
```