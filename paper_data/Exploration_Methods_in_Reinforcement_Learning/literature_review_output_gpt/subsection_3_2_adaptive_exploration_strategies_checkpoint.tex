\subsection*{Adaptive Exploration Strategies}

Adaptive exploration strategies in reinforcement learning (RL) address the challenge of dynamically adjusting exploration behavior based on the context and past experiences. These strategies are crucial for improving sample efficiency and adaptability in complex environments, particularly as state and action spaces expand. This subsection reviews key methodologies, including meta-learning and hierarchical reinforcement learning, that facilitate effective exploration.

One notable approach is presented in \cite{ding2023whs}, which introduces Dual-Adaptive $\epsilon$-greedy Exploration (DAE) within the framework of Incremental Reinforcement Learning (Incremental RL). DAE combines a Meta Policy (Ψ) that adaptively adjusts the exploration probability $\epsilon$ based on the state-specific convergence of exploration, and an Explorer (Φ) that prioritizes actions that have been least tried. This dual mechanism allows agents to efficiently navigate continuously expanding state and action spaces, addressing the inefficiencies of traditional fixed exploration methods. The study highlights the importance of adapting exploration strategies to the evolving nature of real-world environments, where agents must learn to explore new transitions without incurring excessive retraining costs.

Building on the need for adaptive exploration, \cite{mavrin2019iqm} explores distributional reinforcement learning as a means to facilitate efficient exploration. By modeling the distribution of value functions, this work introduces an exploration bonus derived from the upper quantiles of the learned distribution. This method enhances exploration by balancing the agent's intrinsic uncertainties, thereby improving performance in complex environments. However, it still relies on traditional exploration methods that may not fully capitalize on the adaptive capabilities introduced in DAE.

In the context of multi-agent systems, \cite{yang2021ngm} surveys exploration strategies that emphasize uncertainty-oriented and intrinsic motivation-oriented approaches. This survey reveals a critical gap in existing methodologies: while many strategies focus on individual agents, they often neglect the dynamics of cooperation and competition in multi-agent settings. The findings suggest that integrating adaptive exploration strategies, such as those proposed in DAE, could enhance exploration efficiency in multi-agent scenarios by enabling agents to better coordinate their exploration efforts.

Further advancements in adaptive exploration are seen in \cite{hong20182pr}, which proposes a diversity-driven exploration strategy that enhances exploratory behavior by incorporating a distance measure in the loss function. This method effectively prevents agents from getting trapped in local optima, complementing the adaptive strategies of DAE by ensuring a more comprehensive exploration of the state space. However, while diversity-driven methods improve exploration, they may still struggle with the computational overhead associated with tracking diverse actions in expansive environments.

Despite these advancements, challenges remain in the integration of adaptive exploration strategies across various RL paradigms. For instance, \cite{xu2023t6r} highlights the issue of inactivity in visual reinforcement learning agents, suggesting that exploration strategies must also account for the agents' activity levels to maximize their learning potential. This insight underscores the need for future research to develop adaptive exploration frameworks that not only adjust exploration probabilities but also actively promote agent engagement in the learning process.

In conclusion, while significant strides have been made in developing adaptive exploration strategies, unresolved issues persist regarding their integration and effectiveness in diverse and dynamic environments. Future research should focus on unifying different adaptive methodologies, exploring their synergies, and addressing the computational challenges that arise in real-world applications. The evolution of adaptive exploration strategies will be pivotal in advancing the capabilities of RL agents in increasingly complex and changing environments.
```