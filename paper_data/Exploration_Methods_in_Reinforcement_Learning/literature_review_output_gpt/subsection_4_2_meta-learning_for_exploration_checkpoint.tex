\subsection{Meta-Learning for Exploration}

In reinforcement learning (RL), effective exploration strategies are crucial for agents to adapt and perform well across diverse tasks and environments. Meta-learning, or learning to learn, has emerged as a promising approach to enhance exploration by enabling agents to adapt their exploration strategies based on the specific demands of each environment. This subsection reviews recent advancements in meta-learning frameworks, particularly focusing on their application in exploration strategies, such as those employed in Agent57, which dynamically selects exploration policies.

One significant contribution to this field is the work by Ding et al. (2023), which introduces Dual-Adaptive $\epsilon$-greedy Exploration (DAE) within the context of Incremental Reinforcement Learning (Incremental RL) \cite{ding2023whs}. This framework addresses the challenge of expanding state and action spaces in dynamic environments, where traditional exploration methods, like fixed $\epsilon$-greedy strategies, fall short. DAE combines a Meta Policy (Ψ) that adaptively adjusts the exploration probability based on the state-specific convergence of the value function, and an Explorer (Φ) that prioritizes actions that have been least tried. This dual-adaptive mechanism allows the agent to efficiently explore new states and actions while preserving previously learned behaviors, significantly reducing training overhead compared to conventional methods.

Building on the notion of adaptive exploration, Lee et al. (2021) propose PEBBLE, which enhances feedback efficiency in interactive RL by relabeling experiences and incorporating unsupervised pre-training \cite{lee2021qzk}. While DAE focuses on exploration in expanding environments, PEBBLE addresses the challenge of efficiently integrating human feedback into the learning process. By leveraging past experiences and optimizing the reward model through active querying of human preferences, PEBBLE demonstrates improved sample efficiency, particularly in complex tasks. This work highlights the potential for meta-learning to facilitate exploration not only through adaptive strategies but also by optimizing the learning process based on external feedback.

In a different approach, Mavrin et al. (2019) explore distributional reinforcement learning to enhance exploration through uncertainty modeling \cite{mavrin2019iqm}. Their method incorporates intrinsic exploration bonuses derived from the upper quantiles of the learned value distribution, effectively guiding the agent towards states with higher uncertainty. This approach aligns with the principles of meta-learning by allowing the agent to adapt its exploration strategy based on the distribution of learned values, thereby improving performance in challenging environments. However, while Mavrin et al. provide a robust framework for uncertainty-driven exploration, it lacks the adaptability to rapidly changing environments that DAE offers.

The integration of intrinsic motivation into exploration strategies has also been a focal point in the literature. For instance, Houthooft et al. (2016) introduce Variational Information Maximizing Exploration (VIME), which encourages exploration by maximizing information gain about the agent's belief of the environment dynamics \cite{houthooft2016yee}. VIME's approach to exploration is particularly relevant in sparse reward settings, where traditional exploration methods struggle. By utilizing meta-learning principles to adaptively modify the reward structure based on information gain, VIME complements the adaptive exploration strategies proposed by DAE and PEBBLE.

Despite these advancements, several challenges remain in the integration of meta-learning with exploration strategies. The reliance on fixed assumptions regarding the stability of learned transitions, as highlighted in Ding et al. (2023), poses limitations in environments that are inherently non-stationary. Future research should focus on developing more robust meta-learning frameworks that can dynamically adapt to both expanding state spaces and evolving environmental dynamics. Additionally, the interplay between intrinsic motivation and adaptive exploration strategies warrants further exploration to enhance the efficiency and effectiveness of RL agents in diverse and complex tasks.

In conclusion, the exploration landscape in reinforcement learning is rapidly evolving, with meta-learning providing innovative solutions to traditional challenges. The reviewed works demonstrate a clear progression towards more adaptive and efficient exploration strategies, yet unresolved issues related to dynamic environments and the integration of human feedback remain. Addressing these challenges will be crucial for advancing the applicability of RL in real-world scenarios.
```