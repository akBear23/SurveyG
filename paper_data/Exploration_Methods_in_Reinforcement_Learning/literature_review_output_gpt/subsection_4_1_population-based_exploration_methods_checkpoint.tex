\subsection{Population-Based Exploration Methods}

Population-based exploration methods in reinforcement learning (RL) leverage multiple agents to enhance exploration efficiency, addressing the challenges posed by complex environments. These methods allow for diverse exploration strategies, which can lead to improved performance and a better balance between exploration and exploitation. One notable example is Zhu's Two-Stage Evolutionary Reinforcement Learning, which exemplifies how maintaining a population of agents can facilitate a more robust exploration framework.

In the context of population-based methods, Conti et al. (2017) introduced a novel approach that integrates novelty-seeking behaviors within evolution strategies (ES) to enhance exploration in deep RL tasks. Their work demonstrates that by employing populations of agents that prioritize novelty, the algorithm can effectively avoid local optima and achieve superior performance on tasks with sparse rewards \cite{conti2017cr2}. This highlights the potential of collaborative exploration strategies, where agents can share information about their experiences, leading to a more comprehensive understanding of the environment.

Further building on this concept, Yang et al. (2021) conducted a comprehensive survey of exploration methods in both single-agent and multi-agent RL settings. They identified key challenges associated with exploration, particularly in environments characterized by sparse rewards and nonstationary dynamics \cite{yang2021ngm}. Their analysis underscores the importance of coordination among multiple agents to facilitate efficient exploration, suggesting that collaborative learning can significantly enhance the exploration-exploitation balance. This perspective aligns with the findings of Conti et al., emphasizing the role of diverse exploration strategies in overcoming the limitations of traditional methods.

Ding et al. (2023) introduced the Dual-Adaptive $\epsilon$-greedy Exploration (DAE) framework, specifically designed for Incremental Reinforcement Learning (Incremental RL) \cite{ding2023whs}. This framework integrates a Meta Policy and an Explorer to dynamically adjust exploration strategies based on the agent's experiences. By allowing agents to adapt their exploration strategies according to the evolving state and action spaces, DAE addresses the inefficiencies of conventional exploration methods. This work not only formalizes the challenge of Incremental RL but also provides a practical solution for maintaining performance in dynamic environments, further illustrating the advantages of population-based exploration.

However, challenges remain in coordinating multiple agents effectively. For instance, while collaborative exploration can yield diverse strategies, it can also introduce complexities in managing the interactions between agents, particularly in environments with high-dimensional state spaces. The work of Hu et al. (2020) on Voronoi-based multi-robot exploration demonstrates a hierarchical control architecture that minimizes redundant exploration among agents, thereby improving efficiency \cite{hu2020qwm}. This indicates that while population-based methods can enhance exploration, careful consideration of agent coordination is crucial to avoid inefficiencies.

In conclusion, population-based exploration methods represent a promising direction for enhancing exploration efficiency in reinforcement learning. The integration of multiple agents allows for diverse strategies and improved performance, as evidenced by the works of Conti et al., Yang et al., and Ding et al. Nevertheless, the challenges of coordinating these agents and managing their interactions in complex environments remain significant. Future research should focus on developing more sophisticated coordination mechanisms and exploring the scalability of these methods in increasingly dynamic settings.
```