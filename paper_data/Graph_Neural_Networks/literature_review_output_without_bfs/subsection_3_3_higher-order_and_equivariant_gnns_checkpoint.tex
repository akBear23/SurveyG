\subsection{Higher-Order and Equivariant GNNs}

The inherent limitations of standard Graph Neural Networks (GNNs), particularly their equivalence to the 1-Weisfeiler-Lehman (1-WL) test, restrict their ability to distinguish complex graph structures and adequately capture intricate motifs \cite{xu2018c8q, morris20185sd, jegelka20222lq}. This section explores advanced GNN architectures designed to overcome these expressivity bottlenecks and to respect inherent symmetries in data, which is crucial for physical and geometric tasks.

To enhance discriminative power beyond the 1-WL test, a class of **higher-order GNNs** has emerged. Initial efforts, such as k-dimensional GNNs (k-GNNs), directly generalize message passing to operate on k-tuples or k-element subsets of nodes, proving strictly more powerful than 1-GNNs by capturing higher-order structural information like triangle counts \cite{morris20185sd}. While theoretically robust, these models often incur significant computational and memory costs, scaling polynomially or exponentially with the number of nodes, which limits their applicability to large graphs \cite{morris20185sd, garg2020z6o}.

To mitigate the computational overhead of explicit k-tuple processing while retaining enhanced expressiveness, alternative strategies have been developed. Random Node Initialization (RNI) has been shown to enable standard Message Passing Neural Networks (MPNNs) to achieve universal approximation capabilities, effectively breaking the 1-WL barrier by individualizing nodes, though often at the cost of slower convergence \cite{abboud2020x5e}. Similarly, DropGNN leverages random node dropouts across multiple runs during both training and testing to perturb and explore diverse neighborhood patterns, allowing GNNs to distinguish graphs beyond the 1-WL test with relatively low overhead \cite{papp20211ac}.

Another powerful approach involves injecting explicit positional or structural information. Positional Encodings (PEs), often derived from graph Laplacian eigenvectors, provide unique node identities that help GNNs differentiate isomorphic nodes and capture global structural context \cite{dwivedi2021af0}. However, traditional Laplacian PEs suffer from sign ambiguity and instability issues. The Learnable Structural and Positional Encodings (LSPE) framework addresses this by decoupling and concurrently learning both structural and positional representations throughout the GNN layers, often outperforming static PEs \cite{dwivedi2021af0}. Building on this, PEG (Positional Encoding GNN) further improves stability and equivariance for PEs by imposing O(p) equivariance on positional features, making them robust to eigenvalue multiplicities and suitable for large graphs \cite{wang2022p2r}.

Beyond node-level enhancements, models explicitly designed to capture substructures or paths offer increased expressiveness. Substructure Aware Graph Neural Networks (SAGNNs) inject subgraph-level structural information, derived from novel "Cut subgraphs" and random walk return probability encodings, into node features, enabling GNNs to perceive higher-order substructures and surpass the 1-WL limit \cite{zeng20237gv}. For tasks like link prediction, the SEAL framework leverages a $\beta$-decaying heuristic theory to justify learning high-order features from local enclosing subgraphs using GNNs, achieving superior performance by dynamically learning structural patterns rather than relying on fixed heuristics \cite{zhang2018kdl}. Path Neural Networks (PathNNs) take a different route, aggregating information from various paths, with its most expressive variant, $\tilde{AP}$, capable of distinguishing graphs indistinguishable by the 3-WL algorithm through recursively annotated path sets \cite{michel2023hc4}.

Furthermore, dynamic and hybrid approaches are pushing the boundaries of expressiveness and long-range interaction. Cooperative Graph Neural Networks (CO-GNNs) allow nodes to dynamically choose their communication actions (listen, broadcast, isolate) at each layer, leading to adaptive information flow that enhances expressive power beyond 1-WL and mitigates over-smoothing \cite{finkelshtein202301z}. Similarly, Spatio-Spectral Graph Neural Networks (S2GNNs) combine local spatial message passing with global, spectrally bounded but spatially unbounded spectral filters. This hybrid design effectively vanquishes the "over-squashing" problem, enabling robust long-range information exchange and achieving state-of-the-art performance on challenging benchmarks \cite{geisler2024wli}.

A distinct but equally critical area focuses on **equivariant GNNs**, which are designed to respect inherent symmetries in data, particularly for physical and geometric tasks. Many scientific domains, such as molecular modeling, quantum chemistry, and physics, involve 3D structures whose properties must transform predictably under geometric operations like rotation, translation, and reflection \cite{reiser2022b08}. Enforcing this equivariance acts as a powerful inductive bias, leading to more accurate and data-efficient models.

While some GNNs achieve E(n) invariance (e.g., by using relative distances), they often fail to maintain equivariance for vector outputs \cite{satorras2021pzl}. Early equivariant models often relied on computationally expensive higher-order representations like spherical harmonics, limiting their scalability and efficiency. E(n)-Equivariant Graph Neural Networks (EGNNs) offer a simpler and more efficient solution by directly integrating coordinate updates into the message-passing framework. EGNNs update node features and coordinates simultaneously, using squared relative distances for message computation and relative differences for coordinate updates, thereby preserving E(n) equivariance without complex higher-order tensors \cite{satorras2021pzl}. This design scales effectively to arbitrary n-dimensional spaces.

Building on these principles, more specialized equivariant architectures have emerged. GemNet, for instance, provides a universal directional GNN for molecules, leveraging spherical (S2) representations and a novel two-hop message passing scheme that explicitly incorporates interatomic distances, angles, and crucial dihedral angles. This allows GemNet to achieve state-of-the-art accuracy and data efficiency in molecular property prediction \cite{klicpera20215fk}. Another notable example is Neural Equivariant Interatomic Potentials (NequIP), an E(3)-equivariant neural network that employs geometric tensors for interactions, leading to highly data-efficient and accurate interatomic potentials for molecular dynamics simulations \cite{batzner2021t07}.

In conclusion, the development of higher-order and equivariant GNNs represents a significant leap in addressing the fundamental limitations of traditional GNNs. Higher-order models, through various strategies from k-tuples to sophisticated positional and substructural encodings, are pushing the boundaries of structural discriminative power. Concurrently, equivariant GNNs are enabling physically consistent and data-efficient predictions in scientific domains by explicitly respecting geometric symmetries. Despite these advancements, challenges remain in balancing expressiveness with scalability for extremely large graphs, developing more generalizable equivariant models for diverse physical systems, and enhancing the interpretability of complex higher-order features. The continuous development of rigorous benchmarking frameworks remains crucial for fairly evaluating and driving progress in these advanced GNN architectures \cite{dwivedi20239ab}.