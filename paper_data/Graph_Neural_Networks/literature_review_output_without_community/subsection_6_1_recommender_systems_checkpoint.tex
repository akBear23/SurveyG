\subsection{Recommender Systems}
Recommender systems are indispensable components of modern digital ecosystems, tasked with alleviating information overload by predicting user preferences and suggesting relevant items. Graph Neural Networks (GNNs) have emerged as a transformative paradigm in this domain, leveraging their inherent ability to model complex relational data. They excel at capturing intricate user-item interactions, as well as integrating diverse auxiliary information such as social networks, explicit opinions, and item attributes \cite{wu2020dc8, gao2022f3h}. By propagating information across these rich, multi-faceted graph structures, GNNs learn sophisticated latent factors, leading to significantly more accurate, personalized, and increasingly explainable recommendations \cite{he202455s}.

One of the earliest and most critical challenges for GNNs in recommendation was scaling to the massive datasets encountered in industrial applications. Addressing this, \cite{ying20189jc} introduced PinSage, a pioneering framework that demonstrated the viability of GNNs for web-scale recommender systems. While the detailed architectural innovations of PinSage, such as efficient neighborhood sampling and a producer-consumer architecture, are discussed in Section 4.1, its impact on recommender systems was profound. PinSage proved that GNNs could effectively learn high-quality item embeddings from billions of user interactions, leading to substantial improvements in user engagement and driving commercial success at Pinterest. Its success underscored the potential of GNNs to move beyond academic benchmarks to real-world deployment.

Building upon the foundation of scalable GNNs, subsequent research focused on enriching user and item representations by integrating more diverse auxiliary information. \cite{fan2019k6u} proposed GraphRec, a comprehensive GNN framework tailored for social recommendation. GraphRec innovatively models both user-user social graphs and user-item interaction graphs, learning distinct user latent factors from "item-space" and "social-space" perspectives. A key contribution was its use of "opinion embedding vectors" to jointly capture user-item interactions and their associated explicit opinions (e.g., rating scores), moving beyond binary interaction signals. Furthermore, GraphRec employed multiple attention mechanisms to dynamically weigh the contributions of different neighbors and interaction types, leading to more nuanced user modeling and enhanced personalized recommendations by leveraging the principle of social influence \cite{sharma2022liz}.

Beyond social networks, knowledge graphs (KGs) have proven to be another powerful source of auxiliary information for GNN-based recommenders. KGs, which represent entities and their relationships, can significantly alleviate data sparsity and cold-start problems by providing rich semantic context for items and users \cite{he202455s, ye20226hn}. GNNs are particularly adept at traversing and embedding these multi-relational graphs, allowing them to infer complex connections between items and users that are not explicit in interaction data. For instance, models can leverage GNNs to propagate information about item attributes (e.g., genre, director for movies) or hierarchical categories, leading to more informed recommendations, especially for less popular items or new users.

Another significant area where GNNs have excelled is session-based recommendation (SBR), which aims to predict the next item a user will interact with based on their current anonymous session. Traditional sequential models often struggle to capture the complex, non-linear transitions between items within a session. \cite{wu2018t43} introduced SR-GNN, one of the first works to model session sequences as graph-structured data, allowing GNNs to capture intricate item transitions that are difficult for conventional sequential methods. Extending this, \cite{wang2020khd} proposed Global Context Enhanced Graph Neural Networks (GCE-GNN), which learns item embeddings from both session-level graphs (modeling pairwise transitions within a session) and a global graph (modeling transitions across all sessions). By aggregating these two levels of representations with attention mechanisms, GCE-GNN provides a more comprehensive understanding of user preferences, demonstrating the power of GNNs in capturing both local and global sequential patterns.

As GNNs become more prevalent in high-stakes applications like recommender systems, addressing trustworthiness aspects such as fairness and explainability becomes paramount. GNNs, like other machine learning models, can inherit and even amplify biases present in historical interaction data or graph structures, leading to discriminatory recommendations \cite{dai2020p5t}. Research in fair GNNs, such as REDRESS \cite{dong202183w} and FairVGNN \cite{wang2022531}, focuses on mitigating these biases, for instance, by promoting individual fairness or preventing sensitive attribute leakage during feature propagation. While the detailed methodologies for ensuring fairness in GNNs are explored in Section 7.2, their application to recommender systems is crucial for ethical deployment. Furthermore, the initial claim regarding "explainable recommendations" is being actively addressed. For example, \cite{lyu2023ao0} proposed Knowledge Enhanced Graph Neural Networks (KEGNN) for explainable recommendation, which leverages semantic knowledge from external knowledge bases to learn comprehensive user/item representations and generate human-like explanations, thereby making the recommendation process more transparent and trustworthy.

In conclusion, GNNs have profoundly transformed recommender systems by offering a powerful framework to model complex user-item interaction graphs and integrate diverse auxiliary information. The field has progressed from pioneering efforts in scaling GNNs for industrial deployment \cite{ying20189jc} and enriching representations with social \cite{fan2019k6u} and knowledge graph data, to effectively modeling dynamic user behaviors in session-based contexts \cite{wu2018t43, wang2020khd}. Crucially, the community is also addressing the ethical implications, developing methods for fair \cite{dong202183w, dai2020p5t, wang2022531} and explainable recommendations \cite{lyu2023ao0}. While challenges such as structural disparity (as discussed in Section 4.2), cold-start problems, and handling highly dynamic graphs persist, the continuous innovation in GNN architectures promises even more accurate, personalized, and responsible recommendation engines in the future.