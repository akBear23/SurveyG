\subsection*{The Rise of Graph-Structured Data}

The digital age has ushered in an unprecedented era where data is increasingly generated and organized in complex, interconnected structures, fundamentally shifting from traditional tabular formats to intricate graphs. This paradigm is evident across a myriad of domains, from the vast expanse of social networks and the intricate pathways of biological systems to the semantic richness of knowledge graphs, the personalized recommendations of e-commerce platforms, and the dynamic flows within transportation networks \cite{wu2022ptq, wang2023zr0}. Such graph-structured data inherently captures complex, non-i.i.d. relational dependencies and rich structural information that are crucial for understanding the underlying phenomena. The rapid growth of research in Graph Neural Networks (GNNs) itself underscores the critical importance and ubiquity of graph data in modern applications, highlighting a significant paradigm shift in data representation and analysis \cite{wu2022ptq}.

However, traditional machine learning models, primarily designed for Euclidean data where instances are assumed to be independent and identically distributed (i.i.d.), inherently struggle to effectively process and leverage this topological and relational information. Their limitations stem from a fundamental inability to capture the non-linear interactions between entities, the influence propagation across connections, and the rich structural context embedded within these interconnected datasets. Traditional models, by flattening graph structures into feature vectors or treating relationships as independent features, fundamentally discard this crucial relational context. This loss of topological information means they cannot discern how a node's properties are influenced by its neighbors, how information flows through the network, or how global graph patterns emerge from local interactions. Such an approach leads to suboptimal performance and a superficial understanding of the underlying phenomena when confronted with the intricate dependencies inherent in graph data \cite{wang2023zr0}.

The necessity for models that can explicitly leverage graph topology is exemplified in diverse real-world scenarios where interconnectedness is paramount. In social networks, understanding community structures, predicting user behavior, or detecting anomalies requires analyzing the intricate web of connections rather than isolated user profiles. Similarly, web-scale recommender systems, dealing with billions of users and items, fundamentally rely on modeling complex user-item interaction graphs to provide personalized suggestions. Traditional matrix factorization or content-based methods often fall short in capturing the nuanced, multi-hop relationships and the propagation of preferences across the graph, which are crucial for accurate recommendations \cite{ying20189jc, fan2019k6u}. The sheer scale and dynamic nature of these graphs further exacerbate the limitations of conventional approaches, demanding models that can efficiently learn from and adapt to evolving relational data.

Beyond social and recommendation graphs, the prevalence of graph-structured data extends to critical domains like scientific discovery and infrastructure management. In biological systems, protein-protein interaction networks, gene regulatory networks, and molecular graphs are inherently relational, where the function of a component is determined by its interactions within the larger system. Analyzing these structures is vital for drug discovery, disease understanding, and materials science \cite{wu2022ptq}. Furthermore, in smart cities and environmental monitoring, multivariate time series data from sensors often exhibit complex spatial dependencies (e.g., traffic flow between intersections, air quality measurements across a region). Treating these time series independently ignores critical spatial correlations, making traditional models less effective for tasks like forecasting or anomaly detection. The explicit modeling of these non-Euclidean spatial relationships through graph structures has proven crucial for accurate analysis in such domains \cite{wu2020hi3, jin2023ijy}.

In essence, the digital transformation has rendered graph-structured data a ubiquitous and indispensable representation for complex systems across nearly every scientific and industrial domain. The inherent limitations of traditional machine learning, which are ill-equipped to handle the non-i.i.d. and relational nature of such data, have created a compelling need for a new generation of learning paradigms. This fundamental shift underscores the critical importance of developing specialized graph-aware models that can effectively leverage the rich topological and relational information embedded within these interconnected datasets. Graph Neural Networks have emerged as a powerful response to this challenge, promising to unlock deeper insights and drive innovation across fields by directly learning from the structural and relational complexities of the modern data landscape. This review will delve into the foundational concepts, architectural advancements, and diverse applications of these transformative models.