\subsection{Benchmarking and Rigorous Evaluation}
\label{sec:benchmarking}

The rapid proliferation of Graph Neural Network (GNN) architectures, while indicative of their immense potential, has concurrently underscored a critical challenge in the field: the pervasive lack of robust, standardized, and reproducible benchmarking methodologies. Without such foundational frameworks, scientific rigor is compromised, fair comparisons become elusive, and the true impact of architectural advancements can be obscured, ultimately hindering reliable progress and community development. This issue mirrors broader challenges in machine learning research regarding reproducibility and the verifiability of reported gains.

Initially, the success of pioneering GNN models, such as the SEAL framework for link prediction \cite{zhang2018kdl}, which demonstrated "unprecedented performance" by learning general graph structure features from local enclosing subgraphs, highlighted the need for rigorous evaluation. However, the very enthusiasm surrounding these early breakthroughs inadvertently led to inconsistent evaluation practices. A significant pitfall in GNN benchmarking, particularly for tasks like link prediction, has been the prevalence of underreported baseline performance, inconsistent data splits, and unrealistic negative sampling strategies \cite{li2023o4c}. These inconsistencies frequently result in inflated performance claims and impede accurate, apples-to-apples comparisons between models. To address this, \cite{li2023o4c} introduced a refined benchmarking methodology for link prediction. Their key innovation, the Heuristic Related Sampling Technique (HeaRT), generates more challenging and realistic negative samples by restricting them to "corruptions" (sharing one node with the positive sample) and selecting hard negatives based on structural heuristics. This approach ensures a more truthful assessment of model capabilities, moving beyond trivial negative samples that can artificially inflate performance metrics.

Beyond task-specific issues, the broader GNN community struggled with general evaluation challenges. Many studies relied on small, non-discriminative datasets (e.g., Cora, Citeseer) that often failed to statistically differentiate complex GNN architectures from simpler baselines. Furthermore, inconsistent experimental settings, including varying parameter budgets, made fair comparisons nearly impossible, making it difficult to attribute performance gains to genuine architectural innovation rather than increased model capacity \cite{dwivedi20239ab}. To rectify this, \cite{dwivedi20239ab} developed a comprehensive, open-source, and modular benchmarking framework for GNNs. This framework introduced 12 medium-scale datasets, encompassing both real-world graphs (e.g., ZINC, OGB-COLLAB) and synthetic graphs specifically designed to test theoretical graph properties (e.g., distinguishing isomorphic graphs, identifying specific substructures like cycles). Crucially, it enforced fixed parameter budgets (e.g., 100k and 500k parameters) for all models, ensuring that performance differences were genuinely attributable to architectural design. This standardized environment has proven instrumental in validating fundamental GNN components, such as the efficacy of Graph Positional Encoding (PE) using Laplacian eigenvectors for improving Message-Passing GCNs on anonymous graphs \cite{dwivedi20239ab}.

The imperative for rigorous evaluation extends beyond predictive accuracy to critical dimensions like robustness and explainability, which are paramount for real-world deployment. In the realm of robustness, a major concern has been the overly optimistic estimates of GNN defense mechanisms due to their evaluation against weak, non-adaptive attacks \cite{mujkanovic20238fi}. Drawing parallels with the computer vision community, \cite{mujkanovic20238fi} introduced a systematic, 6-step methodology for designing strong *adaptive attacks* for GNNs, which account for an adversary's knowledge of the defense. Their comprehensive analysis revealed that most existing GNN defenses offer "no or only marginal improvement" against such adaptive adversaries, highlighting a critical gap in evaluation practices and the urgent need for more robust benchmarks. Similarly, \cite{gosch20237yi} critically re-evaluated adversarial training for GNNs, demonstrating that prior reported robustness gains were often artifacts of biased *transductive* learning settings, where models could memorize the training graph. They advocated for and validated adversarial training in a *fully inductive setting* with *locally constrained attacks* as a more realistic and challenging benchmark for robustness. Addressing the scalability challenge, \cite{geisler2021dcq} developed sparsity-aware first-order optimization attacks (PR-BCD, GR-BCD) and novel surrogate losses (MCE, tanh margin) to enable rigorous robustness evaluation of GNNs on web-scale graphs with millions of nodes, overcoming the prohibitive memory and computational costs of previous methods.

Furthermore, the evaluation of GNNs under imperfect data conditions and for interpretability has also seen dedicated benchmarking efforts. Recognizing the prevalence of label noise in real-world graph data, \cite{wang2024481} introduced NoisyGL, the first comprehensive benchmark for GNNs under label noise. This benchmark provides unified experimental settings and interfaces to enable fair comparisons and deeper analyses of methods designed to handle noisy labels, a crucial aspect of real-world applicability. For explainability, which is vital for trust and transparency in high-stakes applications, \cite{agarwal2022xfp} developed GRAPHXAI, a general-purpose framework, and SHAPEGG EN, a novel synthetic graph generator. SHAPEGG EN creates diverse datasets with *guaranteed reliable ground-truth explanations*, addressing the critical limitation of existing datasets lacking such information. GRAPHXAI integrates this generator with real-world datasets, GNN models, and a suite of tailored evaluation metrics (e.g., Graph Explanation Accuracy, Faithfulness, Stability, Fairness), enabling systematic benchmarking of GNN explainers and revealing significant weaknesses in current methods, particularly on heterophilic graphs and for fairness properties.

In conclusion, the evolution of GNN research has underscored the paramount importance of robust and standardized benchmarking. From refining task-specific evaluations like link prediction \cite{li2023o4c} to establishing comprehensive frameworks for general GNNs \cite{dwivedi20239ab}, and extending to specialized benchmarks for robustness \cite{mujkanovic20238fi, gosch20237yi, geisler2021dcq}, label noise \cite{wang2024481}, and explainability \cite{agarwal2022xfp}, these efforts are collectively fostering a healthier scientific environment. Such rigorous and reproducible frameworks are indispensable for accurately assessing the true impact of architectural innovations, guiding future research and development effectively, and ensuring the credibility and trustworthy deployment of GNNs as the field continues to mature and integrate with other advanced AI paradigms.