\subsection{Trustworthiness: Privacy, Robustness, Fairness, and Explainability}

The responsible deployment of Graph Neural Networks (GNNs) in high-stakes real-world applications hinges critically on their trustworthiness, encompassing dimensions of privacy preservation, robustness against adversarial manipulations, fairness in predictions, and the explainability of their decisions. Ensuring these properties is paramount for integrating GNNs into critical infrastructure and societal systems, addressing the unique challenges posed by graph-structured data. Comprehensive surveys by \cite{dai2022hsi} and \cite{zhang20222g3} provide foundational overviews, systematically categorizing existing methods and highlighting the distinct challenges of privacy leakage, adversarial attacks, inherent biases, and interpretability in graph domains. These works emphasize that the non-Euclidean, interconnected nature of graph data necessitates specialized approaches for trustworthiness, differentiating GNNs from traditional machine learning models. Furthermore, rigorous evaluation of GNNs, as advocated by benchmarking efforts like \cite{dwivedi20239ab}, is crucial for assessing their trustworthiness, enabling fair and reproducible comparisons essential for validating claims of reliability.

**Privacy Preservation** in GNNs is a critical concern, as graph data often contains sensitive relational information. The unique challenge lies in protecting not only node features but also the intricate graph structure itself. A seminal work by \cite{he2020kz4} introduced and systematically studied "link stealing attacks," demonstrating that an adversary can infer the existence of links in a training graph from the black-box outputs of a GNN model. This vulnerability arises because GNNs' message-passing mechanisms inherently encode structural information into node representations. Beyond link stealing, other privacy threats include membership inference (determining if a specific node was part of the training graph) and attribute inference (recovering sensitive node features) \cite{zhang20222g3}. To counter these threats, research explores several defense paradigms. Differential privacy (DP) can be applied to GNNs by injecting noise during message passing or gradient updates, offering quantifiable privacy guarantees at the cost of some utility \cite{zhang20222g3}. Another promising direction is Federated Learning (FL), which inherently protects data locality by training GNNs on decentralized graph partitions without sharing raw data. Surveys like \cite{liu2022gcg} provide an overview of Federated GNNs (FedGNNs), while recent applications such as \cite{hausleitner2024vw0} demonstrate their utility in privacy-aware disease classification by collaboratively training GNNs on subgraphs of a Protein-Protein Interaction network, integrating human-in-the-loop mechanisms for enhanced transparency.

**Robustness** is a multifaceted challenge for GNNs, encompassing resilience against adversarial attacks, generalization to out-of-distribution (OOD) data, and mitigation of inherent model limitations like oversmoothing. GNNs are particularly vulnerable to adversarial attacks, which can involve subtle perturbations to node features or graph structure (e.g., adding/deleting edges) to induce misclassifications \cite{dai2022hsi, zhang20222g3}. These attacks can be categorized into poisoning attacks (manipulating training data) and evasion attacks (manipulating test data), with defenses often involving adversarial training, certified defenses, or robust aggregation mechanisms. Beyond adversarial threats, understanding and improving GNN generalization is crucial. \cite{ju2023prm} significantly advances theoretical understanding by deriving sharp, non-vacuous PAC-Bayesian bounds for GNN generalization, scaling with the spectral norm of the graph diffusion matrix. These bounds offer orders of magnitude tighter guarantees than previous vacuous bounds, providing a stronger theoretical basis for building reliable GNNs. Addressing a key practical limitation, \cite{kang2024fsk} introduces FROND, a novel framework leveraging fractional calculus in continuous GNNs to model non-local, memory-dependent dynamics. This approach inherently mitigates the pervasive oversmoothing problem, which degrades GNN performance in deep layers, thereby enhancing model robustness and reliability. Furthermore, \cite{xia20247w9} tackles OOD generalization by proposing the Cluster Information Transfer (CIT) mechanism, a plug-in method designed to learn invariant representations against "structure shift" in graph data, crucial for maintaining robustness in dynamic real-world environments.

**Fairness** in GNN predictions is paramount, especially given the potential for GNNs to amplify existing societal biases due to graph homophily (the tendency for similar nodes to connect) and the message-passing mechanism. Bias can manifest as disparate impact (group fairness, e.g., demographic parity or equal opportunity) or disparate treatment (individual fairness, e.g., similar individuals receiving dissimilar outcomes) \cite{zhang20222g3, dai2022hsi}. Addressing this, research has explored various strategies. **Pre-processing** methods aim to debias the input graph data itself. For instance, \cite{dong2021qcg} proposes EDITS, a model-agnostic framework that defines and mitigates "attribute bias" and "structural bias" in attributed networks *before* GNN processing, using Wasserstein-1 distance to quantify bias in both node features and graph structure. This approach offers a universal solution for data debiasing, applicable to any GNN. **In-processing** methods modify the GNN training process. FairGNN \cite{dai2020p5t} introduces an adversarial framework that includes a GCN-based sensitive attribute estimator to predict missing sensitive attribute information, allowing the GNN classifier to be trained to fool an adversary attempting to predict sensitive attributes from its representations. This approach effectively eliminates discrimination while maintaining high accuracy, even with limited sensitive attribute data. For **individual fairness**, \cite{dong202183w} proposes REDRESS, a novel ranking-based framework that refines the notion of individual fairness from a ranking perspective. This method addresses the practical difficulties of Lipschitz-based definitions by formulating a joint optimization problem that balances model utility and ranking-based individual fairness, making it a plug-and-play solution for existing GNN architectures.

The **Interpretability** of GNNs is another critical pillar of trustworthiness, allowing users to understand and trust model decisions, especially in sensitive applications. Recent research has moved beyond merely providing explanations to rigorously ensuring their faithfulness and structural awareness. \cite{chen2024woq} critically examines the interpretability of existing GNN explainers, introducing the Subgraph Multilinear Extension (SubMT) framework to prove limitations of attention-based methods and proposing GMT with random subgraph sampling for more faithful interpretations. This work highlights the need for explanations that truly reflect the model's reasoning process. Building on this, \cite{bui2024zy9} proposes the Myerson-Taylor interaction index and the MAGE explainer, which axiomatically account for graph structure and high-order interactions. This method identifies not only influential nodes and edges but also complex motifs that positively or negatively impact predictions, providing a deeper, more contextually accurate understanding of GNN behavior. Survey papers also categorize explainability methods into intrinsically interpretable GNNs and post-hoc explainers, detailing motivations, challenges, and experimental settings \cite{zhang20222g3, dai2022hsi}.

In conclusion, the pursuit of trustworthy GNNs is rapidly maturing, with substantial progress in developing theoretically grounded and practically robust solutions across privacy, robustness, fairness, and explainability. While the field has made significant strides in understanding and mitigating robustness issues and providing more faithful interpretations, dedicated research into privacy-preserving mechanisms, especially against sophisticated structural inference attacks, and ensuring fairness across diverse populations with practical, scalable solutions remains crucial. Future directions will likely involve more interdisciplinary approaches, combining advanced cryptographic techniques for privacy, robust causal inference methods for fairness, and continued development of transparent and verifiable explanation frameworks, all while striving for holistic solutions that address these interconnected dimensions concurrently.