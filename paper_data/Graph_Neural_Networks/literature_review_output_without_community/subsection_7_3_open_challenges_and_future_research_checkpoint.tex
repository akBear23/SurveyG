\subsection*{Open Challenges and Future Research}

Despite the remarkable progress in Graph Neural Networks (GNNs), the field is still nascent, facing several fundamental open challenges that define critical avenues for future research. These challenges span from enhancing the intrinsic expressive power of GNNs and developing more robust and adaptive models for complex real-world graphs, to exploring novel mathematical foundations and deepening their integration with other advanced AI paradigms. Ultimately, these efforts aim to shape the next generation of graph-aware AI, ensuring its efficiency, scalability, and trustworthiness.

A primary and enduring challenge lies in further enhancing GNN expressivity, particularly for complex graph structures, and overcoming theoretical limitations like the Weisfeiler-Leman (WL) barrier and practical issues such as over-squashing. While Section 3 detailed advancements in higher-order, substructure-aware, geometric, and spatio-spectral GNNs, the quest for universally expressive yet efficient architectures continues. Theoretical analyses, as surveyed by \cite{jegelka20222lq}, underscore the inherent limitations of standard message-passing GNNs, motivating the exploration of alternative paradigms. For instance, \cite{papp20211ac} introduced DropGNN, demonstrating that random node dropouts during both training and testing can increase expressiveness beyond the WL-test with relatively low overhead, by allowing the GNN to observe diverse neighborhood patterns. However, the computational cost of multiple runs and the challenge of ensuring sufficient observation of higher-order dropouts remain practical considerations. Future research must balance theoretical expressivity with computational tractability and generalization capabilities, potentially through novel architectural designs or by integrating insights from other fields.

Beyond expressivity, developing GNNs that are robust to real-world imperfections and can generalize effectively to out-of-distribution (OOD) data is paramount. Real-world graphs often exhibit inherent noise, incompleteness, and structural heterogeneity, including mixed homophilic and heterophilic patterns (structural disparity). While Section 4.2 discussed adaptive filtering mechanisms like NODE-MOE, the challenge of consistently learning invariant representations under significant "structure shift" remains. \cite{xia20247w9} introduced the Cluster Information Transfer (CIT) mechanism to address OOD generalization by learning invariant representations, yet its generalizability across diverse shift types requires further investigation. Similarly, effectively handling heterophilous graphs, where traditional GNNs struggle, necessitates new approaches. \cite{li2022315}'s GloGNN, which learns global node correlations with linear time complexity, offers a promising direction by moving beyond local aggregation, but its applicability to extremely sparse or dynamic graphs needs further exploration.

A particularly critical and emerging area is the integration of *causal inference* into GNNs to move beyond spurious correlations and enhance robustness and interpretability. Traditional GNNs often learn statistical correlations that may not reflect true causal mechanisms, leading to poor generalization on unseen data. Future research will focus on developing GNNs that can disentangle causal and non-causal information. For example, \cite{zhao2024qw6} proposed an Information-based Causal Learning (ICL) framework to optimize causal information flow, while \cite{fan2022m67}'s DisC framework addresses severe bias by learning disentangled causal substructures. However, the challenges lie in rigorously defining causality in complex graph structures, developing scalable methods for causal discovery, and effectively integrating causal mechanisms into GNN architectures without sacrificing predictive performance.

The trustworthiness of GNNs, encompassing privacy, robustness, fairness, and explainability, remains a guiding principle for future development, building upon the foundational work discussed in Section 7.2.
\begin{itemize}
    \item \textbf{Explainability:} While progress has been made in post-hoc explanations (\cite{chen2024woq}, \cite{bui2024zy9}), a key frontier is the development of *inherently interpretable* GNN architectures that provide transparent decision-making processes, rather than just post-hoc rationalizations. Future work needs to establish more rigorous metrics for evaluating the faithfulness and comprehensibility of explanations, especially for complex graph tasks.
    \item \textbf{Fairness:} Addressing algorithmic bias in GNNs is crucial. Research must move towards proactive and holistic fairness solutions that account for the unique ways bias can be amplified through graph structures and message passing (\cite{dai2020p5t}). This includes developing methods for individual fairness (\cite{dong202183w}), model-agnostic debiasing of input graph data (\cite{dong2021qcg}), and ensuring fairness in multi-modal or dynamic graph settings. The challenge of defining and measuring fairness in complex, interconnected data remains significant.
    \item \textbf{Privacy:} The vulnerability of GNNs to privacy attacks, such as link stealing (\cite{he2020kz4}), necessitates robust privacy-preserving mechanisms. Future research will focus on integrating advanced techniques like differential privacy, homomorphic encryption, and secure multi-party computation directly into GNN training and inference pipelines. The emerging field of federated graph neural networks (\cite{liu2022gcg}) offers a promising avenue for privacy-preserving collaborative learning on decentralized graph data, but challenges in data heterogeneity and communication overhead persist. A comprehensive understanding of these issues is provided by surveys like \cite{dai2022hsi}, which will guide future privacy-aware GNN design.
\end{itemize}

A particularly promising and transformative avenue for future research involves deepening the integration of GNNs with other advanced AI paradigms, especially Large Language Models (LLMs), for multi-modal understanding and enhanced transferability. As highlighted by \cite{li202444f}'s Morpher, aligning GNNs with LLMs, even under extremely weak text supervision, enables CLIP-style zero-shot generalization, opening new frontiers for semantic understanding and adaptability in low-resource settings. However, significant challenges remain in effectively aligning diverse modalities (e.g., graph structure, node features, text, images), handling noisy or incomplete multi-modal data, and ensuring that the unique structural information encoded by GNNs is not diluted or misrepresented by the dominant semantic knowledge of LLMs. Future research will explore more sophisticated cross-modal attention mechanisms, unified representation learning, and novel prompting strategies that leverage the strengths of both graph and language models.

Finally, the continuous need for more efficient, scalable, and trustworthy GNNs for emerging applications and ethical considerations remains a guiding principle. Scaling GNNs to truly web-scale, dynamic graphs with billions of nodes and edges, particularly in decentralized or federated environments, requires fundamental architectural innovations beyond current sampling or partitioning strategies. The exploration of novel mathematical foundations, such as fractional calculus within the FROND framework (\cite{kang2024fsk}), offers a fresh perspective on designing inherently robust GNNs that mitigate issues like oversmoothing and model complex, memory-dependent graph dynamics. Furthermore, the establishment of rigorous benchmarking frameworks, as advocated by \cite{dwivedi20239ab}, is crucial for fair comparison and identifying truly impactful architectural advancements. Future research will likely focus on a synergistic approach, combining architectural innovations for enhanced expressivity, advanced mathematical tools for inherent robustness, multi-modal integration for broader applicability, and a strong emphasis on ethical AI principles, all while upholding efficiency and scalability to shape the next generation of graph-aware AI and its profound societal impact.