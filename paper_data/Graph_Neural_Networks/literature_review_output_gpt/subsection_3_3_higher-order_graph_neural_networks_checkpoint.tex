\subsection{Higher-Order Graph Neural Networks}

Higher-order Graph Neural Networks (GNNs) extend traditional message-passing frameworks to capture richer graph structures, addressing the limitations of first-order models in representing complex relationships. This section discusses various methodologies, including k-GNNs and spectral approaches, that enable GNNs to model intricate interactions effectively.

The work by Morris et al. (2018) introduces k-GNNs, which generalize standard GNNs by allowing message passing between k-element subsets of nodes rather than just individual nodes. This advancement significantly enhances the expressive power of GNNs, enabling them to capture higher-order structural information, such as triangles and cliques, which are crucial for tasks like graph classification and regression \cite{morris20185sd}. The authors establish a theoretical connection between k-GNNs and the Weisfeiler-Leman (WL) test, demonstrating that k-GNNs can distinguish non-isomorphic graphs that standard GNNs cannot, thus providing a solid foundation for further exploration into higher-order GNNs.

Building on the theoretical groundwork laid by this earlier work, Klicpera et al. (2018) propose a novel propagation mechanism that decouples prediction from propagation using personalized PageRank, which addresses the oversmoothing problem prevalent in traditional GNNs \cite{klicpera20186xu}. Their method allows for deeper information propagation without increasing model complexity, thereby enhancing the expressive capabilities of GNNs while maintaining computational efficiency. This approach offers a practical solution to the limitations faced by first-order models, paving the way for more sophisticated GNN architectures.

In a complementary vein, Zhang et al. (2018) present the SEAL framework, which focuses on link prediction by learning high-order features from local subgraphs. Their work introduces a theoretical justification for the effectiveness of k-hop enclosing subgraphs, which allows GNNs to capture complex link formation patterns more effectively than traditional heuristic methods \cite{zhang2018kdl}. This research underscores the importance of higher-order information in improving GNN performance, particularly in tasks where understanding the local structure is critical.

Further advancements are made by Satorras and Vinyals (2021), who introduce E(n) equivariant GNNs that enforce equivariance to Euclidean transformations across n-dimensional spaces. This work highlights the significance of geometric properties in GNNs, particularly for applications in physics and material science, where preserving geometric symmetries is crucial \cite{satorras2021pzl}. Their approach demonstrates that higher-order representations can be efficiently learned without relying on complex higher-order structures, thus simplifying the modeling process while enhancing the GNN's expressive power.

Despite these advancements, challenges remain in balancing expressive power with computational efficiency. For instance, while k-GNNs provide a robust framework for capturing higher-order structures, they often introduce increased computational complexity and require careful hyperparameter tuning. Additionally, while techniques like Personalized PageRank enhance propagation depth, they may not fully address the increased risk of overfitting in complex graph structures.

In conclusion, while significant strides have been made in developing higher-order GNNs, unresolved issues persist regarding their scalability and interpretability in real-world applications. Future research should focus on developing more efficient algorithms that can harness the expressive power of higher-order representations while ensuring robustness and generalizability across diverse graph types and tasks.
```