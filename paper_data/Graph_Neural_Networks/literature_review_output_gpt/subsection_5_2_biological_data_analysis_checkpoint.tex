\subsection{Biological Data Analysis}

Graph Neural Networks (GNNs) have emerged as a powerful tool for analyzing complex biological data, particularly in drug discovery and protein-protein interaction prediction. However, traditional GNNs often struggle with expressiveness and efficiency, limiting their ability to capture the intricate relationships inherent in biological networks. Recent advancements have aimed to address these challenges, leading to a more nuanced understanding of how GNNs can be adapted for biological applications.

Ying et al. (2018) introduced a scalable GNN framework designed for web-scale recommender systems, demonstrating the potential of GNNs to handle large and complex networks \cite{ying20189jc}. This work laid the groundwork for subsequent research by illustrating how GNNs could be effectively applied to biological data, where similar scalability is crucial. Following this, Hu et al. (2019) explored pre-training strategies for GNNs, emphasizing their importance in scenarios with limited labeled data, a common challenge in biological research \cite{hu2019r47}. This approach highlights the need for GNNs to adapt to the scarcity of annotated biological datasets, paving the way for more robust models.

Satorras et al. (2021) further advanced the field by proposing E(n)-equivariant GNNs, which are designed to maintain symmetry under Euclidean transformations, thereby improving the model's ability to generalize across different biological structures \cite{satorras2021pzl}. This work addresses the limitations of traditional GNNs, which often fail to capture the geometric properties of biological molecules. By enforcing equivariance, the model enhances its expressiveness, making it more suitable for tasks such as predicting protein interactions and drug efficacy.

In parallel, Michel et al. (2023) introduced Path Neural Networks, which leverage path information to improve GNN expressiveness beyond the 1-WL limit \cite{michel2023hc4}. This approach is particularly relevant for biological applications where understanding the relationships between nodes (e.g., atoms in a molecule) is critical. By focusing on paths, these networks can capture complex interactions that traditional GNNs might overlook.

Despite these advancements, challenges remain. For instance, while Satorras et al. (2021) and Michel et al. (2023) have made significant strides in enhancing expressiveness, the computational efficiency of these models is still a concern, especially when applied to large biological datasets. Additionally, the work by Chamberlain et al. (2022) on subgraph sketching highlights the need for efficient methods that maintain the expressiveness of GNNs without incurring high computational costs \cite{chamberlain2022fym}. 

In conclusion, while significant progress has been made in adapting GNNs for biological data analysis, particularly in drug discovery and protein-protein interaction prediction, there are still unresolved issues regarding expressiveness and efficiency. Future research may benefit from exploring hybrid models that combine the strengths of various GNN architectures and pooling strategies, as well as addressing the computational challenges inherent in analyzing large-scale biological networks.