\subsection{Graph Embeddings}

Graph embeddings have emerged as a crucial technique for transforming graph structures into continuous vector spaces, enabling the application of deep learning methods, particularly Graph Neural Networks (GNNs), to leverage rich relational information inherent in graph data. The challenge of effectively representing graph data while maintaining the relationships between nodes and edges has led to significant advancements in this area.

Early methods such as DeepWalk \cite{perozzi2014deepwalk} and node2vec \cite{grover2016node2vec} laid the groundwork for graph embeddings by employing random walk techniques to capture node relationships. DeepWalk introduced a novel approach by leveraging the skip-gram model from natural language processing to learn embeddings based on local node neighborhoods, effectively treating walks as sentences. Node2vec extended this concept by introducing a flexible framework that allows for biased random walks, enabling the capture of both breadth-first and depth-first search patterns, thus enhancing the quality of learned embeddings. These pioneering works demonstrated the potential of embeddings to facilitate downstream tasks, such as node classification and link prediction, by encoding structural information into low-dimensional spaces.

Building on these foundational methods, subsequent research has sought to address their limitations. For instance, the introduction of Graph Convolutional Networks (GCNs) \cite{kipf2016semi} marked a significant evolution in graph embeddings by incorporating node features into the embedding process. GCNs leverage the spectral graph theory to perform convolution operations on graph data, thus allowing for the aggregation of information from neighboring nodes. However, GCNs face challenges related to over-smoothing, where deeper architectures can lead to indistinguishable node representations \cite{li2018deepergcn}. This limitation prompted further innovations, such as the Personalized PageRank-based approach in the PPNP model \cite{klicpera20186xu}, which decouples the prediction from the propagation mechanism, allowing for deeper layers without the risk of over-smoothing.

Moreover, the need for interpretability and fairness in GNNs has led to the exploration of methods that aim to provide explanations for the learned embeddings and predictions. GNNExplainer \cite{ying2019rza} introduced a framework for generating explanations by identifying important subgraphs and node features that contribute to a GNN's predictions. This work highlighted the importance of understanding not just the embeddings but also the rationale behind predictions, addressing the interpretability gap in earlier embedding methods.

Despite these advancements, challenges remain in ensuring the robustness and fairness of GNNs. Recent studies, such as those by FairGNN \cite{dai2020p5t} and EDITS \cite{dong2021qcg}, emphasize the need for debiasing techniques that can mitigate discrimination in GNNs, particularly in sensitive applications where biased predictions can have severe consequences. These works address the limitations of earlier methods by proposing model-agnostic frameworks that focus on re-balancing input data rather than modifying GNN architectures directly.

In conclusion, while graph embeddings have significantly advanced the capabilities of GNNs, ongoing research is essential to address unresolved issues related to robustness, interpretability, and fairness. Future directions may include the development of more sophisticated embedding techniques that incorporate causal reasoning or adaptive mechanisms to better handle diverse graph structures and ensure equitable outcomes across different demographic groups.
```