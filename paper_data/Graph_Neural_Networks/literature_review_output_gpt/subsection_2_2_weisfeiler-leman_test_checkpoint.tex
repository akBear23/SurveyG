```latex
\subsection{Weisfeiler-Leman Test}

The Weisfeiler-Leman (WL) test is a pivotal concept in graph theory that serves as a benchmark for evaluating the expressiveness of graph-based models, including Graph Neural Networks (GNNs). It offers a systematic approach to distinguishing between non-isomorphic graphs, which is essential for understanding the limitations and capabilities of GNN architectures. The WL test's significance lies in its ability to reveal how well different models can capture the structural nuances of graphs, particularly in the context of higher-order relationships.

The foundational work by \cite{zhang2018kdl} introduced the SEAL framework, which employs subgraph embeddings for link prediction. This approach demonstrated that GNNs can effectively learn from local structures, yet it primarily depended on predefined heuristics, which may hinder adaptability to diverse graph topologies. This limitation was further explored by \cite{ma2021sim}, who empirically challenged the notion that GNNs necessitate homophily for effective performance. Their findings indicated that standard Graph Convolutional Networks (GCNs) could achieve satisfactory results on certain heterophilous graphs, suggesting that the implications of the WL test might not universally apply. This raises critical questions about the expressiveness of GNNs in capturing complex graph structures that deviate from the assumptions of the WL test.

In addressing the over-smoothing problem, which has emerged as a significant concern in deeper GNN architectures, \cite{cai2020k4b} and \cite{zhou20213lg} provided essential insights. They emphasized the necessity for a theoretical framework that enables GNNs to maintain discriminative power across layers. Specifically, \cite{zhou20213lg} proposed a Dirichlet energy constrained learning approach, which aligns the principles of the WL test with practical GNN design. This framework effectively addresses the convergence of node representations, a challenge that previous models struggled to overcome, thereby enhancing the theoretical grounding of GNNs.

The exploration of heterophily was further advanced by \cite{luan202272y}, who introduced the Adaptive Channel Mixing (ACM) framework. This innovative approach allows GNNs to select appropriate filters for different nodes based on their structural characteristics, thereby recognizing the nuanced relationships among nodes that the WL test's binary classification of graph isomorphism may overlook. The ACM framework exemplifies how GNNs can leverage both aggregation and diversification mechanisms, enhancing their performance in heterophilous environments.

Moreover, the work by \cite{xia20247w9} on Cluster Information Transfer (CIT) underscores the importance of invariant representation learning in GNNs. CIT addresses structural shifts without requiring explicit knowledge of graph generation processes, which resonates with the insights derived from the WL test. This approach emphasizes the necessity for GNNs to generalize across diverse structural contexts, thereby broadening their applicability.

In conclusion, while the WL test provides a valuable theoretical foundation for understanding graph isomorphism and expressiveness in GNNs, subsequent research has unveiled the complexities inherent in real-world graph structures, such as heterophily and over-smoothing. The evolution of GNN architectures, from SEAL to ACM and CIT, illustrates a shift towards more adaptable and robust models capable of navigating the challenges posed by diverse graph environments. Future research should continue to explore these dimensions, incorporating sophisticated mechanisms that account for the dynamic nature of graph data while maintaining the theoretical rigor established by the WL test. This ongoing inquiry is crucial for advancing the field and enhancing the practical utility of GNNs in various applications.
```