\subsection{Traffic and Transportation Networks}

The integration of Graph Neural Networks (GNNs) into traffic and transportation networks has emerged as a promising approach to model the dynamic interactions between vehicles, infrastructure, and users. This subsection reviews the application of GNNs in predicting traffic patterns, optimizing routing, and enhancing urban mobility solutions, particularly within the context of smart city initiatives.

One of the foundational works in this area is the introduction of GNNs for link prediction and social recommendation, which laid the groundwork for understanding complex relational data structures in transportation systems. For instance, the SEAL framework proposed by Zhang et al. \cite{zhang2018kdl} leverages GNNs to learn general graph structure features for link prediction, demonstrating the potential of GNNs to capture intricate relationships that are crucial for traffic modeling. This work highlights the importance of learning from local subgraphs, which is particularly relevant in transportation networks where local interactions can significantly influence overall traffic flow.

Building on this, Fan et al. \cite{fan2019k6u} proposed a GNN framework specifically designed for social recommendation, which can be adapted to traffic networks by integrating user-item interactions with social relations. Their approach emphasizes the need for dual-graph integration, which can be directly applied to model the interactions between different types of entities in transportation systems, such as vehicles and traffic signals. The introduction of attention mechanisms to weigh the contributions of various interactions allows for a more nuanced understanding of how different factors influence traffic patterns.

Further advancements were made by Liu et al. \cite{liu2023v3e}, who addressed the challenge of learning robust GNNs in scenarios with incomplete or weak information. Their D2PT framework, which employs dual-channel propagation, is particularly relevant for transportation networks that often deal with incomplete data due to sensor failures or dynamic changes in traffic conditions. By enabling long-range information propagation, D2PT enhances the ability of GNNs to adapt to varying traffic scenarios, thereby improving the robustness of traffic predictions.

In the context of urban mobility, the work of Long et al. \cite{longa202399q} on temporal graphs provides valuable insights into how GNNs can be adapted to account for the dynamic nature of traffic systems. Their comprehensive survey on temporal GNNs emphasizes the importance of capturing evolving relationships over time, which is critical for modeling traffic flows that change throughout the day. This perspective is crucial for developing GNN architectures that can effectively handle the temporal aspects of traffic data.

Moreover, the introduction of the Cluster Information Transfer (CIT) mechanism by Xia et al. \cite{xia20247w9} offers a novel approach to learning invariant representations in the presence of structure shifts, which can occur in traffic networks due to changing road conditions or accidents. By transferring cluster information while preserving essential node features, CIT enhances the generalization ability of GNNs in dynamic environments, making it a valuable tool for traffic prediction tasks.

Despite these advancements, challenges remain in fully leveraging GNNs for traffic and transportation networks. Issues such as over-smoothing, as discussed by Zhou et al. \cite{zhou20213lg}, continue to pose significant barriers to building deep GNNs capable of capturing long-range dependencies in traffic data. Future research directions should focus on developing more robust GNN architectures that can effectively integrate temporal and spatial information while addressing the inherent challenges of dynamic traffic environments.

In conclusion, the application of GNNs in traffic and transportation networks has shown substantial promise, with various studies contributing to the understanding of how these models can be optimized for real-world scenarios. However, ongoing challenges related to data completeness, temporal dynamics, and over-smoothing highlight the need for continued innovation in this rapidly evolving field.
```