\subsection{Recommender Systems}

Recommender systems, essential for navigating vast digital content and alleviating information overload, have been profoundly transformed by Graph Neural Networks (GNNs) \cite{wu2020dc8, gao2022f3h, he202455s}. The inherent graph structure of user-item interactions, social connections, and item knowledge graphs makes GNNs particularly adept at learning rich latent factors and capturing complex relational patterns. This capability leads to more personalized, accurate, and context-aware recommendations that consistently outperform traditional methods by explicitly modeling collaborative signals and multi-hop relationships \cite{wu2020dc8}.

A pioneering challenge for GNNs in recommendation was scaling to web-scale platforms with billions of users and items. \textbf{PinSage} \cite{ying20189jc} addressed this by developing a novel Graph Convolutional Network (GCN) algorithm specifically for Pinterest's massive graph. PinSage overcame the limitations of prior GCNs by introducing efficient, localized convolutions that operate "on-the-fly" through random walk-based neighborhood sampling and an "Importance Pooling" aggregation. This, combined with a producer-consumer minibatch architecture and a MapReduce inference pipeline, enabled the largest-ever application of deep graph embeddings at the time, demonstrating substantial improvements in offline metrics and user engagement in A/B tests. The technical details of these scaling mechanisms, which are critical for industrial deployment, are further elaborated in Section 5.2.

Beyond sheer scale, enhancing recommendation quality often necessitates integrating heterogeneous information. Social recommender systems, for instance, leverage both user-item interactions and user-user social relations to better understand user tastes, capitalizing on phenomena like homophily and social influence \cite{sharma2022liz}. \textbf{GraphRec} \cite{fan2019k6u} is a notable GNN framework that adeptly integrates user-item interactions, associated opinions (e.g., ratings), and heterogeneous social relations. It learns user latent factors from both "item-space" and "social-space" perspectives, employing opinion embedding vectors and attention mechanisms (item, social, and user attention) to dynamically weigh the contributions of diverse information sources. This approach captures nuanced user preferences and social influence, leading to more comprehensive user representations. The comprehensive survey by \cite{sharma2022liz} further categorizes and reviews GNN-based social recommender systems, highlighting their architectural variations and input types for leveraging social graphs as side information.

GNNs also excel at learning richer latent factors and capturing complex relational patterns from interaction graphs, which is vital for precise recommendations. For instance, in link prediction tasks—a core component of many recommender systems aimed at predicting future interactions—\textbf{SEAL} \cite{zhang2018kdl} leverages GNNs to learn general graph structure features from local enclosing subgraphs. This approach, underpinned by a novel "$\beta$-decaying heuristic theory," theoretically justifies learning high-order features from small local contexts. SEAL significantly outperforms traditional heuristic and latent feature methods by adaptively learning patterns specific to link formation, thereby improving the accuracy of predicting new user-item connections.

The dynamic nature of user preferences and item catalogs necessitates GNNs capable of handling **sequential and session-based recommendation tasks**. Traditional sequential models often struggle with complex, non-linear item transitions within a user session. \textbf{SR-GNN} \cite{wu2018t43} pioneered the application of GNNs to session-based recommendation by modeling item sequences as graph-structured data. It uses GNNs to capture complex item transitions within a session, representing each session as a composition of global preference and current interest via an attention network. Building on this, \textbf{GCE-GNN} \cite{wang2020khd} further enhanced session-based recommendation by learning two levels of item embeddings: session-level from the current session graph and global-level from an aggregated global graph of all sessions. A session-aware attention mechanism recursively incorporates neighbors' embeddings on the global graph, allowing for a more subtle exploitation of item transitions across all sessions. More recently, \textbf{SURGE} \cite{chang2021yyt} transforms loose item sequences into dynamic item-item interest graphs, employing metric learning with $\epsilon$-sparseness for robust graph construction and utilizing cluster- and query-aware graph attentive convolutional layers. Its dynamic graph pooling layer adaptively extracts activated core preferences, effectively modeling evolving user interests from long, noisy historical sequences and outperforming state-of-the-art sequential recommenders.

To further enhance the robustness and data efficiency of GNNs in recommendation, especially in data-scarce or cold-start scenarios, **pre-training strategies** are increasingly important. Self-supervised pre-training, as investigated by \cite{hu2019r47}, can learn generalizable representations from abundant unlabeled graph data (e.g., through context prediction or attribute masking). Such pre-trained GNNs are more robust and adaptable for downstream recommendation tasks, even with limited labeled user feedback. This topic, including foundational self-supervised strategies, is further discussed in Section 5.4.

Despite these significant advancements, several challenges persist in the application of GNNs to recommender systems \cite{wu2020dc8, gao2022f3h, he202455s}. Scaling GNNs to truly massive, dynamic graphs with real-time updates while maintaining high expressivity and interpretability remains an active research area. Developing GNN architectures that can seamlessly handle extreme data sparsity and cold-start problems, particularly for new users or items, continues to be a critical hurdle. Furthermore, understanding and mitigating biases inherent in graph data and GNN models to ensure fair and diverse recommendations is crucial for responsible deployment \cite{wu2020dc8}. Research into individual fairness for GNNs, such as the ranking-based approach of REDRESS \cite{dong202183w} or frameworks like FairGNN \cite{dai2020p5t} that address fairness with limited sensitive attribute information, directly contributes to this goal, as detailed in Section 6.3. The comprehensive surveys by \cite{wu2020dc8, gao2022f3h, he202455s} provide systematic taxonomies and highlight critical challenges in graph construction, network design, optimization, and computational efficiency, thereby guiding future research in this rapidly evolving domain.