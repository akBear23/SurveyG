\subsection{Scientific Domains (Chemistry, Materials, Physics)}

Graph Neural Networks (GNNs) have emerged as a transformative paradigm in scientific research, particularly within chemistry, materials science, and physics, owing to the inherently graph-like nature of molecular and crystal structures. Their capacity to directly learn from structural data, coupled with the ability to capture physical symmetries and interactions, makes them invaluable for accelerating discovery, drug design, and material innovation. This section details the application of GNNs across these domains, focusing on their practical impact in molecular property prediction, interatomic potential modeling, and materials discovery.

One of the most prominent applications of GNNs is in **molecular property prediction**, a cornerstone of drug discovery and chemical engineering. Molecules are naturally represented as graphs, with atoms as nodes and chemical bonds as edges, allowing GNNs to directly process their topology and features. For instance, \cite{carlo2024a3g} demonstrated an attention-based GNN approach for predicting Absorption, Distribution, Metabolism, Excretion, and Toxicity (ADMET) properties directly from Simplified Molecular Input Line Entry System (SMILES) notation. This "bottom-up" method processes information from substructures to the whole molecule, bypassing the computationally expensive retrieval of molecular descriptors and showing effectiveness across various regression and classification tasks crucial for early-stage drug development. However, the landscape of molecular property prediction is complex, and GNNs do not always universally outperform traditional methods. A comprehensive comparison by \cite{jiang2020gaq} across 11 diverse datasets revealed that, on average, descriptor-based models (e.g., SVM, XGBoost, RF, DNN) could still outperform graph-based models (e.g., GCN, GAT, MPNN, Attentive FP) in terms of prediction accuracy and computational efficiency, especially for smaller datasets. This highlights the critical need for careful benchmarking and understanding the specific strengths and weaknesses of GNNs in different chemical contexts, suggesting that while GNNs offer a powerful framework, their superiority is not guaranteed across all tasks and data regimes.

Beyond static property prediction, GNNs are revolutionizing **the modeling of interatomic potentials (IAPs)**, which are essential for high-fidelity molecular dynamics (MD) simulations. Traditional *ab initio* methods like Density Functional Theory (DFT) are accurate but computationally prohibitive for large systems or long timescales. GNNs offer a data-driven alternative to learn IAPs from *ab initio* calculations, enabling accelerated simulations. A significant advancement in this area is Neural Equivariant Interatomic Potentials (NequIP) by \cite{batzner2021t07}. NequIP employs E(3)-equivariant convolutions, which ensure that the learned potential respects fundamental physical symmetries (rotations, translations, reflections) without relying on computationally expensive higher-order representations. This approach achieved state-of-the-art accuracy on diverse molecules and materials, demonstrating remarkable data efficiency by outperforming existing models with up to three orders of magnitude fewer training data. Such data efficiency challenges the notion that deep neural networks always require massive datasets and enables the construction of accurate potentials using high-level quantum chemical theories. Complementing NequIP, \cite{klicpera20215fk} introduced GemNet, a universal directional GNN for molecules. GemNet's innovation lies in its explicit incorporation of full geometric information, including interatomic distances, angles, and crucially, dihedral angles, through a novel two-hop message passing scheme based on directed edge embeddings. This architectural design, coupled with a theoretical proof of universality for spherical representations, led to significant performance improvements (e.g., 20-41\% reduction in MAE for force predictions on MD datasets like COLL, MD17, and OC20), particularly for challenging molecules with dynamic, non-planar geometries. These advancements in equivariant GNNs are pivotal for accurately describing complex atomic interactions and dynamics over extended time scales, thereby accelerating materials design and drug discovery.

In **materials science**, GNNs are instrumental in designing new materials and analyzing crystal structures. Crystal structures, with their periodic arrangements of atoms, can be effectively represented as graphs with periodic boundary conditions. GNNs are used to predict properties of novel materials, guide their synthesis, and identify stable structures. The integration of GNNs with other advanced AI paradigms is also emerging, as demonstrated by \cite{li2024gue} with Hybrid-LLM-GNNs for enhanced materials property prediction. This approach combines the structural learning capabilities of GNNs with the semantic reasoning of Large Language Models (LLMs), showcasing a promising direction for leveraging multi-modal data and contextual information to improve predictive accuracy in complex materials informatics tasks.

In summary, GNNs have profoundly impacted scientific domains by offering a powerful framework to learn directly from the intrinsic graph-like nature of molecules and materials. The development of geometrically equivariant architectures, such as NequIP and GemNet, has been crucial for capturing physical symmetries and interactions efficiently, leading to data-efficient and accurate predictions. While challenges remain, including scaling to extremely large and heterogeneous systems, integrating diverse multi-modal data, and enhancing interpretability to provide mechanistic insights, the continuous evolution of GNNs promises further acceleration of scientific discovery and innovation across chemistry, materials science, and physics.