\subsection{Multi-modal and Semantic Understanding Tasks}

While Graph Neural Networks (GNNs) have traditionally excelled in tasks driven by structural patterns and local attribute propagation, a pivotal frontier in their evolution lies in achieving deeper semantic understanding and effectively processing multi-modal data. This involves moving beyond purely topological insights to integrate GNNs with other advanced AI paradigms, particularly Large Language Models (LLMs) and computer vision models, to imbue them with real-world semantic knowledge, enable cross-domain reasoning, and facilitate robust generalization, including zero-shot capabilities. This section highlights cutting-edge applications where such integrations push GNNs into new domains of intelligence and versatility.

The foundation for leveraging GNNs in semantically rich and multi-modal contexts often builds upon advanced knowledge transfer paradigms, such as the pre-training and prompt-based adaptation strategies discussed in Section 5.4. These methods enable GNNs to learn generalizable representations from abundant unlabeled data and efficiently adapt to diverse downstream tasks with limited supervision. With this established capability, the focus shifts to how GNNs can interpret and interact with external semantic information.

One significant avenue for semantic understanding involves the application of GNNs to text-rich graphs and Knowledge Graphs (KGs). In text classification, GNNs are increasingly employed to model complex relationships between words, documents, and entire corpora, moving beyond sequential models to capture global, contextual-aware word relations \cite{wang2023wrg}. By constructing graphs where nodes represent words or documents and edges denote relationships (e.g., co-occurrence, semantic similarity), GNNs can learn rich representations that encode the semantic context of text. Initial node features are often derived from powerful language models (e.g., BERT, ELMo), which GNNs then refine by propagating information across the graph structure, thereby enhancing the semantic understanding for classification tasks \cite{wang2023wrg}.

Knowledge Graphs, which explicitly represent factual information and relationships between entities, are inherently multi-relational graphs that demand deep semantic understanding. GNNs have become indispensable for various KG tasks, including link prediction, knowledge graph alignment, and knowledge graph reasoning \cite{ye20226hn}. By learning embeddings for entities and relations, GNNs enable models to infer missing links or answer complex queries, effectively performing symbolic reasoning over structured semantic knowledge. For instance, GNNs can identify complex interaction patterns in drug-drug interaction graphs or enhance recommender systems by leveraging rich semantic relationships between items and users within a KG \cite{ye20226hn}. Beyond explicit KGs, methods like Knowledge-Enhanced GNNs (KE-GNN) integrate propositional logic rules into the learning process, mapping knowledge into a semantic space and using a teacher-student scheme to guide GNN predictions, thereby enhancing semantic understanding and robustness for tasks like fault scenario identification in communication networks \cite{zhao2024aer}. These approaches demonstrate how GNNs, when augmented with structured knowledge, can achieve more accurate and interpretable semantic reasoning.

The synergy between GNNs and Large Language Models (LLMs) represents a cutting-edge direction for deeper semantic understanding. LLMs, with their vast world knowledge and powerful inference capabilities, can significantly enhance GNNs. One approach is to leverage LLMs to enrich node features or improve graph structure. For example, LLMs can generate more informative textual embeddings for graph nodes, providing a richer semantic context than traditional attribute vectors. Furthermore, LLMs can contribute to the robustness of GNNs against adversarial attacks by identifying malicious edges or inferring missing important connections to recover a more robust graph structure \cite{zhang2024370}. The LLM4RGNN framework, for instance, distills GPT-4's inference capabilities into a local LLM to identify perturbed edges, showcasing how LLMs can semantically "reason" about graph integrity and improve GNN resilience, even in highly perturbed scenarios \cite{zhang2024370}.

A particularly groundbreaking paradigm for integrating GNNs and LLMs is multi-modal prompt learning, which aims to align independently pre-trained GNNs with the rich semantic embedding spaces of LLMs, even under extremely weak text supervision. This approach enables GNNs to perform CLIP-style zero-shot generalization, classifying unseen graph categories by leveraging the semantic understanding encoded in LLMs. A notable example is **Morpher (Multi-modal Prompt Learning for Graph Neural Networks)** \cite{li202444f}. Morpher addresses critical challenges in graph-text alignment, such as data scarcity and the weak nature of text supervision (e.g., single-word labels), which hinder joint pre-training. Its innovations include an improved, stable graph prompt design that carefully balances cross-connections between prompt tokens and input graph nodes with the original graph's inner-connections, preventing prompt features from overwhelming the graph structure. Coupled with tunable text prompts and a cross-modal projector, Morpher aligns graph and text representations in a shared semantic space using contrastive learning, keeping the GNN and LLM backbones frozen. This allows GNNs to dynamically adapt to semantic cues from LLMs, achieving significant accuracy improvements in few-shot and zero-shot graph classification tasks across diverse datasets (e.g., molecular, bioinformatic, citation networks) \cite{li202444f}. Morpher's ability to imbue GNNs with real-world semantic knowledge without extensive paired data or fine-tuning represents a significant leap towards more intelligent and versatile graph-based AI systems.

Beyond text, GNNs are also crucial for multi-modal understanding in computer vision, particularly in tasks involving vision and language. In these applications, GNNs model the relationships between visual entities (e.g., objects in an image, points in a 3D scene) and their linguistic descriptions. For instance, GNNs are used to construct scene graphs from images, where nodes represent objects and edges represent their interactions or attributes, which can then be used for image captioning, visual question answering, or complex scene understanding \cite{chen2022mmu}. By capturing the relational structure inherent in visual data, GNNs provide a powerful mechanism to integrate visual and linguistic information, enabling models to reason about complex visual scenes and generate semantically coherent textual descriptions. The survey by \cite{chen2022mmu} highlights how GNNs and Graph Transformers are applied across various computer vision tasks, including those that explicitly bridge vision and language modalities.

In conclusion, the integration of GNNs with other advanced AI paradigms, particularly LLMs and computer vision models, is rapidly expanding their capabilities beyond purely structural analysis. From leveraging GNNs for semantic understanding in text classification and Knowledge Graphs to employing LLMs for enhancing GNN robustness and enabling multi-modal prompt learning for zero-shot generalization, these applications are transforming GNNs into more semantically aware and versatile systems. Future research will likely explore more complex multi-modal reasoning tasks, develop more robust and interpretable multi-modal prompt designs, and extend zero-shot generalization capabilities to a wider array of graph learning problems, ultimately fostering GNNs that can truly "understand" and interact with the world's knowledge across diverse modalities.