\subsection*{Building Deeper GNN Architectures}
The development of deep Graph Neural Networks (GNNs) is crucial for capturing complex, long-range dependencies in graph-structured data, yet it is significantly hampered by fundamental challenges such as 'oversmoothing' and 'over-squashing'. Oversmoothing causes node representations to become indistinguishable with increasing layers, akin to Laplacian smoothing, while over-squashing limits information flow over long distances by compressing exponentially growing receptive fields into fixed-size vectors. Addressing these issues is paramount for enhancing the representational capacity of GNNs.

Early efforts to enable deeper GNNs primarily focused on decoupling the feature transformation from the propagation step. \cite{klicpera20186xu} pioneered this with Personalized Propagation of Neural Predictions (PPNP) and its scalable approximation APPNP, which leveraged Personalized PageRank to allow arbitrary propagation depth without increasing the neural network's parameter count per layer, effectively mitigating oversmoothing. Building on this, \cite{liu2020w3t} further analyzed the performance degradation in deep GNNs, arguing that the entanglement of transformation and propagation was the primary culprit at moderate depths, and demonstrated that decoupling allowed for significantly deeper models before severe oversmoothing occurred.

To provide a more rigorous understanding of oversmoothing, \cite{cai2020k4b} introduced a novel analytical technique based on Dirichlet energy, proving its exponential convergence to zero with increasing layers across a broader range of non-linearities. This theoretical foundation was further expanded by \cite{rusch2023xev}, who provided an axiomatic, unified definition of oversmoothing, emphasizing exponential convergence of node-similarity measures like Dirichlet energy, and critically evaluating various mitigation strategies. Leveraging these insights, \cite{zhou20213lg} proposed Energetic Graph Neural Networks (EGNNs), which employ a Dirichlet energy-constrained learning principle to guide deep GNN training through orthogonal weight control, lower-bounded residual connections, and Shifted ReLU (SReLU) activation, enabling models with up to 64 layers to achieve state-of-the-art performance.

Beyond direct oversmoothing mitigation, other architectural innovations have sought to enhance GNN depth and expressivity. \cite{bianchi20194ea} introduced Graph Neural Networks with Convolutional ARMA Filters, offering a more flexible frequency response than polynomial filters and capturing longer-range dynamics with fewer parameters through a recursive Graph Convolutional Skip (GCS) layer, implicitly aiding deeper architectures. To address the memory bottleneck that still limits *very* deep GNNs, \cite{li2021orq} introduced Grouped Reversible GNNs, which reduce memory complexity for activations to be independent of network depth, enabling the training of GNNs with over 1000 layers.

The problem of over-squashing, distinct from oversmoothing, was formally diagnosed by \cite{alon2020fok}, who demonstrated that the exponential growth of a node's receptive field leads to information compression and loss in fixed-size message passing vectors, particularly for tasks requiring long-range interactions. As a simple mitigation, they proposed adding a "fully-adjacent layer" to alleviate this bottleneck. Complementing this, \cite{zeng2022jhz} introduced a novel principle to decouple the depth and scope of GNNs, proposing SHADOW-GNNs that apply deep GNNs on shallow, localized subgraphs. This approach theoretically prevents oversmoothing and enhances expressivity beyond the 1-Weisfeiler-Lehman (1-WL) test while significantly reducing computational costs.

More recently, fundamentally different approaches have emerged. \cite{wang2024oi8} proposed Random Walk with Unifying Memory (RUM), a non-convolutional GNN architecture that processes graph information via random walks and RNNs. RUM jointly tackles limited expressiveness, over-smoothing, and over-squashing, demonstrating superior theoretical properties and competitive empirical performance. Building on continuous GNNs, \cite{kang2024fsk} introduced FROND, which leverages fractional calculus to generalize the integer-order differential equations, allowing GNNs to model non-local, memory-dependent dynamics and algebraically mitigate oversmoothing, thereby enhancing expressivity and robustness. Furthermore, \cite{finkelshtein202301z} introduced Cooperative Graph Neural Networks (CO-GNNs), where nodes dynamically choose communication actions, leading to flexible, asynchronous information flow that addresses information bottlenecks and enhances expressive power beyond 1-WL. Finally, \cite{geisler2024wli} proposed Spatio-Spectral Graph Neural Networks (S2GNNs), which synergistically combine local spatial message passing with global spectral filtering, provably vanquishing over-squashing and achieving superior approximation bounds for long-range interactions.

Despite these advancements, building truly robust and universally applicable deep GNNs remains an active area of research. Challenges persist in balancing the computational cost of advanced architectures with the need for high expressivity, especially for extremely large and dynamic graphs. Future directions include developing more adaptive and interpretable mechanisms for controlling information flow, exploring novel mathematical frameworks beyond traditional calculus, and designing architectures that can dynamically adjust their depth and receptive field based on the specific task and graph characteristics.