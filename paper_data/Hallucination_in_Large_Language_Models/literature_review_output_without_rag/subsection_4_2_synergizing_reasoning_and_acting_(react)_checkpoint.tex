\subsection{Synergizing Reasoning and Acting (ReAct)}
Large Language Models (LLMs) frequently suffer from ungrounded hallucinations, generating plausible but factually incorrect information due to their reliance on internal, potentially outdated or fallacious, knowledge \cite{maynez2020h3q}. To address this, a critical paradigm shift involves intertwining an LLM's internal reasoning ("thoughts") with external actions (e.g., API calls, tool use) to dynamically gather information and verify facts, thereby enhancing factual grounding and robustness.

A seminal contribution in this area is the \textit{ReAct} framework, introduced by \cite{yao20229uz}, which synergizes reasoning and acting in language models. ReAct prompts LLMs to generate interleaved verbal reasoning traces ("thoughts") and task-specific actions, allowing the model to dynamically formulate plans ("reason to act") and interact with external environments (e.g., Wikipedia API) to gather information and refine its understanding ("act to reason"). This approach directly mitigates hallucination by grounding LLM responses in verifiable external information, demonstrating significant few-shot performance gains in knowledge-intensive tasks like HotpotQA and FEVER, as well as interactive decision-making benchmarks \cite{yao20229uz}. While ReAct showcased the power of this synergy, its prompting setup had limitations for highly complex behaviors and faced challenges with data scarcity for large-scale annotation.

Building upon the principle of dynamic reasoning and external information retrieval, \cite{trivedi2022qsf} proposed \textit{Interleaving Retrieval with Chain-of-Thought Reasoning (IRCoT)} for knowledge-intensive multi-step questions. IRCoT refines the "act to reason" component by iteratively using intermediate Chain-of-Thought sentences as dynamic queries for knowledge retrieval, thereby overcoming the limitations of static, one-shot retrieval. This adaptive information-seeking mechanism significantly improved retrieval recall and reduced factual errors in generated reasoning steps by up to 50\%, even outperforming much larger models with simpler retrieval methods \cite{trivedi2022qsf}. IRCoT thus demonstrated a more granular and efficient way to integrate external knowledge during complex reasoning, making the grounding process more precise.

Beyond merely incorporating external information, ensuring the verifiability and trustworthiness of LLM outputs is paramount. \cite{gao2023ht7} addressed this by enabling LLMs to generate text with explicit citations through their \textit{ALCE (Automatic LLMs’ Citation Evaluation)} benchmark. ALCE provides a reproducible framework and novel NLI-based metrics to evaluate not only the fluency and correctness of generated text but also the quality of citations, including citation recall and precision \cite{gao2023ht7}. This work directly tackles the problem of ungrounded hallucination by providing a mechanism for LLMs to explicitly link their statements to supporting evidence, thereby enhancing the trustworthiness and interpretability of their responses, a crucial step in operationalizing the external grounding achieved by ReAct-like paradigms.

Despite the promise of these retrieval-augmented and reasoning-enhanced approaches, a critical evaluation of their robustness and limitations is essential. \cite{chen2023h04} provided a comprehensive assessment of Retrieval-Augmented Generation (RAG) systems, which encompass ReAct-like frameworks, through their \textit{RGB} benchmark. RGB systematically diagnoses LLMs' capabilities across four crucial dimensions: Noise Robustness, Negative Rejection, Information Integration, and Counterfactual Robustness \cite{chen2023h04}. Their findings revealed that even state-of-the-art LLMs struggle significantly with these challenges, often failing to reject answering when no relevant information is available, confusing similar noisy information, or even prioritizing factually incorrect retrieved information over their own internal knowledge, despite explicit warnings \cite{chen2023h04}. This highlights that simply providing external tools or dynamic retrieval does not automatically guarantee robust, hallucination-free behavior.

In conclusion, synergizing reasoning and acting, exemplified by the ReAct framework, represents a significant advancement in mitigating LLM hallucination by providing mechanisms for external validation and dynamic information gathering. Approaches like IRCoT have refined this synergy for complex multi-step reasoning, while benchmarks such as ALCE have pushed for greater verifiability through explicit citations. However, as critically evaluated by RGB, current systems still face substantial challenges in effectively handling noisy, irrelevant, or conflicting external information. These limitations underscore that while external grounding is vital, the complete elimination of hallucination remains an elusive goal, a theoretical inevitability for all computable LLMs \cite{xu2024n76}. Future research must therefore focus on enhancing LLMs' ability to critically evaluate retrieved information, integrate diverse knowledge sources robustly, and develop more sophisticated error-handling mechanisms in interactive environments, moving towards more dependable and trustworthy AI agents.