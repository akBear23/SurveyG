\subsection{Knowledge Graph Integration for Trustworthiness}

Large Language Models (LLMs) often suffer from factual inaccuracies and logical inconsistencies, commonly referred to as hallucinations, which severely undermine their trustworthiness. Strategic integration of structured Knowledge Graphs (KGs) with LLMs offers a robust solution by providing a reliable, up-to-date, and verifiable source of information, thereby serving as a powerful grounding mechanism against such issues.

Early efforts to mitigate LLM hallucinations recognized the critical need for external knowledge to ground their responses. For instance, \cite{trivedi2022qsf} demonstrated that interleaving retrieval with Chain-of-Thought (CoT) reasoning could significantly improve factual accuracy in knowledge-intensive multi-step questions. Their IRCoT framework dynamically uses intermediate CoT steps to generate context-aware queries for retrieving relevant paragraphs, grounding the reasoning process and reducing factual errors by up to 50\%. However, these foundational approaches often relied on unstructured text retrieval, which, while effective, could still introduce noise or outdated information, highlighting the need for more structured and verifiable knowledge sources.

Building upon the broader Retrieval-Augmented Generation (RAG) paradigm, which generally involves fetching relevant information from an external corpus to inform LLM responses \cite{gao20232zb}, recent research has increasingly focused on leveraging the inherent structure and verifiability of Knowledge Graphs. \cite{sui20242u1} directly addresses how KGs can enhance LLM trustworthiness by proposing a unified framework that combines "Graph-guided retrieval" and "Graph-guided generation." This approach enables LLMs to query and integrate structured facts from KGs, leading to more accurate and logically consistent answers in open-ended question answering tasks, and introduces the OKGQA benchmark to evaluate such KG-augmented models, even under perturbed KG conditions. This represents a significant step beyond generic text retrieval by providing a rich, semantically structured context that is less prone to misinterpretation or hallucination.

Beyond mere factual accuracy, trustworthiness also encompasses logical consistency, especially in complex reasoning tasks. While LLMs can struggle with maintaining logical coherence, KGs, by their very nature, encode explicit relationships and constraints that can be leveraged for verification. \cite{ghosh2024tj5} emphasizes the importance of evaluating LLM logical consistency in fact-checking, proposing new logical fact-checking (LFC) datasets and quantitative measures to assess their performance on complex propositional logic queries. Although this work does not explicitly integrate KGs, the structured nature of KGs makes them an ideal candidate for providing the ground truth and relational context necessary to enforce and verify such logical consistency in LLM outputs. Further enhancing verifiability, \cite{oh2024xa3} introduced ERBench, a benchmark that utilizes relational databases (conceptually akin to KGs in their structured representation of entities and relationships) to automatically verify not only the LLM's final answers but, crucially, its *rationales*. This capability to scrutinize the reasoning path against structured knowledge is paramount for building truly transparent and trustworthy AI systems, moving beyond simple output correctness to verifiable logical soundness.

In conclusion, the integration of Knowledge Graphs represents a pivotal advancement in addressing the trustworthiness challenges of LLMs. By providing a structured, verifiable, and logically consistent source of external knowledge, KGs enable LLMs to move beyond mere fluency to produce responses that are factually accurate, logically sound, and inherently more reliable. This strategic integration, encompassing graph-guided retrieval, generation, and rationale verification, lays a robust foundation for developing next-generation LLMs that are not only powerful but also transparent and trustworthy in their knowledge-intensive and complex reasoning capabilities. Ongoing challenges include the scalability of KG construction and maintenance, and the seamless, real-time integration of dynamic KGs with evolving LLM architectures.