\subsection{Evaluation Benchmarks for LVLMs, LALMs, and VLLMs}

The rigorous assessment of hallucinations in multimodal models, encompassing Large Vision-Language Models (LVLMs), Large Audio-Language Models (LALMs), and Vision-Language Models (VLLMs), necessitates the development of specialized and nuanced evaluation benchmarks. Traditional metrics often fall short in capturing the complex factual inconsistencies and fabricated information generated by these models, driving the community to innovate sophisticated frameworks for hallucination detection.

Early efforts began to address object-level hallucinations, particularly in emerging modalities. For instance, \cite{Li2023Object} pioneered a benchmark specifically designed for Large Audio-Language Models (LALMs), focusing on evaluating object hallucination by analyzing generated audio descriptions for mentions of non-existent objects. This work established a foundational approach to quantifying spurious object references in a modality distinct from vision. Building upon the need for more granular evaluation in vision-language models, \cite{Chen2023Freeform} introduced an object-centric benchmark tailored for free-form text generations by LVLMs. This framework leverages object detection and grounding techniques to verify the factual consistency of mentioned objects and their attributes against the visual input, moving beyond simple presence or absence to assess the accuracy of descriptive details.

As models grew more capable, the scope of hallucination expanded beyond mere object presence to complex relationships. To address this, \cite{Wang2023Relation} developed Tri-Eval, a novel triplet-level evaluation framework specifically designed to detect and quantify relational hallucinations in LVLMs. This benchmark meticulously examines subject-predicate-object relationships within the generated text, identifying instances where the model fabricates or misrepresents connections between entities depicted in the image, thus offering a more sophisticated measure of factual accuracy. Further refining the understanding of hallucination types, \cite{Zhao2023VLLMHallucination} proposed a comprehensive framework to differentiate between intrinsic and extrinsic hallucinations in VLLMs. Intrinsic hallucinations contradict the visual input, while extrinsic ones are plausible but ungrounded in the image, providing a critical distinction for diagnosing model failures and guiding future improvements.

Beyond general-purpose evaluation, the criticality of model reliability in sensitive domains has led to the creation of domain-specific benchmarks. \cite{Gupta2024Medical} introduced Med-Eval, a specialized benchmark for evaluating LVLMs within medical contexts. This work highlights the paramount importance of factual accuracy and hallucination detection in applications such as clinical report generation and diagnostic assistance, where inaccuracies can have severe consequences, emphasizing the need for expert-curated datasets and domain-specific evaluation criteria.

Furthermore, the robustness of these models under adverse conditions is a significant concern. \cite{Zhang2024Robustness} investigated the resilience of LVLMs and VLLMs by benchmarking their performance against various visual perturbations, including noise, occlusion, and adversarial attacks. Their findings quantify the increased susceptibility to hallucination under such challenging inputs, underscoring the need for models that maintain factual consistency even when confronted with imperfect or manipulated visual data. Finally, a crucial aspect of model evaluation is distinguishing true visual understanding from mere memorization of training data. \cite{Lee2024Understanding} addressed this by introducing a benchmark designed to test true visual understanding in LVLMs. This framework employs novel, out-of-distribution visual concepts and compositional reasoning tasks to ascertain whether models can generalize their knowledge rather than simply recalling learned patterns, ensuring that evaluations reflect genuine comprehension.

In conclusion, the evolution of evaluation benchmarks for LVLMs, LALMs, and VLLMs reflects a growing understanding of the multifaceted nature of hallucinations. While significant progress has been made in developing frameworks for object-level, relational, intrinsic/extrinsic, domain-specific, and robustness-focused evaluations, challenges remain. Future research must focus on dynamic, adaptive benchmarks that can assess emergent hallucination types, better align with human perception of factual accuracy, and continuously evolve with the increasing complexity and capabilities of multimodal AI models.