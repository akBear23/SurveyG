\subsection{The Critical Impact and Significance of Hallucination}

The pervasive phenomenon of hallucination, where Large Language Models (LLMs) generate plausible but factually incorrect or unfaithful information, presents a significant challenge to their practical deployment and trustworthiness across diverse sectors \cite{rejeleene2024okw, ahmadi2024j88, li2025qzg}. This issue extends beyond mere technical imperfection, fundamentally undermining user confidence, propagating misinformation, and introducing substantial risks in sensitive applications such as healthcare, legal analysis, and financial advising \cite{liu2024gxh}. Addressing hallucination is therefore a crucial aspect of responsible AI development and fostering reliable human-AI collaboration.

The erosion of user confidence is a direct and immediate consequence of hallucination. When LLMs consistently produce convincing but false information, users quickly lose trust in their reliability, perceiving them as unreliable sources rather than intelligent assistants. This can be particularly insidious due to phenomena like "sycophantic hallucination," where models generate answers that align with misleading keywords provided by users, thereby amplifying existing biases or misinformation and further eroding confidence in the model's objectivity and factual grounding \cite{rrv2024gw0}. The propagation of misinformation, whether intentional or unintentional, carries severe societal consequences, impacting public discourse, decision-making, and the integrity of information ecosystems \cite{liu2024gxh}. This highlights the critical need for LLMs to not only be accurate but also to be perceived as trustworthy, a perception directly threatened by hallucinatory outputs.

The most critical implications of hallucination manifest in high-stakes domains where factual accuracy is paramount. In \textbf{healthcare}, for instance, LLM hallucinations can lead to dangerous outcomes. Studies have shown that LLMs are highly susceptible to adversarial hallucination attacks in clinical decision support, frequently elaborating on fabricated clinical details embedded in prompts, with hallucination rates ranging from 50\% to 82\% across various models \cite{omar2025cc3, omar2025us3}. Such errors could result in misdiagnoses, incorrect treatment recommendations, or the omission of critical patient information, directly jeopardizing patient safety. Furthermore, in pharmacovigilance, LLM inaccuracies can lead to "never event" errors, such as incorrect drug identification or misrepresentation of adverse event reports, which are wholly preventable and unacceptable in medical contexts \cite{hakim2024d4u}. These findings underscore the urgent need for robust guardrails and verification mechanisms before LLMs can be safely integrated into clinical workflows.

Similarly, in \textbf{legal analysis}, the generation of fabricated legal precedents, non-existent statutes, or incorrect interpretations of law by LLMs can have profound repercussions, leading to flawed legal advice, erroneous court filings, and significant legal disputes \cite{li2025qzg}. The \textbf{financial sector} also faces substantial risks, with LLM-generated misinformation potentially causing incorrect investment advice, market manipulation, or significant economic losses \cite{li2025qzg}. Empirical examinations of LLMs in financial tasks, such as abbreviation recognition, term explanations, and stock price queries, reveal severe hallucination tendencies, with off-the-shelf models often providing outdated or incorrect information \cite{kang20238j0}. The plausible nature of hallucinatory content makes it particularly insidious, as users may not immediately discern the falsehoods, especially in complex or specialized domains where they lack expert knowledge.

Beyond these high-stakes professional domains, the consequences of hallucination are also felt in areas such as \textbf{education}, where students may rely on faulty information for learning, and in \textbf{journalism}, where it threatens to accelerate the spread of sophisticated disinformation and erode public trust in news sources \cite{liu2024gxh}. \textbf{Scientific research} also faces risks, as LLMs might generate plausible but incorrect hypotheses or summaries, leading to misdirection or wasted effort. The widespread impact across these diverse sectors underscores that hallucination is not merely a technical glitch but a fundamental impediment to the beneficial and ethical integration of AI into society.

The significance of hallucination is further amplified by a growing understanding of its fundamental nature. While a detailed theoretical proof will be discussed in Section 2.3, it is important to acknowledge here that emerging arguments suggest hallucination may be an inherent and inevitable characteristic of computable LLMs \cite{li2025qzg}. This perspective reframes the problem not as a temporary bug to be eliminated, but as an enduring aspect that necessitates continuous management and mitigation strategies. Some researchers even propose reframing AI hallucinations as a distinctive feature rather than a mere limitation, arguing that attempts to completely eliminate them might inadvertently lead to increased model rigidity \cite{hamid2024pwn}. Regardless of whether it is viewed as an inherent limitation or a feature to be managed, this fundamental aspect makes the consequences of hallucination even more critical, demanding a proactive and ongoing commitment to ensuring reliability and safety.

In light of these profound implications, mitigating hallucination is presented not merely as a technical challenge but as a crucial imperative for responsible AI development and fostering reliable human-AI collaboration. The pervasive nature of hallucination, as highlighted by \textcite{rejeleene2024okw}, necessitates robust solutions that span from foundational understanding to sophisticated detection and mitigation strategies. Having established the critical impact of hallucination, the following sections of this review will delve into its theoretical underpinnings, beginning with a systematic taxonomy of the phenomenon, followed by an exploration of its root causes, detection methodologies, and various mitigation strategies.