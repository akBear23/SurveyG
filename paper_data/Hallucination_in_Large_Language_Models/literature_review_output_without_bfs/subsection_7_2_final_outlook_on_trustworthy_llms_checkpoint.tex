\subsection{Final Outlook on Trustworthy LLMs}

The trajectory of Large Language Model (LLM) research, particularly concerning hallucination, culminates in a crucial paradigm shift: from the unattainable goal of complete eradication to the imperative of robust, intelligent management. The theoretical proof by \cite{xu2024n76}, demonstrating the inherent inevitability of hallucination for all computable LLMs, serves as a foundational anchor for this outlook. It underscores that future progress hinges not on eliminating this intrinsic characteristic, but on developing sophisticated mechanisms for its detection, mitigation, and, most critically, a deeper causal understanding of its origins. This ensures LLMs can be deployed beneficially and reliably, fostering confidence in their increasingly widespread applications.

A primary pillar of this forward-looking perspective involves **deepening the causal understanding of model failures through white-box analysis**. While early research focused on defining and categorizing hallucinations \cite{maynez2020h3q}, and subsequent efforts developed black-box detection methods like \textit{SelfCheckGPT} \cite{manakul20236ex} and its successors \cite{yang20251dw}, the future demands a more profound mechanistic insight. Surveys like \cite{ye2023yom} have begun to systematically attribute hallucination to factors spanning data collection, knowledge gaps, and optimization processes. Critically, recent work by \cite{mckenna2023pzc} has unveiled specific biases, such as attestation and relative frequency biases, that drive hallucination in inference tasks, revealing how LLMs often rely on superficial statistical patterns rather than robust logical reasoning. Further, white-box approaches like \textit{PoLLMgraph} \cite{zhu2024hll} are pioneering the detection and forecasting of hallucination by analyzing LLM's internal state transition dynamics. This move towards understanding *why* models hallucinate, by probing internal representations and decision-making processes, will enable the development of more principled prevention strategies, moving beyond reactive corrections to proactive architectural and training interventions.

The second crucial theme for trustworthy LLMs is the development of **integrated and adaptive mitigation systems**. No single strategy can fully address the multifaceted nature of hallucination. The future will see a convergence of techniques, combining the strengths of external grounding, internal self-correction, and multimodal verification. Retrieval-Augmented Generation (RAG), exemplified by foundational work like \textit{ReAct} \cite{yao20229uz} and its advancements \cite{trivedi2022qsf}, will continue to evolve, becoming more robust to noisy retrieval and capable of dynamic knowledge refreshing \cite{vu202337s}. However, as highlighted by \cite{chen2023h04}'s \textit{RGB} benchmark, RAG systems still face challenges in noise robustness and information integration. This necessitates hybrid approaches that interleave RAG with sophisticated self-correction mechanisms, as comprehensively surveyed by \cite{pan2024y3a}. These systems will leverage internal consistency checks, Chain-of-Verification \cite{dhuliawala2023rqn}, and even knowledge graph integration \cite{wen2023t6v, sui20242u1} to iteratively refine and validate generated content. Furthermore, in multimodal contexts, adaptive inference strategies like \textit{Dentist} \cite{chang2024u3t} will dynamically apply visual verification or causal interventions \cite{zhou2024lvp} based on query type and perceived uncertainty, balancing modality priors and enhancing grounding. The overarching goal is to create systems that can intelligently assess their own confidence, seek external validation when uncertain, and correct errors autonomously, leading to greater precision in language generation \cite{ahmadi2024j88}.

Thirdly, the future demands **dynamic, 'in-the-wild' evaluation and continuous monitoring**. Current benchmarks, while valuable, often represent static snapshots and may not fully capture the complexities and adversarial conditions of real-world deployment. The proliferation of specialized benchmarks for multimodal hallucination \cite{li2023249, kuan20249pm, wu2024bxt, sungbin2024r2g, bai2024tkm} highlights the need for context-specific evaluations. Future evaluation frameworks must move towards continuous assessment, incorporating human feedback loops, adversarial testing, and real-time anomaly detection to identify emerging hallucination patterns. This includes meta-evaluation of benchmark quality \cite{yan2024ux8} to ensure that our measurement tools are themselves reliable. The challenge of cross-lingual and cross-modal hallucinations \cite{zhang2025pex} further underscores the need for generalizable and robust evaluation methodologies that can adapt to diverse linguistic and sensory inputs.

Finally, the outlook on trustworthy LLMs is inextricably linked to **responsible AI development and socio-technical integration**. Acknowledging the inevitability of hallucination means that LLMs must be deployed with clear awareness of their limitations, not just by developers but also by end-users. This necessitates robust safeguards, transparent communication about model uncertainties, and mechanisms for human oversight and intervention. The ethical implications of hallucination, including the potential for misinformation and erosion of public trust \cite{gao20242nu}, demand a strong commitment to accountability and safety. Future research must not only focus on technical solutions but also on the human-AI interface, designing systems that are inherently interpretable, explainable, and capable of conveying their confidence levels. This holistic approach, encompassing technical advancements, ethical considerations, and user education, will be paramount in fostering confidence and ensuring the beneficial integration of LLMs into society, despite their inherent imperfections.