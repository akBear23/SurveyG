\subsection*{Towards More Robust and Trustworthy LLMs}

The persistent challenge of hallucination fundamentally impedes the widespread adoption and trustworthiness of Large Language Models (LLMs) in critical applications. While theoretical proofs establish hallucination as an inherent and unavoidable limitation for all computable LLMs \cite{xu2024n76, li2025qzg}, the overarching goal of future research has shifted from complete eradication to robust management and mitigation. This necessitates developing models that are not only resilient to factual errors but also consistently trustworthy, providing transparent, verifiable, and explainable outputs \cite{rejeleene2024okw}. Achieving this demands a paradigm shift towards sophisticated, adaptive hybrid approaches that dynamically integrate diverse mitigation strategies, fostering greater confidence in LLM applications across complex domains.

The next frontier in building trustworthy LLMs lies in **adaptive and agentic external grounding and reasoning**. While Retrieval-Augmented Generation (RAG) has emerged as a powerful paradigm (as discussed in Section 4.1), current implementations, even advanced ones, face limitations such as struggling with noisy or irrelevant retrieved documents, negative rejection, and complex information integration \cite{gao20232zb}. A critical challenge is the LLM's ability to accurately judge the *utility* of retrieved passages for answering a question, rather than just their relevance \cite{zhang2024o58}. Future RAG frameworks must evolve beyond static pipelines to become more intelligent and context-aware agents that can dynamically *learn* to retrieve, filter, and synthesize information. This includes developing systems that can perform conditional retrieval, activating external knowledge sources only when potential hallucinations are detected, as exemplified by early steps like DRAD \cite{su2024gnz}. Furthermore, managing long contexts and prioritizing relevant information, as explored by COFT \cite{lv2024k5x}, needs to be integrated with more sophisticated mechanisms for handling contradictory or ambiguous retrieved sources. Beyond document retrieval, the deeper integration of Knowledge Graphs (KGs) (as explored in Section 4.3) is crucial. While MindMap \cite{wen2023t6v} and Chain-of-Knowledge \cite{li2023v3v} represent significant advancements, future research must focus on enabling LLMs to not just *use* KGs, but to *reason with* them in a logically consistent and verifiable manner. This means moving towards methodologies like R$^3$ \cite{toroghi2024mxf}, which aims for "Right for Right Reasons" by explicitly grounding every factual reasoning step on KG triples, thereby providing transparent and auditable processes. Addressing the significant logical inconsistency of LLMs, even when presented with factual KGs \cite{ghosh2024tj5}, is paramount for true trustworthiness, demanding supervised fine-tuning and novel architectures that enforce adherence to logical rules beyond mere factual recall.

Complementing external grounding, **autonomous and inherently robust internal verifiability and self-correction mechanisms** are paramount. Current self-correction strategies (detailed in Section 4.2), such as Chain-of-Verification (CoVe) \cite{dhuliawala2023rqn} and iterative self-reflection \cite{ji2023vhv}, empower LLMs to self-critique. However, these often incur significant computational costs, may not generalize across all tasks, and can still be susceptible to repeating their own errors or being influenced by the order of reasoning \cite{pan2023mwu, pan2024y3a, xie20247zk}. The future demands more efficient, adaptive, and less prompt-dependent self-correction. This involves developing methods that instill robustness directly into the model's foundational training, rather than relying solely on inference-time interventions. Early efforts like ReDRESS \cite{adams202289x} (revising noisy references) and ObjMLM loss \cite{dai20229aa} (enhancing object faithfulness) are steps in this direction. A promising avenue is the development of novel decoding strategies that actively *penalize* untruthful generations. For instance, Induce-then-Contrast Decoding (ICD) \cite{zhang202396g} creates a "factually weak" model to induce hallucinations, then subtracts its knowledge from the original model's output space, effectively guiding the LLM towards factuality. This represents a proactive, internal mechanism for enhancing factual consistency. Furthermore, self-contained hallucination detection approaches leveraging metamorphic relations and prompt mutation, such as MetaQA \cite{yang20251dw}, point towards a future where models can diagnose their own inconsistencies with minimal external reliance.

The integration of **causally-aware multimodal grounding** is another critical future direction, especially given that hallucinations are often exacerbated in multimodal contexts \cite{liu2024sn3}. While specialized benchmarks (Section 5.2) like ECHO and Cover for audio-language models \cite{kuan20249pm}, R-Bench for relationship hallucinations \cite{wu2024bxt}, and AVHBench for cross-modal hallucinations \cite{sungbin2024r2g} continue to diagnose unique challenges, future MMLMs must achieve robust cross-modal consistency through deeper, more principled interventions. This means moving beyond reactive post-correction pipelines like Woodpecker \cite{yin2023hx3} towards methods that embed causal understanding directly into MMLM architectures. The CAUSAL MM framework \cite{zhou2024lvp}, which employs causal inference on attention mechanisms to balance visual and language priors, represents a significant step. Future research must build upon this to develop MMLMs that are inherently robust to modality prior-induced hallucinations from the ground up. This includes advanced training interventions like HSA-DPO \cite{xiao2024hv1} for severity-aware mitigation, Residual Visual Decoding (RVD) \cite{zhong2024mfi} for re-emphasizing visual information, and ICT \cite{chen2024j0g} for enhanced visual grounding. Furthermore, addressing specific multimodal challenges like spatial hallucination in embodied AI, through prompt engineering and reinforcement learning integration \cite{zhang2025p1z}, will be crucial. Self-improvement frameworks for MLLMs using lightweight verification \cite{deng202405j} and counterfactual prompting \cite{kim2024ozf} will continue to evolve, enabling models to proactively guide themselves away from inconsistencies by seeking multi-view information and performing certainty-driven reasoning \cite{qu2024pqc}.

Crucially, fostering trustworthiness extends beyond mere factual accuracy to **inherent explainability and uncertainty quantification**. While verifiability ensures correctness, true trust requires understanding *why* an LLM generated a particular output and *how confident* it is in that output. Future research must focus on developing LLMs that can generate inherently interpretable reasoning paths, rather than just post-hoc explanations. This involves advancing techniques for uncertainty quantification, which can stem from both the provided demonstrations and ambiguities tied to the model’s configurations \cite{ling2024hqv}. By quantifying and exposing these uncertainties, LLMs can provide users with a more nuanced understanding of their reliability. Furthermore, frameworks like LLMAuditor \cite{amirizaniani2024cad}, which integrate human-in-the-loop verification for generating probes to audit LLM inconsistencies, will be vital for ensuring transparency and verifiability at scale. This integration of XAI principles will enable LLMs to not only avoid factual errors but also to articulate their reasoning and limitations, moving towards a future where LLMs are perceived not as black boxes, but as transparent, accountable, and ultimately, more trustworthy collaborators.

Despite significant progress, challenges persist in developing truly robust and trustworthy LLMs. The generalizability of these hybrid mitigation strategies across diverse domains and complex, multi-step reasoning tasks remains an active research area. Balancing computational efficiency with the depth of verification and self-correction is a continuous trade-off. Future research will likely focus on creating more adaptive, context-aware, and causally-informed hybrid systems that seamlessly integrate dynamic knowledge bases, sophisticated self-correction, and fine-grained multimodal grounding. This includes developing more intelligent agents for autonomous verification and fostering a deeper causal understanding of model failures to build truly transparent, verifiable, and explainable LLMs \cite{li2025qzg}. Ultimately, the trajectory is towards LLMs that not only perform tasks accurately but also provide clear, justifiable reasoning, inspiring unwavering confidence in their deployment across critical applications.