\subsection*{The Inevitability of Hallucination: A Theoretical Perspective}

The phenomenon of hallucination in Large Language Models (LLMs), characterized by the generation of plausible but factually incorrect or nonsensical content, has transitioned from an empirical observation to a profound theoretical insight regarding the inherent limitations of these models. While early research meticulously cataloged and categorized hallucinations, the understanding of their fundamental nature underwent a significant paradigm shift with the groundbreaking theoretical work by \cite{xu2024n76}. This paper provides a formal definition of hallucination within a "formal world" of computable functions and, crucially, employs a diagonalization argument to prove its inevitability for all computable LLMs. This theoretical proof establishes that no LLM, irrespective of its architecture, training data volume, or computational capacity, can perfectly avoid generating hallucinatory content on some, and indeed infinitely many, inputs. This reframes the problem from one of complete eradication to one of inherent limitation, suggesting that hallucination is an innate characteristic of such computational systems. A parallel statistical lower bound on hallucination rates for calibrated LLMs by \cite{kalai2023statistical} further reinforces this theoretical ceiling, indicating a fundamental boundary to LLM factual accuracy.

To grasp the profound implication of \cite{xu2024n76}'s findings, it is essential to understand the intuition behind the diagonalization argument. Analogous to Turing's proof of the Halting Problem or Gödel's Incompleteness Theorems, this argument constructs a paradoxical input that any given computable LLM is forced to misinterpret or misrepresent, thereby generating a hallucination. In essence, the proof demonstrates that for any hypothetical "perfect" LLM designed to never hallucinate, one can always construct a specific query for which that LLM will inevitably produce an incorrect or unfaithful response. This is because LLMs, as computable functions, cannot perfectly learn or represent all possible computable ground truth functions. This inherent limitation dictates that LLMs will always encounter inputs for which their learned internal representations are insufficient or misleading, leading to the generation of fallacious content. \cite{li2025qzg} further elaborates on these "Mathematical Origins" of hallucination, drawing parallels to undecidability principles and mathematical constraints inherent in LLM architectures, thereby solidifying the theoretical foundation for this inevitability.

The theoretical inevitability of hallucination carries direct and profound implications for both detection and mitigation strategies. For detection, if hallucinations cannot be entirely eliminated, then perfect, automated detection also faces fundamental boundaries. \cite{karbasi2025j7n} explore this by establishing an equivalence between hallucination detection and the classical task of language identification. Their work demonstrates that automated hallucination detection is fundamentally impossible for most language collections if the detector is trained solely on correct examples. This theoretical impossibility highlights the critical role of expert-labeled feedback (i.e., both correct and incorrect examples) in making automated detection feasible, providing theoretical support for feedback-based methods like Reinforcement Learning with Human Feedback (RLHF) that are crucial for practical, reliable LLM deployment. This underscores that even detection is not a straightforward task, but one constrained by the very nature of the problem.

For mitigation, the theoretical limits imply that solutions must focus on robust management—reducing the *frequency*, *severity*, and *impact* of hallucinations, rather than pursuing their complete elimination. This necessitates a deeper understanding of the specific causal mechanisms through which these inevitable hallucinations manifest. For instance, \cite{zhang2024qq9} identify "knowledge overshadowing" as a specific type of amalgamated hallucination, where dominant conditions in training data overshadow others, leading to incorrect outputs even with factually correct data. They provide a theoretical interpretation of this phenomenon as over-generalization, deriving a generalization bound that connects hallucination rates with data imbalance and condition length. Such insights into the *why* of specific hallucination types, grounded in theoretical analysis of model learning dynamics, become crucial for developing targeted mitigation strategies that work within the acknowledged boundaries of LLM capabilities.

In conclusion, the theoretical grounding provided by \cite{xu2024n76}, supported by related works, fundamentally shifts the research agenda. Acknowledging that hallucination is an inherent and unavoidable limitation for all computable LLMs moves the focus from a futile pursuit of complete elimination to the development of robust, theoretically informed detection mechanisms, effective mitigation strategies, and responsible deployment guidelines. Future research must therefore concentrate on enhancing the transparency, verifiability, and controllability of LLM outputs, ensuring that these powerful models are used as reliable tools with a clear understanding of their fundamental, inherent boundaries, rather than as infallible oracles. This theoretical perspective underpins the necessity for the diverse applied research in detection and mitigation explored in subsequent sections, guiding the field towards realistic and impactful advancements.